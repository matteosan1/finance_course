    \documentclass[12pt,a4paper]{exam}
\usepackage[utf8]{inputenc}
\usepackage[T1]{fontenc}
\usepackage{amsmath}
\usepackage{amsfonts}
%\usepackage{amssymb}
\usepackage{graphicx}
\usepackage{geometry}
\usepackage{enumitem}

\geometry{a4paper, margin=2cm}

\usepackage{cprotect}

\usepackage{xcolor}
\definecolor{maroon}{cmyk}{0, 0.87, 0.68, 0.32}
\definecolor{halfgray}{gray}{0.55}
\definecolor{ipython-frame}{RGB}{207, 207, 207}
\definecolor{ipython-bg}{RGB}{247, 247, 247}
\definecolor{ipython-red}{RGB}{186, 33, 33}
\definecolor{ipython-green}{RGB}{0, 128, 0}
\definecolor{ipython-cyan}{RGB}{64, 128, 128}
\definecolor{ipython-purple}{RGB}{170, 34, 255}
\usepackage{listings}
\lstdefinelanguage{iPython}{
	morekeywords={access,and,del,except,exec,in,is,lambda,not,or,raise},
	morekeywords=[2]{for,print,abs,all,any,basestring,bin,bool,bytearray,callable,chr,classmethod,cmp,compile,complex,delattr,dict,dir,divmod,enumerate,eval,execfile,file,filter,float,format,frozenset,getattr,globals,hasattr,hash,help,hex,id,input,int,isinstance,issubclass,iter,len,list,locals,long,map,max,memoryview,min,next,object,oct,open,ord,pow,property,range,reduce,reload,repr,reversed,round,set,setattr,slice,sorted,staticmethod,str,sum,super,tuple,type,unichr,unicode,vars,xrange,zip,apply,buffer,coerce,intern,elif,else,if,continue,break,while,class,def,return,try,except,import,finally,try,except,from,global,pass, True, False},
	sensitive=true,
	morecomment=[l]\#,%
	morestring=[b]',%
	morestring=[b]",%
	moredelim=**[is][\color{black}]{@@}{@@},
	%%
	%morestring=[s]{'''}{'''},% used for documentation text (mulitiline strings)
	%morestring=[s]{"""}{"""},% added by Philipp Matthias Hahn
	%%
	%morestring=[s]{r'}{'},% `raw' strings
	%morestring=[s]{r"}{"},%
	%morestring=[s]{r'''}{'''},%
	%morestring=[s]{r"""}{"""},%
	%morestring=[s]{u'}{'},% unicode strings
	%morestring=[s]{u"}{"},%
	%morestring=[s]{u'''}{'''},%
	%morestring=[s]{u"""}{"""}%
	%
	% {replace}{replacement}{lenght of replace}
	% *{-}{-}{1} will not replace in comments and so on
	%literate=
	%{\%}{{{\color{ipython-purple}+}}}1,
	%{á}{{\'a}}1 {é}{{\'e}}1 {í}{{\'i}}1 {ó}{{\'o}}1 {ú}{{\'u}}1,
	%{Á}{{\'A}}1 {É}{{\'E}}1 {Í}{{\'I}}1 {Ó}{{\'O}}1 {Ú}{{\'U}}1
	%{à}{{\`a}}1 {è}{{\`e}}1 {ì}{{\`i}}1 {ò}{{\`o}}1 {ù}{{\`u}}1
	%{À}{{\`A}}1 {È}{{\'E}}1 {Ì}{{\`I}}1 {Ò}{{\`O}}1 {Ù}{{\`U}}1
	%{ä}{{\"a}}1 {ë}{{\"e}}1 {ï}{{\"i}}1 {ö}{{\"o}}1 {ü}{{\"u}}1
	%{Ä}{{\"A}}1 {Ë}{{\"E}}1 {Ï}{{\"I}}1 {Ö}{{\"O}}1 {Ü}{{\"U}}1
	%{â}{{\^a}}1 {ê}{{\^e}}1 {î}{{\^i}}1 {ô}{{\^o}}1 {û}{{\^u}}1
	%{Â}{{\^A}}1 {Ê}{{\^E}}1 {Î}{{\^I}}1 {Ô}{{\^O}}1 {Û}{{\^U}}1
	%{œ}{{\oe}}1 {Œ}{{\OE}}1 {æ}{{\ae}}1 {Æ}{{\AE}}1 {ß}{{\ss}}1
	%{ç}{{\c c}}1 {Ç}{{\c C}}1 {ø}{{\o}}1 {å}{{\r a}}1 {Å}{{\r A}}1
	%{€}{{\EUR}}1 {£}{{\pounds}}1
	%
	%{^}{{{\color{ipython_purple}\^{}}}}1
	%{=}{{{\color{ipython_purple}=}}}1
	%%
	%*{-}{{{\color{ipython_purple}-}}}1
	%{*}{{{\color{ipython_purple}$^\ast$}}}1
	%{/}{{{\color{ipython_purple}/}}}1%%
	%{+=}{{{+=}}}1
	%{-=}{{{-=}}}1
	%{*=}{{{$^\ast$=}}}1
	%{/=}{{{/=}}}1,
	%
	identifierstyle=\color{black}\footnotesize\ttfamily,
	commentstyle=\color{ipython-cyan}\footnotesize\itshape\ttfamily,
	stringstyle=\color{ipython-red}\footnotesize\ttfamily,
	keepspaces=true,
	showspaces=false,
	showstringspaces=false,
	rulecolor=\color{ipython-frame},
	frame=single,
	frameround={t}{t}{t}{t},
	%framexleftmargin=6mm,
	%numbers=left,
	%numberstyle=\color{ipython-cyan},
	backgroundcolor=\color{ipython-bg},
	%   extendedchars=true,
	basicstyle=\footnotesize\ttfamily,
	keywordstyle=[2]\color{ipython-green}\bfseries\footnotesize\ttfamily, 
	keywordstyle=\color{ipython-purple}\bfseries\footnotesize\ttfamily
}

\lstdefinelanguage{iOutput} {
	sensitive=true,
	identifierstyle=\color{black}\small\ttfamily,
	stringstyle=\color{ipython-red}\small\ttfamily,
	keepspaces=true,
	showspaces=false,
	showstringspaces=false,
	rulecolor=\color{ipython-frame},
	%frame=single,
	%frameround={t}{t}{t}{t},
	%backgroundcolor=\color{ipython-bg},
	basicstyle=\small\ttfamily,
}

\lstnewenvironment{ipython}[1][]{\lstset{language=iPython,mathescape=true,escapeinside={*@}{@*}}%
}{%
}

\lstnewenvironment{ioutput}[1][]{\lstset{language=iOutput,mathescape=true,escapeinside={*@}{@*}}%
}{%
}

\title{Financial Market Course 24/25\\ Exam}
\author{Prof. Simone Freschi, Prof. Matteo Sani}
\date{$9^{\mathrm{th}}$ January 2025}

%\printanswers
\noprintanswer
\begin{document}
\maketitle
%\addpoints{exam}
\begin{center}
\fbox{\fbox{\parbox{5.5in}{\centering
Answer the questions in the spaces provided. If you run out of room for an answer, continue on the page back.}}}
\end{center}

\begin{center}
\vspace{5mm}
\makebox[0.75\textwidth]{Student's name:\enspace\hrulefill}
\end{center}

\section*{Questions}
\vspace{.5cm}
\begin{questions}


\question 	Write the equation for the price in real terms and nominal terms of an Inflation Linked Bond.
\begin{enumerate}
\item In the chart \ref{prima} is described the dynamic of US nominal rates and Break-even between June 2023 and December 2023. Describe what could have happened and why to prices of 10y TIP (Treasury Inflation Protected Bonds).


\item In the chart \ref{seconda} is described the dynamic of US nominal rates and Break-even between March 2020 and December 2020. Describe what could have happened and why to prices of 10y TIP (Treasury Inflation Protected Bonds).

\end{enumerate}
\makeemptybox{5cm}

\newpage


% \begin{figure}[h]
%     \centering
%     \includegraphics[width=0.75\linewidth]{inflation2.png}
%     \caption{10y US nominal rate and 10y Break-even}
%     \label{seconda}
% \end{figure}

% \begin{figure}[h]
%     \centering
%     \includegraphics[width=0.75\linewidth]{inflation.png}
%     \caption{10y US nominal rate and 10y Break-even}
%     \label{prima}
% \end{figure}

\newpage



\question Describe what are the main characteristics of a Repo Contract. 
\begin{enumerate}
\item In the chart \ref{terza} is described the delivery Basket of the BTP Future March 2025 (IKH5) which is currently trading at 118.63. The bond BTPS 4.35 Nov33 is the cheapest to deliver with a price of 105.50 and a conversion factor of 0.8953. Describe if there is an arbitrage opportunity and what kind of strategy would you implement. (\textit{hint}: check the basis and use the cash and carry strategy) 

\begin{figure}[h]
    \centering
    \includegraphics[width=0.75\linewidth]{BondFuture.png}
    \caption{BTP March 2025 Futures}
    \label{terza}
\end{figure}

\end{enumerate}
\makeemptybox{5cm}



\question Describe the equation for the return attribution model by Brinson, Hood and Beebower (BHB). 
\begin{enumerate}[label=(\alph*),font=\itshape]
\item Explain the difference between Allocation Effect and Security Selection Effect 
\item Imagine that your portfolio is underweight Tech (10\% vs 20\% Benchmark) and the Tech sector is up 15\%. The Tech stocks in your portfolio are up only 25\%. Determine the allocation effect and the selection effect and the overall portfolio result wrt to sector.
\end{enumerate}
\makeemptybox{5cm}

\question
What should we put instead of the \texttt{?} to make the following boolean expression \texttt{True} ?
\begin{ipython}
True and False or ?
\end{ipython}
\makeemptybox{5cm}
\begin{solution}
You need \texttt{?=True} since the boolean operators have the same precedence so the expression is evaluated from left to right.
Hence
\begin{equation*}
  \begin{gathered}
    \underbrace{\texttt{True and False}}_{\texttt{False}}\texttt{ or ?} = \texttt{True} \\
    \texttt{False or ?} = \texttt{True} \implies \texttt{?} = \texttt{True} \\
    \end{gathered}
\end{equation*}
\end{solution}

\question
You are evaluating an investment opportunity. Assuming the initial payment is $P_0 = \$350,000$, the project is expected to generate the following cash flows:
\begin{itemize}
\item $CF(t=\text{1y})=\$100,000$
\item $CF(t=\text{2y})=\$150,000$
\item $CF(t=\text{3y})=\$200,000$
\end{itemize}
Your required rate of return for this investment is 10\%. Is this investment financially viable ?
\makeemptybox{5cm}
\begin{solution}
Calculate the discount factor for each year ($df=1 / (1 + r)^t$) where $r$ is the discount rate and $t$ is the time period.
\begin{itemize}
\item $df(t=\text{1y})=1 / (1 + 0.10)^1 = 0.9091$
\item $df(t=\text{2y})=1 / (1 + 0.10)^2 = 0.8264$
\item $df(t=\text{3y})=1 / (1 + 0.10)^3 = 0.7513$
\end{itemize}

Then calculate the present value of each cash flow ($PV=CF\times df$)
\begin{itemize}
\item $PV(t=\text{1y})=\$100,000\cdot 0.9091 = \$90,910$
\item $PV(t=\text{2y})=\$150,000\cdot 0.8264 = \$123,960$
\item $PV(t=\text{3y})=\$200,000\cdot 0.7513 = \$150,260$
\end{itemize}

Finally calculate the Net Present Value (NPV) of the project ($NPV = \sum PV_i  - P_0$)
Solutions:
\begin{equation*}
NPV = \$90,910 + \$123,960 + \$150,260 - \$350,000 = -\$84,870
\end{equation*}

Since the $NPV$ is negative, the investment is not expected to generate returns that exceed your required rate of return. Therefore, based on this analysis, the investment is not considered financially viable.
\end{solution}

\question 
A financial institution has entered into an interest rate swap with Company A. Under the terms of the swap, it receives 10\% per annum and pays 6-month LIBOR on a principal of \$10 M for 5 years. Payments are made every 6 months. 

Suppose that Company A defaults just before the sixth payment date (at the end of year 3) when the 6-month LIBOR is 9\% per annum.
Assuming the forward rate is 8\% per annum for all maturities, what is the loss to the financial institution ? 
\makeemptybox{3cm}

\begin{solution}
Cash flows before default:
\begin{itemize}
\item The swap has been running for 3 years, meaning 6 payments have been made. We only need to consider the payments after the default to understand its impact.
Halfway through year 3, 6-month LIBOR was 9\% per annum. This means the financial institution paid $9\%/2 * \$10 M = \$450,000$ to Company A. The financial institution receives instead $10\%/2 * \$10 M = \$500,000$ from Company A every six months.
\item The swap was supposed to continue for another 2 years (4 remaining payments). We need to calculate the present value of the remaining cash flows that the financial institution would have received had Company A not defaulted.
Since we are given a flat term structure of 8\% with semiannual compounding for all maturities as of the default date, we can use it to discount the future cash flows. Here's a breakdown of the remaining cash flows and their present values:

The financial institution would have received \$500,000 and paid $8\%/2 * \$10 M = \$400,000$ for a net cash flow: $\$500,000 - \$400,000 = \$100,000$ for each payment. The resulting PVs are:
\begin{itemize}
\item $PV_7 = \$100,000 / (1 + 0.08/2)^1 = \$96,153.85$
\item $PV_8 = \$100,000 / (1 + 0.08/2)^2 = \$92,455.63$
\item $PV_{9} = \$100,000 / (1 + 0.08/2)^3 = \$88,900$
\item $PV_{10} = \$100,000 / (1 + 0.08/2)^4 = \$85,480.40$
\end{itemize}
\end{itemize}

Then total loss is the sum of the present values of the lost cash flows: $\$96,153.85 + \$92,455.63 + \$88,900 + \$85,480.40 + \$50,000 = \$412,990$.
\end{solution}

\question An exotic derivative has a very complicated payoff, such that to valuate it you are forced to use Monte Carlo simulation. 
At first you decide to run a MC involving just 50 scenarios which results in a contract value of 314,159.26€ $\pm$ 2.72\%. After presenting your results you boss is not that happy and ask you to determine more precisely the derivative value at the next group meeting in 1 hour. Assuming each simulation lasts about 1.8 seconds, what is the highest precision you can achieve of the derivative value ?
\makeemptybox{5cm}

\begin{solution}
Given you have to answer in 1 hour, you can run at most: $\cfrac{3600}{1.8}=2000$ experiments. Since the result precision scale as the square root of the number of simulations
\begin{equation*}
\textrm{new uncertainty} = 0.0272 \cdot \sqrt{\frac{50}{2000}} \approx 0.0043
\end{equation*}
\end{solution}


\end{questions}
\end{document}

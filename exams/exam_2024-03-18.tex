\documentclass[12pt,a4paper]{exam}
\usepackage[utf8]{inputenc}
\usepackage[T1]{fontenc}
\usepackage{amsmath}
\usepackage{amsfonts}
%\usepackage{amssymb}
\usepackage{graphicx}
\usepackage{geometry}
\usepackage{enumitem}

\geometry{a4paper, margin=2cm}

\usepackage{cprotect}

\usepackage{xcolor}
\definecolor{maroon}{cmyk}{0, 0.87, 0.68, 0.32}
\definecolor{halfgray}{gray}{0.55}
\definecolor{ipython-frame}{RGB}{207, 207, 207}
\definecolor{ipython-bg}{RGB}{247, 247, 247}
\definecolor{ipython-red}{RGB}{186, 33, 33}
\definecolor{ipython-green}{RGB}{0, 128, 0}
\definecolor{ipython-cyan}{RGB}{64, 128, 128}
\definecolor{ipython-purple}{RGB}{170, 34, 255}

\usepackage{listings}
\lstdefinelanguage{iPython}{
	morekeywords={access,and,del,except,exec,in,is,lambda,not,or,raise},
	morekeywords=[2]{for,print,abs,all,any,basestring,bin,bool,bytearray,callable,chr,classmethod,cmp,compile,complex,delattr,dict,dir,divmod,enumerate,eval,execfile,file,filter,float,format,frozenset,getattr,globals,hasattr,hash,help,hex,id,input,int,isinstance,issubclass,iter,len,list,locals,long,map,max,memoryview,min,next,object,oct,open,ord,pow,property,range,reduce,reload,repr,reversed,round,set,setattr,slice,sorted,staticmethod,str,sum,super,tuple,type,unichr,unicode,vars,xrange,zip,apply,buffer,coerce,intern,elif,else,if,continue,break,while,class,def,return,try,except,import,finally,try,except,from,global,pass, True, False},
	sensitive=true,
	morecomment=[l]\#,%
	morestring=[b]',%
	morestring=[b]",%
	moredelim=**[is][\color{black}]{@@}{@@},
	identifierstyle=\color{black}\footnotesize\ttfamily,
	commentstyle=\color{ipython-cyan}\footnotesize\itshape\ttfamily,
	stringstyle=\color{ipython-red}\footnotesize\ttfamily,
	keepspaces=true,
	showspaces=false,
	showstringspaces=false,
	rulecolor=\color{ipython-frame},
	frame=single,
	frameround={t}{t}{t}{t},
	backgroundcolor=\color{ipython-bg},
	basicstyle=\footnotesize\ttfamily,
	keywordstyle=[2]\color{ipython-green}\bfseries\footnotesize\ttfamily, 
	keywordstyle=\color{ipython-purple}\bfseries\footnotesize\ttfamily
}

\lstdefinelanguage{iOutput} {
	sensitive=true,
	identifierstyle=\color{black}\small\ttfamily,
	stringstyle=\color{ipython-red}\small\ttfamily,
	keepspaces=true,
	showspaces=false,
	showstringspaces=false,
	rulecolor=\color{ipython-frame},
	basicstyle=\small\ttfamily,
}

\lstnewenvironment{ipython}[1][]{\lstset{language=iPython,mathescape=true,escapeinside={*@}{@*}}%
}{%
}

\lstnewenvironment{ioutput}[1][]{\lstset{language=iOutput,mathescape=true,escapeinside={*@}{@*}}%
}{%
}


\title{Financial Market Course 23/24\\ Exam}
\author{Prof. Simone Freschi, Prof. Matteo Sani}
\date{$18^{\mathrm{st}}$ March 2024}

%\printanswers
\noprintanswers
\begin{document}
\maketitle

\begin{center}
\fbox{\fbox{\parbox{5.5in}{
Answer the questions in the spaces provided. If you run out of room for an answer, continue on the page back.}}}
\end{center}

\begin{center}
\vspace{5mm}
\makebox[0.75\textwidth]{Student's name:\enspace\hrulefill}
\end{center}

\section*{Questions}
\vspace{.5cm}
\begin{questions}

\question 
Credit Risk and CVA.
\begin{enumerate}
\item Describe the probability of the default formula using the intensity model. The counterparty BB Company LTD has a CDS equal to 4\%; calculate the probability of default at 1y horizon using the shortcut formula for the intensity value (Recovery Value equal to 40\%).	
\item Describe the CVA adjustment for interest rate swaps. Imagine that BB Company LTD enters into a EUR 100mln 5 years swap (where it receives fixed 5\%), and subsequently, rates increases to 6\%; what will happen to the mark to market of the swap (if you can calculate the approximante change in fair value) ? What will happen to the CVA all the rest being equal ? In case of a decrease of the CDS to 2\% do you expect a different answer ?\end{enumerate}
\fillwithlines{3cm}
\begin{solution}
\end{solution}

\question Describe the asset swap contract for a coupon bond with coupons equal to $C$ and dirty Price equal to $P(t)$.
\begin{enumerate}
\item What will happen to the price if the sprea increase, all the rest being equal ? What does it mean in term of credit quality expectation by the market for the bond's issuer ?
\item What is the difference between asset swap spread and \textbf{G-spread} or \textbf{I-spread} calculated in Bloomberg screen ?
\end{enumerate}
\fillwithlines{3cm}
\begin{solution}
\end{solution}

\question Describe the equation for the return atribution model by Brinson, Hood and Beebower (BHB).
\begin{enumerate}[label=(\alph*)]
\item Explain the difference between Allocation Effect and Security Selection Effect;
\item Imagine that your portfolio is overweight Energy (30\% vs 20\% Benchmark) and the Energy sector is up 15\%. Your energy stock in the portfolio are up only 5\%. Determine the allocation effect and the selection effect and the overall portfolio result.
\end{enumerate}
\fillwithlines{3cm}
\begin{solution}
\end{solution}

\question Risk-Performance Evaluation Measures.
\begin{enumerate}
\item Describe the Capture Ratio Measures. Imagine that Asset manager \textbf{A} has an Upside Capture Ratio of 140\% and Downside Capture Ratio of 110\%; Asset manager \textbf{B} has an Upside Capture Ratio of 90\% and Downside Capture Ratio of 70\%; who is the best according to this measure?
\item Describe the Draw-Down measure and the Max Draw-Down measure. Imagine that Asset manager \textbf{A} has an expected excess return of 10\% and Max Draw-Down equal to 20\%; Asset manager \textbf{B} has an expected excess return of 5\% and Max Draw-Down equal to 7\%; who would you prefer and why? Will your choice be different if you had a target return of more or equal to 10\%?
\end{enumerate}
\fillwithlines{3cm}


\question Consider different bonds with a face value of \$ 100, with the coupon rates as in Tab.\ref{tab:coupons}.

\begin{table}[htb]
  \begin{center}
    \begin{tabular}{|l|c|c|c|c|}
      \hline
      \textbf{Maturity}    & 6M    & 1Y     & 1.5Y   & 2Y  \\ \hline
      \textbf{Coupon (\%)} & 3     & 3.50   & 4.50   & 6 \\ \hline
      \textbf{Price (\$)}  & 98.53 & 103.39 & 105.90 & 110.74 \\ \hline
    \end{tabular}
    \end{center}
    \caption{Coupon-bond characteristics.}
    \label{tab:coupons}
  \end{table}
Each bond has a semi-annual payment schedule. Applying the bootstrap technique determine the implied yield curve.
\fillwithlines{3cm}
\begin{solution}
Considering the first bond we can deduce that
\begin{equation*}
  P_1 = \frac{FV + C_1/2}{(1+\frac{r_{6M}}{2})} \implies r_{6M} = 2\cdot \left(\frac{101.5}{98.53}-1\right) = 6.029\%
\end{equation*}

The second bond has an intermediate copoun (at 6M) before expiry so previous formula becomes

\begin{equation*}
  \begin{gathered}
    P_2 = \frac{C_2/2}{(1+\frac{r_{6M}}{2})} + \frac{FV + C_2/2}{(1+\frac{r_{1Y}}{2})^2} = \frac{1.75}{1.030145} + \frac{101.75}{(1+\frac{r_{1Y}}{2})^2} = \ldots
  \end{gathered}
\end{equation*}

Analogously for the third bond.
%
%\begin{equation*}
%  \begin{gathered}
%    P_3 =  \frac{C_3}{(1+r_{6M})^{\frac{1}{2}}} + \frac{C_3}{(1+r_{1Y})} +  \frac{FV+C_3}{(1+r_{1.5Y})^{\frac{2}{3}}} \\
%    r_{1.5Y} = \left(\frac{104.5}{105.9-8.7793}\right)^{\frac{2}{3}}-1 = 5.00\%
%  \end{gathered}
%\end{equation*}
%
%A different programmatical approach relies on the usage of root finding algorithms on the bond pricing equation to determine the unknown rate:
%\begin{ipython}
%from scipy.optimize import brentq
%
%rate = [0.03, 0.0356, 0.05]
%def price(r, F, C, P):
%    return C/(1+rate[0])**0.5 + C/(1+rate[1]) + C/(1+rate[2])**1.5 \\
%        + (F+C)/(1+r)**2 - P
%
%r = brentq(price, 0, 0.2, args=(100, 100*0.06, 110.74)
%print ("{:.3f}".format(r))
%\end{ipython}
%\begin{ioutput}
%0.065
%\end{ioutput}
\end{solution}

%%%%%%%%%%%%%%%%%%%%%%%%%%%%%%%%%%%%%%%%%%
\question
Suppose \texttt{my\_list} is \texttt{[3, 6, 12, 24, 5, 10, 15, 20]}.
Which of the following statements returns the following list \texttt{[6, 24, 10, 20]} ?

\begin{checkboxes}
\choice \texttt{print (my\_list[1:8])}
\choice \texttt{print (my\_list[::2])}
\choice \texttt{print (my\_list[1::2])}
\choice \texttt{print (my\_list[-1:-7:-2])}
\end{checkboxes}
\begin{solution}
Choice 3 is the correct one, since the operator params are \texttt{[start:end:step]}
\end{solution}
%%%%%%%%%%%%%%%%%%%%%%%%%%%%%%%%%%%%%%%%%%
\question
A 3-year corporate bond pays a coupon of 8\% p.a.. The term structure of risk-free interest rate is flat 4\% p.a. and is expected to stay unchanged in the future. If the risk-neutral conditional probability of default of the issuer is 2\% for a 1-year period and the recovery rate in case of default is 40\% (assumed constant), estimate the CVA.

\fillwithlines{3cm}
\begin{solution}
The CVA is defined as 
\begin{equation*}
	CVA = LGD\sum_i EE_i PD_i DF_i
\end{equation*}
where the sum is on the contract cash-flows, $LGD$ is the loss given default ($1-R$), $EE_i$ the expected exposure, $PD_i$ the default probability of the counter-part, and $DF_i$ the discount factor.

So we need to compute each term for the three periods:

\begin{center}
\begin{tabular}{|c|c|c|c|c|}
\hline
Period & $EE_i$ & $PD_i$ & $DF_i$ & Tot. \\
\hline
1 & $8 + \frac{8}{1+0.04}$ + $\frac{108}{(1+0.04)^2}$ & 2\% & $\frac{1}{1+0.04}$ & 2.22 \\
\hline
2 & $8 + \frac{108}{1+0.04}$ & $(1-2\%)\cdot 2\%$ & $\frac{1}{(1+0.04)^2}$ & 2.03 \\
\hline
3 & 108 & $(1-2\%)^2\cdot 2\%$ & $\frac{1}{(1+0.04)^3}$ & 1.84 \\
\hline
\end{tabular}
\end{center}
Putting all together the CVA results to be 3.654.
\end{solution}

%%%%%%%%%%%%%%%%%%%%%%%%%%%%%%%%%%%%%%%%%%
\question
What is the use of the Loss functions in a Neural Network model ?
\fillwithlines{3cm}
\begin{solution}
The loss function is used as a measure of accuracy to identify whether our neural network has learned the patterns accurately or not with the help of the training data.
This is completed by comparing the training data with the testing data.
\end{solution}

\end{questions}
\end{document}

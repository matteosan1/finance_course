\documentclass[12pt,a4paper]{exam}
\usepackage[utf8]{inputenc}
\usepackage[T1]{fontenc}
\usepackage{amsmath}
\usepackage{amsfonts}
%\usepackage{amssymb}
\usepackage{graphicx}
\usepackage{geometry}
\usepackage{enumitem}

\geometry{a4paper, margin=2cm}

\usepackage{cprotect}

\usepackage{xcolor}
\definecolor{maroon}{cmyk}{0, 0.87, 0.68, 0.32}
\definecolor{halfgray}{gray}{0.55}
\definecolor{ipython-frame}{RGB}{207, 207, 207}
\definecolor{ipython-bg}{RGB}{247, 247, 247}
\definecolor{ipython-red}{RGB}{186, 33, 33}
\definecolor{ipython-green}{RGB}{0, 128, 0}
\definecolor{ipython-cyan}{RGB}{64, 128, 128}
\definecolor{ipython-purple}{RGB}{170, 34, 255}
\usepackage{listings}
\lstdefinelanguage{iPython}{
	morekeywords={access,and,del,except,exec,in,is,lambda,not,or,raise},
	morekeywords=[2]{for,print,abs,all,any,basestring,bin,bool,bytearray,callable,chr,classmethod,cmp,compile,complex,delattr,dict,dir,divmod,enumerate,eval,execfile,file,filter,float,format,frozenset,getattr,globals,hasattr,hash,help,hex,id,input,int,isinstance,issubclass,iter,len,list,locals,long,map,max,memoryview,min,next,object,oct,open,ord,pow,property,range,reduce,reload,repr,reversed,round,set,setattr,slice,sorted,staticmethod,str,sum,super,tuple,type,unichr,unicode,vars,xrange,zip,apply,buffer,coerce,intern,elif,else,if,continue,break,while,class,def,return,try,except,import,finally,try,except,from,global,pass, True, False},
	sensitive=true,
	morecomment=[l]\#,%
	morestring=[b]',%
	morestring=[b]",%
	moredelim=**[is][\color{black}]{@@}{@@},
	identifierstyle=\color{black}\footnotesize\ttfamily,
	commentstyle=\color{ipython-cyan}\footnotesize\itshape\ttfamily,
	stringstyle=\color{ipython-red}\footnotesize\ttfamily,
	keepspaces=true,
	showspaces=false,
	showstringspaces=false,
	rulecolor=\color{ipython-frame},
	frame=single,
	frameround={t}{t}{t}{t},
	backgroundcolor=\color{ipython-bg},
	basicstyle=\footnotesize\ttfamily,
	keywordstyle=[2]\color{ipython-green}\bfseries\footnotesize\ttfamily, 
	keywordstyle=\color{ipython-purple}\bfseries\footnotesize\ttfamily
}

\lstdefinelanguage{iOutput} {
	sensitive=true,
	identifierstyle=\color{black}\small\ttfamily,
	stringstyle=\color{ipython-red}\small\ttfamily,
	keepspaces=true,
	showspaces=false,
	showstringspaces=false,
	rulecolor=\color{ipython-frame},
	basicstyle=\small\ttfamily,
}

\lstnewenvironment{ipython}[1][]{\lstset{language=iPython,mathescape=true,escapeinside={*@}{@*}}%
}{%
}

\lstnewenvironment{ioutput}[1][]{\lstset{language=iOutput,mathescape=true,escapeinside={*@}{@*}}%
}{%
}

\title{Financial Market Course 22/23\\ Exam}
\author{Prof. Simone Freschi, Prof. Matteo Sani}
\date{$21^{\mathrm{st}}$ June 2023}

%\printanswers
\noprintanswers
\begin{document}
\maketitle

\begin{center}
\fbox{\fbox{\parbox{5.5in}{\centering
Answer the questions in the spaces provided. If you run out of room for an answer, continue on the page back.}}}
\end{center}

\begin{center}
\vspace{5mm}
\makebox[0.75\textwidth]{Student's name:\enspace\hrulefill}
\end{center}

\section*{Questions}
\vspace{.5cm}
\begin{questions}
\question
Consider a set of zero-coupon bonds of face value \$ 100, with maturity 6 months, 9 months and 2 years. The prices of the bonds are as below:
\begin{table}[h]
  \begin{center}
    \begin{tabular}{|l|c|c|}
      \hline
      \textbf{Period} & \textbf{Maturity} & \textbf{Price (\$)} \\ \hline
      Months          & 6                 & 99.00               \\ \hline
      Months          & 9                 & 98.50               \\ \hline
      Years           & 2                 & 94.35               \\ \hline
    \end{tabular}
  \end{center}
\end{table}
Determine the implied yield curve including the (approximated value) one year zero-coupon rate.
\fillwithlines{3cm}
\begin{solution}
Considering simple compounding rates $\textbf{ZCB}=\cfrac{FV}{1+r*t}$, from the first bond we get
\begin{equation*}
  r_{6M} = \frac{FV-P}{Pt_y} = \frac{100 - 99}{99\cdot0.5} = 2.0202\% 
\end{equation*}

Similarly for the second and third bonds
\begin{equation*}
  \begin{gathered}
    r_{9M} = \frac{100 - 98.50}{98.50\cdot0.75} = 2.0305\% \\
    r_{2Y} = \frac{100 - 94.35}{94.35\cdot2.0} = 2.9942\%
 \end{gathered}   
\end{equation*}

To determine the one year zero rate we can linearly interpolate between the nine months and two years values
\begin{equation*}
  \begin{gathered}
    r_{2Y} = r_{9M}\cdot w_{9M} + r_{2Y}\cdot w_{2Y},\quad\left(w_{9M}=\frac{t_{1Y}-t_{9M}}{t_{2Y}-t_{9M}};w_{9M}=\frac{t_{2Y}-t_{1Y}}{t_{2Y}-t_{9M}}\right) \\
    r_{2Y} = 0.020305\frac{0.25}{1.25} + 0.29942 \frac{1}{1.25} = 2.4359\%
  \end{gathered}
\end{equation*}
\end{solution}
%%%%%%%%%%%%%%%%%%%%%%%%%%%%%%%%%%%%%%%%%%%%%%%%%%%%%%%%%%%%%%%%%%%%%%%%%%%%%%%%%%%
\question
Consider a Poisson process with constant hazard rate $\lambda$ describing the \emph{default probability} $Q=1-e^{-\lambda t}$ of Nakatomi Trading Corp.. One outstanding property of probability distributions is that $\sum_j p_j = 1$ (i.e. the sum of $p(x)$ over all possible values of $x$ is 1). In light of the previous statement how would you interpret the result
\begin{equation*}
\int_0^{+\infty} Q(t) dt = \int_0^{+\infty}\left(1- e^{-\lambda t}\right) dt \rightarrow \infty
\end{equation*}
\fillwithlines{3cm}
\begin{solution}
The survival probability is a \textbf{cumulative}-conditioned probability, as such its integral over the entire 
sample space is not supposed to equal 1. 
\end{solution}

%%%%%%%%%%%%%%%%%%%%%%%%%%%%%%%%%%%%%%%%%%%%
\question
Write a \texttt{python} function that returns the interest rate given the discount factor $D(0, t)$.
\fillwithlines{3cm}
\begin{solution}
  \begin{ipython}
import numpy as np

def r(d, t):
    return np.log(d)/t
  \end{ipython}
\end{solution}
%%%%%%%%%%%%%%%%%%%%%%%%%%%%%%%%%%%%%%%%%%%
\question
Given the following \texttt{python} expression

\begin{ipython}
a = 36 / 4 * (3 + 2) * 4 + 2

print (a)
\end{ipython}

What is the resulting value stored in variable $a$

\begin{checkboxes}
\choice 182.0
\choice 117
\choice 42
\end{checkboxes}
\begin{solution}
The solution is 182.0 because of the operator precedence that is preserved in a \texttt{python} expression.
\end{solution}
%%%%%%%%%%%%%%%%%%%%%%%%%%%%%%%%%%%%%%%%%%%%%
\question
What is the output of the following code?

\begin{ipython}
var1 = 1
var2 = 2
var3 = "3"

print(var1 + var2 + var3)
\end{ipython}
\fillwithlines{3cm}
\begin{solution}
The result is an error, cannot mix integers and strings in an operation
\begin{ioutput}
TypeError: unsupported operand type(s) for +: 'int' and 'str'
\end{ioutput}
\end{solution}

\end{questions}
\end{document}

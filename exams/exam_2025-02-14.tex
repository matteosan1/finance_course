\documentclass[12pt,a4paper]{exam}
\usepackage[utf8]{inputenc}
\usepackage[T1]{fontenc}
\usepackage{amsmath}
\usepackage{amsfonts}
%\usepackage{amssymb}
\usepackage{graphicx}
\usepackage{geometry}
\usepackage{enumitem}

\geometry{a4paper, margin=2cm}

\usepackage{cprotect}

\usepackage{xcolor}
\definecolor{maroon}{cmyk}{0, 0.87, 0.68, 0.32}
\definecolor{halfgray}{gray}{0.55}
\definecolor{ipython-frame}{RGB}{207, 207, 207}
\definecolor{ipython-bg}{RGB}{247, 247, 247}
\definecolor{ipython-red}{RGB}{186, 33, 33}
\definecolor{ipython-green}{RGB}{0, 128, 0}
\definecolor{ipython-cyan}{RGB}{64, 128, 128}
\definecolor{ipython-purple}{RGB}{170, 34, 255}
\usepackage{listings}
\lstdefinelanguage{iPython}{
	morekeywords={access,and,del,except,exec,in,is,lambda,not,or,raise},
	morekeywords=[2]{for,print,abs,all,any,basestring,bin,bool,bytearray,callable,chr,classmethod,cmp,compile,complex,delattr,dict,dir,divmod,enumerate,eval,execfile,file,filter,float,format,frozenset,getattr,globals,hasattr,hash,help,hex,id,input,int,isinstance,issubclass,iter,len,list,locals,long,map,max,memoryview,min,next,object,oct,open,ord,pow,property,range,reduce,reload,repr,reversed,round,set,setattr,slice,sorted,staticmethod,str,sum,super,tuple,type,unichr,unicode,vars,xrange,zip,apply,buffer,coerce,intern,elif,else,if,continue,break,while,class,def,return,try,except,import,finally,try,except,from,global,pass, True, False},
	sensitive=true,
	morecomment=[l]\#,%
	morestring=[b]',%
	morestring=[b]",%
	moredelim=**[is][\color{black}]{@@}{@@},
	identifierstyle=\color{black}\footnotesize\ttfamily,
	commentstyle=\color{ipython-cyan}\footnotesize\itshape\ttfamily,
	stringstyle=\color{ipython-red}\footnotesize\ttfamily,
	keepspaces=true,
	showspaces=false,
	showstringspaces=false,
	rulecolor=\color{ipython-frame},
	frame=single,
	frameround={t}{t}{t}{t},
	%framexleftmargin=6mm,
	%numbers=left,
	%numberstyle=\color{ipython-cyan},
	backgroundcolor=\color{ipython-bg},
	%   extendedchars=true,
	basicstyle=\footnotesize\ttfamily,
	keywordstyle=[2]\color{ipython-green}\bfseries\footnotesize\ttfamily, 
	keywordstyle=\color{ipython-purple}\bfseries\footnotesize\ttfamily
}

\lstdefinelanguage{iOutput} {
	sensitive=true,
	identifierstyle=\color{black}\small\ttfamily,
	stringstyle=\color{ipython-red}\small\ttfamily,
	keepspaces=true,
	showspaces=false,
	showstringspaces=false,
	rulecolor=\color{ipython-frame},
	basicstyle=\small\ttfamily,
}

\lstnewenvironment{ipython}[1][]{\lstset{language=iPython,mathescape=true,escapeinside={*@}{@*}}%
}{%
}

\lstnewenvironment{ioutput}[1][]{\lstset{language=iOutput,mathescape=true,escapeinside={*@}{@*}}%
}{%
}

\title{Financial Market Course 24/25\\ Exam}
\author{Prof. Simone Freschi, Prof. Matteo Sani}
\date{$14^{\mathrm{th}}$ February 2025}

%\printanswers
\noprintanswer
\begin{document}
\maketitle
%\addpoints{exam}
\begin{center}
\fbox{\fbox{\parbox{5.5in}{\centering
Answer the questions in the spaces provided. If you run out of room for an answer, continue on the page back.}}}
\end{center}

\begin{center}
\vspace{5mm}
\makebox[0.75\textwidth]{Student's name:\enspace\hrulefill}
\end{center}

\section*{Questions}
\vspace{.5cm}
\begin{questions}


% \question 	Write the equation for the price in real terms and nominal terms of an Inflation Linked Bond.
% \begin{enumerate}
% \item In the chart \ref{prima} is described the dynamic of US nominal rates and Break-even between June 2023 and December 2023. Describe what could have happened and why to prices of 10y TIP (Treasury Inflation Protected Bonds).


% \item In the chart \ref{seconda} is described the dynamic of US nominal rates and Break-even between March 2020 and December 2020. Describe what could have happened and why to prices of 10y TIP (Treasury Inflation Protected Bonds).

% \end{enumerate}
% \makeemptybox{5cm}

% \newpage


% % \begin{figure}[h]
% %     \centering
% %     \includegraphics[width=0.75\linewidth]{inflation2.png}
% %     \caption{10y US nominal rate and 10y Break-even}
% %     \label{seconda}
% % \end{figure}

% % \begin{figure}[h]
% %     \centering
% %     \includegraphics[width=0.75\linewidth]{inflation.png}
% %     \caption{10y US nominal rate and 10y Break-even}
% %     \label{prima}
% % \end{figure}

% \newpage



% \question Describe what are the main characteristics of a Repo Contract. 
% \begin{enumerate}
% \item In the chart \ref{terza} is described the delivery Basket of the BTP Future March 2025 (IKH5) which is currently trading at 118.63. The bond BTPS 4.35 Nov33 is the cheapest to deliver with a price of 105.50 and a conversion factor of 0.8953. Describe if there is an arbitrage opportunity and what kind of strategy would you implement. (\textit{hint}: check the basis and use the cash and carry strategy) 

% \begin{figure}[h]
%     \centering
%     \includegraphics[width=0.75\linewidth]{BondFuture.png}
%     \caption{BTP March 2025 Futures}
%     \label{terza}
% \end{figure}

% \end{enumerate}
% \makeemptybox{5cm}



% \question Describe the equation for the return attribution model by Brinson, Hood and Beebower (BHB). 
% \begin{enumerate}[label=(\alph*),font=\itshape]
% \item Explain the difference between Allocation Effect and Security Selection Effect 
% \item Imagine that your portfolio is underweight Tech (10\% vs 20\% Benchmark) and the Tech sector is up 15\%. The Tech stocks in your portfolio are up only 25\%. Determine the allocation effect and the selection effect and the overall portfolio result wrt to sector.
% \end{enumerate}
% \makeemptybox{5cm}

\question
What is the output of the following code ?

\begin{ipython}
d = {"apple":10, "pear":4, "orange":3}

print (d["peach"])
\end{ipython}
\makeemptybox{1cm}
\begin{solution}
\texttt{KeyError: 'peach'}
\end{solution}

\question
Consider a set of zero-coupon bonds of face value \$ 100, with maturity 6 months, 9 months and 1 years. 
The prices of the bonds are as below:
\begin{table}[h]
  \begin{center}
    \begin{tabular}{|l|c|c|}
      \hline
      \textbf{Period} & \textbf{Maturity} & \textbf{Price (\$)} \\ \hline
      Months          & 6                 & 98.808€             \\ \hline
      Years           & 1                 & 97.686€             \\ \hline
    \end{tabular}
  \end{center}
\end{table}
Determine the implied yield curve including the (approximated value) one year zero-coupon rate.
\emph{Consider simple compounding rates.}
\makeemptybox{4cm}
\begin{solution}
Considering simple compounding rates $\textbf{ZCB}=\cfrac{FV}{1+r*t}$, from the first bond we get
\begin{equation*}
  r_{6M} = \frac{FV-P}{Pt_y} = \frac{100 - 98.808}{98.808\cdot0.5} = 2.4126\% 
\end{equation*}

Similarly for the second and third bonds
\begin{equation*}
  \begin{gathered}
    r_{1Y} = \frac{100 - 97.686}{97.686\cdot2.0} = 2.3688\%
 \end{gathered}   
\end{equation*}

To determine the nine-months zero rate we can linearly interpolate between the six months and one year prices and 
then compute the yield as before.
\begin{equation*}
  \begin{gathered}
    P_{9M} = \cfrac{P_{6M} + P_{1Y}}{2} = 98.247€\\
    r_{9M} = \frac{100 - 98.247}{98.247\cdot0.75} = 2.3790\%
  \end{gathered}
\end{equation*}
\end{solution}


\question 
Consider a Poisson process with constant hazard rate $\lambda$ describing the \emph{survival probability} $Q=1-e^{-\lambda t}$ of Satori Corp.. One outstanding property of probability distributions is that $\sum_j p_j = 1$ (i.e. the sum of $p(x)$ over all possible values of $x$ is 1). In light of the previous statement how would you interpret the result
\begin{equation*}
\int_0^{+\infty} Q(t) dt = \int_0^{+\infty}\left(1- e^{-\lambda t}\right) dt \rightarrow \infty
\end{equation*}
\fillwithlines{5cm}

\begin{solution}
The result indicates that the expected total time survived is infinite. This might seem counterintuitive since we know that the survival probability $Q(t)$ goes to zero as $t$ goes to infinity.
The exponential decay of the survival probability, while going to zero, does so slowly enough that the integral of the 
survival curve diverges.  
Even though the probability of surviving any particular long duration gets small, 
there's always some probability of surviving a bit longer, and these "long tail" probabilities add up to infinity when 
you consider all possible durations.

Imagine repeatedly flipping a fair coin. The probability of getting heads on any specific flip is 1/2.  
The probability of getting heads on every flip for an infinite number of flips is zero.  However, 
if you ask "how many flips do I expect to make before I get tails?", the answer is infinity.  
Even though the probability of any given long sequence of heads is tiny, the expected number of flips is infinite 
due to the possibility of arbitrarily long runs of heads.

In simpler terms:  even though you're certain the event will eventually happen ($Q(t)$ goes to zero), the average time you have to wait for it is infinite because there's always a non-zero (though decreasing) chance that it will be delayed even longer.
\end{solution}

\question
Write a \texttt{python} function that returns the interest rate given the discount factor $D(0, t)$.
\makeemptybox{4cm}
\begin{solution}
  \begin{ipython}
import numpy as np

def r(d, t):
    return np.log(d)/t
  \end{ipython}
\end{solution}


\end{questions}
\end{document}


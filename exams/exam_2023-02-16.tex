\documentclass[12pt,a4paper]{exam}

\input{exam_header}

\title{Advanced Financial Modeling Course 23/24\\ Exam}
\author{Prof. Andrea Carapelli, Prof. Matteo Sani}
\date{$16^{\mathrm{th}}$ February 2024}

\printanswers
%\noprintanswers
\begin{document}
\maketitle
%\addpoints{exam}
\begin{center}
\fbox{\fbox{\parbox{5.5in}{\centering
Answer the questions in the spaces provided. If you run out of room for an answer, continue on the page back.}}}
\end{center}

\begin{center}
\vspace{5mm}
\makebox[0.75\textwidth]{Student's name:\enspace\hrulefill}
\end{center}

\section*{Questions}
\vspace{.5cm}
\begin{questions}

%%%%%%%%%%%%%%%%%%%%%%%%%%%%%%%%%%%%%%%%%%%%%%%%%%%%%
\question Consider the process $Y(t) = 2^{W(t)}, where $\{W(T):t\geq 0\}$ is a standard Brownian motion. Is this a martingale ?
\fillwithlines{3cm}
\begin{solution}
With $g(t)=2^{W(t)}$, we find:
\begin{equation*}
dg(t) = \ln2\cdot 2^{W(t)}dW(t) +\cfrac{(\ln2)^2}{2}2^{W(t)}dt
\end{equation*}
Note that $g, g_{x}, g_{xx}$ exist and are continouos. 
Due to the appearance of a $dt$-term, the process is not a martingale.
\end{solution}

%%%%%%%%%%%%%%%%%%%%%%%%%%%%%%%%%%%%%%%%%%%%%%%%%%%%%
\question Let $\{W(T):t\geq 0\}$ be a Brownian motion on a probability space and let $\mathcal{F}_t$ be its natural filtration. Consider a stock with price process $\{S(t):0\leq t \leq T\}$, with 
\begin{equation*}
S(t)=S(0)\exp\left{\int_0^t eì{-u}dW(u) + \int_0^t(1-\frac{1}{2}e^{-2u}du\right}
\end{equation*}
\begin{enumerate}[label=(\alph*),font=\itshape]
\item Let 
\begin{equation*}
X(t)=\int_0^te^{-u}dW(u)+\int_0^t(1-\frac{1}{2}e^{-2u}du\right}
\end{equation*}
and determine the distribution of $X(t).
\item Prove that $\{S(t):t\geq 0\}$ is an Ito process.
\item Let $r$ be a constant interest rate. Find the risk-neutral measure $\tilde{\mathcal{P}}$, equivalent to $\mathcal{P}$, such that the discounted price process $\{we^{-rt}S(t): 0\leq t \leq T\}$ is a martingale under $\tilde{\mathcal{P}}$.
\end{enumerate}
\fillwithlines{3cm}
\begin{solution}
\begin{enumerate}[label=(\alph*),font=\itshape]
\item Let $Y(t)=\int_0^t e^{-u}dW(u)$, or the first term of the $X(t)$ process. From the stochastic calculus results we know that $Y(t)$ is normally distributed with $\mathbb{E}[Y(t)]=0$ and 
\begin{equation*}
\text{Var}[Y(t)]=\int_0^t e^{-2u}du = \frac{1}{2}(1-e^{-2t})$
\end{equation*} 
Since 
\begin{equation*}
X(t) = Y(t) + \int_0^t (1-\frac{1}{2}e^{-2u})du = Y(t) + t + \frac{1}{4}(e^{-2t}-1)$
\end{equation*} 
we see that $X(t)$ is normally distributed, with mean 
\begin{equation*}
\mathbb{E}[X(t)] = t + \frac{1}{4}(e^{-2t}-1)$
\end{equation*} 
and variance
\begin{equation*}
\text{Var}[X(t)} = \text{Var}[Y(t)] = \frac{1}{2}(1-e^{-2t})$
\end{equation*} 
\item With 
\begin{equation*}
X(t) = \int_0^t e^{-u}dW(u) + \int_0^t (1-\frac{1}{2}e^{-2u})du$
\end{equation*} 
we have 
\begin{equation*}
dX(t) = e^{-t}dW(t) + (1-\frac{1}{2}e^{-2t})dt$
\end{equation*} 
and $dX(t)dX(t)=e^{-2t} dt$. Note that $S(t)=S(0)e^{X(t)}$, so let $f(x)=S(0)e^x$, then $f_x(x)=f_{xx}(x)=f(x)$. By the Ito formula we get
\begin{equation*}
\begin{aligned}
dS_t &= df(X_t) = S_tdX+\frac{1}{2}S(t)dXdX = \\
&=S_t\left(e^{-t}dW + (1-\frac{1}{2}e^{-2t})dt\right)+\frac{1}{2}S_t e^{-2t}dt = \\
&=S_t dt+S_t e^{-t}dW
\end{aligned}
\end{equation*}
This shows that $S_t$ is an Ito process.
\item Define 
\begin{equation*}
\theta(r)=\cfrac{1-r}{e^{-t}}=e^t (1-r)
\end{equation*}
Consider the random variable $Z$, defined by
\begin{equation*}
\begin{aligned}
Z &= \exp\left(-\int_0^T \theta(u)dW(u) - \frac{1}{2}\int_0^T \theta^2(u)du\right) = \\
&=\exp\left(-\int_0^T e^u (1-r)dW(u) - \frac{1}{2}\int_0^T e^{2u} (1-r)^2 du\right)
\end{aligned}
\end{equation*}
Define the measure $\tilde{\mathcal{P}}$ by $\tilde{\mathcal{P}}(A) =\int_A Z d\mathcal{P}$ and consider the process 
\begin{equation*}
\begin{aligned}
\tilde{W}_t = \int_0^t \theta(u)dW(u) +W(t) = \int_0^t e^{u}(1-r)du+W(t)=(1-r)(e^t -1)+W(t)
\end{equation*}
By Girsanov Theorem $\tilde{W}$ is a Brownian motion under $\tilde{\mathcal{P}}$ and hence it is a martingale under $\tilde{\mathcal{P}}$. 
Using the SDE of part (a), together with the Ito product rule, we have 
\begin{equation*}
\begin{aligned}
d(e^{-rt}S(t)) &= e^{-rt}dS_t-r e^{-rt}S_t dt = \\
&= e^{-rt}(S_t dt + S_te^{-t}dW)-re^{-rt}S_t = \\
&= e^{-rt}S_t((1-r)dt + e^{-t}dW)) = \\
&= e^{-rt}S_t(e^{-t}\theta_t dt + e^{-t}dW)) = \\
&= e^{-t(r+1)}S_t d\tilde{W}_t
\end{aligned}
\end{equation*}
Since $e^{-rt}S(t)$ is an Ito integral, we see that the discounted price process is a martingale under $\tilde{\mathcal{P}}$.
\end{enumerate}
\end{solution}

%%%%%%%%%%%%%%%%%%%%%%%%%%%%%%%%%%%%%%%%%%%%%%%%%%%%%
\question Suppose that $X(t)$ satisfies the following SDE:
\begin{equation*}
dX_t = 0.04X_t dt + \sigma X_t dW_t
\end{equation*}
and $Y_t$ satisfies:
\begin{equation*}
dY_t = \beta Y_t dt + 0.1 Y_t dW_t
\end{equation*}
Parameters $\beta, \sigma$ are postive and both processes are driven by the same Brownian Motion $W(t)$.
For a given process
\begin{equation*}
Z_t = 2\cfrac{X_t}{Y_t}-\lambda t
\end{equation*}
with $\lambda\in\mathbb{R}^+$.
\begin{enumerate}[label=(\alph*),font=\itshape]
\item Find the SDE for $Z_t$;
\item For which values of $\beta$ and $\lambda$ is the process $Z_t$ a martingale ?

\fillwithlines{3cm}à
\begin{solution}
\begin{enumerate}[label=(\alph*),font=\itshape]
\item We have 
\begin{equation*}
\begin{gathered}
X_t = e^{\sigma W_t-\frac{\sigma^2}{2}t+0.04 t}\\
dX_t = 0.04 X_t dt + \sigma X_t dW_t\\
Y_t = e^{0.1W_t -\frac{0.01}{2}t+\beta t}\\
dY_t = \beta Y_t dt + 0.1 Y_t dW_t
\end{gathered}
\end{equation*}
Using the expressions fot $X_t$ and $Y_t$ we get
\begin{equation*}
Z_t = 2\exp\left((\sigma-0.1)W_t + (0.04+\frac{0.01}{2}-\beta-\frac{\sigma^2}{2})t\right)-\lambda t
\end{equation*}
\item A martingale process does not contain a drift term. We have 
\begin{equation*}
dZ_t = (Z+\lambda t)(0.01+0.04-\beta-0.1\sigma)dt-\lambda dt + (Z+\lambda t)(\sigma-0.1)dW_t
\end{equation*}
With $\beta$ and $\sigma$ constant and $\lambda\in\mathbb{R}^+$, the necessary conditions for a vanishing drift term are $\lambda=0$ and
\begin{equation*}
0.01+0.04-\beta-0.1\sigma=0\implies \beta=0.05-0.15\sigma
\end{equation*}
%To check this result we employ the Ito derivative rules for multivariate functions...
\end{enumerate}
\end{solution}


\end{questions}
\end{document}

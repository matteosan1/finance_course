\documentclass[11pt]{article}

    \usepackage[breakable]{tcolorbox}
    \usepackage{parskip} % Stop auto-indenting (to mimic markdown behaviour)
    
    \usepackage{iftex}
    \ifPDFTeX
    	\usepackage[T1]{fontenc}
    	\usepackage{mathpazo}
    \else
    	\usepackage{fontspec}
    \fi

    % Basic figure setup, for now with no caption control since it's done
    % automatically by Pandoc (which extracts ![](path) syntax from Markdown).
    \usepackage{graphicx}
    % Maintain compatibility with old templates. Remove in nbconvert 6.0
    \let\Oldincludegraphics\includegraphics
    % Ensure that by default, figures have no caption (until we provide a
    % proper Figure object with a Caption API and a way to capture that
    % in the conversion process - todo).
    \usepackage{caption}
    \DeclareCaptionFormat{nocaption}{}
    \captionsetup{format=nocaption,aboveskip=0pt,belowskip=0pt}

    \usepackage[Export]{adjustbox} % Used to constrain images to a maximum size
    \adjustboxset{max size={0.9\linewidth}{0.9\paperheight}}
    \usepackage{float}
    \floatplacement{figure}{H} % forces figures to be placed at the correct location
    \usepackage{xcolor} % Allow colors to be defined
    \usepackage{enumerate} % Needed for markdown enumerations to work
    \usepackage{geometry} % Used to adjust the document margins
    \usepackage{amsmath} % Equations
    \usepackage{amssymb} % Equations
    \usepackage{textcomp} % defines textquotesingle
    % Hack from http://tex.stackexchange.com/a/47451/13684:
    \AtBeginDocument{%
        \def\PYZsq{\textquotesingle}% Upright quotes in Pygmentized code
    }
    \usepackage{upquote} % Upright quotes for verbatim code
    \usepackage{eurosym} % defines \euro
    \usepackage[mathletters]{ucs} % Extended unicode (utf-8) support
    \usepackage{fancyvrb} % verbatim replacement that allows latex
    \usepackage{grffile} % extends the file name processing of package graphics 
                         % to support a larger range
    \makeatletter % fix for grffile with XeLaTeX
    \def\Gread@@xetex#1{%
      \IfFileExists{"\Gin@base".bb}%
      {\Gread@eps{\Gin@base.bb}}%
      {\Gread@@xetex@aux#1}%
    }
    \makeatother

    % The hyperref package gives us a pdf with properly built
    % internal navigation ('pdf bookmarks' for the table of contents,
    % internal cross-reference links, web links for URLs, etc.)
    \usepackage{hyperref}
    % The default LaTeX title has an obnoxious amount of whitespace. By default,
    % titling removes some of it. It also provides customization options.
    \usepackage{titling}
    \usepackage{longtable} % longtable support required by pandoc >1.10
    \usepackage{booktabs}  % table support for pandoc > 1.12.2
    \usepackage[inline]{enumitem} % IRkernel/repr support (it uses the enumerate* environment)
    \usepackage[normalem]{ulem} % ulem is needed to support strikethroughs (\sout)
                                % normalem makes italics be italics, not underlines
    \usepackage{mathrsfs}
    

    
    % Colors for the hyperref package
    \definecolor{urlcolor}{rgb}{0,.145,.698}
    \definecolor{linkcolor}{rgb}{.71,0.21,0.01}
    \definecolor{citecolor}{rgb}{.12,.54,.11}

    % ANSI colors
    \definecolor{ansi-black}{HTML}{3E424D}
    \definecolor{ansi-black-intense}{HTML}{282C36}
    \definecolor{ansi-red}{HTML}{E75C58}
    \definecolor{ansi-red-intense}{HTML}{B22B31}
    \definecolor{ansi-green}{HTML}{00A250}
    \definecolor{ansi-green-intense}{HTML}{007427}
    \definecolor{ansi-yellow}{HTML}{DDB62B}
    \definecolor{ansi-yellow-intense}{HTML}{B27D12}
    \definecolor{ansi-blue}{HTML}{208FFB}
    \definecolor{ansi-blue-intense}{HTML}{0065CA}
    \definecolor{ansi-magenta}{HTML}{D160C4}
    \definecolor{ansi-magenta-intense}{HTML}{A03196}
    \definecolor{ansi-cyan}{HTML}{60C6C8}
    \definecolor{ansi-cyan-intense}{HTML}{258F8F}
    \definecolor{ansi-white}{HTML}{C5C1B4}
    \definecolor{ansi-white-intense}{HTML}{A1A6B2}
    \definecolor{ansi-default-inverse-fg}{HTML}{FFFFFF}
    \definecolor{ansi-default-inverse-bg}{HTML}{000000}

    % commands and environments needed by pandoc snippets
    % extracted from the output of `pandoc -s`
    \providecommand{\tightlist}{%
      \setlength{\itemsep}{0pt}\setlength{\parskip}{0pt}}
    \DefineVerbatimEnvironment{Highlighting}{Verbatim}{commandchars=\\\{\}}
    % Add ',fontsize=\small' for more characters per line
    \newenvironment{Shaded}{}{}
    \newcommand{\KeywordTok}[1]{\textcolor[rgb]{0.00,0.44,0.13}{\textbf{{#1}}}}
    \newcommand{\DataTypeTok}[1]{\textcolor[rgb]{0.56,0.13,0.00}{{#1}}}
    \newcommand{\DecValTok}[1]{\textcolor[rgb]{0.25,0.63,0.44}{{#1}}}
    \newcommand{\BaseNTok}[1]{\textcolor[rgb]{0.25,0.63,0.44}{{#1}}}
    \newcommand{\FloatTok}[1]{\textcolor[rgb]{0.25,0.63,0.44}{{#1}}}
    \newcommand{\CharTok}[1]{\textcolor[rgb]{0.25,0.44,0.63}{{#1}}}
    \newcommand{\StringTok}[1]{\textcolor[rgb]{0.25,0.44,0.63}{{#1}}}
    \newcommand{\CommentTok}[1]{\textcolor[rgb]{0.38,0.63,0.69}{\textit{{#1}}}}
    \newcommand{\OtherTok}[1]{\textcolor[rgb]{0.00,0.44,0.13}{{#1}}}
    \newcommand{\AlertTok}[1]{\textcolor[rgb]{1.00,0.00,0.00}{\textbf{{#1}}}}
    \newcommand{\FunctionTok}[1]{\textcolor[rgb]{0.02,0.16,0.49}{{#1}}}
    \newcommand{\RegionMarkerTok}[1]{{#1}}
    \newcommand{\ErrorTok}[1]{\textcolor[rgb]{1.00,0.00,0.00}{\textbf{{#1}}}}
    \newcommand{\NormalTok}[1]{{#1}}
    
    % Additional commands for more recent versions of Pandoc
    \newcommand{\ConstantTok}[1]{\textcolor[rgb]{0.53,0.00,0.00}{{#1}}}
    \newcommand{\SpecialCharTok}[1]{\textcolor[rgb]{0.25,0.44,0.63}{{#1}}}
    \newcommand{\VerbatimStringTok}[1]{\textcolor[rgb]{0.25,0.44,0.63}{{#1}}}
    \newcommand{\SpecialStringTok}[1]{\textcolor[rgb]{0.73,0.40,0.53}{{#1}}}
    \newcommand{\ImportTok}[1]{{#1}}
    \newcommand{\DocumentationTok}[1]{\textcolor[rgb]{0.73,0.13,0.13}{\textit{{#1}}}}
    \newcommand{\AnnotationTok}[1]{\textcolor[rgb]{0.38,0.63,0.69}{\textbf{\textit{{#1}}}}}
    \newcommand{\CommentVarTok}[1]{\textcolor[rgb]{0.38,0.63,0.69}{\textbf{\textit{{#1}}}}}
    \newcommand{\VariableTok}[1]{\textcolor[rgb]{0.10,0.09,0.49}{{#1}}}
    \newcommand{\ControlFlowTok}[1]{\textcolor[rgb]{0.00,0.44,0.13}{\textbf{{#1}}}}
    \newcommand{\OperatorTok}[1]{\textcolor[rgb]{0.40,0.40,0.40}{{#1}}}
    \newcommand{\BuiltInTok}[1]{{#1}}
    \newcommand{\ExtensionTok}[1]{{#1}}
    \newcommand{\PreprocessorTok}[1]{\textcolor[rgb]{0.74,0.48,0.00}{{#1}}}
    \newcommand{\AttributeTok}[1]{\textcolor[rgb]{0.49,0.56,0.16}{{#1}}}
    \newcommand{\InformationTok}[1]{\textcolor[rgb]{0.38,0.63,0.69}{\textbf{\textit{{#1}}}}}
    \newcommand{\WarningTok}[1]{\textcolor[rgb]{0.38,0.63,0.69}{\textbf{\textit{{#1}}}}}
    
    
    % Define a nice break command that doesn't care if a line doesn't already
    % exist.
    \def\br{\hspace*{\fill} \\* }
    % Math Jax compatibility definitions
    \def\gt{>}
    \def\lt{<}
    \let\Oldtex\TeX
    \let\Oldlatex\LaTeX
    \renewcommand{\TeX}{\textrm{\Oldtex}}
    \renewcommand{\LaTeX}{\textrm{\Oldlatex}}
    % Document parameters
    % Document title
    \title{lecture\_3}
    
    
    
    
    
% Pygments definitions
\makeatletter
\def\PY@reset{\let\PY@it=\relax \let\PY@bf=\relax%
    \let\PY@ul=\relax \let\PY@tc=\relax%
    \let\PY@bc=\relax \let\PY@ff=\relax}
\def\PY@tok#1{\csname PY@tok@#1\endcsname}
\def\PY@toks#1+{\ifx\relax#1\empty\else%
    \PY@tok{#1}\expandafter\PY@toks\fi}
\def\PY@do#1{\PY@bc{\PY@tc{\PY@ul{%
    \PY@it{\PY@bf{\PY@ff{#1}}}}}}}
\def\PY#1#2{\PY@reset\PY@toks#1+\relax+\PY@do{#2}}

\expandafter\def\csname PY@tok@w\endcsname{\def\PY@tc##1{\textcolor[rgb]{0.73,0.73,0.73}{##1}}}
\expandafter\def\csname PY@tok@c\endcsname{\let\PY@it=\textit\def\PY@tc##1{\textcolor[rgb]{0.25,0.50,0.50}{##1}}}
\expandafter\def\csname PY@tok@cp\endcsname{\def\PY@tc##1{\textcolor[rgb]{0.74,0.48,0.00}{##1}}}
\expandafter\def\csname PY@tok@k\endcsname{\let\PY@bf=\textbf\def\PY@tc##1{\textcolor[rgb]{0.00,0.50,0.00}{##1}}}
\expandafter\def\csname PY@tok@kp\endcsname{\def\PY@tc##1{\textcolor[rgb]{0.00,0.50,0.00}{##1}}}
\expandafter\def\csname PY@tok@kt\endcsname{\def\PY@tc##1{\textcolor[rgb]{0.69,0.00,0.25}{##1}}}
\expandafter\def\csname PY@tok@o\endcsname{\def\PY@tc##1{\textcolor[rgb]{0.40,0.40,0.40}{##1}}}
\expandafter\def\csname PY@tok@ow\endcsname{\let\PY@bf=\textbf\def\PY@tc##1{\textcolor[rgb]{0.67,0.13,1.00}{##1}}}
\expandafter\def\csname PY@tok@nb\endcsname{\def\PY@tc##1{\textcolor[rgb]{0.00,0.50,0.00}{##1}}}
\expandafter\def\csname PY@tok@nf\endcsname{\def\PY@tc##1{\textcolor[rgb]{0.00,0.00,1.00}{##1}}}
\expandafter\def\csname PY@tok@nc\endcsname{\let\PY@bf=\textbf\def\PY@tc##1{\textcolor[rgb]{0.00,0.00,1.00}{##1}}}
\expandafter\def\csname PY@tok@nn\endcsname{\let\PY@bf=\textbf\def\PY@tc##1{\textcolor[rgb]{0.00,0.00,1.00}{##1}}}
\expandafter\def\csname PY@tok@ne\endcsname{\let\PY@bf=\textbf\def\PY@tc##1{\textcolor[rgb]{0.82,0.25,0.23}{##1}}}
\expandafter\def\csname PY@tok@nv\endcsname{\def\PY@tc##1{\textcolor[rgb]{0.10,0.09,0.49}{##1}}}
\expandafter\def\csname PY@tok@no\endcsname{\def\PY@tc##1{\textcolor[rgb]{0.53,0.00,0.00}{##1}}}
\expandafter\def\csname PY@tok@nl\endcsname{\def\PY@tc##1{\textcolor[rgb]{0.63,0.63,0.00}{##1}}}
\expandafter\def\csname PY@tok@ni\endcsname{\let\PY@bf=\textbf\def\PY@tc##1{\textcolor[rgb]{0.60,0.60,0.60}{##1}}}
\expandafter\def\csname PY@tok@na\endcsname{\def\PY@tc##1{\textcolor[rgb]{0.49,0.56,0.16}{##1}}}
\expandafter\def\csname PY@tok@nt\endcsname{\let\PY@bf=\textbf\def\PY@tc##1{\textcolor[rgb]{0.00,0.50,0.00}{##1}}}
\expandafter\def\csname PY@tok@nd\endcsname{\def\PY@tc##1{\textcolor[rgb]{0.67,0.13,1.00}{##1}}}
\expandafter\def\csname PY@tok@s\endcsname{\def\PY@tc##1{\textcolor[rgb]{0.73,0.13,0.13}{##1}}}
\expandafter\def\csname PY@tok@sd\endcsname{\let\PY@it=\textit\def\PY@tc##1{\textcolor[rgb]{0.73,0.13,0.13}{##1}}}
\expandafter\def\csname PY@tok@si\endcsname{\let\PY@bf=\textbf\def\PY@tc##1{\textcolor[rgb]{0.73,0.40,0.53}{##1}}}
\expandafter\def\csname PY@tok@se\endcsname{\let\PY@bf=\textbf\def\PY@tc##1{\textcolor[rgb]{0.73,0.40,0.13}{##1}}}
\expandafter\def\csname PY@tok@sr\endcsname{\def\PY@tc##1{\textcolor[rgb]{0.73,0.40,0.53}{##1}}}
\expandafter\def\csname PY@tok@ss\endcsname{\def\PY@tc##1{\textcolor[rgb]{0.10,0.09,0.49}{##1}}}
\expandafter\def\csname PY@tok@sx\endcsname{\def\PY@tc##1{\textcolor[rgb]{0.00,0.50,0.00}{##1}}}
\expandafter\def\csname PY@tok@m\endcsname{\def\PY@tc##1{\textcolor[rgb]{0.40,0.40,0.40}{##1}}}
\expandafter\def\csname PY@tok@gh\endcsname{\let\PY@bf=\textbf\def\PY@tc##1{\textcolor[rgb]{0.00,0.00,0.50}{##1}}}
\expandafter\def\csname PY@tok@gu\endcsname{\let\PY@bf=\textbf\def\PY@tc##1{\textcolor[rgb]{0.50,0.00,0.50}{##1}}}
\expandafter\def\csname PY@tok@gd\endcsname{\def\PY@tc##1{\textcolor[rgb]{0.63,0.00,0.00}{##1}}}
\expandafter\def\csname PY@tok@gi\endcsname{\def\PY@tc##1{\textcolor[rgb]{0.00,0.63,0.00}{##1}}}
\expandafter\def\csname PY@tok@gr\endcsname{\def\PY@tc##1{\textcolor[rgb]{1.00,0.00,0.00}{##1}}}
\expandafter\def\csname PY@tok@ge\endcsname{\let\PY@it=\textit}
\expandafter\def\csname PY@tok@gs\endcsname{\let\PY@bf=\textbf}
\expandafter\def\csname PY@tok@gp\endcsname{\let\PY@bf=\textbf\def\PY@tc##1{\textcolor[rgb]{0.00,0.00,0.50}{##1}}}
\expandafter\def\csname PY@tok@go\endcsname{\def\PY@tc##1{\textcolor[rgb]{0.53,0.53,0.53}{##1}}}
\expandafter\def\csname PY@tok@gt\endcsname{\def\PY@tc##1{\textcolor[rgb]{0.00,0.27,0.87}{##1}}}
\expandafter\def\csname PY@tok@err\endcsname{\def\PY@bc##1{\setlength{\fboxsep}{0pt}\fcolorbox[rgb]{1.00,0.00,0.00}{1,1,1}{\strut ##1}}}
\expandafter\def\csname PY@tok@kc\endcsname{\let\PY@bf=\textbf\def\PY@tc##1{\textcolor[rgb]{0.00,0.50,0.00}{##1}}}
\expandafter\def\csname PY@tok@kd\endcsname{\let\PY@bf=\textbf\def\PY@tc##1{\textcolor[rgb]{0.00,0.50,0.00}{##1}}}
\expandafter\def\csname PY@tok@kn\endcsname{\let\PY@bf=\textbf\def\PY@tc##1{\textcolor[rgb]{0.00,0.50,0.00}{##1}}}
\expandafter\def\csname PY@tok@kr\endcsname{\let\PY@bf=\textbf\def\PY@tc##1{\textcolor[rgb]{0.00,0.50,0.00}{##1}}}
\expandafter\def\csname PY@tok@bp\endcsname{\def\PY@tc##1{\textcolor[rgb]{0.00,0.50,0.00}{##1}}}
\expandafter\def\csname PY@tok@fm\endcsname{\def\PY@tc##1{\textcolor[rgb]{0.00,0.00,1.00}{##1}}}
\expandafter\def\csname PY@tok@vc\endcsname{\def\PY@tc##1{\textcolor[rgb]{0.10,0.09,0.49}{##1}}}
\expandafter\def\csname PY@tok@vg\endcsname{\def\PY@tc##1{\textcolor[rgb]{0.10,0.09,0.49}{##1}}}
\expandafter\def\csname PY@tok@vi\endcsname{\def\PY@tc##1{\textcolor[rgb]{0.10,0.09,0.49}{##1}}}
\expandafter\def\csname PY@tok@vm\endcsname{\def\PY@tc##1{\textcolor[rgb]{0.10,0.09,0.49}{##1}}}
\expandafter\def\csname PY@tok@sa\endcsname{\def\PY@tc##1{\textcolor[rgb]{0.73,0.13,0.13}{##1}}}
\expandafter\def\csname PY@tok@sb\endcsname{\def\PY@tc##1{\textcolor[rgb]{0.73,0.13,0.13}{##1}}}
\expandafter\def\csname PY@tok@sc\endcsname{\def\PY@tc##1{\textcolor[rgb]{0.73,0.13,0.13}{##1}}}
\expandafter\def\csname PY@tok@dl\endcsname{\def\PY@tc##1{\textcolor[rgb]{0.73,0.13,0.13}{##1}}}
\expandafter\def\csname PY@tok@s2\endcsname{\def\PY@tc##1{\textcolor[rgb]{0.73,0.13,0.13}{##1}}}
\expandafter\def\csname PY@tok@sh\endcsname{\def\PY@tc##1{\textcolor[rgb]{0.73,0.13,0.13}{##1}}}
\expandafter\def\csname PY@tok@s1\endcsname{\def\PY@tc##1{\textcolor[rgb]{0.73,0.13,0.13}{##1}}}
\expandafter\def\csname PY@tok@mb\endcsname{\def\PY@tc##1{\textcolor[rgb]{0.40,0.40,0.40}{##1}}}
\expandafter\def\csname PY@tok@mf\endcsname{\def\PY@tc##1{\textcolor[rgb]{0.40,0.40,0.40}{##1}}}
\expandafter\def\csname PY@tok@mh\endcsname{\def\PY@tc##1{\textcolor[rgb]{0.40,0.40,0.40}{##1}}}
\expandafter\def\csname PY@tok@mi\endcsname{\def\PY@tc##1{\textcolor[rgb]{0.40,0.40,0.40}{##1}}}
\expandafter\def\csname PY@tok@il\endcsname{\def\PY@tc##1{\textcolor[rgb]{0.40,0.40,0.40}{##1}}}
\expandafter\def\csname PY@tok@mo\endcsname{\def\PY@tc##1{\textcolor[rgb]{0.40,0.40,0.40}{##1}}}
\expandafter\def\csname PY@tok@ch\endcsname{\let\PY@it=\textit\def\PY@tc##1{\textcolor[rgb]{0.25,0.50,0.50}{##1}}}
\expandafter\def\csname PY@tok@cm\endcsname{\let\PY@it=\textit\def\PY@tc##1{\textcolor[rgb]{0.25,0.50,0.50}{##1}}}
\expandafter\def\csname PY@tok@cpf\endcsname{\let\PY@it=\textit\def\PY@tc##1{\textcolor[rgb]{0.25,0.50,0.50}{##1}}}
\expandafter\def\csname PY@tok@c1\endcsname{\let\PY@it=\textit\def\PY@tc##1{\textcolor[rgb]{0.25,0.50,0.50}{##1}}}
\expandafter\def\csname PY@tok@cs\endcsname{\let\PY@it=\textit\def\PY@tc##1{\textcolor[rgb]{0.25,0.50,0.50}{##1}}}

\def\PYZbs{\char`\\}
\def\PYZus{\char`\_}
\def\PYZob{\char`\{}
\def\PYZcb{\char`\}}
\def\PYZca{\char`\^}
\def\PYZam{\char`\&}
\def\PYZlt{\char`\<}
\def\PYZgt{\char`\>}
\def\PYZsh{\char`\#}
\def\PYZpc{\char`\%}
\def\PYZdl{\char`\$}
\def\PYZhy{\char`\-}
\def\PYZsq{\char`\'}
\def\PYZdq{\char`\"}
\def\PYZti{\char`\~}
% for compatibility with earlier versions
\def\PYZat{@}
\def\PYZlb{[}
\def\PYZrb{]}
\makeatother


    % For linebreaks inside Verbatim environment from package fancyvrb. 
    \makeatletter
        \newbox\Wrappedcontinuationbox 
        \newbox\Wrappedvisiblespacebox 
        \newcommand*\Wrappedvisiblespace {\textcolor{red}{\textvisiblespace}} 
        \newcommand*\Wrappedcontinuationsymbol {\textcolor{red}{\llap{\tiny$\m@th\hookrightarrow$}}} 
        \newcommand*\Wrappedcontinuationindent {3ex } 
        \newcommand*\Wrappedafterbreak {\kern\Wrappedcontinuationindent\copy\Wrappedcontinuationbox} 
        % Take advantage of the already applied Pygments mark-up to insert 
        % potential linebreaks for TeX processing. 
        %        {, <, #, %, $, ' and ": go to next line. 
        %        _, }, ^, &, >, - and ~: stay at end of broken line. 
        % Use of \textquotesingle for straight quote. 
        \newcommand*\Wrappedbreaksatspecials {% 
            \def\PYGZus{\discretionary{\char`\_}{\Wrappedafterbreak}{\char`\_}}% 
            \def\PYGZob{\discretionary{}{\Wrappedafterbreak\char`\{}{\char`\{}}% 
            \def\PYGZcb{\discretionary{\char`\}}{\Wrappedafterbreak}{\char`\}}}% 
            \def\PYGZca{\discretionary{\char`\^}{\Wrappedafterbreak}{\char`\^}}% 
            \def\PYGZam{\discretionary{\char`\&}{\Wrappedafterbreak}{\char`\&}}% 
            \def\PYGZlt{\discretionary{}{\Wrappedafterbreak\char`\<}{\char`\<}}% 
            \def\PYGZgt{\discretionary{\char`\>}{\Wrappedafterbreak}{\char`\>}}% 
            \def\PYGZsh{\discretionary{}{\Wrappedafterbreak\char`\#}{\char`\#}}% 
            \def\PYGZpc{\discretionary{}{\Wrappedafterbreak\char`\%}{\char`\%}}% 
            \def\PYGZdl{\discretionary{}{\Wrappedafterbreak\char`\$}{\char`\$}}% 
            \def\PYGZhy{\discretionary{\char`\-}{\Wrappedafterbreak}{\char`\-}}% 
            \def\PYGZsq{\discretionary{}{\Wrappedafterbreak\textquotesingle}{\textquotesingle}}% 
            \def\PYGZdq{\discretionary{}{\Wrappedafterbreak\char`\"}{\char`\"}}% 
            \def\PYGZti{\discretionary{\char`\~}{\Wrappedafterbreak}{\char`\~}}% 
        } 
        % Some characters . , ; ? ! / are not pygmentized. 
        % This macro makes them "active" and they will insert potential linebreaks 
        \newcommand*\Wrappedbreaksatpunct {% 
            \lccode`\~`\.\lowercase{\def~}{\discretionary{\hbox{\char`\.}}{\Wrappedafterbreak}{\hbox{\char`\.}}}% 
            \lccode`\~`\,\lowercase{\def~}{\discretionary{\hbox{\char`\,}}{\Wrappedafterbreak}{\hbox{\char`\,}}}% 
            \lccode`\~`\;\lowercase{\def~}{\discretionary{\hbox{\char`\;}}{\Wrappedafterbreak}{\hbox{\char`\;}}}% 
            \lccode`\~`\:\lowercase{\def~}{\discretionary{\hbox{\char`\:}}{\Wrappedafterbreak}{\hbox{\char`\:}}}% 
            \lccode`\~`\?\lowercase{\def~}{\discretionary{\hbox{\char`\?}}{\Wrappedafterbreak}{\hbox{\char`\?}}}% 
            \lccode`\~`\!\lowercase{\def~}{\discretionary{\hbox{\char`\!}}{\Wrappedafterbreak}{\hbox{\char`\!}}}% 
            \lccode`\~`\/\lowercase{\def~}{\discretionary{\hbox{\char`\/}}{\Wrappedafterbreak}{\hbox{\char`\/}}}% 
            \catcode`\.\active
            \catcode`\,\active 
            \catcode`\;\active
            \catcode`\:\active
            \catcode`\?\active
            \catcode`\!\active
            \catcode`\/\active 
            \lccode`\~`\~ 	
        }
    \makeatother

    \let\OriginalVerbatim=\Verbatim
    \makeatletter
    \renewcommand{\Verbatim}[1][1]{%
        %\parskip\z@skip
        \sbox\Wrappedcontinuationbox {\Wrappedcontinuationsymbol}%
        \sbox\Wrappedvisiblespacebox {\FV@SetupFont\Wrappedvisiblespace}%
        \def\FancyVerbFormatLine ##1{\hsize\linewidth
            \vtop{\raggedright\hyphenpenalty\z@\exhyphenpenalty\z@
                \doublehyphendemerits\z@\finalhyphendemerits\z@
                \strut ##1\strut}%
        }%
        % If the linebreak is at a space, the latter will be displayed as visible
        % space at end of first line, and a continuation symbol starts next line.
        % Stretch/shrink are however usually zero for typewriter font.
        \def\FV@Space {%
            \nobreak\hskip\z@ plus\fontdimen3\font minus\fontdimen4\font
            \discretionary{\copy\Wrappedvisiblespacebox}{\Wrappedafterbreak}
            {\kern\fontdimen2\font}%
        }%
        
        % Allow breaks at special characters using \PYG... macros.
        \Wrappedbreaksatspecials
        % Breaks at punctuation characters . , ; ? ! and / need catcode=\active 	
        \OriginalVerbatim[#1,codes*=\Wrappedbreaksatpunct]%
    }
    \makeatother

    % Exact colors from NB
    \definecolor{incolor}{HTML}{303F9F}
    \definecolor{outcolor}{HTML}{D84315}
    \definecolor{cellborder}{HTML}{CFCFCF}
    \definecolor{cellbackground}{HTML}{F7F7F7}
    
    % prompt
    \makeatletter
    \newcommand{\boxspacing}{\kern\kvtcb@left@rule\kern\kvtcb@boxsep}
    \makeatother
    \newcommand{\prompt}[4]{
        \ttfamily\llap{{\color{#2}[#3]:\hspace{3pt}#4}}\vspace{-\baselineskip}
    }
    

    
    % Prevent overflowing lines due to hard-to-break entities
    \sloppy 
    % Setup hyperref package
    \hypersetup{
      breaklinks=true,  % so long urls are correctly broken across lines
      colorlinks=true,
      urlcolor=urlcolor,
      linkcolor=linkcolor,
      citecolor=citecolor,
      }
    % Slightly bigger margins than the latex defaults
    
    \geometry{verbose,tmargin=1in,bmargin=1in,lmargin=1in,rmargin=1in}
    
    

\begin{document}
    
    \maketitle
    
    

    
    \hypertarget{interpolation---practical-lesson-3}{%
\section{Interpolation - Practical Lesson
3}\label{interpolation---practical-lesson-3}}

\hypertarget{linear-interpolation}{%
\subsection{Linear interpolation}\label{linear-interpolation}}

Interpolation is a method of constructing new points within the range of
a discrete set of known data points.

It may happen to have few data points, obtained by sampling or
experimenting. These points represent the values of a function \(f(x)\),
where \(x\) is an independent variable (e.g.~in recording a trip:
distances at certain times, \(d = f(t)\)). It is often required to
estimate the value of that function for an intermediate value of the
independent variable (e.g.~in our previous example what is the distance
\(d\) at a new time \(t\) ?).

Let's exercise on linear interpolation with a couple of examples.

\hypertarget{example-1}{%
\paragraph{Example 1}\label{example-1}}

Assume you are going on holidays by car and that luckily there isn't
much traffic so that you can drive at constant speed (which gives a
linear relation between travelled space and time i.e.~\(s = v \cdot t\),
which means that if you plot the distances \(s\) as a function of the
time \(t\) you get a line with slope \(v\)). Given two samples of the
car travelled distance \(s_1\) and \(s_2\) taken at two different times
\(t_1\) and \(t_2\) you can linearly interpolate to find your position
at different times using the following relations:

\[w = \frac{t - t_1}{t_2 - t_1}\] (\(t\) generic time at which we want
to know the distance \(s\))

\[s = (1 - w)\cdot s_1 + w \cdot s_2\]

\textbf{\emph{Derivation}} The equation of a line for two points
\((t_1, s_1)\) and \((t_2, s_2)\) can be written as:

\[\frac{t - t_1}{t_2 - t_1} = \frac{s - s_1}{s_2 - s_1}\]

Setting \(w = \frac{t - t_1}{t_2 - t_1}\) and solving for \(s\) we find
the desired solution:

\[w = \frac{s - s_1}{s_2 - s_1} \Rightarrow (s_2 - s_1)\cdot w = s - s_1 \Rightarrow ...\]

Back to our example, if
\(s_1 = 25.75~\mathrm{km}\;(@t_1 = 15~\mathrm{min})\) and
\(s_2 = 171.7~\mathrm{km}\;(@t_2 = 100~\mathrm{min})\) let's compute:

    \begin{tcolorbox}[breakable, size=fbox, boxrule=1pt, pad at break*=1mm,colback=cellbackground, colframe=cellborder]
\prompt{In}{incolor}{1}{\boxspacing}
\begin{Verbatim}[commandchars=\\\{\}]
\PY{c+c1}{\PYZsh{} let\PYZsq{}s find distance travelled in 1 hour (interpolation)}

\PY{n}{s\PYZus{}1} \PY{o}{=} \PY{l+m+mf}{25.75} \PY{c+c1}{\PYZsh{} distance in km}
\PY{n}{t\PYZus{}1} \PY{o}{=} \PY{l+m+mi}{15}    \PY{c+c1}{\PYZsh{} elapsed time in minutes}
\PY{n}{s\PYZus{}2} \PY{o}{=} \PY{l+m+mf}{171.7}
\PY{n}{t\PYZus{}2} \PY{o}{=} \PY{l+m+mi}{100}

\PY{n}{t} \PY{o}{=} \PY{l+m+mi}{60}

\PY{n}{w} \PY{o}{=} \PY{p}{(}\PY{n}{t} \PY{o}{\PYZhy{}} \PY{n}{t\PYZus{}1}\PY{p}{)}\PY{o}{/}\PY{p}{(}\PY{n}{t\PYZus{}2} \PY{o}{\PYZhy{}} \PY{n}{t\PYZus{}1}\PY{p}{)}
\PY{n}{s} \PY{o}{=} \PY{p}{(}\PY{l+m+mi}{1} \PY{o}{\PYZhy{}} \PY{n}{w}\PY{p}{)}\PY{o}{*}\PY{n}{s\PYZus{}1} \PY{o}{+} \PY{n}{w}\PY{o}{*}\PY{n}{s\PYZus{}2}

\PY{n+nb}{print} \PY{p}{(}\PY{l+s+s2}{\PYZdq{}}\PY{l+s+si}{\PYZob{}:.1f\PYZcb{}}\PY{l+s+s2}{ km}\PY{l+s+s2}{\PYZdq{}}\PY{o}{.}\PY{n}{format}\PY{p}{(}\PY{n}{s}\PY{p}{)}\PY{p}{)}
\end{Verbatim}
\end{tcolorbox}

    \begin{Verbatim}[commandchars=\\\{\}]
103.0 km
    \end{Verbatim}

    If we believe the relation between our variable stays the same, we can
use the same formula to extrapolate values outside our initial sample.

For example if we keep the same constant velocity in our trip we could
check:

    \begin{tcolorbox}[breakable, size=fbox, boxrule=1pt, pad at break*=1mm,colback=cellbackground, colframe=cellborder]
\prompt{In}{incolor}{2}{\boxspacing}
\begin{Verbatim}[commandchars=\\\{\}]
\PY{c+c1}{\PYZsh{} distance travelled in a 3 hour trip (extrapolation)}

\PY{n}{s\PYZus{}1} \PY{o}{=} \PY{l+m+mf}{25.75} \PY{c+c1}{\PYZsh{} distance in km}
\PY{n}{t\PYZus{}1} \PY{o}{=} \PY{l+m+mi}{15}    \PY{c+c1}{\PYZsh{} elapsed time in minutes}
\PY{n}{s\PYZus{}2} \PY{o}{=} \PY{l+m+mf}{171.7}
\PY{n}{t\PYZus{}2} \PY{o}{=} \PY{l+m+mi}{100}

\PY{n}{t} \PY{o}{=} \PY{l+m+mi}{180}

\PY{n}{w} \PY{o}{=} \PY{p}{(}\PY{n}{t} \PY{o}{\PYZhy{}} \PY{n}{t\PYZus{}1}\PY{p}{)}\PY{o}{/}\PY{p}{(}\PY{n}{t\PYZus{}2} \PY{o}{\PYZhy{}} \PY{n}{t\PYZus{}1}\PY{p}{)}
\PY{n}{s} \PY{o}{=} \PY{p}{(}\PY{l+m+mi}{1} \PY{o}{\PYZhy{}} \PY{n}{w}\PY{p}{)}\PY{o}{*}\PY{n}{s\PYZus{}1} \PY{o}{+} \PY{n}{w}\PY{o}{*}\PY{n}{s\PYZus{}2}

\PY{n+nb}{print} \PY{p}{(}\PY{l+s+s2}{\PYZdq{}}\PY{l+s+si}{\PYZob{}:.1f\PYZcb{}}\PY{l+s+s2}{ km}\PY{l+s+s2}{\PYZdq{}}\PY{o}{.}\PY{n}{format}\PY{p}{(}\PY{n}{s}\PY{p}{)}\PY{p}{)}
\end{Verbatim}
\end{tcolorbox}

    \begin{Verbatim}[commandchars=\\\{\}]
309.1 km
    \end{Verbatim}

    \hypertarget{log-linear-interpolation}{%
\subsubsection{Log-linear
interpolation}\label{log-linear-interpolation}}

When the variable we would like to interpolate has an exponential
relation with the unknown we can fall back to the previous case by
applying the logarithm. In this case the previous formulas apply again
except that at the end we have to exponentiate to get back the original
variable:

\[p = \mathrm{exp}(c \cdot h)\]

\[s = \mathrm{log}(p) = \mathrm{log}(\mathrm{exp}(c \cdot h)) = c \cdot h\]

\[w = \frac{h - h_1}{h_2 - h_1}\]

\[s = (1 - w)\cdot s_1 + w \cdot s_2\;\;(\mathrm{remember \;now }\;s = \mathrm{log}(p))\]

\[p = \mathrm{exp}(s)\]

Let's see a practical example.

\hypertarget{example-2}{%
\paragraph{Example 2}\label{example-2}}

Atmospheric pressure decreases with the altitude (i.e.~the highest I
flight the lower is the pressure) following an exponential law:

\[p = p_0\cdot e^{-\alpha h}\]

where

\begin{itemize}
\tightlist
\item
  \(h\) is the altitude
\item
  \(p_0\) is the pressure at sea level
\item
  \(\alpha\) is a constant
\end{itemize}

Taking the logarithm of each side of the equation I get a linear
relation which can be interpolated as before:

\[\tilde{s} = \mathrm{log}(p) = \mathrm{log}(p_0\cdot e^{-\alpha h})\propto - \alpha \cdot h\]

Now assume that we have measured
\(p_1 = 90~\mathrm{kPa}\;(h_1 = 1000~\mathrm{m})\) and
\(p_2 = 40~\mathrm{kPa}\;(h_1 = 7000~\mathrm{m})\) what will be the
atmospheric pressure on top of the Mont Blanc (\(4812~\mathrm{m}\)) ?
and on top of Mount Everest (\(8848~\mathrm{m}\)) ?

    \begin{tcolorbox}[breakable, size=fbox, boxrule=1pt, pad at break*=1mm,colback=cellbackground, colframe=cellborder]
\prompt{In}{incolor}{3}{\boxspacing}
\begin{Verbatim}[commandchars=\\\{\}]
\PY{c+c1}{\PYZsh{} pressure on top of the Mont Blanc (interpolation)}
\PY{k+kn}{from} \PY{n+nn}{math} \PY{k}{import} \PY{n}{log}\PY{p}{,} \PY{n}{exp}

\PY{c+c1}{\PYZsh{} first we take the logarithm of our measurements to use the linear }
\PY{c+c1}{\PYZsh{} relation to interpolate}
\PY{n}{h\PYZus{}1} \PY{o}{=} \PY{l+m+mi}{1000} \PY{c+c1}{\PYZsh{} height in meters}
\PY{n}{s\PYZus{}1} \PY{o}{=} \PY{n}{log}\PY{p}{(}\PY{l+m+mi}{90}\PY{p}{)} \PY{c+c1}{\PYZsh{} logarithm of the pressure at heigth h1}
\PY{n}{h\PYZus{}2} \PY{o}{=} \PY{l+m+mi}{7000} \PY{c+c1}{\PYZsh{} height in meters}
\PY{n}{s\PYZus{}2} \PY{o}{=} \PY{n}{log}\PY{p}{(}\PY{l+m+mi}{40}\PY{p}{)} \PY{c+c1}{\PYZsh{} logarithm of the pressure at heigth h2}

\PY{n}{h} \PY{o}{=} \PY{l+m+mi}{4812}

\PY{n}{w} \PY{o}{=} \PY{p}{(}\PY{n}{h} \PY{o}{\PYZhy{}} \PY{n}{h\PYZus{}1}\PY{p}{)}\PY{o}{/}\PY{p}{(}\PY{n}{h\PYZus{}2} \PY{o}{\PYZhy{}} \PY{n}{h\PYZus{}1}\PY{p}{)}
\PY{n}{s} \PY{o}{=} \PY{p}{(}\PY{l+m+mi}{1} \PY{o}{\PYZhy{}} \PY{n}{w}\PY{p}{)}\PY{o}{*}\PY{n}{s\PYZus{}1} \PY{o}{+} \PY{n}{w}\PY{o}{*}\PY{n}{s\PYZus{}2}

\PY{n+nb}{print} \PY{p}{(}\PY{l+s+s2}{\PYZdq{}}\PY{l+s+si}{\PYZob{}:.1f\PYZcb{}}\PY{l+s+s2}{ kPa}\PY{l+s+s2}{\PYZdq{}}\PY{o}{.}\PY{n}{format}\PY{p}{(}\PY{n}{exp}\PY{p}{(}\PY{n}{s}\PY{p}{)}\PY{p}{)}\PY{p}{)}
\end{Verbatim}
\end{tcolorbox}

    \begin{Verbatim}[commandchars=\\\{\}]
53.8 kPa
    \end{Verbatim}

    \begin{tcolorbox}[breakable, size=fbox, boxrule=1pt, pad at break*=1mm,colback=cellbackground, colframe=cellborder]
\prompt{In}{incolor}{4}{\boxspacing}
\begin{Verbatim}[commandchars=\\\{\}]
\PY{c+c1}{\PYZsh{} pressure on top of the Mount Everest (extrapolation)}
\PY{k+kn}{from} \PY{n+nn}{math} \PY{k}{import} \PY{n}{log}\PY{p}{,} \PY{n}{exp}

\PY{c+c1}{\PYZsh{} first we take the logarithm of our measurements to use the linear }
\PY{c+c1}{\PYZsh{} relation to interpolate}
\PY{n}{h\PYZus{}1} \PY{o}{=} \PY{l+m+mi}{1000} \PY{c+c1}{\PYZsh{} height in meters}
\PY{n}{s\PYZus{}1} \PY{o}{=} \PY{n}{log}\PY{p}{(}\PY{l+m+mi}{90}\PY{p}{)} \PY{c+c1}{\PYZsh{} logarithm of the pressure at heigth h1}
\PY{n}{h\PYZus{}2} \PY{o}{=} \PY{l+m+mi}{7000} \PY{c+c1}{\PYZsh{} height in meters}
\PY{n}{s\PYZus{}2} \PY{o}{=} \PY{n}{log}\PY{p}{(}\PY{l+m+mi}{40}\PY{p}{)} \PY{c+c1}{\PYZsh{} logarithm of the pressure at heigth h2}

\PY{n}{h} \PY{o}{=} \PY{l+m+mi}{8848}

\PY{n}{w} \PY{o}{=} \PY{p}{(}\PY{n}{h} \PY{o}{\PYZhy{}} \PY{n}{h\PYZus{}1}\PY{p}{)}\PY{o}{/}\PY{p}{(}\PY{n}{h\PYZus{}2} \PY{o}{\PYZhy{}} \PY{n}{h\PYZus{}1}\PY{p}{)}
\PY{n}{s} \PY{o}{=} \PY{p}{(}\PY{l+m+mi}{1} \PY{o}{\PYZhy{}} \PY{n}{w}\PY{p}{)}\PY{o}{*}\PY{n}{s\PYZus{}1} \PY{o}{+} \PY{n}{w}\PY{o}{*}\PY{n}{s\PYZus{}2}

\PY{n+nb}{print} \PY{p}{(}\PY{l+s+s2}{\PYZdq{}}\PY{l+s+si}{\PYZob{}:.1f\PYZcb{}}\PY{l+s+s2}{ kPa}\PY{l+s+s2}{\PYZdq{}}\PY{o}{.}\PY{n}{format}\PY{p}{(}\PY{n}{exp}\PY{p}{(}\PY{n}{s}\PY{p}{)}\PY{p}{)}\PY{p}{)}
\end{Verbatim}
\end{tcolorbox}

    \begin{Verbatim}[commandchars=\\\{\}]
31.2 kPa
    \end{Verbatim}

    \begin{figure}
\centering
\includegraphics{Atmospheric_Pressure_vs._Altitude.png}
\caption{Atmospheric pressure versus altitude (wikipedia). Green points
represent our measurements, red points represent
interpolation/extrapolation.}
\end{figure}

    \hypertarget{discount-curve-interpolation}{%
\subsection{Discount curve
interpolation}\label{discount-curve-interpolation}}

Now we can come back to finance and using what we have just learnt try
to write a function which interpolates some given discount factors.

Needed data:

\begin{itemize}
\tightlist
\item
  a list of pillars dates specifying the value dates of the given
  discount factors, \(t_0,...,t_{n-1}\)
\item
  a list of given discount factors, \(D(t_0),...,D(t_{n-1})\)
\item
  a pricing date (`today' date) which corresponds to \(t=0\)
\end{itemize}

The input argument to the function will be the value date at which we
want to interpolate the discount factor.

Since the discount factor can be expresses as \(D=e^{-r(T-t)}\) the
function will use a log-linear interpolation to return the value we are
looking for.

\[D(t) = \mathrm{exp}\Big( (1-w)\cdot \mathrm{ln}(D(t_i)) + w\cdot \mathrm{ln}(D(t_{i+1}))\Big);\;\;\;w=\frac{t-t_i}{t_{i+1}-t_i}\]

where \(i\) is such that \(t_i \le t \le t_{i+1}\). More technically we
can say that we are doing a linear interpolation over time in the log
space:

\[d(t_i):=\mathrm{ln}(D(t_i))\]

\[d(t) = (1-w)d(t_i) + wd(t_{i+1});\;\;\;w=\frac{t-t_i}{t_{i+1}-t_i}\]

\[D(t) = \mathrm{exp}(d(t))\]

where \(i\) is such that \(t_i \le t \le t_{i+1}\)

Instead of reinventing the wheel and perform the interpolation with our
own code, we'll use the function \texttt{interp} provided by the
\texttt{python} module \texttt{numpy}. So first let's try it with some
simple examples:

    \begin{tcolorbox}[breakable, size=fbox, boxrule=1pt, pad at break*=1mm,colback=cellbackground, colframe=cellborder]
\prompt{In}{incolor}{12}{\boxspacing}
\begin{Verbatim}[commandchars=\\\{\}]
\PY{c+c1}{\PYZsh{} the numpy.interp function linearly interpolates these points to }
\PY{c+c1}{\PYZsh{} estimate the value of f at other x coordinates. }
\PY{c+c1}{\PYZsh{} For example, say we want to interpolate the points at x = 2.5::}
\PY{k+kn}{import} \PY{n+nn}{numpy} \PY{k}{as} \PY{n+nn}{np}

\PY{n}{xp} \PY{o}{=} \PY{p}{[}\PY{l+m+mi}{0}\PY{p}{,} \PY{l+m+mi}{1}\PY{p}{,} \PY{l+m+mi}{5}\PY{p}{]}
\PY{n}{fp} \PY{o}{=} \PY{p}{[}\PY{l+m+mi}{0}\PY{p}{,} \PY{l+m+mi}{2}\PY{p}{,} \PY{l+m+mi}{4}\PY{p}{]}
\PY{n}{np}\PY{o}{.}\PY{n}{interp}\PY{p}{(}\PY{l+m+mf}{2.5}\PY{p}{,} \PY{n}{xp}\PY{p}{,} \PY{n}{fp}\PY{p}{)}
\end{Verbatim}
\end{tcolorbox}

            \begin{tcolorbox}[breakable, size=fbox, boxrule=.5pt, pad at break*=1mm, opacityfill=0]
\prompt{Out}{outcolor}{12}{\boxspacing}
\begin{Verbatim}[commandchars=\\\{\}]
2.75
\end{Verbatim}
\end{tcolorbox}
        
    \begin{tcolorbox}[breakable, size=fbox, boxrule=1pt, pad at break*=1mm,colback=cellbackground, colframe=cellborder]
\prompt{In}{incolor}{13}{\boxspacing}
\begin{Verbatim}[commandchars=\\\{\}]
\PY{c+c1}{\PYZsh{} import modules and objects that we need}
\PY{k+kn}{from} \PY{n+nn}{datetime} \PY{k}{import} \PY{n}{date}
\PY{k+kn}{import} \PY{n+nn}{numpy}\PY{o}{,} \PY{n+nn}{math}
\PY{k+kn}{from} \PY{n+nn}{matplotlib} \PY{k}{import} \PY{n}{pyplot} \PY{k}{as} \PY{n}{plt}
\PY{k+kn}{import} \PY{n+nn}{matplotlib}\PY{n+nn}{.}\PY{n+nn}{dates} \PY{k}{as} \PY{n+nn}{mdates} 
\PY{c+c1}{\PYZsh{} with this notation we tell python to use mdates as an alias }
\PY{c+c1}{\PYZsh{} for matplotlib.dates, I told you I\PYZsq{}m lazy...}

\PY{c+c1}{\PYZsh{} define the input data}
\PY{n}{today\PYZus{}date} \PY{o}{=} \PY{n}{date}\PY{p}{(}\PY{l+m+mi}{2019}\PY{p}{,} \PY{l+m+mi}{10}\PY{p}{,} \PY{l+m+mi}{1}\PY{p}{)}

\PY{n}{pillar\PYZus{}dates} \PY{o}{=} \PY{p}{[}\PY{n}{date}\PY{p}{(}\PY{l+m+mi}{2019}\PY{p}{,} \PY{l+m+mi}{10}\PY{p}{,} \PY{l+m+mi}{1}\PY{p}{)}\PY{p}{,} \PY{n}{date}\PY{p}{(}\PY{l+m+mi}{2020}\PY{p}{,} \PY{l+m+mi}{10}\PY{p}{,} \PY{l+m+mi}{1}\PY{p}{)}\PY{p}{,} \PY{n}{date}\PY{p}{(}\PY{l+m+mi}{2021}\PY{p}{,} \PY{l+m+mi}{10}\PY{p}{,} \PY{l+m+mi}{1}\PY{p}{)}\PY{p}{]}
\PY{n}{discount\PYZus{}factors} \PY{o}{=} \PY{p}{[}\PY{l+m+mf}{1.0}\PY{p}{,} \PY{l+m+mf}{0.97}\PY{p}{,} \PY{l+m+mf}{0.72}\PY{p}{]}

\PY{c+c1}{\PYZsh{} let\PYZsq{}s see what this looks like when plotted on a graph}
\PY{c+c1}{\PYZsh{} here a more complicated usage of matplotlib to}
\PY{c+c1}{\PYZsh{} get a nicer plot}
\PY{n}{plt}\PY{o}{.}\PY{n}{plot}\PY{p}{(}\PY{n}{pillar\PYZus{}dates}\PY{p}{,} \PY{n}{discount\PYZus{}factors}\PY{p}{,} \PY{n}{marker}\PY{o}{=}\PY{l+s+s1}{\PYZsq{}}\PY{l+s+s1}{o}\PY{l+s+s1}{\PYZsq{}}\PY{p}{)}
\PY{n}{plt}\PY{o}{.}\PY{n}{gca}\PY{p}{(}\PY{p}{)}\PY{o}{.}\PY{n}{xaxis}\PY{o}{.}\PY{n}{set\PYZus{}major\PYZus{}formatter}\PY{p}{(}\PY{n}{mdates}\PY{o}{.}\PY{n}{DateFormatter}\PY{p}{(}\PY{l+s+s1}{\PYZsq{}}\PY{l+s+s1}{\PYZpc{}}\PY{l+s+s1}{m/}\PY{l+s+si}{\PYZpc{}d}\PY{l+s+s1}{/}\PY{l+s+s1}{\PYZpc{}}\PY{l+s+s1}{Y}\PY{l+s+s1}{\PYZsq{}}\PY{p}{)}\PY{p}{)}
\PY{n}{plt}\PY{o}{.}\PY{n}{gca}\PY{p}{(}\PY{p}{)}\PY{o}{.}\PY{n}{xaxis}\PY{o}{.}\PY{n}{set\PYZus{}major\PYZus{}locator}\PY{p}{(}\PY{n}{mdates}\PY{o}{.}\PY{n}{YearLocator}\PY{p}{(}\PY{p}{)}\PY{p}{)}
\PY{n}{plt}\PY{o}{.}\PY{n}{grid}\PY{p}{(}\PY{k+kc}{True}\PY{p}{)}
\PY{n}{plt}\PY{o}{.}\PY{n}{show}\PY{p}{(}\PY{p}{)}

\PY{c+c1}{\PYZsh{} define the df function}
\PY{k}{def} \PY{n+nf}{df}\PY{p}{(}\PY{n}{d}\PY{p}{)}\PY{p}{:}
    \PY{c+c1}{\PYZsh{} first thing we need to do is to apply the logarithm function }
    \PY{c+c1}{\PYZsh{} to the discount factors since we are doing log‐linear and}
    \PY{c+c1}{\PYZsh{} not just linear interpolation}
    \PY{n}{log\PYZus{}discount\PYZus{}factors} \PY{o}{=} \PY{p}{[}\PY{p}{]}
    \PY{k}{for} \PY{n}{discount\PYZus{}factor} \PY{o+ow}{in} \PY{n}{discount\PYZus{}factors}\PY{p}{:}
        \PY{n}{log\PYZus{}discount\PYZus{}factors}\PY{o}{.}\PY{n}{append}\PY{p}{(}\PY{n}{math}\PY{o}{.}\PY{n}{log}\PY{p}{(}\PY{n}{discount\PYZus{}factor}\PY{p}{)}\PY{p}{)}
    
    \PY{c+c1}{\PYZsh{} perform the linear interpolation of the log discount factors}
    \PY{n}{interpolated\PYZus{}log\PYZus{}discount\PYZus{}factor} \PY{o}{=} \PYZbs{}
        \PY{n}{numpy}\PY{o}{.}\PY{n}{interp}\PY{p}{(}\PY{n}{d}\PY{p}{,} \PY{n}{pillar\PYZus{}dates}\PY{p}{,} \PY{n}{log\PYZus{}discount\PYZus{}factors}\PY{p}{)}
    
    \PY{c+c1}{\PYZsh{} return the interpolated discount factor}
    \PY{k}{return} \PY{n}{math}\PY{o}{.}\PY{n}{exp}\PY{p}{(}\PY{n}{interpolated\PYZus{}log\PYZus{}discount\PYZus{}factor}\PY{p}{)}
\end{Verbatim}
\end{tcolorbox}

    \begin{center}
    \adjustimage{max size={0.9\linewidth}{0.9\paperheight}}{lecture_3_files/lecture_3_10_0.png}
    \end{center}
    { \hspace*{\fill} \\}
    
    This is almost OK, \textbf{but it won't work} because
\texttt{numpy.interp} only accepts numbers/lists of numbers as arguments
i.e.~it doesn't automatically convert or interpret dates as numbers in
any way, so it doesn't know how to interpolate them. So we need to do
the conversion ourselves before passing the data into the
\texttt{numpy.interp} function.

    \begin{tcolorbox}[breakable, size=fbox, boxrule=1pt, pad at break*=1mm,colback=cellbackground, colframe=cellborder]
\prompt{In}{incolor}{7}{\boxspacing}
\begin{Verbatim}[commandchars=\\\{\}]
\PY{k}{def} \PY{n+nf}{df}\PY{p}{(}\PY{n}{d}\PY{p}{)}\PY{p}{:}
    \PY{c+c1}{\PYZsh{} first thing we need to do is to apply the logarithm function}
    \PY{c+c1}{\PYZsh{} to the discount factors since we are doing log‐linear and}
    \PY{c+c1}{\PYZsh{} not just linear interpolation}
    \PY{n}{log\PYZus{}discount\PYZus{}factors} \PY{o}{=} \PY{p}{[}\PY{p}{]}
    \PY{k}{for} \PY{n}{discount\PYZus{}factor} \PY{o+ow}{in} \PY{n}{discount\PYZus{}factors}\PY{p}{:}
        \PY{n}{log\PYZus{}discount\PYZus{}factors}\PY{o}{.}\PY{n}{append}\PY{p}{(}\PY{n}{math}\PY{o}{.}\PY{n}{log}\PY{p}{(}\PY{n}{discount\PYZus{}factor}\PY{p}{)}\PY{p}{)}
    
    \PY{c+c1}{\PYZsh{} convert the pillar dates to pillar \PYZsq{}days\PYZsq{}}
    \PY{c+c1}{\PYZsh{} i.e. number of days from today}
    \PY{c+c1}{\PYZsh{} to write shorter code we can use this NEW notation}
    \PY{c+c1}{\PYZsh{} which condenses for and list creation in one line}
    \PY{n}{pillar\PYZus{}days} \PY{o}{=} \PYZbs{}
        \PY{p}{[}\PY{p}{(}\PY{n}{pillar\PYZus{}date} \PY{o}{\PYZhy{}} \PY{n}{today\PYZus{}date}\PY{p}{)}\PY{o}{.}\PY{n}{days} \PY{k}{for} \PY{n}{pillar\PYZus{}date} \PY{o+ow}{in} \PY{n}{pillar\PYZus{}dates}\PY{p}{]}
    
    \PY{c+c1}{\PYZsh{} obviously we need to do the same to the value date}
    \PY{c+c1}{\PYZsh{} argument of the df function}
    \PY{n}{d\PYZus{}days} \PY{o}{=} \PY{p}{(}\PY{n}{d} \PY{o}{\PYZhy{}} \PY{n}{today\PYZus{}date}\PY{p}{)}\PY{o}{.}\PY{n}{days}
    
    \PY{c+c1}{\PYZsh{} perform the linear interpolation of the log discount factors}
    \PY{n}{interpolated\PYZus{}log\PYZus{}discount\PYZus{}factor} \PY{o}{=} \PYZbs{}
        \PY{n}{numpy}\PY{o}{.}\PY{n}{interp}\PY{p}{(}\PY{n}{d\PYZus{}days}\PY{p}{,} \PY{n}{pillar\PYZus{}days}\PY{p}{,} \PY{n}{log\PYZus{}discount\PYZus{}factors}\PY{p}{)}
    
    \PY{c+c1}{\PYZsh{} return the interpolated discount factor}
    \PY{k}{return} \PY{n}{math}\PY{o}{.}\PY{n}{exp}\PY{p}{(}\PY{n}{interpolated\PYZus{}log\PYZus{}discount\PYZus{}factor}\PY{p}{)}
\end{Verbatim}
\end{tcolorbox}

    \begin{tcolorbox}[breakable, size=fbox, boxrule=1pt, pad at break*=1mm,colback=cellbackground, colframe=cellborder]
\prompt{In}{incolor}{8}{\boxspacing}
\begin{Verbatim}[commandchars=\\\{\}]
\PY{c+c1}{\PYZsh{} now we can use the df function to get discount factors}
\PY{c+c1}{\PYZsh{} on value dates between the given pillar dates}
\PY{n}{d0} \PY{o}{=} \PY{n}{date}\PY{p}{(}\PY{l+m+mi}{2020}\PY{p}{,} \PY{l+m+mi}{1}\PY{p}{,} \PY{l+m+mi}{1}\PY{p}{)}
\PY{n}{df0} \PY{o}{=} \PY{n}{df}\PY{p}{(}\PY{n}{d0}\PY{p}{)}
\PY{n+nb}{print} \PY{p}{(}\PY{n}{df0}\PY{p}{)}
\end{Verbatim}
\end{tcolorbox}

    \begin{Verbatim}[commandchars=\\\{\}]
0.9923728228571693
    \end{Verbatim}

    \begin{tcolorbox}[breakable, size=fbox, boxrule=1pt, pad at break*=1mm,colback=cellbackground, colframe=cellborder]
\prompt{In}{incolor}{9}{\boxspacing}
\begin{Verbatim}[commandchars=\\\{\}]
\PY{n}{d1} \PY{o}{=} \PY{n}{date}\PY{p}{(}\PY{l+m+mi}{2021}\PY{p}{,} \PY{l+m+mi}{1}\PY{p}{,} \PY{l+m+mi}{1}\PY{p}{)}
\PY{n}{df1} \PY{o}{=} \PY{n}{df}\PY{p}{(}\PY{n}{d1}\PY{p}{)}
\PY{n+nb}{print} \PY{p}{(}\PY{n}{df1}\PY{p}{)}
\end{Verbatim}
\end{tcolorbox}

    \begin{Verbatim}[commandchars=\\\{\}]
0.8997999273630835
    \end{Verbatim}

    \begin{tcolorbox}[breakable, size=fbox, boxrule=1pt, pad at break*=1mm,colback=cellbackground, colframe=cellborder]
\prompt{In}{incolor}{10}{\boxspacing}
\begin{Verbatim}[commandchars=\\\{\}]
\PY{c+c1}{\PYZsh{} let\PYZsq{}s see what these look like when plotted on a semi‐log graph}

\PY{k+kn}{from} \PY{n+nn}{matplotlib} \PY{k}{import} \PY{n}{pyplot} \PY{k}{as} \PY{n}{plt}
\PY{k+kn}{import} \PY{n+nn}{matplotlib}\PY{n+nn}{.}\PY{n+nn}{dates} \PY{k}{as} \PY{n+nn}{mdates}

\PY{n}{plt}\PY{o}{.}\PY{n}{semilogy}\PY{p}{(}\PY{n}{pillar\PYZus{}dates}\PY{p}{,} \PY{n}{discount\PYZus{}factors}\PY{p}{,} \PY{n}{marker}\PY{o}{=}\PY{l+s+s1}{\PYZsq{}}\PY{l+s+s1}{o}\PY{l+s+s1}{\PYZsq{}}\PY{p}{)}
\PY{n}{plt}\PY{o}{.}\PY{n}{semilogy}\PY{p}{(}\PY{n}{d0}\PY{p}{,}\PY{n}{df0} \PY{p}{,} \PY{n}{marker}\PY{o}{=}\PY{l+s+s1}{\PYZsq{}}\PY{l+s+s1}{X}\PY{l+s+s1}{\PYZsq{}}\PY{p}{)}
\PY{n}{plt}\PY{o}{.}\PY{n}{semilogy}\PY{p}{(}\PY{n}{d1}\PY{p}{,}\PY{n}{df1} \PY{p}{,} \PY{n}{marker}\PY{o}{=}\PY{l+s+s1}{\PYZsq{}}\PY{l+s+s1}{X}\PY{l+s+s1}{\PYZsq{}}\PY{p}{)}
\PY{n}{plt}\PY{o}{.}\PY{n}{gca}\PY{p}{(}\PY{p}{)}\PY{o}{.}\PY{n}{xaxis}\PY{o}{.}\PY{n}{set\PYZus{}major\PYZus{}formatter}\PY{p}{(}\PY{n}{mdates}\PY{o}{.}\PY{n}{DateFormatter}\PY{p}{(}\PY{l+s+s1}{\PYZsq{}}\PY{l+s+s1}{\PYZpc{}}\PY{l+s+s1}{m/}\PY{l+s+si}{\PYZpc{}d}\PY{l+s+s1}{/}\PY{l+s+s1}{\PYZpc{}}\PY{l+s+s1}{Y}\PY{l+s+s1}{\PYZsq{}}\PY{p}{)}\PY{p}{)}
\PY{n}{plt}\PY{o}{.}\PY{n}{gca}\PY{p}{(}\PY{p}{)}\PY{o}{.}\PY{n}{xaxis}\PY{o}{.}\PY{n}{set\PYZus{}major\PYZus{}locator}\PY{p}{(}\PY{n}{mdates}\PY{o}{.}\PY{n}{YearLocator}\PY{p}{(}\PY{p}{)}\PY{p}{)}
\PY{n}{plt}\PY{o}{.}\PY{n}{grid}\PY{p}{(}\PY{k+kc}{True}\PY{p}{)}
\PY{n}{plt}\PY{o}{.}\PY{n}{show}\PY{p}{(}\PY{p}{)}
\end{Verbatim}
\end{tcolorbox}

    \begin{center}
    \adjustimage{max size={0.9\linewidth}{0.9\paperheight}}{lecture_3_files/lecture_3_15_0.png}
    \end{center}
    { \hspace*{\fill} \\}
    
    \begin{tcolorbox}[breakable, size=fbox, boxrule=1pt, pad at break*=1mm,colback=cellbackground, colframe=cellborder]
\prompt{In}{incolor}{11}{\boxspacing}
\begin{Verbatim}[commandchars=\\\{\}]
\PY{c+c1}{\PYZsh{} let\PYZsq{}s see what these look like when plotted on a linear graph}

\PY{k+kn}{from} \PY{n+nn}{matplotlib} \PY{k}{import} \PY{n}{pyplot} \PY{k}{as} \PY{n}{plt}
\PY{k+kn}{import} \PY{n+nn}{matplotlib}\PY{n+nn}{.}\PY{n+nn}{dates} \PY{k}{as} \PY{n+nn}{mdates}
\PY{n}{plt}\PY{o}{.}\PY{n}{plot}\PY{p}{(}\PY{n}{pillar\PYZus{}dates}\PY{p}{,} \PY{n}{discount\PYZus{}factors}\PY{p}{,} \PY{n}{marker}\PY{o}{=}\PY{l+s+s1}{\PYZsq{}}\PY{l+s+s1}{o}\PY{l+s+s1}{\PYZsq{}}\PY{p}{)}
\PY{n}{plt}\PY{o}{.}\PY{n}{plot}\PY{p}{(}\PY{n}{d0}\PY{p}{,}\PY{n}{df0} \PY{p}{,} \PY{n}{marker}\PY{o}{=}\PY{l+s+s1}{\PYZsq{}}\PY{l+s+s1}{X}\PY{l+s+s1}{\PYZsq{}}\PY{p}{)}
\PY{n}{plt}\PY{o}{.}\PY{n}{plot}\PY{p}{(}\PY{n}{d1}\PY{p}{,}\PY{n}{df1} \PY{p}{,} \PY{n}{marker}\PY{o}{=}\PY{l+s+s1}{\PYZsq{}}\PY{l+s+s1}{X}\PY{l+s+s1}{\PYZsq{}}\PY{p}{)}
\PY{n}{plt}\PY{o}{.}\PY{n}{gca}\PY{p}{(}\PY{p}{)}\PY{o}{.}\PY{n}{xaxis}\PY{o}{.}\PY{n}{set\PYZus{}major\PYZus{}formatter}\PY{p}{(}\PY{n}{mdates}\PY{o}{.}\PY{n}{DateFormatter}\PY{p}{(}\PY{l+s+s1}{\PYZsq{}}\PY{l+s+s1}{\PYZpc{}}\PY{l+s+s1}{m/}\PY{l+s+si}{\PYZpc{}d}\PY{l+s+s1}{/}\PY{l+s+s1}{\PYZpc{}}\PY{l+s+s1}{Y}\PY{l+s+s1}{\PYZsq{}}\PY{p}{)}\PY{p}{)}
\PY{n}{plt}\PY{o}{.}\PY{n}{gca}\PY{p}{(}\PY{p}{)}\PY{o}{.}\PY{n}{xaxis}\PY{o}{.}\PY{n}{set\PYZus{}major\PYZus{}locator}\PY{p}{(}\PY{n}{mdates}\PY{o}{.}\PY{n}{YearLocator}\PY{p}{(}\PY{p}{)}\PY{p}{)}
\PY{n}{plt}\PY{o}{.}\PY{n}{grid}\PY{p}{(}\PY{k+kc}{True}\PY{p}{)}
\PY{n}{plt}\PY{o}{.}\PY{n}{show}\PY{p}{(}\PY{p}{)}
\end{Verbatim}
\end{tcolorbox}

    \begin{center}
    \adjustimage{max size={0.9\linewidth}{0.9\paperheight}}{lecture_3_files/lecture_3_16_0.png}
    \end{center}
    { \hspace*{\fill} \\}
    
    \hypertarget{exercises}{%
\subsection{Exercises}\label{exercises}}

\hypertarget{exercise-3.1}{%
\subsubsection{Exercise 3.1}\label{exercise-3.1}}

Take the code for the Black-Scholes formula from Exercise 2.3 and wrap
it in a function. Then, use this function to calculate the prices of
calls with various strikes, using the following data.

\begin{Shaded}
\begin{Highlighting}[]
\NormalTok{S_t }\OperatorTok{=} \DecValTok{800}
\CommentTok{# strikes expressed as % of spot price}
\NormalTok{moneyness }\OperatorTok{=}\NormalTok{ [ }\FloatTok{0.5}\NormalTok{, }\FloatTok{0.75}\NormalTok{, }\FloatTok{0.825}\NormalTok{, }\FloatTok{1.0}\NormalTok{, }\FloatTok{1.125}\NormalTok{, }\FloatTok{1.25}\NormalTok{, }\FloatTok{1.5}\NormalTok{ ]   }
\NormalTok{vol }\OperatorTok{=} \FloatTok{0.3}
\NormalTok{ttm }\OperatorTok{=} \FloatTok{0.75}
\NormalTok{r }\OperatorTok{=} \FloatTok{0.005}
\end{Highlighting}
\end{Shaded}

The output should be a dictionary mapping strikes to call prices.

\hypertarget{exercise-3.2}{%
\subsubsection{Exercise 3.2}\label{exercise-3.2}}

Python has a useful command called \texttt{assert} which can be used for
checking that a given condition is satisfied, and raising an error if
the condition is not satisfied.

The following line does not cause an error, in fact it does nothing

\begin{Shaded}
\begin{Highlighting}[]
\ControlFlowTok{assert} \DecValTok{1} \OperatorTok{<} \DecValTok{2}
\end{Highlighting}
\end{Shaded}

This causes an error

\begin{Shaded}
\begin{Highlighting}[]
\ControlFlowTok{assert} \DecValTok{1} \OperatorTok{>} \DecValTok{2}
\end{Highlighting}
\end{Shaded}

\texttt{assert} can take a second argument with a message to display in
case of failure.

\begin{Shaded}
\begin{Highlighting}[]
\ControlFlowTok{assert} \DecValTok{1} \OperatorTok{>} \DecValTok{2}\NormalTok{, }\StringTok{"Two is bigger than one"}
\end{Highlighting}
\end{Shaded}

Take the \texttt{df} function from this lesson and modify it by adding
some assertions to check that:

\begin{itemize}
\tightlist
\item
  the pillar date list contains at least 2 elements;
\item
  the pillar date list is the same length as the discount factor list;
\item
  the first pillar date is equal to the today date;
\item
  the value date argument `d' is greater or equal to the first pillar
  date and also less than or equal to the last pillar date.
\end{itemize}

Then try using the function with some invalid data to make sure that
your assertions are correctly checking the desired conditions.

\hypertarget{exercise-3.3}{%
\subsubsection{Exercise 3.3}\label{exercise-3.3}}

Python has a module called \texttt{matplotlib} which can be used for
plotting graphs and charts. In particular, we can use a sub-module
called \texttt{pyplot} which provides slightly easier-to-use interface
for plotting interactively.

\begin{Shaded}
\begin{Highlighting}[]
\ImportTok{from}\NormalTok{ matplotlib }\ImportTok{import}\NormalTok{ pyplot}

\CommentTok{# plot some data}
\NormalTok{pyplot.plot(}
\NormalTok{    [}\DecValTok{1}\NormalTok{, }\DecValTok{2}\NormalTok{, }\DecValTok{3}\NormalTok{],   }\CommentTok{# x-axis coordinates}
\NormalTok{    [}\DecValTok{5}\NormalTok{, }\DecValTok{3}\NormalTok{, }\DecValTok{10}\NormalTok{],  }\CommentTok{# y-axis coordinates}
\NormalTok{    marker}\OperatorTok{=}\StringTok{'o'}   \CommentTok{# we want the points to be marked with circles}
\NormalTok{)}
\end{Highlighting}
\end{Shaded}

Use this function to plot the call prices from exercise 3.1. Remember to
use \texttt{help} and \texttt{dir} to have some help (or to look in
Google ;-)).

    \hypertarget{advanced-hint}{%
\subsection{Advanced hint}\label{advanced-hint}}

Interpolation using \texttt{scipy.interpolate}:
https://docs.scipy.org/doc/scipy-0.15.1/reference/interpolate.html\#module-scipy.interpolate

    \begin{tcolorbox}[breakable, size=fbox, boxrule=1pt, pad at break*=1mm,colback=cellbackground, colframe=cellborder]
\prompt{In}{incolor}{ }{\boxspacing}
\begin{Verbatim}[commandchars=\\\{\}]

\end{Verbatim}
\end{tcolorbox}

    \begin{tcolorbox}[breakable, size=fbox, boxrule=1pt, pad at break*=1mm,colback=cellbackground, colframe=cellborder]
\prompt{In}{incolor}{ }{\boxspacing}
\begin{Verbatim}[commandchars=\\\{\}]

\end{Verbatim}
\end{tcolorbox}


    % Add a bibliography block to the postdoc
    
    
    
\end{document}

\documentclass[11pt]{article}

    \usepackage[breakable]{tcolorbox}
    \usepackage{parskip} % Stop auto-indenting (to mimic markdown behaviour)
    
    \usepackage{iftex}
    \ifPDFTeX
    	\usepackage[T1]{fontenc}
    	\usepackage{mathpazo}
    \else
    	\usepackage{fontspec}
    \fi

    % Basic figure setup, for now with no caption control since it's done
    % automatically by Pandoc (which extracts ![](path) syntax from Markdown).
    \usepackage{graphicx}
    % Maintain compatibility with old templates. Remove in nbconvert 6.0
    \let\Oldincludegraphics\includegraphics
    % Ensure that by default, figures have no caption (until we provide a
    % proper Figure object with a Caption API and a way to capture that
    % in the conversion process - todo).
    \usepackage{caption}
    \DeclareCaptionFormat{nocaption}{}
    \captionsetup{format=nocaption,aboveskip=0pt,belowskip=0pt}

    \usepackage[Export]{adjustbox} % Used to constrain images to a maximum size
    \adjustboxset{max size={0.9\linewidth}{0.9\paperheight}}
    \usepackage{float}
    \floatplacement{figure}{H} % forces figures to be placed at the correct location
    \usepackage{xcolor} % Allow colors to be defined
    \usepackage{enumerate} % Needed for markdown enumerations to work
    \usepackage{geometry} % Used to adjust the document margins
    \usepackage{amsmath} % Equations
    \usepackage{amssymb} % Equations
    \usepackage{textcomp} % defines textquotesingle
    % Hack from http://tex.stackexchange.com/a/47451/13684:
    \AtBeginDocument{%
        \def\PYZsq{\textquotesingle}% Upright quotes in Pygmentized code
    }
    \usepackage{upquote} % Upright quotes for verbatim code
    \usepackage{eurosym} % defines \euro
    \usepackage[mathletters]{ucs} % Extended unicode (utf-8) support
    \usepackage{fancyvrb} % verbatim replacement that allows latex
    \usepackage{grffile} % extends the file name processing of package graphics 
                         % to support a larger range
    \makeatletter % fix for grffile with XeLaTeX
    \def\Gread@@xetex#1{%
      \IfFileExists{"\Gin@base".bb}%
      {\Gread@eps{\Gin@base.bb}}%
      {\Gread@@xetex@aux#1}%
    }
    \makeatother

    % The hyperref package gives us a pdf with properly built
    % internal navigation ('pdf bookmarks' for the table of contents,
    % internal cross-reference links, web links for URLs, etc.)
    \usepackage{hyperref}
    % The default LaTeX title has an obnoxious amount of whitespace. By default,
    % titling removes some of it. It also provides customization options.
    \usepackage{titling}
    \usepackage{longtable} % longtable support required by pandoc >1.10
    \usepackage{booktabs}  % table support for pandoc > 1.12.2
    \usepackage[inline]{enumitem} % IRkernel/repr support (it uses the enumerate* environment)
    \usepackage[normalem]{ulem} % ulem is needed to support strikethroughs (\sout)
                                % normalem makes italics be italics, not underlines
    \usepackage{mathrsfs}
    

    
    % Colors for the hyperref package
    \definecolor{urlcolor}{rgb}{0,.145,.698}
    \definecolor{linkcolor}{rgb}{.71,0.21,0.01}
    \definecolor{citecolor}{rgb}{.12,.54,.11}

    % ANSI colors
    \definecolor{ansi-black}{HTML}{3E424D}
    \definecolor{ansi-black-intense}{HTML}{282C36}
    \definecolor{ansi-red}{HTML}{E75C58}
    \definecolor{ansi-red-intense}{HTML}{B22B31}
    \definecolor{ansi-green}{HTML}{00A250}
    \definecolor{ansi-green-intense}{HTML}{007427}
    \definecolor{ansi-yellow}{HTML}{DDB62B}
    \definecolor{ansi-yellow-intense}{HTML}{B27D12}
    \definecolor{ansi-blue}{HTML}{208FFB}
    \definecolor{ansi-blue-intense}{HTML}{0065CA}
    \definecolor{ansi-magenta}{HTML}{D160C4}
    \definecolor{ansi-magenta-intense}{HTML}{A03196}
    \definecolor{ansi-cyan}{HTML}{60C6C8}
    \definecolor{ansi-cyan-intense}{HTML}{258F8F}
    \definecolor{ansi-white}{HTML}{C5C1B4}
    \definecolor{ansi-white-intense}{HTML}{A1A6B2}
    \definecolor{ansi-default-inverse-fg}{HTML}{FFFFFF}
    \definecolor{ansi-default-inverse-bg}{HTML}{000000}

    % commands and environments needed by pandoc snippets
    % extracted from the output of `pandoc -s`
    \providecommand{\tightlist}{%
      \setlength{\itemsep}{0pt}\setlength{\parskip}{0pt}}
    \DefineVerbatimEnvironment{Highlighting}{Verbatim}{commandchars=\\\{\}}
    % Add ',fontsize=\small' for more characters per line
    \newenvironment{Shaded}{}{}
    \newcommand{\KeywordTok}[1]{\textcolor[rgb]{0.00,0.44,0.13}{\textbf{{#1}}}}
    \newcommand{\DataTypeTok}[1]{\textcolor[rgb]{0.56,0.13,0.00}{{#1}}}
    \newcommand{\DecValTok}[1]{\textcolor[rgb]{0.25,0.63,0.44}{{#1}}}
    \newcommand{\BaseNTok}[1]{\textcolor[rgb]{0.25,0.63,0.44}{{#1}}}
    \newcommand{\FloatTok}[1]{\textcolor[rgb]{0.25,0.63,0.44}{{#1}}}
    \newcommand{\CharTok}[1]{\textcolor[rgb]{0.25,0.44,0.63}{{#1}}}
    \newcommand{\StringTok}[1]{\textcolor[rgb]{0.25,0.44,0.63}{{#1}}}
    \newcommand{\CommentTok}[1]{\textcolor[rgb]{0.38,0.63,0.69}{\textit{{#1}}}}
    \newcommand{\OtherTok}[1]{\textcolor[rgb]{0.00,0.44,0.13}{{#1}}}
    \newcommand{\AlertTok}[1]{\textcolor[rgb]{1.00,0.00,0.00}{\textbf{{#1}}}}
    \newcommand{\FunctionTok}[1]{\textcolor[rgb]{0.02,0.16,0.49}{{#1}}}
    \newcommand{\RegionMarkerTok}[1]{{#1}}
    \newcommand{\ErrorTok}[1]{\textcolor[rgb]{1.00,0.00,0.00}{\textbf{{#1}}}}
    \newcommand{\NormalTok}[1]{{#1}}
    
    % Additional commands for more recent versions of Pandoc
    \newcommand{\ConstantTok}[1]{\textcolor[rgb]{0.53,0.00,0.00}{{#1}}}
    \newcommand{\SpecialCharTok}[1]{\textcolor[rgb]{0.25,0.44,0.63}{{#1}}}
    \newcommand{\VerbatimStringTok}[1]{\textcolor[rgb]{0.25,0.44,0.63}{{#1}}}
    \newcommand{\SpecialStringTok}[1]{\textcolor[rgb]{0.73,0.40,0.53}{{#1}}}
    \newcommand{\ImportTok}[1]{{#1}}
    \newcommand{\DocumentationTok}[1]{\textcolor[rgb]{0.73,0.13,0.13}{\textit{{#1}}}}
    \newcommand{\AnnotationTok}[1]{\textcolor[rgb]{0.38,0.63,0.69}{\textbf{\textit{{#1}}}}}
    \newcommand{\CommentVarTok}[1]{\textcolor[rgb]{0.38,0.63,0.69}{\textbf{\textit{{#1}}}}}
    \newcommand{\VariableTok}[1]{\textcolor[rgb]{0.10,0.09,0.49}{{#1}}}
    \newcommand{\ControlFlowTok}[1]{\textcolor[rgb]{0.00,0.44,0.13}{\textbf{{#1}}}}
    \newcommand{\OperatorTok}[1]{\textcolor[rgb]{0.40,0.40,0.40}{{#1}}}
    \newcommand{\BuiltInTok}[1]{{#1}}
    \newcommand{\ExtensionTok}[1]{{#1}}
    \newcommand{\PreprocessorTok}[1]{\textcolor[rgb]{0.74,0.48,0.00}{{#1}}}
    \newcommand{\AttributeTok}[1]{\textcolor[rgb]{0.49,0.56,0.16}{{#1}}}
    \newcommand{\InformationTok}[1]{\textcolor[rgb]{0.38,0.63,0.69}{\textbf{\textit{{#1}}}}}
    \newcommand{\WarningTok}[1]{\textcolor[rgb]{0.38,0.63,0.69}{\textbf{\textit{{#1}}}}}
    
    
    % Define a nice break command that doesn't care if a line doesn't already
    % exist.
    \def\br{\hspace*{\fill} \\* }
    % Math Jax compatibility definitions
    \def\gt{>}
    \def\lt{<}
    \let\Oldtex\TeX
    \let\Oldlatex\LaTeX
    \renewcommand{\TeX}{\textrm{\Oldtex}}
    \renewcommand{\LaTeX}{\textrm{\Oldlatex}}
    % Document parameters
    % Document title
    \title{lesson3}
    
    
    
    
    
% Pygments definitions
\makeatletter
\def\PY@reset{\let\PY@it=\relax \let\PY@bf=\relax%
    \let\PY@ul=\relax \let\PY@tc=\relax%
    \let\PY@bc=\relax \let\PY@ff=\relax}
\def\PY@tok#1{\csname PY@tok@#1\endcsname}
\def\PY@toks#1+{\ifx\relax#1\empty\else%
    \PY@tok{#1}\expandafter\PY@toks\fi}
\def\PY@do#1{\PY@bc{\PY@tc{\PY@ul{%
    \PY@it{\PY@bf{\PY@ff{#1}}}}}}}
\def\PY#1#2{\PY@reset\PY@toks#1+\relax+\PY@do{#2}}

\expandafter\def\csname PY@tok@w\endcsname{\def\PY@tc##1{\textcolor[rgb]{0.73,0.73,0.73}{##1}}}
\expandafter\def\csname PY@tok@c\endcsname{\let\PY@it=\textit\def\PY@tc##1{\textcolor[rgb]{0.25,0.50,0.50}{##1}}}
\expandafter\def\csname PY@tok@cp\endcsname{\def\PY@tc##1{\textcolor[rgb]{0.74,0.48,0.00}{##1}}}
\expandafter\def\csname PY@tok@k\endcsname{\let\PY@bf=\textbf\def\PY@tc##1{\textcolor[rgb]{0.00,0.50,0.00}{##1}}}
\expandafter\def\csname PY@tok@kp\endcsname{\def\PY@tc##1{\textcolor[rgb]{0.00,0.50,0.00}{##1}}}
\expandafter\def\csname PY@tok@kt\endcsname{\def\PY@tc##1{\textcolor[rgb]{0.69,0.00,0.25}{##1}}}
\expandafter\def\csname PY@tok@o\endcsname{\def\PY@tc##1{\textcolor[rgb]{0.40,0.40,0.40}{##1}}}
\expandafter\def\csname PY@tok@ow\endcsname{\let\PY@bf=\textbf\def\PY@tc##1{\textcolor[rgb]{0.67,0.13,1.00}{##1}}}
\expandafter\def\csname PY@tok@nb\endcsname{\def\PY@tc##1{\textcolor[rgb]{0.00,0.50,0.00}{##1}}}
\expandafter\def\csname PY@tok@nf\endcsname{\def\PY@tc##1{\textcolor[rgb]{0.00,0.00,1.00}{##1}}}
\expandafter\def\csname PY@tok@nc\endcsname{\let\PY@bf=\textbf\def\PY@tc##1{\textcolor[rgb]{0.00,0.00,1.00}{##1}}}
\expandafter\def\csname PY@tok@nn\endcsname{\let\PY@bf=\textbf\def\PY@tc##1{\textcolor[rgb]{0.00,0.00,1.00}{##1}}}
\expandafter\def\csname PY@tok@ne\endcsname{\let\PY@bf=\textbf\def\PY@tc##1{\textcolor[rgb]{0.82,0.25,0.23}{##1}}}
\expandafter\def\csname PY@tok@nv\endcsname{\def\PY@tc##1{\textcolor[rgb]{0.10,0.09,0.49}{##1}}}
\expandafter\def\csname PY@tok@no\endcsname{\def\PY@tc##1{\textcolor[rgb]{0.53,0.00,0.00}{##1}}}
\expandafter\def\csname PY@tok@nl\endcsname{\def\PY@tc##1{\textcolor[rgb]{0.63,0.63,0.00}{##1}}}
\expandafter\def\csname PY@tok@ni\endcsname{\let\PY@bf=\textbf\def\PY@tc##1{\textcolor[rgb]{0.60,0.60,0.60}{##1}}}
\expandafter\def\csname PY@tok@na\endcsname{\def\PY@tc##1{\textcolor[rgb]{0.49,0.56,0.16}{##1}}}
\expandafter\def\csname PY@tok@nt\endcsname{\let\PY@bf=\textbf\def\PY@tc##1{\textcolor[rgb]{0.00,0.50,0.00}{##1}}}
\expandafter\def\csname PY@tok@nd\endcsname{\def\PY@tc##1{\textcolor[rgb]{0.67,0.13,1.00}{##1}}}
\expandafter\def\csname PY@tok@s\endcsname{\def\PY@tc##1{\textcolor[rgb]{0.73,0.13,0.13}{##1}}}
\expandafter\def\csname PY@tok@sd\endcsname{\let\PY@it=\textit\def\PY@tc##1{\textcolor[rgb]{0.73,0.13,0.13}{##1}}}
\expandafter\def\csname PY@tok@si\endcsname{\let\PY@bf=\textbf\def\PY@tc##1{\textcolor[rgb]{0.73,0.40,0.53}{##1}}}
\expandafter\def\csname PY@tok@se\endcsname{\let\PY@bf=\textbf\def\PY@tc##1{\textcolor[rgb]{0.73,0.40,0.13}{##1}}}
\expandafter\def\csname PY@tok@sr\endcsname{\def\PY@tc##1{\textcolor[rgb]{0.73,0.40,0.53}{##1}}}
\expandafter\def\csname PY@tok@ss\endcsname{\def\PY@tc##1{\textcolor[rgb]{0.10,0.09,0.49}{##1}}}
\expandafter\def\csname PY@tok@sx\endcsname{\def\PY@tc##1{\textcolor[rgb]{0.00,0.50,0.00}{##1}}}
\expandafter\def\csname PY@tok@m\endcsname{\def\PY@tc##1{\textcolor[rgb]{0.40,0.40,0.40}{##1}}}
\expandafter\def\csname PY@tok@gh\endcsname{\let\PY@bf=\textbf\def\PY@tc##1{\textcolor[rgb]{0.00,0.00,0.50}{##1}}}
\expandafter\def\csname PY@tok@gu\endcsname{\let\PY@bf=\textbf\def\PY@tc##1{\textcolor[rgb]{0.50,0.00,0.50}{##1}}}
\expandafter\def\csname PY@tok@gd\endcsname{\def\PY@tc##1{\textcolor[rgb]{0.63,0.00,0.00}{##1}}}
\expandafter\def\csname PY@tok@gi\endcsname{\def\PY@tc##1{\textcolor[rgb]{0.00,0.63,0.00}{##1}}}
\expandafter\def\csname PY@tok@gr\endcsname{\def\PY@tc##1{\textcolor[rgb]{1.00,0.00,0.00}{##1}}}
\expandafter\def\csname PY@tok@ge\endcsname{\let\PY@it=\textit}
\expandafter\def\csname PY@tok@gs\endcsname{\let\PY@bf=\textbf}
\expandafter\def\csname PY@tok@gp\endcsname{\let\PY@bf=\textbf\def\PY@tc##1{\textcolor[rgb]{0.00,0.00,0.50}{##1}}}
\expandafter\def\csname PY@tok@go\endcsname{\def\PY@tc##1{\textcolor[rgb]{0.53,0.53,0.53}{##1}}}
\expandafter\def\csname PY@tok@gt\endcsname{\def\PY@tc##1{\textcolor[rgb]{0.00,0.27,0.87}{##1}}}
\expandafter\def\csname PY@tok@err\endcsname{\def\PY@bc##1{\setlength{\fboxsep}{0pt}\fcolorbox[rgb]{1.00,0.00,0.00}{1,1,1}{\strut ##1}}}
\expandafter\def\csname PY@tok@kc\endcsname{\let\PY@bf=\textbf\def\PY@tc##1{\textcolor[rgb]{0.00,0.50,0.00}{##1}}}
\expandafter\def\csname PY@tok@kd\endcsname{\let\PY@bf=\textbf\def\PY@tc##1{\textcolor[rgb]{0.00,0.50,0.00}{##1}}}
\expandafter\def\csname PY@tok@kn\endcsname{\let\PY@bf=\textbf\def\PY@tc##1{\textcolor[rgb]{0.00,0.50,0.00}{##1}}}
\expandafter\def\csname PY@tok@kr\endcsname{\let\PY@bf=\textbf\def\PY@tc##1{\textcolor[rgb]{0.00,0.50,0.00}{##1}}}
\expandafter\def\csname PY@tok@bp\endcsname{\def\PY@tc##1{\textcolor[rgb]{0.00,0.50,0.00}{##1}}}
\expandafter\def\csname PY@tok@fm\endcsname{\def\PY@tc##1{\textcolor[rgb]{0.00,0.00,1.00}{##1}}}
\expandafter\def\csname PY@tok@vc\endcsname{\def\PY@tc##1{\textcolor[rgb]{0.10,0.09,0.49}{##1}}}
\expandafter\def\csname PY@tok@vg\endcsname{\def\PY@tc##1{\textcolor[rgb]{0.10,0.09,0.49}{##1}}}
\expandafter\def\csname PY@tok@vi\endcsname{\def\PY@tc##1{\textcolor[rgb]{0.10,0.09,0.49}{##1}}}
\expandafter\def\csname PY@tok@vm\endcsname{\def\PY@tc##1{\textcolor[rgb]{0.10,0.09,0.49}{##1}}}
\expandafter\def\csname PY@tok@sa\endcsname{\def\PY@tc##1{\textcolor[rgb]{0.73,0.13,0.13}{##1}}}
\expandafter\def\csname PY@tok@sb\endcsname{\def\PY@tc##1{\textcolor[rgb]{0.73,0.13,0.13}{##1}}}
\expandafter\def\csname PY@tok@sc\endcsname{\def\PY@tc##1{\textcolor[rgb]{0.73,0.13,0.13}{##1}}}
\expandafter\def\csname PY@tok@dl\endcsname{\def\PY@tc##1{\textcolor[rgb]{0.73,0.13,0.13}{##1}}}
\expandafter\def\csname PY@tok@s2\endcsname{\def\PY@tc##1{\textcolor[rgb]{0.73,0.13,0.13}{##1}}}
\expandafter\def\csname PY@tok@sh\endcsname{\def\PY@tc##1{\textcolor[rgb]{0.73,0.13,0.13}{##1}}}
\expandafter\def\csname PY@tok@s1\endcsname{\def\PY@tc##1{\textcolor[rgb]{0.73,0.13,0.13}{##1}}}
\expandafter\def\csname PY@tok@mb\endcsname{\def\PY@tc##1{\textcolor[rgb]{0.40,0.40,0.40}{##1}}}
\expandafter\def\csname PY@tok@mf\endcsname{\def\PY@tc##1{\textcolor[rgb]{0.40,0.40,0.40}{##1}}}
\expandafter\def\csname PY@tok@mh\endcsname{\def\PY@tc##1{\textcolor[rgb]{0.40,0.40,0.40}{##1}}}
\expandafter\def\csname PY@tok@mi\endcsname{\def\PY@tc##1{\textcolor[rgb]{0.40,0.40,0.40}{##1}}}
\expandafter\def\csname PY@tok@il\endcsname{\def\PY@tc##1{\textcolor[rgb]{0.40,0.40,0.40}{##1}}}
\expandafter\def\csname PY@tok@mo\endcsname{\def\PY@tc##1{\textcolor[rgb]{0.40,0.40,0.40}{##1}}}
\expandafter\def\csname PY@tok@ch\endcsname{\let\PY@it=\textit\def\PY@tc##1{\textcolor[rgb]{0.25,0.50,0.50}{##1}}}
\expandafter\def\csname PY@tok@cm\endcsname{\let\PY@it=\textit\def\PY@tc##1{\textcolor[rgb]{0.25,0.50,0.50}{##1}}}
\expandafter\def\csname PY@tok@cpf\endcsname{\let\PY@it=\textit\def\PY@tc##1{\textcolor[rgb]{0.25,0.50,0.50}{##1}}}
\expandafter\def\csname PY@tok@c1\endcsname{\let\PY@it=\textit\def\PY@tc##1{\textcolor[rgb]{0.25,0.50,0.50}{##1}}}
\expandafter\def\csname PY@tok@cs\endcsname{\let\PY@it=\textit\def\PY@tc##1{\textcolor[rgb]{0.25,0.50,0.50}{##1}}}

\def\PYZbs{\char`\\}
\def\PYZus{\char`\_}
\def\PYZob{\char`\{}
\def\PYZcb{\char`\}}
\def\PYZca{\char`\^}
\def\PYZam{\char`\&}
\def\PYZlt{\char`\<}
\def\PYZgt{\char`\>}
\def\PYZsh{\char`\#}
\def\PYZpc{\char`\%}
\def\PYZdl{\char`\$}
\def\PYZhy{\char`\-}
\def\PYZsq{\char`\'}
\def\PYZdq{\char`\"}
\def\PYZti{\char`\~}
% for compatibility with earlier versions
\def\PYZat{@}
\def\PYZlb{[}
\def\PYZrb{]}
\makeatother


    % For linebreaks inside Verbatim environment from package fancyvrb. 
    \makeatletter
        \newbox\Wrappedcontinuationbox 
        \newbox\Wrappedvisiblespacebox 
        \newcommand*\Wrappedvisiblespace {\textcolor{red}{\textvisiblespace}} 
        \newcommand*\Wrappedcontinuationsymbol {\textcolor{red}{\llap{\tiny$\m@th\hookrightarrow$}}} 
        \newcommand*\Wrappedcontinuationindent {3ex } 
        \newcommand*\Wrappedafterbreak {\kern\Wrappedcontinuationindent\copy\Wrappedcontinuationbox} 
        % Take advantage of the already applied Pygments mark-up to insert 
        % potential linebreaks for TeX processing. 
        %        {, <, #, %, $, ' and ": go to next line. 
        %        _, }, ^, &, >, - and ~: stay at end of broken line. 
        % Use of \textquotesingle for straight quote. 
        \newcommand*\Wrappedbreaksatspecials {% 
            \def\PYGZus{\discretionary{\char`\_}{\Wrappedafterbreak}{\char`\_}}% 
            \def\PYGZob{\discretionary{}{\Wrappedafterbreak\char`\{}{\char`\{}}% 
            \def\PYGZcb{\discretionary{\char`\}}{\Wrappedafterbreak}{\char`\}}}% 
            \def\PYGZca{\discretionary{\char`\^}{\Wrappedafterbreak}{\char`\^}}% 
            \def\PYGZam{\discretionary{\char`\&}{\Wrappedafterbreak}{\char`\&}}% 
            \def\PYGZlt{\discretionary{}{\Wrappedafterbreak\char`\<}{\char`\<}}% 
            \def\PYGZgt{\discretionary{\char`\>}{\Wrappedafterbreak}{\char`\>}}% 
            \def\PYGZsh{\discretionary{}{\Wrappedafterbreak\char`\#}{\char`\#}}% 
            \def\PYGZpc{\discretionary{}{\Wrappedafterbreak\char`\%}{\char`\%}}% 
            \def\PYGZdl{\discretionary{}{\Wrappedafterbreak\char`\$}{\char`\$}}% 
            \def\PYGZhy{\discretionary{\char`\-}{\Wrappedafterbreak}{\char`\-}}% 
            \def\PYGZsq{\discretionary{}{\Wrappedafterbreak\textquotesingle}{\textquotesingle}}% 
            \def\PYGZdq{\discretionary{}{\Wrappedafterbreak\char`\"}{\char`\"}}% 
            \def\PYGZti{\discretionary{\char`\~}{\Wrappedafterbreak}{\char`\~}}% 
        } 
        % Some characters . , ; ? ! / are not pygmentized. 
        % This macro makes them "active" and they will insert potential linebreaks 
        \newcommand*\Wrappedbreaksatpunct {% 
            \lccode`\~`\.\lowercase{\def~}{\discretionary{\hbox{\char`\.}}{\Wrappedafterbreak}{\hbox{\char`\.}}}% 
            \lccode`\~`\,\lowercase{\def~}{\discretionary{\hbox{\char`\,}}{\Wrappedafterbreak}{\hbox{\char`\,}}}% 
            \lccode`\~`\;\lowercase{\def~}{\discretionary{\hbox{\char`\;}}{\Wrappedafterbreak}{\hbox{\char`\;}}}% 
            \lccode`\~`\:\lowercase{\def~}{\discretionary{\hbox{\char`\:}}{\Wrappedafterbreak}{\hbox{\char`\:}}}% 
            \lccode`\~`\?\lowercase{\def~}{\discretionary{\hbox{\char`\?}}{\Wrappedafterbreak}{\hbox{\char`\?}}}% 
            \lccode`\~`\!\lowercase{\def~}{\discretionary{\hbox{\char`\!}}{\Wrappedafterbreak}{\hbox{\char`\!}}}% 
            \lccode`\~`\/\lowercase{\def~}{\discretionary{\hbox{\char`\/}}{\Wrappedafterbreak}{\hbox{\char`\/}}}% 
            \catcode`\.\active
            \catcode`\,\active 
            \catcode`\;\active
            \catcode`\:\active
            \catcode`\?\active
            \catcode`\!\active
            \catcode`\/\active 
            \lccode`\~`\~ 	
        }
    \makeatother

    \let\OriginalVerbatim=\Verbatim
    \makeatletter
    \renewcommand{\Verbatim}[1][1]{%
        %\parskip\z@skip
        \sbox\Wrappedcontinuationbox {\Wrappedcontinuationsymbol}%
        \sbox\Wrappedvisiblespacebox {\FV@SetupFont\Wrappedvisiblespace}%
        \def\FancyVerbFormatLine ##1{\hsize\linewidth
            \vtop{\raggedright\hyphenpenalty\z@\exhyphenpenalty\z@
                \doublehyphendemerits\z@\finalhyphendemerits\z@
                \strut ##1\strut}%
        }%
        % If the linebreak is at a space, the latter will be displayed as visible
        % space at end of first line, and a continuation symbol starts next line.
        % Stretch/shrink are however usually zero for typewriter font.
        \def\FV@Space {%
            \nobreak\hskip\z@ plus\fontdimen3\font minus\fontdimen4\font
            \discretionary{\copy\Wrappedvisiblespacebox}{\Wrappedafterbreak}
            {\kern\fontdimen2\font}%
        }%
        
        % Allow breaks at special characters using \PYG... macros.
        \Wrappedbreaksatspecials
        % Breaks at punctuation characters . , ; ? ! and / need catcode=\active 	
        \OriginalVerbatim[#1,codes*=\Wrappedbreaksatpunct]%
    }
    \makeatother

    % Exact colors from NB
    \definecolor{incolor}{HTML}{303F9F}
    \definecolor{outcolor}{HTML}{D84315}
    \definecolor{cellborder}{HTML}{CFCFCF}
    \definecolor{cellbackground}{HTML}{F7F7F7}
    
    % prompt
    \makeatletter
    \newcommand{\boxspacing}{\kern\kvtcb@left@rule\kern\kvtcb@boxsep}
    \makeatother
    \newcommand{\prompt}[4]{
        \ttfamily\llap{{\color{#2}[#3]:\hspace{3pt}#4}}\vspace{-\baselineskip}
    }
    

    
    % Prevent overflowing lines due to hard-to-break entities
    \sloppy 
    % Setup hyperref package
    \hypersetup{
      breaklinks=true,  % so long urls are correctly broken across lines
      colorlinks=true,
      urlcolor=urlcolor,
      linkcolor=linkcolor,
      citecolor=citecolor,
      }
    % Slightly bigger margins than the latex defaults
    
    \geometry{verbose,tmargin=1in,bmargin=1in,lmargin=1in,rmargin=1in}
    
    

\begin{document}
    
    \maketitle
    
    

    
    \hypertarget{payment-dates-generator}{%
\subsection{Payment Dates Generator}\label{payment-dates-generator}}

Before going on with the lesson let's develop a \(\tt{python}\) utility.
Since from now on we need to create many lists of dates (e.g.~payment
dates for our contracts) we will write a function that does that for us
(it is actually Ex. 3.5). This function will be then put in
\(\tt{finmarkets}\) module beside \(\tt{DiscountCurve}\) class.

The function will take as input a starting date (the first date of the
list), and a maturity (in months) which represents the length of the
list. The tenor for the moment will be 12 months by default, when needed
we will modify it.

Notice that if the maturity is not a multiple of 12 months the last
period will be truncated to the last computed date. generate\_swap\_date
come creare finmarkets file adding help to functions

    \begin{tcolorbox}[breakable, size=fbox, boxrule=1pt, pad at break*=1mm,colback=cellbackground, colframe=cellborder]
\prompt{In}{incolor}{10}{\boxspacing}
\begin{Verbatim}[commandchars=\\\{\}]
\PY{k+kn}{from} \PY{n+nn}{datetime} \PY{k}{import} \PY{n}{date}
\PY{k+kn}{from} \PY{n+nn}{dateutil}\PY{n+nn}{.}\PY{n+nn}{relativedelta} \PY{k}{import} \PY{n}{relativedelta}

\PY{k}{def} \PY{n+nf}{generate\PYZus{}swap\PYZus{}dates}\PY{p}{(}\PY{n}{starting\PYZus{}date}\PY{p}{,} \PY{n}{maturity\PYZus{}months}\PY{p}{)}\PY{p}{:}
    \PY{n}{dates} \PY{o}{=} \PY{p}{[}\PY{p}{]}
    \PY{k}{for} \PY{n}{i} \PY{o+ow}{in} \PY{n+nb}{range}\PY{p}{(}\PY{l+m+mi}{0}\PY{p}{,} \PY{n}{maturity\PYZus{}months}\PY{p}{,} \PY{l+m+mi}{12}\PY{p}{)}\PY{p}{:}
        \PY{n}{dates}\PY{o}{.}\PY{n}{append}\PY{p}{(}\PY{n}{starting\PYZus{}date} \PY{o}{+} \PY{n}{relativedelta}\PY{p}{(}\PY{n}{months}\PY{o}{=}\PY{n}{i}\PY{p}{)}\PY{p}{)}
    \PY{n}{dates}\PY{o}{.}\PY{n}{append}\PY{p}{(}\PY{n}{starting\PYZus{}date} \PY{o}{+} \PY{n}{relativedelta}\PY{p}{(}\PY{n}{months}\PY{o}{=}\PY{n}{maturity\PYZus{}months}\PY{p}{)}\PY{p}{)}
    \PY{k}{return} \PY{n}{dates}
\end{Verbatim}
\end{tcolorbox}

    \begin{tcolorbox}[breakable, size=fbox, boxrule=1pt, pad at break*=1mm,colback=cellbackground, colframe=cellborder]
\prompt{In}{incolor}{11}{\boxspacing}
\begin{Verbatim}[commandchars=\\\{\}]
\PY{n+nb}{print} \PY{p}{(}\PY{n}{generate\PYZus{}swap\PYZus{}dates}\PY{p}{(}\PY{n}{date}\PY{o}{.}\PY{n}{today}\PY{p}{(}\PY{p}{)}\PY{p}{,} \PY{l+m+mi}{25}\PY{p}{)}\PY{p}{)}
\end{Verbatim}
\end{tcolorbox}

    \begin{Verbatim}[commandchars=\\\{\}]
[datetime.date(2020, 10, 20), datetime.date(2021, 10, 20), datetime.date(2022,
10, 20), datetime.date(2022, 11, 20)]
    \end{Verbatim}

    To add the function to \(\tt{finmarkets.py}\) go to \(\tt{Jupyter}\)
main page, on top/right choose \(\tt{New/Text File}\), choose the right
name for your file and start to code.

    \hypertarget{overnight-index-swap}{%
\subsection{Overnight Index Swap}\label{overnight-index-swap}}

Interest rate swaps (IRS) are usually used to mitigate the risks of
fluctuations of varying interest rates, or to benefit from lower rates.
We will always look at these products from the point of view of the
\textbf{receiver of the floating leg}.

Overnight Index Swaps (OIS) are a particular kind of IRS which pay a
floating coupon, determined by overnight rate fixings over the reference
periods, against a fixed coupon. By definition an OIS is defined by:

\begin{itemize}
\tightlist
\item
  a notional amount \(N\);
\item
  a starting date \(d_0\);
\item
  a sequence of payment dates \(d_1,...,d_n\);
\item
  a fixed rate \(K\).
\end{itemize}

For simplicity in the following we are assuming that the fixed and
floating legs of our OIS have the same notional and payment dates,
although this is not necessarily always the case in practice.

\hypertarget{ois-valuation}{%
\subsubsection{OIS Valuation}\label{ois-valuation}}

To evaluate the net present value (NPV) of such products the cash flows
of each leg have to be calculated; today's NPV then is the sum of all
the discounted cash flows.

\hypertarget{floating-leg}{%
\paragraph{Floating leg}\label{floating-leg}}

At each payment date, the floating leg pays a cash flow determined as
follows:

\[f_{\mathrm{float},~i} = N \Bigg\{\prod_{d=d_{i-1}}^{d=d_i-1}\Big(1+r_{\mathrm{O/N}}(d)\cdot\frac{1}{360}\Big) -1 \Bigg\}\]

Strictly speaking this formula is valid for an EONIA swaps
(i.e.\textasciitilde{}for OIS swaps in EUR) other currencies might have
different conventions. The \(\frac{1}{360}\) fraction appears because
EONIA rates are quoted using the ACT/360 day-count convention. In
addition we are making the simplifying assumption of ignoring weekends
and holidays, so we assume that each overnight rate is valid for only
one day. The sum of the discounted expected values of these cash flows
is

\[\mathrm{NPV}_{\mathrm{float}} = \sum_{i=1}^{n}D(d_i)\mathbb{E}[f_{\mathrm{float},~i}]\]
where \(D(d)\) is the discount factor with expiry \(d\). On the other
hand, by definition (remember last lesson on forward rates), we also
have the following relationship

\[\mathbb{E}[f_{\mathrm{float},~i}] = N\cdot\Big(\frac{D_{\mathrm{OIS}}(d_{i-1})}{D_{\mathrm{OIS}}(d_{i})} - 1\Big)\]
hence
\[\mathrm{NPV}_{\mathrm{float}} = N\cdot \sum_{i=1}^{n}D(d_i) \Big(\frac{D_{\mathrm{OIS}}(d_{i-1})}{D_{\mathrm{OIS}}(d_{i})} - 1\Big)\]
where \(D_{\mathrm{OIS}}(d)\) is the discount factor implied by OIS
prices (we will see how to derive it).

Since the correct curve to discount OIS is the overnight index itself we
have that \(D = D_{\mathrm{OIS}}\) so the NPV simplifies to

\[\begin{equation}
  \begin{split}
    \mathrm{NPV}_{\mathrm{float}} & = N\cdot\sum_{i=1}^{n}[D(d_{i-1}) - D(d_i)] =  \\
    &= N\cdot[(D(d_{0}) - D(d_{1})) + (D(d_{1}) - D(d_{2})) + ... + (D(d_{n-1}) - D(d_{n}))]\\
    &= N \cdot [D(d_0) - D(d_n)]
  \end{split}
\end{equation}
\]

\hypertarget{fixed-leg}{%
\paragraph{Fixed leg}\label{fixed-leg}}

The calculation for the fixed leg is simpler; each cash flow is equal to

\[f_{\mathrm{fixed},~i}=N\cdot K\cdot \frac{d_i - d_{i-1}}{360}\] so the
NPV of the fixed leg is

\[\mathrm{NPV}_{\mathrm{fixed}} = N\cdot K\cdot \sum_{i=1}^{n}D_{\mathrm{OIS}}(d_{i})\frac{d_i - d_{i-1}}{360}\]

\hypertarget{discount-factor-determination-from-market-quotes}{%
\subsubsection{Discount Factor Determination from Market
Quotes}\label{discount-factor-determination-from-market-quotes}}

Our ultimate goal is to take a series of Overnight Index Swap
quotations, and determine the discount factors implied by their prices.
To do this we will build a class to represent OIS and compute its value
given particular discount curve. Then we will use this class, with a
numerical optimizer, to \emph{invert} the relation which ties the NPV to
the discount curve so that the implied discount factors can be
determined from OIS prices (market quotes).

    \begin{tcolorbox}[breakable, size=fbox, boxrule=1pt, pad at break*=1mm,colback=cellbackground, colframe=cellborder]
\prompt{In}{incolor}{23}{\boxspacing}
\begin{Verbatim}[commandchars=\\\{\}]
\PY{k}{class} \PY{n+nc}{OvernightIndexSwap}\PY{p}{:}
    \PY{k}{def} \PY{n+nf}{\PYZus{}\PYZus{}init\PYZus{}\PYZus{}}\PY{p}{(}\PY{n+nb+bp}{self}\PY{p}{,} \PY{n}{notional}\PY{p}{,} \PY{n}{payment\PYZus{}dates}\PY{p}{,} \PY{n}{fixed\PYZus{}rate}\PY{p}{)}\PY{p}{:}
        \PY{n+nb+bp}{self}\PY{o}{.}\PY{n}{notional} \PY{o}{=} \PY{n}{notional}
        \PY{n+nb+bp}{self}\PY{o}{.}\PY{n}{payment\PYZus{}dates} \PY{o}{=} \PY{n}{payment\PYZus{}dates}
        \PY{n+nb+bp}{self}\PY{o}{.}\PY{n}{fixed\PYZus{}rate} \PY{o}{=} \PY{n}{fixed\PYZus{}rate}
        
    \PY{k}{def} \PY{n+nf}{npv\PYZus{}floating\PYZus{}leg}\PY{p}{(}\PY{n+nb+bp}{self}\PY{p}{,} \PY{n}{discount\PYZus{}curve}\PY{p}{)}\PY{p}{:}
        \PY{k}{return} \PY{n+nb+bp}{self}\PY{o}{.}\PY{n}{notional} \PY{o}{*} \PY{p}{(}\PY{n}{discount\PYZus{}curve}\PY{o}{.}\PY{n}{df}\PY{p}{(}\PY{n+nb+bp}{self}\PY{o}{.}\PY{n}{payment\PYZus{}dates}\PY{p}{[}\PY{l+m+mi}{0}\PY{p}{]}\PY{p}{)} \PY{o}{\PYZhy{}} 
                                \PY{n}{discount\PYZus{}curve}\PY{o}{.}\PY{n}{df}\PY{p}{(}\PY{n+nb+bp}{self}\PY{o}{.}\PY{n}{payment\PYZus{}dates}\PY{p}{[}\PY{o}{\PYZhy{}}\PY{l+m+mi}{1}\PY{p}{]}\PY{p}{)}\PY{p}{)}
    
    \PY{k}{def} \PY{n+nf}{npv\PYZus{}fixed\PYZus{}leg}\PY{p}{(}\PY{n+nb+bp}{self}\PY{p}{,} \PY{n}{discount\PYZus{}curve}\PY{p}{)}\PY{p}{:}
        \PY{n}{npv} \PY{o}{=} \PY{l+m+mi}{0}
        \PY{k}{for} \PY{n}{i} \PY{o+ow}{in} \PY{n+nb}{range}\PY{p}{(}\PY{l+m+mi}{1}\PY{p}{,} \PY{n+nb}{len}\PY{p}{(}\PY{n+nb+bp}{self}\PY{o}{.}\PY{n}{payment\PYZus{}dates}\PY{p}{)}\PY{p}{)}\PY{p}{:}
            \PY{n}{tau} \PY{o}{=} \PY{p}{(}\PY{n+nb+bp}{self}\PY{o}{.}\PY{n}{payment\PYZus{}dates}\PY{p}{[}\PY{n}{i}\PY{p}{]} \PY{o}{\PYZhy{}} \PY{n+nb+bp}{self}\PY{o}{.}\PY{n}{payment\PYZus{}dates}\PY{p}{[}\PY{n}{i}\PY{o}{\PYZhy{}}\PY{l+m+mi}{1}\PY{p}{]}\PY{p}{)}\PY{o}{.}\PY{n}{days}\PY{o}{/}\PY{l+m+mi}{360}
            \PY{n}{npv} \PY{o}{+}\PY{o}{=} \PY{n}{discount\PYZus{}curve}\PY{o}{.}\PY{n}{df}\PY{p}{(}\PY{n+nb+bp}{self}\PY{o}{.}\PY{n}{payment\PYZus{}dates}\PY{p}{[}\PY{n}{i}\PY{p}{]}\PY{p}{)} \PY{o}{*} \PY{n}{tau}
            
        \PY{k}{return} \PY{n+nb+bp}{self}\PY{o}{.}\PY{n}{notional} \PY{o}{*} \PY{n+nb+bp}{self}\PY{o}{.}\PY{n}{fixed\PYZus{}rate} \PY{o}{*} \PY{n}{npv}
    
    \PY{k}{def} \PY{n+nf}{npv}\PY{p}{(}\PY{n+nb+bp}{self}\PY{p}{,} \PY{n}{discount\PYZus{}curve}\PY{p}{)}\PY{p}{:}
        \PY{k}{return} \PY{n+nb+bp}{self}\PY{o}{.}\PY{n}{npv\PYZus{}floating\PYZus{}leg}\PY{p}{(}\PY{n}{discount\PYZus{}curve}\PY{p}{)} \PY{o}{\PYZhy{}} \PYZbs{}
            \PY{n+nb+bp}{self}\PY{o}{.}\PY{n}{npv\PYZus{}fixed\PYZus{}leg}\PY{p}{(}\PY{n}{discount\PYZus{}curve}\PY{p}{)}
\end{Verbatim}
\end{tcolorbox}

    To test the newly developed class we need a discount curve. In the
following example a fake curve will be defined, and then used to build
an OIS contract.

    \begin{tcolorbox}[breakable, size=fbox, boxrule=1pt, pad at break*=1mm,colback=cellbackground, colframe=cellborder]
\prompt{In}{incolor}{24}{\boxspacing}
\begin{Verbatim}[commandchars=\\\{\}]
\PY{k+kn}{from} \PY{n+nn}{datetime} \PY{k}{import} \PY{n}{date}
\PY{k+kn}{from} \PY{n+nn}{finmarkets} \PY{k}{import} \PY{n}{DiscountCurve}

\PY{n}{ois} \PY{o}{=} \PY{n}{OvernightIndexSwap}\PY{p}{(}
            \PY{c+c1}{\PYZsh{} the notional, one million}
            \PY{l+m+mf}{1e6}\PY{p}{,}
            \PY{c+c1}{\PYZsh{} the list of product dates,}
            \PY{c+c1}{\PYZsh{} i.e. the start date then the payment dates}
            \PY{p}{[}\PY{n}{date}\PY{p}{(}\PY{l+m+mi}{2020}\PY{p}{,} \PY{l+m+mi}{10}\PY{p}{,} \PY{l+m+mi}{21}\PY{p}{)}\PY{p}{,} \PY{n}{date}\PY{p}{(}\PY{l+m+mi}{2021}\PY{p}{,} \PY{l+m+mi}{1}\PY{p}{,} \PY{l+m+mi}{21}\PY{p}{)}\PY{p}{,}
             \PY{n}{date}\PY{p}{(}\PY{l+m+mi}{2021}\PY{p}{,} \PY{l+m+mi}{4}\PY{p}{,} \PY{l+m+mi}{21}\PY{p}{)}\PY{p}{,} \PY{n}{date}\PY{p}{(}\PY{l+m+mi}{2021}\PY{p}{,} \PY{l+m+mi}{7}\PY{p}{,} \PY{l+m+mi}{21}\PY{p}{)}\PY{p}{,}
             \PY{n}{date}\PY{p}{(}\PY{l+m+mi}{2021}\PY{p}{,} \PY{l+m+mi}{10}\PY{p}{,} \PY{l+m+mi}{21}\PY{p}{)}\PY{p}{]}\PY{p}{,}
            \PY{c+c1}{\PYZsh{} the fixed rate, 2.5\PYZpc{}}
            \PY{l+m+mf}{0.025}\PY{p}{)}

\PY{c+c1}{\PYZsh{} fake discount curve}
\PY{n}{curve} \PY{o}{=} \PY{n}{DiscountCurve}\PY{p}{(}\PY{n}{date}\PY{p}{(}\PY{l+m+mi}{2020}\PY{p}{,} \PY{l+m+mi}{10}\PY{p}{,} \PY{l+m+mi}{21}\PY{p}{)}\PY{p}{,}
                      \PY{p}{[}\PY{n}{date}\PY{p}{(}\PY{l+m+mi}{2020}\PY{p}{,} \PY{l+m+mi}{10}\PY{p}{,} \PY{l+m+mi}{21}\PY{p}{)}\PY{p}{,} \PY{n}{date}\PY{p}{(}\PY{l+m+mi}{2021}\PY{p}{,} \PY{l+m+mi}{6}\PY{p}{,} \PY{l+m+mi}{1}\PY{p}{)}\PY{p}{,}
                       \PY{n}{date}\PY{p}{(}\PY{l+m+mi}{2022}\PY{p}{,} \PY{l+m+mi}{1}\PY{p}{,} \PY{l+m+mi}{1}\PY{p}{)}\PY{p}{]}\PY{p}{,}
                      \PY{p}{[}\PY{l+m+mf}{1.0}\PY{p}{,} \PY{l+m+mf}{0.98}\PY{p}{,} \PY{l+m+mf}{0.82}\PY{p}{]}\PY{p}{)}

\PY{n}{ois}\PY{o}{.}\PY{n}{npv}\PY{p}{(}\PY{n}{curve}\PY{p}{)}
\end{Verbatim}
\end{tcolorbox}

            \begin{tcolorbox}[breakable, size=fbox, boxrule=.5pt, pad at break*=1mm, opacityfill=0]
\prompt{Out}{outcolor}{24}{\boxspacing}
\begin{Verbatim}[commandchars=\\\{\}]
105332.19237766218
\end{Verbatim}
\end{tcolorbox}
        
    \hypertarget{bootstrapping-technique}{%
\subsection{Bootstrapping Technique}\label{bootstrapping-technique}}

As we said before we would like to determine a \emph{real} discount
curve starting from the market quotes of a set of Overnight Index Swaps
with different maturities, this will be done via a technique called
bootstrapping. This is the ABC of financial mathematics, since you
almost always need a discount curve to price every contract. We are
going to concentrate on EONIA swaps in order to build an EUR discount
curve. The underlying assumption is that market quotes represent the
\textbf{fair} prices of the OIS so they make the swap NPVs null (the
fair price is an estimate of what a willing buyer would pay a willing
seller for a given asset, assuming both have a reasonable knowledge of
the asset's worthness).

\hypertarget{building-ois-instances}{%
\subsubsection{Building OIS Instances}\label{building-ois-instances}}

The first step involves getting data, the swap market quotes. This is
not quite easy since EONIA swap market is over the counter (OTC) and
it's not straightforward to access it. Anyway in the file
\href{https://drive.google.com/file/d/1LCEDmheKqwPXFpJ25hFz32QI5im2UJO1/view?usp=sharing}{ois\_data.xlsx}
it is available a data-set of of swap quotes (which actually are rates
rather prices) suitable for our needs. With the help of the
\(\tt{pandas}\) module the data-set can be inspected:

    \begin{tcolorbox}[breakable, size=fbox, boxrule=1pt, pad at break*=1mm,colback=cellbackground, colframe=cellborder]
\prompt{In}{incolor}{26}{\boxspacing}
\begin{Verbatim}[commandchars=\\\{\}]
\PY{c+c1}{\PYZsh{} read from ois\PYZus{}data.xlsx and set an observation\PYZus{}date}
\PY{k+kn}{import} \PY{n+nn}{pandas}
\PY{k+kn}{from} \PY{n+nn}{datetime} \PY{k}{import} \PY{n}{date}

\PY{n}{mq} \PY{o}{=} \PY{n}{pandas}\PY{o}{.}\PY{n}{read\PYZus{}excel}\PY{p}{(}\PY{l+s+s2}{\PYZdq{}}\PY{l+s+s2}{ois\PYZus{}data.xlsx}\PY{l+s+s2}{\PYZdq{}}\PY{p}{)}

\PY{n+nb}{print} \PY{p}{(}\PY{n}{mq}\PY{o}{.}\PY{n}{head}\PY{p}{(}\PY{p}{)}\PY{p}{)}
\end{Verbatim}
\end{tcolorbox}

    \begin{Verbatim}[commandchars=\\\{\}]
   months  quote
0       1 -0.350
1       2 -0.347
2       3 -0.348
3       4 -0.350
4       5 -0.350
    \end{Verbatim}

    Let's say we want to build a 15 months swap instance using data
contained in \(\tt{ois\_data}\) file. Be careful when doing this
operation and double check the units of rates, quotes, etc\ldots{}in
this case for example quotes are expressed in percent so you need to
multiply by 0.01 before using them. Another detail to check is that 15
months quote is not the fifteenth entry in the \(\tt{DataFrame}\)
(actually it is the twelfth).

    \begin{tcolorbox}[breakable, size=fbox, boxrule=1pt, pad at break*=1mm,colback=cellbackground, colframe=cellborder]
\prompt{In}{incolor}{28}{\boxspacing}
\begin{Verbatim}[commandchars=\\\{\}]
\PY{c+c1}{\PYZsh{} create an OvernightIndexSwap with a market quote}
\PY{n}{start\PYZus{}date} \PY{o}{=} \PY{n}{date}\PY{o}{.}\PY{n}{today}\PY{p}{(}\PY{p}{)}
\PY{n}{ois} \PY{o}{=} \PY{n}{OvernightIndexSwap}\PY{p}{(}\PY{l+m+mf}{1e6}\PY{p}{,} \PY{n}{generate\PYZus{}swap\PYZus{}dates}\PY{p}{(}\PY{n}{start\PYZus{}date}\PY{p}{,} \PY{l+m+mi}{15}\PY{p}{)}\PY{p}{,}
                        \PY{n}{mq}\PY{p}{[}\PY{l+s+s1}{\PYZsq{}}\PY{l+s+s1}{quote}\PY{l+s+s1}{\PYZsq{}}\PY{p}{]}\PY{o}{.}\PY{n}{tolist}\PY{p}{(}\PY{p}{)}\PY{p}{[}\PY{l+m+mi}{12}\PY{p}{]}\PY{p}{)}

\PY{n+nb}{print} \PY{p}{(}\PY{n}{ois}\PY{o}{.}\PY{n}{payment\PYZus{}dates}\PY{p}{[}\PY{o}{\PYZhy{}}\PY{l+m+mi}{1}\PY{p}{]}\PY{p}{)}
\end{Verbatim}
\end{tcolorbox}

    \begin{Verbatim}[commandchars=\\\{\}]
2022-01-20
    \end{Verbatim}

    \hypertarget{bootstrap-algorithm}{%
\subsubsection{Bootstrap Algorithm}\label{bootstrap-algorithm}}

Keep aside for a moment our swaps and introduce the \emph{bootstrap
algorithm}. In finance, bootstrap is a method for constructing a
(zero-coupon) fixed-income yield curve from the prices of a set of
coupon-bearing products, e.g.~bonds and swaps. The term structure of
spot returns is obtained from the bond yields by solving for them
recursively, by \emph{forward substitution}: this iterative process is
what is called the bootstrap method. The usefulness of bootstrap is that
using only a few carefully selected zero-coupon products, it becomes
possible to derive swap forward and spot rates for all maturities given
the solved curve.

To illustrate the algorithm let's consider the following example which
can be solved, at least parrtially, analytically. We have some coupon
paying bonds (coupon of 4\%, 5\%, 6\%, 7\% and 8\% respectively) with
maturities ranging from 1 to 5 years, each having a value of €100 and
traded at par (traded at its face value). To determine the zero-coupon
yield curve proceed as follows:

\begin{itemize}
\tightlist
\item
  at the end of first year the discounted cash flow of the first bond is
  €104 (face value plus the coupon) times the discount factor, so the
  implied \emph{fair} rate
\end{itemize}

\[100 = \cfrac{104}{(1 + S_{1y})}\implies S_{1y} =  104/100 - 1 = 4\%\]

\begin{itemize}
\tightlist
\item
  at the end of second year the sum of the cash flows of the 2nd bond
  can be compared to its trading price to compute the 2-year spot rate
  \(S_{2y}\) (using the previously derived value of \(S_{1y}\))
\end{itemize}

\[100 = \cfrac{5}{(1 + S_{1y})} + \cfrac{105}{(1 + S_{2y})^{2}}\]

\[\begin{equation*}
\begin{split}
& 100 = 5 / (1 + 0.04) + 105 / (1 + S_{2y})^{2}\qquad\Rightarrow\qquad S_{2y}^2  + 2 S_{2y}  - 0.103030 = 0 \\
& S_{2y} = - 1 \pm \sqrt{1 + 0.103030} = \begin{cases}-2.05023 \\ 0.0503\end{cases}
\end{split}
\end{equation*}\]

From the third year on, still keeping the same reasoning, we obtain
equation of third order or more which are not easily analitically
solvable. For example following the equation for the fifth bond after
five years:

\[100 = \cfrac{8} {(1 + S_{1y})} + \cfrac{8} {(1 + S_{2y})^{2}}+ \cfrac{8} {(1 + S_{3y})^{3}} + \cfrac{8} {(1 + S_{4y})^{4}} + \cfrac{108} {(1 + S_{5y})^{5}}\]

Assuming we have already determined the previous rates:

\begin{longtable}[]{@{}cccc@{}}
\toprule
years & coupon rate & bond price & rate\tabularnewline
\midrule
\endhead
1 & 1.00 \% & 100 & 4.00\%\tabularnewline
2 & 2.00 \% & 100 & 5.03\%\tabularnewline
3 & 3.00 \% & 100 & 6.08\%\tabularnewline
4 & 4.00 \% & 100 & 7.19\%\tabularnewline
5 & 5.00 \% & 100 & ???\tabularnewline
\bottomrule
\end{longtable}

we can solve also the last one numerically. This can be done by using
\(\tt{brentq}\) function, that find the zeros of an user-defined
function given a interval.

    \begin{tcolorbox}[breakable, size=fbox, boxrule=1pt, pad at break*=1mm,colback=cellbackground, colframe=cellborder]
\prompt{In}{incolor}{41}{\boxspacing}
\begin{Verbatim}[commandchars=\\\{\}]
\PY{c+c1}{\PYZsh{} find zeros of previous eq.}
\PY{k+kn}{from} \PY{n+nn}{scipy}\PY{n+nn}{.}\PY{n+nn}{optimize} \PY{k}{import} \PY{n}{brentq}

\PY{k}{def} \PY{n+nf}{func}\PY{p}{(}\PY{n}{x}\PY{p}{)}\PY{p}{:}
    \PY{k}{return} \PY{l+m+mi}{100} \PY{o}{\PYZhy{}} \PY{p}{(}\PY{l+m+mi}{8}\PY{o}{/}\PY{p}{(}\PY{l+m+mi}{1}\PY{o}{+}\PY{l+m+mf}{0.04}\PY{p}{)}\PY{p}{)} \PY{o}{\PYZhy{}} \PY{p}{(}\PY{l+m+mi}{8}\PY{o}{/}\PY{p}{(}\PY{l+m+mi}{1}\PY{o}{+}\PY{l+m+mf}{0.0503}\PY{p}{)}\PY{o}{*}\PY{o}{*}\PY{l+m+mi}{2}\PY{p}{)} \PY{o}{\PYZhy{}} \PY{p}{(}\PY{l+m+mi}{8}\PY{o}{/}\PY{p}{(}\PY{l+m+mi}{1}\PY{o}{+}\PY{l+m+mf}{0.0608}\PY{p}{)}\PY{o}{*}\PY{o}{*}\PY{l+m+mi}{3}\PY{p}{)} \PY{o}{\PYZhy{}} \PYZbs{}
            \PY{p}{(}\PY{l+m+mi}{8}\PY{o}{/}\PY{p}{(}\PY{l+m+mi}{1}\PY{o}{+}\PY{l+m+mf}{0.0719}\PY{p}{)}\PY{o}{*}\PY{o}{*}\PY{l+m+mi}{4}\PY{p}{)} \PY{o}{\PYZhy{}} \PY{p}{(}\PY{l+m+mi}{108}\PY{o}{/}\PY{p}{(}\PY{l+m+mi}{1}\PY{o}{+}\PY{n}{x}\PY{p}{)}\PY{o}{*}\PY{o}{*}\PY{l+m+mi}{5}\PY{p}{)}
    
\PY{k+kn}{from} \PY{n+nn}{matplotlib} \PY{k}{import} \PY{n}{pyplot} \PY{k}{as} \PY{n}{plt}
\PY{k+kn}{import} \PY{n+nn}{numpy}

\PY{n}{x} \PY{o}{=} \PY{n}{numpy}\PY{o}{.}\PY{n}{arange}\PY{p}{(}\PY{l+m+mf}{0.01}\PY{p}{,} \PY{o}{.}\PY{l+m+mi}{8}\PY{p}{,} \PY{l+m+mf}{0.01}\PY{p}{)}
\PY{n}{plt}\PY{o}{.}\PY{n}{plot}\PY{p}{(}\PY{n}{x}\PY{p}{,} \PY{n}{func}\PY{p}{(}\PY{n}{x}\PY{p}{)}\PY{p}{,} \PY{n}{label}\PY{o}{=}\PY{l+s+s2}{\PYZdq{}}\PY{l+s+s2}{func(x)}\PY{l+s+s2}{\PYZdq{}}\PY{p}{)}
\PY{n}{plt}\PY{o}{.}\PY{n}{grid}\PY{p}{(}\PY{k+kc}{True}\PY{p}{)}
\PY{n}{plt}\PY{o}{.}\PY{n}{xlabel}\PY{p}{(}\PY{l+s+s2}{\PYZdq{}}\PY{l+s+s2}{x}\PY{l+s+s2}{\PYZdq{}}\PY{p}{)}
\PY{n}{plt}\PY{o}{.}\PY{n}{legend}\PY{p}{(}\PY{p}{)}
\PY{n}{plt}\PY{o}{.}\PY{n}{show}\PY{p}{(}\PY{p}{)}
    

\PY{n}{r} \PY{o}{=} \PY{n}{brentq}\PY{p}{(}\PY{n}{func}\PY{p}{,} \PY{l+m+mi}{0}\PY{p}{,} \PY{l+m+mf}{0.50}\PY{p}{)}
\PY{n+nb}{print} \PY{p}{(}\PY{n}{r}\PY{p}{)}
\end{Verbatim}
\end{tcolorbox}

    \begin{center}
    \adjustimage{max size={0.9\linewidth}{0.9\paperheight}}{lesson3_files/lesson3_13_0.png}
    \end{center}
    { \hspace*{\fill} \\}
    
    \begin{Verbatim}[commandchars=\\\{\}]
0.08358879752352916
    \end{Verbatim}

    The very same mechanism can be generalized and extended to more
maturities to get a more detailed yield curve. In general terms the
previous system can be written as:

\[\begin{equation*}
\begin{cases}
f_1(S_1, p_1) = 0 \\
f_2(S_1, S_2, p_2) = 0 \\
f_3(S_1, S_2, S_3, p_3) = 0 \\
f_4(S_1, S_2, S_3, S_4, p_4) = 0 \\
\cdots
\end{cases}
\end{equation*}
\] where \(S_i\) are the unknown spot rates and \(p_i\) the market
quotes of the considered products.

The iterative procedure we have applied before exploits the first
equation to find \(S_1 = f_1^{-1}(p_1)\), the second to find
\(S_2 = f_2^{-1}(S_1, p_2)\) and so on and so forth.

This algorithm works since each equation will determine exactly one free
spot rate which is not already determined by the others.

\hypertarget{bootstrap-as-minimization-problem}{%
\subsubsection{Bootstrap as Minimization
Problem}\label{bootstrap-as-minimization-problem}}

Instead of iteratively finding the solution of each equation as before,
we could define a vector of spot rates
\(\mathbf{S} = (S_1, S_2, S_3,\ldots)\) seeking for a particular
\(\mathbf{\hat{S}}\) which solves the following equation:

\[F = f_1^2(S_1) + f_2^2(S_1, S_2) + f_3^2(S_1, S_2, S_3) + f_4^2(S_1, S_2, S_3, S_4) + \ldots = 0\]

Under this terms the bootstrap technique can be considered as a
minimization problem, indeed we need to find \(\mathbf{\hat{S}}\) which
makes \(F\) zero, or at least \emph{minimize} it making \(F\) as close
as possible to 0.

Notice that each \(f_i\) is squared since we want all of them to be
minimized and not only \(F\) globally (without the squared there may be
cancellation effects between the terms of the sum).

    \hypertarget{minimization-algorithm}{%
\subsubsection{Minimization Algorithm}\label{minimization-algorithm}}

A minimization algorithm follows these steps:

\begin{itemize}
\tightlist
\item
  define an \emph{objective function} i.e.~the function that is actually
  minimized to reach our goal;
\item
  set the initial value of the unknown parameters and their range of
  variability;
\item
  the minimizer will compute the objective function value;
\item
  then it will move the parameter values in such a way to find a smaller
  value of the objective function (e.g.~following the derivative w.r.t.
  each parameter);
\item
  if there are contraints the step above will take them into account;
\item
  the last two steps will be repeated until further variations of the
  \(\mathbf{x}\) values won't change significantly the objective
  function (i.e.~we have found a minimum of the function so the
  minimisation process is completed !).
\end{itemize}

Let's see two examples of \(\tt{python}\) applications of the
minimization function \(\tt{scipy.optimize.minimize}\)

    \hypertarget{example}{%
\subsubsection{Example}\label{example}}

Find the dimensions that will minimize the costs to manufacture a
circular cylindrical can of volume, \(33~\mathrm{cm}^3\).

Clearly to minimize the costs the company needs to reduce the can
surface, given the required volume.

\[ S(r, h) = 2\pi rh + 2\cdot(\pi r^2) \]

On the other hand we want the volume to be \(33~\mathrm{cm}^3\) so we
can remove \(h\) from the previous equation:

\[ V = \pi r^2 h = 33\quad\implies h = \cfrac{33}{\pi r^2} \]

So in the end the surface function to be minimized is:

\[ S(r) = 2\pi rh + 2\cdot(\pi r^2) = \cfrac{2\cdot 33}{r} + 2\cdot(\pi r^2)\]

So we implement the objective function \(\tt{x[0]}\) is the can radius:

    \begin{tcolorbox}[breakable, size=fbox, boxrule=1pt, pad at break*=1mm,colback=cellbackground, colframe=cellborder]
\prompt{In}{incolor}{43}{\boxspacing}
\begin{Verbatim}[commandchars=\\\{\}]
\PY{k+kn}{from} \PY{n+nn}{math} \PY{k}{import} \PY{n}{pi}

\PY{k}{def} \PY{n+nf}{objective\PYZus{}function}\PY{p}{(}\PY{n}{x}\PY{p}{)}\PY{p}{:}
    \PY{k}{return} \PY{l+m+mi}{2}\PY{o}{*}\PY{l+m+mi}{33}\PY{o}{/}\PY{n}{x}\PY{p}{[}\PY{l+m+mi}{0}\PY{p}{]} \PY{o}{+} \PY{l+m+mi}{2}\PY{o}{*}\PY{n}{pi}\PY{o}{*}\PY{n}{x}\PY{p}{[}\PY{l+m+mi}{0}\PY{p}{]}\PY{o}{*}\PY{o}{*}\PY{l+m+mi}{2}
\end{Verbatim}
\end{tcolorbox}

    Set the limits to our unknown variable and its initial value:

    \begin{tcolorbox}[breakable, size=fbox, boxrule=1pt, pad at break*=1mm,colback=cellbackground, colframe=cellborder]
\prompt{In}{incolor}{44}{\boxspacing}
\begin{Verbatim}[commandchars=\\\{\}]
\PY{n}{x0} \PY{o}{=} \PY{p}{[}\PY{l+m+mi}{1}\PY{p}{]}
\PY{n}{bounds} \PY{o}{=} \PY{p}{[}\PY{p}{(}\PY{l+m+mf}{0.01}\PY{p}{,} \PY{l+m+mi}{100}\PY{p}{)}\PY{p}{]}
\end{Verbatim}
\end{tcolorbox}

    Finally we run the minimization:

    \begin{tcolorbox}[breakable, size=fbox, boxrule=1pt, pad at break*=1mm,colback=cellbackground, colframe=cellborder]
\prompt{In}{incolor}{45}{\boxspacing}
\begin{Verbatim}[commandchars=\\\{\}]
\PY{k+kn}{from} \PY{n+nn}{scipy}\PY{n+nn}{.}\PY{n+nn}{optimize} \PY{k}{import} \PY{n}{minimize}

\PY{n}{r} \PY{o}{=} \PY{n}{minimize}\PY{p}{(}\PY{n}{objective\PYZus{}function}\PY{p}{,} \PY{n}{x0}\PY{p}{,} \PY{n}{bounds}\PY{o}{=}\PY{n}{bounds}\PY{p}{)}
\PY{n+nb}{print} \PY{p}{(}\PY{n}{r}\PY{p}{)}
\end{Verbatim}
\end{tcolorbox}

    \begin{Verbatim}[commandchars=\\\{\}]
      fun: 56.95394839811317
 hess\_inv: <1x1 LbfgsInvHessProduct with dtype=float64>
      jac: array([0.])
  message: b'CONVERGENCE: NORM\_OF\_PROJECTED\_GRADIENT\_<=\_PGTOL'
     nfev: 18
      nit: 6
   status: 0
  success: True
        x: array([1.73824646])
    \end{Verbatim}

    So to minimize the cost the company should produce cans with a radius of
about 1.74 cm (so clearly Coke didn't care about costs of can
production).

    \hypertarget{example-with-constraint}{%
\subsubsection{Example with Constraint}\label{example-with-constraint}}

We are going to fence in a rectangular field. If we look at the field
from above the cost of the vertical sides are €10/m, the cost of the
bottom is €2/m and the cost of the top is €7/m. If we have €700
determine the dimensions of the field that will maximize the enclosed
area.

In this example there are two differences w.r.t before:

\begin{itemize}
\tightlist
\item
  we want to maximize a quantity (not minimize);
\item
  there is a contraint (we have a limited amount of money).
\end{itemize}

So let's repeat the steps as before. The objective is to maximize the
enclosed area \(A\) so we can \emph{minimize} the quantity \(-A\). If we
define the length and the width of the field with \(\tt{x[0]}\) and
\(\tt{x[1]}\) (items of the list \(\tt{x}\) we get:

    \begin{tcolorbox}[breakable, size=fbox, boxrule=1pt, pad at break*=1mm,colback=cellbackground, colframe=cellborder]
\prompt{In}{incolor}{1}{\boxspacing}
\begin{Verbatim}[commandchars=\\\{\}]
\PY{k}{def} \PY{n+nf}{objective\PYZus{}function}\PY{p}{(}\PY{n}{x}\PY{p}{)}\PY{p}{:}
    \PY{k}{return} \PY{o}{\PYZhy{}} \PY{n}{x}\PY{p}{[}\PY{l+m+mi}{0}\PY{p}{]}\PY{o}{*}\PY{n}{x}\PY{p}{[}\PY{l+m+mi}{1}\PY{p}{]}
\end{Verbatim}
\end{tcolorbox}

    Now we can set the boundaries for length and width and their initial
values (1 m each):

    \begin{tcolorbox}[breakable, size=fbox, boxrule=1pt, pad at break*=1mm,colback=cellbackground, colframe=cellborder]
\prompt{In}{incolor}{47}{\boxspacing}
\begin{Verbatim}[commandchars=\\\{\}]
\PY{n}{x0} \PY{o}{=} \PY{p}{[}\PY{l+m+mi}{1}\PY{p}{,} \PY{l+m+mi}{1}\PY{p}{]}
\PY{n}{bounds} \PY{o}{=} \PY{p}{[}\PY{p}{(}\PY{l+m+mf}{0.01}\PY{p}{,} \PY{l+m+mi}{100}\PY{p}{)} \PY{k}{for} \PY{n}{\PYZus{}} \PY{o+ow}{in} \PY{n+nb}{range}\PY{p}{(}\PY{n+nb}{len}\PY{p}{(}\PY{n}{x0}\PY{p}{)}\PY{p}{)}\PY{p}{]}
\end{Verbatim}
\end{tcolorbox}

    We have also to impose the constraint on the money. This is done by
defining a function that compute the money spent with the fence and
compare it to €700. The constraint is passed to the minimizer with a
dictionary which has two keys: \(\tt{type}\) with value \(\tt{eq}\)
(like equality) since we want to spend all of our available money so the
fence has to cost €700

\[\mathrm{fence~cost} = l\cdot10 + l\cdot10 + w\cdot2 + w\cdot7 = 700\]
\[l\cdot10 + l\cdot10 + w\cdot2 + w\cdot7 - 700 = 0\]

, \(\tt{fun}\) whose value is the constraint function.

    \begin{tcolorbox}[breakable, size=fbox, boxrule=1pt, pad at break*=1mm,colback=cellbackground, colframe=cellborder]
\prompt{In}{incolor}{48}{\boxspacing}
\begin{Verbatim}[commandchars=\\\{\}]
\PY{k}{def} \PY{n+nf}{cons}\PY{p}{(}\PY{n}{x}\PY{p}{)}\PY{p}{:}
    \PY{k}{return} \PY{l+m+mi}{700} \PY{o}{\PYZhy{}} \PY{n}{x}\PY{p}{[}\PY{l+m+mi}{0}\PY{p}{]}\PY{o}{*}\PY{l+m+mi}{20} \PY{o}{\PYZhy{}} \PY{n}{x}\PY{p}{[}\PY{l+m+mi}{1}\PY{p}{]}\PY{o}{*}\PY{l+m+mi}{2} \PY{o}{\PYZhy{}} \PY{n}{x}\PY{p}{[}\PY{l+m+mi}{1}\PY{p}{]}\PY{o}{*}\PY{l+m+mi}{7}

\PY{n}{constraints} \PY{o}{=} \PY{p}{\PYZob{}}\PY{l+s+s1}{\PYZsq{}}\PY{l+s+s1}{type}\PY{l+s+s1}{\PYZsq{}}\PY{p}{:}\PY{l+s+s1}{\PYZsq{}}\PY{l+s+s1}{eq}\PY{l+s+s1}{\PYZsq{}}\PY{p}{,} \PY{l+s+s1}{\PYZsq{}}\PY{l+s+s1}{fun}\PY{l+s+s1}{\PYZsq{}}\PY{p}{:}\PY{n}{cons}\PY{p}{\PYZcb{}}
\end{Verbatim}
\end{tcolorbox}

    Finally we can call the minimizer.

    \begin{tcolorbox}[breakable, size=fbox, boxrule=1pt, pad at break*=1mm,colback=cellbackground, colframe=cellborder]
\prompt{In}{incolor}{50}{\boxspacing}
\begin{Verbatim}[commandchars=\\\{\}]
\PY{n}{r} \PY{o}{=} \PY{n}{minimize}\PY{p}{(}\PY{n}{objective\PYZus{}function}\PY{p}{,} \PY{n}{x0}\PY{p}{,} \PY{n}{bounds}\PY{o}{=}\PY{n}{bounds}\PY{p}{,} \PY{n}{constraints}\PY{o}{=}\PY{n}{constraints}\PY{p}{)}
\PY{n+nb}{print} \PY{p}{(}\PY{n}{r}\PY{p}{)}
\end{Verbatim}
\end{tcolorbox}

    \begin{Verbatim}[commandchars=\\\{\}]
     fun: -680.5555555555482
     jac: array([-38.88889313, -17.5       ])
 message: 'Optimization terminated successfully.'
    nfev: 16
     nit: 4
    njev: 4
  status: 0
 success: True
       x: array([17.49999818, 38.88889293])
    \end{Verbatim}

    So the field will come out \(17.5\) m long and \(38.9\) m wide.

    \hypertarget{ois-example}{%
\subsubsection{OIS Example}\label{ois-example}}

Back to our Overnight Index Swap, the general idea here is to find the
discount curve \(\mathcal{C}\) such that it prices as much correctly as
possible each OIS by minimizing the sum of the squared NPVs (our
\(f_i\)):

\[\mathrm{min}_{\mathcal{C}} \Big\{\sum_{i=1}^{n}\mathrm{NPV}^2(\mathrm{OIS}_i, \mathcal{C})\Big\}\]

A discount curve is characterized by pillar dates and the corresponding
discount factors. The description of the problem we have given above
does not, in theory, specifies any constraint on the number of pillar
dates of the discount curve \(\mathcal{C}\) we are going to find.
However, the pillar dates determine the number of unknown variables
(i.e.\textasciitilde{}the dimensionality \(N\) of the optimization
problem). A curve with \(N\) pillar dates has also \(N\) discount
factors (note that the first discount factor with value date equal to
the today date, is constrained to 1). \textbf{In practice, therefore, it
makes sense to choose the pillar dates in such a way that there are
exactly the right number of degrees of freedom in the optimization to
match data, so equal number of pillars and market quotes.} Hence the
natural choice is to choose the pillar dates of the discount curve equal
to the set of expiry dates of the swaps.

Once we've fixed \(\mathbf{d}\) to be a vector of pillar dates equal to
the expiry dates of the OIS swaps, and we use the notation
\(\mathbf{x}\) to represent the vector of unknown pillar discount
factors, then the problem becomes:

\[ F= \mathrm{min}_{\mathbf{x}} \Big\{\sum_{i=1}^{N}\mathrm{NPV}^2(\mathrm{OIS}_i, \mathcal{C}(\mathbf{d}, \mathbf{x}))\Big\}\qquad (f_i^2 = \mathrm{NPV}^2(\mathrm{OIS}_i, \mathcal{C}(\mathbf{d}, \mathbf{x})))\]
which is our optimization problem (\textbf{to find the minimum of the
above expression as a function of x}) that can be solved as before using
the available numerical optimization routines in \(\tt{python}\).

So first let's create the swaps according to all the available market
quotes and also the pillar dates of our final discount curve:

    \begin{tcolorbox}[breakable, size=fbox, boxrule=1pt, pad at break*=1mm,colback=cellbackground, colframe=cellborder]
\prompt{In}{incolor}{66}{\boxspacing}
\begin{Verbatim}[commandchars=\\\{\}]
\PY{c+c1}{\PYZsh{} creates the OIS from market quotes}
\PY{n}{observation\PYZus{}date} \PY{o}{=} \PY{n}{date}\PY{p}{(}\PY{l+m+mi}{2020}\PY{p}{,} \PY{l+m+mi}{10}\PY{p}{,} \PY{l+m+mi}{21}\PY{p}{)}
\PY{n}{pillar\PYZus{}dates} \PY{o}{=} \PY{p}{[}\PY{n}{observation\PYZus{}date}\PY{p}{]}
\PY{n}{swaps} \PY{o}{=} \PY{p}{[}\PY{p}{]} \PY{c+c1}{\PYZsh{} container of the OIS objects}

\PY{k}{for} \PY{n}{i} \PY{o+ow}{in} \PY{n+nb}{range}\PY{p}{(}\PY{n+nb}{len}\PY{p}{(}\PY{n}{mq}\PY{p}{)}\PY{p}{)}\PY{p}{:}
    \PY{n}{swap} \PY{o}{=} \PY{n}{OvernightIndexSwap}\PY{p}{(}\PY{l+m+mf}{1e6}\PY{p}{,} 
                             \PY{n}{generate\PYZus{}swap\PYZus{}dates}\PY{p}{(}\PY{n}{observation\PYZus{}date}\PY{p}{,}
                                                \PY{n}{mq}\PY{p}{[}\PY{l+s+s1}{\PYZsq{}}\PY{l+s+s1}{months}\PY{l+s+s1}{\PYZsq{}}\PY{p}{]}\PY{o}{.}\PY{n}{tolist}\PY{p}{(}\PY{p}{)}\PY{p}{[}\PY{n}{i}\PY{p}{]}\PY{p}{)}\PY{p}{,}
                             \PY{n}{mq}\PY{p}{[}\PY{l+s+s1}{\PYZsq{}}\PY{l+s+s1}{quote}\PY{l+s+s1}{\PYZsq{}}\PY{p}{]}\PY{o}{.}\PY{n}{tolist}\PY{p}{(}\PY{p}{)}\PY{p}{[}\PY{n}{i}\PY{p}{]}\PY{o}{*}\PY{l+m+mf}{0.01}\PY{p}{)}
    \PY{n}{swaps}\PY{o}{.}\PY{n}{append}\PY{p}{(}\PY{n}{swap}\PY{p}{)}
    \PY{n}{pillar\PYZus{}dates}\PY{o}{.}\PY{n}{append}\PY{p}{(}\PY{n}{swap}\PY{o}{.}\PY{n}{payment\PYZus{}dates}\PY{p}{[}\PY{o}{\PYZhy{}}\PY{l+m+mi}{1}\PY{p}{]}\PY{p}{)}                
\end{Verbatim}
\end{tcolorbox}

    So implement the method to the swaps we have just created.

\begin{itemize}
\tightlist
\item
  define the objective function: the sum of the squared NPVs of the OIS
\end{itemize}

    \begin{tcolorbox}[breakable, size=fbox, boxrule=1pt, pad at break*=1mm,colback=cellbackground, colframe=cellborder]
\prompt{In}{incolor}{67}{\boxspacing}
\begin{Verbatim}[commandchars=\\\{\}]
\PY{c+c1}{\PYZsh{} define objective function}
\PY{k+kn}{from} \PY{n+nn}{finmarkets} \PY{k}{import} \PY{n}{DiscountCurve}

\PY{k}{def} \PY{n+nf}{objective\PYZus{}function}\PY{p}{(}\PY{n}{x}\PY{p}{)}\PY{p}{:}
    \PY{n}{curve} \PY{o}{=} \PY{n}{DiscountCurve}\PY{p}{(}\PY{n}{observation\PYZus{}date}\PY{p}{,} 
                          \PY{n}{pillar\PYZus{}dates}\PY{p}{,} 
                          \PY{n}{x}\PY{p}{)}
    
    \PY{n}{s} \PY{o}{=} \PY{l+m+mi}{0}
    \PY{k}{for} \PY{n}{swap} \PY{o+ow}{in} \PY{n}{swaps}\PY{p}{:}
        \PY{n}{s} \PY{o}{+}\PY{o}{=} \PY{n}{swap}\PY{o}{.}\PY{n}{npv}\PY{p}{(}\PY{n}{curve}\PY{p}{)}\PY{o}{*}\PY{o}{*}\PY{l+m+mi}{2}
        
    \PY{k}{return} \PY{n}{s}
\end{Verbatim}
\end{tcolorbox}

    \begin{itemize}
\tightlist
\item
  set the initial value of the discount factors (\(x_i\)) to 1 with a
  range of variability \([ 0.01, 10]\), in addition the first element of
  the list, today's discount factor, will be fixed to 1 (variability
  \([1, 1]\))
\end{itemize}

    \begin{tcolorbox}[breakable, size=fbox, boxrule=1pt, pad at break*=1mm,colback=cellbackground, colframe=cellborder]
\prompt{In}{incolor}{70}{\boxspacing}
\begin{Verbatim}[commandchars=\\\{\}]
\PY{c+c1}{\PYZsh{} set boundaries and guess values}
\PY{n}{x0} \PY{o}{=} \PY{p}{[}\PY{l+m+mf}{1.0} \PY{k}{for} \PY{n}{\PYZus{}} \PY{o+ow}{in} \PY{n+nb}{range}\PY{p}{(}\PY{n+nb}{len}\PY{p}{(}\PY{n}{pillar\PYZus{}dates}\PY{p}{)}\PY{p}{)}\PY{p}{]}
\PY{n}{bounds} \PY{o}{=} \PY{p}{[}\PY{p}{(}\PY{l+m+mf}{0.01}\PY{p}{,} \PY{l+m+mi}{10}\PY{p}{)} \PY{k}{for} \PY{n}{\PYZus{}} \PY{o+ow}{in} \PY{n+nb}{range}\PY{p}{(}\PY{n+nb}{len}\PY{p}{(}\PY{n}{pillar\PYZus{}dates}\PY{p}{)}\PY{p}{)}\PY{p}{]}
\PY{n}{bounds}\PY{p}{[}\PY{l+m+mi}{0}\PY{p}{]} \PY{o}{=} \PY{p}{(}\PY{l+m+mi}{1}\PY{p}{,}\PY{l+m+mi}{1}\PY{p}{)}
\end{Verbatim}
\end{tcolorbox}

    \begin{itemize}
\tightlist
\item
  finally we can launch the minimizer to find the discount factors
  (\(\mathbf{x}\))
\end{itemize}

    \begin{tcolorbox}[breakable, size=fbox, boxrule=1pt, pad at break*=1mm,colback=cellbackground, colframe=cellborder]
\prompt{In}{incolor}{71}{\boxspacing}
\begin{Verbatim}[commandchars=\\\{\}]
\PY{c+c1}{\PYZsh{} minimize}
\PY{k+kn}{from} \PY{n+nn}{scipy}\PY{n+nn}{.}\PY{n+nn}{optimize} \PY{k}{import} \PY{n}{minimize}

\PY{n}{r} \PY{o}{=} \PY{n}{minimize}\PY{p}{(}\PY{n}{objective\PYZus{}function}\PY{p}{,} \PY{n}{x0}\PY{p}{,} \PY{n}{bounds}\PY{o}{=}\PY{n}{bounds}\PY{p}{)}
\PY{n+nb}{print} \PY{p}{(}\PY{n}{r}\PY{p}{)}
\end{Verbatim}
\end{tcolorbox}

    \begin{Verbatim}[commandchars=\\\{\}]
      fun: 0.0007426436661625521
 hess\_inv: <34x34 LbfgsInvHessProduct with dtype=float64>
      jac: array([ 6.42449420e+05,  8.32031905e-01,  8.90730179e-01,
9.99809437e-01,
        1.16338016e+00,  1.35595593e+00,  1.62708876e+00,  1.94173743e+00,
        2.32604199e+00,  2.75664459e+00,  3.26255699e+00,  3.81275975e+00,
       -8.91479061e+00, -6.38485847e+00, -6.96267903e+00, -7.33997369e+00,
        7.89445143e-01,  4.87668799e+00,  8.93476963e+00,  1.12652614e+01,
        1.05447179e+01,  6.40675283e+00,  4.05126471e-01, -5.17133222e+00,
       -8.13900454e+00, -7.08502139e+00, -1.49907905e+01, -1.13916619e+01,
       -3.29560798e+00,  2.82810043e+01, -6.64421843e+00, -3.85619132e+00,
       -8.55228617e+00,  6.72518724e+00])
  message: b'CONVERGENCE: REL\_REDUCTION\_OF\_F\_<=\_FACTR*EPSMCH'
     nfev: 875
      nit: 12
   status: 0
  success: True
        x: array([1.        , 1.00030147, 1.00058831, 1.00089012, 1.00119726,
       1.00147021, 1.00177765, 1.00207128, 1.00237487, 1.00266885,
       1.00297281, 1.00326758, 1.00356124, 1.00445968, 1.00530957,
       1.00614269, 1.00693061, 1.00906201, 1.0093198 , 1.00710112,
       1.0018986 , 0.99379504, 0.9833297 , 0.97101001, 0.95723164,
       0.94268861, 0.92772535, 0.88314869, 0.8178113 , 0.76554845,
       0.71988664, 0.64350636, 0.59281978, 0.54547324])
    \end{Verbatim}

    Some diagnostic number/plot.

    \begin{tcolorbox}[breakable, size=fbox, boxrule=1pt, pad at break*=1mm,colback=cellbackground, colframe=cellborder]
\prompt{In}{incolor}{18}{\boxspacing}
\begin{Verbatim}[commandchars=\\\{\}]
\PY{c+c1}{\PYZsh{} print initial and final objective function values}
\end{Verbatim}
\end{tcolorbox}

    \begin{Verbatim}[commandchars=\\\{\}]
Initial objective function value  931188216.6666666
Final objective function value  0.000819919032900304
    \end{Verbatim}

    

    Finally we can create the discount curve implied by the market quote of
our swaps.

    \begin{tcolorbox}[breakable, size=fbox, boxrule=1pt, pad at break*=1mm,colback=cellbackground, colframe=cellborder]
\prompt{In}{incolor}{21}{\boxspacing}
\begin{Verbatim}[commandchars=\\\{\}]
\PY{c+c1}{\PYZsh{} create the discount curve with our factors}
\PY{k+kn}{from} \PY{n+nn}{math} \PY{k}{import} \PY{n}{log}
\end{Verbatim}
\end{tcolorbox}

    \begin{Verbatim}[commandchars=\\\{\}]
40y df: 0.9891780176191146
40y rate: 0.0002720241491103593
    \end{Verbatim}

    \begin{tcolorbox}[breakable, size=fbox, boxrule=1pt, pad at break*=1mm,colback=cellbackground, colframe=cellborder]
\prompt{In}{incolor}{30}{\boxspacing}
\begin{Verbatim}[commandchars=\\\{\}]
\PY{k+kn}{from} \PY{n+nn}{matplotlib} \PY{k}{import} \PY{n}{pyplot} \PY{k}{as} \PY{n}{plt}

\PY{n}{dates} \PY{o}{=} \PY{p}{[}\PY{p}{(}\PY{n}{d}\PY{o}{\PYZhy{}}\PY{n}{observation\PYZus{}date}\PY{p}{)}\PY{o}{.}\PY{n}{days}\PY{o}{/}\PY{l+m+mf}{365.} \PY{k}{for} \PY{n}{d} \PY{o+ow}{in} \PY{n}{curve}\PY{o}{.}\PY{n}{pillar\PYZus{}dates}\PY{p}{]}
\PY{n}{plt}\PY{o}{.}\PY{n}{plot}\PY{p}{(}\PY{n}{dates}\PY{p}{,} \PY{n}{result}\PY{o}{.}\PY{n}{x}\PY{p}{,} \PY{n}{label}\PY{o}{=}\PY{l+s+s2}{\PYZdq{}}\PY{l+s+s2}{Discount Curve}\PY{l+s+s2}{\PYZdq{}}\PY{p}{)}
\PY{n}{plt}\PY{o}{.}\PY{n}{grid}\PY{p}{(}\PY{k+kc}{True}\PY{p}{)}
\PY{n}{plt}\PY{o}{.}\PY{n}{xlabel}\PY{p}{(}\PY{l+s+s2}{\PYZdq{}}\PY{l+s+s2}{years}\PY{l+s+s2}{\PYZdq{}}\PY{p}{)}
\PY{n}{plt}\PY{o}{.}\PY{n}{ylabel}\PY{p}{(}\PY{l+s+s2}{\PYZdq{}}\PY{l+s+s2}{disc. factor}\PY{l+s+s2}{\PYZdq{}}\PY{p}{)}
\PY{n}{plt}\PY{o}{.}\PY{n}{legend}\PY{p}{(}\PY{p}{)}
\PY{n}{plt}\PY{o}{.}\PY{n}{show}\PY{p}{(}\PY{p}{)}
\end{Verbatim}
\end{tcolorbox}

    \begin{center}
    \adjustimage{max size={0.9\linewidth}{0.9\paperheight}}{lesson3_files/lesson3_45_0.png}
    \end{center}
    { \hspace*{\fill} \\}
    
    \begin{tcolorbox}[breakable, size=fbox, boxrule=1pt, pad at break*=1mm,colback=cellbackground, colframe=cellborder]
\prompt{In}{incolor}{16}{\boxspacing}
\begin{Verbatim}[commandchars=\\\{\}]
\PY{k+kn}{from} \PY{n+nn}{scipy}\PY{n+nn}{.}\PY{n+nn}{optimize} \PY{k}{import} \PY{n}{minimize}

\PY{k}{def} \PY{n+nf}{of}\PY{p}{(}\PY{n}{x}\PY{p}{)}\PY{p}{:}
    \PY{k}{return} \PY{l+m+mi}{20000}\PY{o}{*}\PY{n}{x}\PY{p}{[}\PY{l+m+mi}{0}\PY{p}{]} \PY{o}{+} \PY{l+m+mi}{25000}\PY{o}{*}\PY{n}{x}\PY{p}{[}\PY{l+m+mi}{1}\PY{p}{]}

\PY{k}{def} \PY{n+nf}{cons1}\PY{p}{(}\PY{n}{x}\PY{p}{)}\PY{p}{:}
    \PY{k}{return} \PY{l+m+mi}{400}\PY{o}{*}\PY{n}{x}\PY{p}{[}\PY{l+m+mi}{0}\PY{p}{]} \PY{o}{+} \PY{l+m+mi}{300}\PY{o}{*}\PY{n}{x}\PY{p}{[}\PY{l+m+mi}{1}\PY{p}{]} \PY{o}{\PYZhy{}} \PY{l+m+mi}{25000}

\PY{k}{def} \PY{n+nf}{cons2}\PY{p}{(}\PY{n}{x}\PY{p}{)}\PY{p}{:}
    \PY{k}{return} \PY{l+m+mi}{300}\PY{o}{*}\PY{n}{x}\PY{p}{[}\PY{l+m+mi}{0}\PY{p}{]} \PY{o}{+} \PY{l+m+mi}{400}\PY{o}{*}\PY{n}{x}\PY{p}{[}\PY{l+m+mi}{1}\PY{p}{]} \PY{o}{\PYZhy{}} \PY{l+m+mi}{27000}

\PY{k}{def} \PY{n+nf}{cons3}\PY{p}{(}\PY{n}{x}\PY{p}{)}\PY{p}{:}
    \PY{k}{return} \PY{l+m+mi}{200}\PY{o}{*}\PY{n}{x}\PY{p}{[}\PY{l+m+mi}{0}\PY{p}{]} \PY{o}{+} \PY{l+m+mi}{500}\PY{o}{*}\PY{n}{x}\PY{p}{[}\PY{l+m+mi}{1}\PY{p}{]} \PY{o}{\PYZhy{}} \PY{l+m+mi}{30000}

\PY{n}{cons} \PY{o}{=} \PY{p}{[}\PY{p}{\PYZob{}}\PY{l+s+s2}{\PYZdq{}}\PY{l+s+s2}{type}\PY{l+s+s2}{\PYZdq{}}\PY{p}{:}\PY{l+s+s2}{\PYZdq{}}\PY{l+s+s2}{ineq}\PY{l+s+s2}{\PYZdq{}}\PY{p}{,} \PY{l+s+s2}{\PYZdq{}}\PY{l+s+s2}{fun}\PY{l+s+s2}{\PYZdq{}}\PY{p}{:}\PY{n}{cons1}\PY{p}{\PYZcb{}}\PY{p}{,}
        \PY{p}{\PYZob{}}\PY{l+s+s2}{\PYZdq{}}\PY{l+s+s2}{type}\PY{l+s+s2}{\PYZdq{}}\PY{p}{:}\PY{l+s+s2}{\PYZdq{}}\PY{l+s+s2}{ineq}\PY{l+s+s2}{\PYZdq{}}\PY{p}{,} \PY{l+s+s2}{\PYZdq{}}\PY{l+s+s2}{fun}\PY{l+s+s2}{\PYZdq{}}\PY{p}{:}\PY{n}{cons2}\PY{p}{\PYZcb{}}\PY{p}{,}
        \PY{p}{\PYZob{}}\PY{l+s+s2}{\PYZdq{}}\PY{l+s+s2}{type}\PY{l+s+s2}{\PYZdq{}}\PY{p}{:}\PY{l+s+s2}{\PYZdq{}}\PY{l+s+s2}{ineq}\PY{l+s+s2}{\PYZdq{}}\PY{p}{,} \PY{l+s+s2}{\PYZdq{}}\PY{l+s+s2}{fun}\PY{l+s+s2}{\PYZdq{}}\PY{p}{:}\PY{n}{cons3}\PY{p}{\PYZcb{}}\PY{p}{]}
        
\PY{n}{x0} \PY{o}{=} \PY{p}{[}\PY{l+m+mi}{10}\PY{p}{,} \PY{l+m+mi}{10}\PY{p}{]}
\PY{n}{bounds} \PY{o}{=} \PY{p}{[}\PY{p}{(}\PY{l+m+mi}{0}\PY{p}{,} \PY{l+m+mi}{100}\PY{p}{)} \PY{k}{for} \PY{n}{\PYZus{}} \PY{o+ow}{in} \PY{n+nb}{range}\PY{p}{(}\PY{n+nb}{len}\PY{p}{(}\PY{n}{x0}\PY{p}{)}\PY{p}{)}\PY{p}{]}

\PY{n}{r} \PY{o}{=} \PY{n}{minimize}\PY{p}{(}\PY{n}{of}\PY{p}{,} \PY{n}{x0}\PY{p}{,} \PY{n}{bounds}\PY{o}{=}\PY{n}{bounds}\PY{p}{,} \PY{n}{constraints}\PY{o}{=}\PY{n}{cons}\PY{p}{)}
\PY{n+nb}{print} \PY{p}{(}\PY{n}{r}\PY{p}{)}
\end{Verbatim}
\end{tcolorbox}

    \begin{Verbatim}[commandchars=\\\{\}]
     fun: 1750002.070622686
     jac: array([20000., 25000.])
 message: 'Optimization terminated successfully.'
    nfev: 8
     nit: 6
    njev: 2
  status: 0
 success: True
       x: array([25.00004033, 50.00005056])
    \end{Verbatim}


    % Add a bibliography block to the postdoc
    
    
    
\end{document}

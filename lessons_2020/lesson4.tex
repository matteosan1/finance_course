\documentclass[11pt]{article}

    \usepackage[breakable]{tcolorbox}
    \usepackage{parskip} % Stop auto-indenting (to mimic markdown behaviour)
    
    \usepackage{iftex}
    \ifPDFTeX
    	\usepackage[T1]{fontenc}
    	\usepackage{mathpazo}
    \else
    	\usepackage{fontspec}
    \fi

    % Basic figure setup, for now with no caption control since it's done
    % automatically by Pandoc (which extracts ![](path) syntax from Markdown).
    \usepackage{graphicx}
    % Maintain compatibility with old templates. Remove in nbconvert 6.0
    \let\Oldincludegraphics\includegraphics
    % Ensure that by default, figures have no caption (until we provide a
    % proper Figure object with a Caption API and a way to capture that
    % in the conversion process - todo).
    \usepackage{caption}
    \DeclareCaptionFormat{nocaption}{}
    \captionsetup{format=nocaption,aboveskip=0pt,belowskip=0pt}

    \usepackage[Export]{adjustbox} % Used to constrain images to a maximum size
    \adjustboxset{max size={0.9\linewidth}{0.9\paperheight}}
    \usepackage{float}
    \floatplacement{figure}{H} % forces figures to be placed at the correct location
    \usepackage{xcolor} % Allow colors to be defined
    \usepackage{enumerate} % Needed for markdown enumerations to work
    \usepackage{geometry} % Used to adjust the document margins
    \usepackage{amsmath} % Equations
    \usepackage{amssymb} % Equations
    \usepackage{textcomp} % defines textquotesingle
    % Hack from http://tex.stackexchange.com/a/47451/13684:
    \AtBeginDocument{%
        \def\PYZsq{\textquotesingle}% Upright quotes in Pygmentized code
    }
    \usepackage{upquote} % Upright quotes for verbatim code
    \usepackage{eurosym} % defines \euro
    \usepackage[mathletters]{ucs} % Extended unicode (utf-8) support
    \usepackage{fancyvrb} % verbatim replacement that allows latex
    \usepackage{grffile} % extends the file name processing of package graphics 
                         % to support a larger range
    \makeatletter % fix for grffile with XeLaTeX
    \def\Gread@@xetex#1{%
      \IfFileExists{"\Gin@base".bb}%
      {\Gread@eps{\Gin@base.bb}}%
      {\Gread@@xetex@aux#1}%
    }
    \makeatother

    % The hyperref package gives us a pdf with properly built
    % internal navigation ('pdf bookmarks' for the table of contents,
    % internal cross-reference links, web links for URLs, etc.)
    \usepackage{hyperref}
    % The default LaTeX title has an obnoxious amount of whitespace. By default,
    % titling removes some of it. It also provides customization options.
    \usepackage{titling}
    \usepackage{longtable} % longtable support required by pandoc >1.10
    \usepackage{booktabs}  % table support for pandoc > 1.12.2
    \usepackage[inline]{enumitem} % IRkernel/repr support (it uses the enumerate* environment)
    \usepackage[normalem]{ulem} % ulem is needed to support strikethroughs (\sout)
                                % normalem makes italics be italics, not underlines
    \usepackage{mathrsfs}
    

    
    % Colors for the hyperref package
    \definecolor{urlcolor}{rgb}{0,.145,.698}
    \definecolor{linkcolor}{rgb}{.71,0.21,0.01}
    \definecolor{citecolor}{rgb}{.12,.54,.11}

    % ANSI colors
    \definecolor{ansi-black}{HTML}{3E424D}
    \definecolor{ansi-black-intense}{HTML}{282C36}
    \definecolor{ansi-red}{HTML}{E75C58}
    \definecolor{ansi-red-intense}{HTML}{B22B31}
    \definecolor{ansi-green}{HTML}{00A250}
    \definecolor{ansi-green-intense}{HTML}{007427}
    \definecolor{ansi-yellow}{HTML}{DDB62B}
    \definecolor{ansi-yellow-intense}{HTML}{B27D12}
    \definecolor{ansi-blue}{HTML}{208FFB}
    \definecolor{ansi-blue-intense}{HTML}{0065CA}
    \definecolor{ansi-magenta}{HTML}{D160C4}
    \definecolor{ansi-magenta-intense}{HTML}{A03196}
    \definecolor{ansi-cyan}{HTML}{60C6C8}
    \definecolor{ansi-cyan-intense}{HTML}{258F8F}
    \definecolor{ansi-white}{HTML}{C5C1B4}
    \definecolor{ansi-white-intense}{HTML}{A1A6B2}
    \definecolor{ansi-default-inverse-fg}{HTML}{FFFFFF}
    \definecolor{ansi-default-inverse-bg}{HTML}{000000}

    % commands and environments needed by pandoc snippets
    % extracted from the output of `pandoc -s`
    \providecommand{\tightlist}{%
      \setlength{\itemsep}{0pt}\setlength{\parskip}{0pt}}
    \DefineVerbatimEnvironment{Highlighting}{Verbatim}{commandchars=\\\{\}}
    % Add ',fontsize=\small' for more characters per line
    \newenvironment{Shaded}{}{}
    \newcommand{\KeywordTok}[1]{\textcolor[rgb]{0.00,0.44,0.13}{\textbf{{#1}}}}
    \newcommand{\DataTypeTok}[1]{\textcolor[rgb]{0.56,0.13,0.00}{{#1}}}
    \newcommand{\DecValTok}[1]{\textcolor[rgb]{0.25,0.63,0.44}{{#1}}}
    \newcommand{\BaseNTok}[1]{\textcolor[rgb]{0.25,0.63,0.44}{{#1}}}
    \newcommand{\FloatTok}[1]{\textcolor[rgb]{0.25,0.63,0.44}{{#1}}}
    \newcommand{\CharTok}[1]{\textcolor[rgb]{0.25,0.44,0.63}{{#1}}}
    \newcommand{\StringTok}[1]{\textcolor[rgb]{0.25,0.44,0.63}{{#1}}}
    \newcommand{\CommentTok}[1]{\textcolor[rgb]{0.38,0.63,0.69}{\textit{{#1}}}}
    \newcommand{\OtherTok}[1]{\textcolor[rgb]{0.00,0.44,0.13}{{#1}}}
    \newcommand{\AlertTok}[1]{\textcolor[rgb]{1.00,0.00,0.00}{\textbf{{#1}}}}
    \newcommand{\FunctionTok}[1]{\textcolor[rgb]{0.02,0.16,0.49}{{#1}}}
    \newcommand{\RegionMarkerTok}[1]{{#1}}
    \newcommand{\ErrorTok}[1]{\textcolor[rgb]{1.00,0.00,0.00}{\textbf{{#1}}}}
    \newcommand{\NormalTok}[1]{{#1}}
    
    % Additional commands for more recent versions of Pandoc
    \newcommand{\ConstantTok}[1]{\textcolor[rgb]{0.53,0.00,0.00}{{#1}}}
    \newcommand{\SpecialCharTok}[1]{\textcolor[rgb]{0.25,0.44,0.63}{{#1}}}
    \newcommand{\VerbatimStringTok}[1]{\textcolor[rgb]{0.25,0.44,0.63}{{#1}}}
    \newcommand{\SpecialStringTok}[1]{\textcolor[rgb]{0.73,0.40,0.53}{{#1}}}
    \newcommand{\ImportTok}[1]{{#1}}
    \newcommand{\DocumentationTok}[1]{\textcolor[rgb]{0.73,0.13,0.13}{\textit{{#1}}}}
    \newcommand{\AnnotationTok}[1]{\textcolor[rgb]{0.38,0.63,0.69}{\textbf{\textit{{#1}}}}}
    \newcommand{\CommentVarTok}[1]{\textcolor[rgb]{0.38,0.63,0.69}{\textbf{\textit{{#1}}}}}
    \newcommand{\VariableTok}[1]{\textcolor[rgb]{0.10,0.09,0.49}{{#1}}}
    \newcommand{\ControlFlowTok}[1]{\textcolor[rgb]{0.00,0.44,0.13}{\textbf{{#1}}}}
    \newcommand{\OperatorTok}[1]{\textcolor[rgb]{0.40,0.40,0.40}{{#1}}}
    \newcommand{\BuiltInTok}[1]{{#1}}
    \newcommand{\ExtensionTok}[1]{{#1}}
    \newcommand{\PreprocessorTok}[1]{\textcolor[rgb]{0.74,0.48,0.00}{{#1}}}
    \newcommand{\AttributeTok}[1]{\textcolor[rgb]{0.49,0.56,0.16}{{#1}}}
    \newcommand{\InformationTok}[1]{\textcolor[rgb]{0.38,0.63,0.69}{\textbf{\textit{{#1}}}}}
    \newcommand{\WarningTok}[1]{\textcolor[rgb]{0.38,0.63,0.69}{\textbf{\textit{{#1}}}}}
    
    
    % Define a nice break command that doesn't care if a line doesn't already
    % exist.
    \def\br{\hspace*{\fill} \\* }
    % Math Jax compatibility definitions
    \def\gt{>}
    \def\lt{<}
    \let\Oldtex\TeX
    \let\Oldlatex\LaTeX
    \renewcommand{\TeX}{\textrm{\Oldtex}}
    \renewcommand{\LaTeX}{\textrm{\Oldlatex}}
    % Document parameters
    % Document title
    \title{lesson4}
    
    
    
    
    
% Pygments definitions
\makeatletter
\def\PY@reset{\let\PY@it=\relax \let\PY@bf=\relax%
    \let\PY@ul=\relax \let\PY@tc=\relax%
    \let\PY@bc=\relax \let\PY@ff=\relax}
\def\PY@tok#1{\csname PY@tok@#1\endcsname}
\def\PY@toks#1+{\ifx\relax#1\empty\else%
    \PY@tok{#1}\expandafter\PY@toks\fi}
\def\PY@do#1{\PY@bc{\PY@tc{\PY@ul{%
    \PY@it{\PY@bf{\PY@ff{#1}}}}}}}
\def\PY#1#2{\PY@reset\PY@toks#1+\relax+\PY@do{#2}}

\expandafter\def\csname PY@tok@w\endcsname{\def\PY@tc##1{\textcolor[rgb]{0.73,0.73,0.73}{##1}}}
\expandafter\def\csname PY@tok@c\endcsname{\let\PY@it=\textit\def\PY@tc##1{\textcolor[rgb]{0.25,0.50,0.50}{##1}}}
\expandafter\def\csname PY@tok@cp\endcsname{\def\PY@tc##1{\textcolor[rgb]{0.74,0.48,0.00}{##1}}}
\expandafter\def\csname PY@tok@k\endcsname{\let\PY@bf=\textbf\def\PY@tc##1{\textcolor[rgb]{0.00,0.50,0.00}{##1}}}
\expandafter\def\csname PY@tok@kp\endcsname{\def\PY@tc##1{\textcolor[rgb]{0.00,0.50,0.00}{##1}}}
\expandafter\def\csname PY@tok@kt\endcsname{\def\PY@tc##1{\textcolor[rgb]{0.69,0.00,0.25}{##1}}}
\expandafter\def\csname PY@tok@o\endcsname{\def\PY@tc##1{\textcolor[rgb]{0.40,0.40,0.40}{##1}}}
\expandafter\def\csname PY@tok@ow\endcsname{\let\PY@bf=\textbf\def\PY@tc##1{\textcolor[rgb]{0.67,0.13,1.00}{##1}}}
\expandafter\def\csname PY@tok@nb\endcsname{\def\PY@tc##1{\textcolor[rgb]{0.00,0.50,0.00}{##1}}}
\expandafter\def\csname PY@tok@nf\endcsname{\def\PY@tc##1{\textcolor[rgb]{0.00,0.00,1.00}{##1}}}
\expandafter\def\csname PY@tok@nc\endcsname{\let\PY@bf=\textbf\def\PY@tc##1{\textcolor[rgb]{0.00,0.00,1.00}{##1}}}
\expandafter\def\csname PY@tok@nn\endcsname{\let\PY@bf=\textbf\def\PY@tc##1{\textcolor[rgb]{0.00,0.00,1.00}{##1}}}
\expandafter\def\csname PY@tok@ne\endcsname{\let\PY@bf=\textbf\def\PY@tc##1{\textcolor[rgb]{0.82,0.25,0.23}{##1}}}
\expandafter\def\csname PY@tok@nv\endcsname{\def\PY@tc##1{\textcolor[rgb]{0.10,0.09,0.49}{##1}}}
\expandafter\def\csname PY@tok@no\endcsname{\def\PY@tc##1{\textcolor[rgb]{0.53,0.00,0.00}{##1}}}
\expandafter\def\csname PY@tok@nl\endcsname{\def\PY@tc##1{\textcolor[rgb]{0.63,0.63,0.00}{##1}}}
\expandafter\def\csname PY@tok@ni\endcsname{\let\PY@bf=\textbf\def\PY@tc##1{\textcolor[rgb]{0.60,0.60,0.60}{##1}}}
\expandafter\def\csname PY@tok@na\endcsname{\def\PY@tc##1{\textcolor[rgb]{0.49,0.56,0.16}{##1}}}
\expandafter\def\csname PY@tok@nt\endcsname{\let\PY@bf=\textbf\def\PY@tc##1{\textcolor[rgb]{0.00,0.50,0.00}{##1}}}
\expandafter\def\csname PY@tok@nd\endcsname{\def\PY@tc##1{\textcolor[rgb]{0.67,0.13,1.00}{##1}}}
\expandafter\def\csname PY@tok@s\endcsname{\def\PY@tc##1{\textcolor[rgb]{0.73,0.13,0.13}{##1}}}
\expandafter\def\csname PY@tok@sd\endcsname{\let\PY@it=\textit\def\PY@tc##1{\textcolor[rgb]{0.73,0.13,0.13}{##1}}}
\expandafter\def\csname PY@tok@si\endcsname{\let\PY@bf=\textbf\def\PY@tc##1{\textcolor[rgb]{0.73,0.40,0.53}{##1}}}
\expandafter\def\csname PY@tok@se\endcsname{\let\PY@bf=\textbf\def\PY@tc##1{\textcolor[rgb]{0.73,0.40,0.13}{##1}}}
\expandafter\def\csname PY@tok@sr\endcsname{\def\PY@tc##1{\textcolor[rgb]{0.73,0.40,0.53}{##1}}}
\expandafter\def\csname PY@tok@ss\endcsname{\def\PY@tc##1{\textcolor[rgb]{0.10,0.09,0.49}{##1}}}
\expandafter\def\csname PY@tok@sx\endcsname{\def\PY@tc##1{\textcolor[rgb]{0.00,0.50,0.00}{##1}}}
\expandafter\def\csname PY@tok@m\endcsname{\def\PY@tc##1{\textcolor[rgb]{0.40,0.40,0.40}{##1}}}
\expandafter\def\csname PY@tok@gh\endcsname{\let\PY@bf=\textbf\def\PY@tc##1{\textcolor[rgb]{0.00,0.00,0.50}{##1}}}
\expandafter\def\csname PY@tok@gu\endcsname{\let\PY@bf=\textbf\def\PY@tc##1{\textcolor[rgb]{0.50,0.00,0.50}{##1}}}
\expandafter\def\csname PY@tok@gd\endcsname{\def\PY@tc##1{\textcolor[rgb]{0.63,0.00,0.00}{##1}}}
\expandafter\def\csname PY@tok@gi\endcsname{\def\PY@tc##1{\textcolor[rgb]{0.00,0.63,0.00}{##1}}}
\expandafter\def\csname PY@tok@gr\endcsname{\def\PY@tc##1{\textcolor[rgb]{1.00,0.00,0.00}{##1}}}
\expandafter\def\csname PY@tok@ge\endcsname{\let\PY@it=\textit}
\expandafter\def\csname PY@tok@gs\endcsname{\let\PY@bf=\textbf}
\expandafter\def\csname PY@tok@gp\endcsname{\let\PY@bf=\textbf\def\PY@tc##1{\textcolor[rgb]{0.00,0.00,0.50}{##1}}}
\expandafter\def\csname PY@tok@go\endcsname{\def\PY@tc##1{\textcolor[rgb]{0.53,0.53,0.53}{##1}}}
\expandafter\def\csname PY@tok@gt\endcsname{\def\PY@tc##1{\textcolor[rgb]{0.00,0.27,0.87}{##1}}}
\expandafter\def\csname PY@tok@err\endcsname{\def\PY@bc##1{\setlength{\fboxsep}{0pt}\fcolorbox[rgb]{1.00,0.00,0.00}{1,1,1}{\strut ##1}}}
\expandafter\def\csname PY@tok@kc\endcsname{\let\PY@bf=\textbf\def\PY@tc##1{\textcolor[rgb]{0.00,0.50,0.00}{##1}}}
\expandafter\def\csname PY@tok@kd\endcsname{\let\PY@bf=\textbf\def\PY@tc##1{\textcolor[rgb]{0.00,0.50,0.00}{##1}}}
\expandafter\def\csname PY@tok@kn\endcsname{\let\PY@bf=\textbf\def\PY@tc##1{\textcolor[rgb]{0.00,0.50,0.00}{##1}}}
\expandafter\def\csname PY@tok@kr\endcsname{\let\PY@bf=\textbf\def\PY@tc##1{\textcolor[rgb]{0.00,0.50,0.00}{##1}}}
\expandafter\def\csname PY@tok@bp\endcsname{\def\PY@tc##1{\textcolor[rgb]{0.00,0.50,0.00}{##1}}}
\expandafter\def\csname PY@tok@fm\endcsname{\def\PY@tc##1{\textcolor[rgb]{0.00,0.00,1.00}{##1}}}
\expandafter\def\csname PY@tok@vc\endcsname{\def\PY@tc##1{\textcolor[rgb]{0.10,0.09,0.49}{##1}}}
\expandafter\def\csname PY@tok@vg\endcsname{\def\PY@tc##1{\textcolor[rgb]{0.10,0.09,0.49}{##1}}}
\expandafter\def\csname PY@tok@vi\endcsname{\def\PY@tc##1{\textcolor[rgb]{0.10,0.09,0.49}{##1}}}
\expandafter\def\csname PY@tok@vm\endcsname{\def\PY@tc##1{\textcolor[rgb]{0.10,0.09,0.49}{##1}}}
\expandafter\def\csname PY@tok@sa\endcsname{\def\PY@tc##1{\textcolor[rgb]{0.73,0.13,0.13}{##1}}}
\expandafter\def\csname PY@tok@sb\endcsname{\def\PY@tc##1{\textcolor[rgb]{0.73,0.13,0.13}{##1}}}
\expandafter\def\csname PY@tok@sc\endcsname{\def\PY@tc##1{\textcolor[rgb]{0.73,0.13,0.13}{##1}}}
\expandafter\def\csname PY@tok@dl\endcsname{\def\PY@tc##1{\textcolor[rgb]{0.73,0.13,0.13}{##1}}}
\expandafter\def\csname PY@tok@s2\endcsname{\def\PY@tc##1{\textcolor[rgb]{0.73,0.13,0.13}{##1}}}
\expandafter\def\csname PY@tok@sh\endcsname{\def\PY@tc##1{\textcolor[rgb]{0.73,0.13,0.13}{##1}}}
\expandafter\def\csname PY@tok@s1\endcsname{\def\PY@tc##1{\textcolor[rgb]{0.73,0.13,0.13}{##1}}}
\expandafter\def\csname PY@tok@mb\endcsname{\def\PY@tc##1{\textcolor[rgb]{0.40,0.40,0.40}{##1}}}
\expandafter\def\csname PY@tok@mf\endcsname{\def\PY@tc##1{\textcolor[rgb]{0.40,0.40,0.40}{##1}}}
\expandafter\def\csname PY@tok@mh\endcsname{\def\PY@tc##1{\textcolor[rgb]{0.40,0.40,0.40}{##1}}}
\expandafter\def\csname PY@tok@mi\endcsname{\def\PY@tc##1{\textcolor[rgb]{0.40,0.40,0.40}{##1}}}
\expandafter\def\csname PY@tok@il\endcsname{\def\PY@tc##1{\textcolor[rgb]{0.40,0.40,0.40}{##1}}}
\expandafter\def\csname PY@tok@mo\endcsname{\def\PY@tc##1{\textcolor[rgb]{0.40,0.40,0.40}{##1}}}
\expandafter\def\csname PY@tok@ch\endcsname{\let\PY@it=\textit\def\PY@tc##1{\textcolor[rgb]{0.25,0.50,0.50}{##1}}}
\expandafter\def\csname PY@tok@cm\endcsname{\let\PY@it=\textit\def\PY@tc##1{\textcolor[rgb]{0.25,0.50,0.50}{##1}}}
\expandafter\def\csname PY@tok@cpf\endcsname{\let\PY@it=\textit\def\PY@tc##1{\textcolor[rgb]{0.25,0.50,0.50}{##1}}}
\expandafter\def\csname PY@tok@c1\endcsname{\let\PY@it=\textit\def\PY@tc##1{\textcolor[rgb]{0.25,0.50,0.50}{##1}}}
\expandafter\def\csname PY@tok@cs\endcsname{\let\PY@it=\textit\def\PY@tc##1{\textcolor[rgb]{0.25,0.50,0.50}{##1}}}

\def\PYZbs{\char`\\}
\def\PYZus{\char`\_}
\def\PYZob{\char`\{}
\def\PYZcb{\char`\}}
\def\PYZca{\char`\^}
\def\PYZam{\char`\&}
\def\PYZlt{\char`\<}
\def\PYZgt{\char`\>}
\def\PYZsh{\char`\#}
\def\PYZpc{\char`\%}
\def\PYZdl{\char`\$}
\def\PYZhy{\char`\-}
\def\PYZsq{\char`\'}
\def\PYZdq{\char`\"}
\def\PYZti{\char`\~}
% for compatibility with earlier versions
\def\PYZat{@}
\def\PYZlb{[}
\def\PYZrb{]}
\makeatother


    % For linebreaks inside Verbatim environment from package fancyvrb. 
    \makeatletter
        \newbox\Wrappedcontinuationbox 
        \newbox\Wrappedvisiblespacebox 
        \newcommand*\Wrappedvisiblespace {\textcolor{red}{\textvisiblespace}} 
        \newcommand*\Wrappedcontinuationsymbol {\textcolor{red}{\llap{\tiny$\m@th\hookrightarrow$}}} 
        \newcommand*\Wrappedcontinuationindent {3ex } 
        \newcommand*\Wrappedafterbreak {\kern\Wrappedcontinuationindent\copy\Wrappedcontinuationbox} 
        % Take advantage of the already applied Pygments mark-up to insert 
        % potential linebreaks for TeX processing. 
        %        {, <, #, %, $, ' and ": go to next line. 
        %        _, }, ^, &, >, - and ~: stay at end of broken line. 
        % Use of \textquotesingle for straight quote. 
        \newcommand*\Wrappedbreaksatspecials {% 
            \def\PYGZus{\discretionary{\char`\_}{\Wrappedafterbreak}{\char`\_}}% 
            \def\PYGZob{\discretionary{}{\Wrappedafterbreak\char`\{}{\char`\{}}% 
            \def\PYGZcb{\discretionary{\char`\}}{\Wrappedafterbreak}{\char`\}}}% 
            \def\PYGZca{\discretionary{\char`\^}{\Wrappedafterbreak}{\char`\^}}% 
            \def\PYGZam{\discretionary{\char`\&}{\Wrappedafterbreak}{\char`\&}}% 
            \def\PYGZlt{\discretionary{}{\Wrappedafterbreak\char`\<}{\char`\<}}% 
            \def\PYGZgt{\discretionary{\char`\>}{\Wrappedafterbreak}{\char`\>}}% 
            \def\PYGZsh{\discretionary{}{\Wrappedafterbreak\char`\#}{\char`\#}}% 
            \def\PYGZpc{\discretionary{}{\Wrappedafterbreak\char`\%}{\char`\%}}% 
            \def\PYGZdl{\discretionary{}{\Wrappedafterbreak\char`\$}{\char`\$}}% 
            \def\PYGZhy{\discretionary{\char`\-}{\Wrappedafterbreak}{\char`\-}}% 
            \def\PYGZsq{\discretionary{}{\Wrappedafterbreak\textquotesingle}{\textquotesingle}}% 
            \def\PYGZdq{\discretionary{}{\Wrappedafterbreak\char`\"}{\char`\"}}% 
            \def\PYGZti{\discretionary{\char`\~}{\Wrappedafterbreak}{\char`\~}}% 
        } 
        % Some characters . , ; ? ! / are not pygmentized. 
        % This macro makes them "active" and they will insert potential linebreaks 
        \newcommand*\Wrappedbreaksatpunct {% 
            \lccode`\~`\.\lowercase{\def~}{\discretionary{\hbox{\char`\.}}{\Wrappedafterbreak}{\hbox{\char`\.}}}% 
            \lccode`\~`\,\lowercase{\def~}{\discretionary{\hbox{\char`\,}}{\Wrappedafterbreak}{\hbox{\char`\,}}}% 
            \lccode`\~`\;\lowercase{\def~}{\discretionary{\hbox{\char`\;}}{\Wrappedafterbreak}{\hbox{\char`\;}}}% 
            \lccode`\~`\:\lowercase{\def~}{\discretionary{\hbox{\char`\:}}{\Wrappedafterbreak}{\hbox{\char`\:}}}% 
            \lccode`\~`\?\lowercase{\def~}{\discretionary{\hbox{\char`\?}}{\Wrappedafterbreak}{\hbox{\char`\?}}}% 
            \lccode`\~`\!\lowercase{\def~}{\discretionary{\hbox{\char`\!}}{\Wrappedafterbreak}{\hbox{\char`\!}}}% 
            \lccode`\~`\/\lowercase{\def~}{\discretionary{\hbox{\char`\/}}{\Wrappedafterbreak}{\hbox{\char`\/}}}% 
            \catcode`\.\active
            \catcode`\,\active 
            \catcode`\;\active
            \catcode`\:\active
            \catcode`\?\active
            \catcode`\!\active
            \catcode`\/\active 
            \lccode`\~`\~ 	
        }
    \makeatother

    \let\OriginalVerbatim=\Verbatim
    \makeatletter
    \renewcommand{\Verbatim}[1][1]{%
        %\parskip\z@skip
        \sbox\Wrappedcontinuationbox {\Wrappedcontinuationsymbol}%
        \sbox\Wrappedvisiblespacebox {\FV@SetupFont\Wrappedvisiblespace}%
        \def\FancyVerbFormatLine ##1{\hsize\linewidth
            \vtop{\raggedright\hyphenpenalty\z@\exhyphenpenalty\z@
                \doublehyphendemerits\z@\finalhyphendemerits\z@
                \strut ##1\strut}%
        }%
        % If the linebreak is at a space, the latter will be displayed as visible
        % space at end of first line, and a continuation symbol starts next line.
        % Stretch/shrink are however usually zero for typewriter font.
        \def\FV@Space {%
            \nobreak\hskip\z@ plus\fontdimen3\font minus\fontdimen4\font
            \discretionary{\copy\Wrappedvisiblespacebox}{\Wrappedafterbreak}
            {\kern\fontdimen2\font}%
        }%
        
        % Allow breaks at special characters using \PYG... macros.
        \Wrappedbreaksatspecials
        % Breaks at punctuation characters . , ; ? ! and / need catcode=\active 	
        \OriginalVerbatim[#1,codes*=\Wrappedbreaksatpunct]%
    }
    \makeatother

    % Exact colors from NB
    \definecolor{incolor}{HTML}{303F9F}
    \definecolor{outcolor}{HTML}{D84315}
    \definecolor{cellborder}{HTML}{CFCFCF}
    \definecolor{cellbackground}{HTML}{F7F7F7}
    
    % prompt
    \makeatletter
    \newcommand{\boxspacing}{\kern\kvtcb@left@rule\kern\kvtcb@boxsep}
    \makeatother
    \newcommand{\prompt}[4]{
        \ttfamily\llap{{\color{#2}[#3]:\hspace{3pt}#4}}\vspace{-\baselineskip}
    }
    

    
    % Prevent overflowing lines due to hard-to-break entities
    \sloppy 
    % Setup hyperref package
    \hypersetup{
      breaklinks=true,  % so long urls are correctly broken across lines
      colorlinks=true,
      urlcolor=urlcolor,
      linkcolor=linkcolor,
      citecolor=citecolor,
      }
    % Slightly bigger margins than the latex defaults
    
    \geometry{verbose,tmargin=1in,bmargin=1in,lmargin=1in,rmargin=1in}
    
    

\begin{document}
    
    \maketitle
    
    

    
    \hypertarget{monte-carlo-simulation}{%
\section{Monte Carlo Simulation}\label{monte-carlo-simulation}}

The modern version of the Monte Carlo method was invented in the late
1940s by Stanislaw Ulam, while he was working on nuclear weapons
projects at the Los Alamos National Laboratory.

Monte Carlo methods, or Monte Carlo experiments, are a broad class of
computational algorithms that rely on repeated random sampling to obtain
numerical results. The underlying concept is to use randomness to solve
problems that might be deterministic in principle. Monte Carlo methods
are mainly used in three problem classes: optimization, numerical
integration, and generating draws from a probability distribution.

Monte Carlo simulation is widely used in many fields: Engineering,
Physics, Computational biology, Computer graphics, Applied statistics,
Artificial intelligence for games, Search and rescue and of course
Finance and business.

\hypertarget{the-algorithm}{%
\subsubsection{The Algorithm}\label{the-algorithm}}

In principle, Monte Carlo methods can be used to solve any problem
having a probabilistic interpretation. By the law of large numbers, the
expected value of some random variable can be approximated by taking the
empirical mean of independent samples of the variable.

\[ \mu_n = \cfrac{1}{n}\sum_i^n Y_i \]

Monte Carlo methods vary, but tend to follow a particular pattern:

\begin{itemize}
\tightlist
\item
  define a domain of possible inputs;
\item
  generate inputs randomly from a probability distribution over the
  domain of inputs;
\item
  perform a deterministic computation on the inputs;
\item
  aggregate the results.
\end{itemize}

    \hypertarget{pseudo-random-numbers}{%
\subsection{Pseudo-Random Numbers}\label{pseudo-random-numbers}}

Monte Carlo methods require large amounts of random numbers to generate
the inputs, and it was their use that spurred the development of
pseudorandom number generators.

This is the main reason why every programming language has libraries
that allows to produce huge series of random numbers (with a periodicity
of \(2^{19937}\)). Those numbers are produced by algorithms that take as
input a \emph{seed} which determines univokely the series. This means
that setting the same seed you will produce the same set of numbers
every time (which is great for debugging purpouses).

In \(\tt{python}\) the right module to use is \texttt{random} which has
the following useful functions:

\begin{itemize}
\tightlist
\item
  \texttt{seed} set the seed of the random number generator;
\item
  \texttt{random} returns a random number between 0 and 1 (with uniform
  probability);
\item
  \texttt{randint(min,\ max)} returns an integer random number between
  \texttt{min} and \texttt{max} (with uniform probability);
\item
  \texttt{sample(aList,\ k=n)} samples n elements from the list
  \texttt{aList}.
\end{itemize}

As usual for a more detailed description check \texttt{help(random)}.

    \begin{tcolorbox}[breakable, size=fbox, boxrule=1pt, pad at break*=1mm,colback=cellbackground, colframe=cellborder]
\prompt{In}{incolor}{8}{\boxspacing}
\begin{Verbatim}[commandchars=\\\{\}]
\PY{k+kn}{import} \PY{n+nn}{random} 

\PY{n}{random}\PY{o}{.}\PY{n}{seed}\PY{p}{(}\PY{l+m+mi}{1}\PY{p}{)}
\PY{n+nb}{print} \PY{p}{(}\PY{l+s+s2}{\PYZdq{}}\PY{l+s+s2}{seed is 1}\PY{l+s+s2}{\PYZdq{}}\PY{p}{)}
\PY{n+nb}{print}\PY{p}{(}\PY{n}{random}\PY{o}{.}\PY{n}{random}\PY{p}{(}\PY{p}{)}\PY{p}{)}
\PY{n+nb}{print}\PY{p}{(}\PY{n}{random}\PY{o}{.}\PY{n}{random}\PY{p}{(}\PY{p}{)}\PY{p}{)}

\PY{n}{random}\PY{o}{.}\PY{n}{seed}\PY{p}{(}\PY{l+m+mi}{2}\PY{p}{)}
\PY{n+nb}{print} \PY{p}{(}\PY{l+s+s2}{\PYZdq{}}\PY{l+s+s2}{seed is 2}\PY{l+s+s2}{\PYZdq{}}\PY{p}{)}
\PY{n+nb}{print}\PY{p}{(}\PY{n}{random}\PY{o}{.}\PY{n}{random}\PY{p}{(}\PY{p}{)}\PY{p}{)}
\PY{n+nb}{print}\PY{p}{(}\PY{n}{random}\PY{o}{.}\PY{n}{random}\PY{p}{(}\PY{p}{)}\PY{p}{)}

\PY{n}{random}\PY{o}{.}\PY{n}{seed}\PY{p}{(}\PY{l+m+mi}{1}\PY{p}{)}
\PY{n+nb}{print} \PY{p}{(}\PY{l+s+s2}{\PYZdq{}}\PY{l+s+s2}{seed is 1 again}\PY{l+s+s2}{\PYZdq{}}\PY{p}{)}
\PY{n+nb}{print}\PY{p}{(}\PY{n}{random}\PY{o}{.}\PY{n}{random}\PY{p}{(}\PY{p}{)}\PY{p}{)}
\PY{n+nb}{print}\PY{p}{(}\PY{n}{random}\PY{o}{.}\PY{n}{random}\PY{p}{(}\PY{p}{)}\PY{p}{)}

\PY{n+nb}{print}\PY{p}{(}\PY{n}{random}\PY{o}{.}\PY{n}{randint}\PY{p}{(}\PY{l+m+mi}{1}\PY{p}{,} \PY{l+m+mi}{10}\PY{p}{)}\PY{p}{)}
\PY{n}{aList} \PY{o}{=} \PY{p}{[}\PY{l+s+s1}{\PYZsq{}}\PY{l+s+s1}{a}\PY{l+s+s1}{\PYZsq{}}\PY{p}{,} \PY{l+s+s1}{\PYZsq{}}\PY{l+s+s1}{b}\PY{l+s+s1}{\PYZsq{}}\PY{p}{,} \PY{l+s+s1}{\PYZsq{}}\PY{l+s+s1}{c}\PY{l+s+s1}{\PYZsq{}}\PY{p}{,} \PY{l+s+s1}{\PYZsq{}}\PY{l+s+s1}{d}\PY{l+s+s1}{\PYZsq{}}\PY{p}{,} \PY{l+s+s1}{\PYZsq{}}\PY{l+s+s1}{f}\PY{l+s+s1}{\PYZsq{}}\PY{p}{]}
\PY{n+nb}{print} \PY{p}{(}\PY{n}{random}\PY{o}{.}\PY{n}{sample}\PY{p}{(}\PY{n}{aList}\PY{p}{,} \PY{n}{k}\PY{o}{=}\PY{l+m+mi}{2}\PY{p}{)}\PY{p}{)}
\end{Verbatim}
\end{tcolorbox}

    \begin{Verbatim}[commandchars=\\\{\}]
seed is 1
0.13436424411240122
0.8474337369372327
seed is 2
0.9560342718892494
0.9478274870593494
seed is 1 again
0.13436424411240122
0.8474337369372327
2
['c', 'a']
    \end{Verbatim}

    Below an example of uniform distribution.

    \begin{tcolorbox}[breakable, size=fbox, boxrule=1pt, pad at break*=1mm,colback=cellbackground, colframe=cellborder]
\prompt{In}{incolor}{14}{\boxspacing}
\begin{Verbatim}[commandchars=\\\{\}]
\PY{n}{numbers} \PY{o}{=} \PY{p}{[}\PY{p}{]}
\PY{k}{for} \PY{n}{\PYZus{}} \PY{o+ow}{in} \PY{n+nb}{range}\PY{p}{(}\PY{l+m+mi}{10000}\PY{p}{)}\PY{p}{:}
  \PY{n}{numbers}\PY{o}{.}\PY{n}{append}\PY{p}{(}\PY{n}{random}\PY{o}{.}\PY{n}{randint}\PY{p}{(}\PY{l+m+mi}{0}\PY{p}{,} \PY{l+m+mi}{5}\PY{p}{)}\PY{p}{)}

\PY{k+kn}{from} \PY{n+nn}{matplotlib} \PY{k}{import} \PY{n}{pyplot} \PY{k}{as} \PY{n}{plt}
\PY{n}{plt}\PY{o}{.}\PY{n}{hist}\PY{p}{(}\PY{n}{numbers}\PY{p}{,} \PY{l+m+mi}{6}\PY{p}{,} \PY{n+nb}{range}\PY{o}{=}\PY{p}{[}\PY{o}{\PYZhy{}}\PY{l+m+mf}{0.5}\PY{p}{,} \PY{l+m+mf}{5.5}\PY{p}{]}\PY{p}{)}
\PY{n}{plt}\PY{o}{.}\PY{n}{title}\PY{p}{(}\PY{l+s+s2}{\PYZdq{}}\PY{l+s+s2}{Uniform distribution from randint}\PY{l+s+s2}{\PYZdq{}}\PY{p}{)}
\PY{n}{plt}\PY{o}{.}\PY{n}{show}\PY{p}{(}\PY{p}{)}
\end{Verbatim}
\end{tcolorbox}

    \begin{center}
    \adjustimage{max size={0.9\linewidth}{0.9\paperheight}}{lesson4_files/lesson4_4_0.png}
    \end{center}
    { \hspace*{\fill} \\}
    
    Other modules provide random number generators. Below an example with
\texttt{numpy.random} which allows among others to throw random numbers
according to a standard normal distribution (\(\mathcal{N}(0, 1)\)).

    \begin{tcolorbox}[breakable, size=fbox, boxrule=1pt, pad at break*=1mm,colback=cellbackground, colframe=cellborder]
\prompt{In}{incolor}{11}{\boxspacing}
\begin{Verbatim}[commandchars=\\\{\}]
\PY{k+kn}{from} \PY{n+nn}{numpy}\PY{n+nn}{.}\PY{n+nn}{random} \PY{k}{import} \PY{n}{normal}
\PY{k+kn}{from} \PY{n+nn}{matplotlib} \PY{k}{import} \PY{n}{pyplot} \PY{k}{as} \PY{n}{plt}

\PY{n}{gauss} \PY{o}{=} \PY{p}{[}\PY{p}{]}
\PY{k}{for} \PY{n}{\PYZus{}} \PY{o+ow}{in} \PY{n+nb}{range}\PY{p}{(}\PY{l+m+mi}{50000}\PY{p}{)}\PY{p}{:}
  \PY{n}{gauss}\PY{o}{.}\PY{n}{append}\PY{p}{(}\PY{n}{normal}\PY{p}{(}\PY{p}{)}\PY{p}{)}
  
\PY{n}{plt}\PY{o}{.}\PY{n}{hist}\PY{p}{(}\PY{n}{gauss}\PY{p}{,} \PY{l+m+mi}{100}\PY{p}{,} \PY{n+nb}{range}\PY{o}{=}\PY{p}{[}\PY{o}{\PYZhy{}}\PY{l+m+mi}{4}\PY{p}{,} \PY{l+m+mi}{4}\PY{p}{]}\PY{p}{)}
\PY{n}{plt}\PY{o}{.}\PY{n}{title}\PY{p}{(}\PY{l+s+s2}{\PYZdq{}}\PY{l+s+s2}{Example of gaussian distribution from numpy}\PY{l+s+s2}{\PYZdq{}}\PY{p}{)}
\PY{n}{plt}\PY{o}{.}\PY{n}{show}\PY{p}{(}\PY{p}{)}
\end{Verbatim}
\end{tcolorbox}

    \begin{center}
    \adjustimage{max size={0.9\linewidth}{0.9\paperheight}}{lesson4_files/lesson4_6_0.png}
    \end{center}
    { \hspace*{\fill} \\}
    
    \hypertarget{example-of-monte-carlo-simulation}{%
\paragraph{Example of Monte Carlo
Simulation}\label{example-of-monte-carlo-simulation}}

Let's check the probability to get two kings drawing randomly two cards
from a deck. Using a frequentist approach, we can calculate the
probability of an event as the ratio of the number of favorable outcomes
of an experiment (number of successes) and the number of all possible
outcomes so for our example:

\[P_\textrm{two kings} = \frac{4}{40} \cdot \frac{3}{39} = \frac{1}{130} \approx 0.0077\]

Let's now try with a Monte Carlo simulation.

    \begin{tcolorbox}[breakable, size=fbox, boxrule=1pt, pad at break*=1mm,colback=cellbackground, colframe=cellborder]
\prompt{In}{incolor}{10}{\boxspacing}
\begin{Verbatim}[commandchars=\\\{\}]
\PY{k+kn}{from} \PY{n+nn}{random} \PY{k}{import} \PY{n}{sample}\PY{p}{,} \PY{n}{choices}\PY{p}{,} \PY{n}{seed}

\PY{n}{seed}\PY{p}{(}\PY{l+m+mi}{1}\PY{p}{)}

\PY{n}{deck} \PY{o}{=} \PY{p}{[}\PY{l+s+s2}{\PYZdq{}}\PY{l+s+s2}{A}\PY{l+s+s2}{\PYZdq{}}\PY{p}{,} \PY{l+s+s2}{\PYZdq{}}\PY{l+s+s2}{2}\PY{l+s+s2}{\PYZdq{}}\PY{p}{,} \PY{l+s+s2}{\PYZdq{}}\PY{l+s+s2}{3}\PY{l+s+s2}{\PYZdq{}}\PY{p}{,} \PY{l+s+s2}{\PYZdq{}}\PY{l+s+s2}{4}\PY{l+s+s2}{\PYZdq{}}\PY{p}{,} \PY{l+s+s2}{\PYZdq{}}\PY{l+s+s2}{5}\PY{l+s+s2}{\PYZdq{}}\PY{p}{,} \PY{l+s+s2}{\PYZdq{}}\PY{l+s+s2}{6}\PY{l+s+s2}{\PYZdq{}}\PY{p}{,} \PY{l+s+s2}{\PYZdq{}}\PY{l+s+s2}{7}\PY{l+s+s2}{\PYZdq{}}\PY{p}{,} \PY{l+s+s2}{\PYZdq{}}\PY{l+s+s2}{J}\PY{l+s+s2}{\PYZdq{}}\PY{p}{,} \PY{l+s+s2}{\PYZdq{}}\PY{l+s+s2}{Q}\PY{l+s+s2}{\PYZdq{}}\PY{p}{,} \PY{l+s+s2}{\PYZdq{}}\PY{l+s+s2}{K}\PY{l+s+s2}{\PYZdq{}}\PY{p}{]} \PY{o}{*} \PY{l+m+mi}{4}

\PY{n}{trials} \PY{o}{=} \PY{l+m+mi}{1000000}
\PY{n}{success} \PY{o}{=} \PY{l+m+mi}{0}

\PY{k}{for} \PY{n}{i} \PY{o+ow}{in} \PY{n+nb}{range}\PY{p}{(}\PY{n}{trials}\PY{p}{)}\PY{p}{:}
  \PY{n}{cards} \PY{o}{=} \PY{n}{sample}\PY{p}{(}\PY{n}{deck}\PY{p}{,} \PY{n}{k}\PY{o}{=}\PY{l+m+mi}{2}\PY{p}{)}
  \PY{k}{if} \PY{n}{i} \PY{o}{\PYZlt{}} \PY{l+m+mi}{10}\PY{p}{:}
    \PY{n+nb}{print} \PY{p}{(}\PY{n}{cards}\PY{p}{)}
  \PY{k}{if} \PY{n}{cards} \PY{o}{==} \PY{p}{[}\PY{l+s+s2}{\PYZdq{}}\PY{l+s+s2}{K}\PY{l+s+s2}{\PYZdq{}}\PY{p}{,} \PY{l+s+s2}{\PYZdq{}}\PY{l+s+s2}{K}\PY{l+s+s2}{\PYZdq{}}\PY{p}{]}\PY{p}{:}
    \PY{n}{success} \PY{o}{+}\PY{o}{=} \PY{l+m+mi}{1}
    
\PY{n+nb}{print} \PY{p}{(}\PY{l+s+s2}{\PYZdq{}}\PY{l+s+s2}{The probability to draw two kings is }\PY{l+s+si}{\PYZob{}:.4f\PYZcb{}}\PY{l+s+s2}{\PYZdq{}}\PY{o}{.}\PY{n}{format}\PY{p}{(}\PY{n}{success}\PY{o}{/}\PY{n}{trials}\PY{p}{)}\PY{p}{)}
\end{Verbatim}
\end{tcolorbox}

    \begin{Verbatim}[commandchars=\\\{\}]
['Q', '7']
['5', '7']
['J', '2']
['Q', 'A']
['5', '4']
['7', '2']
['2', '5']
['J', 'Q']
['A', 'Q']
['J', '5']
The probability to draw two kings is 0.0077
    \end{Verbatim}

    \hypertarget{another-mc-example}{%
\paragraph{Another MC Example}\label{another-mc-example}}

Two dice are rolled, find the probability that the sum is:

\begin{itemize}
\tightlist
\item
  equal to 1;
\item
  equal to 4;
\item
  less than 13.
\end{itemize}

Before looking at the python implementation let's compute what we should
get: the possible combinations of the outcomes of two dice are 36 (to
realize it you can simply write them down). It is not possible to get 1
since thee dice have no face with 0 so the first probability should come
out 0. The sum of the two dice is always less than 13 (the maximum is
12\ldots{}) so the answer to point 3 is 1. We can get a sum of 4 in 3
cases (1-3, 3-1 or 2-2) so the probability will result 3/36 = 1/12 =
0.0833

    \begin{tcolorbox}[breakable, size=fbox, boxrule=1pt, pad at break*=1mm,colback=cellbackground, colframe=cellborder]
\prompt{In}{incolor}{2}{\boxspacing}
\begin{Verbatim}[commandchars=\\\{\}]
\PY{k+kn}{import} \PY{n+nn}{random}

\PY{n}{random}\PY{o}{.}\PY{n}{seed}\PY{p}{(}\PY{l+m+mi}{1}\PY{p}{)}
\PY{n}{successes} \PY{o}{=} \PY{p}{\PYZob{}}\PY{l+s+s2}{\PYZdq{}}\PY{l+s+s2}{=0}\PY{l+s+s2}{\PYZdq{}}\PY{p}{:}\PY{l+m+mf}{0.0}\PY{p}{,} \PY{l+s+s2}{\PYZdq{}}\PY{l+s+s2}{=4}\PY{l+s+s2}{\PYZdq{}}\PY{p}{:}\PY{l+m+mf}{0.0}\PY{p}{,} \PY{l+s+s2}{\PYZdq{}}\PY{l+s+s2}{\PYZlt{}13}\PY{l+s+s2}{\PYZdq{}}\PY{p}{:}\PY{l+m+mf}{0.0}\PY{p}{\PYZcb{}}
\PY{n}{trials} \PY{o}{=} \PY{l+m+mi}{100000}

\PY{k}{for} \PY{n}{\PYZus{}} \PY{o+ow}{in} \PY{n+nb}{range}\PY{p}{(}\PY{n}{trials}\PY{p}{)}\PY{p}{:}
    \PY{n}{d1} \PY{o}{=} \PY{n}{random}\PY{o}{.}\PY{n}{randint}\PY{p}{(}\PY{l+m+mi}{1}\PY{p}{,} \PY{l+m+mi}{6}\PY{p}{)}
    \PY{n}{d2} \PY{o}{=} \PY{n}{random}\PY{o}{.}\PY{n}{randint}\PY{p}{(}\PY{l+m+mi}{1}\PY{p}{,} \PY{l+m+mi}{6}\PY{p}{)}
    \PY{k}{if} \PY{p}{(}\PY{n}{d1} \PY{o}{+} \PY{n}{d2}\PY{p}{)} \PY{o}{==} \PY{l+m+mi}{0}\PY{p}{:}
        \PY{n}{successes}\PY{p}{[}\PY{l+s+s2}{\PYZdq{}}\PY{l+s+s2}{=0}\PY{l+s+s2}{\PYZdq{}}\PY{p}{]} \PY{o}{+}\PY{o}{=} \PY{l+m+mf}{1.0}
    \PY{k}{if} \PY{p}{(}\PY{n}{d1} \PY{o}{+} \PY{n}{d2}\PY{p}{)} \PY{o}{==} \PY{l+m+mi}{4}\PY{p}{:}
        \PY{n}{successes}\PY{p}{[}\PY{l+s+s2}{\PYZdq{}}\PY{l+s+s2}{=4}\PY{l+s+s2}{\PYZdq{}}\PY{p}{]} \PY{o}{+}\PY{o}{=} \PY{l+m+mf}{1.0}
    \PY{k}{if} \PY{p}{(}\PY{n}{d1} \PY{o}{+} \PY{n}{d2}\PY{p}{)} \PY{o}{\PYZlt{}} \PY{l+m+mi}{13}\PY{p}{:}
        \PY{n}{successes}\PY{p}{[}\PY{l+s+s2}{\PYZdq{}}\PY{l+s+s2}{\PYZlt{}13}\PY{l+s+s2}{\PYZdq{}}\PY{p}{]} \PY{o}{+}\PY{o}{=} \PY{l+m+mf}{1.0}

\PY{k}{for} \PY{n}{k}\PY{p}{,}\PY{n}{v} \PY{o+ow}{in} \PY{n}{successes}\PY{o}{.}\PY{n}{items}\PY{p}{(}\PY{p}{)}\PY{p}{:}
    \PY{n+nb}{print} \PY{p}{(}\PY{l+s+s2}{\PYZdq{}}\PY{l+s+s2}{P(}\PY{l+s+si}{\PYZob{}\PYZcb{}}\PY{l+s+s2}{): }\PY{l+s+si}{\PYZob{}:.3f\PYZcb{}}\PY{l+s+s2}{\PYZdq{}}\PY{o}{.}\PY{n}{format}\PY{p}{(}\PY{n}{k}\PY{p}{,} \PY{n}{v}\PY{o}{/}\PY{n}{trials}\PY{p}{)}\PY{p}{)}
\end{Verbatim}
\end{tcolorbox}

    \begin{Verbatim}[commandchars=\\\{\}]
P(=0): 0.000
P(=4): 0.084
P(<13): 1.000
    \end{Verbatim}

    \hypertarget{accuracy-of-monte-carlo-simulations}{%
\subsubsection{Accuracy of Monte Carlo
SImulations}\label{accuracy-of-monte-carlo-simulations}}

Since we rely on a frequentist approach naively we can say that the
lower is the probability we need to estimate the higher has to be the
number of simulated trials. This is because to get a reasonable number
of ``success'' so that the uncertainty in the probability is small, we
have to try many times. This is apparent playing with the number of
trials in the above simulations. Monte Carlo Simulation is not always
the best approach to follow !

To quantify observe that from the law of large numbers is:

\$\mathbb{E}(\mu\_n) = \cfrac{1}{n}\sum\_i\^{}n \mathbb{Y_i} = \mu \$\$

The variance on our best estimate of \(\mu\) is:

\[ \mathbb{E}((\mu_n - \mu )^2) = \cfrac{\sigma^2}{n} \] where
\(\sigma^2 = \mathrm{Var}(Y)\).

While it is obvious that the estimate should get worse with increased
variance and better with increased sample size, this equation gives us
the exact rate of exchange. The root mean squared error (RMSE) of
\(\mu_n = \sigma/\sqrt{n}\):

\begin{itemize}
\tightlist
\item
  to get one more decimal digit of accuracy is like asking for an RMSE
  one tenth as large, and that requires a 100-fold increase in
  computation.
\item
  to get three more digits of accuracy requires one million times as
  much computation.
\end{itemize}

From that it is clear that simple Monte Carlo computation is poorly
suited for problems that must be answered with high precision.

On the other hand if raising \(n\) from \(n_1\) to \(n_2\) only makes
our accuracy a little better, then reducing \(n\) from \(n_2\) to
\(n_1\) must only make our accuracy a little worse. We might well decide
that software that runs slower, reducing \(n\) for a given time budget,
is worthwhile if it provides some other benefit, such as a more rapid
programming.

    \hypertarget{derivation-of-log-normal-stochastic-differential-equation}{%
\paragraph{Derivation of log-normal Stochastic Differential
Equation}\label{derivation-of-log-normal-stochastic-differential-equation}}

Stock prices deviate from a steady state as a result of the random
fluctuations given by the trades. Considering a stock with a price
\(S_t\) and an expected rate of return \(\mu\), then the relative change
in its price during a period \(dt\) can be decomposed in two parts:

\begin{itemize}
\tightlist
\item
  a deterministic part that is the expected return from the stock hold
  during the time period \(dt\) (\(\mu S_tdt\))
\item
  a stochastic part which reflects the random changes of the market
  (e.g.~as a response to external effects such as unexpected news). A
  reasonable assumption is to take this contribution proportional to the
  stock (\(\sigma S_tdB_t\) where \(dB_t\) is a random walk process
  equal to \(\mathcal{N}(0,1)\sqrt{t}\)).
\end{itemize}

    The resulting differential equation is:

\[dS_t = \mu S_tdt + \sigma S_tdB_t\] or
\[\frac{dS_t}{S_t} = d\textrm{log}(S_t) = \mu dt + \sigma dB_t\]

    The solution of this equation can be derived by applying the
It\(\hat{o}\)'s formula which states that for any given function
\(G(S, t)\) where \(S\) satisfies the following stochastic differential
equation \(dS=a\cdot dt +b\cdot dB_t\) it holds:

\[dG=\big(a\frac{\partial G}{\partial S} + \frac{\partial G}{\partial t} + \frac{1}{2}b^2\frac{\partial^2 G}{\partial S^2} \big)dt + b \frac{\partial G}{\partial S}dB\]

Considering \(G = \textrm{log}(S_t)\) we have:

\[\frac{\partial G}{\partial S} = \frac{1}{S_t}\]

\[\frac{\partial G}{\partial t} = 0\]

\[\frac{\partial^2 G}{\partial S^2} = -\frac{1}{S_t^{2}}\]

    By inserting these into It\(\hat{o}\)'s formula we get:

\[d(\textrm{log} S_t) = \big(\mu S_t \frac{1}{S_t} + \frac{1}{2}\sigma^2 S_t^2 (-\frac{1}{S_t^2})\big)dt + \sigma\epsilon\sqrt{dt}\]

\[d(\textrm{log} S_t) = \textrm{log} (S_t) - \textrm{log} (S_{t-1}) = \textrm{log} \frac{S_t}{S_{t-1}} = \big(\mu - \frac{1}{2}\sigma^2\big)dt + \sigma\epsilon\sqrt{dt}\]

\[S_t = S_{t-1}e^{\big(\mu - \frac{1}{2}\sigma^2\big)dt + \sigma\epsilon\sqrt{dt}}\]

    As can be seen from the following equation:

\[d(\textrm{log} S_t) = \big(\mu - \frac{1}{2}\sigma^2\big)dt + \sigma\epsilon\sqrt{dt}\]

the change in \(\textrm{log} S_t\) has a constant \emph{drift}
\(\mu - \frac{1}{2}\sigma^2\) and a constance variance rate \(\sigma^2\)
(remember that \(\epsilon\) is a normally distributed random variable
(\(\mathcal{N}(0,1)\)). So you have a constant plus a gaussian
distributed variable, therefore \(\textrm{log} S_t\) at some time \(T\)
is normally distributed with:

\[\textrm{log}S_t - \textrm{log}S_0 \approx\mathcal{N}\big[\big(\mu-\frac{\sigma^2}{2}\big)T, \sigma^2 T\big]\]

This equation shows that \(\textrm{log}S_t\) is normally distributed,
but \textbf{a variable whose logarithm is normally distributed is said
to be log-normal}. Hence the model we have just developed implies that
the stock price at time T, given today's price, is lognormally
distributed.

Lognormality is important because we need to ensure that a stock price
will never be negative. Indeed looking at the initial \(dS\) equation we
have that:

\[dS_t = \mu S_tdt + \sigma S_tdB_t\]

which shows that the closer is \(S_t\) to 0 the smaller is the \(dS\)
variation (so it will never go below 0).

    \hypertarget{confidence-interval}{%
\subsubsection{Confidence Interval}\label{confidence-interval}}

From the central limit theorem we know that as \(n\) becomes large

\[ \mu_n = \cfrac{1}{n}\sum_i^n Y_i \]

is distributed according to \(\mathcal{N}(\mu, \sigma^2/n)\). Hence:

\[\mu_n - \mu \approx \mathcal{N}(0, \sigma^2/n) \]

Referring to the previous Figure we can write:

\[ \mathbb{P}\left(\mu_n - \cfrac{1.96\sigma}{\sqrt{n}}\le \mu \le \mu_n + \cfrac{1.96\sigma}{\sqrt{n}}\right) = 0.95\]

This interval is called \textbf{95\% confidence interval} because it
covers 95\% of the area under the Gaussian.

It measures the accuracy of our simulation in the sense that if you
repeat many times the above simulation, the fraction of calculated
confidence intervals that contains the true parameter would tend toward
95\%. Below a program to compute 95\% confidence level of a fake
simulation.

    \begin{tcolorbox}[breakable, size=fbox, boxrule=1pt, pad at break*=1mm,colback=cellbackground, colframe=cellborder]
\prompt{In}{incolor}{37}{\boxspacing}
\begin{Verbatim}[commandchars=\\\{\}]
\PY{k+kn}{import} \PY{n+nn}{numpy} \PY{k}{as} \PY{n+nn}{np}

\PY{n}{samples} \PY{o}{=} \PY{p}{[}\PY{l+m+mf}{1.}\PY{p}{,}\PY{l+m+mf}{2.}\PY{p}{,}\PY{l+m+mf}{3.}\PY{p}{,}\PY{l+m+mf}{4.}\PY{p}{,}\PY{l+m+mf}{4.}\PY{p}{,}\PY{l+m+mf}{4.}\PY{p}{,}\PY{l+m+mf}{5.}\PY{p}{,}\PY{l+m+mf}{5.}\PY{p}{,}\PY{l+m+mf}{5.}\PY{p}{,}\PY{l+m+mf}{5.}\PY{p}{,}\PY{l+m+mf}{4.}\PY{p}{,}\PY{l+m+mf}{4.}\PY{p}{,}\PY{l+m+mf}{4.}\PY{p}{,}\PY{l+m+mf}{6.}\PY{p}{,}\PY{l+m+mf}{7.}\PY{p}{,}\PY{l+m+mf}{8.}\PY{p}{]}

\PY{n}{X} \PY{o}{=} \PY{n}{np}\PY{o}{.}\PY{n}{array}\PY{p}{(}\PY{n}{samples}\PY{p}{)}
\PY{n}{m}\PY{p}{,} \PY{n}{se} \PY{o}{=} \PY{n}{np}\PY{o}{.}\PY{n}{mean}\PY{p}{(}\PY{n}{X}\PY{p}{)}\PY{p}{,} \PY{n}{np}\PY{o}{.}\PY{n}{std}\PY{p}{(}\PY{n}{X}\PY{p}{)}
\PY{n}{h} \PY{o}{=} \PY{l+m+mf}{1.96}\PY{o}{*}\PY{n}{se}\PY{o}{/}\PY{n}{np}\PY{o}{.}\PY{n}{sqrt}\PY{p}{(}\PY{n+nb}{len}\PY{p}{(}\PY{n}{samples}\PY{p}{)}\PY{p}{)}
\PY{n+nb}{print} \PY{p}{(}\PY{l+s+s2}{\PYZdq{}}\PY{l+s+s2}{95}\PY{l+s+si}{\PYZpc{} c}\PY{l+s+s2}{onfidence interval: }\PY{l+s+si}{\PYZob{}\PYZcb{}}\PY{l+s+s2}{ +\PYZhy{} }\PY{l+s+si}{\PYZob{}\PYZcb{}}\PY{l+s+s2}{\PYZdq{}}\PY{o}{.}\PY{n}{format}\PY{p}{(}\PY{n}{m}\PY{p}{,} \PY{n}{h}\PY{p}{)}\PY{p}{)}
\end{Verbatim}
\end{tcolorbox}

    \begin{Verbatim}[commandchars=\\\{\}]
95\% confidence interval: 4.4375 +- 0.8119957569932247
    \end{Verbatim}

    A naive understanding of the interest rate markets would lead one to
believe that these different markets (O/N, 1M, 3M, etc) can all be
priced with a single discount curve - indeed in periods of low market
stress, this has been the case.

In reality, the details of the liquidity and counterparty risk involved
in each type of transaction are such that there is a basis between these
markets, and therefore each one has a different discount (or rate) curve
associated with it.

    Given the fact that LIBOR curves do not represent a \emph{pure} interest
rate market but incorporate elements of liquidity and counterparty risk,
it is surprising that, so many years after the financial crisis, they
still maintain so much importance as benchmark rate markets for a wide
range of purposes.

Indeed, though OIS markets are fully liquid enough to extract
information about forward interest rates, there is no functioning
options market from which to extract information about the volatility of
those interest rates. The liquidity in rate volatility continues to be
present only in the LIBOR swaptions markets.

As such, it continues to be important for banks to be able to price and
calibrate market parameters against LIBOR instruments.

    \hypertarget{interest-rate-swaps-and-swaptions}{%
\section{Interest Rate Swaps and
Swaptions}\label{interest-rate-swaps-and-swaptions}}

    \hypertarget{interest-rate-swaps}{%
\subsection{Interest Rate Swaps}\label{interest-rate-swaps}}

Interest rate swaps (IRS) consist of a floating leg and a fixed leg. The
contract parameters are:

\begin{itemize}
\tightlist
\item
  start date \(d_0\)
\item
  notional \(N\)
\item
  fixed rate \(K\)
\item
  floating rate tenor (months)
\item
  maturity (years)
\end{itemize}

The floating leg pays the reference LIBOR fixing at a frequency equal to
the tenor of the floating rate - so for example an IRS on a 3-month
LIBOR will pay a floating coupon every three months, an IRS on 6-month
EURIBOR pays the floating coupon every six months and so on.

The fixed leg pays a predetermined cash flow at annual frequency,
regardless of the tenor of the underlying floating rate. For simplicity
we will only consider swaps with maturities which are multiples of 1
year.

    Before going into the deatils of the valuation of IRSs, we need to
modify the \texttt{generate\_swap\_dates} function in our
\texttt{finmarkets} module to generate the payment dates for both the
fixed and floating legs, as follows:

    \begin{tcolorbox}[breakable, size=fbox, boxrule=1pt, pad at break*=1mm,colback=cellbackground, colframe=cellborder]
\prompt{In}{incolor}{14}{\boxspacing}
\begin{Verbatim}[commandchars=\\\{\}]
\PY{k+kn}{from} \PY{n+nn}{datetime} \PY{k}{import} \PY{n}{date}
\PY{k+kn}{from} \PY{n+nn}{dateutil}\PY{n+nn}{.}\PY{n+nn}{relativedelta} \PY{k}{import} \PY{n}{relativedelta}
    
\PY{k}{def} \PY{n+nf}{generate\PYZus{}swap\PYZus{}dates}\PY{p}{(}\PY{n}{start\PYZus{}date}\PY{p}{,} \PY{n}{n\PYZus{}months}\PY{p}{,} \PY{n}{tenor\PYZus{}months}\PY{o}{=}\PY{l+m+mi}{12}\PY{p}{)}\PY{p}{:}
    \PY{n}{dates} \PY{o}{=} \PY{p}{[}\PY{p}{]}
    \PY{k}{for} \PY{n}{n} \PY{o+ow}{in} \PY{n+nb}{range}\PY{p}{(}\PY{l+m+mi}{0}\PY{p}{,} \PY{n}{n\PYZus{}months}\PY{p}{,} \PY{n}{tenor\PYZus{}months}\PY{p}{)}\PY{p}{:}
        \PY{n}{dates}\PY{o}{.}\PY{n}{append}\PY{p}{(}\PY{n}{start\PYZus{}date} \PY{o}{+} \PY{n}{relativedelta}\PY{p}{(}\PY{n}{months}\PY{o}{=}\PY{n}{n}\PY{p}{)}\PY{p}{)}
    \PY{n}{dates}\PY{o}{.}\PY{n}{append}\PY{p}{(}\PY{n}{start\PYZus{}date} \PY{o}{+} \PY{n}{relativedelta}\PY{p}{(}\PY{n}{months}\PY{o}{=}\PY{n}{n\PYZus{}months}\PY{p}{)}\PY{p}{)}
    \PY{k}{return} \PY{n}{dates}

\PY{n}{generate\PYZus{}swap\PYZus{}dates}\PY{p}{(}\PY{n}{date}\PY{o}{.}\PY{n}{today}\PY{p}{(}\PY{p}{)}\PY{p}{,} \PY{l+m+mi}{16}\PY{p}{,} \PY{l+m+mi}{3}\PY{p}{)}
\end{Verbatim}
\end{tcolorbox}

            \begin{tcolorbox}[breakable, size=fbox, boxrule=.5pt, pad at break*=1mm, opacityfill=0]
\prompt{Out}{outcolor}{14}{\boxspacing}
\begin{Verbatim}[commandchars=\\\{\}]
[datetime.date(2020, 10, 15),
 datetime.date(2021, 1, 15),
 datetime.date(2021, 4, 15),
 datetime.date(2021, 7, 15),
 datetime.date(2021, 10, 15),
 datetime.date(2022, 1, 15),
 datetime.date(2022, 2, 15)]
\end{Verbatim}
\end{tcolorbox}
        
    Using this function and the contract parameters we will be able to
determine a sequence of payment dates for each of the two legs.

    Let \(d_0=d_0^{\mathrm{fixed}},...,d_p^{\mathrm{fixed}}\) be the fixed
leg payment dates and
\(d_0=d_0^{\mathrm{float}},...,d_p^{\mathrm{float}}\) be the floating
leg payment dates, and let's use the following notation:

\begin{itemize}
\tightlist
\item
  \(d\) the pricing date
\item
  \(D(d, d')\) the discount factor observed in date \(d\) for the value
  date \(d'\)
\item
  \(F(d, d', d'')\) the forward rate observed in date \(d\) for the
  period \([d', d'']\). The rate tenor is \(\tau = d'' - d'\).
\end{itemize}

    \hypertarget{irs-valuation}{%
\subsubsection{IRS Valuation}\label{irs-valuation}}

The NPV of the fixed leg is calculated as follows:

\[\mathrm{NPV}_{\mathrm{fixed}}(d; K) = N\cdot K\cdot\sum_{i=1}^{n}D(d, d_{i}^{\mathrm{fixed}})\]

while the NPV of the floating leg is calculated as follows:

\[\mathrm{NPV}_{\mathrm{float}}(d) = N\cdot\sum_{i=1}^{m}F(d, d_{j-1}^{\mathrm{float}}, d_{j}^{\mathrm{float}}) \cdot \frac{d_{j}^{\mathrm{float}}-d_{j-1}^{\mathrm{float}}}{360}
\cdot D(d, d_{i}^{\mathrm{float}})\]

Therefore the NPV of the swap (seen from the point of view of the
counter-party which receives the floating leg) is

\[\mathrm{NPV}(d; K) = \mathrm{NPV}_{\mathrm{float}}(d) - \mathrm{NPV}_{\mathrm{fixed}}(d;K)\]

    For reasons which will become apparent later, it's actually more
convenient to express the NPV of an IRS as a function of the fair value
fixed rate \(S\) of the IRS, also known as the \textbf{swap rate}. \(S\)
is the value of K which makes \(\mathrm{NPV}(d)=0\).

On the basis of the previous expressions, we can easiy calculate \(S\)
as:

\[\mathrm{NPV}_{\mathrm{fixed}}(d;S) = \mathrm{NPV}_{\mathrm{float}}(d)\]
\[N\cdot S\cdot\sum_{i=1}^{n}D(d, d_{i}^{\mathrm{fixed}}) = N\cdot\sum_{i=1}^{m}F(d, d_{j-1}^{\mathrm{float}}, d_{j}^{\mathrm{float}}) \cdot \frac{d_{j}^{\mathrm{float}}-d_{j-1}^{\mathrm{float}}}{360} \cdot D(d, d_{i}^{\mathrm{float}})\]
\[S=\frac{\sum_{i=1}^{m}F(d, d_{j-1}^{\mathrm{float}}, d_{j}^{\mathrm{float}}) \cdot \frac{d_{j}^{\mathrm{float}}-d_{j-1}^{\mathrm{float}}}{360}
\cdot D(d, d_{i}^{\mathrm{float}})}{\sum_{i=1}^{n}D(d, d_i^{\mathrm{fixed}})} \]

    Once we have calculated \(S\), we can express the \(\mathrm{NPV}\) of an
IRS as follows:

\[\begin{align}&\mathrm{NPV}(d; K) = \mathrm{NPV}_{\mathrm{float}}(d) - \mathrm{NPV}_{\mathrm{fixed}}(d; K) = & \\ \\ &= \underbrace{\mathrm{NPV}_{\mathrm{float}}(d) - \mathrm{NPV}_{\mathrm{fixed}}(d; S)}_{\mathrm{=\;0}} + \mathrm{NPV}_{\mathrm{fixed}}(d;S) - \mathrm{NPV}_{\mathrm{fixed}}(d;K) & \\ & = N\cdot(S-K)\cdot\underbrace{\sum_{i=1}^{n}D(d, d_{i}^{\mathrm{fixed}})}_{\mathrm{'annuity'}}\end{align}\]

    For convenience the relevant inputs that will be used later (discount
and libor curve definitions) have been saved in the files
\(\href{https://repl.it/@MatteoSani/support6}{\texttt{discount_curve.xlsx}}\)
and
\(\href{https://repl.it/@MatteoSani/support6}{\texttt{libor.xlsx}}\).

    \begin{tcolorbox}[breakable, size=fbox, boxrule=1pt, pad at break*=1mm,colback=cellbackground, colframe=cellborder]
\prompt{In}{incolor}{15}{\boxspacing}
\begin{Verbatim}[commandchars=\\\{\}]
\PY{k+kn}{import} \PY{n+nn}{pandas} \PY{k}{as} \PY{n+nn}{pd}
\PY{k+kn}{from} \PY{n+nn}{datetime} \PY{k}{import} \PY{n}{date}
\PY{k+kn}{from} \PY{n+nn}{finmarkets} \PY{k}{import} \PY{n}{DiscountCurve}\PY{p}{,} \PY{n}{ForwardRateCurve}

\PY{n}{pricing\PYZus{}date} \PY{o}{=} \PY{n}{date}\PY{p}{(}\PY{l+m+mi}{2019}\PY{p}{,} \PY{l+m+mi}{11}\PY{p}{,} \PY{l+m+mi}{23}\PY{p}{)}
\PY{n}{start\PYZus{}date} \PY{o}{=} \PY{n}{date}\PY{p}{(}\PY{l+m+mi}{2021}\PY{p}{,} \PY{l+m+mi}{11}\PY{p}{,} \PY{l+m+mi}{23}\PY{p}{)}
\PY{n}{exercise\PYZus{}date} \PY{o}{=} \PY{n}{date}\PY{p}{(}\PY{l+m+mi}{2020}\PY{p}{,} \PY{l+m+mi}{11}\PY{p}{,} \PY{l+m+mi}{23}\PY{p}{)}

\PY{n}{discount\PYZus{}data} \PY{o}{=} \PY{n}{pd}\PY{o}{.}\PY{n}{read\PYZus{}excel}\PY{p}{(}\PY{l+s+s1}{\PYZsq{}}\PY{l+s+s1}{discount\PYZus{}curve.xlsx}\PY{l+s+s1}{\PYZsq{}}\PY{p}{)}
\PY{n}{libor\PYZus{}data} \PY{o}{=} \PY{n}{pd}\PY{o}{.}\PY{n}{read\PYZus{}excel}\PY{p}{(}\PY{l+s+s1}{\PYZsq{}}\PY{l+s+s1}{libor.xlsx}\PY{l+s+s1}{\PYZsq{}}\PY{p}{)}

\PY{n}{dc} \PY{o}{=} \PY{n}{DiscountCurve}\PY{p}{(}\PY{n}{pricing\PYZus{}date}\PY{p}{,} 
                   \PY{n}{discount\PYZus{}data}\PY{p}{[}\PY{l+s+s1}{\PYZsq{}}\PY{l+s+s1}{pillar}\PY{l+s+s1}{\PYZsq{}}\PY{p}{]}\PY{o}{.}\PY{n}{dt}\PY{o}{.}\PY{n}{date}\PY{o}{.}\PY{n}{tolist}\PY{p}{(}\PY{p}{)}\PY{p}{,}
                   \PY{n}{discount\PYZus{}data}\PY{p}{[}\PY{l+s+s1}{\PYZsq{}}\PY{l+s+s1}{discount\PYZus{}factor}\PY{l+s+s1}{\PYZsq{}}\PY{p}{]}\PY{o}{.}\PY{n}{tolist}\PY{p}{(}\PY{p}{)}\PY{p}{)}

\PY{n}{fr} \PY{o}{=} \PY{n}{ForwardRateCurve}\PY{p}{(}\PY{n}{libor\PYZus{}data}\PY{p}{[}\PY{l+s+s1}{\PYZsq{}}\PY{l+s+s1}{date}\PY{l+s+s1}{\PYZsq{}}\PY{p}{]}\PY{o}{.}\PY{n}{dt}\PY{o}{.}\PY{n}{date}\PY{o}{.}\PY{n}{tolist}\PY{p}{(}\PY{p}{)}\PY{p}{,}
                      \PY{n}{libor\PYZus{}data}\PY{p}{[}\PY{l+s+s1}{\PYZsq{}}\PY{l+s+s1}{rate}\PY{l+s+s1}{\PYZsq{}}\PY{p}{]}\PY{o}{.}\PY{n}{tolist}\PY{p}{(}\PY{p}{)}\PY{p}{)}

\PY{n+nb}{print}\PY{p}{(}\PY{n}{dc}\PY{o}{.}\PY{n}{df}\PY{p}{(}\PY{n}{date}\PY{p}{(}\PY{l+m+mi}{2020}\PY{p}{,} \PY{l+m+mi}{1}\PY{p}{,} \PY{l+m+mi}{1}\PY{p}{)}\PY{p}{)}\PY{p}{)}
\PY{n+nb}{print} \PY{p}{(}\PY{n}{fr}\PY{o}{.}\PY{n}{forward\PYZus{}rate}\PY{p}{(}\PY{n}{date}\PY{p}{(}\PY{l+m+mi}{2020}\PY{p}{,} \PY{l+m+mi}{1}\PY{p}{,} \PY{l+m+mi}{1}\PY{p}{)}\PY{p}{)}\PY{p}{)}
\end{Verbatim}
\end{tcolorbox}

    \begin{Verbatim}[commandchars=\\\{\}]
1.0003778376026249
0.01000266393442623
    \end{Verbatim}

    Now we can implement an \texttt{InterestRateSwap} class to valuate IRS
contracts.

    \begin{tcolorbox}[breakable, size=fbox, boxrule=1pt, pad at break*=1mm,colback=cellbackground, colframe=cellborder]
\prompt{In}{incolor}{16}{\boxspacing}
\begin{Verbatim}[commandchars=\\\{\}]
\PY{k}{class} \PY{n+nc}{InterestRateSwap}\PY{p}{:}
    
    \PY{k}{def} \PY{n+nf}{\PYZus{}\PYZus{}init\PYZus{}\PYZus{}}\PY{p}{(}\PY{n+nb+bp}{self}\PY{p}{,} \PY{n}{start\PYZus{}date}\PY{p}{,} \PY{n}{notional}\PY{p}{,} 
                 \PY{n}{fixed\PYZus{}rate}\PY{p}{,} \PY{n}{tenor\PYZus{}months}\PY{p}{,} 
                 \PY{n}{maturity\PYZus{}years}\PY{p}{)}\PY{p}{:}
        \PY{n+nb+bp}{self}\PY{o}{.}\PY{n}{notional} \PY{o}{=} \PY{n}{notional}
        \PY{n+nb+bp}{self}\PY{o}{.}\PY{n}{fixed\PYZus{}rate} \PY{o}{=} \PY{n}{fixed\PYZus{}rate}
        \PY{n+nb+bp}{self}\PY{o}{.}\PY{n}{fixed\PYZus{}leg\PYZus{}dates} \PY{o}{=} \PYZbs{}
            \PY{n}{generate\PYZus{}swap\PYZus{}dates}\PY{p}{(}\PY{n}{start\PYZus{}date}\PY{p}{,} \PY{l+m+mi}{12} \PY{o}{*} \PY{n}{maturity\PYZus{}years}\PY{p}{)}
        \PY{n+nb+bp}{self}\PY{o}{.}\PY{n}{floating\PYZus{}leg\PYZus{}dates} \PY{o}{=} \PYZbs{}
            \PY{n}{generate\PYZus{}swap\PYZus{}dates}\PY{p}{(}\PY{n}{start\PYZus{}date}\PY{p}{,} \PY{l+m+mi}{12} \PY{o}{*} \PY{n}{maturity\PYZus{}years}\PY{p}{,}
                                \PY{n}{tenor\PYZus{}months}\PY{p}{)}
        
    \PY{k}{def} \PY{n+nf}{annuity}\PY{p}{(}\PY{n+nb+bp}{self}\PY{p}{,} \PY{n}{discount\PYZus{}curve}\PY{p}{)}\PY{p}{:}
        \PY{n}{a} \PY{o}{=} \PY{l+m+mi}{0}
        \PY{k}{for} \PY{n}{i} \PY{o+ow}{in} \PY{n+nb}{range}\PY{p}{(}\PY{l+m+mi}{1}\PY{p}{,} \PY{n+nb}{len}\PY{p}{(}\PY{n+nb+bp}{self}\PY{o}{.}\PY{n}{fixed\PYZus{}leg\PYZus{}dates}\PY{p}{)}\PY{p}{)}\PY{p}{:}
            \PY{n}{a} \PY{o}{+}\PY{o}{=} \PY{n}{discount\PYZus{}curve}\PY{o}{.}\PY{n}{df}\PY{p}{(}\PY{n+nb+bp}{self}\PY{o}{.}\PY{n}{fixed\PYZus{}leg\PYZus{}dates}\PY{p}{[}\PY{n}{i}\PY{p}{]}\PY{p}{)}
        \PY{k}{return} \PY{n}{a}

    \PY{k}{def} \PY{n+nf}{swap\PYZus{}rate}\PY{p}{(}\PY{n+nb+bp}{self}\PY{p}{,} \PY{n}{discount\PYZus{}curve}\PY{p}{,} \PY{n}{libor\PYZus{}curve}\PY{p}{)}\PY{p}{:}
        \PY{n}{s} \PY{o}{=} \PY{l+m+mi}{0}
        \PY{k}{for} \PY{n}{j} \PY{o+ow}{in} \PY{n+nb}{range}\PY{p}{(}\PY{l+m+mi}{1}\PY{p}{,} \PY{n+nb}{len}\PY{p}{(}\PY{n+nb+bp}{self}\PY{o}{.}\PY{n}{floating\PYZus{}leg\PYZus{}dates}\PY{p}{)}\PY{p}{)}\PY{p}{:}
            \PY{n}{F} \PY{o}{=} \PY{n}{libor\PYZus{}curve}\PY{o}{.}\PY{n}{forward\PYZus{}rate}\PY{p}{(}\PY{n+nb+bp}{self}\PY{o}{.}\PY{n}{floating\PYZus{}leg\PYZus{}dates}\PY{p}{[}\PY{n}{j}\PY{o}{\PYZhy{}}\PY{l+m+mi}{1}\PY{p}{]}\PY{p}{)}
            \PY{n}{tau} \PY{o}{=} \PY{p}{(}\PY{n+nb+bp}{self}\PY{o}{.}\PY{n}{floating\PYZus{}leg\PYZus{}dates}\PY{p}{[}\PY{n}{j}\PY{p}{]} \PY{o}{\PYZhy{}} \PYZbs{}
                   \PY{n+nb+bp}{self}\PY{o}{.}\PY{n}{floating\PYZus{}leg\PYZus{}dates}\PY{p}{[}\PY{n}{j}\PY{o}{\PYZhy{}}\PY{l+m+mi}{1}\PY{p}{]}\PY{p}{)}\PY{o}{.}\PY{n}{days} \PY{o}{/} \PY{l+m+mi}{360}
            \PY{n}{P} \PY{o}{=} \PY{n}{discount\PYZus{}curve}\PY{o}{.}\PY{n}{df}\PY{p}{(}\PY{n+nb+bp}{self}\PY{o}{.}\PY{n}{floating\PYZus{}leg\PYZus{}dates}\PY{p}{[}\PY{n}{j}\PY{p}{]}\PY{p}{)}
            \PY{n}{s} \PY{o}{+}\PY{o}{=} \PY{n}{F} \PY{o}{*} \PY{n}{tau} \PY{o}{*} \PY{n}{P}
        \PY{k}{return} \PY{n}{s} \PY{o}{/} \PY{n+nb+bp}{self}\PY{o}{.}\PY{n}{annuity}\PY{p}{(}\PY{n}{discount\PYZus{}curve}\PY{p}{)}
        
    \PY{k}{def} \PY{n+nf}{npv}\PY{p}{(}\PY{n+nb+bp}{self}\PY{p}{,} \PY{n}{discount\PYZus{}curve}\PY{p}{,} \PY{n}{libor\PYZus{}curve}\PY{p}{)}\PY{p}{:}
        \PY{n}{S} \PY{o}{=} \PY{n+nb+bp}{self}\PY{o}{.}\PY{n}{swap\PYZus{}rate}\PY{p}{(}\PY{n}{discount\PYZus{}curve}\PY{p}{,} \PY{n}{libor\PYZus{}curve}\PY{p}{)}
        \PY{n}{A} \PY{o}{=} \PY{n+nb+bp}{self}\PY{o}{.}\PY{n}{annuity}\PY{p}{(}\PY{n}{discount\PYZus{}curve}\PY{p}{)}
        \PY{k}{return} \PY{n+nb+bp}{self}\PY{o}{.}\PY{n}{notional} \PY{o}{*} \PY{p}{(}\PY{n}{S} \PY{o}{\PYZhy{}} \PY{n+nb+bp}{self}\PY{o}{.}\PY{n}{fixed\PYZus{}rate}\PY{p}{)} \PY{o}{*} \PY{n}{A}
\end{Verbatim}
\end{tcolorbox}

    Let's test our class instantiating an IRS with 1M notional, fixed rate
of 5\%, 6 month tenor and a maturity of 4 years; discount and libor
curves are the same as before.

    \begin{tcolorbox}[breakable, size=fbox, boxrule=1pt, pad at break*=1mm,colback=cellbackground, colframe=cellborder]
\prompt{In}{incolor}{19}{\boxspacing}
\begin{Verbatim}[commandchars=\\\{\}]
\PY{n}{pricing\PYZus{}date} \PY{o}{=} \PY{n}{date}\PY{p}{(}\PY{l+m+mi}{2019}\PY{p}{,} \PY{l+m+mi}{11}\PY{p}{,} \PY{l+m+mi}{23}\PY{p}{)}
\PY{n}{irs} \PY{o}{=} \PY{n}{InterestRateSwap}\PY{p}{(}\PY{n}{pricing\PYZus{}date}\PY{p}{,} \PY{l+m+mf}{1e6}\PY{p}{,} \PY{l+m+mf}{0.05}\PY{p}{,} \PY{l+m+mi}{6}\PY{p}{,} \PY{l+m+mi}{4}\PY{p}{)}
\PY{n+nb}{print} \PY{p}{(}\PY{l+s+s2}{\PYZdq{}}\PY{l+s+si}{\PYZob{}:.2f\PYZcb{}}\PY{l+s+s2}{ EUR}\PY{l+s+s2}{\PYZdq{}}\PY{o}{.}\PY{n}{format}\PY{p}{(}\PY{n}{irs}\PY{o}{.}\PY{n}{npv}\PY{p}{(}\PY{n}{dc}\PY{p}{,} \PY{n}{fr}\PY{p}{)}\PY{p}{)}\PY{p}{)}
\end{Verbatim}
\end{tcolorbox}

    \begin{Verbatim}[commandchars=\\\{\}]
-160130.58 EUR
    \end{Verbatim}

    Can you guess what could be the swap rate given that the npv is negative
?

(Remember that we are looking at this contracts from the point of view
of the receiver of the floating leg\ldots{})

    \begin{tcolorbox}[breakable, size=fbox, boxrule=1pt, pad at break*=1mm,colback=cellbackground, colframe=cellborder]
\prompt{In}{incolor}{28}{\boxspacing}
\begin{Verbatim}[commandchars=\\\{\}]
\PY{n+nb}{print} \PY{p}{(}\PY{l+s+s2}{\PYZdq{}}\PY{l+s+si}{\PYZob{}\PYZcb{}}\PY{l+s+s2}{\PYZdq{}}\PY{o}{.}\PY{n}{format}\PY{p}{(}\PY{n}{irs}\PY{o}{.}\PY{n}{swap\PYZus{}rate}\PY{p}{(}\PY{n}{dc}\PY{p}{,} \PY{n}{fr}\PY{p}{)}\PY{p}{)}\PY{p}{)}
\end{Verbatim}
\end{tcolorbox}

    \begin{Verbatim}[commandchars=\\\{\}]
0.010254255993254186
    \end{Verbatim}

    To check if the we have computed correctly the swap rate we can
instanciate a new IRS with fixed rate equal to the just calculated swap
rate and print its NPV, it should come very close to 0.

    \begin{tcolorbox}[breakable, size=fbox, boxrule=1pt, pad at break*=1mm,colback=cellbackground, colframe=cellborder]
\prompt{In}{incolor}{30}{\boxspacing}
\begin{Verbatim}[commandchars=\\\{\}]
\PY{n}{irs2} \PY{o}{=} \PY{n}{InterestRateSwap}\PY{p}{(}\PY{n}{pricing\PYZus{}date}\PY{p}{,} \PY{l+m+mf}{1e6}\PY{p}{,} \PY{l+m+mf}{0.01025425}\PY{p}{,} \PY{l+m+mi}{6}\PY{p}{,} \PY{l+m+mi}{4}\PY{p}{)}
\PY{n+nb}{print} \PY{p}{(}\PY{l+s+s2}{\PYZdq{}}\PY{l+s+si}{\PYZob{}:.2f\PYZcb{}}\PY{l+s+s2}{ EUR}\PY{l+s+s2}{\PYZdq{}}\PY{o}{.}\PY{n}{format}\PY{p}{(}\PY{n}{irs2}\PY{o}{.}\PY{n}{npv}\PY{p}{(}\PY{n}{dc}\PY{p}{,} \PY{n}{fr}\PY{p}{)}\PY{p}{)}\PY{p}{)}
\end{Verbatim}
\end{tcolorbox}

    \begin{Verbatim}[commandchars=\\\{\}]
0.02 EUR
    \end{Verbatim}

    \hypertarget{interest-rate-swaptions}{%
\subsection{Interest Rate Swaptions}\label{interest-rate-swaptions}}

Swaptions are the equivalent of European options for the interest rate
markets. They give the option holder the right but not the obligation,
at the exercise date \(d_{ex}\), to enter into an Interest Rate Swap at
a pre-determined fixed rate.

Clearly the option holder will only choose to do this if the NPV of the
underlying swap at \(d_{ex}\) is positive - looking at the expression
for the NPV of the IRS in terms of the swap rate \(S\) therefore, we can
see that the payoff of the swaption is

\[N\cdot \mathrm{max}(0, S(d_{\mathrm{ex}}) - K)\cdot\sum D(d_{\mathrm{ex}}, d_i^{\mathrm{fixed}})\]

The key issue is now to estimate \(S(d_{\mathrm{ex}})\) in order to
evaluate the payoff of a swaption. This will be shown with two
alternative approaches.

    \hypertarget{evaluation-through-black-scholes-formula}{%
\paragraph{Evaluation through Black-Scholes
formula}\label{evaluation-through-black-scholes-formula}}

In this case, to evaluate the NPV of this payoff, we'll use a
generalization of the Black-Scholes-Merton formula applied to swaptions:

\[\mathrm{NPV} = N\cdot A\cdot [S \mathcal{N}(d_+) - K\mathcal{N}(d_-)]\]

where \(\mathcal{N}\) represent a normal distribution

\[d_{\pm} = \frac{\mathrm{log}(\frac{S}{K}) \pm \frac{1}{2}\sigma^{2}T}{\sigma\sqrt{T}}\qquad(\sigma~\textrm{is the volatility of the swap rate})\\\]
\[A =\sum_{i=1}^{p}D(d, d_{i}^{\mathrm{fixed}})\qquad\mathrm{(annuity})\]

As an example let's consider a swaption whose underlying 6M-IRS has a
notional of 1M, fixed rate of 1\%, and a maturity of 4 years. In
addition we assume a volatility associated to the swap rate of about
7\%.

    \begin{tcolorbox}[breakable, size=fbox, boxrule=1pt, pad at break*=1mm,colback=cellbackground, colframe=cellborder]
\prompt{In}{incolor}{35}{\boxspacing}
\begin{Verbatim}[commandchars=\\\{\}]
\PY{k+kn}{from} \PY{n+nn}{math} \PY{k}{import} \PY{n}{log}
\PY{k+kn}{from} \PY{n+nn}{scipy}\PY{n+nn}{.}\PY{n+nn}{stats} \PY{k}{import} \PY{n}{norm} 
\PY{k+kn}{from} \PY{n+nn}{dateutil}\PY{n+nn}{.}\PY{n+nn}{relativedelta} \PY{k}{import} \PY{n}{relativedelta}

\PY{k}{def} \PY{n+nf}{npvSwaptionBS}\PY{p}{(}\PY{n}{irs}\PY{p}{,} \PY{n}{sigma}\PY{p}{,} 
                  \PY{n}{pricing\PYZus{}date}\PY{p}{,}
                  \PY{n}{exercise\PYZus{}date}\PY{p}{,} 
                  \PY{n}{discount\PYZus{}curve}\PY{p}{,} \PY{n}{libor\PYZus{}curve}\PY{p}{)}\PY{p}{:}
    \PY{n}{T} \PY{o}{=} \PY{p}{(}\PY{n}{exercise\PYZus{}date} \PY{o}{\PYZhy{}} \PY{n}{pricing\PYZus{}date}\PY{p}{)}\PY{o}{.}\PY{n}{days} \PY{o}{/} \PY{l+m+mi}{365}
    \PY{n}{A} \PY{o}{=} \PY{n}{irs}\PY{o}{.}\PY{n}{annuity}\PY{p}{(}\PY{n}{discount\PYZus{}curve}\PY{p}{)}
    \PY{n}{S} \PY{o}{=} \PY{n}{irs}\PY{o}{.}\PY{n}{swap\PYZus{}rate}\PY{p}{(}\PY{n}{discount\PYZus{}curve}\PY{p}{,} \PY{n}{libor\PYZus{}curve}\PY{p}{)}
    \PY{n}{K} \PY{o}{=} \PY{n}{irs}\PY{o}{.}\PY{n}{fixed\PYZus{}rate}
    \PY{n}{N} \PY{o}{=} \PY{n}{irs}\PY{o}{.}\PY{n}{notional}
    
    \PY{n}{d\PYZus{}plus} \PY{o}{=} \PY{p}{(}\PY{n}{log}\PY{p}{(}\PY{n}{S}\PY{o}{/}\PY{n}{K}\PY{p}{)} \PY{o}{+} \PY{l+m+mf}{0.5} \PY{o}{*} \PY{n}{sigma}\PY{o}{*}\PY{o}{*}\PY{l+m+mi}{2} \PY{o}{*} \PY{n}{T}\PY{p}{)} \PY{o}{/} \PY{p}{(}\PY{n}{sigma} \PY{o}{*} \PY{n}{T}\PY{o}{*}\PY{o}{*}\PY{l+m+mf}{0.5}\PY{p}{)}
    \PY{n}{d\PYZus{}minus} \PY{o}{=} \PY{p}{(}\PY{n}{log}\PY{p}{(}\PY{n}{S}\PY{o}{/}\PY{n}{K}\PY{p}{)} \PY{o}{\PYZhy{}} \PY{l+m+mf}{0.5} \PY{o}{*} \PY{n}{sigma}\PY{o}{*}\PY{o}{*}\PY{l+m+mi}{2} \PY{o}{*} \PY{n}{T}\PY{p}{)} \PY{o}{/} \PY{p}{(}\PY{n}{sigma} \PY{o}{*} \PY{n}{T}\PY{o}{*}\PY{o}{*}\PY{l+m+mf}{0.5}\PY{p}{)}
    \PY{k}{return} \PY{n}{irs}\PY{o}{.}\PY{n}{notional} \PY{o}{*} \PY{n}{A} \PY{o}{*} \PY{p}{(}\PY{n}{S} \PY{o}{*} \PY{n}{norm}\PY{o}{.}\PY{n}{cdf}\PY{p}{(}\PY{n}{d\PYZus{}plus}\PY{p}{)} \PY{o}{\PYZhy{}} \PY{n}{K} \PY{o}{*} \PY{n}{norm}\PY{o}{.}\PY{n}{cdf}\PY{p}{(}\PY{n}{d\PYZus{}minus}\PY{p}{)}\PY{p}{)}

\PY{n}{sigma} \PY{o}{=} \PY{l+m+mf}{0.07}
\PY{n}{irs} \PY{o}{=} \PY{n}{InterestRateSwap}\PY{p}{(}\PY{n}{pricing\PYZus{}date}\PY{p}{,} \PY{l+m+mf}{1e6}\PY{p}{,} \PY{l+m+mf}{0.01}\PY{p}{,} \PY{l+m+mi}{6}\PY{p}{,} \PY{l+m+mi}{4}\PY{p}{)}
\PY{n}{exercise\PYZus{}date} \PY{o}{=} \PY{n}{start\PYZus{}date} \PY{o}{+} \PY{n}{relativedelta}\PY{p}{(}\PY{n}{years}\PY{o}{=}\PY{l+m+mi}{4}\PY{p}{)}

\PY{n}{npv} \PY{o}{=} \PY{n}{npvSwaptionBS}\PY{p}{(}\PY{n}{irs}\PY{p}{,} \PY{n}{sigma}\PY{p}{,} \PY{n}{pricing\PYZus{}date}\PY{p}{,} 
                    \PY{n}{exercise\PYZus{}date}\PY{p}{,} \PY{n}{dc}\PY{p}{,} \PY{n}{fr}\PY{p}{)}
\PY{n+nb}{print}\PY{p}{(}\PY{l+s+s2}{\PYZdq{}}\PY{l+s+s2}{Swaption NPV with BS: }\PY{l+s+si}{\PYZob{}:.3f\PYZcb{}}\PY{l+s+s2}{ EUR}\PY{l+s+s2}{\PYZdq{}}\PY{o}{.}\PY{n}{format}\PY{p}{(}\PY{n}{npv}\PY{p}{)}\PY{p}{)}
\end{Verbatim}
\end{tcolorbox}

    \begin{Verbatim}[commandchars=\\\{\}]
Swaption NPV with BS: 3330.741 EUR
    \end{Verbatim}

    \hypertarget{evaluation-through-monte-carlo-simulation}{%
\paragraph{Evaluation through Monte-Carlo
Simulation}\label{evaluation-through-monte-carlo-simulation}}

In this second case we start from the current swap rate \(S(d)\)
evaluated at the pricing date \(d\), and assume that it follows a
log-normal stochastic process, i.e.~its distribution at
\(d_{\mathrm{ex}}\) (exercise date) is
\(S(d_{\mathrm{ex}}) = S(d)\mathrm{exp}(-\frac{1}{2}\sigma^{2}T+\sigma\sqrt{T}\epsilon)\)
where \(\epsilon\approx\mathcal{N}(0,1)\). Notice that it is assumed
that the \emph{drift} rate in the evolution of the swap rate is zero.
Given that the discounted payoff is given by:

\[N\cdot \mathrm{max}(0, S(d_{\mathrm{ex}}) - K)\cdot\sum D(d_{\mathrm{ex}}, d_i^{\mathrm{fixed}})\]

to perform the simulation we can:

\begin{itemize}
\tightlist
\item
  sample the normal distribution \(\mathcal{N}(0, 1)\) to calculate a
  large number of scenarios for \(S(d_{\mathrm{ex}})\);
\item
  evaluate the underlying swap's NPV at the expiry date, and
  consequently the swaption's payoff, for each scenario;
\item
  take the average of these values to get the final estimate.
\end{itemize}

    \begin{tcolorbox}[breakable, size=fbox, boxrule=1pt, pad at break*=1mm,colback=cellbackground, colframe=cellborder]
\prompt{In}{incolor}{42}{\boxspacing}
\begin{Verbatim}[commandchars=\\\{\}]
\PY{k+kn}{import} \PY{n+nn}{numpy} \PY{k}{as} \PY{n+nn}{np}
\PY{k+kn}{from} \PY{n+nn}{math} \PY{k}{import} \PY{n}{exp}\PY{p}{,} \PY{n}{sqrt}
\PY{k+kn}{from} \PY{n+nn}{numpy}\PY{n+nn}{.}\PY{n+nn}{random} \PY{k}{import} \PY{n}{normal}\PY{p}{,} \PY{n}{seed}

\PY{c+c1}{\PYZsh{} define the number of Monte Carlo scenarios}
\PY{n}{n\PYZus{}scenarios} \PY{o}{=} \PY{l+m+mi}{50000}
\PY{n}{discounted\PYZus{}payoffs} \PY{o}{=} \PY{p}{[}\PY{p}{]}
\PY{n}{seed}\PY{p}{(}\PY{l+m+mi}{1}\PY{p}{)}

\PY{n}{T} \PY{o}{=} \PY{p}{(}\PY{n}{exercise\PYZus{}date} \PY{o}{\PYZhy{}} \PY{n}{pricing\PYZus{}date}\PY{p}{)}\PY{o}{.}\PY{n}{days} \PY{o}{/} \PY{l+m+mi}{365}
\PY{n}{A} \PY{o}{=} \PY{n}{irs}\PY{o}{.}\PY{n}{annuity}\PY{p}{(}\PY{n}{dc}\PY{p}{)}
\PY{n}{S} \PY{o}{=} \PY{n}{irs}\PY{o}{.}\PY{n}{swap\PYZus{}rate}\PY{p}{(}\PY{n}{dc}\PY{p}{,} \PY{n}{fr}\PY{p}{)}
    
\PY{k}{for} \PY{n}{i\PYZus{}scenario} \PY{o+ow}{in} \PY{n+nb}{range}\PY{p}{(}\PY{n}{n\PYZus{}scenarios}\PY{p}{)}\PY{p}{:}
    \PY{n}{S\PYZus{}simulated} \PY{o}{=} \PY{n}{S} \PY{o}{*} \PY{n}{exp}\PY{p}{(}\PY{o}{\PYZhy{}}\PY{l+m+mf}{0.5} \PY{o}{*} \PY{n}{sigma} \PY{o}{*} \PY{n}{sigma} \PY{o}{*} \PY{n}{T} \PY{o}{+}
                          \PY{n}{sigma} \PY{o}{*} \PY{n}{sqrt}\PY{p}{(}\PY{n}{T}\PY{p}{)} \PY{o}{*} \PY{n}{normal}\PY{p}{(}\PY{p}{)}\PY{p}{)}
    
    \PY{c+c1}{\PYZsh{} calculate the swap NPV in this scenario}
    \PY{n}{swap\PYZus{}npv} \PY{o}{=} \PY{n}{irs}\PY{o}{.}\PY{n}{notional} \PY{o}{*} \PY{p}{(}\PY{n}{S\PYZus{}simulated} \PY{o}{\PYZhy{}} \PY{n}{irs}\PY{o}{.}\PY{n}{fixed\PYZus{}rate}\PY{p}{)} \PY{o}{*} \PY{n}{A}
    
    \PY{c+c1}{\PYZsh{} add the discounted payoff of the swaption, in this scenario, to the list}
    \PY{n}{discounted\PYZus{}payoffs}\PY{o}{.}\PY{n}{append}\PY{p}{(}\PY{n+nb}{max}\PY{p}{(}\PY{l+m+mi}{0}\PY{p}{,} \PY{n}{swap\PYZus{}npv}\PY{p}{)}\PY{p}{)}
    
\PY{c+c1}{\PYZsh{} calculate the NPV of the swaption }
\PY{c+c1}{\PYZsh{} by taking the average of the discounted }
\PY{c+c1}{\PYZsh{} payoffs across all the scenarios}
\PY{n}{npv\PYZus{}mc} \PY{o}{=} \PY{n}{np}\PY{o}{.}\PY{n}{mean}\PY{p}{(}\PY{n}{discounted\PYZus{}payoffs}\PY{p}{)}
    
\PY{c+c1}{\PYZsh{} calculate the MC error estimate as}
\PY{c+c1}{\PYZsh{} 99\PYZpc{} confidence interval}
\PY{n}{npv\PYZus{}error} \PY{o}{=} \PY{l+m+mf}{2.57} \PY{o}{*} \PY{n}{np}\PY{o}{.}\PY{n}{std}\PY{p}{(}\PY{n}{discounted\PYZus{}payoffs}\PY{p}{)}\PY{o}{/}\PY{n}{sqrt}\PY{p}{(}\PY{n}{n\PYZus{}scenarios}\PY{p}{)}

\PY{n+nb}{print}\PY{p}{(}\PY{l+s+s2}{\PYZdq{}}\PY{l+s+s2}{Swaption NPV: }\PY{l+s+si}{\PYZob{}:.2f\PYZcb{}}\PY{l+s+s2}{ EUR (+/‐ }\PY{l+s+si}{\PYZob{}:.2f\PYZcb{}}\PY{l+s+s2}{ EUR with 99}\PY{l+s+si}{\PYZpc{} c}\PY{l+s+s2}{onfidence)}\PY{l+s+s2}{\PYZdq{}}\PYZbs{}
      \PY{o}{.}\PY{n}{format}\PY{p}{(}\PY{n}{npv\PYZus{}mc}\PY{p}{,} \PY{n}{npv\PYZus{}error}\PY{p}{)}\PY{p}{)}
\end{Verbatim}
\end{tcolorbox}

    \begin{Verbatim}[commandchars=\\\{\}]
Swaption NPV: 3351.42 EUR (+/‐ 56.66 EUR with 99\% confidence)
    \end{Verbatim}

    Note that this is not \emph{strictly speaking} the correct way of
calculating the swaption NPV, the reason being that one should calculate
the swap NPV at the expiry date of the swaption, apply the payoff
function max(0, \ldots{}) and \emph{then} discount from the expiry date
to today.

However, it's simpler to calculate it as above and it doesn't make any
difference for the result, since

\[ DF\cdot \mathrm{max}(0, \mathrm{SwapNPVAtExpiry}) = \mathrm{max}(0, DF \cdot\mathrm{SwapNPVAtExpiry}) \]

    The NPV calculated via the Black-Scholes-Merton formula falls within the
confidence interval produced by the Monte Carlo simulation, so we can
assert that the two methods are in agreement:

\begin{itemize}
\tightlist
\item
  Swaption NPV (BS): €3330.74
\item
  Swaption NPV (MC): €3351.42
\end{itemize}

    \begin{tcolorbox}[breakable, size=fbox, boxrule=1pt, pad at break*=1mm,colback=cellbackground, colframe=cellborder]
\prompt{In}{incolor}{ }{\boxspacing}
\begin{Verbatim}[commandchars=\\\{\}]
\PY{k}{class} \PY{n+nc}{InterestRateSwaption}\PY{p}{:}
    
    \PY{k}{def} \PY{n+nf}{\PYZus{}\PYZus{}init\PYZus{}\PYZus{}}\PY{p}{(}\PY{n+nb+bp}{self}\PY{p}{,} \PY{n}{exercise\PYZus{}date}\PY{p}{,} \PY{n}{irs}\PY{p}{)}\PY{p}{:}
        \PY{n+nb+bp}{self}\PY{o}{.}\PY{n}{exercise\PYZus{}date} \PY{o}{=} \PY{n}{exercise\PYZus{}date}
        \PY{n+nb+bp}{self}\PY{o}{.}\PY{n}{irs} \PY{o}{=} \PY{n}{irs}
        
    \PY{k}{def} \PY{n+nf}{npv\PYZus{}bs}\PY{p}{(}\PY{n+nb+bp}{self}\PY{p}{,} \PY{n}{discount\PYZus{}curve}\PY{p}{,} \PY{n}{libor\PYZus{}curve}\PY{p}{,} \PY{n}{sigma}\PY{p}{)}\PY{p}{:}
        
        \PY{n}{A} \PY{o}{=} \PY{n+nb+bp}{self}\PY{o}{.}\PY{n}{irs}\PY{o}{.}\PY{n}{annuity}\PY{p}{(}\PY{n}{discount\PYZus{}curve}\PY{p}{)}
        \PY{n}{S} \PY{o}{=} \PY{n+nb+bp}{self}\PY{o}{.}\PY{n}{irs}\PY{o}{.}\PY{n}{swap\PYZus{}rate}\PY{p}{(}\PY{n}{discount\PYZus{}curve}\PY{p}{,} \PY{n}{libor\PYZus{}curve}\PY{p}{)}

        \PY{n}{T} \PY{o}{=} \PY{p}{(}\PY{n+nb+bp}{self}\PY{o}{.}\PY{n}{exercise\PYZus{}date} \PY{o}{\PYZhy{}} \PY{n}{discount\PYZus{}curve}\PY{o}{.}\PY{n}{today}\PY{p}{)}\PY{o}{.}\PY{n}{days} \PY{o}{/} \PY{l+m+mi}{365}

        \PY{n}{d1} \PY{o}{=} \PY{p}{(}\PY{n}{math}\PY{o}{.}\PY{n}{log}\PY{p}{(}\PY{n}{S}\PY{o}{/}\PY{n+nb+bp}{self}\PY{o}{.}\PY{n}{irs}\PY{o}{.}\PY{n}{fixed\PYZus{}rate}\PY{p}{)} \PY{o}{+} \PY{l+m+mf}{0.5} \PY{o}{*} \PY{n}{sigma}\PY{o}{*}\PY{o}{*}\PY{l+m+mi}{2} \PY{o}{*} \PY{n}{T}\PY{p}{)} \PY{o}{/} \PY{p}{(}\PY{n}{sigma} \PY{o}{*} \PY{n}{T}\PY{o}{*}\PY{o}{*}\PY{l+m+mf}{0.5}\PY{p}{)}
        \PY{n}{d2} \PY{o}{=} \PY{n}{d1} \PY{o}{\PYZhy{}} \PY{p}{(}\PY{n}{sigma} \PY{o}{*} \PY{n}{T}\PY{o}{*}\PY{o}{*}\PY{l+m+mf}{0.5}\PY{p}{)}

        \PY{n}{npv} \PY{o}{=} \PY{n+nb+bp}{self}\PY{o}{.}\PY{n}{irs}\PY{o}{.}\PY{n}{notional} \PY{o}{*} \PY{n}{A} \PY{o}{*} \PY{p}{(}\PY{n}{S} \PY{o}{*} \PY{n}{scipy}\PY{o}{.}\PY{n}{stats}\PY{o}{.}\PY{n}{norm}\PY{o}{.}\PY{n}{cdf}\PY{p}{(}\PY{n}{d1}\PY{p}{)} \PY{o}{\PYZhy{}} 
                                       \PY{n+nb+bp}{self}\PY{o}{.}\PY{n}{irs}\PY{o}{.}\PY{n}{fixed\PYZus{}rate} \PY{o}{*} \PY{n}{scipy}\PY{o}{.}\PY{n}{stats}\PY{o}{.}\PY{n}{norm}\PY{o}{.}\PY{n}{cdf}\PY{p}{(}\PY{n}{d2}\PY{p}{)}\PY{p}{)}
        
        \PY{k}{return} \PY{n}{npv}
    
    \PY{k}{def} \PY{n+nf}{npv\PYZus{}mc}\PY{p}{(}\PY{n+nb+bp}{self}\PY{p}{,} \PY{n}{discount\PYZus{}curve}\PY{p}{,} \PY{n}{libor\PYZus{}curve}\PY{p}{,} \PY{n}{sigma}\PY{p}{,} \PY{n}{n\PYZus{}scenarios}\PY{o}{=}\PY{l+m+mi}{10000}\PY{p}{)}\PY{p}{:}
        
        \PY{n}{A} \PY{o}{=} \PY{n+nb+bp}{self}\PY{o}{.}\PY{n}{irs}\PY{o}{.}\PY{n}{annuity}\PY{p}{(}\PY{n}{discount\PYZus{}curve}\PY{p}{)}
        \PY{n}{S} \PY{o}{=} \PY{n+nb+bp}{self}\PY{o}{.}\PY{n}{irs}\PY{o}{.}\PY{n}{swap\PYZus{}rate}\PY{p}{(}\PY{n}{discount\PYZus{}curve}\PY{p}{,} \PY{n}{libor\PYZus{}curve}\PY{p}{)}

        \PY{n}{T} \PY{o}{=} \PY{p}{(}\PY{n+nb+bp}{self}\PY{o}{.}\PY{n}{exercise\PYZus{}date} \PY{o}{\PYZhy{}} \PY{n}{discount\PYZus{}curve}\PY{o}{.}\PY{n}{today}\PY{p}{)}\PY{o}{.}\PY{n}{days} \PY{o}{/} \PY{l+m+mi}{365}
        \PY{n}{discounted\PYZus{}payoffs} \PY{o}{=} \PY{p}{[}\PY{p}{]}

        \PY{k}{for} \PY{n}{i\PYZus{}scenario} \PY{o+ow}{in} \PY{n+nb}{range}\PY{p}{(}\PY{n}{n\PYZus{}scenarios}\PY{p}{)}\PY{p}{:}
            \PY{n}{S\PYZus{}simulated} \PY{o}{=} \PY{n}{S} \PY{o}{*} \PY{n}{math}\PY{o}{.}\PY{n}{exp}\PY{p}{(}\PY{o}{\PYZhy{}}\PY{l+m+mf}{0.5} \PY{o}{*} \PY{n}{sigma} \PY{o}{*} \PY{n}{sigma} \PY{o}{*} \PY{n}{T} \PY{o}{+}
                                       \PY{n}{sigma} \PY{o}{*} \PY{n}{math}\PY{o}{.}\PY{n}{sqrt}\PY{p}{(}\PY{n}{T}\PY{p}{)} \PY{o}{*} \PY{n}{numpy}\PY{o}{.}\PY{n}{random}\PY{o}{.}\PY{n}{normal}\PY{p}{(}\PY{p}{)}\PY{p}{)}

            \PY{n}{swap\PYZus{}npv} \PY{o}{=} \PY{n+nb+bp}{self}\PY{o}{.}\PY{n}{irs}\PY{o}{.}\PY{n}{notional} \PY{o}{*} \PY{p}{(}\PY{n}{S\PYZus{}simulated} \PY{o}{\PYZhy{}} \PY{n+nb+bp}{self}\PY{o}{.}\PY{n}{irs}\PY{o}{.}\PY{n}{fixed\PYZus{}rate}\PY{p}{)} \PY{o}{*} \PY{n}{A}
            \PY{n}{discounted\PYZus{}payoffs}\PY{o}{.}\PY{n}{append}\PY{p}{(}\PY{n+nb}{max}\PY{p}{(}\PY{l+m+mi}{0}\PY{p}{,} \PY{n}{swap\PYZus{}npv}\PY{p}{)}\PY{p}{)}

        \PY{n}{npv\PYZus{}mc} \PY{o}{=} \PY{n}{numpy}\PY{o}{.}\PY{n}{mean}\PY{p}{(}\PY{n}{discounted\PYZus{}payoffs}\PY{p}{)}
        \PY{n}{npv\PYZus{}error} \PY{o}{=} \PY{l+m+mi}{3} \PY{o}{*} \PY{n}{numpy}\PY{o}{.}\PY{n}{std}\PY{p}{(}\PY{n}{discounted\PYZus{}payoffs}\PY{p}{)} \PY{o}{/} \PY{n}{math}\PY{o}{.}\PY{n}{sqrt}\PY{p}{(}\PY{n}{n\PYZus{}scenarios}\PY{p}{)}
        
        \PY{k}{return} \PY{n}{npv\PYZus{}mc}\PY{p}{,} \PY{n}{npv\PYZus{}error}
\end{Verbatim}
\end{tcolorbox}

    \hypertarget{exercise-6.3}{%
\subsubsection{Exercise 6.3}\label{exercise-6.3}}

Using the function \texttt{normal} of \texttt{numpy.random} simulate the
price of a stock which evolves according to a log-normal stochastic
process with a daily rate of return \(\mu=0.1\) and a volatility
\(\sigma=0.15\) for 30 days.

Also plot the price. Try to play with \(\mu\) and \(\sigma\) to see how
the plot changes.

    \hypertarget{exercise-6.4}{%
\subsubsection{Exercise 6.4}\label{exercise-6.4}}

Suppouse that the Libor Forward rates are those defined here in
\(\href{https://repl.it/@MatteoSani/support6}{\textrm{curve_data.py}}\).
Determine the value of an option to pay a fixed rate of 4\% and receives
LIBOR on a 5 year swap starting in 1 year. Assume the notional is 100
EUR, the exercise date is on October, 30th 2020 and the swap rate
volatility is 15\%.


    % Add a bibliography block to the postdoc
    
    
    
\end{document}

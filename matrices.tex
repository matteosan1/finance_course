\documentclass[11pt]{article}

    \usepackage[breakable]{tcolorbox}
    \usepackage{parskip} % Stop auto-indenting (to mimic markdown behaviour)
    
    \usepackage{iftex}
    \ifPDFTeX
    	\usepackage[T1]{fontenc}
    	\usepackage{mathpazo}
    \else
    	\usepackage{fontspec}
    \fi

    % Basic figure setup, for now with no caption control since it's done
    % automatically by Pandoc (which extracts ![](path) syntax from Markdown).
    \usepackage{graphicx}
    % Maintain compatibility with old templates. Remove in nbconvert 6.0
    \let\Oldincludegraphics\includegraphics
    % Ensure that by default, figures have no caption (until we provide a
    % proper Figure object with a Caption API and a way to capture that
    % in the conversion process - todo).
    \usepackage{caption}
    \DeclareCaptionFormat{nocaption}{}
    \captionsetup{format=nocaption,aboveskip=0pt,belowskip=0pt}

    \usepackage[Export]{adjustbox} % Used to constrain images to a maximum size
    \adjustboxset{max size={0.9\linewidth}{0.9\paperheight}}
    \usepackage{float}
    \floatplacement{figure}{H} % forces figures to be placed at the correct location
    \usepackage{xcolor} % Allow colors to be defined
    \usepackage{enumerate} % Needed for markdown enumerations to work
    \usepackage{geometry} % Used to adjust the document margins
    \usepackage{amsmath} % Equations
    \usepackage{amssymb} % Equations
    \usepackage{textcomp} % defines textquotesingle
    % Hack from http://tex.stackexchange.com/a/47451/13684:
    \AtBeginDocument{%
        \def\PYZsq{\textquotesingle}% Upright quotes in Pygmentized code
    }
    \usepackage{upquote} % Upright quotes for verbatim code
    \usepackage{eurosym} % defines \euro
    \usepackage[mathletters]{ucs} % Extended unicode (utf-8) support
    \usepackage{fancyvrb} % verbatim replacement that allows latex
    \usepackage{grffile} % extends the file name processing of package graphics 
                         % to support a larger range
    \makeatletter % fix for grffile with XeLaTeX
    \def\Gread@@xetex#1{%
      \IfFileExists{"\Gin@base".bb}%
      {\Gread@eps{\Gin@base.bb}}%
      {\Gread@@xetex@aux#1}%
    }
    \makeatother

    % The hyperref package gives us a pdf with properly built
    % internal navigation ('pdf bookmarks' for the table of contents,
    % internal cross-reference links, web links for URLs, etc.)
    \usepackage{hyperref}
    % The default LaTeX title has an obnoxious amount of whitespace. By default,
    % titling removes some of it. It also provides customization options.
    \usepackage{titling}
    \usepackage{longtable} % longtable support required by pandoc >1.10
    \usepackage{booktabs}  % table support for pandoc > 1.12.2
    \usepackage[inline]{enumitem} % IRkernel/repr support (it uses the enumerate* environment)
    \usepackage[normalem]{ulem} % ulem is needed to support strikethroughs (\sout)
                                % normalem makes italics be italics, not underlines
    \usepackage{mathrsfs}
    

    
    % Colors for the hyperref package
    \definecolor{urlcolor}{rgb}{0,.145,.698}
    \definecolor{linkcolor}{rgb}{.71,0.21,0.01}
    \definecolor{citecolor}{rgb}{.12,.54,.11}

    % ANSI colors
    \definecolor{ansi-black}{HTML}{3E424D}
    \definecolor{ansi-black-intense}{HTML}{282C36}
    \definecolor{ansi-red}{HTML}{E75C58}
    \definecolor{ansi-red-intense}{HTML}{B22B31}
    \definecolor{ansi-green}{HTML}{00A250}
    \definecolor{ansi-green-intense}{HTML}{007427}
    \definecolor{ansi-yellow}{HTML}{DDB62B}
    \definecolor{ansi-yellow-intense}{HTML}{B27D12}
    \definecolor{ansi-blue}{HTML}{208FFB}
    \definecolor{ansi-blue-intense}{HTML}{0065CA}
    \definecolor{ansi-magenta}{HTML}{D160C4}
    \definecolor{ansi-magenta-intense}{HTML}{A03196}
    \definecolor{ansi-cyan}{HTML}{60C6C8}
    \definecolor{ansi-cyan-intense}{HTML}{258F8F}
    \definecolor{ansi-white}{HTML}{C5C1B4}
    \definecolor{ansi-white-intense}{HTML}{A1A6B2}
    \definecolor{ansi-default-inverse-fg}{HTML}{FFFFFF}
    \definecolor{ansi-default-inverse-bg}{HTML}{000000}

    % commands and environments needed by pandoc snippets
    % extracted from the output of `pandoc -s`
    \providecommand{\tightlist}{%
      \setlength{\itemsep}{0pt}\setlength{\parskip}{0pt}}
    \DefineVerbatimEnvironment{Highlighting}{Verbatim}{commandchars=\\\{\}}
    % Add ',fontsize=\small' for more characters per line
    \newenvironment{Shaded}{}{}
    \newcommand{\KeywordTok}[1]{\textcolor[rgb]{0.00,0.44,0.13}{\textbf{{#1}}}}
    \newcommand{\DataTypeTok}[1]{\textcolor[rgb]{0.56,0.13,0.00}{{#1}}}
    \newcommand{\DecValTok}[1]{\textcolor[rgb]{0.25,0.63,0.44}{{#1}}}
    \newcommand{\BaseNTok}[1]{\textcolor[rgb]{0.25,0.63,0.44}{{#1}}}
    \newcommand{\FloatTok}[1]{\textcolor[rgb]{0.25,0.63,0.44}{{#1}}}
    \newcommand{\CharTok}[1]{\textcolor[rgb]{0.25,0.44,0.63}{{#1}}}
    \newcommand{\StringTok}[1]{\textcolor[rgb]{0.25,0.44,0.63}{{#1}}}
    \newcommand{\CommentTok}[1]{\textcolor[rgb]{0.38,0.63,0.69}{\textit{{#1}}}}
    \newcommand{\OtherTok}[1]{\textcolor[rgb]{0.00,0.44,0.13}{{#1}}}
    \newcommand{\AlertTok}[1]{\textcolor[rgb]{1.00,0.00,0.00}{\textbf{{#1}}}}
    \newcommand{\FunctionTok}[1]{\textcolor[rgb]{0.02,0.16,0.49}{{#1}}}
    \newcommand{\RegionMarkerTok}[1]{{#1}}
    \newcommand{\ErrorTok}[1]{\textcolor[rgb]{1.00,0.00,0.00}{\textbf{{#1}}}}
    \newcommand{\NormalTok}[1]{{#1}}
    
    % Additional commands for more recent versions of Pandoc
    \newcommand{\ConstantTok}[1]{\textcolor[rgb]{0.53,0.00,0.00}{{#1}}}
    \newcommand{\SpecialCharTok}[1]{\textcolor[rgb]{0.25,0.44,0.63}{{#1}}}
    \newcommand{\VerbatimStringTok}[1]{\textcolor[rgb]{0.25,0.44,0.63}{{#1}}}
    \newcommand{\SpecialStringTok}[1]{\textcolor[rgb]{0.73,0.40,0.53}{{#1}}}
    \newcommand{\ImportTok}[1]{{#1}}
    \newcommand{\DocumentationTok}[1]{\textcolor[rgb]{0.73,0.13,0.13}{\textit{{#1}}}}
    \newcommand{\AnnotationTok}[1]{\textcolor[rgb]{0.38,0.63,0.69}{\textbf{\textit{{#1}}}}}
    \newcommand{\CommentVarTok}[1]{\textcolor[rgb]{0.38,0.63,0.69}{\textbf{\textit{{#1}}}}}
    \newcommand{\VariableTok}[1]{\textcolor[rgb]{0.10,0.09,0.49}{{#1}}}
    \newcommand{\ControlFlowTok}[1]{\textcolor[rgb]{0.00,0.44,0.13}{\textbf{{#1}}}}
    \newcommand{\OperatorTok}[1]{\textcolor[rgb]{0.40,0.40,0.40}{{#1}}}
    \newcommand{\BuiltInTok}[1]{{#1}}
    \newcommand{\ExtensionTok}[1]{{#1}}
    \newcommand{\PreprocessorTok}[1]{\textcolor[rgb]{0.74,0.48,0.00}{{#1}}}
    \newcommand{\AttributeTok}[1]{\textcolor[rgb]{0.49,0.56,0.16}{{#1}}}
    \newcommand{\InformationTok}[1]{\textcolor[rgb]{0.38,0.63,0.69}{\textbf{\textit{{#1}}}}}
    \newcommand{\WarningTok}[1]{\textcolor[rgb]{0.38,0.63,0.69}{\textbf{\textit{{#1}}}}}
    
    
    % Define a nice break command that doesn't care if a line doesn't already
    % exist.
    \def\br{\hspace*{\fill} \\* }
    % Math Jax compatibility definitions
    \def\gt{>}
    \def\lt{<}
    \let\Oldtex\TeX
    \let\Oldlatex\LaTeX
    \renewcommand{\TeX}{\textrm{\Oldtex}}
    \renewcommand{\LaTeX}{\textrm{\Oldlatex}}
    % Document parameters
    % Document title
    \title{matrices}
    
    
    
    
    
% Pygments definitions
\makeatletter
\def\PY@reset{\let\PY@it=\relax \let\PY@bf=\relax%
    \let\PY@ul=\relax \let\PY@tc=\relax%
    \let\PY@bc=\relax \let\PY@ff=\relax}
\def\PY@tok#1{\csname PY@tok@#1\endcsname}
\def\PY@toks#1+{\ifx\relax#1\empty\else%
    \PY@tok{#1}\expandafter\PY@toks\fi}
\def\PY@do#1{\PY@bc{\PY@tc{\PY@ul{%
    \PY@it{\PY@bf{\PY@ff{#1}}}}}}}
\def\PY#1#2{\PY@reset\PY@toks#1+\relax+\PY@do{#2}}

\expandafter\def\csname PY@tok@w\endcsname{\def\PY@tc##1{\textcolor[rgb]{0.73,0.73,0.73}{##1}}}
\expandafter\def\csname PY@tok@c\endcsname{\let\PY@it=\textit\def\PY@tc##1{\textcolor[rgb]{0.25,0.50,0.50}{##1}}}
\expandafter\def\csname PY@tok@cp\endcsname{\def\PY@tc##1{\textcolor[rgb]{0.74,0.48,0.00}{##1}}}
\expandafter\def\csname PY@tok@k\endcsname{\let\PY@bf=\textbf\def\PY@tc##1{\textcolor[rgb]{0.00,0.50,0.00}{##1}}}
\expandafter\def\csname PY@tok@kp\endcsname{\def\PY@tc##1{\textcolor[rgb]{0.00,0.50,0.00}{##1}}}
\expandafter\def\csname PY@tok@kt\endcsname{\def\PY@tc##1{\textcolor[rgb]{0.69,0.00,0.25}{##1}}}
\expandafter\def\csname PY@tok@o\endcsname{\def\PY@tc##1{\textcolor[rgb]{0.40,0.40,0.40}{##1}}}
\expandafter\def\csname PY@tok@ow\endcsname{\let\PY@bf=\textbf\def\PY@tc##1{\textcolor[rgb]{0.67,0.13,1.00}{##1}}}
\expandafter\def\csname PY@tok@nb\endcsname{\def\PY@tc##1{\textcolor[rgb]{0.00,0.50,0.00}{##1}}}
\expandafter\def\csname PY@tok@nf\endcsname{\def\PY@tc##1{\textcolor[rgb]{0.00,0.00,1.00}{##1}}}
\expandafter\def\csname PY@tok@nc\endcsname{\let\PY@bf=\textbf\def\PY@tc##1{\textcolor[rgb]{0.00,0.00,1.00}{##1}}}
\expandafter\def\csname PY@tok@nn\endcsname{\let\PY@bf=\textbf\def\PY@tc##1{\textcolor[rgb]{0.00,0.00,1.00}{##1}}}
\expandafter\def\csname PY@tok@ne\endcsname{\let\PY@bf=\textbf\def\PY@tc##1{\textcolor[rgb]{0.82,0.25,0.23}{##1}}}
\expandafter\def\csname PY@tok@nv\endcsname{\def\PY@tc##1{\textcolor[rgb]{0.10,0.09,0.49}{##1}}}
\expandafter\def\csname PY@tok@no\endcsname{\def\PY@tc##1{\textcolor[rgb]{0.53,0.00,0.00}{##1}}}
\expandafter\def\csname PY@tok@nl\endcsname{\def\PY@tc##1{\textcolor[rgb]{0.63,0.63,0.00}{##1}}}
\expandafter\def\csname PY@tok@ni\endcsname{\let\PY@bf=\textbf\def\PY@tc##1{\textcolor[rgb]{0.60,0.60,0.60}{##1}}}
\expandafter\def\csname PY@tok@na\endcsname{\def\PY@tc##1{\textcolor[rgb]{0.49,0.56,0.16}{##1}}}
\expandafter\def\csname PY@tok@nt\endcsname{\let\PY@bf=\textbf\def\PY@tc##1{\textcolor[rgb]{0.00,0.50,0.00}{##1}}}
\expandafter\def\csname PY@tok@nd\endcsname{\def\PY@tc##1{\textcolor[rgb]{0.67,0.13,1.00}{##1}}}
\expandafter\def\csname PY@tok@s\endcsname{\def\PY@tc##1{\textcolor[rgb]{0.73,0.13,0.13}{##1}}}
\expandafter\def\csname PY@tok@sd\endcsname{\let\PY@it=\textit\def\PY@tc##1{\textcolor[rgb]{0.73,0.13,0.13}{##1}}}
\expandafter\def\csname PY@tok@si\endcsname{\let\PY@bf=\textbf\def\PY@tc##1{\textcolor[rgb]{0.73,0.40,0.53}{##1}}}
\expandafter\def\csname PY@tok@se\endcsname{\let\PY@bf=\textbf\def\PY@tc##1{\textcolor[rgb]{0.73,0.40,0.13}{##1}}}
\expandafter\def\csname PY@tok@sr\endcsname{\def\PY@tc##1{\textcolor[rgb]{0.73,0.40,0.53}{##1}}}
\expandafter\def\csname PY@tok@ss\endcsname{\def\PY@tc##1{\textcolor[rgb]{0.10,0.09,0.49}{##1}}}
\expandafter\def\csname PY@tok@sx\endcsname{\def\PY@tc##1{\textcolor[rgb]{0.00,0.50,0.00}{##1}}}
\expandafter\def\csname PY@tok@m\endcsname{\def\PY@tc##1{\textcolor[rgb]{0.40,0.40,0.40}{##1}}}
\expandafter\def\csname PY@tok@gh\endcsname{\let\PY@bf=\textbf\def\PY@tc##1{\textcolor[rgb]{0.00,0.00,0.50}{##1}}}
\expandafter\def\csname PY@tok@gu\endcsname{\let\PY@bf=\textbf\def\PY@tc##1{\textcolor[rgb]{0.50,0.00,0.50}{##1}}}
\expandafter\def\csname PY@tok@gd\endcsname{\def\PY@tc##1{\textcolor[rgb]{0.63,0.00,0.00}{##1}}}
\expandafter\def\csname PY@tok@gi\endcsname{\def\PY@tc##1{\textcolor[rgb]{0.00,0.63,0.00}{##1}}}
\expandafter\def\csname PY@tok@gr\endcsname{\def\PY@tc##1{\textcolor[rgb]{1.00,0.00,0.00}{##1}}}
\expandafter\def\csname PY@tok@ge\endcsname{\let\PY@it=\textit}
\expandafter\def\csname PY@tok@gs\endcsname{\let\PY@bf=\textbf}
\expandafter\def\csname PY@tok@gp\endcsname{\let\PY@bf=\textbf\def\PY@tc##1{\textcolor[rgb]{0.00,0.00,0.50}{##1}}}
\expandafter\def\csname PY@tok@go\endcsname{\def\PY@tc##1{\textcolor[rgb]{0.53,0.53,0.53}{##1}}}
\expandafter\def\csname PY@tok@gt\endcsname{\def\PY@tc##1{\textcolor[rgb]{0.00,0.27,0.87}{##1}}}
\expandafter\def\csname PY@tok@err\endcsname{\def\PY@bc##1{\setlength{\fboxsep}{0pt}\fcolorbox[rgb]{1.00,0.00,0.00}{1,1,1}{\strut ##1}}}
\expandafter\def\csname PY@tok@kc\endcsname{\let\PY@bf=\textbf\def\PY@tc##1{\textcolor[rgb]{0.00,0.50,0.00}{##1}}}
\expandafter\def\csname PY@tok@kd\endcsname{\let\PY@bf=\textbf\def\PY@tc##1{\textcolor[rgb]{0.00,0.50,0.00}{##1}}}
\expandafter\def\csname PY@tok@kn\endcsname{\let\PY@bf=\textbf\def\PY@tc##1{\textcolor[rgb]{0.00,0.50,0.00}{##1}}}
\expandafter\def\csname PY@tok@kr\endcsname{\let\PY@bf=\textbf\def\PY@tc##1{\textcolor[rgb]{0.00,0.50,0.00}{##1}}}
\expandafter\def\csname PY@tok@bp\endcsname{\def\PY@tc##1{\textcolor[rgb]{0.00,0.50,0.00}{##1}}}
\expandafter\def\csname PY@tok@fm\endcsname{\def\PY@tc##1{\textcolor[rgb]{0.00,0.00,1.00}{##1}}}
\expandafter\def\csname PY@tok@vc\endcsname{\def\PY@tc##1{\textcolor[rgb]{0.10,0.09,0.49}{##1}}}
\expandafter\def\csname PY@tok@vg\endcsname{\def\PY@tc##1{\textcolor[rgb]{0.10,0.09,0.49}{##1}}}
\expandafter\def\csname PY@tok@vi\endcsname{\def\PY@tc##1{\textcolor[rgb]{0.10,0.09,0.49}{##1}}}
\expandafter\def\csname PY@tok@vm\endcsname{\def\PY@tc##1{\textcolor[rgb]{0.10,0.09,0.49}{##1}}}
\expandafter\def\csname PY@tok@sa\endcsname{\def\PY@tc##1{\textcolor[rgb]{0.73,0.13,0.13}{##1}}}
\expandafter\def\csname PY@tok@sb\endcsname{\def\PY@tc##1{\textcolor[rgb]{0.73,0.13,0.13}{##1}}}
\expandafter\def\csname PY@tok@sc\endcsname{\def\PY@tc##1{\textcolor[rgb]{0.73,0.13,0.13}{##1}}}
\expandafter\def\csname PY@tok@dl\endcsname{\def\PY@tc##1{\textcolor[rgb]{0.73,0.13,0.13}{##1}}}
\expandafter\def\csname PY@tok@s2\endcsname{\def\PY@tc##1{\textcolor[rgb]{0.73,0.13,0.13}{##1}}}
\expandafter\def\csname PY@tok@sh\endcsname{\def\PY@tc##1{\textcolor[rgb]{0.73,0.13,0.13}{##1}}}
\expandafter\def\csname PY@tok@s1\endcsname{\def\PY@tc##1{\textcolor[rgb]{0.73,0.13,0.13}{##1}}}
\expandafter\def\csname PY@tok@mb\endcsname{\def\PY@tc##1{\textcolor[rgb]{0.40,0.40,0.40}{##1}}}
\expandafter\def\csname PY@tok@mf\endcsname{\def\PY@tc##1{\textcolor[rgb]{0.40,0.40,0.40}{##1}}}
\expandafter\def\csname PY@tok@mh\endcsname{\def\PY@tc##1{\textcolor[rgb]{0.40,0.40,0.40}{##1}}}
\expandafter\def\csname PY@tok@mi\endcsname{\def\PY@tc##1{\textcolor[rgb]{0.40,0.40,0.40}{##1}}}
\expandafter\def\csname PY@tok@il\endcsname{\def\PY@tc##1{\textcolor[rgb]{0.40,0.40,0.40}{##1}}}
\expandafter\def\csname PY@tok@mo\endcsname{\def\PY@tc##1{\textcolor[rgb]{0.40,0.40,0.40}{##1}}}
\expandafter\def\csname PY@tok@ch\endcsname{\let\PY@it=\textit\def\PY@tc##1{\textcolor[rgb]{0.25,0.50,0.50}{##1}}}
\expandafter\def\csname PY@tok@cm\endcsname{\let\PY@it=\textit\def\PY@tc##1{\textcolor[rgb]{0.25,0.50,0.50}{##1}}}
\expandafter\def\csname PY@tok@cpf\endcsname{\let\PY@it=\textit\def\PY@tc##1{\textcolor[rgb]{0.25,0.50,0.50}{##1}}}
\expandafter\def\csname PY@tok@c1\endcsname{\let\PY@it=\textit\def\PY@tc##1{\textcolor[rgb]{0.25,0.50,0.50}{##1}}}
\expandafter\def\csname PY@tok@cs\endcsname{\let\PY@it=\textit\def\PY@tc##1{\textcolor[rgb]{0.25,0.50,0.50}{##1}}}

\def\PYZbs{\char`\\}
\def\PYZus{\char`\_}
\def\PYZob{\char`\{}
\def\PYZcb{\char`\}}
\def\PYZca{\char`\^}
\def\PYZam{\char`\&}
\def\PYZlt{\char`\<}
\def\PYZgt{\char`\>}
\def\PYZsh{\char`\#}
\def\PYZpc{\char`\%}
\def\PYZdl{\char`\$}
\def\PYZhy{\char`\-}
\def\PYZsq{\char`\'}
\def\PYZdq{\char`\"}
\def\PYZti{\char`\~}
% for compatibility with earlier versions
\def\PYZat{@}
\def\PYZlb{[}
\def\PYZrb{]}
\makeatother


    % For linebreaks inside Verbatim environment from package fancyvrb. 
    \makeatletter
        \newbox\Wrappedcontinuationbox 
        \newbox\Wrappedvisiblespacebox 
        \newcommand*\Wrappedvisiblespace {\textcolor{red}{\textvisiblespace}} 
        \newcommand*\Wrappedcontinuationsymbol {\textcolor{red}{\llap{\tiny$\m@th\hookrightarrow$}}} 
        \newcommand*\Wrappedcontinuationindent {3ex } 
        \newcommand*\Wrappedafterbreak {\kern\Wrappedcontinuationindent\copy\Wrappedcontinuationbox} 
        % Take advantage of the already applied Pygments mark-up to insert 
        % potential linebreaks for TeX processing. 
        %        {, <, #, %, $, ' and ": go to next line. 
        %        _, }, ^, &, >, - and ~: stay at end of broken line. 
        % Use of \textquotesingle for straight quote. 
        \newcommand*\Wrappedbreaksatspecials {% 
            \def\PYGZus{\discretionary{\char`\_}{\Wrappedafterbreak}{\char`\_}}% 
            \def\PYGZob{\discretionary{}{\Wrappedafterbreak\char`\{}{\char`\{}}% 
            \def\PYGZcb{\discretionary{\char`\}}{\Wrappedafterbreak}{\char`\}}}% 
            \def\PYGZca{\discretionary{\char`\^}{\Wrappedafterbreak}{\char`\^}}% 
            \def\PYGZam{\discretionary{\char`\&}{\Wrappedafterbreak}{\char`\&}}% 
            \def\PYGZlt{\discretionary{}{\Wrappedafterbreak\char`\<}{\char`\<}}% 
            \def\PYGZgt{\discretionary{\char`\>}{\Wrappedafterbreak}{\char`\>}}% 
            \def\PYGZsh{\discretionary{}{\Wrappedafterbreak\char`\#}{\char`\#}}% 
            \def\PYGZpc{\discretionary{}{\Wrappedafterbreak\char`\%}{\char`\%}}% 
            \def\PYGZdl{\discretionary{}{\Wrappedafterbreak\char`\$}{\char`\$}}% 
            \def\PYGZhy{\discretionary{\char`\-}{\Wrappedafterbreak}{\char`\-}}% 
            \def\PYGZsq{\discretionary{}{\Wrappedafterbreak\textquotesingle}{\textquotesingle}}% 
            \def\PYGZdq{\discretionary{}{\Wrappedafterbreak\char`\"}{\char`\"}}% 
            \def\PYGZti{\discretionary{\char`\~}{\Wrappedafterbreak}{\char`\~}}% 
        } 
        % Some characters . , ; ? ! / are not pygmentized. 
        % This macro makes them "active" and they will insert potential linebreaks 
        \newcommand*\Wrappedbreaksatpunct {% 
            \lccode`\~`\.\lowercase{\def~}{\discretionary{\hbox{\char`\.}}{\Wrappedafterbreak}{\hbox{\char`\.}}}% 
            \lccode`\~`\,\lowercase{\def~}{\discretionary{\hbox{\char`\,}}{\Wrappedafterbreak}{\hbox{\char`\,}}}% 
            \lccode`\~`\;\lowercase{\def~}{\discretionary{\hbox{\char`\;}}{\Wrappedafterbreak}{\hbox{\char`\;}}}% 
            \lccode`\~`\:\lowercase{\def~}{\discretionary{\hbox{\char`\:}}{\Wrappedafterbreak}{\hbox{\char`\:}}}% 
            \lccode`\~`\?\lowercase{\def~}{\discretionary{\hbox{\char`\?}}{\Wrappedafterbreak}{\hbox{\char`\?}}}% 
            \lccode`\~`\!\lowercase{\def~}{\discretionary{\hbox{\char`\!}}{\Wrappedafterbreak}{\hbox{\char`\!}}}% 
            \lccode`\~`\/\lowercase{\def~}{\discretionary{\hbox{\char`\/}}{\Wrappedafterbreak}{\hbox{\char`\/}}}% 
            \catcode`\.\active
            \catcode`\,\active 
            \catcode`\;\active
            \catcode`\:\active
            \catcode`\?\active
            \catcode`\!\active
            \catcode`\/\active 
            \lccode`\~`\~ 	
        }
    \makeatother

    \let\OriginalVerbatim=\Verbatim
    \makeatletter
    \renewcommand{\Verbatim}[1][1]{%
        %\parskip\z@skip
        \sbox\Wrappedcontinuationbox {\Wrappedcontinuationsymbol}%
        \sbox\Wrappedvisiblespacebox {\FV@SetupFont\Wrappedvisiblespace}%
        \def\FancyVerbFormatLine ##1{\hsize\linewidth
            \vtop{\raggedright\hyphenpenalty\z@\exhyphenpenalty\z@
                \doublehyphendemerits\z@\finalhyphendemerits\z@
                \strut ##1\strut}%
        }%
        % If the linebreak is at a space, the latter will be displayed as visible
        % space at end of first line, and a continuation symbol starts next line.
        % Stretch/shrink are however usually zero for typewriter font.
        \def\FV@Space {%
            \nobreak\hskip\z@ plus\fontdimen3\font minus\fontdimen4\font
            \discretionary{\copy\Wrappedvisiblespacebox}{\Wrappedafterbreak}
            {\kern\fontdimen2\font}%
        }%
        
        % Allow breaks at special characters using \PYG... macros.
        \Wrappedbreaksatspecials
        % Breaks at punctuation characters . , ; ? ! and / need catcode=\active 	
        \OriginalVerbatim[#1,codes*=\Wrappedbreaksatpunct]%
    }
    \makeatother

    % Exact colors from NB
    \definecolor{incolor}{HTML}{303F9F}
    \definecolor{outcolor}{HTML}{D84315}
    \definecolor{cellborder}{HTML}{CFCFCF}
    \definecolor{cellbackground}{HTML}{F7F7F7}
    
    % prompt
    \makeatletter
    \newcommand{\boxspacing}{\kern\kvtcb@left@rule\kern\kvtcb@boxsep}
    \makeatother
    \newcommand{\prompt}[4]{
        \ttfamily\llap{{\color{#2}[#3]:\hspace{3pt}#4}}\vspace{-\baselineskip}
    }
    

    
    % Prevent overflowing lines due to hard-to-break entities
    \sloppy 
    % Setup hyperref package
    \hypersetup{
      breaklinks=true,  % so long urls are correctly broken across lines
      colorlinks=true,
      urlcolor=urlcolor,
      linkcolor=linkcolor,
      citecolor=citecolor,
      }
    % Slightly bigger margins than the latex defaults
    
    \geometry{verbose,tmargin=1in,bmargin=1in,lmargin=1in,rmargin=1in}
    
    

\begin{document}
    
    \maketitle
    
    

    
    \hypertarget{matrices}{%
\subsection{Matrices}\label{matrices}}

In mathematics, a matrix (plural matrices) is a rectangular array of
numbers, symbols, or expressions, arranged in rows and columns. Matrices
are commonly written in box brackets. The horizontal and vertical lines
of entries in a matrix are called rows and columns, respectively. The
size of a matrix is defined by the number of rows and columns that it
contains. A matrix with \(m\) rows and \(n\) columns is called an
\(m\times n\) matrix or \(m\)-by-\(n\) matrix, while \(m\) and \(n\) are
called its dimensions. The dimensions of the following matrix are
\(2\times 3\) (read ``two by three''), because there are two rows and
three columns.

\begin{bmatrix}
1 & 2 & 3\\
4 & 5 & 6
\end{bmatrix}

The individual items (numbers, symbols or expressions) in a matrix are
called its elements or entries.

In \texttt{python} matrices can be represented as \texttt{numpy.array},
so the example above will become:

    \begin{tcolorbox}[breakable, size=fbox, boxrule=1pt, pad at break*=1mm,colback=cellbackground, colframe=cellborder]
\prompt{In}{incolor}{17}{\boxspacing}
\begin{Verbatim}[commandchars=\\\{\}]
\PY{k+kn}{import} \PY{n+nn}{numpy} \PY{k}{as} \PY{n+nn}{np}

\PY{n+nb}{print} \PY{p}{(}\PY{n}{np}\PY{o}{.}\PY{n}{array}\PY{p}{(}\PY{p}{[}\PY{p}{[}\PY{l+m+mi}{1}\PY{p}{,}\PY{l+m+mi}{2}\PY{p}{,}\PY{l+m+mi}{3}\PY{p}{]}\PY{p}{,}\PY{p}{[}\PY{l+m+mi}{4}\PY{p}{,}\PY{l+m+mi}{5}\PY{p}{,}\PY{l+m+mi}{6}\PY{p}{]}\PY{p}{]}\PY{p}{)}\PY{p}{)}
\end{Verbatim}
\end{tcolorbox}

    \begin{Verbatim}[commandchars=\\\{\}]
[[1 2 3]
 [4 5 6]]
    \end{Verbatim}

    Essentially a \texttt{numpy.array} is a list of lists each one
representing a matrix row. Arrays can have any dimension so they can be
used to represent also vectors in \texttt{python}. There are two special
types of arrays \texttt{zeros} and \texttt{ones} whose name already
clarify their meaning:

    \begin{tcolorbox}[breakable, size=fbox, boxrule=1pt, pad at break*=1mm,colback=cellbackground, colframe=cellborder]
\prompt{In}{incolor}{18}{\boxspacing}
\begin{Verbatim}[commandchars=\\\{\}]
\PY{n}{A} \PY{o}{=} \PY{n}{np}\PY{o}{.}\PY{n}{zeros}\PY{p}{(}\PY{n}{shape}\PY{o}{=}\PY{p}{(}\PY{l+m+mi}{3}\PY{p}{,} \PY{l+m+mi}{3}\PY{p}{)}\PY{p}{)}
\PY{n}{B} \PY{o}{=} \PY{n}{np}\PY{o}{.}\PY{n}{ones}\PY{p}{(}\PY{n}{shape}\PY{o}{=}\PY{p}{(}\PY{l+m+mi}{4}\PY{p}{,} \PY{l+m+mi}{4}\PY{p}{)}\PY{p}{)}

\PY{n+nb}{print} \PY{p}{(}\PY{n}{A}\PY{p}{)}
\PY{n+nb}{print} \PY{p}{(}\PY{p}{)}
\PY{n+nb}{print} \PY{p}{(}\PY{n}{B}\PY{p}{)}
\end{Verbatim}
\end{tcolorbox}

    \begin{Verbatim}[commandchars=\\\{\}]
[[0. 0. 0.]
 [0. 0. 0.]
 [0. 0. 0.]]

[[1. 1. 1. 1.]
 [1. 1. 1. 1.]
 [1. 1. 1. 1.]
 [1. 1. 1. 1.]]
    \end{Verbatim}

    \hypertarget{adding-and-subtracting-matrices}{%
\subsubsection{Adding and Subtracting
Matrices}\label{adding-and-subtracting-matrices}}

We use matrices to list data or to represent systems. Because the
entries are numbers, we can perform operations on matrices. We add or
subtract matrices by adding or subtracting corresponding entries.

In order to do this, the entries must correspond. Therefore, addition
and subtraction of matrices is only possible when the matrices have the
same dimensions.\\
Adding matrices is very simple. Just add each element in the first
matrix to the corresponding element in the second matrix.

\[
\begin{bmatrix}
1 & 2 & 3 \\
4 & 5 & 6
\end{bmatrix}
+
\begin{bmatrix}
10 & 20 & 30\\
40 & 50 & 60
\end{bmatrix}
=
\begin{bmatrix}
11 & 22 & 33\\
44 & 55 & 66
\end{bmatrix}
\]

As you might guess, subtracting works much the same way except that you
subtract instead of adding.

\[
\begin{bmatrix}
10 & 20 & 30\\
40 & 50 & 60
\end{bmatrix}
-
\begin{bmatrix}
1 & 2 & 3 \\
4 & 5 & 6
\end{bmatrix}
=
\begin{bmatrix}
9 & 18 & 27\\
36 & 45 & 54
\end{bmatrix}
\]

Adding and subtracting \texttt{numpy.array} is as easy as that:

    \begin{tcolorbox}[breakable, size=fbox, boxrule=1pt, pad at break*=1mm,colback=cellbackground, colframe=cellborder]
\prompt{In}{incolor}{19}{\boxspacing}
\begin{Verbatim}[commandchars=\\\{\}]
\PY{n}{A} \PY{o}{=} \PY{n}{np}\PY{o}{.}\PY{n}{array}\PY{p}{(}\PY{p}{[}\PY{p}{[}\PY{l+m+mi}{1}\PY{p}{,}\PY{l+m+mi}{2}\PY{p}{,}\PY{l+m+mi}{3}\PY{p}{]}\PY{p}{,}\PY{p}{[}\PY{l+m+mi}{4}\PY{p}{,}\PY{l+m+mi}{5}\PY{p}{,}\PY{l+m+mi}{6}\PY{p}{]}\PY{p}{]}\PY{p}{)}
\PY{n}{B} \PY{o}{=} \PY{n}{np}\PY{o}{.}\PY{n}{array}\PY{p}{(}\PY{p}{[}\PY{p}{[}\PY{l+m+mi}{10}\PY{p}{,}\PY{l+m+mi}{20}\PY{p}{,}\PY{l+m+mi}{30}\PY{p}{]}\PY{p}{,}\PY{p}{[}\PY{l+m+mi}{40}\PY{p}{,}\PY{l+m+mi}{50}\PY{p}{,}\PY{l+m+mi}{60}\PY{p}{]}\PY{p}{]}\PY{p}{)}

\PY{n+nb}{print} \PY{p}{(}\PY{n}{A}\PY{o}{+}\PY{n}{B}\PY{p}{)}
\PY{n+nb}{print} \PY{p}{(}\PY{n}{B}\PY{o}{\PYZhy{}}\PY{n}{A}\PY{p}{)}
\end{Verbatim}
\end{tcolorbox}

    \begin{Verbatim}[commandchars=\\\{\}]
[[11 22 33]
 [44 55 66]]
[[ 9 18 27]
 [36 45 54]]
    \end{Verbatim}

    \hypertarget{scalar-multiplication}{%
\subsubsection{Scalar Multiplication}\label{scalar-multiplication}}

Multiplying a matrix by a scalar \(c\) means you add the matrix to
itself \(c\) times, or simply multiply each element by that constant.

\[
3 \cdot
\begin{bmatrix}
1 & 2 & 3 \\
4 & 5 & 6
\end{bmatrix}
=
\begin{bmatrix}
3 & 6 & 9 \\
12 & 15 & 18
\end{bmatrix}
\]

Scalar multiplication of \texttt{numpy.array} is:

    \begin{tcolorbox}[breakable, size=fbox, boxrule=1pt, pad at break*=1mm,colback=cellbackground, colframe=cellborder]
\prompt{In}{incolor}{20}{\boxspacing}
\begin{Verbatim}[commandchars=\\\{\}]
\PY{n}{c} \PY{o}{=} \PY{l+m+mi}{3}
\PY{n}{A} \PY{o}{=} \PY{n}{np}\PY{o}{.}\PY{n}{array}\PY{p}{(}\PY{p}{[}\PY{p}{[}\PY{l+m+mi}{1}\PY{p}{,}\PY{l+m+mi}{2}\PY{p}{,}\PY{l+m+mi}{3}\PY{p}{]}\PY{p}{,}\PY{p}{[}\PY{l+m+mi}{4}\PY{p}{,}\PY{l+m+mi}{5}\PY{p}{,}\PY{l+m+mi}{6}\PY{p}{]}\PY{p}{]}\PY{p}{)}

\PY{n+nb}{print} \PY{p}{(}\PY{n}{c}\PY{o}{*}\PY{n}{A}\PY{p}{)}
\end{Verbatim}
\end{tcolorbox}

    \begin{Verbatim}[commandchars=\\\{\}]
[[ 3  6  9]
 [12 15 18]]
    \end{Verbatim}

    \hypertarget{matrix-multiplication}{%
\subsubsection{Matrix Multiplication}\label{matrix-multiplication}}

Matrix multiplication is multiplying every element of each row of the
first matrix times every element of each column in the second matrix.
When multiplying matrices, the elements of the rows in the first matrix
are multiplied with corresponding columns in the second matrix. Each
entry of the resultant matrix is computed one at a time.

Let's see with an example: \[ 
\begin{bmatrix}
1 & 2 \\
3 & 4
\end{bmatrix}
\cdot
\begin{bmatrix}
5 & 6 \\
7 & 8
\end{bmatrix}
= ?
\]

First ask: Do the number of columns in the first matrix equal the number
of rows in the second ? If so the product exists. Then start with
producing the product for the first row, first column element. Take the
first row of the first matrix and multiply by the first column of the
second like this:

\[ 
\begin{bmatrix}
1 & 2 \\
3 & 4
\end{bmatrix}
\cdot
\begin{bmatrix}
5 & 6 \\
7 & 8
\end{bmatrix}
=
\begin{bmatrix}
(1\cdot 5) + (2\cdot 7) & X \\
X & X
\end{bmatrix}
\]

Continue the pattern with the first row of the first matrix with the
second column of the second matrix:

\[ 
\begin{bmatrix}
1 & 2 \\
3 & 4
\end{bmatrix}
\cdot
\begin{bmatrix}
5 & 6 \\
7 & 8
\end{bmatrix}
=
\begin{bmatrix}
(1\cdot 5) + (2\cdot 7) & (1\cdot 6) + (2\cdot 8)  \\
X & X
\end{bmatrix}
\]

The do the same with the second row of the first matrix and you are
done:

\[ 
\begin{bmatrix}
1 & 2 \\
3 & 4
\end{bmatrix}
\cdot
\begin{bmatrix}
5 & 6 \\
7 & 8
\end{bmatrix}
=
\begin{bmatrix}
(1\cdot 5) + (2\cdot 7) & (1\cdot 6) + (2\cdot 8)  \\
(3\cdot 5) + (4\cdot 7) & (3\cdot 6) + (4\cdot 8) 
\end{bmatrix}
=
\begin{bmatrix}
19 & 22 \\
43 & 50 
\end{bmatrix}
\]

In \texttt{numpy} array multiplication can be done like this:

    \begin{tcolorbox}[breakable, size=fbox, boxrule=1pt, pad at break*=1mm,colback=cellbackground, colframe=cellborder]
\prompt{In}{incolor}{21}{\boxspacing}
\begin{Verbatim}[commandchars=\\\{\}]
\PY{n}{A} \PY{o}{=} \PY{n}{np}\PY{o}{.}\PY{n}{array}\PY{p}{(}\PY{p}{[}\PY{p}{[}\PY{l+m+mi}{1}\PY{p}{,}\PY{l+m+mi}{2}\PY{p}{]}\PY{p}{,}\PY{p}{[}\PY{l+m+mi}{3}\PY{p}{,}\PY{l+m+mi}{4}\PY{p}{]}\PY{p}{]}\PY{p}{)}
\PY{n}{B} \PY{o}{=} \PY{n}{np}\PY{o}{.}\PY{n}{array}\PY{p}{(}\PY{p}{[}\PY{p}{[}\PY{l+m+mi}{5}\PY{p}{,}\PY{l+m+mi}{6}\PY{p}{]}\PY{p}{,}\PY{p}{[}\PY{l+m+mi}{7}\PY{p}{,}\PY{l+m+mi}{8}\PY{p}{]}\PY{p}{]}\PY{p}{)}

\PY{n+nb}{print} \PY{p}{(}\PY{n}{np}\PY{o}{.}\PY{n}{dot}\PY{p}{(}\PY{n}{A}\PY{p}{,} \PY{n}{B}\PY{p}{)}\PY{p}{)}
\end{Verbatim}
\end{tcolorbox}

    \begin{Verbatim}[commandchars=\\\{\}]
[[19 22]
 [43 50]]
    \end{Verbatim}

    \hypertarget{the-identity-matrix}{%
\subsubsection{The identity matrix}\label{the-identity-matrix}}

\([I]\) is defined so that \([A][I]=[A]\), i.e.~it is the matrix version
of multiplying a number by one. What matrix has this property? A first
guess might be a matrix full of 1s, but that does not work:

\[
\begin{bmatrix}
1 & 2 \\
3 & 4
\end{bmatrix}
\begin{bmatrix}
1 & 1 \\
1 & 1
\end{bmatrix}
=
\begin{bmatrix}
3 & 3 \\
7 & 7
\end{bmatrix}
\]

So our initial guess was wrong ! The matrix that does work is a diagonal
stretch of 1s, with all other elements being 0:

\[
\begin{bmatrix}
1 & 2 \\
3 & 4
\end{bmatrix}
\begin{bmatrix}
1 & 0 \\
0 & 1
\end{bmatrix}
=
\begin{bmatrix}
1 & 2 \\
3 & 4
\end{bmatrix}
\]

So \$ I =

\begin{bmatrix}
1 & 0 \\
0 & 1
\end{bmatrix}

\$ is the identity matrix for \(2\times 2\) matrices.

The \texttt{numpy} equivalent of the identity matrix is given by
\texttt{numpy.identity(n)} with n the dimension of the matrix. So for
example:

    \begin{tcolorbox}[breakable, size=fbox, boxrule=1pt, pad at break*=1mm,colback=cellbackground, colframe=cellborder]
\prompt{In}{incolor}{22}{\boxspacing}
\begin{Verbatim}[commandchars=\\\{\}]
\PY{n}{A} \PY{o}{=} \PY{n}{np}\PY{o}{.}\PY{n}{array}\PY{p}{(}\PY{p}{[}\PY{p}{[}\PY{l+m+mi}{1}\PY{p}{,}\PY{l+m+mi}{2}\PY{p}{]}\PY{p}{,}\PY{p}{[}\PY{l+m+mi}{3}\PY{p}{,}\PY{l+m+mi}{4}\PY{p}{]}\PY{p}{]}\PY{p}{)}
\PY{n}{I} \PY{o}{=} \PY{n}{np}\PY{o}{.}\PY{n}{identity}\PY{p}{(}\PY{l+m+mi}{2}\PY{p}{)}

\PY{n+nb}{print} \PY{p}{(}\PY{n}{A}\PY{p}{)}
\PY{n+nb}{print} \PY{p}{(}\PY{p}{)}
\PY{n+nb}{print} \PY{p}{(}\PY{n}{np}\PY{o}{.}\PY{n}{dot}\PY{p}{(}\PY{n}{A}\PY{p}{,} \PY{n}{I}\PY{p}{)}\PY{p}{)}
\end{Verbatim}
\end{tcolorbox}

    \begin{Verbatim}[commandchars=\\\{\}]
[[1 2]
 [3 4]]

[[1. 2.]
 [3. 4.]]
    \end{Verbatim}

    \hypertarget{matrix-equations}{%
\subsubsection{Matrix Equations}\label{matrix-equations}}

Matrices can be used to compactly write and work with systems of
multiple linear equations. As we have learned in previous sections,
matrices can be manipulated in any way that a normal equation can be.
This is very helpful when we start to work with systems of equations. It
is helpful to understand how to organize matrices to solve these
systems.

\hypertarget{writing-a-system-of-equations-with-matrices}{%
\subsubsection{Writing a System of Equations with
Matrices}\label{writing-a-system-of-equations-with-matrices}}

It is possible to solve this system using the elimination or
substitution method, but it is also possible to do it with a matrix
operation. Before we start setting up the matrices, it is important to
do the following:

\begin{itemize}
\tightlist
\item
  make sure that all of the equations are written in a similar manner,
  meaning the variables need to all be in the same order;
\item
  make sure that one side of the equation is only variables and their
  coefficients, and the other side is just constants;
\end{itemize}

Solving a system of linear equations using the inverse of a matrix
requires the definition of two new matrices:  \(X\) is the matrix
representing the variables of the system, and \(B\) is the matrix
representing the constants. Using matrix multiplication, we may define a
system of equations with the same number of equations as variables as:

\[ A\cdot X = B\]

where \(A\) is the coefficient matrix, \(X\) is the variable matrix, and
\(B\) is the constant matrix. Given the system:

\[
\begin{cases}
x + 8y = 7 \\
2x − 8y = −3
\end{cases}
\]

The corresponding matrices are then:

\[
A=
\begin{bmatrix}
1 & 8\\
2 & −8
\end{bmatrix}
;
X=
\begin{bmatrix}
x\\
y
\end{bmatrix}
;
A=
\begin{bmatrix}
7\\
-3
\end{bmatrix}
\]

Thus, to solve a system \(AX=B\), for \(X\), multiply both sides by the
inverse of \(A\) and we shall obtain the solution:

\[A^{-1}AX=A^{-1}B \implies X = A^{-1}B \]

Provided the inverse \(A^{-1}\) exists, this formula will solve the
system. If the coefficient matrix is not invertible, the system could be
inconsistent and have no solution, or be dependent and have infinitely
many solutions.

    \hypertarget{the-inverse-of-a-matrix}{%
\subsubsection{The Inverse of a Matrix}\label{the-inverse-of-a-matrix}}

The inverse of matrix \(A\) is \(A^{-1}\), and is defined by the
property:

\[ AA^{-1}=I \]

Hence the matrix \(B\) is the inverse of the matrix \(A\) if when
multiplied together, \(A\cdot B\) gives the identity matrix. Using the
definition let's try to find the inverse of:

\[
\begin{bmatrix}
3 & 4\\
5 & 6
\end{bmatrix}
\]

First, let the following be true:

\[
\begin{bmatrix}
3 & 4\\
5 & 6
\end{bmatrix}
\begin{bmatrix}
a & b\\
c & d
\end{bmatrix}
=
\begin{bmatrix}
1 & 0\\
0 & 1
\end{bmatrix}
\]

When multiplying this mystery matrix by our original matrix, the result
is

\[
\begin{bmatrix}
3a+4c & 3b+4d\\
5a+6c & 5b+6d
\end{bmatrix}
=
\begin{bmatrix}
1 & 0\\
0 & 1
\end{bmatrix}
\]

For two matrices to be equal, every element in the left must equal its
corresponding element on the right. So, for these two matrices to equal
each other:

\[
\begin{cases}
3a+4c=1\\
3b+4d=0\\
5a+6c=0\\
5b+6d=1
\end{cases}
\]

Solving this simple system we get the following result:

\[
\begin{cases}
a=−3\\
b=2\\
c=2.5\\
d=−1.5
\end{cases}
\]

Having solved for the four variables, the result is the inverse

\[
\begin{bmatrix}
−3 & 2\\
2.5 & −1.5
\end{bmatrix}
\]

The quick check to be sure it is correct is to multiply it by the
original matrix and see if the identify matrix results, this is left as
an exercise to the reader.

The \texttt{linalg.inv()} function can be used to find the inverse of a
matrix in \texttt{python}:

    \begin{tcolorbox}[breakable, size=fbox, boxrule=1pt, pad at break*=1mm,colback=cellbackground, colframe=cellborder]
\prompt{In}{incolor}{23}{\boxspacing}
\begin{Verbatim}[commandchars=\\\{\}]
\PY{k+kn}{from} \PY{n+nn}{numpy}\PY{n+nn}{.}\PY{n+nn}{linalg} \PY{k}{import} \PY{n}{inv}

\PY{n}{A} \PY{o}{=} \PY{n}{np}\PY{o}{.}\PY{n}{array}\PY{p}{(}\PY{p}{[}\PY{p}{[}\PY{l+m+mi}{3}\PY{p}{,}\PY{l+m+mi}{4}\PY{p}{]}\PY{p}{,}\PY{p}{[}\PY{l+m+mi}{5}\PY{p}{,}\PY{l+m+mi}{6}\PY{p}{]}\PY{p}{]}\PY{p}{)}
\PY{n+nb}{print} \PY{p}{(}\PY{n}{inv}\PY{p}{(}\PY{n}{A}\PY{p}{)}\PY{p}{)}
\end{Verbatim}
\end{tcolorbox}

    \begin{Verbatim}[commandchars=\\\{\}]
[[-3.   2. ]
 [ 2.5 -1.5]]
    \end{Verbatim}

    \hypertarget{solving-systems-of-equations-using-matrix-inverses}{%
\subsubsection{Solving Systems of Equations Using Matrix
Inverses}\label{solving-systems-of-equations-using-matrix-inverses}}

A system of equations can be readily solved using the concepts of the
inverse matrix and matrix multiplication.

\hypertarget{solve-the-following-system-of-linear-equations}{%
\paragraph{Solve the following system of linear
equations:}\label{solve-the-following-system-of-linear-equations}}

\[
\begin{cases}
x+2y−z=11\\
2x−y+3z=7\\
7x−3y−2z=2
\end{cases}
\]

Set up the three necesary matrices:

\[
A=
\begin{bmatrix}
1 & 2 & −1 \\ 
2 & −1 & 3 \\
7 & −3 & −2
\end{bmatrix}
;
B=
\begin{bmatrix}
11\\
7\\
2
\end{bmatrix}
;
X=
\begin{bmatrix}
x\\
y \\ 
z
\end{bmatrix}
\]

Since to solve this system we have to find the inverse matrix of \(A\)
and multiply it to \(B\) we have all the ingredients to do it in
\texttt{python}:

    \begin{tcolorbox}[breakable, size=fbox, boxrule=1pt, pad at break*=1mm,colback=cellbackground, colframe=cellborder]
\prompt{In}{incolor}{24}{\boxspacing}
\begin{Verbatim}[commandchars=\\\{\}]
\PY{n}{A} \PY{o}{=} \PY{n}{np}\PY{o}{.}\PY{n}{array}\PY{p}{(}\PY{p}{[}\PY{p}{[}\PY{l+m+mi}{1}\PY{p}{,}\PY{l+m+mi}{2}\PY{p}{,}\PY{o}{\PYZhy{}}\PY{l+m+mi}{1}\PY{p}{]}\PY{p}{,}\PY{p}{[}\PY{l+m+mi}{2}\PY{p}{,}\PY{o}{\PYZhy{}}\PY{l+m+mi}{1}\PY{p}{,}\PY{l+m+mi}{3}\PY{p}{]}\PY{p}{,}\PY{p}{[}\PY{l+m+mi}{7}\PY{p}{,}\PY{o}{\PYZhy{}}\PY{l+m+mi}{3}\PY{p}{,}\PY{o}{\PYZhy{}}\PY{l+m+mi}{2}\PY{p}{]}\PY{p}{]}\PY{p}{)}
\PY{n}{B} \PY{o}{=} \PY{n}{np}\PY{o}{.}\PY{n}{array}\PY{p}{(}\PY{p}{[}\PY{l+m+mi}{11}\PY{p}{,}\PY{l+m+mi}{7}\PY{p}{,}\PY{l+m+mi}{2}\PY{p}{]}\PY{p}{)}

\PY{n}{A\PYZus{}inv} \PY{o}{=} \PY{n}{inv}\PY{p}{(}\PY{n}{A}\PY{p}{)}
\PY{n}{sol} \PY{o}{=} \PY{n}{np}\PY{o}{.}\PY{n}{dot}\PY{p}{(}\PY{n}{A\PYZus{}inv}\PY{p}{,} \PY{n}{B}\PY{p}{)}

\PY{n+nb}{print} \PY{p}{(}\PY{n}{sol}\PY{p}{)}
\end{Verbatim}
\end{tcolorbox}

    \begin{Verbatim}[commandchars=\\\{\}]
[3. 5. 2.]
    \end{Verbatim}

    So the solution of the system is: \[
\begin{cases}
x=3\\
y=5\\
z=2
\end{cases}
\]

    \begin{tcolorbox}[breakable, size=fbox, boxrule=1pt, pad at break*=1mm,colback=cellbackground, colframe=cellborder]
\prompt{In}{incolor}{7}{\boxspacing}
\begin{Verbatim}[commandchars=\\\{\}]
\PY{k+kn}{import} \PY{n+nn}{random}

\PY{n}{random}\PY{o}{.}\PY{n}{seed}\PY{p}{(}\PY{l+m+mi}{1}\PY{p}{)}

\PY{n}{successes} \PY{o}{=} \PY{p}{\PYZob{}}\PY{l+s+s2}{\PYZdq{}}\PY{l+s+s2}{=0}\PY{l+s+s2}{\PYZdq{}}\PY{p}{:}\PY{l+m+mf}{0.0}\PY{p}{,} \PY{l+s+s2}{\PYZdq{}}\PY{l+s+s2}{=4}\PY{l+s+s2}{\PYZdq{}}\PY{p}{:}\PY{l+m+mf}{0.0}\PY{p}{,} \PY{l+s+s2}{\PYZdq{}}\PY{l+s+s2}{\PYZlt{}13}\PY{l+s+s2}{\PYZdq{}}\PY{p}{:}\PY{l+m+mf}{0.0}\PY{p}{\PYZcb{}}
\PY{n}{trials} \PY{o}{=} \PY{l+m+mi}{100000}
\PY{k}{for} \PY{n}{\PYZus{}} \PY{o+ow}{in} \PY{n+nb}{range}\PY{p}{(}\PY{n}{trials}\PY{p}{)}\PY{p}{:}
    \PY{n}{d1} \PY{o}{=} \PY{n}{random}\PY{o}{.}\PY{n}{randint}\PY{p}{(}\PY{l+m+mi}{1}\PY{p}{,} \PY{l+m+mi}{6}\PY{p}{)}
    \PY{n}{d2} \PY{o}{=} \PY{n}{random}\PY{o}{.}\PY{n}{randint}\PY{p}{(}\PY{l+m+mi}{1}\PY{p}{,} \PY{l+m+mi}{6}\PY{p}{)}
    \PY{k}{if} \PY{p}{(}\PY{n}{d1} \PY{o}{+} \PY{n}{d2}\PY{p}{)} \PY{o}{==} \PY{l+m+mi}{0}\PY{p}{:}
        \PY{n}{successes}\PY{p}{[}\PY{l+s+s2}{\PYZdq{}}\PY{l+s+s2}{=0}\PY{l+s+s2}{\PYZdq{}}\PY{p}{]} \PY{o}{+}\PY{o}{=} \PY{l+m+mf}{1.0}
    \PY{k}{if} \PY{p}{(}\PY{n}{d1} \PY{o}{+} \PY{n}{d2}\PY{p}{)} \PY{o}{==} \PY{l+m+mi}{4}\PY{p}{:}
        \PY{n}{successes}\PY{p}{[}\PY{l+s+s2}{\PYZdq{}}\PY{l+s+s2}{=4}\PY{l+s+s2}{\PYZdq{}}\PY{p}{]} \PY{o}{+}\PY{o}{=} \PY{l+m+mf}{1.0}
    \PY{k}{if} \PY{p}{(}\PY{n}{d1} \PY{o}{+} \PY{n}{d2}\PY{p}{)} \PY{o}{\PYZlt{}} \PY{l+m+mi}{13}\PY{p}{:}
        \PY{n}{successes}\PY{p}{[}\PY{l+s+s2}{\PYZdq{}}\PY{l+s+s2}{\PYZlt{}13}\PY{l+s+s2}{\PYZdq{}}\PY{p}{]} \PY{o}{+}\PY{o}{=} \PY{l+m+mf}{1.0}
    
\PY{k}{for} \PY{n}{k}\PY{p}{,}\PY{n}{v} \PY{o+ow}{in} \PY{n}{successes}\PY{o}{.}\PY{n}{items}\PY{p}{(}\PY{p}{)}\PY{p}{:}
    \PY{n+nb}{print} \PY{p}{(}\PY{l+s+s2}{\PYZdq{}}\PY{l+s+s2}{P(}\PY{l+s+si}{\PYZob{}\PYZcb{}}\PY{l+s+s2}{): }\PY{l+s+si}{\PYZob{}:.3f\PYZcb{}}\PY{l+s+s2}{\PYZdq{}}\PY{o}{.}\PY{n}{format}\PY{p}{(}\PY{n}{k}\PY{p}{,} \PY{n}{v}\PY{o}{/}\PY{n}{trials}\PY{p}{)}\PY{p}{)}
\end{Verbatim}
\end{tcolorbox}

    \begin{Verbatim}[commandchars=\\\{\}]
P(=0): 0.000
P(=4): 0.084
P(<13): 1.000
    \end{Verbatim}

    \begin{tcolorbox}[breakable, size=fbox, boxrule=1pt, pad at break*=1mm,colback=cellbackground, colframe=cellborder]
\prompt{In}{incolor}{17}{\boxspacing}
\begin{Verbatim}[commandchars=\\\{\}]
\PY{k+kn}{import} \PY{n+nn}{numpy} \PY{k}{as} \PY{n+nn}{np}
\PY{k+kn}{from} \PY{n+nn}{scipy}\PY{n+nn}{.}\PY{n+nn}{stats} \PY{k}{import} \PY{n}{norm}

\PY{n}{samples} \PY{o}{=} \PY{p}{[}\PY{l+m+mf}{1.}\PY{p}{,}\PY{l+m+mf}{2.}\PY{p}{,}\PY{l+m+mf}{3.}\PY{p}{,}\PY{l+m+mf}{4.}\PY{p}{,}\PY{l+m+mf}{4.}\PY{p}{,}\PY{l+m+mf}{4.}\PY{p}{,}\PY{l+m+mf}{5.}\PY{p}{,}\PY{l+m+mf}{5.}\PY{p}{,}\PY{l+m+mf}{5.}\PY{p}{,}\PY{l+m+mf}{5.}\PY{p}{,}\PY{l+m+mf}{4.}\PY{p}{,}\PY{l+m+mf}{4.}\PY{p}{,}\PY{l+m+mf}{4.}\PY{p}{,}\PY{l+m+mf}{6.}\PY{p}{,}\PY{l+m+mf}{7.}\PY{p}{,}\PY{l+m+mf}{8.}\PY{p}{]}

\PY{n}{alpha} \PY{o}{=} \PY{l+m+mf}{0.95}
\PY{n}{a} \PY{o}{=} \PY{n}{np}\PY{o}{.}\PY{n}{array}\PY{p}{(}\PY{n}{samples}\PY{p}{)}

\PY{n}{A} \PY{o}{=} \PY{n}{norm}\PY{o}{.}\PY{n}{ppf}\PY{p}{(}\PY{p}{(}\PY{l+m+mi}{1} \PY{o}{+} \PY{n}{alpha}\PY{p}{)}\PY{o}{/}\PY{l+m+mi}{2}\PY{p}{)}
\PY{n}{m}\PY{p}{,} \PY{n}{se} \PY{o}{=} \PY{n}{np}\PY{o}{.}\PY{n}{mean}\PY{p}{(}\PY{n}{a}\PY{p}{)}\PY{p}{,} \PY{n}{np}\PY{o}{.}\PY{n}{std}\PY{p}{(}\PY{n}{a}\PY{p}{)}
\PY{n}{h} \PY{o}{=} \PY{n}{A}\PY{o}{*}\PY{n}{se}\PY{o}{/}\PY{n}{np}\PY{o}{.}\PY{n}{sqrt}\PY{p}{(}\PY{n+nb}{len}\PY{p}{(}\PY{n}{samples}\PY{p}{)}\PY{p}{)}

\PY{n+nb}{print} \PY{p}{(}\PY{l+s+s2}{\PYZdq{}}\PY{l+s+si}{\PYZob{}:.0f\PYZcb{}}\PY{l+s+si}{\PYZpc{} c}\PY{l+s+s2}{onfidence interval: }\PY{l+s+si}{\PYZob{}\PYZcb{}}\PY{l+s+s2}{ +\PYZhy{} }\PY{l+s+si}{\PYZob{}\PYZcb{}}\PY{l+s+s2}{\PYZdq{}}\PY{o}{.}\PY{n}{format}\PY{p}{(}\PY{n}{alpha}\PY{o}{*}\PY{l+m+mi}{100}\PY{p}{,} \PY{n}{m}\PY{p}{,} \PY{n}{h}\PY{p}{)}\PY{p}{)}
\end{Verbatim}
\end{tcolorbox}

    \begin{Verbatim}[commandchars=\\\{\}]
95\% confidence interval: 4.4375 +- 0.8119808363806419
    \end{Verbatim}

    \begin{tcolorbox}[breakable, size=fbox, boxrule=1pt, pad at break*=1mm,colback=cellbackground, colframe=cellborder]
\prompt{In}{incolor}{54}{\boxspacing}
\begin{Verbatim}[commandchars=\\\{\}]
\PY{k+kn}{import} \PY{n+nn}{random}

\PY{n}{flasks} \PY{o}{=} \PY{p}{[}\PY{l+s+s2}{\PYZdq{}}\PY{l+s+s2}{C}\PY{l+s+s2}{\PYZdq{}}\PY{p}{]}\PY{o}{*}\PY{l+m+mi}{54} \PY{o}{+} \PY{p}{[}\PY{l+s+s2}{\PYZdq{}}\PY{l+s+s2}{U}\PY{l+s+s2}{\PYZdq{}}\PY{p}{]} \PY{o}{*} \PY{l+m+mi}{6}

\PY{n}{random}\PY{o}{.}\PY{n}{seed}\PY{p}{(}\PY{l+m+mi}{1}\PY{p}{)}
\PY{n}{trials} \PY{o}{=} \PY{l+m+mi}{1000}
\PY{n}{success} \PY{o}{=} \PY{l+m+mf}{0.}
\PY{k}{for} \PY{n}{\PYZus{}} \PY{o+ow}{in} \PY{n+nb}{range}\PY{p}{(}\PY{n}{trials}\PY{p}{)}\PY{p}{:}
    \PY{n}{draw} \PY{o}{=} \PY{n}{random}\PY{o}{.}\PY{n}{sample}\PY{p}{(}\PY{n}{flasks}\PY{p}{,} \PY{l+m+mi}{5}\PY{p}{)}
    \PY{k}{if} \PY{n}{draw}\PY{o}{.}\PY{n}{count}\PY{p}{(}\PY{l+s+s2}{\PYZdq{}}\PY{l+s+s2}{C}\PY{l+s+s2}{\PYZdq{}}\PY{p}{)} \PY{o}{==} \PY{l+m+mi}{3}\PY{p}{:}
        \PY{n}{success} \PY{o}{+}\PY{o}{=} \PY{l+m+mf}{1.}
        
\PY{n+nb}{print} \PY{p}{(}\PY{l+s+s2}{\PYZdq{}}\PY{l+s+s2}{Probbility: }\PY{l+s+si}{\PYZob{}:.1f\PYZcb{}}\PY{l+s+s2}{\PYZpc{}}\PY{l+s+s2}{\PYZdq{}}\PY{o}{.}\PY{n}{format}\PY{p}{(}\PY{n}{success}\PY{o}{/}\PY{n+nb}{float}\PY{p}{(}\PY{n}{trials}\PY{p}{)}\PY{o}{*}\PY{l+m+mi}{100}\PY{p}{)}\PY{p}{)}
\end{Verbatim}
\end{tcolorbox}

    \begin{Verbatim}[commandchars=\\\{\}]
Probbility: 6.7\%
    \end{Verbatim}

    \begin{tcolorbox}[breakable, size=fbox, boxrule=1pt, pad at break*=1mm,colback=cellbackground, colframe=cellborder]
\prompt{In}{incolor}{56}{\boxspacing}
\begin{Verbatim}[commandchars=\\\{\}]
\PY{k+kn}{from} \PY{n+nn}{scipy}\PY{n+nn}{.}\PY{n+nn}{stats} \PY{k}{import} \PY{n}{norm}
\PY{k+kn}{import} \PY{n+nn}{numpy} \PY{k}{as} \PY{n+nn}{np}
\PY{n}{temperatures} \PY{o}{=}\PY{p}{[}\PY{l+m+mf}{65.98}\PY{p}{,} \PY{l+m+mf}{68.43}\PY{p}{,} \PY{l+m+mf}{67.53}\PY{p}{,} \PY{l+m+mf}{66.27}\PY{p}{,} \PY{l+m+mf}{67.12}\PY{p}{,} \PY{l+m+mf}{68.54}\PY{p}{,} \PY{l+m+mf}{66.20}\PY{p}{,} \PY{l+m+mf}{66.96}\PY{p}{,} \PY{l+m+mf}{66.31}\PY{p}{,} \PY{l+m+mf}{66.09}\PY{p}{,}
\PY{l+m+mf}{66.85}\PY{p}{,} \PY{l+m+mf}{65.70}\PY{p}{,} \PY{l+m+mf}{65.57}\PY{p}{,} \PY{l+m+mf}{64.35}\PY{p}{,} \PY{l+m+mf}{67.35}\PY{p}{,} \PY{l+m+mf}{64.08}\PY{p}{,} \PY{l+m+mf}{64.60}\PY{p}{,} \PY{l+m+mf}{65.25}\PY{p}{,} \PY{l+m+mf}{65.84}\PY{p}{,} \PY{l+m+mf}{65.93}\PY{p}{,} \PY{l+m+mf}{64.18}\PY{p}{,} \PY{l+m+mf}{62.89}\PY{p}{,} \PY{l+m+mf}{64.54}\PY{p}{,} \PY{l+m+mf}{61.77}\PY{p}{,} \PY{l+m+mf}{63.70}\PY{p}{,} \PY{l+m+mf}{66.34}\PY{p}{,} \PY{l+m+mf}{66.45}\PY{p}{,} \PY{l+m+mf}{66.16}\PY{p}{,} \PY{l+m+mf}{65.89}\PY{p}{,} \PY{l+m+mf}{64.09}\PY{p}{,} \PY{l+m+mf}{63.66}\PY{p}{,} \PY{l+m+mf}{64.53}\PY{p}{,} \PY{l+m+mf}{65.35}\PY{p}{,} \PY{l+m+mf}{67.37}\PY{p}{,} \PY{l+m+mf}{65.75}\PY{p}{,} \PY{l+m+mf}{63.41}\PY{p}{,} \PY{l+m+mf}{63.27}\PY{p}{,} \PY{l+m+mf}{64.98}\PY{p}{,} \PY{l+m+mf}{61.92}\PY{p}{,} \PY{l+m+mf}{64.09}\PY{p}{,} \PY{l+m+mf}{63.50}\PY{p}{,} \PY{l+m+mf}{64.42}\PY{p}{,} \PY{l+m+mf}{66.25}\PY{p}{,} \PY{l+m+mf}{65.95}\PY{p}{,} \PY{l+m+mf}{64.96}\PY{p}{,} \PY{l+m+mf}{62.73}\PY{p}{,}
               \PY{l+m+mf}{66.49}\PY{p}{,} \PY{l+m+mf}{66.38}\PY{p}{,} \PY{l+m+mf}{65.21}\PY{p}{,} \PY{l+m+mf}{66.43}\PY{p}{,} \PY{l+m+mf}{63.30}\PY{p}{,} \PY{l+m+mf}{67.37}\PY{p}{,} \PY{l+m+mf}{65.66}\PY{p}{,} \PY{l+m+mf}{64.96}\PY{p}{,} \PY{l+m+mf}{68.99}\PY{p}{,} \PY{l+m+mf}{66.42}\PY{p}{,} \PY{l+m+mf}{63.46}\PY{p}{,} \PY{l+m+mf}{64.69}\PY{p}{,} \PY{l+m+mf}{65.55}\PY{p}{,} \PY{l+m+mf}{63.30}\PY{p}{,} \PY{l+m+mf}{64.67}\PY{p}{,} \PY{l+m+mf}{64.90}\PY{p}{,} \PY{l+m+mf}{63.70}\PY{p}{,} \PY{l+m+mf}{62.73}\PY{p}{,} \PY{l+m+mf}{63.14}\PY{p}{,} \PY{l+m+mf}{65.59}\PY{p}{,} \PY{l+m+mf}{64.04}\PY{p}{,} \PY{l+m+mf}{64.92}\PY{p}{,} \PY{l+m+mf}{66.24}\PY{p}{,} \PY{l+m+mf}{66.09}\PY{p}{,} \PY{l+m+mf}{62.31}\PY{p}{,} \PY{l+m+mf}{64.40}\PY{p}{,} \PY{l+m+mf}{64.44}\PY{p}{,} \PY{l+m+mf}{64.33}\PY{p}{,} \PY{l+m+mf}{64.58}\PY{p}{,} \PY{l+m+mf}{65.55}\PY{p}{,} \PY{l+m+mf}{63.90}\PY{p}{,} \PY{l+m+mf}{64.06}\PY{p}{,} \PY{l+m+mf}{64.18}\PY{p}{,} \PY{l+m+mf}{63.41}\PY{p}{,} \PY{l+m+mf}{66.45}\PY{p}{,} \PY{l+m+mf}{64.74}\PY{p}{,} \PY{l+m+mf}{65.43}\PY{p}{,} \PY{l+m+mf}{64.18}\PY{p}{,} \PY{l+m+mf}{65.10}\PY{p}{,} \PY{l+m+mf}{65.88}\PY{p}{,} \PY{l+m+mf}{63.50}\PY{p}{,} \PY{l+m+mf}{63.73}\PY{p}{,} \PY{l+m+mf}{66.07}\PY{p}{,} \PY{l+m+mf}{66.54}\PY{p}{,} \PY{l+m+mf}{64.54}\PY{p}{,} \PY{l+m+mf}{64.99}\PY{p}{,} \PY{l+m+mf}{65.26}\PY{p}{,} \PY{l+m+mf}{64.33}\PY{p}{,} \PY{l+m+mf}{66.63}\PY{p}{,} \PY{l+m+mf}{65.70}\PY{p}{,} \PY{l+m+mf}{66.36}\PY{p}{,} \PY{l+m+mf}{64.80}\PY{p}{,} \PY{l+m+mf}{63.84}\PY{p}{,} \PY{l+m+mf}{67.73}\PY{p}{,} \PY{l+m+mf}{64.69}\PY{p}{,} \PY{l+m+mf}{68.13}\PY{p}{,} \PY{l+m+mf}{64.11}\PY{p}{,} \PY{l+m+mf}{66.67}\PY{p}{,} \PY{l+m+mf}{62.74}\PY{p}{,} \PY{l+m+mf}{65.07}\PY{p}{,} \PY{l+m+mf}{67.28}\PY{p}{,} \PY{l+m+mf}{66.54}\PY{p}{,} \PY{l+m+mf}{65.44}\PY{p}{,} \PY{l+m+mf}{65.64}\PY{p}{,} \PY{l+m+mf}{64.58}\PY{p}{,} \PY{l+m+mf}{63.64}\PY{p}{,} \PY{l+m+mf}{62.73}\PY{p}{,} \PY{l+m+mf}{66.09}\PY{p}{,} \PY{l+m+mf}{65.68}\PY{p}{,} \PY{l+m+mf}{64.40}\PY{p}{,} \PY{l+m+mf}{65.59}\PY{p}{,} \PY{l+m+mf}{64.22}\PY{p}{,} \PY{l+m+mf}{62.76}\PY{p}{,} \PY{l+m+mf}{63.25}\PY{p}{,} \PY{l+m+mf}{65.14}\PY{p}{,} \PY{l+m+mf}{64.80}\PY{p}{,} \PY{l+m+mf}{65.46}\PY{p}{,} \PY{l+m+mf}{66.40}\PY{p}{,} \PY{l+m+mf}{62.94}\PY{p}{,} \PY{l+m+mf}{65.57}\PY{p}{]}
\PY{n}{temperatures} \PY{o}{=} \PY{n}{np}\PY{o}{.}\PY{n}{array}\PY{p}{(}\PY{n}{temperatures}\PY{p}{)}

\PY{n}{alpha} \PY{o}{=} \PY{l+m+mf}{0.99}

\PY{n}{A} \PY{o}{=} \PY{n}{norm}\PY{o}{.}\PY{n}{ppf}\PY{p}{(}\PY{p}{(}\PY{l+m+mi}{1} \PY{o}{+} \PY{n}{alpha}\PY{p}{)}\PY{o}{/}\PY{l+m+mi}{2}\PY{p}{)}
\PY{n}{m}\PY{p}{,} \PY{n}{se} \PY{o}{=} \PY{n}{np}\PY{o}{.}\PY{n}{mean}\PY{p}{(}\PY{n}{temperatures}\PY{p}{)}\PY{p}{,} \PY{n}{np}\PY{o}{.}\PY{n}{std}\PY{p}{(}\PY{n}{temperatures}\PY{p}{)}
\PY{n}{h} \PY{o}{=} \PY{n}{A}\PY{o}{*}\PY{n}{se}\PY{o}{/}\PY{n}{np}\PY{o}{.}\PY{n}{sqrt}\PY{p}{(}\PY{n+nb}{len}\PY{p}{(}\PY{n}{temperatures}\PY{p}{)}\PY{p}{)}

\PY{n+nb}{print} \PY{p}{(}\PY{l+s+s2}{\PYZdq{}}\PY{l+s+s2}{Avg temperature in Septepmber (US): }\PY{l+s+si}{\PYZob{}\PYZcb{}}\PY{l+s+s2}{\PYZdq{}}\PY{o}{.}\PY{n}{format}\PY{p}{(}\PY{n}{m}\PY{p}{)}\PY{p}{)} 
\PY{n+nb}{print} \PY{p}{(}\PY{l+s+s2}{\PYZdq{}}\PY{l+s+si}{\PYZob{}:.0f\PYZcb{}}\PY{l+s+si}{\PYZpc{} c}\PY{l+s+s2}{onfidence interval: +\PYZhy{} }\PY{l+s+si}{\PYZob{}\PYZcb{}}\PY{l+s+s2}{\PYZdq{}}\PY{o}{.}\PY{n}{format}\PY{p}{(}\PY{n}{alpha}\PY{o}{*}\PY{l+m+mi}{100}\PY{p}{,} \PY{n}{h}\PY{p}{)}\PY{p}{)}
\end{Verbatim}
\end{tcolorbox}

    \begin{Verbatim}[commandchars=\\\{\}]
Avg temperature in Septepmber (US): 65.10992063492064
99\% confidence interval: +- 0.33013045424273185
    \end{Verbatim}

    \begin{tcolorbox}[breakable, size=fbox, boxrule=1pt, pad at break*=1mm,colback=cellbackground, colframe=cellborder]
\prompt{In}{incolor}{63}{\boxspacing}
\begin{Verbatim}[commandchars=\\\{\}]
\PY{k+kn}{from} \PY{n+nn}{scipy}\PY{n+nn}{.}\PY{n+nn}{stats} \PY{k}{import} \PY{n}{binom}

\PY{n}{b} \PY{o}{=} \PY{n}{binom}\PY{p}{(}\PY{l+m+mi}{100}\PY{p}{,} \PY{l+m+mf}{0.02}\PY{p}{)}
\PY{n+nb}{print}\PY{p}{(}\PY{l+s+s2}{\PYZdq{}}\PY{l+s+s2}{P(\PYZgt{}=1) : }\PY{l+s+si}{\PYZob{}\PYZcb{}}\PY{l+s+s2}{\PYZdq{}}\PY{o}{.}\PY{n}{format}\PY{p}{(}\PY{l+m+mi}{1} \PY{o}{\PYZhy{}} \PY{n}{b}\PY{o}{.}\PY{n}{cdf}\PY{p}{(}\PY{l+m+mi}{0}\PY{p}{)}\PY{p}{)}\PY{p}{)}
\PY{n+nb}{print}\PY{p}{(}\PY{l+s+s2}{\PYZdq{}}\PY{l+s+s2}{P(\PYZgt{}=10): }\PY{l+s+si}{\PYZob{}\PYZcb{}}\PY{l+s+s2}{\PYZdq{}}\PY{o}{.}\PY{n}{format}\PY{p}{(}\PY{l+m+mi}{1} \PY{o}{\PYZhy{}} \PY{n}{b}\PY{o}{.}\PY{n}{cdf}\PY{p}{(}\PY{l+m+mi}{9}\PY{p}{)}\PY{p}{)}\PY{p}{)}
\end{Verbatim}
\end{tcolorbox}

    \begin{Verbatim}[commandchars=\\\{\}]
P(>=1) : 0.8673804441052471
P(>=10): 3.441680604299169e-05
    \end{Verbatim}

    \begin{tcolorbox}[breakable, size=fbox, boxrule=1pt, pad at break*=1mm,colback=cellbackground, colframe=cellborder]
\prompt{In}{incolor}{ }{\boxspacing}
\begin{Verbatim}[commandchars=\\\{\}]

\end{Verbatim}
\end{tcolorbox}


    % Add a bibliography block to the postdoc
    
    
    
\end{document}

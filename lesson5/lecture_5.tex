
% Default to the notebook output style

    


% Inherit from the specified cell style.




    
\documentclass[11pt]{article}

    
    
    \usepackage[T1]{fontenc}
    % Nicer default font (+ math font) than Computer Modern for most use cases
    \usepackage{mathpazo}

    % Basic figure setup, for now with no caption control since it's done
    % automatically by Pandoc (which extracts ![](path) syntax from Markdown).
    \usepackage{graphicx}
    % We will generate all images so they have a width \maxwidth. This means
    % that they will get their normal width if they fit onto the page, but
    % are scaled down if they would overflow the margins.
    \makeatletter
    \def\maxwidth{\ifdim\Gin@nat@width>\linewidth\linewidth
    \else\Gin@nat@width\fi}
    \makeatother
    \let\Oldincludegraphics\includegraphics
    % Set max figure width to be 80% of text width, for now hardcoded.
    \renewcommand{\includegraphics}[1]{\Oldincludegraphics[width=.8\maxwidth]{#1}}
    % Ensure that by default, figures have no caption (until we provide a
    % proper Figure object with a Caption API and a way to capture that
    % in the conversion process - todo).
    \usepackage{caption}
    \DeclareCaptionLabelFormat{nolabel}{}
    \captionsetup{labelformat=nolabel}

    \usepackage{adjustbox} % Used to constrain images to a maximum size 
    \usepackage{xcolor} % Allow colors to be defined
    \usepackage{enumerate} % Needed for markdown enumerations to work
    \usepackage{geometry} % Used to adjust the document margins
    \usepackage{amsmath} % Equations
    \usepackage{amssymb} % Equations
    \usepackage{textcomp} % defines textquotesingle
    % Hack from http://tex.stackexchange.com/a/47451/13684:
    \AtBeginDocument{%
        \def\PYZsq{\textquotesingle}% Upright quotes in Pygmentized code
    }
    \usepackage{upquote} % Upright quotes for verbatim code
    \usepackage{eurosym} % defines \euro
    \usepackage[mathletters]{ucs} % Extended unicode (utf-8) support
    \usepackage[utf8x]{inputenc} % Allow utf-8 characters in the tex document
    \usepackage{fancyvrb} % verbatim replacement that allows latex
    \usepackage{grffile} % extends the file name processing of package graphics 
                         % to support a larger range 
    % The hyperref package gives us a pdf with properly built
    % internal navigation ('pdf bookmarks' for the table of contents,
    % internal cross-reference links, web links for URLs, etc.)
    \usepackage{hyperref}
    \usepackage{longtable} % longtable support required by pandoc >1.10
    \usepackage{booktabs}  % table support for pandoc > 1.12.2
    \usepackage[inline]{enumitem} % IRkernel/repr support (it uses the enumerate* environment)
    \usepackage[normalem]{ulem} % ulem is needed to support strikethroughs (\sout)
                                % normalem makes italics be italics, not underlines
    \usepackage{mathrsfs}
    

    
    
    % Colors for the hyperref package
    \definecolor{urlcolor}{rgb}{0,.145,.698}
    \definecolor{linkcolor}{rgb}{.71,0.21,0.01}
    \definecolor{citecolor}{rgb}{.12,.54,.11}

    % ANSI colors
    \definecolor{ansi-black}{HTML}{3E424D}
    \definecolor{ansi-black-intense}{HTML}{282C36}
    \definecolor{ansi-red}{HTML}{E75C58}
    \definecolor{ansi-red-intense}{HTML}{B22B31}
    \definecolor{ansi-green}{HTML}{00A250}
    \definecolor{ansi-green-intense}{HTML}{007427}
    \definecolor{ansi-yellow}{HTML}{DDB62B}
    \definecolor{ansi-yellow-intense}{HTML}{B27D12}
    \definecolor{ansi-blue}{HTML}{208FFB}
    \definecolor{ansi-blue-intense}{HTML}{0065CA}
    \definecolor{ansi-magenta}{HTML}{D160C4}
    \definecolor{ansi-magenta-intense}{HTML}{A03196}
    \definecolor{ansi-cyan}{HTML}{60C6C8}
    \definecolor{ansi-cyan-intense}{HTML}{258F8F}
    \definecolor{ansi-white}{HTML}{C5C1B4}
    \definecolor{ansi-white-intense}{HTML}{A1A6B2}
    \definecolor{ansi-default-inverse-fg}{HTML}{FFFFFF}
    \definecolor{ansi-default-inverse-bg}{HTML}{000000}

    % commands and environments needed by pandoc snippets
    % extracted from the output of `pandoc -s`
    \providecommand{\tightlist}{%
      \setlength{\itemsep}{0pt}\setlength{\parskip}{0pt}}
    \DefineVerbatimEnvironment{Highlighting}{Verbatim}{commandchars=\\\{\}}
    % Add ',fontsize=\small' for more characters per line
    \newenvironment{Shaded}{}{}
    \newcommand{\KeywordTok}[1]{\textcolor[rgb]{0.00,0.44,0.13}{\textbf{{#1}}}}
    \newcommand{\DataTypeTok}[1]{\textcolor[rgb]{0.56,0.13,0.00}{{#1}}}
    \newcommand{\DecValTok}[1]{\textcolor[rgb]{0.25,0.63,0.44}{{#1}}}
    \newcommand{\BaseNTok}[1]{\textcolor[rgb]{0.25,0.63,0.44}{{#1}}}
    \newcommand{\FloatTok}[1]{\textcolor[rgb]{0.25,0.63,0.44}{{#1}}}
    \newcommand{\CharTok}[1]{\textcolor[rgb]{0.25,0.44,0.63}{{#1}}}
    \newcommand{\StringTok}[1]{\textcolor[rgb]{0.25,0.44,0.63}{{#1}}}
    \newcommand{\CommentTok}[1]{\textcolor[rgb]{0.38,0.63,0.69}{\textit{{#1}}}}
    \newcommand{\OtherTok}[1]{\textcolor[rgb]{0.00,0.44,0.13}{{#1}}}
    \newcommand{\AlertTok}[1]{\textcolor[rgb]{1.00,0.00,0.00}{\textbf{{#1}}}}
    \newcommand{\FunctionTok}[1]{\textcolor[rgb]{0.02,0.16,0.49}{{#1}}}
    \newcommand{\RegionMarkerTok}[1]{{#1}}
    \newcommand{\ErrorTok}[1]{\textcolor[rgb]{1.00,0.00,0.00}{\textbf{{#1}}}}
    \newcommand{\NormalTok}[1]{{#1}}
    
    % Additional commands for more recent versions of Pandoc
    \newcommand{\ConstantTok}[1]{\textcolor[rgb]{0.53,0.00,0.00}{{#1}}}
    \newcommand{\SpecialCharTok}[1]{\textcolor[rgb]{0.25,0.44,0.63}{{#1}}}
    \newcommand{\VerbatimStringTok}[1]{\textcolor[rgb]{0.25,0.44,0.63}{{#1}}}
    \newcommand{\SpecialStringTok}[1]{\textcolor[rgb]{0.73,0.40,0.53}{{#1}}}
    \newcommand{\ImportTok}[1]{{#1}}
    \newcommand{\DocumentationTok}[1]{\textcolor[rgb]{0.73,0.13,0.13}{\textit{{#1}}}}
    \newcommand{\AnnotationTok}[1]{\textcolor[rgb]{0.38,0.63,0.69}{\textbf{\textit{{#1}}}}}
    \newcommand{\CommentVarTok}[1]{\textcolor[rgb]{0.38,0.63,0.69}{\textbf{\textit{{#1}}}}}
    \newcommand{\VariableTok}[1]{\textcolor[rgb]{0.10,0.09,0.49}{{#1}}}
    \newcommand{\ControlFlowTok}[1]{\textcolor[rgb]{0.00,0.44,0.13}{\textbf{{#1}}}}
    \newcommand{\OperatorTok}[1]{\textcolor[rgb]{0.40,0.40,0.40}{{#1}}}
    \newcommand{\BuiltInTok}[1]{{#1}}
    \newcommand{\ExtensionTok}[1]{{#1}}
    \newcommand{\PreprocessorTok}[1]{\textcolor[rgb]{0.74,0.48,0.00}{{#1}}}
    \newcommand{\AttributeTok}[1]{\textcolor[rgb]{0.49,0.56,0.16}{{#1}}}
    \newcommand{\InformationTok}[1]{\textcolor[rgb]{0.38,0.63,0.69}{\textbf{\textit{{#1}}}}}
    \newcommand{\WarningTok}[1]{\textcolor[rgb]{0.38,0.63,0.69}{\textbf{\textit{{#1}}}}}
    
    
    % Define a nice break command that doesn't care if a line doesn't already
    % exist.
    \def\br{\hspace*{\fill} \\* }
    % Math Jax compatibility definitions
    \def\gt{>}
    \def\lt{<}
    \let\Oldtex\TeX
    \let\Oldlatex\LaTeX
    \renewcommand{\TeX}{\textrm{\Oldtex}}
    \renewcommand{\LaTeX}{\textrm{\Oldlatex}}
    % Document parameters
    % Document title
    \title{Swaps and Modules - Practical Lesson 5}
    \author{Matteo Sani \\ \href{mailto:matteosan1@gmail.com}{matteosan1@gmail.com}}
    
    
    
    

    % Pygments definitions
    
\makeatletter
\def\PY@reset{\let\PY@it=\relax \let\PY@bf=\relax%
    \let\PY@ul=\relax \let\PY@tc=\relax%
    \let\PY@bc=\relax \let\PY@ff=\relax}
\def\PY@tok#1{\csname PY@tok@#1\endcsname}
\def\PY@toks#1+{\ifx\relax#1\empty\else%
    \PY@tok{#1}\expandafter\PY@toks\fi}
\def\PY@do#1{\PY@bc{\PY@tc{\PY@ul{%
    \PY@it{\PY@bf{\PY@ff{#1}}}}}}}
\def\PY#1#2{\PY@reset\PY@toks#1+\relax+\PY@do{#2}}

\expandafter\def\csname PY@tok@w\endcsname{\def\PY@tc##1{\textcolor[rgb]{0.73,0.73,0.73}{##1}}}
\expandafter\def\csname PY@tok@c\endcsname{\let\PY@it=\textit\def\PY@tc##1{\textcolor[rgb]{0.25,0.50,0.50}{##1}}}
\expandafter\def\csname PY@tok@cp\endcsname{\def\PY@tc##1{\textcolor[rgb]{0.74,0.48,0.00}{##1}}}
\expandafter\def\csname PY@tok@k\endcsname{\let\PY@bf=\textbf\def\PY@tc##1{\textcolor[rgb]{0.00,0.50,0.00}{##1}}}
\expandafter\def\csname PY@tok@kp\endcsname{\def\PY@tc##1{\textcolor[rgb]{0.00,0.50,0.00}{##1}}}
\expandafter\def\csname PY@tok@kt\endcsname{\def\PY@tc##1{\textcolor[rgb]{0.69,0.00,0.25}{##1}}}
\expandafter\def\csname PY@tok@o\endcsname{\def\PY@tc##1{\textcolor[rgb]{0.40,0.40,0.40}{##1}}}
\expandafter\def\csname PY@tok@ow\endcsname{\let\PY@bf=\textbf\def\PY@tc##1{\textcolor[rgb]{0.67,0.13,1.00}{##1}}}
\expandafter\def\csname PY@tok@nb\endcsname{\def\PY@tc##1{\textcolor[rgb]{0.00,0.50,0.00}{##1}}}
\expandafter\def\csname PY@tok@nf\endcsname{\def\PY@tc##1{\textcolor[rgb]{0.00,0.00,1.00}{##1}}}
\expandafter\def\csname PY@tok@nc\endcsname{\let\PY@bf=\textbf\def\PY@tc##1{\textcolor[rgb]{0.00,0.00,1.00}{##1}}}
\expandafter\def\csname PY@tok@nn\endcsname{\let\PY@bf=\textbf\def\PY@tc##1{\textcolor[rgb]{0.00,0.00,1.00}{##1}}}
\expandafter\def\csname PY@tok@ne\endcsname{\let\PY@bf=\textbf\def\PY@tc##1{\textcolor[rgb]{0.82,0.25,0.23}{##1}}}
\expandafter\def\csname PY@tok@nv\endcsname{\def\PY@tc##1{\textcolor[rgb]{0.10,0.09,0.49}{##1}}}
\expandafter\def\csname PY@tok@no\endcsname{\def\PY@tc##1{\textcolor[rgb]{0.53,0.00,0.00}{##1}}}
\expandafter\def\csname PY@tok@nl\endcsname{\def\PY@tc##1{\textcolor[rgb]{0.63,0.63,0.00}{##1}}}
\expandafter\def\csname PY@tok@ni\endcsname{\let\PY@bf=\textbf\def\PY@tc##1{\textcolor[rgb]{0.60,0.60,0.60}{##1}}}
\expandafter\def\csname PY@tok@na\endcsname{\def\PY@tc##1{\textcolor[rgb]{0.49,0.56,0.16}{##1}}}
\expandafter\def\csname PY@tok@nt\endcsname{\let\PY@bf=\textbf\def\PY@tc##1{\textcolor[rgb]{0.00,0.50,0.00}{##1}}}
\expandafter\def\csname PY@tok@nd\endcsname{\def\PY@tc##1{\textcolor[rgb]{0.67,0.13,1.00}{##1}}}
\expandafter\def\csname PY@tok@s\endcsname{\def\PY@tc##1{\textcolor[rgb]{0.73,0.13,0.13}{##1}}}
\expandafter\def\csname PY@tok@sd\endcsname{\let\PY@it=\textit\def\PY@tc##1{\textcolor[rgb]{0.73,0.13,0.13}{##1}}}
\expandafter\def\csname PY@tok@si\endcsname{\let\PY@bf=\textbf\def\PY@tc##1{\textcolor[rgb]{0.73,0.40,0.53}{##1}}}
\expandafter\def\csname PY@tok@se\endcsname{\let\PY@bf=\textbf\def\PY@tc##1{\textcolor[rgb]{0.73,0.40,0.13}{##1}}}
\expandafter\def\csname PY@tok@sr\endcsname{\def\PY@tc##1{\textcolor[rgb]{0.73,0.40,0.53}{##1}}}
\expandafter\def\csname PY@tok@ss\endcsname{\def\PY@tc##1{\textcolor[rgb]{0.10,0.09,0.49}{##1}}}
\expandafter\def\csname PY@tok@sx\endcsname{\def\PY@tc##1{\textcolor[rgb]{0.00,0.50,0.00}{##1}}}
\expandafter\def\csname PY@tok@m\endcsname{\def\PY@tc##1{\textcolor[rgb]{0.40,0.40,0.40}{##1}}}
\expandafter\def\csname PY@tok@gh\endcsname{\let\PY@bf=\textbf\def\PY@tc##1{\textcolor[rgb]{0.00,0.00,0.50}{##1}}}
\expandafter\def\csname PY@tok@gu\endcsname{\let\PY@bf=\textbf\def\PY@tc##1{\textcolor[rgb]{0.50,0.00,0.50}{##1}}}
\expandafter\def\csname PY@tok@gd\endcsname{\def\PY@tc##1{\textcolor[rgb]{0.63,0.00,0.00}{##1}}}
\expandafter\def\csname PY@tok@gi\endcsname{\def\PY@tc##1{\textcolor[rgb]{0.00,0.63,0.00}{##1}}}
\expandafter\def\csname PY@tok@gr\endcsname{\def\PY@tc##1{\textcolor[rgb]{1.00,0.00,0.00}{##1}}}
\expandafter\def\csname PY@tok@ge\endcsname{\let\PY@it=\textit}
\expandafter\def\csname PY@tok@gs\endcsname{\let\PY@bf=\textbf}
\expandafter\def\csname PY@tok@gp\endcsname{\let\PY@bf=\textbf\def\PY@tc##1{\textcolor[rgb]{0.00,0.00,0.50}{##1}}}
\expandafter\def\csname PY@tok@go\endcsname{\def\PY@tc##1{\textcolor[rgb]{0.53,0.53,0.53}{##1}}}
\expandafter\def\csname PY@tok@gt\endcsname{\def\PY@tc##1{\textcolor[rgb]{0.00,0.27,0.87}{##1}}}
\expandafter\def\csname PY@tok@err\endcsname{\def\PY@bc##1{\setlength{\fboxsep}{0pt}\fcolorbox[rgb]{1.00,0.00,0.00}{1,1,1}{\strut ##1}}}
\expandafter\def\csname PY@tok@kc\endcsname{\let\PY@bf=\textbf\def\PY@tc##1{\textcolor[rgb]{0.00,0.50,0.00}{##1}}}
\expandafter\def\csname PY@tok@kd\endcsname{\let\PY@bf=\textbf\def\PY@tc##1{\textcolor[rgb]{0.00,0.50,0.00}{##1}}}
\expandafter\def\csname PY@tok@kn\endcsname{\let\PY@bf=\textbf\def\PY@tc##1{\textcolor[rgb]{0.00,0.50,0.00}{##1}}}
\expandafter\def\csname PY@tok@kr\endcsname{\let\PY@bf=\textbf\def\PY@tc##1{\textcolor[rgb]{0.00,0.50,0.00}{##1}}}
\expandafter\def\csname PY@tok@bp\endcsname{\def\PY@tc##1{\textcolor[rgb]{0.00,0.50,0.00}{##1}}}
\expandafter\def\csname PY@tok@fm\endcsname{\def\PY@tc##1{\textcolor[rgb]{0.00,0.00,1.00}{##1}}}
\expandafter\def\csname PY@tok@vc\endcsname{\def\PY@tc##1{\textcolor[rgb]{0.10,0.09,0.49}{##1}}}
\expandafter\def\csname PY@tok@vg\endcsname{\def\PY@tc##1{\textcolor[rgb]{0.10,0.09,0.49}{##1}}}
\expandafter\def\csname PY@tok@vi\endcsname{\def\PY@tc##1{\textcolor[rgb]{0.10,0.09,0.49}{##1}}}
\expandafter\def\csname PY@tok@vm\endcsname{\def\PY@tc##1{\textcolor[rgb]{0.10,0.09,0.49}{##1}}}
\expandafter\def\csname PY@tok@sa\endcsname{\def\PY@tc##1{\textcolor[rgb]{0.73,0.13,0.13}{##1}}}
\expandafter\def\csname PY@tok@sb\endcsname{\def\PY@tc##1{\textcolor[rgb]{0.73,0.13,0.13}{##1}}}
\expandafter\def\csname PY@tok@sc\endcsname{\def\PY@tc##1{\textcolor[rgb]{0.73,0.13,0.13}{##1}}}
\expandafter\def\csname PY@tok@dl\endcsname{\def\PY@tc##1{\textcolor[rgb]{0.73,0.13,0.13}{##1}}}
\expandafter\def\csname PY@tok@s2\endcsname{\def\PY@tc##1{\textcolor[rgb]{0.73,0.13,0.13}{##1}}}
\expandafter\def\csname PY@tok@sh\endcsname{\def\PY@tc##1{\textcolor[rgb]{0.73,0.13,0.13}{##1}}}
\expandafter\def\csname PY@tok@s1\endcsname{\def\PY@tc##1{\textcolor[rgb]{0.73,0.13,0.13}{##1}}}
\expandafter\def\csname PY@tok@mb\endcsname{\def\PY@tc##1{\textcolor[rgb]{0.40,0.40,0.40}{##1}}}
\expandafter\def\csname PY@tok@mf\endcsname{\def\PY@tc##1{\textcolor[rgb]{0.40,0.40,0.40}{##1}}}
\expandafter\def\csname PY@tok@mh\endcsname{\def\PY@tc##1{\textcolor[rgb]{0.40,0.40,0.40}{##1}}}
\expandafter\def\csname PY@tok@mi\endcsname{\def\PY@tc##1{\textcolor[rgb]{0.40,0.40,0.40}{##1}}}
\expandafter\def\csname PY@tok@il\endcsname{\def\PY@tc##1{\textcolor[rgb]{0.40,0.40,0.40}{##1}}}
\expandafter\def\csname PY@tok@mo\endcsname{\def\PY@tc##1{\textcolor[rgb]{0.40,0.40,0.40}{##1}}}
\expandafter\def\csname PY@tok@ch\endcsname{\let\PY@it=\textit\def\PY@tc##1{\textcolor[rgb]{0.25,0.50,0.50}{##1}}}
\expandafter\def\csname PY@tok@cm\endcsname{\let\PY@it=\textit\def\PY@tc##1{\textcolor[rgb]{0.25,0.50,0.50}{##1}}}
\expandafter\def\csname PY@tok@cpf\endcsname{\let\PY@it=\textit\def\PY@tc##1{\textcolor[rgb]{0.25,0.50,0.50}{##1}}}
\expandafter\def\csname PY@tok@c1\endcsname{\let\PY@it=\textit\def\PY@tc##1{\textcolor[rgb]{0.25,0.50,0.50}{##1}}}
\expandafter\def\csname PY@tok@cs\endcsname{\let\PY@it=\textit\def\PY@tc##1{\textcolor[rgb]{0.25,0.50,0.50}{##1}}}

\def\PYZbs{\char`\\}
\def\PYZus{\char`\_}
\def\PYZob{\char`\{}
\def\PYZcb{\char`\}}
\def\PYZca{\char`\^}
\def\PYZam{\char`\&}
\def\PYZlt{\char`\<}
\def\PYZgt{\char`\>}
\def\PYZsh{\char`\#}
\def\PYZpc{\char`\%}
\def\PYZdl{\char`\$}
\def\PYZhy{\char`\-}
\def\PYZsq{\char`\'}
\def\PYZdq{\char`\"}
\def\PYZti{\char`\~}
% for compatibility with earlier versions
\def\PYZat{@}
\def\PYZlb{[}
\def\PYZrb{]}
\makeatother


    % Exact colors from NB
    \definecolor{incolor}{rgb}{0.0, 0.0, 0.5}
    \definecolor{outcolor}{rgb}{0.545, 0.0, 0.0}



    
    % Prevent overflowing lines due to hard-to-break entities
    \sloppy 
    % Setup hyperref package
    \hypersetup{
      breaklinks=true,  % so long urls are correctly broken across lines
      colorlinks=true,
      urlcolor=urlcolor,
      linkcolor=linkcolor,
      citecolor=citecolor,
      }
    % Slightly bigger margins than the latex defaults
    
    \geometry{verbose,tmargin=1in,bmargin=1in,lmargin=1in,rmargin=1in}
    
    

    \begin{document}
    
    
    \maketitle
    
    

    
    \hypertarget{swaps-and-modules---practical-lesson-5}{%
\section{Swaps and Modules}\label{swaps-and-modules---practical-lesson-5}}

\hypertarget{recap}{%
\subsection{Recap}\label{recap}}

\begin{itemize}
\tightlist
\item
  basic Python (mostly not related directly to finance)
\item
  how to implement a discount factor interpolation function
\item
  qrapping up functionality in classes in order to work with multiple
  data sets more easily
\item
  libor forward rate calculator
\end{itemize}

\hypertarget{todays-lesson}{%
\subsection{Today's lesson}\label{todays-lesson}}

We're going to look at: * modules, and start building up our library of
finance-related functionality * implementing an Overnight Index Swap
class for calculating the NPV of an OIS.

\hypertarget{modules}{%
\section{Modules}\label{modules}}

An interactive session (e.g notebook or interactive shell) is great for
quick testing and exploratory use, but once you have some code
(i.e.~functions or classes) which you'd like to reuse often, rather than
copy/pasting it every time you need it, you can save it in a .py file
and use it from your session (aka you can create your own library).

These work just like the modules we have been importing up to now,
except they're written by us! Take a look at this video
(https://www.youtube.com/watch?v=AqCl65wxikw) for an example of how it's
done for Jupyter notebook.

We're going to start writing a module called \textbf{finmarkets}, and
over the course of the remaining lessons we'll add functionality related
to the theory lessons.

So first of all let's create a new file called finmarkets.py and copy
into it the \texttt{DiscountCurve} class we wrote last time.

    \begin{Verbatim}[commandchars=\\\{\}]
{\color{incolor}In [{\color{incolor}1}]:} \PY{k+kn}{from} \PY{n+nn}{datetime} \PY{k}{import} \PY{n}{date}
        \PY{k+kn}{from} \PY{n+nn}{finmarkets} \PY{k}{import} \PY{n}{DiscountCurve}
        
        \PY{n}{curve} \PY{o}{=} \PY{n}{DiscountCurve}\PY{p}{(}\PY{n}{date}\PY{p}{(}\PY{l+m+mi}{2019}\PY{p}{,} \PY{l+m+mi}{1}\PY{p}{,} \PY{l+m+mi}{1}\PY{p}{)}\PY{p}{,}
                              \PY{p}{[}\PY{n}{date}\PY{p}{(}\PY{l+m+mi}{2019}\PY{p}{,} \PY{l+m+mi}{1}\PY{p}{,} \PY{l+m+mi}{1}\PY{p}{)}\PY{p}{,} 
                               \PY{n}{date}\PY{p}{(}\PY{l+m+mi}{2019}\PY{p}{,} \PY{l+m+mi}{6}\PY{p}{,} \PY{l+m+mi}{1}\PY{p}{)}\PY{p}{,} 
                               \PY{n}{date}\PY{p}{(}\PY{l+m+mi}{2020}\PY{p}{,} \PY{l+m+mi}{1}\PY{p}{,} \PY{l+m+mi}{1}\PY{p}{)}\PY{p}{]}\PY{p}{,}
                              \PY{p}{[}\PY{l+m+mf}{1.0}\PY{p}{,} \PY{l+m+mf}{0.98}\PY{p}{,} \PY{l+m+mf}{0.82}\PY{p}{]}\PY{p}{)}
        \PY{n}{curve}\PY{o}{.}\PY{n}{df}\PY{p}{(}\PY{n}{date}\PY{p}{(}\PY{l+m+mi}{2019}\PY{p}{,} \PY{l+m+mi}{7}\PY{p}{,} \PY{l+m+mi}{1}\PY{p}{)}\PY{p}{)}
\end{Verbatim}

\begin{Verbatim}[commandchars=\\\{\}]
{\color{outcolor}Out[{\color{outcolor}1}]:} 0.9558151167629666
\end{Verbatim}
            
    We will use this discount curve later in this lesson.

\hypertarget{overnight-index-swap}{%
\section{Overnight Index Swap}\label{overnight-index-swap}}

Overnight Index Swap (OIS) are products which pay a floating coupon,
determined by overnight rate fixings over the reference periods, against
a fixed coupon. We will always look at these products from the point of
view of the \textbf{receiver of the floating leg}. Therefore an OIS is
defined by:

\begin{itemize}
\tightlist
\item
  a notional amount \(N\)
\item
  a start date \(d_0\)
\item
  a sequence of payment dates \(d_1,...,d_n\)
\item
  a fixed rate \(K\)
\end{itemize}

For simplicity we're assuming that the fixed and floating legs have the
same notional and payment dates, although this is not necessarily always
the case in practice.

At each payment date, the floating leg pays a cash flow determined as
follows:

\[f_{\mathrm{float},~i} = N \Bigg\{\prod_{d=d_{i-1}}^{d=d_i-1}\Big(1+r_{o/n}(d)\cdot\frac{1}{360}\Big) -1 \Bigg\}\]

(This formula is valid for an EONIA swap, i.e.~for OIS swaps in EUR,
other currencies might have different conventions. The \(\frac{1}{360}\)
fraction appears because EONIA rates are quoted using the ACT/360
daycount convention and here we're making a simplifying assumption of
ignoring weekends and holidays, so we assume that each overnight rate is
valid for only one day.)

The sum of the discounted expected values of these cashflows is

\[\mathrm{NPV}_{\mathrm{float}} = \sum_{i=1}^{n}D(d_i)\mathbb{E}[f_{\mathrm{float},~i}]\]

where \(D(d)\) is the discount factor with expiry \(d\). On the other
hand, by definition (remember practical lesson 4 with forward rates), we
also have the following relationship

\[\mathbb{E}[f_{\mathrm{float},~i}] = N\cdot\Big(\frac{D_{ois}(d_{i-1})}{D_{ois}(d_{i})} - 1\Big) \]

where \(D_{ois}(d)\) is the discount factor implied by OIS prices.

In a previous theory lesson we mentioned that the correct curve to use
for discounting the flows of a collateralized contract is the one
associated with the collateral. Since OIS contracts are collateralized
with cash, and cash accrues daily interes at the overnight rate, the OIS
curve is itself the correct curve with which to discount the flows of an
OIS contract !

In summary, \(D = D_{ois}\) so the NPV simplifies to

\[\mathrm{NPV}_{\mathrm{float}} = N\cdot\sum_{i=1}^{n}[D(d_{i-1}) - D(d_i)] = N \cdot [D(d_0) - D(d_n)]\]

Each cash flow of the fixed leg is equal to

\[f_{\mathrm{fix},~i}=N\cdot K\cdot \frac{d_i - d_{i-1}}{360}\]

so the NPV of the fixed leg is

\[\mathrm{NPV}_{\mathrm{fix}} = N\cdot K\cdot \sum_{i=1}^{n}D(d_{i})\frac{d_i - d_{i-1}}{360}\]

Ultimately the aim will be to take a series of OIS quotations, and
determine the discount factors implied by their prices. To do this we'll
build a pricing function (or rather a class), which takes discount curve
as the input and produces the net present value (NPV) of the OIS as the
output. Next lesson we'll put this function inside a numerical optimizer
to invert the process and hence to determine the implied discount
factors from the prices.

    \begin{Verbatim}[commandchars=\\\{\}]
{\color{incolor}In [{\color{incolor}2}]:} \PY{k}{class} \PY{n+nc}{OvernightIndexSwap}\PY{p}{(}\PY{n+nb}{object}\PY{p}{)}\PY{p}{:}
        
            \PY{c+c1}{\PYZsh{} this method is called to build the instance,}
            \PY{c+c1}{\PYZsh{} we take some data arguments and save them as}
            \PY{c+c1}{\PYZsh{} attributes of self }
            \PY{c+c1}{\PYZsh{} n.b.: payment\PYZus{}dates should be a list of dates,}
            \PY{c+c1}{\PYZsh{} including the start date as the first element}
            \PY{k}{def} \PY{n+nf}{\PYZus{}\PYZus{}init\PYZus{}\PYZus{}}\PY{p}{(}\PY{n+nb+bp}{self}\PY{p}{,} \PY{n}{notional}\PY{p}{,} \PY{n}{payment\PYZus{}dates}\PY{p}{,} \PY{n}{fixed\PYZus{}rate}\PY{p}{)}\PY{p}{:}
                \PY{n+nb+bp}{self}\PY{o}{.}\PY{n}{notional} \PY{o}{=} \PY{n}{notional}
                \PY{n+nb+bp}{self}\PY{o}{.}\PY{n}{payment\PYZus{}dates} \PY{o}{=} \PY{n}{payment\PYZus{}dates}
                \PY{n+nb+bp}{self}\PY{o}{.}\PY{n}{fixed\PYZus{}rate} \PY{o}{=} \PY{n}{fixed\PYZus{}rate}
                
            \PY{c+c1}{\PYZsh{} this method takes a discount curve and calculates}
            \PY{c+c1}{\PYZsh{} the NPV of the floating leg using that curve}
            \PY{k}{def} \PY{n+nf}{npv\PYZus{}floating\PYZus{}leg}\PY{p}{(}\PY{n+nb+bp}{self}\PY{p}{,} \PY{n}{discount\PYZus{}curve}\PY{p}{)}\PY{p}{:}
                \PY{c+c1}{\PYZsh{} self.payment\PYZus{}date s[0] is the start date of the swap}
                \PY{c+c1}{\PYZsh{} self.payment\PYZus{}date s[-1] is the last payment date of the swap}
                \PY{k}{return} \PY{n+nb+bp}{self}\PY{o}{.}\PY{n}{notional} \PY{o}{*} \PY{p}{(}\PY{n}{discount\PYZus{}curve}\PY{o}{.}\PY{n}{df}\PY{p}{(}\PY{n+nb+bp}{self}\PY{o}{.}\PY{n}{payment\PYZus{}dates}\PY{p}{[}\PY{l+m+mi}{0}\PY{p}{]}\PY{p}{)} \PY{o}{\PYZhy{}} 
                                        \PY{n}{discount\PYZus{}curve}\PY{o}{.}\PY{n}{df}\PY{p}{(}\PY{n+nb+bp}{self}\PY{o}{.}\PY{n}{payment\PYZus{}dates}\PY{p}{[}\PY{o}{\PYZhy{}}\PY{l+m+mi}{1}\PY{p}{]}\PY{p}{)}\PY{p}{)}
            
            \PY{c+c1}{\PYZsh{} this method takes a discount curve and calculates the NPV}
            \PY{c+c1}{\PYZsh{} of the fixed leg using that curve}
            \PY{k}{def} \PY{n+nf}{npv\PYZus{}fixed\PYZus{}leg}\PY{p}{(}\PY{n+nb+bp}{self}\PY{p}{,} \PY{n}{discount\PYZus{}curve}\PY{p}{)}\PY{p}{:}
                \PY{n}{npv} \PY{o}{=} \PY{l+m+mi}{0}
                \PY{c+c1}{\PYZsh{} we loop from i=1 up to but not including the length of the date list}
                \PY{k}{for} \PY{n}{i} \PY{o+ow}{in} \PY{n+nb}{range}\PY{p}{(}\PY{l+m+mi}{1}\PY{p}{,} \PY{n+nb}{len}\PY{p}{(}\PY{n+nb+bp}{self}\PY{o}{.}\PY{n}{payment\PYZus{}dates}\PY{p}{)}\PY{p}{)}\PY{p}{:} 
                    \PY{c+c1}{\PYZsh{} we can do i-1, because the loop starts with i=1}
                    \PY{n}{start\PYZus{}date} \PY{o}{=} \PY{n+nb+bp}{self}\PY{o}{.}\PY{n}{payment\PYZus{}dates}\PY{p}{[}\PY{n}{i}\PY{o}{\PYZhy{}}\PY{l+m+mi}{1}\PY{p}{]} 
                    \PY{n}{end\PYZus{}date} \PY{o}{=} \PY{n+nb+bp}{self}\PY{o}{.}\PY{n}{payment\PYZus{}dates}\PY{p}{[}\PY{n}{i}\PY{p}{]}
                    \PY{n}{tau} \PY{o}{=} \PY{p}{(}\PY{n}{end\PYZus{}date} \PY{o}{\PYZhy{}} \PY{n}{start\PYZus{}date}\PY{p}{)}\PY{o}{.}\PY{n}{days} \PY{o}{/} \PY{l+m+mi}{360}
                    \PY{n}{df} \PY{o}{=} \PY{n}{discount\PYZus{}curve}\PY{o}{.}\PY{n}{df}\PY{p}{(}\PY{n}{end\PYZus{}date}\PY{p}{)}
                    \PY{n}{npv} \PY{o}{=} \PY{n}{npv} \PY{o}{+} \PY{n}{df} \PY{o}{*} \PY{n}{tau}
                    \PY{k}{return} \PY{n+nb+bp}{self}\PY{o}{.}\PY{n}{notional} \PY{o}{*} \PY{n+nb+bp}{self}\PY{o}{.}\PY{n}{fixed\PYZus{}rate} \PY{o}{*} \PY{n}{npv}
            
            \PY{c+c1}{\PYZsh{} this method calculates the NPV of the OIS swap}
            \PY{c+c1}{\PYZsh{} n.b.: inside this method we call the other two }
            \PY{c+c1}{\PYZsh{} methods of the class on the same instance \PYZsq{}self\PYZsq{},}
            \PY{c+c1}{\PYZsh{} using self.npv\PYZus{}XXX\PYZus{}leg(...), and we pass the }
            \PY{c+c1}{\PYZsh{} discount\PYZus{}curve we received as an argument}
            \PY{k}{def} \PY{n+nf}{npv}\PY{p}{(}\PY{n+nb+bp}{self}\PY{p}{,} \PY{n}{discount\PYZus{}curve}\PY{p}{)}\PY{p}{:}
                \PY{n}{float\PYZus{}npv} \PY{o}{=} \PY{n+nb+bp}{self}\PY{o}{.}\PY{n}{npv\PYZus{}floating\PYZus{}leg}\PY{p}{(}\PY{n}{discount\PYZus{}curve}\PY{p}{)}
                \PY{n}{fixed\PYZus{}npv} \PY{o}{=} \PY{n+nb+bp}{self}\PY{o}{.}\PY{n}{npv\PYZus{}fixed\PYZus{}leg}\PY{p}{(}\PY{n}{discount\PYZus{}curve}\PY{p}{)}
                \PY{k}{return} \PY{n}{float\PYZus{}npv} \PY{o}{\PYZhy{}} \PY{n}{fixed\PYZus{}npv}
\end{Verbatim}

    \begin{Verbatim}[commandchars=\\\{\}]
{\color{incolor}In [{\color{incolor}3}]:} \PY{k+kn}{from} \PY{n+nn}{datetime} \PY{k}{import} \PY{n}{date}
        
        \PY{n}{ois} \PY{o}{=} \PY{n}{OvernightIndexSwap}\PY{p}{(}
            \PY{c+c1}{\PYZsh{} the notional, one million}
            \PY{l+m+mf}{1e6}\PY{p}{,}
            \PY{c+c1}{\PYZsh{} the list of product dates, }
            \PY{c+c1}{\PYZsh{} i.e. the start date then the payment dates}
            \PY{p}{[}\PY{n}{date}\PY{p}{(}\PY{l+m+mi}{2019}\PY{p}{,} \PY{l+m+mi}{1}\PY{p}{,} \PY{l+m+mi}{1}\PY{p}{)}\PY{p}{,} 
             \PY{n}{date}\PY{p}{(}\PY{l+m+mi}{2019}\PY{p}{,} \PY{l+m+mi}{4}\PY{p}{,} \PY{l+m+mi}{1}\PY{p}{)}\PY{p}{,} 
             \PY{n}{date}\PY{p}{(}\PY{l+m+mi}{2019}\PY{p}{,} \PY{l+m+mi}{7}\PY{p}{,} \PY{l+m+mi}{1}\PY{p}{)}\PY{p}{,} 
             \PY{n}{date}\PY{p}{(}\PY{l+m+mi}{2019}\PY{p}{,} \PY{l+m+mi}{10}\PY{p}{,} \PY{l+m+mi}{1}\PY{p}{)}\PY{p}{,}
             \PY{n}{date}\PY{p}{(}\PY{l+m+mi}{2020}\PY{p}{,} \PY{l+m+mi}{1}\PY{p}{,} \PY{l+m+mi}{1}\PY{p}{)}\PY{p}{]}\PY{p}{,}
            \PY{c+c1}{\PYZsh{} the fixed rate, 2.5\PYZpc{}}
            \PY{l+m+mf}{0.025}
        \PY{p}{)}
\end{Verbatim}

    We can now use the curve we have prepared at the beginning of the lesson
and that we stored in a variable called curve. Let's now evaluate the
NPV of the OIS.

    \begin{Verbatim}[commandchars=\\\{\}]
{\color{incolor}In [{\color{incolor}4}]:} \PY{n}{ois}\PY{o}{.}\PY{n}{npv}\PY{p}{(}\PY{n}{curve}\PY{p}{)}
\end{Verbatim}

\begin{Verbatim}[commandchars=\\\{\}]
{\color{outcolor}Out[{\color{outcolor}4}]:} 173824.80713628858
\end{Verbatim}
            
    \hypertarget{exercises}{%
\subsection{Exercises}\label{exercises}}

\hypertarget{exercise-5.1}{%
\subsubsection{Exercise 5.1}\label{exercise-5.1}}

Take the \texttt{OvernightIndexSwap} class from the lesson and add a new
method called fair\_value\_strike which takes a discount curve object
and returns the fixed rate which would make the OIS have zero NPV.

\emph{Hints}: * first take the formulas for the NPV of the fixed leg and
the NPV of the floating leg, put one equal to the other and solve for
\(K\); * then implement that in Python.

\hypertarget{exercise-5.2}{%
\subsubsection{Exercise 5.2}\label{exercise-5.2}}

Take the \texttt{OvernightIndexSwap} class, add it to
\texttt{finmarkets.py} and try importing and using it.

\hypertarget{exercise-5.3}{%
\subsubsection{Exercise 5.3}\label{exercise-5.3}}

In the next lesson we're going to build lots of
\texttt{OvernightIndexSwap} objects, one for each market quote we have.
The market quotes will consist of fixed strikes for 1M, 2M, 3M,
\ldots{}, 12M, 15M, 18M, 2Y, 3Y, \ldots{}, 30Y and 40Y swaps.

It would be very boring to write a long list of payment dates for each
one of these, plus they'd need to be updated every day. Write a function
which given a start date and the number of months, returns a list of
dates of \textbf{annual} frequency starting from the start date and
ending after the specified number of months.

For example

2016-11-17 start date 12 months \(\rightarrow\) 2016-11-17, 2017-11-17
2016-11-17 start date 24 months \(\rightarrow\) 2016-11-17, 2017-11-17,
2018-11-17

Note that if the number of months is not a multiple of 12, the last
period should simply be shorter than 12 months. For example

2016-11-17 start date 9 months \(\rightarrow\) 2016-11-17, 2017-08-17
2016-11-17 start date 15 months \(\rightarrow\) 2016-11-17, 2017-11-17,
2018-02-17

Here's some skeleton code to help you get started:

\begin{Shaded}
\begin{Highlighting}[]
\ImportTok{from}\NormalTok{ dateutil }\ImportTok{import}\NormalTok{ relativedelta}

\KeywordTok{def}\NormalTok{ generate_swap_dates(start_date, n_months):}
\NormalTok{    dates }\OperatorTok{=}\NormalTok{ []}
    \CommentTok{# your code here which adds all the relevant dates to the dates list}
    \ControlFlowTok{return}\NormalTok{ dates}
\end{Highlighting}
\end{Shaded}

\begin{Shaded}
\begin{Highlighting}[]
\CommentTok{# some tests to check if the function is working correctly}
\ImportTok{from}\NormalTok{ datetime }\ImportTok{import}\NormalTok{ date}

\ControlFlowTok{assert}\NormalTok{ generate_swap_dates(date(}\DecValTok{2016}\NormalTok{, }\DecValTok{11}\NormalTok{, }\DecValTok{17}\NormalTok{), }\DecValTok{12}\NormalTok{) }\OperatorTok{==}\NormalTok{ [date(}\DecValTok{2016}\NormalTok{, }\DecValTok{11}\NormalTok{, }\DecValTok{17}\NormalTok{), }
\NormalTok{                                                       date(}\DecValTok{2017}\NormalTok{, }\DecValTok{11}\NormalTok{, }\DecValTok{17}\NormalTok{)]}
\ControlFlowTok{assert}\NormalTok{ generate_swap_dates(date(}\DecValTok{2016}\NormalTok{, }\DecValTok{11}\NormalTok{, }\DecValTok{17}\NormalTok{), }\DecValTok{24}\NormalTok{) }\OperatorTok{==}\NormalTok{ [date(}\DecValTok{2016}\NormalTok{, }\DecValTok{11}\NormalTok{, }\DecValTok{17}\NormalTok{), }
\NormalTok{                                                       date(}\DecValTok{2017}\NormalTok{, }\DecValTok{11}\NormalTok{, }\DecValTok{17}\NormalTok{), }
\NormalTok{                                                       date(}\DecValTok{2018}\NormalTok{, }\DecValTok{11}\NormalTok{, }\DecValTok{17}\NormalTok{)]}

\ControlFlowTok{assert}\NormalTok{ generate_swap_dates(date(}\DecValTok{2016}\NormalTok{, }\DecValTok{11}\NormalTok{, }\DecValTok{17}\NormalTok{), }\DecValTok{9}\NormalTok{) }\OperatorTok{==}\NormalTok{ [date(}\DecValTok{2016}\NormalTok{, }\DecValTok{11}\NormalTok{, }\DecValTok{17}\NormalTok{), }
\NormalTok{                                                      date(}\DecValTok{2017}\NormalTok{, }\DecValTok{8}\NormalTok{, }\DecValTok{17}\NormalTok{)]}
\ControlFlowTok{assert}\NormalTok{ generate_swap_dates(date(}\DecValTok{2016}\NormalTok{, }\DecValTok{11}\NormalTok{, }\DecValTok{17}\NormalTok{), }\DecValTok{15}\NormalTok{) }\OperatorTok{==}\NormalTok{ [date(}\DecValTok{2016}\NormalTok{, }\DecValTok{11}\NormalTok{, }\DecValTok{17}\NormalTok{), }
\NormalTok{                                                       date(}\DecValTok{2017}\NormalTok{, }\DecValTok{11}\NormalTok{, }\DecValTok{17}\NormalTok{), }
\NormalTok{                                                       date(}\DecValTok{2018}\NormalTok{, }\DecValTok{2}\NormalTok{, }\DecValTok{17}\NormalTok{)]}
\end{Highlighting}
\end{Shaded}


    % Add a bibliography block to the postdoc
    
    
    
    \end{document}

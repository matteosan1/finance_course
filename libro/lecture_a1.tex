\chapter{Binomial Distribution}\label{binomial-distribution}

In statistics, the binomial distribution is a discrete probability of
independent events, where each event has exactly two possible outcomes.
For example, if we toss a coin 10 times and we are interested in how
many times it will land on heads, the probability of it lending one,
two, three, up to 10 head in a row will form a distribution of
probabilities or binomial distribution. When we're interested in a
specific outcome, let's say 3 heads out of 10 tosses, we call it
binomial probability, which is a probability of exactly \(k\) successes
on \(n\) repeated trials.

Following the characteristics of this distribution:

\begin{itemize}
\tightlist
\item
  the number of observations or trials is fixed;
\item
  each observation or trial is independent. Like in the coin toss, the
  previous toss doesn't affect the following;
\item
  the probability of success is exactly the same from one trial to
  another. Each coin toss has 50\% chances to fall in heads or tails.
\end{itemize}

\section{Probability Mass Function}\label{probability-mass-function}

The probabilty mass function (PMF) is just the equivalent of the
probability density function (PDF) in the discrete case. It describes the
probability of an outcome, and for the binomial distribution is

\[ \mathrm{PMF}_{\mathrm{binomial}} = \cfrac{n!}{k!(n-k)!}p^k (1-p)^{n-k}\]
where \(n\) is the number of observations, \(k\) the desired outcome,
\(p\) the probability of an outcome. To have a better understanding of
how this formula works let's take a look at two examples.

\section{A Fair Coin Flip}\label{a-fair-coin-flip}

If we flip a fair coin 100 times, what is the probability of getting
exactly 50 heads ? Now we might think that since we have equal chances
of getting heads or tails, the answer would be 50\% probability. On the
other hand, what are the odds that we will get 49 heads, or 51 ? Almost
50\% chance too, but what exactly ? Let's first answer the first
question and then try it for all the possible outcomes to see how our
probabilities are distributed. So our parameters are the following:
\(n = 100\), \(k = 50\), \(p = 0.5\). Plugging numbers into the formula
and we get

\[\mathbb{P}(50) = \cfrac{100!}{50!(100 - 50)!}0.5^{50}(1-0.5)^{(100-50)} \approx 8\% \]

So there is an 8\% chance of getting exactly 50 heads on 100 coin
tosses. The chart below shows how to get this number in \(\tt{python}\)
and the binomial distribution for all possible outcomes. The probability distribution
is shown in Fig.~\ref{fig:binomial_coin_flip}.

\begin{tcolorbox}[breakable, size=fbox, boxrule=1pt, pad at break*=1mm,colback=cellbackground, colframe=cellborder]
\begin{Verbatim}[commandchars=\\\{\}]
\PY{k+kn}{from} \PY{n+nn}{scipy}\PY{n+nn}{.}\PY{n+nn}{stats} \PY{k}{import} \PY{n}{binom} \PY{c+c1}{\PYZsh{} access binomial distribution}

\PY{n}{b} \PY{o}{=} \PY{n}{binom}\PY{p}{(}\PY{l+m+mi}{100}\PY{p}{,} \PY{l+m+mf}{0.5}\PY{p}{)} \PY{c+c1}{\PYZsh{} params (n, p)}

\PY{n+nb}{print} \PY{p}{(}\PY{l+s+s2}{\PYZdq{}}\PY{l+s+s2}{P(50): }\PY{l+s+si}{\PYZob{}:.2f\PYZcb{}}\PY{l+s+s2}{\PYZpc{}}\PY{l+s+s2}{\PYZdq{}}\PY{o}{.}\PY{n}{format}\PY{p}{(}\PY{n}{b}\PY{o}{.}\PY{n}{pmf}\PY{p}{(}\PY{l+m+mi}{50}\PY{p}{)}\PY{o}{*}\PY{l+m+mi}{100}\PY{p}{)}\PY{p}{)}

P(50): 7.96\%
\end{Verbatim}
\end{tcolorbox}

\begin{figure}[ht]
\centering
    \includegraphics[width=0.7\textwidth]{binomial_files/binomial_2_0.png}
    \caption{Binomial distribution relative to 100 coin flips.}
    \label{fig:binomial_coin_flip}
\end{figure}
    
\section{Success of Marketing Campaign}\label{success-of-marketing-campaign}

In 2014, 8.5\% of adults aged 18 years and older had diabetes. Let's go
back to 2014 and pretend we're working for the company that developed a
drug for treating diabetes and our job is to make phone calls to find
potential clients. The only statistic we know is that 8.5\% of adults in
the US have diabetes. What is the probability that out of 50 calls at
least 5 customers would have diabetes ? Once again let's plug numbers
into the formula: \(n = 50\), \(k > 5\), \(p = 0.085\)

\[\sum_{k=5}^{n}\mathbb{P}(k) = \sum_{k=5}^{n}\Big(\cfrac{50!}{k!(50 - k)!}0.085^{k}(1-0.085)^{(50-k)}\Big) \approx 42.2\% \]
there is a 42.2\% chance that out of 50 calls we will find at least five
customers with diabetes. Again in \(\tt{python}\), this time instead of
summing PMFs we exploit the binomial CDF (see Fig.~\ref{fig:binomial_cdf})

\begin{tcolorbox}[breakable, size=fbox, boxrule=1pt, pad at break*=1mm,colback=cellbackground, colframe=cellborder]
\begin{Verbatim}[commandchars=\\\{\}]
\PY{n}{b} \PY{o}{=} \PY{n}{binom}\PY{p}{(}\PY{l+m+mi}{50}\PY{p}{,} \PY{l+m+mf}{0.085}\PY{p}{)} \PY{c+c1}{\PYZsh{} params (n, p)}

\PY{n}{prob} \PY{o}{=} \PY{n}{b}\PY{o}{.}\PY{n}{cdf}\PY{p}{(}\PY{l+m+mi}{50}\PY{p}{)}\PY{o}{\PYZhy{}}\PY{n}{b}\PY{o}{.}\PY{n}{cdf}\PY{p}{(}\PY{l+m+mi}{4}\PY{p}{)}

\PY{n+nb}{print} \PY{p}{(}\PY{l+s+s2}{\PYZdq{}}\PY{l+s+s2}{P(\PYZgt{}=5): }\PY{l+s+si}{\PYZob{}:.1f\PYZcb{}}\PY{l+s+s2}{\PYZpc{}}\PY{l+s+s2}{\PYZdq{}}\PY{o}{.}\PY{n}{format}\PY{p}{(}\PY{n}{prob}\PY{o}{*}\PY{l+m+mi}{100}\PY{p}{)}\PY{p}{)}

P(>=5): 42.2\%
\end{Verbatim}
\end{tcolorbox}

    \begin{figure}[htb]
    \centering
    \includegraphics[width=0.7\textwidth]{binomial_files/binomial_5_0.png}
    \caption{Binomial distribution of 50 calls with ($p=0.085$), in red the integrated probability of finding 5 or more successes.}
    \label{fig:binomial_cdf}
    \end{figure}
    
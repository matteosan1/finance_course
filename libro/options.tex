\chapter{Black-Scholes Model}
\label{ch:BS}

\section{Black-Scholes Model}
Suppose that stock price $S$ follows a geometric Brownian motion

\begin{equation}
dS = \mu Sdt + \sigma SdW
\label{eq:bs_gbm}
\end{equation}

we are going to derive a partial differential equation (PDE) for the price of a derivative based on such a stock.

The model proposed by Black and Scholes relies on the following set of additional assumptions (besides the geometric Brownian motion):
\begin{itemize}
\tightlist 
\item constant risk-less interest rate $r$;
\item no transaction costs;
\item it is possible to buy/sell any (also fractional) number of stocks; similarly with the cash;
\item no restrictions on short selling;
\item option is of European type.
\end{itemize}

After the derivation of the model equation we will determine the explicit solution for European call and put options.

\subsection{Merton Derivation}
In this Section we are going to provide the steps followed by Merton to derive the Black-Scholes PDE.

Firstly, let's consider the case of a non-dividend paying stock. Imagine a portfolio consisting of options, stocks and cash with the properties that at each time $t$ the portfolio has zero value and it is \emph{self-financing}, which means that if there is no exogenous infusion or withdrawal of money, the purchase of a new asset must be financed by the sale of an old one.

Let be 
\begin{itemize}
\item $Q_S$, the number of stoks, each of them with value $S$;
\item $Q_V$, number of options, each of them with value $V$;
\item $B$, cash on the account, which is continuously compounded using the risk-free rate $r$.
\end{itemize}

The required properties can be formulated mathematically as
\begin{equation}
\begin{cases}
SQ_S+VQ_V+B=0 \\
SdQ_S + V dQ_V +\delta B = 0 \\
dB = rBdt+\delta B
\end{cases}
\label{eq:bs_properties}
\end{equation}

Differentiating the first of Eq.~\ref{eq:bs_properties}
\begin{equation}
\begin{split}
&d(SQ_S + VQ_V +B) = d(SQ_S + VQ_V) + \overbrace{dB}^{rBdt+\delta B} = 0 \\
& \overbrace{SdQ_s + VdQ_V +\delta B}^{=0} + Q_S dS + Q_V dV + rBdt = 0 \\
& Q_SdS+Q_VdV \overbrace{-r(SQ_S + VQ_V)}^{rB}dt=0\\
\end{split} 
\end{equation}
Dividing by $Q_V$ and setting $\Delta = -\frac{Q_S}{Q_V}$
\begin{equation}
dV -rVdt-\Delta (dS - rSdt)=0
\end{equation}
$dS$ can be determined from Eq.~\ref{eq:bs_gbm}, while $dV$ applying It$\hat{o}$'s lemma. Then we choose $\Delta$ (i.e. the ratio between the number of stocks and options) so that it eliminates the randomness (i.e. the coefficient at $dW$ is zero).
  
\begin{equation}
dV = \left(\cfrac{\partial V}{\partial t} + \mu S  \cfrac{\partial V}{\partial S} + \cfrac{1}{2} \sigma^2 S^2  \cfrac{\partial^2 V}{\partial S^2} \right) dt + \sigma S \cfrac{\partial V}{\partial S} dW
\end{equation} 
Therefore:
\begin{equation}
dP =  \left(\cfrac{\partial V}{\partial t} + \mu S  \cfrac{\partial V}{\partial S} + \cfrac{1}{2} \sigma^2 S^2  \cfrac{\partial^2 V}{\partial S^2} + \delta \mu S\right) dt + \left(\sigma S  \cfrac{\partial V}{\partial S} + \delta \sigma S \right)dW
\end{equation}
We eliminate the randomness
\begin{equation}
\delta = -\cfrac{\partial V}{\partial S}
\end{equation}

\begin{equation}
\cfrac{\partial V}{\partial t} + \cfrac{1}{2} \sigma^2 S^2 \cfrac{\partial^2 V}{\partial S^2}+ rS \cfrac{\partial V}{\partial S} − rV = 0
\end{equation}

\section{Solutions of Black-Scholes PDE}
Now we need to find a solution $V(S, t)$ to the partial differential equation which holds for $S >0$, $t \in [0, T)$.

So far we have not used the fact that we consider an option (in general the PDE holds for any derivative that pays a payoff at time $T$ depending on the stock price at this time).
Clearly the type of derivative determines the terminal condition at time $T$, so in general $V(S, T)$ = payoff of the derivative.

\subsection{Simple Solutions}
How to price the derivatives with the following payoffs:
\begin{itemize}
\tightlist
\item $V(S, T) = S$, it is in fact a stock so $V(S, t) = S$
\item $V(S, T) = E$ with a certainty we obtain the cash $E$ ($V(S, t) = Ee^{−r(T−t)}$)
\end{itemize}
by substitution into the PDE it is possible to check that they are indeed solutions.
%EXERCISES:
%Find the price of a derivative with payoff
%$V(S, T) =S^n$,where $n\in N$.

%HINT: Look for the solution in the form
%$V(S, t) =A(t)S^n$

%Find all solutions to the Black-Scholes PDE, which are independent of time, i.e., for which
%$V(S, t) = V(S)$

\subsection{Binary Options}
Let us consider a binary option, which pays \$1 if the stock price is higher that $K$ at expiration time, otherwise its payoff is zero.
In this case
\begin{equation}
V(S, T) = 
	\begin{cases}
		1, \quad\textrm{if }S > K\\
		0, \quad\textrm{otherwise}	
	\end{cases}
\end{equation}

The main idea is to transform the Black-Scholes PDE to a heat equation with some variable transformations.

The first step is to apply the transformation $x = \textrm{ln}(S/K)$ and a new function $Z(x, \tau) = V (Ee^x, T -\tau)$. 
The PDE for $Z(x, \tau)$ becomes:
\begin{gather}
\cfrac{\partial Z}{\partial \tau} -\cfrac{1}{2}\sigma^2 \cfrac{\partial^2 Z}{\partial x^2}+\left(\cfrac{\sigma^2}{2}−r\right)  \cfrac{\partial Z}{\partial x} + rZ = 0\\
Z(x, 0) =V(Ee^x, T)
\end{gather}

Then we want the PDE to become a heat equation. This is done by considering the new function $u(x,\tau)=e^{\alpha x + \beta\tau}Z(x,\tau)$ where the constant $\alpha, \beta \in \mathbb{R}$ are chosen so that the PDE for $u$ is the heat equation.

The corresponding PDE for $u$ is
\begin{gather}
\cfrac{\partial u}{\partial \tau} -\cfrac{\sigma^2}{2} \cfrac{\partial^2 u}{\partial x^2}+A\cfrac{\partial u}{\partial x} + Bu = 0\\ \\
u(x,0)=e^{\alpha x}Z(x,0)=e^{\alpha x}V(Ke^x, T)
\end{gather}
where
\begin{equation}
\begin{cases}
A=\alpha\sigma^2 + \cfrac{\sigma^2}{2}-r\\
B=(1+\alpha)r-\beta-\cfrac{\alpha^2\sigma^2+\alpha\sigma^2}{2}
\end{cases}
\end{equation}

In order to have $A=B=0$ to get the heat equation we set
\begin{equation}
\begin{cases}
\alpha=\cfrac{r}{\sigma^2}-\cfrac{1}{2}\\
\beta=\cfrac{r}{2}+\cfrac{\sigma^2}{8}+\cfrac{r^2}{2\sigma^2}
\end{cases}
\end{equation}

The solution $u(x,\tau)$ of the resulting PDE is given by hte \emph{Green formula}
\begin{equation}
u(x, \tau)=\cfrac{1}{\sqrt{2\sigma^2\pi\tau}}\int^{+\infty}_{-\infty}e^{-\frac{(x-s)^2}{2\sigma^2\tau}}u(s,0)ds
\end{equation}

After evaluating the integration it is possible to find the original solution $V(S,t)$ performing backward substitutions

\begin{equation}
V(S, t) = e^{-r(T-t)}N(d_2)
\end{equation}
where $d_2=\frac{\textrm{log}(\frac{S}{K}+\left(r-\frac{\sigma^2}{2}\right)(T-t)}{\sigma\sqrt{T-t}}$.

\subsection{Call Option}
In this case
\begin{equation}
V(S,T)=\textrm{max}(0,S-K)=
\begin{cases}
S-K,\quad \textrm{if}~S>K\\
0, \quad\textrm{otherwise}
\end{cases}
\end{equation}

The steps are similar to the previous example (i.e. same sequence of transformations, initial conditions for the heat equation):
\begin{equation}
u(x,0)=
\begin{cases}
e^{\alpha x}(S-K),\quad \textrm{if}~x>0\\
0, \quad\textrm{otherwise}
\end{cases}
\end{equation}
and similar evaluation of the integral.

Finally the option price
\begin{equation}
V(S,t) = SN(d_1)-Ke^{-r(T-t)}N(d_2)
\end{equation}
where $N$ is the distribution function of a normalized normal distribution and  $d_1=\frac{\textrm{log}(\frac{S}{K}+\left(r+\frac{\sigma^2}{2}\right)(T-t)}{\sigma\sqrt{T-t}},\quad d_2=d_1-\sigma\sqrt{T-t}$.

Figure~\ref{fig:call_option} shows the option price as a function of the stock price for various values of $t$.

\begin{figure}[htb]
	\centering
	\includegraphics[width=0.9\textwidth]{figures/call_option_price}
	\caption{Option price as a function of the underlying price for various values of $t$.}
	\label{fig:call_option}
\end{figure} 

\chapter{Black-Scholes Model}

\section{Black-Scholes PDE: Derivation}
Suppose that stock price S follows a geometric Brownian motion

\begin{equation}
dS = \mu Sdt + \sigma SdW
\end{equation}

Further assumptions (besides GBP):
\begin{itemize}
\tightlist 
\item constant risk-less interest rate r;
\item no transaction costs;
\item it is possible to buy/sell any (also fractional) number of stocks; similarly with the cash;
\item no restrictions on short selling;
\item option is of European type.
\end{itemize}

We derive a partial differential equation for the price of a derivative.
Explicit solution for European call and put options.

Firstly, let us consider the case of a non-dividend paying stock.

$S$ = stock price,
$t$= time
$V=V(S, t)$ = option price

Define a portfolio made of 1 option and $\delta$ stocks. The value $P$ of this portfolio is:

\begin{equation}
P = V + \delta S
\end{equation}
So the change in the portfolio value
\begin{equation}
 dP = dV + \delta dS
\end{equation}

From the assumptions of $S$ following a geometric Brownian motion and from the It$\hat{o}$'s lemma:
\begin{equation}
dV = \left(\cfrac{\partial V}{\partial t} + \mu S  \cfrac{\partial V}{\partial S} + \cfrac{1}{2} \sigma^2 S^2  \cfrac{\partial^2 V}{\partial S^2} \right) dt + \sigma S \cfrac{\partial V}{\partial S} dW
\end{equation} 
Therefore:
\begin{equation}
dP =  \left(\cfrac{\partial V}{\partial t} + \mu S  \cfrac{\partial V}{\partial S} + \cfrac{1}{2} \sigma^2 S^2  \cfrac{\partial^2 V}{\partial S^2} + \delta \mu S\right) dt + \left(\sigma S  \cfrac{\partial V}{\partial S} + \delta \sigma S \right)dW
\end{equation}
We eliminate the randomness
\begin{equation}
\delta = -\cfrac{\partial V}{\partial S}
\end{equation}
Non-stochastic portfolio, from no-arbitrage consideration its value has to be the same as
if being on a bank account with interest rate
\begin{equation}
dP = rP dt
\end{equation}

Equality between the two expressions for $dP$ and substituting $P = V + \delta S$ 
\begin{equation}
\cfrac{\partial V}{\partial t} + \cfrac{1}{2} \sigma^2 S^2 \cfrac{\partial^2 V}{\partial S^2}+ rS \cfrac{\partial V}{\partial S} − rV = 0
\end{equation}

\subsection{Dividends in the Black-Scholes Derivation}
We consider continuous dividend rate $q$, i.e. holding a stock with value $S$ during the time differential $dt$ brings dividends $qSdt$. In this case the change in the portfolio value equals

\begin{equation}
dP = dV + \delta dS + \delta qSdt
\end{equation}

We proceed in the same way as before and obtain
\begin{equation}
\cfrac{\partial V}{\partial t} + \cfrac{1}{2} \sigma^2 S^2 \cfrac{\partial^2 V}{\partial S^2}+ (q-r)S \cfrac{\partial V}{\partial S} − rV = 0
\end{equation}

\section{Black-Scholes PDE: Solutions}

Mathematical formulation of the model:
Find solution
V(S, t) to the partial differential equation (so called Black-Scholes PDE)
% \partial V \partial t + 12 \sigma 2S2  \partial 2V \partial S2 + rS  \partial V \partial S − rV = 0
which holds for $S >0$, $t \in [0, T)$.

So far we have not used the fact that we consider an option. PDE holds for any derivative that pays a payoff at time $T$ depending on the stock price at this time
Type of the derivative determines the terminal condition at time $T$
In general $V(S, T)$ = payoff of the derivative 

\subsection{Black-Scholes PDE: Simple Solutions}
SOME SIMPLE "DERIVATIVES":
How to price the derivatives with the following payoffs:
\begin{itemize}
\tightlist
\item $V(S, T) = S$, it is in fact a stock $V(S, t) = S$
\item $V(S, T) = E$ with a certainty we obtain the cash $E$ ($V(S, t) = Ee^{−r(T−t)}$)
\end{itemize}
by substitution into the PDE we see that they are indeed solutions.
%EXERCISES:
%Find the price of a derivative with payoff
%$V(S, T) =S^n$,where $n\in N$.

%HINT: Look for the solution in the form
%$V(S, t) =A(t)S^n$

Find all solutions to the Black-Scholes PDE, which are independent of time, i.e., for which
$V(S, t) = V(S)$

\subsection{Binary Options}
Let us consider a binary option, which pays \$1 if the stock price is higher that $E$ at expiration time, otherwise its payoff is zero.
In this case
\begin{equation}
V(S, T) = 
	\begin{cases}
		1, \quad\textrm{if }S > E\\
		0, \quad\textrm{otherwise}	
	\end{cases}
\end{equation}


The main idea is to transform the Black-Scholes PDE to a heat equation.
Transformations are independent of the derivative type; it affects only the initial condition of the heat equation

%Partial differential equation
% \partial V \partial t +12 \sigma 2S2  \partial 2V \partial S2 + rS  \partial V \partial S − rV = 0
%which holds for S >0, t∈ [0, T).

Terminal condition $V(S, T)$ = payoff of the derivative for $S > 0$.

%STEP 1:
%Transformation x = ln(S/E) ∈ R, τ = T − t ∈ [0, T] and a
%new function
%Z (x, τ ) = V (Eex, T −τ)
%PDE for
%Z(x, τ), x∈R, τ∈ [0, T]:
% \partial Z \partial τ− 12 \sigma 2 \partial 2Z \partial x2+\sigma 22−r  \partial Z \partial x + rZ = 0,
%Z(x, 0) =V(Eex, T)
%
%STEP 2:
%Transformation to heat equation
%New function
%u(x, τ ) =eαx+βτZ(x, τ), where the constants
%α, β∈R are chosen so that the PDE for u is the heat equation
%
%Black-Scholes PDE: transformations
%PDE for
%u: \partial u \partial τ−\sigma 22 \partial 2u \partial x2+A \partial u \partial x+ Bu = 0 ,u(x, 0) = eαx Z(x, 0) = eαxV(Eex, T),
%where
%A= α\sigma 2+\sigma 22− r, B = (1 +α)r−β−α2\sigma 2+ α\sigma 22
%
%In order to have
%A=B = 0, we set
%α=r\sigma 2−12, β=r2+\sigma 28+r22\sigma 2
%
%STEP 3:
%Solution
%u(x, τ) of the PDE  \partial u \partial τ−\sigma 22 \partial 
%2
%u
% \partial x
%2 = 0 is given by
%Green formula
%u
%(x, τ ) =
%1
%√
%2
%\sigma 
%2πτ
%Z
%∞
%−∞
%e
%−
%(x
%−
%s)
%2
%2
%\sigma 
%2
%τ
%u
%(s, 0) ds .
%• We evaluate the integral and perform backward
%substitutions
%u
%(x, τ
%)
%→
%Z
%(x, τ
%)
%→
%V
%(S, t
%)
%V. Black-Scholes model: Derivation and solution – p.18/36
%Black-Scholes PDE: binary option (continued)
%• Transformations from the previous slides
%• We obtain the heat equation  \partial u \partial τ
%−
%\sigma 
%2
%2
% \partial 
%2
%u
% \partial x
%2 = 0 with initial
%condition
%u
%(x, 0) =
%eαx
%V
%(Ee
%x, T) =
%(
%eαx if Ee
%x > E
%0 otherwise
%=
%(
%eαx if x >
%0
%0 otherwise
%• Solution
%u
%(x, τ
%):
%u
%(x, τ ) =
%1
%√
%2π\sigma 
%2
%τ
%Z
%∞
%0
%e
%−
%(x
%−
%s)
%2
%2
%\sigma 
%2
%τ
%eαsds
%=
%. . .
%=
%eαx
%+
%1
%2
%\sigma 
%2τα
%2
%N
%
%x
%+
%\sigma 
%2τα
%\sigma 
%√
%τ
%
%where
%N
%(
%y) =
%√1
%2
%π
%R
%y
%−∞
%e
%−
%ξ
%2
%2 dξ is the cumulative distribution
%function of a normalized normal distribution
%V. Black-Scholes model: Derivation and solution – p.19/36
%Black-Scholes PDE: binary option (continued)
%• Option price
%V
%(S, t
%):
%V
%(S, t) =
%e
%−
%r(T
%−
%t)N
%(
%d
%2
%),
%where
%d
%2
%=
%log
%(
%S
%E
%)
%+
%
%r
%−
%\sigma 
%2
%2
%
%(T
%−
%t)
%\sigma 
%√
%T
%−
%t
%V. Black-Scholes model: Derivation and solution – p.20/36
%Black-Scholes PDE: call option
%• In this case
%V
%(S, T) = max(0, S
%−
%E) =
%(
%S
%−
%E if S > E
%0 otherwise
%• The same sequence of transformations; inital condition for
%the heat equation:
%u
%(x, 0) =
%(
%eαx
%(
%S
%−
%E
%) if x >
%0
%0 otherwise
%and similar evaluation of the integral
%• Option price:
%V
%(S, t) = SN
%(
%d
%1
%)
%− Ee
%−
%r(T
%−
%t)N
%(
%d
%2
%),
%where
%N is the distribution function of a normalized normal
%distribution and
%d
%1
%=
%ln
%S
%E +(
%r
%+
%\sigma 
%2
%2 )(
%T
%−
%t)
%\sigma 
%√
%T
%−
%t
%, d
%2
%=
%d
%1
%−
%\sigma 
%√
%T
%−
%t
%V. Black-Scholes model: Derivation and solution – p.21/36
%Black-Scholes PDE: call option
%HOMEWORK:
%Solve the Black-Scholes PDE for a call option on a stock which
%pays continuous dividends and write it in the form
%V
%(S, t) = Se
%−
%q(T
%−
%t)N
%(
%d
%1
%)
%− Ee
%−
%r(T
%−
%t)N
%(
%d
%2
%),
%where
%N
%(
%x) =
%√1
%2
%π
%R
%x
%−∞
%e
%−
%ξ
%2
%2 dξ is the distribution function of a
%normalized normal distribution
%N(0, 1) and
%d
%1
%=
%ln
%S
%E + (
%r
%−
%q
%+
%\sigma 
%2
%2 )(
%T
%−
%t
%)
%\sigma 
%√
%T
%−
%t , d
%2
%=
%d
%1
%−
%\sigma 
%√
%T
%−
%t
%NOTE: The PDE is different, so the transformations have to be
%adjusted (do the same steps for the new equation)
%V. Black-Scholes model: Derivation and solution – p.22/36
%Black-Scholes PDE: call option
%Payoff (i.e., terminal condition at time
%t
%=
%T = 1) and solution
%V
%(S, t
%) for selected times
%t:
%20 40 60 25 35 30 50 70 45 55 65
%0
%20
%10
%−2
%2
%4
%6
%8
%12
%14
%16
%18
%22
%stock price
%option price
%payoff
%t = 0
%t = 0.25
%t = 0.5
%t = 0.75
%V. Black-Scholes model: Derivation and solution – p.23/36
%Black-Scholes PDE: put option
%FORMULATION OF THE PROBLEM
%• Partial differential equation
% \partial V
% \partial t
%+
%1
%2
%\sigma 
%2
%S
%2
% \partial 
%2
%V
% \partial S
%2
%+ rS  \partial V \partial S
%− rV = 0
%which holds for S >
%0, t
%∈ [0, T].
%• Terminal condition:
%V
%(S, T) = max(0, E
%−
%S
%)
%for S >
%0
%V. Black-Scholes model: Derivation and solution – p.24/36
%Black-Scholes PDE: put option
%APPROACH I.
%• The same sequence of computations as in the case of a
%call option
%APPROACH II.
%• We use the linearity of the Black- Scholes PDE and the
%solution for a call which we have already found
%We show the application of the latter approach.
%V. Black-Scholes model: Derivation and solution – p.25/36
%Black-Scholes PDE: putoption
%• Recall that for the payoffs of a call and a put we have
%−[call payoff] + [put payoff] + [stock price] =
%E
%• Hence:
%[put payoff] = [call payoff] −
%S
%+
%E
%• Black-Scholes PDE is linear: a linear combination of
%solutions is again a solution
%V. Black-Scholes model: Derivation and solution – p.26/36
%Black-Scholes PDE: put option
%• Recall the solutions for
%V
%(S, T) =
%S and
%V
%(S, T) =
%E
%(page 13):
%terminal condition solution
%max(0, S
%−
%E
%)
%V call
%(S, t
%)
%S
%S
%E Ee
%−
%r(T
%−
%t)
%• From the linearity:
%terminal condition solution
%max(0, S
%−
%E
%)
%−
%S
%+
%E
%V call
%(S, t
%)
%−
%S
%+ Ee
%−
%r(T
%−
%t)
%• Since [put payoff] = max(0, S
%−
%E
%)
%−
%S
%+
%E, we get
%V put
%(S, t) =
%V call
%(S, t
%)
%−
%S
%+ Ee
%−
%r(T
%−
%t)
%V. Black-Scholes model: Derivation and solution – p.27/36
%Solution for a put option
%• The solution
%V put
%(S, t) =
%V call
%(S, t
%)
%−
%S
%+ Ee
%−
%r(T
%−
%t)
%can be written in a similar form as the solution for a call
%option:
%V ep
%(S, t) = Ee
%−
%r(T
%−
%t)N
%(
%−
%d
%2
%)
%− SN
%(
%−
%d
%1
%),
%where N, d
%1, d
%2 are the same as before
%V. Black-Scholes model: Derivation and solution – p.28/36
%Put option - example
%Payoff (i.e. terminal condition at time
%t
%=
%T = 1) and solution
%V
%(S, t
%) for selected times
%t:
%35 40 45 50 55 60 65
%0
%10
%2
%4
%6
%8
%12
%14
%16
%18
%stock price
%option price
%payoff
%t = 0
%t = 0.25
%t = 0.5
%t = 0.75
%V. Black-Scholes model: Derivation and solution – p.29/36
%Put option - alternative computation
%Comics about negative volatility on the webpage of Espen Haug:
%http://www.espenhaug.com/collector/collector.html
%V. Black-Scholes model: Derivation and solution – p.30/36
%Put option - alternative computation
%• A nightmare about negative volatility:
%• Not only a dream... according to internet, it really exists
%and is connected with professor Shiryaev from Moscow...
%V. Black-Scholes model: Derivation and solution – p.31/36
%Put option - alternative computation
%QUESTION: Why does this computation work?
%V. Black-Scholes model: Derivation and solution – p.32/36
%Stocks paying dividends
%•
%HOMEWORK:
%Solve the Black-Scholes equation for a put option, if the
%underlying stock pays continuous dividends.
%HINT:
%◦ In this case,
%V (S, t) =
%S is not a solution
%◦ What is the solution satisfying the terminal condition
%V (S, T) =
%S? Use
%financial interpretation and check your answer by substituting it into the PDE
%•
%HOMEWORK:
%Denote
%V
%(S, t; E, r, q
%) the price of an option with exercise
%price
%E, if the interest rate is
%r and the dividend rate is
%q. Show that
%V put
%(S, t; E, r, q) =
%V call
%(E, t; S, q, r
%)
%HINT: How do the terms
%d
%1
%d
%2 change when replacing
%S
%↔
%E, r
%↔
%q
%?
%V. Black-Scholes model: Derivation and solution – p.33/36
%Combined strategies
%• From the linearity of the Black-Scholes PDE: if the strategy
%is a linear combination of call and put options, then its price
%is the same linear combination of the call and put options
%prices
%• It does not necessarily hold in other models:
%◦ consider a model with some transaction costs; it is not
%equivalent
%◦ whether we hedge the options independenty
%◦ or we hedge the portfolio - in this case, we might be
%able to reduce transaction costs
%V. Black-Scholes model: Derivation and solution – p.34/36
%Combined strategies
%EXAMPLE:
%• we buy call options with exerise prices E1, E3 and sell two
%call options with exercise prices E2, with exercise prices
%satisfying E1 < E2 < E3 and E1 + E3 = 2E2.
%• Payoff of the strategy can be written as
%V (S, T) = max(S−E1, 0) − 2 max(S−E2, 0)+max(S−E3, 0)
%• Hence its Black-Scholes price is:
%V (S, t) = V call(S, t; E1) − 2V call(S, t; E2) + V call(S, t; E3)
%V. Black-Scholes model: Derivation and solution – p.35/36
%Combined strategies
%• Numerical example - butterfly with T = 1:
%10 30 50 70 90 15 25 35 20 40 60 80 45 55 65 75 85
%0
%20
%10
%2
%4
%6
%8
%12
%14
%16
%18
%stock price
%price of the strategy
%payoff
%t = 0
%t = 0.25
%t = 0.5
%t = 0.75
%Butterfly option strategy
%V. Black-Scholes model: De
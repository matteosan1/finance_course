\chapter{Interpolation, Discount Factors and Forward Rates}\label{interpolation---practical-lesson-3}

In this chapter we will start to see the first applications of \texttt{python} to financial calculations.
In particular we will consider discount curves and forward rates, implementing the first utilities that will fill our financial module.
In doing so we will review a widely used mathematical tool: \emph{interpolation}.

\section{Linear interpolation}\label{linear-interpolation}

Consider to have few data points, obtained by sampling or experimenting. These points represent the values of a not well known function \(f(x)\), where \(x\) is an independent variable (e.g.~in recording a trip: distances at certain times, \(d = f(t)\)).

It may be necessary to estimate values of the function $f$ at values for which we don't have samples.
Interpolation is a method of "constructing" new points within the range of the known data.

Let's clarify the technique with an example.
Assume you are going on holidays by car and that luckily there isn't much traffic so that you can drive at constant speed (which gives a linear relation between traveled space and time i.e.~\(s = v \cdot t\), which means that if you plot the distances \(s\) as a function of the time \(t\) you get a line with slope \(v\)), see Fig.~\ref{fig:samples_for_interpolation}.

\begin{figure}
  \centering
  \includegraphics[width=0.7\textwidth]{interp_example1.png}
  \caption{An example of sampling of traveled distances at some time. The red point shows an additional sample taken after the trip velocity has been reduced.}
  \label{fig:samples_for_interpolation}
\end{figure}

Given two samples of the car traveled distance \(s_1\) and \(s_2\) taken at two different times \(t_1\) and \(t_2\) you can linearly interpolate to find your position at different times using the following relations:

\[s = (1 - w)\cdot s_1 + w \cdot s_2\]
where $t$ is a generic time at which we want to know the distance $s$ and \(w = \cfrac{t - t_1}{t_2 - t_1}\).

\subsubsection{Derivation}
The equation of a line for two points
\((t_1, s_1)\) and \((t_2, s_2)\) can be written as:

\[\frac{t - t_1}{t_2 - t_1} = \frac{s - s_1}{s_2 - s_1}\]

Setting again \(w = \cfrac{t - t_1}{t_2 - t_1}\) and solving for \(s\) we find the desired solution:

\[(s_2 - s_1)\cdot w = s - s_1\quad\Rightarrow\quad s = (1 - w)\cdot s_1 + w \cdot s_2\]

This formula can also be understood as a weighted average where the weights are inversely related to the distance from the end points to the unknown point ($w_1 = (1 - w) = \cfrac{t_2 - t}{t_2 -t_1}, w_2 = w$), the closer point has more influence than the farther point.

Back to our example, if
\(s_1 = 25.75~\mathrm{km}\;(@t_1 = 15~\mathrm{min})\) and
\(s_2 = 171.7~\mathrm{km}\;(@t_2 = 100~\mathrm{min})\) let's find distance traveled in 1 hour (interpolation):

\begin{tcolorbox}[breakable, size=fbox, boxrule=1pt, pad at break*=1mm,colback=cellbackground, colframe=cellborder]
\begin{Verbatim}[commandchars=\\\{\}]
\PY{n}{s\PYZus{}1} \PY{o}{=} \PY{l+m+mf}{25.75} \PY{c+c1}{\PYZsh{} distance in km}
\PY{n}{t\PYZus{}1} \PY{o}{=} \PY{l+m+mi}{15}    \PY{c+c1}{\PYZsh{} elapsed time in minutes}
\PY{n}{s\PYZus{}2} \PY{o}{=} \PY{l+m+mf}{171.7}
\PY{n}{t\PYZus{}2} \PY{o}{=} \PY{l+m+mi}{100}

\PY{n}{t} \PY{o}{=} \PY{l+m+mi}{60}

\PY{n}{w} \PY{o}{=} \PY{p}{(}\PY{n}{t} \PY{o}{\PYZhy{}} \PY{n}{t\PYZus{}1}\PY{p}{)}\PY{o}{/}\PY{p}{(}\PY{n}{t\PYZus{}2} \PY{o}{\PYZhy{}} \PY{n}{t\PYZus{}1}\PY{p}{)}
\PY{n}{s} \PY{o}{=} \PY{p}{(}\PY{l+m+mi}{1} \PY{o}{\PYZhy{}} \PY{n}{w}\PY{p}{)}\PY{o}{*}\PY{n}{s\PYZus{}1} \PY{o}{+} \PY{n}{w}\PY{o}{*}\PY{n}{s\PYZus{}2}

\PY{n+nb}{print} \PY{p}{(}\PY{l+s+s2}{\PYZdq{}}\PY{l+s+si}{\PYZob{}:.1f\PYZcb{}}\PY{l+s+s2}{ km}\PY{l+s+s2}{\PYZdq{}}\PY{o}{.}\PY{n}{format}\PY{p}{(}\PY{n}{s}\PY{p}{)}\PY{p}{)}

103.0 km
\end{Verbatim}
\end{tcolorbox}

Always interpret critically your results to guess if they make sense or not. In the previous example we certainly expected something between 25.75 and 171.7~km (our range ends) furthermore since we are looking for the distance at a time which is almost halfway the interval, the result will be somehow in the middle or around 98.6~km. This is indeed more or less what we have got.
This simple reasoning should be applied every time you have a result to quickly judge it.

If we believe the relation between our variable stays the same ($f(t)$ still linear), we can use the same formula to \emph{extrapolate} values \emph{outside} our initial sample. For example if we keep the same constant velocity in our trip we could check the distance traveled after 3 hours:

\begin{tcolorbox}[breakable, size=fbox, boxrule=1pt, pad at break*=1mm,colback=cellbackground, colframe=cellborder]
\begin{Verbatim}[commandchars=\\\{\}]
\PY{n}{s\PYZus{}1} \PY{o}{=} \PY{l+m+mf}{25.75} \PY{c+c1}{\PYZsh{} distance in km}
\PY{n}{t\PYZus{}1} \PY{o}{=} \PY{l+m+mi}{15}    \PY{c+c1}{\PYZsh{} elapsed time in minutes}
\PY{n}{s\PYZus{}2} \PY{o}{=} \PY{l+m+mf}{171.7}
\PY{n}{t\PYZus{}2} \PY{o}{=} \PY{l+m+mi}{100}

\PY{n}{t} \PY{o}{=} \PY{l+m+mi}{180}

\PY{n}{w} \PY{o}{=} \PY{p}{(}\PY{n}{t} \PY{o}{\PYZhy{}} \PY{n}{t\PYZus{}1}\PY{p}{)}\PY{o}{/}\PY{p}{(}\PY{n}{t\PYZus{}2} \PY{o}{\PYZhy{}} \PY{n}{t\PYZus{}1}\PY{p}{)}
\PY{n}{s} \PY{o}{=} \PY{p}{(}\PY{l+m+mi}{1} \PY{o}{\PYZhy{}} \PY{n}{w}\PY{p}{)}\PY{o}{*}\PY{n}{s\PYZus{}1} \PY{o}{+} \PY{n}{w}\PY{o}{*}\PY{n}{s\PYZus{}2}

\PY{n+nb}{print} \PY{p}{(}\PY{l+s+s2}{\PYZdq{}}\PY{l+s+si}{\PYZob{}:.1f\PYZcb{}}\PY{l+s+s2}{ km}\PY{l+s+s2}{\PYZdq{}}\PY{o}{.}\PY{n}{format}\PY{p}{(}\PY{n}{s}\PY{p}{)}\PY{p}{)}

309.1 km
\end{Verbatim}
\end{tcolorbox}

\subsection{Log-linear interpolation}\label{log-linear-interpolation}
When the function $f$ that we want to interpolate is an exponential we can fall back to the previous case by a simple variable transformation. 
Assume the following is the relationship between $p$ and $h$, two generic variables:

\[p = \mathrm{exp}(c \cdot h)\]

Applying the logarithm to both sides of the equation gives:

\[s = \mathrm{log}(p) = \mathrm{log}(\mathrm{exp}(c \cdot h)) = c \cdot h\]
so there is linear relation between the new variable $s$ and $h$. At this point we can use the results of the previous section to interpolate for values of $s$, just remember to exponentiate the result to get the correct $p$. In formulas:

\[w = \frac{h - h_1}{h_2 - h_1}\]

\[s = (1 - w)\cdot s_1 + w \cdot s_2\qquad (\mathrm{remember \;now }\;s = \mathrm{log}(p))\]

\[p = \mathrm{exp}(s)\]

Let's see another example. Atmospheric pressure decreases with the altitude (i.e.~the highest you flight the lower is the pressure) following an exponential law:

\[p = p_0\cdot e^{-\alpha h}\]
where
\begin{itemize}
\tightlist
\item
  \(h\) is the altitude
\item
  \(p_0\) is the pressure at sea level
\item
  \(\alpha\) is a constant
\end{itemize}

Taking the logarithm of each side of the equation I get a linear relation which can be interpolated as seen before:

\[s = \mathrm{log}(p) = \mathrm{log}(p_0\cdot e^{-\alpha h})\propto - \alpha \cdot h\]

Now assume that we have measured
\(p_1 = 90~\mathrm{kPa}\;(h_1 = 1000~\mathrm{m})\) and
\(p_2 = 40~\mathrm{kPa}\;(h_1 = 7000~\mathrm{m})\) what will be the
atmospheric pressure on top of the Mont Blanc (\(4812~\mathrm{m}\)) ? and on top of Mount Everest (\(8848~\mathrm{m}\)) ?

\begin{tcolorbox}[breakable, size=fbox, boxrule=1pt, pad at break*=1mm,colback=cellbackground, colframe=cellborder]
\begin{Verbatim}[commandchars=\\\{\}]
\PY{c+c1}{\PYZsh{} pressure on top of the Mont Blanc (interpolation)}
\PY{k+kn}{from} \PY{n+nn}{math} \PY{k}{import} \PY{n}{log}\PY{p}{,} \PY{n}{exp}

\PY{c+c1}{\PYZsh{} first we take the logarithm of our measurements to use the linear }
\PY{c+c1}{\PYZsh{} relation to interpolate}
\PY{n}{h\PYZus{}1} \PY{o}{=} \PY{l+m+mi}{1000} \PY{c+c1}{\PYZsh{} height in meters}
\PY{n}{s\PYZus{}1} \PY{o}{=} \PY{n}{log}\PY{p}{(}\PY{l+m+mi}{90}\PY{p}{)} \PY{c+c1}{\PYZsh{} logarithm of the pressure at heigth h1}
\PY{n}{h\PYZus{}2} \PY{o}{=} \PY{l+m+mi}{7000} \PY{c+c1}{\PYZsh{} height in meters}
\PY{n}{s\PYZus{}2} \PY{o}{=} \PY{n}{log}\PY{p}{(}\PY{l+m+mi}{40}\PY{p}{)} \PY{c+c1}{\PYZsh{} logarithm of the pressure at heigth h2}

\PY{n}{h} \PY{o}{=} \PY{l+m+mi}{4812}

\PY{n}{w} \PY{o}{=} \PY{p}{(}\PY{n}{h} \PY{o}{\PYZhy{}} \PY{n}{h\PYZus{}1}\PY{p}{)}\PY{o}{/}\PY{p}{(}\PY{n}{h\PYZus{}2} \PY{o}{\PYZhy{}} \PY{n}{h\PYZus{}1}\PY{p}{)}
\PY{n}{s} \PY{o}{=} \PY{p}{(}\PY{l+m+mi}{1} \PY{o}{\PYZhy{}} \PY{n}{w}\PY{p}{)}\PY{o}{*}\PY{n}{s\PYZus{}1} \PY{o}{+} \PY{n}{w}\PY{o}{*}\PY{n}{s\PYZus{}2}

\PY{n+nb}{print} \PY{p}{(}\PY{l+s+s2}{\PYZdq{}}\PY{l+s+si}{\PYZob{}:.1f\PYZcb{}}\PY{l+s+s2}{ kPa}\PY{l+s+s2}{\PYZdq{}}\PY{o}{.}\PY{n}{format}\PY{p}{(}\PY{n}{exp}\PY{p}{(}\PY{n}{s}\PY{p}{)}\PY{p}{)}\PY{p}{)}

53.8 kPa
\end{Verbatim}
\end{tcolorbox}

\begin{tcolorbox}[breakable, size=fbox, boxrule=1pt, pad at break*=1mm,colback=cellbackground, colframe=cellborder]
\begin{Verbatim}[commandchars=\\\{\}]
\PY{c+c1}{\PYZsh{} pressure on top of the Mount Everest (extrapolation)}
\PY{k+kn}{from} \PY{n+nn}{math} \PY{k}{import} \PY{n}{log}\PY{p}{,} \PY{n}{exp}

\PY{c+c1}{\PYZsh{} first we take the logarithm of our measurements to use the linear }
\PY{c+c1}{\PYZsh{} relation to interpolate}
\PY{n}{h\PYZus{}1} \PY{o}{=} \PY{l+m+mi}{1000} \PY{c+c1}{\PYZsh{} height in meters}
\PY{n}{s\PYZus{}1} \PY{o}{=} \PY{n}{log}\PY{p}{(}\PY{l+m+mi}{90}\PY{p}{)} \PY{c+c1}{\PYZsh{} logarithm of the pressure at heigth h1}
\PY{n}{h\PYZus{}2} \PY{o}{=} \PY{l+m+mi}{7000} \PY{c+c1}{\PYZsh{} height in meters}
\PY{n}{s\PYZus{}2} \PY{o}{=} \PY{n}{log}\PY{p}{(}\PY{l+m+mi}{40}\PY{p}{)} \PY{c+c1}{\PYZsh{} logarithm of the pressure at heigth h2}
\PY{n}{h} \PY{o}{=} \PY{l+m+mi}{8848}

\PY{n}{w} \PY{o}{=} \PY{p}{(}\PY{n}{h} \PY{o}{\PYZhy{}} \PY{n}{h\PYZus{}1}\PY{p}{)}\PY{o}{/}\PY{p}{(}\PY{n}{h\PYZus{}2} \PY{o}{\PYZhy{}} \PY{n}{h\PYZus{}1}\PY{p}{)}
\PY{n}{s} \PY{o}{=} \PY{p}{(}\PY{l+m+mi}{1} \PY{o}{\PYZhy{}} \PY{n}{w}\PY{p}{)}\PY{o}{*}\PY{n}{s\PYZus{}1} \PY{o}{+} \PY{n}{w}\PY{o}{*}\PY{n}{s\PYZus{}2}

\PY{n+nb}{print} \PY{p}{(}\PY{l+s+s2}{\PYZdq{}}\PY{l+s+si}{\PYZob{}:.1f\PYZcb{}}\PY{l+s+s2}{ kPa}\PY{l+s+s2}{\PYZdq{}}\PY{o}{.}\PY{n}{format}\PY{p}{(}\PY{n}{exp}\PY{p}{(}\PY{n}{s}\PY{p}{)}\PY{p}{)}\PY{p}{)}

31.2 kPa
\end{Verbatim}
\end{tcolorbox}

In this case we check our results by plotting the found pressures on top of the $P$ vs $h$ plot shown on Wikipedia, see Fig~\ref{fig:Pvsh}.

\begin{figure}
\centering
\includegraphics[width=0.7\linewidth]{Atmospheric_Pressure_vs._Altitude.png}
\caption{Atmospheric pressure versus altitude (Wikipedia). Green points
represent our measurements, red points represent
interpolation/extrapolation.}
\label{fig:Pvsh}
\end{figure}

\subsection{Limitations of Interpolation}
Interpolation is just an approximation and works well when either the function $f$ is linear or we are trying to interpolate between two points that are close enough to believe that $f$ is almost linear in that interval.

It can be easily demonstrated that the linear approximation between two points of a given function $f(x)$ gets worse with the second derivative of the function that is approximated ($f''(x)$). This is intuitively correct: the "curvier" the function is, the worse the approximation made with simple linear interpolation becomes, see Fig.~\ref{fig:sine_interp} where we try to interpolate a sine function.

\begin{figure}
  \centering
  \includegraphics[width=0.7\textwidth]{wrong_interp.png}
  \caption{Trying to approximate a sine function with a line is clearly not going to work unless the interpolation interval is very small.}
  \label{fig:sine_interp}
\end{figure}

To improve the approximation accuracy with complicated curves a polynomial of higher order can be used ($𝑝(𝑥)=𝑎_0 + 𝑎_1 𝑥+ 𝑎_2 𝑥^2+\cdots$), for example in the evaluation of the natural logarithm and trigonometric functions. It has to be clear however that going to higher degrees does not always help (for those interested \href{https://en.wikipedia.org/wiki/Runge%27s_phenomenon}{see Runge's phenomenon}).

\section{Discount curve interpolation}\label{discount-curve-interpolation}

Finally we can come back to finance. Since discount factors are derived from a discrete set of dates we may need to find the factor at some different date and clearly we can use interpolation to do it.
Now we will see how to implement a \texttt{python} function which interpolates some given discount factors.
Needed data:

\begin{itemize}
\tightlist
\item a list of pillars dates specifying the value dates of the given discount factors, \(t_0,...,t_{n-1}\);
\item a list of given discount factors, \(D(t_0),...,D(t_{n-1})\);
\item a pricing date (`today' date) which corresponds to \(t=0\).
\end{itemize}

The input argument to the function will be the value date at which we want to interpolate the discount factor. Since the discount factor can be expressed as \(D=e^{-r(T-t)}\) the function will use a log-linear interpolation to return the value at a date not included in the given pillars.
More technically we can say that we are doing a linear interpolation over time in the log space:

\[d(t_i):=\mathrm{ln}(D(t_i))\]

\[d(t) = (1-w)d(t_i) + wd(t_{i+1});\qquad w=\frac{t-t_i}{t_{i+1}-t_i}\]

\[D(t) = \mathrm{exp}(d(t))\]
where again \(i\) is such that \(t_i \le t \le t_{i+1}\)

This time instead of reinventing the wheel and performing the interpolation with our own code, we'll use the function \texttt{interp} provided by the module \texttt{numpy}; this function linearly interpolates between the provided points to estimate the value of $f$ at some "new" $x$.
Say we want to interpolate the points at $x = 2.5$ given the following values:

\begin{tcolorbox}[breakable, size=fbox, boxrule=1pt, pad at break*=1mm,colback=cellbackground, colframe=cellborder]
\begin{Verbatim}[commandchars=\\\{\}]
\PY{k+kn}{import} \PY{n+nn}{numpy} \PY{k}{as} \PY{n+nn}{np}

\PY{n}{xp} \PY{o}{=} \PY{p}{[}\PY{l+m+mi}{0}\PY{p}{,} \PY{l+m+mi}{1}\PY{p}{,} \PY{l+m+mi}{5}\PY{p}{]}
\PY{n}{fp} \PY{o}{=} \PY{p}{[}\PY{l+m+mi}{0}\PY{p}{,} \PY{l+m+mi}{2}\PY{p}{,} \PY{l+m+mi}{4}\PY{p}{]}
\PY{n}{np}\PY{o}{.}\PY{n}{interp}\PY{p}{(}\PY{l+m+mf}{2.5}\PY{p}{,} \PY{n}{xp}\PY{p}{,} \PY{n}{fp}\PY{p}{)}

2.75
\end{Verbatim}
\end{tcolorbox}

Assume we have three discount factors instead:
\begin{tcolorbox}[breakable, size=fbox, boxrule=1pt, pad at break*=1mm,colback=cellbackground, colframe=cellborder]
\begin{Verbatim}[commandchars=\\\{\}]
\PY{c+c1}{\PYZsh{} import modules and objects that we need}
\PY{k+kn}{from} \PY{n+nn}{datetime} \PY{k}{import} \PY{n}{date}
\PY{k+kn}{import} \PY{n+nn}{numpy}\PY{o}{,} \PY{n+nn}{math}
\PY{k+kn}{from} \PY{n+nn}{matplotlib} \PY{k}{import} \PY{n}{pyplot} \PY{k}{as} \PY{n}{plt}
\PY{k+kn}{import} \PY{n+nn}{matplotlib}\PY{n+nn}{.}\PY{n+nn}{dates} \PY{k}{as} \PY{n+nn}{mdates} 
\PY{c+c1}{\PYZsh{} with this notation we tell python to use mdates as an alias }
\PY{c+c1}{\PYZsh{} for matplotlib.dates}

\PY{c+c1}{\PYZsh{} define the input data}
\PY{n}{today\PYZus{}date} \PY{o}{=} \PY{n}{date}\PY{p}{(}\PY{l+m+mi}{2019}\PY{p}{,} \PY{l+m+mi}{10}\PY{p}{,} \PY{l+m+mi}{1}\PY{p}{)}

\PY{n}{pillar\PYZus{}dates} \PY{o}{=} \PY{p}{[}\PY{n}{date}\PY{p}{(}\PY{l+m+mi}{2019}\PY{p}{,} \PY{l+m+mi}{10}\PY{p}{,} \PY{l+m+mi}{1}\PY{p}{)}\PY{p}{,} \PY{n}{date}\PY{p}{(}\PY{l+m+mi}{2020}\PY{p}{,} \PY{l+m+mi}{10}\PY{p}{,} \PY{l+m+mi}{1}\PY{p}{)}\PY{p}{,} \PY{n}{date}\PY{p}{(}\PY{l+m+mi}{2021}\PY{p}{,} \PY{l+m+mi}{10}\PY{p}{,} \PY{l+m+mi}{1}\PY{p}{)}\PY{p}{]}
\PY{n}{discount\PYZus{}factors} \PY{o}{=} \PY{p}{[}\PY{l+m+mf}{1.0}\PY{p}{,} \PY{l+m+mf}{0.97}\PY{p}{,} \PY{l+m+mf}{0.72}\PY{p}{]}
\end{Verbatim}
\end{tcolorbox}
    
Let's see what this fake discount curve looks like when plotted on a graph:

\begin{tcolorbox}[breakable, size=fbox, boxrule=1pt, pad at break*=1mm,colback=cellbackground, colframe=cellborder]
\begin{Verbatim}[commandchars=\\\{\}]
\PY{n}{plt}\PY{o}{.}\PY{n}{plot}\PY{p}{(}\PY{n}{pillar\PYZus{}dates}\PY{p}{,} \PY{n}{discount\PYZus{}factors}\PY{p}{,} \PY{n}{marker}\PY{o}{=}\PY{l+s+s1}{\PYZsq{}}\PY{l+s+s1}{o}\PY{l+s+s1}{\PYZsq{}}\PY{p}{)}
\PY{n}{plt}\PY{o}{.}\PY{n}{gca}\PY{p}{(}\PY{p}{)}\PY{o}{.}\PY{n}{xaxis}\PY{o}{.}\PY{n}{set\PYZus{}major\PYZus{}formatter}\PY{p}{(}\PY{n}{mdates}\PY{o}{.}\PY{n}{DateFormatter}\PY{p}{(}\PY{l+s+s1}{\PYZsq{}}\PY{l+s+s1}{\PYZpc{}}\PY{l+s+s1}{m/}\PY{l+s+si}{\PYZpc{}d}\PY{l+s+s1}{/}\PY{l+s+s1}{\PYZpc{}}\PY{l+s+s1}{Y}\PY{l+s+s1}{\PYZsq{}}\PY{p}{)}\PY{p}{)}
\PY{n}{plt}\PY{o}{.}\PY{n}{gca}\PY{p}{(}\PY{p}{)}\PY{o}{.}\PY{n}{xaxis}\PY{o}{.}\PY{n}{set\PYZus{}major\PYZus{}locator}\PY{p}{(}\PY{n}{mdates}\PY{o}{.}\PY{n}{YearLocator}\PY{p}{(}\PY{p}{)}\PY{p}{)}
\PY{n}{plt}\PY{o}{.}\PY{n}{grid}\PY{p}{(}\PY{k+kc}{True}\PY{p}{)}
\PY{n}{plt}\PY{o}{.}\PY{n}{show}\PY{p}{(}\PY{p}{)}
\end{Verbatim}
\end{tcolorbox}
\vfill
\begin{figure}[h]
  \centering
  \includegraphics[width=0.7\textwidth]{lecture_3_10_0.png}
\end{figure}
    
Since it is a computation that from now on we need to perform quite often it is convenient to write a function that compute the discount factor at arbitrary dates.

\begin{tcolorbox}[breakable, size=fbox, boxrule=1pt, pad at break*=1mm,colback=cellbackground, colframe=cellborder]
\begin{Verbatim}[commandchars=\\\{\}]
\PY{c+c1}{\PYZsh{} define the df function}
\PY{k}{def} \PY{n+nf}{df}\PY{p}{(}\PY{n}{d, pillar_dates, discount_factors}\PY{p}{)}\PY{p}{:}
    \PY{c+c1}{\PYZsh{} first thing we need to do is to apply the logarithm function}
    \PY{c+c1}{\PYZsh{} to the discount factors since we are doing log-linear and}
    \PY{c+c1}{\PYZsh{} not just linear interpolation}
    \PY{n}{log\PYZus{}discount\PYZus{}factors} \PY{o}{=} \PY{p}{[}\PY{p}{]}
    \PY{k}{for} \PY{n}{discount\PYZus{}factor} \PY{o+ow}{in} \PY{n}{discount\PYZus{}factors}\PY{p}{:}
        \PY{n}{log\PYZus{}discount\PYZus{}factors}\PY{o}{.}\PY{n}{append}\PY{p}{(}\PY{n}{math}\PY{o}{.}\PY{n}{log}\PY{p}{(}\PY{n}{discount\PYZus{}factor}\PY{p}{)}\PY{p}{)}
    
    \PY{c+c1}{\PYZsh{} perform the linear interpolation of the log discount factors}
    \PY{n}{interpolated\PYZus{}log\PYZus{}discount\PYZus{}factor} \PY{o}{=} \PYZbs{}
        \PY{n}{numpy}\PY{o}{.}\PY{n}{interp}\PY{p}{(}\PY{n}{d}\PY{p}{,} \PY{n}{pillar\PYZus{}dates}\PY{p}{,} \PY{n}{log\PYZus{}discount\PYZus{}factors}\PY{p}{)}
    
    \PY{c+c1}{\PYZsh{} return the interpolated discount factor}
    \PY{k}{return} \PY{n}{math}\PY{o}{.}\PY{n}{exp}\PY{p}{(}\PY{n}{interpolated\PYZus{}log\PYZus{}discount\PYZus{}factor}\PY{p}{)}
\end{Verbatim}
\end{tcolorbox}

This is almost OK, \textbf{but it won't work} because \texttt{numpy.interp} only accepts numbers or a list of numbers as argument i.e.~it doesn't automatically convert or interpret dates as numbers so doesn't know how to interpolate them. So we need to do the conversion ourselves before passing the dates into the interpolation function.
The following updated version of our function converts the pillar dates into ``pillar days'' i.e. each date is replaced by the number of days today ($t_0$):

\begin{tcolorbox}[breakable, size=fbox, boxrule=1pt, pad at break*=1mm,colback=cellbackground, colframe=cellborder]
\begin{Verbatim}[commandchars=\\\{\}]
\PY{k}{def} \PY{n+nf}{df}\PY{p}{(}\PY{n}{d, today_date, pillar_dates, discount_factors}\PY{p}{)}\PY{p}{:}
    \PY{c+c1}{\PYZsh{} first thing we need to do is to apply the logarithm function}
    \PY{c+c1}{\PYZsh{} to the discount factors since we are doing log-linear and}
    \PY{c+c1}{\PYZsh{} not just linear interpolation}
    \PY{n}{log\PYZus{}discount\PYZus{}factors} \PY{o}{=} \PY{p}{[}\PY{p}{]}
    \PY{k}{for} \PY{n}{discount\PYZus{}factor} \PY{o+ow}{in} \PY{n}{discount\PYZus{}factors}\PY{p}{:}
        \PY{n}{log\PYZus{}discount\PYZus{}factors}\PY{o}{.}\PY{n}{append}\PY{p}{(}\PY{n}{math}\PY{o}{.}\PY{n}{log}\PY{p}{(}\PY{n}{discount\PYZus{}factor}\PY{p}{)}\PY{p}{)}
    
    \PY{c+c1}{\PYZsh{} convert the pillar dates to pillar \PYZsq{}days\PYZsq{}}
    \PY{c+c1}{\PYZsh{} i.e. number of days from today}
    \PY{c+c1}{\PYZsh{} to write shorter code we can use this NEW notation}
    \PY{c+c1}{\PYZsh{} which condenses for and list creation in one line}
    \PY{n}{pillar\PYZus{}days} \PY{o}{=} \PYZbs{}
        \PY{p}{[}\PY{p}{(}\PY{n}{pillar\PYZus{}date} \PY{o}{\PYZhy{}} \PY{n}{today\PYZus{}date}\PY{p}{)}\PY{o}{.}\PY{n}{days} \PY{k}{for} \PY{n}{pillar\PYZus{}date} \PY{o+ow}{in} \PY{n}{pillar\PYZus{}dates}\PY{p}{]}
    
    \PY{c+c1}{\PYZsh{} obviously we need to do the same to the value date}
    \PY{c+c1}{\PYZsh{} argument of the df function}
    \PY{n}{d\PYZus{}days} \PY{o}{=} \PY{p}{(}\PY{n}{d} \PY{o}{\PYZhy{}} \PY{n}{today\PYZus{}date}\PY{p}{)}\PY{o}{.}\PY{n}{days}
    
    \PY{c+c1}{\PYZsh{} perform the linear interpolation of the log discount factors}
    \PY{n}{interpolated\PYZus{}log\PYZus{}discount\PYZus{}factor} \PY{o}{=} \PYZbs{}
        \PY{n}{numpy}\PY{o}{.}\PY{n}{interp}\PY{p}{(}\PY{n}{d\PYZus{}days}\PY{p}{,} \PY{n}{pillar\PYZus{}days}\PY{p}{,} \PY{n}{log\PYZus{}discount\PYZus{}factors}\PY{p}{)}
    
    \PY{c+c1}{\PYZsh{} return the interpolated discount factor}
    \PY{k}{return} \PY{n}{math}\PY{o}{.}\PY{n}{exp}\PY{p}{(}\PY{n}{interpolated\PYZus{}log\PYZus{}discount\PYZus{}factor}\PY{p}{)}
\end{Verbatim}
\end{tcolorbox}

Now we can use the \texttt{df} function to get discount factors on value dates between the given pillar dates:

\begin{tcolorbox}[breakable, size=fbox, boxrule=1pt, pad at break*=1mm,colback=cellbackground, colframe=cellborder]
\begin{Verbatim}[commandchars=\\\{\}]
\PY{n}{d0} \PY{o}{=} \PY{n}{date}\PY{p}{(}\PY{l+m+mi}{2020}\PY{p}{,} \PY{l+m+mi}{1}\PY{p}{,} \PY{l+m+mi}{1}\PY{p}{)}
\PY{n}{df0} \PY{o}{=} \PY{n}{df}\PY{p}{(}\PY{n}{d0, today_date, pillar_dates, discount_factors}\PY{p}{)}
\PY{n+nb}{print} \PY{p}{(}\PY{n}{df0}\PY{p}{)}

0.9923728228571693
\end{Verbatim}
\end{tcolorbox}

\begin{tcolorbox}[breakable, size=fbox, boxrule=1pt, pad at break*=1mm,colback=cellbackground, colframe=cellborder]
\begin{Verbatim}[commandchars=\\\{\}]
\PY{n}{d1} \PY{o}{=} \PY{n}{date}\PY{p}{(}\PY{l+m+mi}{2021}\PY{p}{,} \PY{l+m+mi}{1}\PY{p}{,} \PY{l+m+mi}{1}\PY{p}{)}
\PY{n}{df1} \PY{o}{=} \PY{n}{df}\PY{p}{(}\PY{n}{d1, today_date, pillar_dates, discount_factors}\PY{p}{)}
\PY{n+nb}{print} \PY{p}{(}\PY{n}{df1}\PY{p}{)}

0.8997999273630835
\end{Verbatim}
\end{tcolorbox}

Another very useful way to check the correctness of a result is by plotting data, so let's see what these look like when plotted on a semi-log graph and if they make sense:
    
\begin{tcolorbox}[breakable, size=fbox, boxrule=1pt, pad at break*=1mm,colback=cellbackground, colframe=cellborder]
\begin{Verbatim}[commandchars=\\\{\}]
\PY{k+kn}{from} \PY{n+nn}{matplotlib} \PY{k}{import} \PY{n}{pyplot} \PY{k}{as} \PY{n}{plt}
\PY{k+kn}{import} \PY{n+nn}{matplotlib}\PY{n+nn}{.}\PY{n+nn}{dates} \PY{k}{as} \PY{n+nn}{mdates}

\PY{n}{plt}\PY{o}{.}\PY{n}{semilogy}\PY{p}{(}\PY{n}{pillar\PYZus{}dates}\PY{p}{,} \PY{n}{discount\PYZus{}factors}\PY{p}{,} \PY{n}{marker}\PY{o}{=}\PY{l+s+s1}{\PYZsq{}}\PY{l+s+s1}{o}\PY{l+s+s1}{\PYZsq{}}\PY{p}{)}
\PY{n}{plt}\PY{o}{.}\PY{n}{semilogy}\PY{p}{(}\PY{n}{d0}\PY{p}{,}\PY{n}{df0} \PY{p}{,} \PY{n}{marker}\PY{o}{=}\PY{l+s+s1}{\PYZsq{}}\PY{l+s+s1}{X}\PY{l+s+s1}{\PYZsq{}}\PY{p}{)}
\PY{n}{plt}\PY{o}{.}\PY{n}{semilogy}\PY{p}{(}\PY{n}{d1}\PY{p}{,}\PY{n}{df1} \PY{p}{,} \PY{n}{marker}\PY{o}{=}\PY{l+s+s1}{\PYZsq{}}\PY{l+s+s1}{X}\PY{l+s+s1}{\PYZsq{}}\PY{p}{)}
\PY{n}{plt}\PY{o}{.}\PY{n}{gca}\PY{p}{(}\PY{p}{)}\PY{o}{.}\PY{n}{xaxis}\PY{o}{.}\PY{n}{set\PYZus{}major\PYZus{}formatter}\PY{p}{(}\PY{n}{mdates}\PY{o}{.}\PY{n}{DateFormatter}\PY{p}{(}\PY{l+s+s1}{\PYZsq{}}\PY{l+s+s1}{\PYZpc{}}\PY{l+s+s1}{m/}\PY{l+s+si}{\PYZpc{}d}\PY{l+s+s1}{/}\PY{l+s+s1}{\PYZpc{}}\PY{l+s+s1}{Y}\PY{l+s+s1}{\PYZsq{}}\PY{p}{)}\PY{p}{)}
\PY{n}{plt}\PY{o}{.}\PY{n}{gca}\PY{p}{(}\PY{p}{)}\PY{o}{.}\PY{n}{xaxis}\PY{o}{.}\PY{n}{set\PYZus{}major\PYZus{}locator}\PY{p}{(}\PY{n}{mdates}\PY{o}{.}\PY{n}{YearLocator}\PY{p}{(}\PY{p}{)}\PY{p}{)}
\PY{n}{plt}\PY{o}{.}\PY{n}{grid}\PY{p}{(}\PY{k+kc}{True}\PY{p}{)}
\PY{n}{plt}\PY{o}{.}\PY{n}{show}\PY{p}{(}\PY{p}{)}
\end{Verbatim}
\end{tcolorbox}
\vfill
\begin{figure}[h]
  \centering
  \includegraphics[width=0.7\textwidth]{lecture_3_15_0.png}
\end{figure}

Let's see what these look like when plotted on a linear graph instead:
    
\begin{tcolorbox}[breakable, size=fbox, boxrule=1pt, pad at break*=1mm,colback=cellbackground, colframe=cellborder]
\begin{Verbatim}[commandchars=\\\{\}]
\PY{k+kn}{from} \PY{n+nn}{matplotlib} \PY{k}{import} \PY{n}{pyplot} \PY{k}{as} \PY{n}{plt}
\PY{k+kn}{import} \PY{n+nn}{matplotlib}\PY{n+nn}{.}\PY{n+nn}{dates} \PY{k}{as} \PY{n+nn}{mdates}
\PY{n}{plt}\PY{o}{.}\PY{n}{plot}\PY{p}{(}\PY{n}{pillar\PYZus{}dates}\PY{p}{,} \PY{n}{discount\PYZus{}factors}\PY{p}{,} \PY{n}{marker}\PY{o}{=}\PY{l+s+s1}{\PYZsq{}}\PY{l+s+s1}{o}\PY{l+s+s1}{\PYZsq{}}\PY{p}{)}
\PY{n}{plt}\PY{o}{.}\PY{n}{plot}\PY{p}{(}\PY{n}{d0}\PY{p}{,}\PY{n}{df0} \PY{p}{,} \PY{n}{marker}\PY{o}{=}\PY{l+s+s1}{\PYZsq{}}\PY{l+s+s1}{X}\PY{l+s+s1}{\PYZsq{}}\PY{p}{)}
\PY{n}{plt}\PY{o}{.}\PY{n}{plot}\PY{p}{(}\PY{n}{d1}\PY{p}{,}\PY{n}{df1} \PY{p}{,} \PY{n}{marker}\PY{o}{=}\PY{l+s+s1}{\PYZsq{}}\PY{l+s+s1}{X}\PY{l+s+s1}{\PYZsq{}}\PY{p}{)}
\PY{n}{plt}\PY{o}{.}\PY{n}{gca}\PY{p}{(}\PY{p}{)}\PY{o}{.}\PY{n}{xaxis}\PY{o}{.}\PY{n}{set\PYZus{}major\PYZus{}formatter}\PY{p}{(}\PY{n}{mdates}\PY{o}{.}\PY{n}{DateFormatter}\PY{p}{(}\PY{l+s+s1}{\PYZsq{}}\PY{l+s+s1}{\PYZpc{}}\PY{l+s+s1}{m/}\PY{l+s+si}{\PYZpc{}d}\PY{l+s+s1}{/}\PY{l+s+s1}{\PYZpc{}}\PY{l+s+s1}{Y}\PY{l+s+s1}{\PYZsq{}}\PY{p}{)}\PY{p}{)}
\PY{n}{plt}\PY{o}{.}\PY{n}{gca}\PY{p}{(}\PY{p}{)}\PY{o}{.}\PY{n}{xaxis}\PY{o}{.}\PY{n}{set\PYZus{}major\PYZus{}locator}\PY{p}{(}\PY{n}{mdates}\PY{o}{.}\PY{n}{YearLocator}\PY{p}{(}\PY{p}{)}\PY{p}{)}
\PY{n}{plt}\PY{o}{.}\PY{n}{grid}\PY{p}{(}\PY{k+kc}{True}\PY{p}{)}
\PY{n}{plt}\PY{o}{.}\PY{n}{show}\PY{p}{(}\PY{p}{)}
\end{Verbatim}
\end{tcolorbox}
\vfill
\begin{figure}[h]
  \centering
  \includegraphics[width=0.7\textwidth]{lecture_3_16_0.png}
\end{figure}

Discrepancies in the linear plot are most likely due to rounding.

\section{Forward Rates}\label{calculating-forward-rates}

A forward rate is an interest rate applicable to a financial transaction that will take place in the future. Forward rates are calculated from the spot rate by exploiting the no arbitrage condition which states that investing at rate \(r_1\) for the period \((0, T_1)\) and then \emph{re-investing} at rate \(r_{1,2}\) for the time period \((T_1, T_2)\) is equivalent to invest at rate \(r_2\) for the full time period \((0, T_2)\). Essentially two investors shouldn't be able to earn money from arbitraging between different interest periods. That said:

\[(1+r_1 T_1)(1+r_{1,2}(T_2 - T_1)) = 1 + r_2 T_2\]

Solving for \(r_{1,2}\) leads to

\[F(T_1, T_2) = r_{1,2} = \frac{1}{T_2-T_1}\Big(\frac{D(T_1)}{D(T_2)} - 1 \Big)~~~~\textrm{(where $D{(T_i)}=\frac{1}{1+r_iT_{i}}$)}\]

\begin{tcolorbox}[breakable, size=fbox, boxrule=1pt, pad at break*=1mm,colback=cellbackground, colframe=cellborder]
\begin{Verbatim}[commandchars=\\\{\}]
\PY{k+kn}{from} \PY{n+nn}{datetime} \PY{k}{import} \PY{n}{date}
\PY{k+kn}{import} \PY{n+nn}{numpy}\PY{o}{,} \PY{n+nn}{math}

\PY{n}{today\PYZus{}date} \PY{o}{=} \PY{n}{date} \PY{p}{(}\PY{l+m+mi}{2019}\PY{p}{,} \PY{l+m+mi}{1}\PY{p}{,} \PY{l+m+mi}{1}\PY{p}{)}

\PY{n}{pillar\PYZus{}dates} \PY{o}{=} \PY{p}{[}\PY{n}{date}\PY{p}{(}\PY{l+m+mi}{2019} \PY{p}{,} \PY{l+m+mi}{1} \PY{p}{,}\PY{l+m+mi}{1}\PY{p}{)}\PY{p}{,} 
                \PY{n}{date}\PY{p}{(}\PY{l+m+mi}{2020}\PY{p}{,} \PY{l+m+mi}{1}\PY{p}{,} \PY{l+m+mi}{1}\PY{p}{)}\PY{p}{,} 
                \PY{n}{date}\PY{p}{(}\PY{l+m+mi}{2021}\PY{p}{,} \PY{l+m+mi}{10} \PY{p}{,}\PY{l+m+mi}{1}\PY{p}{)}\PY{p}{]}
\PY{n}{discount\PYZus{}factors} \PY{o}{=} \PY{p}{[}\PY{l+m+mf}{1.0}\PY{p}{,} \PY{l+m+mf}{0.97}\PY{p}{,} \PY{l+m+mf}{0.72}\PY{p}{]}

\PY{k}{def} \PY{n+nf}{forward\PYZus{}rate}\PY{p}{(}\PY{n}{t1}\PY{p}{,} \PY{n}{t2, today_date, pillar_dates, discount_factors}\PY{p}{)}\PY{p}{:}
    \PY{k}{return} \PY{l+m+mf}{365.0}\PY{o}{/}\PY{p}{(}\PY{n}{t2}\PY{o}{\PYZhy{}}\PY{n}{t1}\PY{p}{)}\PY{o}{.}\PY{n}{days} \PY{o}{*}
        \PY{p}{(}\PY{n}{df}\PY{p}{(}\PY{n}{t1, today_date, pillar_dates, discount_factors}\PY{p}{)} \PY{o}{/}
        \PY{n}{df}\PY{p}{(}\PY{n}{t2, today_date, pillar_dates, discount_factors}\PY{p}{)} \PY{o}{\PYZhy{}} \PY{l+m+mi}{1}\PY{p}{)}

\PY{n}{forward\PYZus{}rate}\PY{p}{(}\PY{n}{date}\PY{p}{(}\PY{l+m+mi}{2019}\PY{p}{,} \PY{l+m+mi}{2}\PY{p}{,} \PY{l+m+mi}{1}\PY{p}{)}\PY{p}{,} \PY{n}{date}\PY{p}{(}\PY{l+m+mi}{2019}\PY{p}{,} \PY{l+m+mi}{8}\PY{p}{,} \PY{l+m+mi}{1}\PY{p}{),}
             \PY{p}{today_date, pillar_dates, discount_factors}\PY{p}{)}
\end{Verbatim}
\end{tcolorbox}

\subsection{2008 Financial Crisis}\label{financial-crisis}

Looking at the historical series of the Euribor (6M) rate versus the Eonia Overnight Indexed Swap (OIS-6M) rate over the time interval 2006-2011 it becomes apparent how before August 2007 the two rates display strictly overlapping trends differing of no more than 6 bps.

\begin{figure}[h]
\centering
\includegraphics[width=0.7\linewidth]{credit_crunch.png}
\end{figure}

In August 2007 however we observe a sudden increase of the Euribor rate and a simultaneous decrease of the OIS rate that leads to the explosion of the corresponding basis spread, touching the peak of 222 bps in October 2008, when Lehman Brothers filed for bankruptcy. Successively the basis has sensibly reduced and stabilized between 40 bps and 60 bps (notice that the pre-crisis level has never been recovered). The same effect is observed for other similar couples of series, e.g.~Euribor 3M vs OIS 3M.

The reason of the abrupt divergence between the Euribor and OIS rates can be explained by considering both the monetary policy decisions adopted by international authorities in response to the financial turmoil, and the impact of the credit crunch on both credit and liquidity risk perception of the market, coupled with the different financial meaning and dynamics of these rates.

\begin{itemize}
\tightlist
\item
  The Euribor rate is the reference rate for over-the-counter (OTC)
  transactions in the Euro area. It is defined as the rate at which
  Euro inter-bank deposits are being offered within the EMU zone by one
  prime bank to another at 11:00 a.m. Brussels time. The rate fixings
  for a strip of 15 maturities (from one day to one year) are
  constructed as the average of the rates submitted (excluding the
  highest and lowest 15\% tails) by a panel of 42 banks, selected
  among the EU banks with the highest volume of business in the Euro
  zone money markets, plus some large international bank from non-EU
  countries with important euro zone operations. \emph{Thus, Euribor
  rates reflect the average cost of funding of banks in the inter bank
  market at each given maturity. During the crisis the solvency and
  solidity of the whole financial sector was brought into question and
  the credit and liquidity risk and uremia associated to inter-bank
  counter-parties sharply increased.} The Euribor rates immediately
  reflected these dynamics and raise to their highest values over more
  than 10 years. As seen in the plot above, the Euribor 6M rate suddenly
  increased on August 2007 and reached 5.49\% on 10th October 2008.
\item
  The Eonia rate is the reference rate for overnight OTC transactions in
  the Euro area. It is constructed as the average rate of the overnight
  transactions (one day maturity deposits) executed during a given
  business day by a panel of banks on the inter-bank money market,
  weighted with the corresponding transaction volumes. \emph{The Eonia
  Contribution Panel coincides with the Euribor Contribution Panel, thus
  Eonia rate includes information on the short term (overnight)
  liquidity expectations of banks in the Euro money market. It is also
  used by the European Central Bank (ECB) as a method of effecting and
  observing the transmission of its monetary policy actions. During the
  crisis the central banks were mainly concerned about stabilizing the
  level of liquidity in the market, thus they reduced the level of the
  official rates.} Furthermore, the daily tenor of the Eonia rate makes
  negligible the credit and liquidity risks reflected on it: for this
  reason the OIS rates are considered the best proxies available in the
  market for the risk-free rate.
\end{itemize}

As a practical result of the divergence of the two indices, after the 2008 financial crisis, it is not possible anymore to use a single discount curve to correctly price forward rates of all tenors. For example, if we want to calculate the net present value of a forward 6-month Libor coupon, we need to simultaneously use two different discount curves:

\begin{itemize}
\tightlist
\item the 6-month Libor curve for determining the forward rate;
\item the EONIA curve for discounting the expected cash flow.
\end{itemize}

Our financial library will have to implement the following calculation:

\[\mathrm{NPV} = D_{\mathrm{EONIA}}(T_1) \times \frac{1}{T_2-T_1}\Big(\frac{D_{\mathrm{LIBOR}}(T_1)}{D_{\mathrm{LIBOR}}(T_2)} - 1 \Big)\]
and this will asked to be done in the exercises relative to this chapter.

To exploit the Object Oriented capabilities we will implement a \texttt{DiscountCurve} class, below a skeleton class to give you an idea of how could be this new class.
\begin{Shaded}
\begin{Highlighting}[]
\CommentTok{# here goes import statement of the needed modules}
\ImportTok{import}\NormalTok{ ABCD}
\ImportTok{from}\NormalTok{ XYZ }\ImportTok{import}\NormalTok{ xyz}

\CommentTok{# usually classes have CamelCase naming convention}
\KeywordTok{class}\NormalTok{ DiscountCurve:}

    \CommentTok{# the special __init__ method defines }
    \CommentTok{# how to construct instances of the class}
    \CommentTok{# so you need to identify the attributes you need to store }
    \CommentTok{# in the class defining a discount curve}
    \KeywordTok{def} \FunctionTok{__init__}\NormalTok{(}\VariableTok{self}\NormalTok{, ...):}

    \CommentTok{# then we want to add a method to compute the discount}
    \CommentTok{# factor at an arbitrary value date }
    \CommentTok{# using the data stored in the instance}
    \KeywordTok{def}\NormalTok{ df(}\VariableTok{self}\NormalTok{, param1, param2, ...):}
      \CommentTok{# the implementation can follow what we did in the }
      \CommentTok{# function we wrote last week but this time has to }
      \CommentTok{# use the class attributes}
      
    \CommentTok{# finally we want a method to calculates the forward rate }
    \CommentTok{# based on the discount curve data stored in the instance}
    \KeywordTok{def}\NormalTok{ forward_rate(}\VariableTok{self}\NormalTok{, param1, param2, ...):}
        \CommentTok{# here of course we can use the df method }
        \CommentTok{# implemented above to calculate the forward rate}
\end{Highlighting}
\end{Shaded}

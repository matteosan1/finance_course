    For a more detailed and comprhensive documentation I suggest to check
\href{https://matplotlib.org}{this page}

\hypertarget{canvas}{%
\subsubsection{Canvas}\label{canvas}}

The first element we need to plot is a \emph{canvas}, the white
rectangular space where we draw our data. \(\tt{Matplotlib}\) provides a
default canvas if no specific command is given (a 640x480 pixels space)
but you can customize it by specifying many different parameters, the
most commonly are:

\begin{itemize}
\tightlist
\item
  \(\tt{figsize}\): width, height in inches. If not provided, defaults
  to \([6.4, 4.8]\).
\item
  \(\tt{dpi}\): resolution of the figure. If not provided, defaults to
  100.
\item
  \(\tt{facecolor}\): the background color. If not provided, defaults to
  \(\tt{'white'}\), or \(\tt{'w'}\).
\end{itemize}

In the following example we create a canvas (8 inches x 5 inches) with a
resolution of 80 dots per inches (dpi) and a green background color.

    \begin{tcolorbox}[breakable, size=fbox, boxrule=1pt, pad at break*=1mm,colback=cellbackground, colframe=cellborder]
\prompt{In}{incolor}{1}{\boxspacing}
\begin{Verbatim}[commandchars=\\\{\}]
\PY{k+kn}{from} \PY{n+nn}{matplotlib} \PY{k}{import} \PY{n}{pyplot} \PY{k}{as} \PY{n}{plt}
\PY{n}{fig} \PY{o}{=} \PY{n}{plt}\PY{o}{.}\PY{n}{figure}\PY{p}{(}\PY{n}{figsize}\PY{o}{=}\PY{p}{(}\PY{l+m+mi}{8}\PY{p}{,}\PY{l+m+mi}{5}\PY{p}{)}\PY{p}{,} \PY{n}{dpi}\PY{o}{=}\PY{l+m+mi}{80}\PY{p}{,} \PY{n}{facecolor}\PY{o}{=}\PY{l+s+s1}{\PYZsq{}}\PY{l+s+s1}{green}\PY{l+s+s1}{\PYZsq{}}\PY{p}{)}
\end{Verbatim}
\end{tcolorbox}

    Once we have a canvas we cannot even draw it unless we a some plot to
show. So let's see the most used type of plots we can use.

\hypertarget{chart-object}{%
\subsubsection{Chart Object}\label{chart-object}}

Below a concise list of the main chart objects that are commonly used
when presenting financial data.

\hypertarget{histograms}{%
\paragraph{Histograms}\label{histograms}}

A histogram is an approximate representation of the distribution of
numerical data. To construct a histogram, the first step is to ``bin''
(or ``bucket'') the range of values (i.e.~divide the entire range of
values into a series of intervals) and then count how many values fall
into each interval.

If the bins are of equal size, a rectangle is erected over the bin with
height proportional to the frequency (i.e.~the number of cases in each
bin).

However, bins need not be of equal width; in that case, the erected
rectangle is defined to have its area proportional to the frequency of
cases in the bin. The vertical axis is then not the frequency but
frequency density (i.e.~the number of cases per unit of the variable on
the horizontal axis).

To create an histogram with \(\tt{matplotlib}\) it is enough to pass the
a list (or a \(\tt{numpy.array}\)) with data to represent to the
function \(\tt{hist}\) and then call \(\tt{show()}\) to actually draw
the plot (I have also used the previously defined canvas just to show
the results).

    \begin{tcolorbox}[breakable, size=fbox, boxrule=1pt, pad at break*=1mm,colback=cellbackground, colframe=cellborder]
\prompt{In}{incolor}{2}{\boxspacing}
\begin{Verbatim}[commandchars=\\\{\}]
\PY{n}{y} \PY{o}{=} \PY{p}{[}\PY{l+m+mf}{40.9}\PY{p}{,} \PY{l+m+mf}{49.6}\PY{p}{,} \PY{l+m+mf}{36.5}\PY{p}{,} \PY{l+m+mf}{43.7}\PY{p}{,} \PY{l+m+mf}{52.4}\PY{p}{,} \PY{l+m+mf}{37.0}\PY{p}{,} \PY{l+m+mf}{42.4}\PY{p}{,} \PY{l+m+mf}{38.4}\PY{p}{,} \PY{l+m+mf}{39.9}\PY{p}{,} \PY{l+m+mf}{40.6}\PY{p}{,} \PY{l+m+mf}{35.7}\PY{p}{,} \PY{l+m+mf}{42.4}\PY{p}{,} 
     \PY{l+m+mf}{47.7}\PY{p}{,} \PY{l+m+mf}{40.6}\PY{p}{,} \PY{l+m+mf}{47.6}\PY{p}{,} \PY{l+m+mf}{38.3}\PY{p}{,} \PY{l+m+mf}{43.6}\PY{p}{,} \PY{l+m+mf}{38.5}\PY{p}{,} \PY{l+m+mf}{43.5}\PY{p}{,} \PY{l+m+mf}{42.9}\PY{p}{,} \PY{l+m+mf}{42.1}\PY{p}{,} \PY{l+m+mf}{39.8}\PY{p}{,} \PY{l+m+mf}{35.8}\PY{p}{,} \PY{l+m+mf}{44.1}\PY{p}{,} 
     \PY{l+m+mf}{43.0}\PY{p}{,} \PY{l+m+mf}{45.4}\PY{p}{,} \PY{l+m+mf}{52.3}\PY{p}{,} \PY{l+m+mf}{41.5}\PY{p}{,} \PY{l+m+mf}{42.6}\PY{p}{,} \PY{l+m+mf}{45.4}\PY{p}{,} \PY{l+m+mf}{38.6}\PY{p}{,} \PY{l+m+mf}{37.3}\PY{p}{,} \PY{l+m+mf}{37.8}\PY{p}{,} \PY{l+m+mf}{48.1}\PY{p}{,} \PY{l+m+mf}{44.8}\PY{p}{,} \PY{l+m+mf}{32.7}\PY{p}{,} 
     \PY{l+m+mf}{42.3}\PY{p}{,} \PY{l+m+mf}{33.8}\PY{p}{,} \PY{l+m+mf}{43.1}\PY{p}{,} \PY{l+m+mf}{40.3}\PY{p}{]}

\PY{n}{fig} \PY{o}{=} \PY{n}{plt}\PY{o}{.}\PY{n}{figure}\PY{p}{(}\PY{n}{figsize}\PY{o}{=}\PY{p}{(}\PY{l+m+mi}{8}\PY{p}{,}\PY{l+m+mi}{5}\PY{p}{)}\PY{p}{,} \PY{n}{dpi}\PY{o}{=}\PY{l+m+mi}{80}\PY{p}{,} \PY{n}{facecolor}\PY{o}{=}\PY{l+s+s1}{\PYZsq{}}\PY{l+s+s1}{green}\PY{l+s+s1}{\PYZsq{}}\PY{p}{)}
\PY{n}{plt}\PY{o}{.}\PY{n}{hist}\PY{p}{(}\PY{n}{y}\PY{p}{)}
\PY{n}{plt}\PY{o}{.}\PY{n}{show}\PY{p}{(}\PY{p}{)}
\end{Verbatim}
\end{tcolorbox}

    \begin{center}
    \adjustimage{max size={0.9\linewidth}{0.9\paperheight}}{matplotlib_files/matplotlib_3_0.png}
    \end{center}
    { \hspace*{\fill} \\}
    
    If you are not satisfied with the default binning and/or you want to
change the range shown in the plot, those can be specified in the call
to \(\tt{hist}\) like this

    \begin{tcolorbox}[breakable, size=fbox, boxrule=1pt, pad at break*=1mm,colback=cellbackground, colframe=cellborder]
\prompt{In}{incolor}{3}{\boxspacing}
\begin{Verbatim}[commandchars=\\\{\}]
\PY{n}{fig} \PY{o}{=} \PY{n}{plt}\PY{o}{.}\PY{n}{figure}\PY{p}{(}\PY{n}{figsize}\PY{o}{=}\PY{p}{(}\PY{l+m+mi}{8}\PY{p}{,}\PY{l+m+mi}{5}\PY{p}{)}\PY{p}{)}
\PY{c+c1}{\PYZsh{} after data there is the number of equally spaced bins}
\PY{c+c1}{\PYZsh{} to be used, then the range}
\PY{n}{plt}\PY{o}{.}\PY{n}{hist}\PY{p}{(}\PY{n}{y}\PY{p}{,} \PY{l+m+mi}{50}\PY{p}{,} \PY{n+nb}{range}\PY{o}{=}\PY{p}{(}\PY{l+m+mi}{30}\PY{p}{,} \PY{l+m+mi}{50}\PY{p}{)}\PY{p}{)}
\PY{n}{plt}\PY{o}{.}\PY{n}{show}\PY{p}{(}\PY{p}{)}
\end{Verbatim}
\end{tcolorbox}

    \begin{center}
    \adjustimage{max size={0.9\linewidth}{0.9\paperheight}}{matplotlib_files/matplotlib_5_0.png}
    \end{center}
    { \hspace*{\fill} \\}
    
    Passing other parameters it is possible to modify the style of the
histogram, like its color. Colors can be specified by their names. For
the basic built-in colors, you can use a single letter

\begin{itemize}
\tightlist
\item
  b: blue
\item
  g: green
\item
  r: red
\item
  c: cyan
\item
  m: magenta
\item
  y: yellow
\item
  k: black
\item
  w: white
\end{itemize}

Plotting multiple histograms on the same canvas is as easy as calling
various times \(\tt{hist}\) (binnig should be the same for each
histogram).

To each object that is plotted can be associated a label, passing the
corresponding parameter to the object call, so that we can build a
legend. Latex symbols can be used whenever text is used, the characters
have to enclosed between dollar symbols (i.e.~``\(\mu\)''). Legend that
will be shown on the canvas with the command \(\tt{legend()}\).

A grid can be added to the plot with \(\tt{plt.grid(True)}\) for a nicer
look.

    \begin{tcolorbox}[breakable, size=fbox, boxrule=1pt, pad at break*=1mm,colback=cellbackground, colframe=cellborder]
\prompt{In}{incolor}{4}{\boxspacing}
\begin{Verbatim}[commandchars=\\\{\}]
\PY{n}{y1} \PY{o}{=} \PY{p}{[}\PY{l+m+mf}{40.9}\PY{p}{,} \PY{l+m+mf}{49.6}\PY{p}{,} \PY{l+m+mf}{36.5}\PY{p}{,} \PY{l+m+mf}{43.7}\PY{p}{,} \PY{l+m+mf}{52.4}\PY{p}{,} \PY{l+m+mf}{37.0}\PY{p}{,} \PY{l+m+mf}{42.4}\PY{p}{,} \PY{l+m+mf}{38.4}\PY{p}{,} \PY{l+m+mf}{39.9}\PY{p}{,} \PY{l+m+mf}{40.6}\PY{p}{,} \PY{l+m+mf}{35.7}\PY{p}{,} \PY{l+m+mf}{42.4}\PY{p}{,} 
      \PY{l+m+mf}{47.7}\PY{p}{,} \PY{l+m+mf}{40.6}\PY{p}{,} \PY{l+m+mf}{47.6}\PY{p}{,} \PY{l+m+mf}{38.3}\PY{p}{,} \PY{l+m+mf}{43.6}\PY{p}{,} \PY{l+m+mf}{38.5}\PY{p}{,} \PY{l+m+mf}{43.5}\PY{p}{,} \PY{l+m+mf}{42.9}\PY{p}{,} \PY{l+m+mf}{42.1}\PY{p}{,} \PY{l+m+mf}{39.8}\PY{p}{,} \PY{l+m+mf}{35.8}\PY{p}{,} \PY{l+m+mf}{44.1}\PY{p}{,} 
      \PY{l+m+mf}{43.0}\PY{p}{,} \PY{l+m+mf}{45.4}\PY{p}{,} \PY{l+m+mf}{52.3}\PY{p}{,} \PY{l+m+mf}{41.5}\PY{p}{,} \PY{l+m+mf}{42.6}\PY{p}{,} \PY{l+m+mf}{45.4}\PY{p}{,} \PY{l+m+mf}{38.6}\PY{p}{,} \PY{l+m+mf}{37.3}\PY{p}{,} \PY{l+m+mf}{37.8}\PY{p}{,} \PY{l+m+mf}{48.1}\PY{p}{,} \PY{l+m+mf}{44.8}\PY{p}{,} \PY{l+m+mf}{32.7}\PY{p}{,} 
      \PY{l+m+mf}{42.3}\PY{p}{,} \PY{l+m+mf}{33.8}\PY{p}{,} \PY{l+m+mf}{43.1}\PY{p}{,} \PY{l+m+mf}{40.3}\PY{p}{]}
\PY{n}{y2} \PY{o}{=} \PY{p}{[}\PY{l+m+mf}{40.6}\PY{p}{,} \PY{l+m+mf}{40.5}\PY{p}{,} \PY{l+m+mf}{37.3}\PY{p}{,} \PY{l+m+mf}{37.6}\PY{p}{,} \PY{l+m+mf}{39.0}\PY{p}{,} \PY{l+m+mf}{38.5}\PY{p}{,} \PY{l+m+mf}{36.0}\PY{p}{,} \PY{l+m+mf}{39.0}\PY{p}{,} \PY{l+m+mf}{36.8}\PY{p}{,} \PY{l+m+mf}{41.4}\PY{p}{,} \PY{l+m+mf}{41.9}\PY{p}{,} \PY{l+m+mf}{39.9}\PY{p}{,} 
      \PY{l+m+mf}{39.0}\PY{p}{,} \PY{l+m+mf}{37.7}\PY{p}{,} \PY{l+m+mf}{35.0}\PY{p}{,} \PY{l+m+mf}{37.9}\PY{p}{,} \PY{l+m+mf}{35.2}\PY{p}{,} \PY{l+m+mf}{39.5}\PY{p}{,} \PY{l+m+mf}{37.7}\PY{p}{,} \PY{l+m+mf}{38.4}\PY{p}{,} \PY{l+m+mf}{42.4}\PY{p}{,} \PY{l+m+mf}{38.1}\PY{p}{,} \PY{l+m+mf}{39.0}\PY{p}{,} \PY{l+m+mf}{34.7}\PY{p}{,} 
      \PY{l+m+mf}{37.1}\PY{p}{,} \PY{l+m+mf}{36.6}\PY{p}{,} \PY{l+m+mf}{37.0}\PY{p}{,} \PY{l+m+mf}{40.8}\PY{p}{,} \PY{l+m+mf}{39.0}\PY{p}{,} \PY{l+m+mf}{41.5}\PY{p}{]}

\PY{n}{fig} \PY{o}{=} \PY{n}{plt}\PY{o}{.}\PY{n}{figure}\PY{p}{(}\PY{n}{figsize}\PY{o}{=}\PY{p}{(}\PY{l+m+mi}{8}\PY{p}{,}\PY{l+m+mi}{5}\PY{p}{)}\PY{p}{)}
\PY{n}{plt}\PY{o}{.}\PY{n}{hist}\PY{p}{(}\PY{n}{y1}\PY{p}{,} \PY{l+m+mi}{50}\PY{p}{,} \PY{n+nb}{range}\PY{o}{=}\PY{p}{(}\PY{l+m+mi}{30}\PY{p}{,} \PY{l+m+mi}{50}\PY{p}{)}\PY{p}{,} \PY{n}{color}\PY{o}{=}\PY{l+s+s1}{\PYZsq{}}\PY{l+s+s1}{red}\PY{l+s+s1}{\PYZsq{}}\PY{p}{,} \PY{n}{label}\PY{o}{=}\PY{l+s+s1}{\PYZsq{}}\PY{l+s+s1}{Red Hist}\PY{l+s+s1}{\PYZsq{}}\PY{p}{)}
\PY{n}{plt}\PY{o}{.}\PY{n}{hist}\PY{p}{(}\PY{n}{y2}\PY{p}{,} \PY{l+m+mi}{50}\PY{p}{,} \PY{n+nb}{range}\PY{o}{=}\PY{p}{(}\PY{l+m+mi}{30}\PY{p}{,} \PY{l+m+mi}{50}\PY{p}{)}\PY{p}{,} \PY{n}{color}\PY{o}{=}\PY{l+s+s1}{\PYZsq{}}\PY{l+s+s1}{yellow}\PY{l+s+s1}{\PYZsq{}}\PY{p}{,} \PY{n}{label}\PY{o}{=}\PY{l+s+s1}{\PYZsq{}}\PY{l+s+s1}{Yellow Hist}\PY{l+s+s1}{\PYZsq{}}\PY{p}{)}
\PY{n}{plt}\PY{o}{.}\PY{n}{legend}\PY{p}{(}\PY{p}{)}
\PY{n}{plt}\PY{o}{.}\PY{n}{grid}\PY{p}{(}\PY{k+kc}{True}\PY{p}{)}
\PY{n}{plt}\PY{o}{.}\PY{n}{show}\PY{p}{(}\PY{p}{)}
\end{Verbatim}
\end{tcolorbox}

    \begin{center}
    \adjustimage{max size={0.9\linewidth}{0.9\paperheight}}{matplotlib_files/matplotlib_7_0.png}
    \end{center}
    { \hspace*{\fill} \\}
    
    \hypertarget{scatter}{%
\paragraph{Scatter}\label{scatter}}

A scatter plot is a type of plot using Cartesian coordinates to display
values for typically two variables for a set of data. If the points are
coded (color/shape/size), one additional variable can be displayed.

The data are displayed as a collection of points, each having the value
of one variable determining the position on the horizontal axis and the
value of the other variable determining the position on the vertical
axis.

To create such an object it is enough to call \(\tt{scatter}\) and pass
the \(x\) and \(y\) lists .

    \begin{tcolorbox}[breakable, size=fbox, boxrule=1pt, pad at break*=1mm,colback=cellbackground, colframe=cellborder]
\prompt{In}{incolor}{18}{\boxspacing}
\begin{Verbatim}[commandchars=\\\{\}]
\PY{n}{y1} \PY{o}{=} \PY{p}{[}\PY{l+m+mf}{26.2}\PY{p}{,} \PY{l+m+mf}{5.4}\PY{p}{,} \PY{l+m+mf}{7.7}\PY{p}{,} \PY{l+m+mf}{3.8}\PY{p}{,} \PY{l+m+mf}{24.7}\PY{p}{,} \PY{o}{\PYZhy{}}\PY{l+m+mf}{5.5}\PY{p}{,} \PY{l+m+mf}{36.4}\PY{p}{,} \PY{l+m+mf}{12.9}\PY{p}{,} \PY{l+m+mf}{25.2}\PY{p}{,} \PY{l+m+mf}{21.0}\PY{p}{,} \PY{l+m+mf}{39.6}\PY{p}{,} 
      \PY{l+m+mf}{5.9}\PY{p}{,} \PY{l+m+mf}{24.8}\PY{p}{,} \PY{l+m+mf}{25.7}\PY{p}{,} \PY{l+m+mf}{42.3}\PY{p}{,} \PY{l+m+mf}{21.5}\PY{p}{,} \PY{l+m+mf}{32.3}\PY{p}{,} \PY{l+m+mf}{26.7}\PY{p}{,} \PY{l+m+mf}{37.4}\PY{p}{,} \PY{l+m+mf}{44.3}\PY{p}{,} \PY{l+m+mf}{29.0}\PY{p}{,} 
      \PY{l+m+mf}{52.9}\PY{p}{,} \PY{l+m+mf}{52.0}\PY{p}{,} \PY{l+m+mf}{49.5}\PY{p}{,} \PY{l+m+mf}{55.0}\PY{p}{,} \PY{l+m+mf}{40.7}\PY{p}{,} \PY{l+m+mf}{47.8}\PY{p}{,} \PY{l+m+mf}{41.1}\PY{p}{,} \PY{l+m+mf}{49.3}\PY{p}{,} \PY{l+m+mf}{58.8}\PY{p}{,} \PY{l+m+mf}{48.1}\PY{p}{,} 
      \PY{l+m+mf}{52.5}\PY{p}{,} \PY{l+m+mf}{51.1}\PY{p}{,} \PY{l+m+mf}{51.0}\PY{p}{,} \PY{l+m+mf}{54.3}\PY{p}{,} \PY{l+m+mf}{62.4}\PY{p}{,} \PY{l+m+mf}{52.8}\PY{p}{,} \PY{l+m+mf}{67.8}\PY{p}{,} \PY{l+m+mf}{83.6}\PY{p}{,} \PY{l+m+mf}{75.9}\PY{p}{]}

\PY{n}{fig} \PY{o}{=} \PY{n}{plt}\PY{o}{.}\PY{n}{figure}\PY{p}{(}\PY{n}{figsize}\PY{o}{=}\PY{p}{(}\PY{l+m+mi}{8}\PY{p}{,}\PY{l+m+mi}{5}\PY{p}{)}\PY{p}{)}
\PY{n}{x} \PY{o}{=} \PY{n+nb}{range}\PY{p}{(}\PY{n+nb}{len}\PY{p}{(}\PY{n}{y1}\PY{p}{)}\PY{p}{)}
\PY{n}{plt}\PY{o}{.}\PY{n}{scatter}\PY{p}{(}\PY{n}{x}\PY{p}{,} \PY{n}{y1}\PY{p}{)}
\PY{n}{plt}\PY{o}{.}\PY{n}{show}\PY{p}{(}\PY{p}{)}
\end{Verbatim}
\end{tcolorbox}

    \begin{center}
    \adjustimage{max size={0.9\linewidth}{0.9\paperheight}}{matplotlib_files/matplotlib_9_0.png}
    \end{center}
    { \hspace*{\fill} \\}
    
    Other parameters can be passes to \(\tt{scatter}\) the most useful are

\begin{itemize}
\tightlist
\item
  \(\tt{label}\): label the scatter plot fot the legend;
\item
  \(\tt{color}\): set the color of the marker;
\item
  \(\tt{marker}\): set the sityle of the marker defined with a single
  character (+,*, o, x\ldots{});
\item
  \(\tt{s=}\): set the size of the marker
\end{itemize}

Anfd of course multiple scatter plot can be shown together in the same
canvas calling \(\tt{scatter}\) accordingly.

    \begin{tcolorbox}[breakable, size=fbox, boxrule=1pt, pad at break*=1mm,colback=cellbackground, colframe=cellborder]
\prompt{In}{incolor}{19}{\boxspacing}
\begin{Verbatim}[commandchars=\\\{\}]
\PY{n}{y2} \PY{o}{=} \PY{p}{[}\PY{l+m+mf}{22.2}\PY{p}{,} \PY{l+m+mf}{3.4}\PY{p}{,} \PY{l+m+mf}{7.7}\PY{p}{,} \PY{l+m+mf}{5.8}\PY{p}{,} \PY{l+m+mf}{28.7}\PY{p}{,} \PY{l+m+mf}{0.5}\PY{p}{,} \PY{l+m+mf}{44.4}\PY{p}{,} \PY{l+m+mf}{22.9}\PY{p}{,} \PY{l+m+mf}{37.2}\PY{p}{,} \PY{l+m+mf}{35.0}\PY{p}{,} \PY{l+m+mf}{55.6}\PY{p}{,} 
      \PY{l+m+mf}{23.9}\PY{p}{,} \PY{l+m+mf}{44.8}\PY{p}{,} \PY{l+m+mf}{47.7}\PY{p}{,} \PY{l+m+mf}{66.3}\PY{p}{,} \PY{l+m+mf}{47.5}\PY{p}{,} \PY{l+m+mf}{60.3}\PY{p}{,} \PY{l+m+mf}{56.7}\PY{p}{,} \PY{l+m+mf}{69.4}\PY{p}{,} \PY{l+m+mf}{78.3}\PY{p}{,} \PY{l+m+mf}{65.0}\PY{p}{,} 
      \PY{l+m+mf}{90.9}\PY{p}{,} \PY{l+m+mf}{92.0}\PY{p}{,} \PY{l+m+mf}{91.5}\PY{p}{,} \PY{l+m+mf}{99.0}\PY{p}{,} \PY{l+m+mf}{86.7}\PY{p}{,} \PY{l+m+mf}{95.8}\PY{p}{,} \PY{l+m+mf}{91.1}\PY{p}{,} \PY{l+m+mf}{101.3}\PY{p}{,} \PY{l+m+mf}{112.8}\PY{p}{,} 
      \PY{l+m+mf}{104.1}\PY{p}{,} \PY{l+m+mf}{110.5}\PY{p}{,} \PY{l+m+mf}{111.1}\PY{p}{,} \PY{l+m+mf}{113.0}\PY{p}{,} \PY{l+m+mf}{118.3}\PY{p}{,} \PY{l+m+mf}{128.4}\PY{p}{,} \PY{l+m+mf}{120.8}\PY{p}{,} \PY{l+m+mf}{137.8}\PY{p}{,} \PY{l+m+mf}{155.6}\PY{p}{,} \PY{l+m+mf}{149.9}\PY{p}{]}

\PY{n}{fig} \PY{o}{=} \PY{n}{plt}\PY{o}{.}\PY{n}{figure}\PY{p}{(}\PY{n}{figsize}\PY{o}{=}\PY{p}{(}\PY{l+m+mi}{8}\PY{p}{,}\PY{l+m+mi}{5}\PY{p}{)}\PY{p}{)}
\PY{n}{plt}\PY{o}{.}\PY{n}{scatter}\PY{p}{(}\PY{n+nb}{range}\PY{p}{(}\PY{n+nb}{len}\PY{p}{(}\PY{n}{y1}\PY{p}{)}\PY{p}{)}\PY{p}{,} \PY{n}{y1}\PY{p}{,} \PY{n}{s}\PY{o}{=}\PY{l+m+mi}{10}\PY{p}{,} \PY{n}{marker}\PY{o}{=}\PY{l+s+s2}{\PYZdq{}}\PY{l+s+s2}{o}\PY{l+s+s2}{\PYZdq{}}\PY{p}{,} \PY{n}{color}\PY{o}{=}\PY{l+s+s2}{\PYZdq{}}\PY{l+s+s2}{lightblue}\PY{l+s+s2}{\PYZdq{}}\PY{p}{,} \PY{n}{label}\PY{o}{=}\PY{l+s+s2}{\PYZdq{}}\PY{l+s+s2}{1st Scatter}\PY{l+s+s2}{\PYZdq{}}\PY{p}{)}
\PY{n}{plt}\PY{o}{.}\PY{n}{scatter}\PY{p}{(}\PY{n+nb}{range}\PY{p}{(}\PY{n+nb}{len}\PY{p}{(}\PY{n}{y2}\PY{p}{)}\PY{p}{)}\PY{p}{,} \PY{n}{y2}\PY{p}{,} \PY{n}{s}\PY{o}{=}\PY{l+m+mi}{30}\PY{p}{,} \PY{n}{marker}\PY{o}{=}\PY{l+s+s2}{\PYZdq{}}\PY{l+s+s2}{*}\PY{l+s+s2}{\PYZdq{}}\PY{p}{,} \PY{n}{color}\PY{o}{=}\PY{l+s+s2}{\PYZdq{}}\PY{l+s+s2}{brown}\PY{l+s+s2}{\PYZdq{}}\PY{p}{,} \PY{n}{label}\PY{o}{=}\PY{l+s+s2}{\PYZdq{}}\PY{l+s+s2}{2nd Scatter}\PY{l+s+s2}{\PYZdq{}}\PY{p}{)}
\PY{n}{plt}\PY{o}{.}\PY{n}{legend}\PY{p}{(}\PY{p}{)}
\PY{n}{plt}\PY{o}{.}\PY{n}{show}\PY{p}{(}\PY{p}{)}
\end{Verbatim}
\end{tcolorbox}

    \begin{center}
    \adjustimage{max size={0.9\linewidth}{0.9\paperheight}}{matplotlib_files/matplotlib_11_0.png}
    \end{center}
    { \hspace*{\fill} \\}
    
    \hypertarget{plot}{%
\paragraph{Plot}\label{plot}}

If you want to still plot \(x\) and \(y\) points but they should be
connected with a line you can use the function \(\tt{plot}\). It acts
similarly to \(\tt{scatter}\) but the graphical result will be
different.

For example below the same two scatter plots of the previous example are
plotted with \(\tt{plot}\)

    \begin{tcolorbox}[breakable, size=fbox, boxrule=1pt, pad at break*=1mm,colback=cellbackground, colframe=cellborder]
\prompt{In}{incolor}{20}{\boxspacing}
\begin{Verbatim}[commandchars=\\\{\}]
\PY{n}{fig} \PY{o}{=} \PY{n}{plt}\PY{o}{.}\PY{n}{figure}\PY{p}{(}\PY{n}{figsize}\PY{o}{=}\PY{p}{(}\PY{l+m+mi}{8}\PY{p}{,}\PY{l+m+mi}{5}\PY{p}{)}\PY{p}{)}
\PY{n}{plt}\PY{o}{.}\PY{n}{plot}\PY{p}{(}\PY{n}{x}\PY{p}{,} \PY{n}{y1}\PY{p}{,} \PY{n}{label}\PY{o}{=}\PY{l+s+s2}{\PYZdq{}}\PY{l+s+s2}{1st Plot}\PY{l+s+s2}{\PYZdq{}}\PY{p}{)}
\PY{n}{plt}\PY{o}{.}\PY{n}{plot}\PY{p}{(}\PY{n}{x}\PY{p}{,} \PY{n}{y2}\PY{p}{,} \PY{n}{label}\PY{o}{=}\PY{l+s+s2}{\PYZdq{}}\PY{l+s+s2}{2nd Plot}\PY{l+s+s2}{\PYZdq{}}\PY{p}{)}
\PY{n}{plt}\PY{o}{.}\PY{n}{legend}\PY{p}{(}\PY{p}{)}
\PY{n}{plt}\PY{o}{.}\PY{n}{show}\PY{p}{(}\PY{p}{)}
\end{Verbatim}
\end{tcolorbox}

    \begin{center}
    \adjustimage{max size={0.9\linewidth}{0.9\paperheight}}{matplotlib_files/matplotlib_13_0.png}
    \end{center}
    { \hspace*{\fill} \\}
    
    In this case the parameters ther parameters allow to control the line
style and the possibility to add the marker at each point

\begin{itemize}
\tightlist
\item
  \(\tt{label}\): label the scatter plot fot the legend;
\item
  \(\tt{color}\): set the color of the marker and the line;
\item
  \(\tt{marker}\): set the sityle of the marker defined with a single
  character (+,*, o, x\ldots{});
\item
  \(\tt{s}\): set the size of the marker;
\item
  \(\tt{linestyle}\): style of the line (`-', `--', `-.', `:');
\item
  \(\tt{linewidth}\): thickness of the line.
\end{itemize}

    \begin{tcolorbox}[breakable, size=fbox, boxrule=1pt, pad at break*=1mm,colback=cellbackground, colframe=cellborder]
\prompt{In}{incolor}{21}{\boxspacing}
\begin{Verbatim}[commandchars=\\\{\}]
\PY{n}{fig} \PY{o}{=} \PY{n}{plt}\PY{o}{.}\PY{n}{figure}\PY{p}{(}\PY{n}{figsize}\PY{o}{=}\PY{p}{(}\PY{l+m+mi}{8}\PY{p}{,}\PY{l+m+mi}{5}\PY{p}{)}\PY{p}{)}
\PY{n}{plt}\PY{o}{.}\PY{n}{plot}\PY{p}{(}\PY{n}{x}\PY{p}{,} \PY{n}{y1}\PY{p}{,} \PY{n}{marker}\PY{o}{=}\PY{l+s+s2}{\PYZdq{}}\PY{l+s+s2}{+}\PY{l+s+s2}{\PYZdq{}}\PY{p}{,} \PY{n}{linestyle}\PY{o}{=}\PY{l+s+s2}{\PYZdq{}}\PY{l+s+s2}{:}\PY{l+s+s2}{\PYZdq{}}\PY{p}{,} \PY{n}{color}\PY{o}{=}\PY{l+s+s2}{\PYZdq{}}\PY{l+s+s2}{darkgreen}\PY{l+s+s2}{\PYZdq{}}\PY{p}{,} \PY{n}{label}\PY{o}{=}\PY{l+s+s2}{\PYZdq{}}\PY{l+s+s2}{1st Plot}\PY{l+s+s2}{\PYZdq{}}\PY{p}{)}
\PY{n}{plt}\PY{o}{.}\PY{n}{plot}\PY{p}{(}\PY{n}{x}\PY{p}{,} \PY{n}{y2}\PY{p}{,} \PY{n}{marker}\PY{o}{=}\PY{l+s+s2}{\PYZdq{}}\PY{l+s+s2}{x}\PY{l+s+s2}{\PYZdq{}}\PY{p}{,} \PY{n}{linewidth}\PY{o}{=}\PY{l+m+mi}{5}\PY{p}{,} \PY{n}{color}\PY{o}{=}\PY{l+s+s2}{\PYZdq{}}\PY{l+s+s2}{pink}\PY{l+s+s2}{\PYZdq{}}\PY{p}{,} \PY{n}{label}\PY{o}{=}\PY{l+s+s2}{\PYZdq{}}\PY{l+s+s2}{2nd Plot}\PY{l+s+s2}{\PYZdq{}}\PY{p}{)}
\PY{n}{plt}\PY{o}{.}\PY{n}{legend}\PY{p}{(}\PY{p}{)}
\PY{n}{plt}\PY{o}{.}\PY{n}{show}\PY{p}{(}\PY{p}{)}
\end{Verbatim}
\end{tcolorbox}

    \begin{center}
    \adjustimage{max size={0.9\linewidth}{0.9\paperheight}}{matplotlib_files/matplotlib_15_0.png}
    \end{center}
    { \hspace*{\fill} \\}
    
    \hypertarget{plotting-a-function}{%
\subsubsection{Plotting a Function}\label{plotting-a-function}}

Functions can be plotted using the \(\tt{plot}\) method. They can be
functions from \(\tt{python}\) modules or user-defined.

As an exmple let's try to plot \(\mathrm{sin}(x/2)\), in the example we
will use \(\tt{numpy.arange}\) function which extend the functionalities
of the standard \(\tt{range}\) allowing for non integer steps.

In such cases it is always recommended to use \(\tt{numpy}\) functions
and not those defined in \(\tt{math}\) since they also accept arrays in
input for a faster processing (i.e.~it is not neede to loop over each
\(x\) value to compute the value function since it is done
automatically).

    \begin{tcolorbox}[breakable, size=fbox, boxrule=1pt, pad at break*=1mm,colback=cellbackground, colframe=cellborder]
\prompt{In}{incolor}{22}{\boxspacing}
\begin{Verbatim}[commandchars=\\\{\}]
\PY{k+kn}{from} \PY{n+nn}{numpy} \PY{k}{import} \PY{n}{arange}\PY{p}{,} \PY{n}{sin}

\PY{k}{def} \PY{n+nf}{func}\PY{p}{(}\PY{n}{x}\PY{p}{)}\PY{p}{:}
    \PY{k}{return} \PY{n}{sin}\PY{p}{(}\PY{n}{x}\PY{o}{/}\PY{l+m+mi}{2}\PY{p}{)}

\PY{n}{xs} \PY{o}{=} \PY{n}{arange}\PY{p}{(}\PY{o}{\PYZhy{}}\PY{l+m+mi}{10}\PY{p}{,} \PY{l+m+mi}{10}\PY{p}{,} \PY{l+m+mf}{0.01}\PY{p}{)}

\PY{n}{fig} \PY{o}{=} \PY{n}{plt}\PY{o}{.}\PY{n}{figure}\PY{p}{(}\PY{n}{figsize}\PY{o}{=}\PY{p}{(}\PY{l+m+mi}{8}\PY{p}{,}\PY{l+m+mi}{5}\PY{p}{)}\PY{p}{)}
\PY{n}{plt}\PY{o}{.}\PY{n}{plot}\PY{p}{(}\PY{n}{xs}\PY{p}{,} \PY{n}{func}\PY{p}{(}\PY{n}{xs}\PY{p}{)}\PY{p}{)}
\PY{n}{plt}\PY{o}{.}\PY{n}{show}\PY{p}{(}\PY{p}{)}
\end{Verbatim}
\end{tcolorbox}

    \begin{center}
    \adjustimage{max size={0.9\linewidth}{0.9\paperheight}}{matplotlib_files/matplotlib_17_0.png}
    \end{center}
    { \hspace*{\fill} \\}
    
    \hypertarget{axes}{%
\subsubsection{Axes}\label{axes}}

The axis of a any plot can be formatted with many options.

First of all the axis range can be controlled with
\(\tt{xlim(min, max}\) and \(\tt{ylim(min, max}\). Then labels can be
added with \(\tt{xlabel(myLabel)}\) and \(\tt{ylabel(myLabel)}\). A
global title to the plot can also be specified with
\(\tt{plt.title(myTitle)}\).

Referring to the previous example we can write

    \begin{tcolorbox}[breakable, size=fbox, boxrule=1pt, pad at break*=1mm,colback=cellbackground, colframe=cellborder]
\prompt{In}{incolor}{23}{\boxspacing}
\begin{Verbatim}[commandchars=\\\{\}]
\PY{n}{fig} \PY{o}{=} \PY{n}{plt}\PY{o}{.}\PY{n}{figure}\PY{p}{(}\PY{n}{figsize}\PY{o}{=}\PY{p}{(}\PY{l+m+mi}{8}\PY{p}{,}\PY{l+m+mi}{5}\PY{p}{)}\PY{p}{)}
\PY{n}{plt}\PY{o}{.}\PY{n}{plot}\PY{p}{(}\PY{n}{x}\PY{p}{,} \PY{n}{y1}\PY{p}{,} \PY{n}{marker}\PY{o}{=}\PY{l+s+s2}{\PYZdq{}}\PY{l+s+s2}{+}\PY{l+s+s2}{\PYZdq{}}\PY{p}{,} \PY{n}{linestyle}\PY{o}{=}\PY{l+s+s2}{\PYZdq{}}\PY{l+s+s2}{:}\PY{l+s+s2}{\PYZdq{}}\PY{p}{,} \PY{n}{color}\PY{o}{=}\PY{l+s+s2}{\PYZdq{}}\PY{l+s+s2}{darkgreen}\PY{l+s+s2}{\PYZdq{}}\PY{p}{,} \PY{n}{label}\PY{o}{=}\PY{l+s+s2}{\PYZdq{}}\PY{l+s+s2}{1st Plot}\PY{l+s+s2}{\PYZdq{}}\PY{p}{)}
\PY{n}{plt}\PY{o}{.}\PY{n}{plot}\PY{p}{(}\PY{n}{x}\PY{p}{,} \PY{n}{y2}\PY{p}{,} \PY{n}{marker}\PY{o}{=}\PY{l+s+s2}{\PYZdq{}}\PY{l+s+s2}{x}\PY{l+s+s2}{\PYZdq{}}\PY{p}{,} \PY{n}{linewidth}\PY{o}{=}\PY{l+m+mi}{5}\PY{p}{,} \PY{n}{color}\PY{o}{=}\PY{l+s+s2}{\PYZdq{}}\PY{l+s+s2}{pink}\PY{l+s+s2}{\PYZdq{}}\PY{p}{,} \PY{n}{label}\PY{o}{=}\PY{l+s+s2}{\PYZdq{}}\PY{l+s+s2}{2nd Plot}\PY{l+s+s2}{\PYZdq{}}\PY{p}{)}
\PY{n}{plt}\PY{o}{.}\PY{n}{title}\PY{p}{(}\PY{l+s+s2}{\PYZdq{}}\PY{l+s+s2}{Plot Title}\PY{l+s+s2}{\PYZdq{}}\PY{p}{)}
\PY{n}{plt}\PY{o}{.}\PY{n}{xlabel}\PY{p}{(}\PY{l+s+s2}{\PYZdq{}}\PY{l+s+s2}{x label}\PY{l+s+s2}{\PYZdq{}}\PY{p}{)}
\PY{n}{plt}\PY{o}{.}\PY{n}{ylabel}\PY{p}{(}\PY{l+s+s2}{\PYZdq{}}\PY{l+s+s2}{y label}\PY{l+s+s2}{\PYZdq{}}\PY{p}{)}
\PY{n}{plt}\PY{o}{.}\PY{n}{legend}\PY{p}{(}\PY{p}{)}
\PY{n}{plt}\PY{o}{.}\PY{n}{show}\PY{p}{(}\PY{p}{)}
\end{Verbatim}
\end{tcolorbox}

    \begin{center}
    \adjustimage{max size={0.9\linewidth}{0.9\paperheight}}{matplotlib_files/matplotlib_19_0.png}
    \end{center}
    { \hspace*{\fill} \\}
    
    It can also be specified if axis has to be in linear or log scale with
\(\tt{xscale("log")}\) and/or \(\tt{yscale("log")}\).

    \begin{tcolorbox}[breakable, size=fbox, boxrule=1pt, pad at break*=1mm,colback=cellbackground, colframe=cellborder]
\prompt{In}{incolor}{24}{\boxspacing}
\begin{Verbatim}[commandchars=\\\{\}]
\PY{n}{fig} \PY{o}{=} \PY{n}{plt}\PY{o}{.}\PY{n}{figure}\PY{p}{(}\PY{n}{figsize}\PY{o}{=}\PY{p}{(}\PY{l+m+mi}{8}\PY{p}{,}\PY{l+m+mi}{5}\PY{p}{)}\PY{p}{)}
\PY{n}{plt}\PY{o}{.}\PY{n}{plot}\PY{p}{(}\PY{n}{x}\PY{p}{,} \PY{n}{y1}\PY{p}{,} \PY{n}{marker}\PY{o}{=}\PY{l+s+s2}{\PYZdq{}}\PY{l+s+s2}{+}\PY{l+s+s2}{\PYZdq{}}\PY{p}{,} \PY{n}{linestyle}\PY{o}{=}\PY{l+s+s2}{\PYZdq{}}\PY{l+s+s2}{:}\PY{l+s+s2}{\PYZdq{}}\PY{p}{,} \PY{n}{color}\PY{o}{=}\PY{l+s+s2}{\PYZdq{}}\PY{l+s+s2}{darkgreen}\PY{l+s+s2}{\PYZdq{}}\PY{p}{,} \PY{n}{label}\PY{o}{=}\PY{l+s+s2}{\PYZdq{}}\PY{l+s+s2}{1st Plot}\PY{l+s+s2}{\PYZdq{}}\PY{p}{)}
\PY{n}{plt}\PY{o}{.}\PY{n}{plot}\PY{p}{(}\PY{n}{x}\PY{p}{,} \PY{n}{y2}\PY{p}{,} \PY{n}{marker}\PY{o}{=}\PY{l+s+s2}{\PYZdq{}}\PY{l+s+s2}{x}\PY{l+s+s2}{\PYZdq{}}\PY{p}{,} \PY{n}{linewidth}\PY{o}{=}\PY{l+m+mi}{5}\PY{p}{,} \PY{n}{color}\PY{o}{=}\PY{l+s+s2}{\PYZdq{}}\PY{l+s+s2}{pink}\PY{l+s+s2}{\PYZdq{}}\PY{p}{,} \PY{n}{label}\PY{o}{=}\PY{l+s+s2}{\PYZdq{}}\PY{l+s+s2}{2nd Plot}\PY{l+s+s2}{\PYZdq{}}\PY{p}{)}
\PY{n}{plt}\PY{o}{.}\PY{n}{title}\PY{p}{(}\PY{l+s+s2}{\PYZdq{}}\PY{l+s+s2}{Plot Title}\PY{l+s+s2}{\PYZdq{}}\PY{p}{)}
\PY{n}{plt}\PY{o}{.}\PY{n}{xlabel}\PY{p}{(}\PY{l+s+s2}{\PYZdq{}}\PY{l+s+s2}{x label}\PY{l+s+s2}{\PYZdq{}}\PY{p}{)}
\PY{n}{plt}\PY{o}{.}\PY{n}{ylabel}\PY{p}{(}\PY{l+s+s2}{\PYZdq{}}\PY{l+s+s2}{y label}\PY{l+s+s2}{\PYZdq{}}\PY{p}{)}
\PY{n}{plt}\PY{o}{.}\PY{n}{xscale}\PY{p}{(}\PY{l+s+s2}{\PYZdq{}}\PY{l+s+s2}{log}\PY{l+s+s2}{\PYZdq{}}\PY{p}{)}
\PY{n}{plt}\PY{o}{.}\PY{n}{yscale}\PY{p}{(}\PY{l+s+s2}{\PYZdq{}}\PY{l+s+s2}{log}\PY{l+s+s2}{\PYZdq{}}\PY{p}{)}
\PY{n}{plt}\PY{o}{.}\PY{n}{legend}\PY{p}{(}\PY{p}{)}
\PY{n}{plt}\PY{o}{.}\PY{n}{show}\PY{p}{(}\PY{p}{)}
\end{Verbatim}
\end{tcolorbox}

    \begin{center}
    \adjustimage{max size={0.9\linewidth}{0.9\paperheight}}{matplotlib_files/matplotlib_21_0.png}
    \end{center}
    { \hspace*{\fill} \\}
    
    \hypertarget{date-on-axis}{%
\paragraph{Date on Axis}\label{date-on-axis}}

A special need, very common in finance, is to have dates on the \(x\)
axis. This can be done like shown in the next example:

    \begin{tcolorbox}[breakable, size=fbox, boxrule=1pt, pad at break*=1mm,colback=cellbackground, colframe=cellborder]
\prompt{In}{incolor}{25}{\boxspacing}
\begin{Verbatim}[commandchars=\\\{\}]
\PY{k+kn}{import} \PY{n+nn}{datetime} \PY{k}{as} \PY{n+nn}{dt}
\PY{k+kn}{import} \PY{n+nn}{matplotlib}\PY{n+nn}{.}\PY{n+nn}{dates} \PY{k}{as} \PY{n+nn}{mdates}

\PY{n}{dates} \PY{o}{=} \PY{p}{[}\PY{l+s+s1}{\PYZsq{}}\PY{l+s+s1}{01/02/1991}\PY{l+s+s1}{\PYZsq{}}\PY{p}{,}\PY{l+s+s1}{\PYZsq{}}\PY{l+s+s1}{01/03/1991}\PY{l+s+s1}{\PYZsq{}}\PY{p}{,}\PY{l+s+s1}{\PYZsq{}}\PY{l+s+s1}{01/04/1991}\PY{l+s+s1}{\PYZsq{}}\PY{p}{]}
\PY{n}{xd} \PY{o}{=} \PY{p}{[}\PY{n}{dt}\PY{o}{.}\PY{n}{datetime}\PY{o}{.}\PY{n}{strptime}\PY{p}{(}\PY{n}{d}\PY{p}{,}\PY{l+s+s1}{\PYZsq{}}\PY{l+s+s1}{\PYZpc{}}\PY{l+s+s1}{m/}\PY{l+s+si}{\PYZpc{}d}\PY{l+s+s1}{/}\PY{l+s+s1}{\PYZpc{}}\PY{l+s+s1}{Y}\PY{l+s+s1}{\PYZsq{}}\PY{p}{)}\PY{o}{.}\PY{n}{date}\PY{p}{(}\PY{p}{)} \PY{k}{for} \PY{n}{d} \PY{o+ow}{in} \PY{n}{dates}\PY{p}{]}
\PY{n}{yd} \PY{o}{=} \PY{n+nb}{range}\PY{p}{(}\PY{n+nb}{len}\PY{p}{(}\PY{n}{xd}\PY{p}{)}\PY{p}{)}

\PY{n}{fig} \PY{o}{=} \PY{n}{plt}\PY{o}{.}\PY{n}{figure}\PY{p}{(}\PY{n}{figsize}\PY{o}{=}\PY{p}{(}\PY{l+m+mi}{8}\PY{p}{,}\PY{l+m+mi}{5}\PY{p}{)}\PY{p}{)}
\PY{n}{plt}\PY{o}{.}\PY{n}{gca}\PY{p}{(}\PY{p}{)}\PY{o}{.}\PY{n}{xaxis}\PY{o}{.}\PY{n}{set\PYZus{}major\PYZus{}formatter}\PY{p}{(}\PY{n}{mdates}\PY{o}{.}\PY{n}{DateFormatter}\PY{p}{(}\PY{l+s+s1}{\PYZsq{}}\PY{l+s+s1}{\PYZpc{}}\PY{l+s+s1}{m/}\PY{l+s+si}{\PYZpc{}d}\PY{l+s+s1}{/}\PY{l+s+s1}{\PYZpc{}}\PY{l+s+s1}{Y}\PY{l+s+s1}{\PYZsq{}}\PY{p}{)}\PY{p}{)}
\PY{n}{plt}\PY{o}{.}\PY{n}{gca}\PY{p}{(}\PY{p}{)}\PY{o}{.}\PY{n}{xaxis}\PY{o}{.}\PY{n}{set\PYZus{}major\PYZus{}locator}\PY{p}{(}\PY{n}{mdates}\PY{o}{.}\PY{n}{DayLocator}\PY{p}{(}\PY{p}{)}\PY{p}{)}
\PY{n}{plt}\PY{o}{.}\PY{n}{plot}\PY{p}{(}\PY{n}{xd}\PY{p}{,} \PY{n}{yd}\PY{p}{)}
\PY{n}{plt}\PY{o}{.}\PY{n}{gcf}\PY{p}{(}\PY{p}{)}\PY{o}{.}\PY{n}{autofmt\PYZus{}xdate}\PY{p}{(}\PY{p}{)} \PY{c+c1}{\PYZsh{} this makes a prettier formatting}
\PY{n}{plt}\PY{o}{.}\PY{n}{show}\PY{p}{(}\PY{p}{)}
\end{Verbatim}
\end{tcolorbox}

    \begin{center}
    \adjustimage{max size={0.9\linewidth}{0.9\paperheight}}{matplotlib_files/matplotlib_23_0.png}
    \end{center}
    { \hspace*{\fill} \\}
    
    \hypertarget{subplots}{%
\subsubsection{Subplots}\label{subplots}}

It may happen that we need to plot two or more plots side by side or in
a grid fashion.

The canvas can be then split into subcanvases with the \(\tt{subplot}\)
commands which takes in input the number of rows, columns and the index
of the current subplot. It returns the \emph{subcanvas} object that can
be used to draw like previously shown.

There is one complication though, some of the commands change when
dealing with subplot, \(\tt{xlabel}\) and \(\tt{ylabel}\) for example
becomes \(\tt{set_xlabel}\) and \(\tt{set_ylabel}\). Similarly for
\(\tt{xlim}\) and \(\tt{ylim}\).

As an example imagine that we want to plot the previous plots side by
side on two canvases instead of togheter in the same

    \begin{tcolorbox}[breakable, size=fbox, boxrule=1pt, pad at break*=1mm,colback=cellbackground, colframe=cellborder]
\prompt{In}{incolor}{26}{\boxspacing}
\begin{Verbatim}[commandchars=\\\{\}]
\PY{n}{fig} \PY{o}{=} \PY{n}{plt}\PY{o}{.}\PY{n}{figure}\PY{p}{(}\PY{n}{figsize}\PY{o}{=}\PY{p}{(}\PY{l+m+mi}{8}\PY{p}{,}\PY{l+m+mi}{5}\PY{p}{)}\PY{p}{)}

\PY{n}{sub1} \PY{o}{=} \PY{n}{plt}\PY{o}{.}\PY{n}{subplot}\PY{p}{(}\PY{l+m+mi}{1}\PY{p}{,} \PY{l+m+mi}{2}\PY{p}{,} \PY{l+m+mi}{1}\PY{p}{)}
\PY{n}{sub1}\PY{o}{.}\PY{n}{plot}\PY{p}{(}\PY{n}{x}\PY{p}{,} \PY{n}{y1}\PY{p}{,} \PY{n}{marker}\PY{o}{=}\PY{l+s+s2}{\PYZdq{}}\PY{l+s+s2}{+}\PY{l+s+s2}{\PYZdq{}}\PY{p}{,} \PY{n}{linestyle}\PY{o}{=}\PY{l+s+s2}{\PYZdq{}}\PY{l+s+s2}{:}\PY{l+s+s2}{\PYZdq{}}\PY{p}{,} \PY{n}{color}\PY{o}{=}\PY{l+s+s2}{\PYZdq{}}\PY{l+s+s2}{darkgreen}\PY{l+s+s2}{\PYZdq{}}\PY{p}{,} \PY{n}{label}\PY{o}{=}\PY{l+s+s2}{\PYZdq{}}\PY{l+s+s2}{1st Plot}\PY{l+s+s2}{\PYZdq{}}\PY{p}{)}
\PY{n}{sub1}\PY{o}{.}\PY{n}{set\PYZus{}xlabel}\PY{p}{(}\PY{l+s+s2}{\PYZdq{}}\PY{l+s+s2}{x label}\PY{l+s+s2}{\PYZdq{}}\PY{p}{)}
\PY{n}{sub1}\PY{o}{.}\PY{n}{set\PYZus{}ylabel}\PY{p}{(}\PY{l+s+s2}{\PYZdq{}}\PY{l+s+s2}{y label}\PY{l+s+s2}{\PYZdq{}}\PY{p}{)}
\PY{n}{sub1}\PY{o}{.}\PY{n}{legend}\PY{p}{(}\PY{p}{)}

\PY{n}{sub2} \PY{o}{=} \PY{n}{plt}\PY{o}{.}\PY{n}{subplot}\PY{p}{(}\PY{l+m+mi}{1}\PY{p}{,} \PY{l+m+mi}{2}\PY{p}{,} \PY{l+m+mi}{2}\PY{p}{)}
\PY{n}{sub2}\PY{o}{.}\PY{n}{plot}\PY{p}{(}\PY{n}{x}\PY{p}{,} \PY{n}{y2}\PY{p}{,} \PY{n}{marker}\PY{o}{=}\PY{l+s+s2}{\PYZdq{}}\PY{l+s+s2}{x}\PY{l+s+s2}{\PYZdq{}}\PY{p}{,} \PY{n}{linewidth}\PY{o}{=}\PY{l+m+mi}{5}\PY{p}{,} \PY{n}{color}\PY{o}{=}\PY{l+s+s2}{\PYZdq{}}\PY{l+s+s2}{pink}\PY{l+s+s2}{\PYZdq{}}\PY{p}{,} \PY{n}{label}\PY{o}{=}\PY{l+s+s2}{\PYZdq{}}\PY{l+s+s2}{2nd Plot}\PY{l+s+s2}{\PYZdq{}}\PY{p}{)}
\PY{n}{sub2}\PY{o}{.}\PY{n}{set\PYZus{}xlabel}\PY{p}{(}\PY{l+s+s2}{\PYZdq{}}\PY{l+s+s2}{x label}\PY{l+s+s2}{\PYZdq{}}\PY{p}{)}
\PY{n}{sub2}\PY{o}{.}\PY{n}{set\PYZus{}ylabel}\PY{p}{(}\PY{l+s+s2}{\PYZdq{}}\PY{l+s+s2}{y label}\PY{l+s+s2}{\PYZdq{}}\PY{p}{)}
\PY{n}{sub2}\PY{o}{.}\PY{n}{legend}\PY{p}{(}\PY{p}{)}

\PY{n}{plt}\PY{o}{.}\PY{n}{show}\PY{p}{(}\PY{p}{)}
\end{Verbatim}
\end{tcolorbox}

    \begin{center}
    \adjustimage{max size={0.9\linewidth}{0.9\paperheight}}{matplotlib_files/matplotlib_25_0.png}
    \end{center}
    { \hspace*{\fill} \\}
    
    Subplots can also be arranged in another way like

    \begin{tcolorbox}[breakable, size=fbox, boxrule=1pt, pad at break*=1mm,colback=cellbackground, colframe=cellborder]
\prompt{In}{incolor}{27}{\boxspacing}
\begin{Verbatim}[commandchars=\\\{\}]
\PY{n}{fig} \PY{o}{=} \PY{n}{plt}\PY{o}{.}\PY{n}{figure}\PY{p}{(}\PY{n}{figsize}\PY{o}{=}\PY{p}{(}\PY{l+m+mi}{8}\PY{p}{,}\PY{l+m+mi}{5}\PY{p}{)}\PY{p}{)}

\PY{n}{sub1} \PY{o}{=} \PY{n}{plt}\PY{o}{.}\PY{n}{subplot}\PY{p}{(}\PY{l+m+mi}{2}\PY{p}{,} \PY{l+m+mi}{1}\PY{p}{,} \PY{l+m+mi}{1}\PY{p}{)}
\PY{n}{sub1}\PY{o}{.}\PY{n}{plot}\PY{p}{(}\PY{n}{x}\PY{p}{,} \PY{n}{y1}\PY{p}{,} \PY{n}{marker}\PY{o}{=}\PY{l+s+s2}{\PYZdq{}}\PY{l+s+s2}{+}\PY{l+s+s2}{\PYZdq{}}\PY{p}{,} \PY{n}{linestyle}\PY{o}{=}\PY{l+s+s2}{\PYZdq{}}\PY{l+s+s2}{:}\PY{l+s+s2}{\PYZdq{}}\PY{p}{,} \PY{n}{color}\PY{o}{=}\PY{l+s+s2}{\PYZdq{}}\PY{l+s+s2}{darkgreen}\PY{l+s+s2}{\PYZdq{}}\PY{p}{,} \PY{n}{label}\PY{o}{=}\PY{l+s+s2}{\PYZdq{}}\PY{l+s+s2}{1st Plot}\PY{l+s+s2}{\PYZdq{}}\PY{p}{)}
\PY{n}{sub1}\PY{o}{.}\PY{n}{set\PYZus{}xlabel}\PY{p}{(}\PY{l+s+s2}{\PYZdq{}}\PY{l+s+s2}{x label}\PY{l+s+s2}{\PYZdq{}}\PY{p}{)}
\PY{n}{sub1}\PY{o}{.}\PY{n}{set\PYZus{}ylabel}\PY{p}{(}\PY{l+s+s2}{\PYZdq{}}\PY{l+s+s2}{y label}\PY{l+s+s2}{\PYZdq{}}\PY{p}{)}
\PY{n}{sub1}\PY{o}{.}\PY{n}{legend}\PY{p}{(}\PY{p}{)}

\PY{n}{sub2} \PY{o}{=} \PY{n}{plt}\PY{o}{.}\PY{n}{subplot}\PY{p}{(}\PY{l+m+mi}{2}\PY{p}{,} \PY{l+m+mi}{1}\PY{p}{,} \PY{l+m+mi}{2}\PY{p}{)}
\PY{n}{sub2}\PY{o}{.}\PY{n}{plot}\PY{p}{(}\PY{n}{x}\PY{p}{,} \PY{n}{y2}\PY{p}{,} \PY{n}{marker}\PY{o}{=}\PY{l+s+s2}{\PYZdq{}}\PY{l+s+s2}{x}\PY{l+s+s2}{\PYZdq{}}\PY{p}{,} \PY{n}{linewidth}\PY{o}{=}\PY{l+m+mi}{5}\PY{p}{,} \PY{n}{color}\PY{o}{=}\PY{l+s+s2}{\PYZdq{}}\PY{l+s+s2}{pink}\PY{l+s+s2}{\PYZdq{}}\PY{p}{,} \PY{n}{label}\PY{o}{=}\PY{l+s+s2}{\PYZdq{}}\PY{l+s+s2}{2nd Plot}\PY{l+s+s2}{\PYZdq{}}\PY{p}{)}
\PY{n}{sub2}\PY{o}{.}\PY{n}{set\PYZus{}xlabel}\PY{p}{(}\PY{l+s+s2}{\PYZdq{}}\PY{l+s+s2}{x label}\PY{l+s+s2}{\PYZdq{}}\PY{p}{)}
\PY{n}{sub2}\PY{o}{.}\PY{n}{set\PYZus{}ylabel}\PY{p}{(}\PY{l+s+s2}{\PYZdq{}}\PY{l+s+s2}{y label}\PY{l+s+s2}{\PYZdq{}}\PY{p}{)}
\PY{n}{sub2}\PY{o}{.}\PY{n}{legend}\PY{p}{(}\PY{p}{)}

\PY{n}{plt}\PY{o}{.}\PY{n}{show}\PY{p}{(}\PY{p}{)}
\end{Verbatim}
\end{tcolorbox}

    \begin{center}
    \adjustimage{max size={0.9\linewidth}{0.9\paperheight}}{matplotlib_files/matplotlib_27_0.png}
    \end{center}
    { \hspace*{\fill} \\}
    
    \hypertarget{lines-and-text}{%
\subsubsection{Lines and Text}\label{lines-and-text}}

To make plots more informative it can be useful to add text or lines on
a plot.

\hypertarget{lines}{%
\paragraph{Lines}\label{lines}}

Lines can be drawn with \(\tt{hlines}\) (for horizontal lines) and
\(\tt{hvines}\) (for vertical lines). Both commands take in input the
\(y\), \(x_{min}\), \(x_{max}\) and \(x\), \(y_{min}\), \(y_{max}\)
respectively. Beware, coordinates are relative to the axis scale.

Other parameters allow to format the line in terms of color, style,
width\(\ldots\).

    \begin{tcolorbox}[breakable, size=fbox, boxrule=1pt, pad at break*=1mm,colback=cellbackground, colframe=cellborder]
\prompt{In}{incolor}{28}{\boxspacing}
\begin{Verbatim}[commandchars=\\\{\}]
\PY{n}{fig} \PY{o}{=} \PY{n}{plt}\PY{o}{.}\PY{n}{figure}\PY{p}{(}\PY{n}{figsize}\PY{o}{=}\PY{p}{(}\PY{l+m+mi}{8}\PY{p}{,}\PY{l+m+mi}{5}\PY{p}{)}\PY{p}{)}
\PY{n}{plt}\PY{o}{.}\PY{n}{plot}\PY{p}{(}\PY{n}{x}\PY{p}{,} \PY{n}{y1}\PY{p}{,} \PY{n}{marker}\PY{o}{=}\PY{l+s+s2}{\PYZdq{}}\PY{l+s+s2}{+}\PY{l+s+s2}{\PYZdq{}}\PY{p}{,} \PY{n}{linestyle}\PY{o}{=}\PY{l+s+s2}{\PYZdq{}}\PY{l+s+s2}{:}\PY{l+s+s2}{\PYZdq{}}\PY{p}{,} \PY{n}{color}\PY{o}{=}\PY{l+s+s2}{\PYZdq{}}\PY{l+s+s2}{darkgreen}\PY{l+s+s2}{\PYZdq{}}\PY{p}{,} \PY{n}{label}\PY{o}{=}\PY{l+s+s2}{\PYZdq{}}\PY{l+s+s2}{1st Plot}\PY{l+s+s2}{\PYZdq{}}\PY{p}{)}
\PY{n}{plt}\PY{o}{.}\PY{n}{plot}\PY{p}{(}\PY{n}{x}\PY{p}{,} \PY{n}{y2}\PY{p}{,} \PY{n}{marker}\PY{o}{=}\PY{l+s+s2}{\PYZdq{}}\PY{l+s+s2}{x}\PY{l+s+s2}{\PYZdq{}}\PY{p}{,} \PY{n}{linewidth}\PY{o}{=}\PY{l+m+mi}{5}\PY{p}{,} \PY{n}{color}\PY{o}{=}\PY{l+s+s2}{\PYZdq{}}\PY{l+s+s2}{pink}\PY{l+s+s2}{\PYZdq{}}\PY{p}{,} \PY{n}{label}\PY{o}{=}\PY{l+s+s2}{\PYZdq{}}\PY{l+s+s2}{2nd Plot}\PY{l+s+s2}{\PYZdq{}}\PY{p}{)}
\PY{n}{plt}\PY{o}{.}\PY{n}{title}\PY{p}{(}\PY{l+s+s2}{\PYZdq{}}\PY{l+s+s2}{Plot Title}\PY{l+s+s2}{\PYZdq{}}\PY{p}{)}
\PY{n}{plt}\PY{o}{.}\PY{n}{xlabel}\PY{p}{(}\PY{l+s+s2}{\PYZdq{}}\PY{l+s+s2}{x label}\PY{l+s+s2}{\PYZdq{}}\PY{p}{)}
\PY{n}{plt}\PY{o}{.}\PY{n}{ylabel}\PY{p}{(}\PY{l+s+s2}{\PYZdq{}}\PY{l+s+s2}{y label}\PY{l+s+s2}{\PYZdq{}}\PY{p}{)}
\PY{n}{plt}\PY{o}{.}\PY{n}{hlines}\PY{p}{(}\PY{l+m+mi}{65}\PY{p}{,} \PY{l+m+mi}{0}\PY{p}{,} \PY{l+m+mf}{37.5}\PY{p}{,} \PY{n}{color}\PY{o}{=}\PY{l+s+s1}{\PYZsq{}}\PY{l+s+s1}{red}\PY{l+s+s1}{\PYZsq{}}\PY{p}{,} \PY{n}{linestyle}\PY{o}{=}\PY{l+s+s2}{\PYZdq{}}\PY{l+s+s2}{\PYZhy{}.}\PY{l+s+s2}{\PYZdq{}}\PY{p}{,} \PY{n}{label}\PY{o}{=}\PY{l+s+s1}{\PYZsq{}}\PY{l+s+s1}{Critical Thr.}\PY{l+s+s1}{\PYZsq{}}\PY{p}{)}
\PY{n}{plt}\PY{o}{.}\PY{n}{legend}\PY{p}{(}\PY{p}{)}
\PY{n}{plt}\PY{o}{.}\PY{n}{show}\PY{p}{(}\PY{p}{)}
\end{Verbatim}
\end{tcolorbox}

    \begin{center}
    \adjustimage{max size={0.9\linewidth}{0.9\paperheight}}{matplotlib_files/matplotlib_29_0.png}
    \end{center}
    { \hspace*{\fill} \\}
    
    Notice as the \(y\) value is 65 and \(x\) goes from 0 to 37.5 which
corresponds to each axis.

Analogously for a vertical line, except that we have to first specify
the \(x\) value and then the \(y\) range.

    \begin{tcolorbox}[breakable, size=fbox, boxrule=1pt, pad at break*=1mm,colback=cellbackground, colframe=cellborder]
\prompt{In}{incolor}{29}{\boxspacing}
\begin{Verbatim}[commandchars=\\\{\}]
\PY{n}{fig} \PY{o}{=} \PY{n}{plt}\PY{o}{.}\PY{n}{figure}\PY{p}{(}\PY{n}{figsize}\PY{o}{=}\PY{p}{(}\PY{l+m+mi}{8}\PY{p}{,}\PY{l+m+mi}{5}\PY{p}{)}\PY{p}{)}
\PY{n}{plt}\PY{o}{.}\PY{n}{plot}\PY{p}{(}\PY{n}{x}\PY{p}{,} \PY{n}{y1}\PY{p}{,} \PY{n}{marker}\PY{o}{=}\PY{l+s+s2}{\PYZdq{}}\PY{l+s+s2}{+}\PY{l+s+s2}{\PYZdq{}}\PY{p}{,} \PY{n}{linestyle}\PY{o}{=}\PY{l+s+s2}{\PYZdq{}}\PY{l+s+s2}{:}\PY{l+s+s2}{\PYZdq{}}\PY{p}{,} \PY{n}{color}\PY{o}{=}\PY{l+s+s2}{\PYZdq{}}\PY{l+s+s2}{darkgreen}\PY{l+s+s2}{\PYZdq{}}\PY{p}{,} \PY{n}{label}\PY{o}{=}\PY{l+s+s2}{\PYZdq{}}\PY{l+s+s2}{1st Plot}\PY{l+s+s2}{\PYZdq{}}\PY{p}{)}
\PY{n}{plt}\PY{o}{.}\PY{n}{plot}\PY{p}{(}\PY{n}{x}\PY{p}{,} \PY{n}{y2}\PY{p}{,} \PY{n}{marker}\PY{o}{=}\PY{l+s+s2}{\PYZdq{}}\PY{l+s+s2}{x}\PY{l+s+s2}{\PYZdq{}}\PY{p}{,} \PY{n}{linewidth}\PY{o}{=}\PY{l+m+mi}{5}\PY{p}{,} \PY{n}{color}\PY{o}{=}\PY{l+s+s2}{\PYZdq{}}\PY{l+s+s2}{pink}\PY{l+s+s2}{\PYZdq{}}\PY{p}{,} \PY{n}{label}\PY{o}{=}\PY{l+s+s2}{\PYZdq{}}\PY{l+s+s2}{2nd Plot}\PY{l+s+s2}{\PYZdq{}}\PY{p}{)}
\PY{n}{plt}\PY{o}{.}\PY{n}{title}\PY{p}{(}\PY{l+s+s2}{\PYZdq{}}\PY{l+s+s2}{Plot Title}\PY{l+s+s2}{\PYZdq{}}\PY{p}{)}
\PY{n}{plt}\PY{o}{.}\PY{n}{xlabel}\PY{p}{(}\PY{l+s+s2}{\PYZdq{}}\PY{l+s+s2}{x label}\PY{l+s+s2}{\PYZdq{}}\PY{p}{)}
\PY{n}{plt}\PY{o}{.}\PY{n}{ylabel}\PY{p}{(}\PY{l+s+s2}{\PYZdq{}}\PY{l+s+s2}{y label}\PY{l+s+s2}{\PYZdq{}}\PY{p}{)}
\PY{n}{plt}\PY{o}{.}\PY{n}{vlines}\PY{p}{(}\PY{l+m+mi}{22}\PY{p}{,} \PY{l+m+mi}{0}\PY{p}{,} \PY{l+m+mi}{120}\PY{p}{,} \PY{n}{color}\PY{o}{=}\PY{l+s+s1}{\PYZsq{}}\PY{l+s+s1}{gray}\PY{l+s+s1}{\PYZsq{}}\PY{p}{,} \PY{n}{label}\PY{o}{=}\PY{l+s+s1}{\PYZsq{}}\PY{l+s+s1}{Splitting Thr.}\PY{l+s+s1}{\PYZsq{}}\PY{p}{)}
\PY{n}{plt}\PY{o}{.}\PY{n}{legend}\PY{p}{(}\PY{p}{)}
\PY{n}{plt}\PY{o}{.}\PY{n}{show}\PY{p}{(}\PY{p}{)}
\end{Verbatim}
\end{tcolorbox}

    \begin{center}
    \adjustimage{max size={0.9\linewidth}{0.9\paperheight}}{matplotlib_files/matplotlib_31_0.png}
    \end{center}
    { \hspace*{\fill} \\}
    
    \hypertarget{text}{%
\paragraph{Text}\label{text}}

Like lines text can be added to the plot with the command \(\tt{text}\).
This function takes the \(x\) and \(y\) coordinates of starting point of
the text and the actual text string. Like before coordinates are
relative to the axis plot.

Other options can be used to format the text style, their meaning should
be straightforward:

\begin{itemize}
\tightlist
\item
  \(\tt{backgroundcolor}\);
\item
  \(\tt{color}\);
\item
  \(\tt{fontfamily}\): (FONTNAME, `serif', `sans-serif', `cursive',
  `fantasy', `monospace');
\item
  \(\tt{fontsize}\); (size in points, `xx-small', `x-small', `small',
  `medium', `large', `x-large', `xx-large');
\item
  \(\tt{fontstyle}\): (`normal', `italic', `oblique');
\item
  \(\tt{horizontalalignment}\): (`center', `right', `left');
\item
  \(\tt{rotation}\): takes the angle in degrees;
\item
  \(\tt{verticalalignment}\): (`center', `top', `bottom', `baseline',
  `center\_baseline').
\end{itemize}

    \begin{tcolorbox}[breakable, size=fbox, boxrule=1pt, pad at break*=1mm,colback=cellbackground, colframe=cellborder]
\prompt{In}{incolor}{31}{\boxspacing}
\begin{Verbatim}[commandchars=\\\{\}]
\PY{n}{fig} \PY{o}{=} \PY{n}{plt}\PY{o}{.}\PY{n}{figure}\PY{p}{(}\PY{n}{figsize}\PY{o}{=}\PY{p}{(}\PY{l+m+mi}{8}\PY{p}{,}\PY{l+m+mi}{5}\PY{p}{)}\PY{p}{)}
\PY{n}{plt}\PY{o}{.}\PY{n}{plot}\PY{p}{(}\PY{n}{x}\PY{p}{,} \PY{n}{y1}\PY{p}{,} \PY{n}{marker}\PY{o}{=}\PY{l+s+s2}{\PYZdq{}}\PY{l+s+s2}{+}\PY{l+s+s2}{\PYZdq{}}\PY{p}{,} \PY{n}{linestyle}\PY{o}{=}\PY{l+s+s2}{\PYZdq{}}\PY{l+s+s2}{:}\PY{l+s+s2}{\PYZdq{}}\PY{p}{,} \PY{n}{color}\PY{o}{=}\PY{l+s+s2}{\PYZdq{}}\PY{l+s+s2}{darkgreen}\PY{l+s+s2}{\PYZdq{}}\PY{p}{,} \PY{n}{label}\PY{o}{=}\PY{l+s+s2}{\PYZdq{}}\PY{l+s+s2}{1st Plot}\PY{l+s+s2}{\PYZdq{}}\PY{p}{)}
\PY{n}{plt}\PY{o}{.}\PY{n}{plot}\PY{p}{(}\PY{n}{x}\PY{p}{,} \PY{n}{y2}\PY{p}{,} \PY{n}{marker}\PY{o}{=}\PY{l+s+s2}{\PYZdq{}}\PY{l+s+s2}{x}\PY{l+s+s2}{\PYZdq{}}\PY{p}{,} \PY{n}{linewidth}\PY{o}{=}\PY{l+m+mi}{5}\PY{p}{,} \PY{n}{color}\PY{o}{=}\PY{l+s+s2}{\PYZdq{}}\PY{l+s+s2}{pink}\PY{l+s+s2}{\PYZdq{}}\PY{p}{,} \PY{n}{label}\PY{o}{=}\PY{l+s+s2}{\PYZdq{}}\PY{l+s+s2}{2nd Plot}\PY{l+s+s2}{\PYZdq{}}\PY{p}{)}
\PY{n}{plt}\PY{o}{.}\PY{n}{title}\PY{p}{(}\PY{l+s+s2}{\PYZdq{}}\PY{l+s+s2}{Plot Title}\PY{l+s+s2}{\PYZdq{}}\PY{p}{)}
\PY{n}{plt}\PY{o}{.}\PY{n}{xlabel}\PY{p}{(}\PY{l+s+s2}{\PYZdq{}}\PY{l+s+s2}{x label}\PY{l+s+s2}{\PYZdq{}}\PY{p}{)}
\PY{n}{plt}\PY{o}{.}\PY{n}{ylabel}\PY{p}{(}\PY{l+s+s2}{\PYZdq{}}\PY{l+s+s2}{y label}\PY{l+s+s2}{\PYZdq{}}\PY{p}{)}
\PY{n}{plt}\PY{o}{.}\PY{n}{vlines}\PY{p}{(}\PY{l+m+mi}{22}\PY{p}{,} \PY{l+m+mi}{0}\PY{p}{,} \PY{l+m+mi}{120}\PY{p}{,} \PY{n}{color}\PY{o}{=}\PY{l+s+s1}{\PYZsq{}}\PY{l+s+s1}{gray}\PY{l+s+s1}{\PYZsq{}}\PY{p}{)}
\PY{n}{plt}\PY{o}{.}\PY{n}{text}\PY{p}{(}\PY{l+m+mi}{23}\PY{p}{,} \PY{l+m+mi}{5}\PY{p}{,} \PY{l+s+s2}{\PYZdq{}}\PY{l+s+s2}{Splitting Thr.}\PY{l+s+s2}{\PYZdq{}}\PY{p}{,} \PY{n}{rotation}\PY{o}{=}\PY{l+m+mi}{90}\PY{p}{)}
\PY{n}{plt}\PY{o}{.}\PY{n}{hlines}\PY{p}{(}\PY{l+m+mi}{65}\PY{p}{,} \PY{l+m+mi}{0}\PY{p}{,} \PY{l+m+mf}{37.5}\PY{p}{,} \PY{n}{color}\PY{o}{=}\PY{l+s+s1}{\PYZsq{}}\PY{l+s+s1}{red}\PY{l+s+s1}{\PYZsq{}}\PY{p}{,} \PY{n}{linestyle}\PY{o}{=}\PY{l+s+s2}{\PYZdq{}}\PY{l+s+s2}{\PYZhy{}.}\PY{l+s+s2}{\PYZdq{}}\PY{p}{)}
\PY{n}{plt}\PY{o}{.}\PY{n}{text}\PY{p}{(}\PY{l+m+mi}{25}\PY{p}{,} \PY{l+m+mi}{75}\PY{p}{,} \PY{l+s+s2}{\PYZdq{}}\PY{l+s+s2}{Critical Thr.}\PY{l+s+s2}{\PYZdq{}}\PY{p}{,} \PY{n}{backgroundcolor}\PY{o}{=}\PY{l+s+s2}{\PYZdq{}}\PY{l+s+s2}{red}\PY{l+s+s2}{\PYZdq{}}\PY{p}{,} \PY{n}{color}\PY{o}{=}\PY{l+s+s2}{\PYZdq{}}\PY{l+s+s2}{white}\PY{l+s+s2}{\PYZdq{}}\PY{p}{)}
\PY{n}{plt}\PY{o}{.}\PY{n}{legend}\PY{p}{(}\PY{p}{)}
\PY{n}{plt}\PY{o}{.}\PY{n}{show}\PY{p}{(}\PY{p}{)}
\end{Verbatim}
\end{tcolorbox}

    \begin{center}
    \adjustimage{max size={0.9\linewidth}{0.9\paperheight}}{matplotlib_files/matplotlib_33_0.png}
    \end{center}
    { \hspace*{\fill} \\}
    
    \hypertarget{saving}{%
\subsubsection{Saving}\label{saving}}

A plot can be saved into a file for later usage with \(\tt{savefig}\)
which takes in input the file name. The format of the file will be
inferred from the file name extension (.jpg, .png, .gif\ldots{})

    \begin{tcolorbox}[breakable, size=fbox, boxrule=1pt, pad at break*=1mm,colback=cellbackground, colframe=cellborder]
\prompt{In}{incolor}{ }{\boxspacing}
\begin{Verbatim}[commandchars=\\\{\}]
\PY{n}{plt}\PY{o}{.}\PY{n}{savefig}\PY{p}{(}\PY{l+s+s2}{\PYZdq{}}\PY{l+s+s2}{myfigure.png}\PY{l+s+s2}{\PYZdq{}}\PY{p}{)}
\end{Verbatim}
\end{tcolorbox}


    % Add a bibliography block to the postdoc
    
    
    
\end{document}

\chapter{Value At Risk}
\label{var-and-credit-risk}

\section{VaR}
\label{value-at-risk}

The \emph{Value at Risk} (VaR) of a portfolio is used when it is required to know to a certain confidence level how much will be the maximum loss in the next $N$ days. By definition it is a function of two parameters: the time horizon (i.e. $N$ days) and the confidence level. 
It can be interpreted as the loss level over a certain time horizon that has a probability of only \((100 - X)\%\) of being exceeded. 

Mathematically the VaR is the loss corresponding to the \((100-X)\textrm{th}\) percentile of the portfolio change in value distribution over the next $N$ days (e.g. in Figure~\ref{fig:var_loss} the graphical representation of the VaR assuming a normal distribution for the changes in value is shown, with $N=1$ and $X=95$).

\begin{figure}[htb]
\centering
  \includegraphics[width=0.6\linewidth]{figures/95_var.png}
  \caption{Example of 95\% VaR on the distribution of changes in value.}
  \label{fig:var_loss}
\end{figure}
    
VaR is useful to summarize all the information about the risk of a portfolio in one single number, but this can also be considered its main limitation as it implies too much simplification of such a complex task.

Concerning the time horizon parameter it is usually set to $N=1$ since it is not easy to estimate market variables over periods longer than one day. Nevertheless to generalize the VaR estimate it is assumed

\begin{equation}
\textrm{N-day VaR} = \textrm{1-day VaR}\cdot \sqrt{N}
\label{eq:var_horizon}
\end{equation}
This relation is true only if the portfolio changes in value over the considered period of time have independent and identical normal distributions with zero mean, otherwise it is just an approximation.

\section{How to Estimate the VaR}
\label{how-to-estimate-the-var}

In the proposed examples historical series of Apple and Netflix are used also it has been assumed to have a portfolio made of \textbf{60\% of AAPL and 40\% NFLX stocks}.
 
You can download the dateset with \texttt{yfinance} but you need some massaging to rearrange a bit the dataframe or check out \href{https://raw.githubusercontent.com/matteosan1/finance_course/develop/libro/input_files/historical_data.csv}{historical\_data.csv} as in the following code.

\begin{ipython}
import pandas as pd

df = pd.read_csv('historical_data.csv', index_col='Date')	
print (df.head())
\end{ipython}
\begin{ioutput}
                 aapl       nflx
Date
2014-01-02  17.598297  51.831429
2014-01-03  17.211735  51.871429
2014-01-06  17.305593  51.367142
2014-01-07  17.181829  48.500000
2014-01-08  17.290642  48.712856
\end{ioutput}
\noindent
In the following we are going to add a new column with the daily return to the dataframe.

\begin{ipython}
import numpy as np

w = np.array([0.6, 0.4])
df['aapl_rets'] = df['aapl'].pct_change()
df['nflx_rets'] = df['nflx'].pct_change()

print (df.head())
\end{ipython}
\begin{ioutput}
                 aapl       nflx  aapl_rets  nflx_rets
Date                                                  
2014-01-02  17.542171  51.831429        NaN        NaN
2014-01-03  17.156841  51.871429  -0.021966   0.000772
2014-01-06  17.250401  51.367142   0.005453  -0.009722
2014-01-07  17.127035  48.500000  -0.007151  -0.055817
2014-01-08  17.235500  48.712856   0.006333   0.004389
\end{ioutput}

The portfolio value $P_i$ is determined by the \emph{scalar product} between the invested amount (the weights $w_i$) and the asset prices ($v_i$), i.e. $\sum_{i} w_i \cdot v_i$. In \texttt{python}, when using \texttt{numpy.array} for weights and values, it is indicated with the method \texttt{.dot}. For more information about scalar product, vector and matrices see Chapter~\ref{sec:matrices}.

\subsection{Historical Simulation}
\label{historical-simulation}

To estimate the VaR from historical series, we need to collect the necessary market variables (i.e. the asset prices) over a quite long period in the past.

The historical sequence of daily price variations will provide different scenarios to be applied to today's asset prices. 
The daily evolution of the portfolio can be simulated by rescaling each asset value (e.g. \(v_1(t), v_2(t)\)) according to their variations between in two consecutive days \(i\) and \(i-1\)

\begin{equation}
P_i(t_n+1) = \left(w_1 \cdot v_1(t_n)\cfrac{(v_1(t_i)-v_1(t_{i-1}))}{v_1(t_{i-1})} + w_2\cdot  v_2(t_n)\cfrac{(v_2(t_i)-v_2(t_{i-1}))}{v_2(t_{i-1})}\right)
\end{equation}

With the set of portfolio values the distribution of portfolio change in value (\(\Delta P\)) can be computed, and the VaR estimate will be the $(100 - X)\%$ percentile of this distribution. 

Clearly such kind of simulation heavily relies on the assumption that past behaviors are indicative of what might happen in the future, that's why it is important that out historical series was as large as possible.


%At each simulation the portfolio value $P_i$ is determined by the \emph{scalar product} between the invested amount (the weights $w_i$) and the price of each portfolio component ($p_i$), i.e. $\sum_{i} w_i \cdot p_i$. In \texttt{python} it is indicated with the method \texttt{.dot}. For more information about scalar product, vector and matrices see Chapter~\ref{sec:matrices}.

%Finally the distribution of possible changes in the portfolio value \(P_i\) is drawn and the VaR computed by taking the appropriate percentile.
Figure~\ref{fig:hist_var} shows the resulting VaR. The calculation is carried out in two complementary ways: expliciting all the for loops and, more compactly, using \texttt{numpy.array}.

\begin{ipython}
import numpy as np

rets = []
for idx in df.index[1:]:
rets.append(w[0]*df['aapl_rets'].iloc[idx] + w[1]*df['nflx_rets'].iloc[idx])

portfolio_price = w[0] * df['aapl'].iloc[-1] + w[1] * df['nflx'].iloc[-1]
hist_var = portfolio_price*np.percentile(rets, 5)

print ('Historical VAR is {:.3f}'.format(hist_var))
\end{ipython}
\begin{ioutput}
Historical VAR is -3.350
\end{ioutput}

\begin{ipython}
rets = df[['aapl_rets', 'nflx_rets']].iloc[1:].dot(w)
portfolio_price = df[['aapl','nflx']].iloc[-1].dot(w)
hist_var = portfolio_price*np.percentile(rets, 5)
	
print ('Historical VAR is {:.3f}'.format(hist_var))
\end{ipython}
\begin{ioutput}
Historical VAR is -3.350
\end{ioutput}

\begin{figure}[htb]
\centering
\includegraphics[width=0.7\textwidth]{figures/historical_var}
\caption{Distribution of the changes of values estimated from historical data. The red line shows the 99\% VaR.}
\label{fig:hist_var}
\end{figure}

%\subsection{Model Approach}
%\label{model-approach}
%
%The portfolio $P$ consists of different amounts $w_i$ invested on two assets. If with $\Delta x_i$ we denote the daily return of the $i^{th}$ asset the portfolio change in value can be expressed as
%
%\begin{equation}
%\Delta P = \sum_{i=1}^n w_i \Delta x_i
%\end{equation}
%
%If we then assume that the asset variations are normally distributed with zero mean (in this approach is typical to assume the expected change in a market variable over the considered period equal to zero), \(\Delta P\) will be also normally distributed (as a sum of normal distribution) with zero mean.
%
%To estimate the VaR we just need to compute the standard deviation of $\Delta P$. In the general case with many different assets, $\sigma_i$ indicates the daily volatility of the $i^{th}$ asset and $\rho_{ij}$ the correlation coefficient between the assets $i$ and $j$. The variance of $\Delta P$ can then be expressed as
%
%\begin{align}
%\begin{split}
%\sigma^2_P & = \sum_{i=1}^{n}\sum_{j=1}^{n}\rho_{ij}w_i w_j \sigma_i \sigma_j \\
%& = \sum_{i=1}^{n} w_i^2 \sigma_i^2 + 2 \sum_{i=1}^{n}\sum_{j<i}^{n}\rho_{ij}w_i w_j \sigma_i \sigma_j 
%\end{split}
%\end{align}
%\noindent
%If we are interested in a longer time horizon we can use Eq.~\ref{eq:var_horizon}.
%
%Once we have the variance of \(\Delta P\) it is easy to determine the appropriate percentile using the equations described in Appendix~\ref{transformation-to-standard-normal}.

\subsection{Monte Carlo Simulation}
\label{monte-carlo-simulation}

A very useful alternative to the historical approach is to use Monte Carlo simulation to generate the probability distribution for $\Delta P$. 

Imagine we need to compute the 1-day VaR of our portfolio. The simulation can be done either generating random returns from a distribution with mean and standard deviation obtained from the historical data of each stock, or by evolving the asset prices in one day, for example using a geometric Brownian motion.

Let's start from the first case, computing mean and standard deviation from the historical dataset, then we will sample various simulated returns from a multivariate Gaussian with such mean and variance. One useful aspect of this method is that other distributions than Gaussian could be used. 

Once we have the distribution of returns the VaR can be computed as usual; the result is shown in Fig.~\ref{fig:mc1_var}.

\begin{ipython}
from scipy.stats import multivariate_normal

mean = df[['aapl_rets', 'nflx_rets']].iloc[1:].mean()
cov = df[['aapl_rets', 'nflx_rets']].iloc[1:].cov()
mvnorm = multivariate_normal(mean=mean, cov=cov)

np.random.seed(1)
sim_returns = mvnorm.rvs(10000)
p_returns = sim_returns.dot(w)
mc_var = portfolio_price * np.percentile(p_returns, 5)

print('Simulated VAR is {:.3f}'.format(mc_var))
\end{ipython}
\begin{ioutput}
Simulated VAR is -3.439
\end{ioutput}

\begin{figure}[htb]
\centering
\includegraphics[width=0.7\textwidth]{figures/sim1_var}
\caption{Distribution of the changes of values estimated from simulated data. The red line shows the 95\% VaR.}
\label{fig:mc1_var}
\end{figure}

This result can be compared to the VaR estimate determined with a simulation of the stock price daily evolution. We will use the geometric Brownian motion described in Section~\ref{derivation-of-log-normal-stochastic-differential-equation} where $\mu$ and $\sigma$ are the mean and variance estimated from the historical series. Figure~\ref{fig:mc2_var} shows the resulting distribution of the returns.

\begin{ipython}
from numpy.random import normal, seed

T = 1
dP = []
for exp in range(trials):
	seed(exp)

    s = w[0]*df['aapl'].iloc[-1]*np.exp((mean.iloc[0] - 0.5*cov.iloc[0, 0])*T + 
        np.sqrt(cov.iloc[0, 0]*T)*normal())
    s += w[1]*df['nflx'].iloc[-1]*np.exp((mean.iloc[1] - 0.5*cov.iloc[1, 1])*T + 
         np.sqrt(cov.iloc[1, 1]*T)*normal())
    dP.append(portfolio_price - s)

mc_var = np.percentile(dP, 5)
print('Simulated VAR is {:.5f}'.format(mc_var))
\end{ipython}
\begin{ioutput}
Simulated VAR is -5.67443
\end{ioutput}

\begin{figure}[htb]
\centering
\includegraphics[width=0.7\textwidth]{figures/sim2_var}
\caption{Distribution of the changes of values estimated from simulation of the evolution of the stock prices. The red line shows the 99\% VaR.}
\label{fig:mc2_var}
\end{figure}

The large difference with the previous estimates is most likely due to the different shape of the $\Delta P$ distribution tails.

\subsection{Stress and Back Testing}
\label{stress-testing-and-back-testing}

In addition to calculating the value at risk of a portfolio, it is generally useful to check how it would behave under the most extreme market moves seen in the last years.

This kind of test is called \emph{stress test} and it is done by extracting from the historical series particular days with exceptionally large variation of the market variables.
 
The idea is to take into account extreme events that can happen more frequently in reality than in a simulation (where usually Gaussian tails are assumed). For example a 5-standard deviation ($5\sigma$) move is expected to happen once every 7000 years but in practice can be observed twice over 10 years.

\begin{attention}
\subsubsection{$n\sigma$ Event Likelihood}
To compute the occurence frequency of a $n\sigma$ event consider that it has a probability to happen equal to $n\sigma$ (referred to a standard Gaussian distribution) so it is 1 over $252\cdot P(|x| \ge n\sigma$) (the 252 factor comes to the number of working days per year).

\begin{attpython}
from scipy.stats import norm

prob = norm.cdf(-5) * 2 # e.g. consider +- 5sigma movements
nyears = 1/prob/252
print (nyears)
\end{attpython}
\begin{ioutput}
6921.737673091067
\end{ioutput}
\noindent
So about seven thousand years as stated before.
\end{attention}

Another important check that could be done is the \emph{back testing} which consists of assessing how well the VaR estimate would have performed in the past. Basically it has to be tested how often the daily loss exceeded the N-days X\% VaR just computed. If it happens on about (100-X)\% of the times we can be confident that our estimate is correct. Clearly back-testing makes sense only if VaR has been estimated on an independet historical sample with respect to that used in the test.

\section{Credit VaR}
\label{credit-var-cr-var}

%exposure at any given future time is the
%larger between zero and the market value of the portfolio of derivative
%positions with a counterparty that would be lost if the counterparty
%were to default with zero recovery at that time.

\emph{Credit VaR} is defined in the usual way Value at Risk measures are (i.e. as percentile of a loss distribution).
In this case we are concerned with the default risk associated to one or multiple counter-parties in a specific portfolio instead of to the market risk.

To derive the loss distribution we need to consider the exposure $\textrm{EE}(\tau)$ defined as the sum of the discounted cash flows at the default date $\tau$. The corresponding loss is then given by

\begin{equation}
L_{\tau, \hat{T}} = (1 - R) \cdot \textrm{EE}(\tau)
\end{equation}
where \(\hat{T}\) is the risk horizon and $L$ is non-zero only in scenarios of early counter-party default. 

Given the above definitions we can express the Credit VaR as the X-quantile of $L_{\tau, \hat{T}}$.
With respect to the Value at Risk, the time horizon is usually set to one year and the percentile to $99.9^{th}$, so it returns the loss that is exceeded only in 1 case out of 1000. 

Credit VaR is actually either the difference of the percentile from the mean, or the percentile itself. There is more than one possible definition, anyway we  will use the latter.

%Consider your portfolio has a call option on equity with a final maturity of two years. To get the Credit-Var, roughly, you simulate the underlying equity up to one year, and obtain a number of scenarios for the underlying equity in one year. Also, you need to simulate the default scenarios up to one year, to know in each scenario whether the counter-parties have defaulted or not. 
%
%And then in each scenario at one year, if the counter-party
%has defaulted there will be a recovery value and all else will be lost.
%Otherwise, we price the call option over the remaining year using for
%example a Black Scholes formula. But this price is like taking the
%expected value of the call option payoff in two years, conditional on
%each scenario for the underlying equity in one year. Because this is
%pricing, this expected value will be taken under the pricing measure Q,
%not P. This gives the Black Scholes formula if the underlying equity
%follows a geometric brownian motion under Q.

\subsection{Credit VaR and MC Simulation}

Credit VaR can be calculated through a simulation of the evolution of a portfolio up to the risk horizon; the simulation must of course include possible defaults of the counter-parties. 

In each experiment the portfolio is priced obtaining a number of scenarios to draw the loss distribution. It is then straightforward to derive the Credit VaR.

Consider a portfolio of twenty zero coupon bonds each one with a default probability of 8\% and the same face value (\euro{100}). The recovery rate in case of default is $R=40\%$ and the risk free rate is 1\%.

\begin{ipython}
import numpy as np
from datetime import date
from dateutil.relativedelta import relativedelta
from scipy.stats import uniform

bonds = 20
DP = 0.08
FV = [100 for _ in range(20)]
R = 0.4
r = 0.01

pillars = [date.today() + relativedelta(years=i) for i in range(2)]
df = np.exp(-r)

np.random.seed(2)
scenarios = 10000
losses = []
for _ in range(scenarios):
    loss = 0
    unif = uniform.rvs(size=bonds)
    for i in range(bonds):
        if unif[i] < DP:
            loss += (1 - R)*FV[i]*df
    losses.append(loss)

print (np.percentile(losses, [99.9]))
\end{ipython}
\begin{ioutput}
[356.41794015]
\end{ioutput}
\noindent
Figure~\ref{fig:credit_var} (left) shows the loss distribution of our simple portfolio.

This particularly simple example could have been solved in a different way. Indeed we are dealing with independent and equiprobable default events, so the distribution of the losses (which is a scaled version of the distribution of defaults, since the loss given default is the same for each ZCB) could have been estimated simply with the binomial distribution. Figure~\ref{fig:credit_var} (right) shows the corresponding binomial distribution, notice how the shape is qualitatively similar to that of the left plot.

\begin{figure}[htb]
\centering
\includegraphics[width=0.45\textwidth]{figures/credit_var_zcb}
\includegraphics[width=0.45\textwidth]{figures/binomial_zcb}
\caption{(left) Distribution of losses in a portfolio made of twenty zero coupon bonds. The distribution is discrete since all the ZCB have the same face value and recovery rate. The red line indicates the Credit VaR. (right) 
Binomial distribution for $n=20$ and $p=0.08$. Notice the similarities between the two curves.}
\label{fig:credit_var}
\end{figure}

\begin{ipython}
from scipy.stats import binom

bi = binom(20, 0.08)
q = bi.ppf(0.999)

print ((1 - R)*FV[i]*df*q)
\end{ipython}
\begin{ioutput}
356.4179401497005
\end{ioutput}

Indeed the two results are the same, since in the first place we have made a MC simulation of a binomial distribution.

\subsection{Credit VaR and One Factor Copula Model}
Consider a portfolio made of similar assets. As an approximation assume that the probability of default is the same for each counter-party and that the correlation between each pair is the same and equal to $\rho$. Under these assumption the One Factor Copula model can be used to describe the default correlations (see Eq.~\ref{eq:gaussian_one_factor_copula}

\begin{equation}
\mathcal{Q}_M(T) = \Phi\Big(\cfrac{\Phi^{-1}[Q(T)]-M\sqrt{\rho}}{\sqrt{1-\rho}}\Big)
\label{eq:conditional_default_prob}
\end{equation}
where $\Phi$ is the cumulative distribution function of the standard normal.

Since there are $n$ counter-parties with the same default probability $\mathcal{Q}_M(T)$ the percentage of defaults at time $T$ is $\mathcal{Q}_M$ itself ($\textrm{\% of defaults} = \textrm{n\_defaults}/n = n\cdot \mathcal{Q}_M/n$). Hence Eq.~\ref{eq:conditional_default_prob} gives the percentage of defaults by time $T$ given $M$. 

Since $M$ is distributed according to a standard Gaussian we can be $X\%$ certain that its value will be \emph{greater} than $\hat{m} = \Phi^{-1}(1-X)=-\Phi^{-1}(X)$, where the equality holds due to the symmetry of the Gaussian distribution (see Figure~\ref{fig:certain_for_X}).

\begin{figure}[htb]
	\centering
	\includegraphics[width=0.7\textwidth]{figures/certain_for_X}
	\caption{$X\%$ probability to get an higher value a threshold for a normally distributed random variable.}
	\label{fig:certain_for_X}
\end{figure} 

Once the time $T$ has been chosen the only random variable appearing in the conditional default probability expression of Eq.~\ref{eq:conditional_default_prob} is $M$, therefore we can be $X\%$ certain that the percentage of defaults over $T$ years on a large portfolio will be \textbf{less} than $V(X,T)$ where
\[
V(X,T)= \Phi\Big(\cfrac{\Phi^{-1}[Q(T)]-\hat{m}\sqrt{\rho}}{\sqrt{1-\rho}}\Big) = \Phi\Big(\cfrac{\Phi^{-1}[Q(T)]+\Phi^{-1}(X)\sqrt{\rho}}{\sqrt{1-\rho}}\Big)
\]
When the confidence level is $X\%$ and the time horizon is $T$, a rough estimate of the Credit VaR is therefore $P(1-R)V(X,T)$, where $P$ is the portfolio size and $R$ is the recovery rate.

Suppose that a bank has a total of \euro{100} million of retail exposures. The 1-year probability of default averages to 2\% and the recovery rate averages to 60\%. The copula correlation parameter is estimated as 0.1.

\begin{ipython}
from scipy.stats import norm
from math import sqrt

X = 0.999
rho = 0.1
R = 0.6
Q = 0.02
exposure = 100e6
num = norm.ppf(Q) + sqrt(rho)*norm.ppf(X)
den = sqrt(1-rho)
V = norm.cdf(num/den)
cr_var = exposure*V*(1-R)

print ("Cr-VaR: {:.0f}".format(round(cr_var, -4)))
\end{ipython}
\begin{ioutput}
Cr-VaR: 5130000
\end{ioutput}
\noindent
The 1-year 99.9\% Credit VaR is therefore \euro{5.13} million.

\subsection{CreditMetrics}
Another popular approach to compute Credti VaR is \emph{CreditMetrics}. It involves estimating a probability distribution of credit losses by carrying out Monte Carlo simulations of the counter-party credit rating changes.

Imagine we would like to determine the probability distribution of losses over 1-year period. On each simulation, we are going to determine the credit rating of each counter-party using the estimated probability of migration between one rate to another (or to default). Since the portfolio value depends on its asset ratings we can determine the eventual losses. 

As an example consider Table~\ref{tab:credit_ratings} which shows the percentage probability of a bond moving from one category to another during a 1-year period.

\begin{table}[htb]
	\centering
	\begin{tabular}{|l|c|c|c|c|c|c|c|c|}
	\hline
	Initial rating & AAA & AA & A & BBB & BB & B & CCC & default \\
	\hline
	\hline
	AAA & 90.81 & 8.33 & 0.68 & 0.06 & 0.08 & 0.02 & 0.01& 0.01 \\ 
	\hline
	AA & 0.70 & 90.65 & 7.79 & 0.64 & 0.06 & 0.13 & 0.02 & 0.01 \\ 
	\hline
	A & 0.09 & 2.27 & 91.05 & 5.52 & 0.74 & 0.26 & 0.01 & 0.06 \\ 
	\hline
	BBB & 0.02 & 0.33 & 5.95 & 85.93 & 5.30 & 1.17 & 1.12 & 0.18 \\
	\hline
	BB & 0.03 & 0.14 & 0.67 & 7.73 & 80.53 & 8.84 & 1.00 & 1.06 \\
	\hline
	B & 0.01 & 0.11 & 0.24 & 0.43 & 6.48 & 83.46 & 4.07 & 5.20 \\
	\hline
	CCC & 0.21 & 0 & 0.22 & 1.30 & 2.38 & 11.24 & 64.86 & 19.79 \\		
	\hline
\end{tabular}
\caption{Example of table with transition probabilities (in percent) between different credit rating categories.}
\label{tab:credit_ratings}
\end{table}

For a correct implementation of this technique credit rate changes cannot be assumed independent, hence a copula approach could be implemented.
Another possibility is the application of Markov chains~\ref{sec:markov_chain}, with the transition matrix which can be deduced by the numbers in Table~\ref{tab:credit_ratings}.

In fact the main difficulty in this application is precisely the determination of the transition matrix. These probabilities could be estimated by analysing historical data from credit rating agencies, such as Standard\&Poor, Moody’s and Fitch. But this could lead to unreliable numbers in case the future does not develop as smoothly as the past. It can therefore be more reliable to base the estimates on a combination of empirical data and more subjective, qualitative data such as opinions from experts. This is because the market view is a mixture of beliefs determined by both historical ratings and a more extreme view of the ratings. 

Another problem with deciding the transition matrix is that maybe it is not appropriate to use a \emph{homogeneous} Markov chain to model credit risk over time. In this kind of chain the transition matrix is considered constant, but it is clearly a crude approximation of reality which doesn't capture the time-varying behaviour of the default risk. A non-homogeneous model could be more realistic, but on the other hand more complicated to use. 

%As an example suppose to simulate rating change of a portfolio of nine bonds with various ratings over 1-year period. The correlation between them is 0.2.
%
%\begin{codebox}
%\begin{Verbatim}[commandchars=\\\{\}]
%\PY{k+kn}{from} \PY{n+nn}{scipy}\PY{n+nn}{.}\PY{n+nn}{stats} \PY{k}{import} \PY{n}{multivariate\PYZus{}normal}\PY{p}{,} \PY{n}{norm}
%\PY{k+kn}{import} \PY{n+nn}{numpy}
%		
%\PY{c+c1}{\PYZsh{} AAA, AA, A, BBB, BB, B, CCC, Def}
%\PY{n}{table} \PY{o}{=} \PY{p}{[}\PY{p}{[}\PY{l+m+mf}{90.81}\PY{p}{,} \PY{l+m+mf}{8.33}\PY{p}{,} \PY{l+m+mf}{0.68}\PY{p}{,} \PY{l+m+mf}{0.06}\PY{p}{,} \PY{l+m+mf}{0.08}\PY{p}{,} \PY{l+m+mf}{0.02}\PY{p}{,} \PY{l+m+mf}{0.01}\PY{p}{,} \PY{l+m+mf}{0.01}\PY{p}{]}\PY{p}{,}
%         \PY{p}{[}\PY{l+m+mf}{0.70}\PY{p}{,} \PY{l+m+mf}{90.65}\PY{p}{,} \PY{l+m+mf}{7.79}\PY{p}{,} \PY{l+m+mf}{0.64}\PY{p}{,} \PY{l+m+mf}{0.06}\PY{p}{,} \PY{l+m+mf}{0.13}\PY{p}{,} \PY{l+m+mf}{0.02}\PY{p}{,} \PY{l+m+mf}{0.01}\PY{p}{]}\PY{p}{,}
%         \PY{p}{[}\PY{l+m+mf}{0.09}\PY{p}{,} \PY{l+m+mf}{2.27}\PY{p}{,} \PY{l+m+mf}{91.05}\PY{p}{,} \PY{l+m+mf}{5.52}\PY{p}{,} \PY{l+m+mf}{0.74}\PY{p}{,} \PY{l+m+mf}{0.26}\PY{p}{,} \PY{l+m+mf}{0.01}\PY{p}{,} \PY{l+m+mf}{0.06}\PY{p}{]}\PY{p}{,}
%         \PY{p}{[}\PY{l+m+mf}{0.02}\PY{p}{,} \PY{l+m+mf}{0.33}\PY{p}{,} \PY{l+m+mf}{5.95}\PY{p}{,} \PY{l+m+mf}{85.93}\PY{p}{,} \PY{l+m+mf}{5.30}\PY{p}{,} \PY{l+m+mf}{1.17}\PY{p}{,} \PY{l+m+mf}{1.12}\PY{p}{,} \PY{l+m+mf}{0.18}\PY{p}{]}\PY{p}{,}
%         \PY{p}{[}\PY{l+m+mf}{0.03}\PY{p}{,} \PY{l+m+mf}{0.14}\PY{p}{,} \PY{l+m+mf}{0.67}\PY{p}{,} \PY{l+m+mf}{7.73}\PY{p}{,} \PY{l+m+mf}{80.53}\PY{p}{,} \PY{l+m+mf}{8.84}\PY{p}{,} \PY{l+m+mf}{1.00}\PY{p}{,} \PY{l+m+mf}{1.06}\PY{p}{]}\PY{p}{,}
%         \PY{p}{[}\PY{l+m+mf}{0.01}\PY{p}{,} \PY{l+m+mf}{0.11}\PY{p}{,} \PY{l+m+mf}{0.24}\PY{p}{,} \PY{l+m+mf}{0.43}\PY{p}{,} \PY{l+m+mf}{6.48}\PY{p}{,} \PY{l+m+mf}{83.46}\PY{p}{,} \PY{l+m+mf}{4.07}\PY{p}{,} \PY{l+m+mf}{5.20}\PY{p}{]}\PY{p}{,}
%         \PY{p}{[}\PY{l+m+mf}{0.21}\PY{p}{,} \PY{l+m+mi}{0}\PY{p}{,} \PY{l+m+mf}{0.22}\PY{p}{,} \PY{l+m+mf}{1.30}\PY{p}{,} \PY{l+m+mf}{2.38}\PY{p}{,} \PY{l+m+mf}{11.24}\PY{p}{,} \PY{l+m+mf}{64.86}\PY{p}{,} \PY{l+m+mf}{19.79}\PY{p}{]}\PY{p}{]}
%				
%\PY{n}{t} \PY{o}{=} \PY{n}{numpy}\PY{o}{.}\PY{n}{array}\PY{p}{(}\PY{n}{table}\PY{p}{)}
%\PY{n}{table\PYZus{}gauss} \PY{o}{=} \PY{n}{norm}\PY{o}{.}\PY{n}{ppf}\PY{p}{(}\PY{n}{np}\PY{o}{.}\PY{n}{cumsum}\PY{p}{(}\PY{n}{t}\PY{o}{/}\PY{l+m+mf}{100.}\PY{p}{,} \PY{n}{axis}\PY{o}{=}\PY{l+m+mi}{1}\PY{p}{)}\PY{p}{)}
%\PY{n}{table\PYZus{}gauss}\PY{p}{[}\PY{p}{:}\PY{p}{,} \PY{o}{\PYZhy{}}\PY{l+m+mi}{1}\PY{p}{]} \PY{o}{=} \PY{n}{np}\PY{o}{.}\PY{n}{inf}
%		
%\PY{n}{N} \PY{o}{=} \PY{p}{[}\PY{l+m+mi}{100}\PY{p}{,} \PY{l+m+mi}{95}\PY{p}{,} \PY{l+m+mi}{92}\PY{p}{,} \PY{l+m+mi}{85}\PY{p}{,} \PY{l+m+mi}{80}\PY{p}{,} \PY{l+m+mi}{70}\PY{p}{,} \PY{l+m+mi}{60}\PY{p}{]}
%\PY{n}{portfolio} \PY{o}{=} \PY{p}{[}\PY{l+m+mi}{2}\PY{p}{,} \PY{l+m+mi}{3}\PY{p}{,} \PY{l+m+mi}{3}\PY{p}{,} \PY{l+m+mi}{4}\PY{p}{,} \PY{l+m+mi}{5}\PY{p}{,} \PY{l+m+mi}{6}\PY{p}{,} \PY{l+m+mi}{3}\PY{p}{,} \PY{l+m+mi}{4}\PY{p}{,} \PY{l+m+mi}{2}\PY{p}{]}
%\PY{n}{R} \PY{o}{=} \PY{l+m+mf}{0.4}
%		
%\PY{n}{p0} \PY{o}{=} \PY{l+m+mi}{0}
%\PY{k}{for} \PY{n}{i} \PY{o+ow}{in} \PY{n}{portfolio}\PY{p}{:}
%    \PY{n}{p0} \PY{o}{+}\PY{o}{=} \PY{n}{N}\PY{p}{[}\PY{n}{i}\PY{p}{]}
%		
%\PY{n}{numpy}\PY{o}{.}\PY{n}{random}\PY{o}{.}\PY{n}{seed}\PY{p}{(}\PY{l+m+mi}{1}\PY{p}{)}
%\PY{n}{mvnorm} \PY{o}{=} \PY{n}{multivariate\PYZus{}normal}\PY{p}{(}\PY{n}{mean}\PY{o}{=}\PY{p}{[}\PY{l+m+mi}{0} \PY{k}{for} \PY{n}{\PYZus{}} \PY{o+ow}{in} \PY{n+nb}{range}\PY{p}{(}\PY{l+m+mi}{9}\PY{p}{)}\PY{p}{]}\PY{p}{,}
%          \PY{n}{cov}\PY{o}{=}\PY{p}{[}\PY{p}{[}\PY{l+m+mi}{1}\PY{p}{,} \PY{l+m+mf}{0.2}\PY{p}{,} \PY{l+m+mf}{0.2}\PY{p}{,} \PY{l+m+mf}{0.2}\PY{p}{,} \PY{l+m+mf}{0.2}\PY{p}{,} \PY{l+m+mf}{0.2}\PY{p}{,} \PY{l+m+mf}{0.2}\PY{p}{,} \PY{l+m+mf}{0.2}\PY{p}{,} \PY{l+m+mf}{0.2}\PY{p}{]}\PY{p}{,}
%          \PY{p}{[}\PY{l+m+mf}{0.2}\PY{p}{,} \PY{l+m+mi}{1}\PY{p}{,} \PY{l+m+mf}{0.2}\PY{p}{,} \PY{l+m+mf}{0.2}\PY{p}{,} \PY{l+m+mf}{0.2}\PY{p}{,} \PY{l+m+mf}{0.2}\PY{p}{,} \PY{l+m+mf}{0.2}\PY{p}{,} \PY{l+m+mf}{0.2}\PY{p}{,} \PY{l+m+mf}{0.2}\PY{p}{]}\PY{p}{,}
%          \PY{p}{[}\PY{l+m+mf}{0.2}\PY{p}{,} \PY{l+m+mf}{0.2}\PY{p}{,} \PY{l+m+mi}{1}\PY{p}{,} \PY{l+m+mf}{0.2}\PY{p}{,} \PY{l+m+mf}{0.2}\PY{p}{,} \PY{l+m+mf}{0.2}\PY{p}{,} \PY{l+m+mf}{0.2}\PY{p}{,} \PY{l+m+mf}{0.2}\PY{p}{,} \PY{l+m+mf}{0.2}\PY{p}{]}\PY{p}{,}
%          \PY{p}{[}\PY{l+m+mf}{0.2}\PY{p}{,} \PY{l+m+mf}{0.2}\PY{p}{,} \PY{l+m+mf}{0.2}\PY{p}{,} \PY{l+m+mi}{1}\PY{p}{,} \PY{l+m+mf}{0.2}\PY{p}{,} \PY{l+m+mf}{0.2}\PY{p}{,} \PY{l+m+mf}{0.2}\PY{p}{,} \PY{l+m+mf}{0.2}\PY{p}{,} \PY{l+m+mf}{0.2}\PY{p}{]}\PY{p}{,}
%          \PY{p}{[}\PY{l+m+mf}{0.2}\PY{p}{,} \PY{l+m+mf}{0.2}\PY{p}{,} \PY{l+m+mf}{0.2}\PY{p}{,} \PY{l+m+mf}{0.2}\PY{p}{,} \PY{l+m+mi}{1}\PY{p}{,} \PY{l+m+mf}{0.2}\PY{p}{,} \PY{l+m+mf}{0.2}\PY{p}{,} \PY{l+m+mf}{0.2}\PY{p}{,} \PY{l+m+mf}{0.2}\PY{p}{]}\PY{p}{,}
%          \PY{p}{[}\PY{l+m+mf}{0.2}\PY{p}{,} \PY{l+m+mf}{0.2}\PY{p}{,} \PY{l+m+mf}{0.2}\PY{p}{,} \PY{l+m+mf}{0.2}\PY{p}{,} \PY{l+m+mf}{0.2}\PY{p}{,} \PY{l+m+mi}{1}\PY{p}{,} \PY{l+m+mf}{0.2}\PY{p}{,} \PY{l+m+mf}{0.2}\PY{p}{,} \PY{l+m+mf}{0.2}\PY{p}{]}\PY{p}{,}
%          \PY{p}{[}\PY{l+m+mf}{0.2}\PY{p}{,} \PY{l+m+mf}{0.2}\PY{p}{,} \PY{l+m+mf}{0.2}\PY{p}{,} \PY{l+m+mf}{0.2}\PY{p}{,} \PY{l+m+mf}{0.2}\PY{p}{,} \PY{l+m+mf}{0.2}\PY{p}{,} \PY{l+m+mi}{1}\PY{p}{,} \PY{l+m+mf}{0.2}\PY{p}{,} \PY{l+m+mf}{0.2}\PY{p}{]}\PY{p}{,}
%          \PY{p}{[}\PY{l+m+mf}{0.2}\PY{p}{,} \PY{l+m+mf}{0.2}\PY{p}{,} \PY{l+m+mf}{0.2}\PY{p}{,} \PY{l+m+mf}{0.2}\PY{p}{,} \PY{l+m+mf}{0.2}\PY{p}{,} \PY{l+m+mf}{0.2}\PY{p}{,} \PY{l+m+mf}{0.2}\PY{p}{,} \PY{l+m+mi}{1}\PY{p}{,} \PY{l+m+mf}{0.2}\PY{p}{]}\PY{p}{,}
%          \PY{p}{[}\PY{l+m+mf}{0.2}\PY{p}{,} \PY{l+m+mf}{0.2}\PY{p}{,} \PY{l+m+mf}{0.2}\PY{p}{,} \PY{l+m+mf}{0.2}\PY{p}{,} \PY{l+m+mf}{0.2}\PY{p}{,} \PY{l+m+mf}{0.2}\PY{p}{,} \PY{l+m+mf}{0.2}\PY{p}{,} \PY{l+m+mf}{0.2}\PY{p}{,} \PY{l+m+mi}{1}\PY{p}{]}\PY{p}{]}\PY{p}{)}
%		
%\PY{n}{trials} \PY{o}{=} \PY{l+m+mi}{1000000}
%\PY{n}{x\PYZus{}prob} \PY{o}{=} \PY{n}{mvnorm}\PY{o}{.}\PY{n}{rvs}\PY{p}{(}\PY{n}{size}\PY{o}{=}\PY{n}{trials}\PY{p}{)}
%		
%\PY{n}{dp} \PY{o}{=} \PY{p}{[}\PY{p}{]}
%\PY{k}{for} \PY{n}{x} \PY{o+ow}{in} \PY{n}{x\PYZus{}prob}\PY{p}{:}
%    \PY{n}{p} \PY{o}{=} \PY{l+m+mi}{0}
%    \PY{k}{for} \PY{n}{j} \PY{o+ow}{in} \PY{n+nb}{range}\PY{p}{(}\PY{n+nb}{len}\PY{p}{(}\PY{n}{portfolio}\PY{p}{)}\PY{p}{)}\PY{p}{:}
%        \PY{n}{ip} \PY{o}{=} \PY{l+m+mi}{0}
%        \PY{k}{while} \PY{n}{x}\PY{p}{[}\PY{n}{j}\PY{p}{]} \PY{o}{\PYZgt{}} \PY{n}{table\PYZus{}gauss}\PY{p}{[}\PY{n}{portfolio}\PY{p}{[}\PY{n}{j}\PY{p}{]}\PY{p}{,} \PY{n}{ip}\PY{p}{]}\PY{p}{:}
%            \PY{n}{ip} \PY{o}{+}\PY{o}{=} \PY{l+m+mi}{1}
%            \PY{k}{if} \PY{n}{ip} \PY{o}{==} \PY{l+m+mi}{7}\PY{p}{:}
%                \PY{n}{p} \PY{o}{+}\PY{o}{=} \PY{n}{N}\PY{p}{[}\PY{n}{portfolio}\PY{p}{[}\PY{n}{j}\PY{p}{]}\PY{p}{]}\PY{o}{*}\PY{p}{(}\PY{l+m+mi}{1}\PY{o}{\PYZhy{}}\PY{n}{R}\PY{p}{)}
%            \PY{k}{else}\PY{p}{:}
%                \PY{n}{p} \PY{o}{+}\PY{o}{=} \PY{n}{N}\PY{p}{[}\PY{n}{ip}\PY{p}{]}
%	
%        \PY{n}{r} \PY{o}{=} \PY{n+nb}{max}\PY{p}{(}\PY{l+m+mi}{0}\PY{p}{,} \PY{o}{\PYZhy{}}\PY{p}{(}\PY{n}{p} \PY{o}{\PYZhy{}} \PY{n}{p0}\PY{p}{)}\PY{p}{)}
%        \PY{k}{if} \PY{n}{r} \PY{o}{!=} \PY{l+m+mi}{0}\PY{p}{:}
%            \PY{n}{dp}\PY{o}{.}\PY{n}{append}\PY{p}{(}\PY{n}{r}\PY{p}{)}
%		
%\PY{n}{crvar} \PY{o}{=} \PY{n}{numpy}\PY{o}{.}\PY{n}{percentile}\PY{p}{(}\PY{n}{dp}\PY{p}{,} \PY{p}{[}\PY{l+m+mf}{99.9}\PY{p}{]}\PY{p}{)}
%\PY{n+nb}{print} \PY{p}{(}\PY{n}{crvar}\PY{p}{)}
%
%[124.]
%\end{Verbatim}
%\end{codebox}
%
%\begin{figure}[htb]
%	\centering
%	\includegraphics[width=0.7\textwidth]{figures/credit_metrics.png}
%	\caption{Distribution of the losses estimated from simulation of credit rating changes within 1 year. The red line shows the 99.9\% Cr-VaR.}
%\end{figure}

\section{Credit Valuation Adjustment}
\label{credit-valuation-adjustment}

Suppose you have a portfolio of derivatives. If the counter-party defaults and the present value of the portfolio at default is positive to the surviving party (you), then the actual gain is only given by the recovery fraction of the value. If however the present value is negative to you, you have to pay it in full to the liquidators of the defaulted entity.

This behaviour creates an asymmetry which can be corrected by changing the definition of the deal value as the value without counter-party risk minus a positive adjustment, called \emph{Credit Valuation Adjustment} (CVA).

The CVA can be expressed in the following way:

\begin{equation}
\text{CVA} = (1-R) \int_0^T D(t) \cdot \textrm{EE}(t) dQ(t)
\label{eq:cva}
\end{equation}
where $T$ is the latest maturity in the portfolio, $D$ is the discount factor, EE is the expected exposure or \(\mathbb{E}[\text{max(0, NPV}_\text{portfolio})]\), and $dQ$ is the probability of default between $t$ and $t+dt$.

For an easier computation it is natural to discretize the above integral and use a time grid going from 0 to the portfolio maturity:

\begin{equation}
\text{CVA} = (1-R) \sum_i^n D(t_i) \cdot \mathrm{EE}(t_i) Q(t_{i-1}, t_i)
\label{eq:cva_discrete}
\end{equation}

It is important to note the difference between Credit VaR and CVA. While the former measures the risk of losses faced due to the possible default of some counter-party, the latter measures the pricing component of this risk, i.e. the price adjustment of a contract due to this risk.

\subsection{Debit Valuation Adjustment}

The adjustment seen from the point of view of our counter-party is positive, and is called Debit Valuation Adjustment, DVA. It is positive because the early default of the client itself would imply a discount on the its payment obligations, and this means a gain. So the client marks an adjustment over the risk free price by adding the positive amount called DVA. 

When both parties have the possibility to default, they consistently include both defaults into the valuation. So they will mark a positive CVA to be subtracted and a positive DVA to be added to the default risk free price of the deal. The CVA of one party will be the DVA of the other one and vice versa.

\[
\textrm{price}=\textrm{default risk free price + DVA - CVA}
\]
Now, since
\[
\textrm{default risk free price(A)} = - \textrm{default risk free price(A)}
\]
\[
\textrm{DVA(A)} = \textrm{CVA(B)}
\]
\[
\textrm{DVA(B)} = \textrm{CVA(A)}
\]
we get that eventually
\[
\textrm{price(A)} = -\textrm{price(B)}
\]
so that both parties agree on the price, or, we could say, there is money conservation.

\subsection{CVA Computation}

The computation of the CVA is easily carried on with Monte Carlo simulation. 
First compute the portfolio value at each time point for each MC scenario. Then calculate the CVA using either Eq.~\ref{eq:cva} or its discrete form in Eq.~\ref{eq:cva_discrete}. Finally average the CVA of all the scenarios to get the estimate.

In case of zero coupon bonds the computation of the CVA can be further simplified. Indeed in this case the investor exposure is equal to the bond face value, so it is enough to loop through each day from the pricing date to the bond maturity and compute the CVA using Eq.~\ref{eq:cva_discrete}.

Imagine a 3-years zero coupon bond with a face value of $FV=$~\euro{100}. The bond issuer has the following default probabilities 10\%, 20\% and 30\% for 1, 2 and 3 years respectively and the recovery rate is 40\%. The risk free rate is instead 3\% flat. 

To compute the CVA we need to first define a discount curve and the issuer credit curve. Then we perform a daily loop to sum up all the contributions to the CVA and finally set the price of the bond to the default-risk-free price minus the CVA.

\begin{ipython}
from dateutil.relativedelta import relativedelta
from finmarkets import DiscountCurve, CreditCurve
import math

T = 3
r = 0.03
R = 0.4
FV = 100

pillars = [date.today() + relativedelta(years=i) for i in range(T+1)]
dfs = [math.exp(-r*i) for i in range(T+1)]
dc = DiscountCurve(pillars, dfs)
cc = CreditCurve(pillars, [1, 0.9, 0.8, 0.7])
PV = FV * math.exp(-r*3)

cva = 0
d = date.today()
while d <= pillars[-1]:
    cva += dc.df(d)*(cc.ndp(d) - cc.ndp(d+relativedelta(days=1)))
    d += relativedelta(days=1)
cva *= (1-R) * FV

print ("CVA: {:.2f}".format(cva))
print ("Adjusted Price: {:.2f}".format(PV - cva))
\end{ipython}
\begin{ioutput}
CVA: 17.21
Adjusted Price: 74.18
\end{ioutput}

\section*{Exercises}
\begin{question}
Given the historical series of three stock prices in the file
$\href{https://raw.githubusercontent.com/matteosan1/finance_course/develop/libro/input_files/historical.csv}{\textrm{historical.csv}}$
compute the 1-day 95\% VaR for a portfolio consisting of 40 FOX shares, 35 ABC shares and 25 CBS shares. 
Today's price is the last entry of the series.

\noindent\textbf{Hint:} when simulating the historical scenarios take care of possible NaN values
in the series. 
\end{question}

\cprotEnv\begin{solution}
\begin{ipython}
import pandas as pd
import numpy as np

df = pd.read_csv("https://raw.githubusercontent.com/matteosan1/finance_course/develop/libro/input_files/historical.csv")
df['FOX_RETS'] = np.log1p(df['FOX'].pct_change())
df['CBS_RETS'] = np.log1p(df['CBS'].pct_change())
df['ABC_RETS'] = np.log1p(df['ABC'].pct_change())
\end{ipython}
\begin{ioutput}
         date    FOX    CBS    ABC  FOX_RETS  CBS_RETS  ABC_RETS
0  2018-03-27  36.08  52.35  84.01       NaN       NaN       NaN
1  2018-03-26  36.58  51.69  84.95  0.013763 -0.012688  0.011127
2  2018-03-23  35.45  49.27  84.03 -0.031378 -0.047949 -0.010889
3  2018-03-22  36.18  50.26  85.00  0.020383  0.019894  0.011477
4  2018-03-21  36.30  50.87  89.50  0.003311  0.012064  0.051587
\end{ioutput}
\begin{ipython}
w = np.array([0.4, 0.35, 0.25])

df = df.dropna()
dP = []
current_value = df[['FOX', 'ABC', 'CBS']].iloc[-1].values
for i in range(len(df)):
    variation = (current_value * df[['FOX_RETS', 'ABC_RETS', 'CBS_RETS']].iloc[i])
    dP.append(variation.dot(w))

hist_var = np.percentile(dP, 1)
print ('Historical VaR is  {:.3f}'.format(hist_var))\end{ipython}
\begin{ioutput}
Historical VAR is -1.385
\end{ioutput}
\begin{figure}[htbp]
	\centering
\includegraphics[width=0.7\linewidth]{figures/hist_var_ex}
\end{figure}
\end{solution}

\begin{question}
You have a 3-years call with strike \euro{110}. The underlying initial price is \euro{100} and the mean rate of return is 0.05 with a volatility of 0.15. The risk-free rate is 0.03 flat.
Compute the CVA of the contract assuming a recovery rate of 40\% and default probabilities for the underlying of 10\%, 20\% and 30\% for first, second and third year respectively.
\end{question}

\cprotEnv\begin{solution}
Below ten simulations of the underlying price in the next three years.

\begin{figure}[htbp]
	\centering
\includegraphics[width=0.7\linewidth]{figures/underlying_simulation}
\end{figure}
This is a first implementation of the CVA calculation, not optimized in terms of speed. It takes about 700 seconds to run 1000 simulations.

\begin{ipython}
from datetime import date
from dateutil.relativedelta import relativedelta
from finmarkets import call_price, CreditCurve
from scipy.stats import norm
import numpy as np
import time

np.random.seed(1)
dt = 1/365
K = 110
sigma = 0.15
mu = 0.05
r = 0.03
T = 3
R = 0.4
S = [1, 0.9, 0.8, 0.7]
obs_date = date.today()
pillars = [obs_date + relativedelta(years=i) for i in range(T+1)]
cc = CreditCurve(pillars, S)
t1 = time.time()
scenarios = 1000
cvas = []
St = S0*np.ones(shape=(T*365, scenarios))
for s in range(scenarios):
    cva = 0
    St = 100
    for t in range(0, 365*T):
        St = St * np.exp((mu - 0.5 * sigma**2) * dt + sigma
            * np.sqrt(dt) * norm.rvs(size=1))
        cva += call_price(dt*t, St, K, r, sigma, T)
            *(cc.ndp(obs_date+relativedelta(days=t))-
            cc.ndp(obs_date+relativedelta(days=t+1)))

        cvas.append(cva*(1-R))

print (np.mean(cvas))
print (time.time()-t1)
\end{ipython}
\begin{ioutput}
\end{ioutput}
The next implementation, the one actually used, exploits the \texttt{numpy.array} and it is about 30\% faster.

\begin{ipython}
from datetime import date
from dateutil.relativedelta import relativedelta
from finmarkets import call_price, CreditCurve
from scipy.stats import norm
import numpy as np
import time

np.random.seed(1)
dt = 1/365
K = 110
sigma = 0.15
mu = 0.05
r = 0.03
T = 3
R = 0.4
S = [1, 0.9, 0.8, 0.7]
obs_date = date.today()
pillars = [obs_date + relativedelta(years=i) for i in range(T+1)]
cc = CreditCurve(pillars, S)
t1 = time.time()
scenarios = 1000
cvas = []
St = S0*np.ones(shape=(T*365, scenarios))
for i in range(T*365):
    norms = norm.rvs(size=scenarios)
    St[i, :] = St[i-1, :] * np.exp((mu - 0.5 * sigma**2) * dt + sigma
        * np.sqrt(dt) * norms[:])

for s in range(scenarios):
    cva = 0
    for t in range(365*T):
        cva += call_price(dt*t, St[i, s], K, r, sigma, T)* \
            (cc.ndp(obs_date+relativedelta(days=t))-
             cc.ndp(obs_date+relativedelta(days=t+1)))

        cvas.append(cva*(1-R))

print (np.mean(cvas))
print (time.time()-t1)
\end{ipython}
\begin{ioutput}
3.7521017896018973
459.46695613861084
\end{ioutput}
\end{solution}

\begin{question}
Consider a 1-years call with strike \euro{110}. The underlying initial price is \euro{100} and the mean rate of return is 0.05 with a volatility of 0.15. The risk-free rate is 0.03 flat.
Compute the 99.9\% Credit VaR assuming a recovery rate of 40\% and default probabilities for the underlying of 30\% within next year.
\end{question}

\cprotEnv\begin{solution}

\begin{ipython}
# 460s for 1000 simulations
from datetime import date
from dateutil.relativedelta import relativedelta
from finmarkets import call_price, CreditCurve
from scipy.stats import norm
import numpy as np

np.random.seed(1)
dt = 1/365
K = 110
sigma = 0.15
mu = 0.05
r = 0.03
T = 1
R = 0.4
S = [1, 0.7]
obs_date = date.today()
pillars = [obs_date + relativedelta(years=i) for i in range(T+1)]
cc = CreditCurve(pillars, S)
scenarios = 1000
losses = []
St = S0*np.ones(shape=(T*365, scenarios))
for i in range(T*365):
    norms = norm.rvs(size=scenarios)
    St[i, :] = St[i-1, :] * np.exp((mu - 0.5 * sigma**2) * dt + sigma
        * np.sqrt(dt) * norms[:])

for s in range(scenarios):
    loss = 0
    for t in range(365*T):
        loss += call_price(dt*t, St[i, s], K, r, sigma, T)* \
            (cc.ndp(obs_date+relativedelta(days=t))-
             cc.ndp(obs_date+relativedelta(days=t+1)))

        losses.append(cva*(1-R))

print (np.percentile(cvas, [99.9]))
\end{ipython}
\begin{ioutput}
[9.96586061]
\end{ioutput}

\begin{figure}[htbp]
	\centering
\includegraphics[width=0.7\linewidth]{figures/cr_var_ex}
\end{figure}
\end{solution}







\begin{thebibliography}{9}
\bibitem{bib:var} J. C. Hull, \emph{Options, Futures and Other Derivatives, 7th Ed.}, Value at Risk (Ch. 20), Pearson Prentice Hall, 2009
\bibitem{bib:credit_var} J. C. Hull, \emph{Options, Futures and Other Derivatives, 7th Ed.}, Credit Risk (Ch. 22), Pearson Prentice Hall, 2009
\bibitem{bib:creditmetrics}RiskMetrics Group, \href{https://www.msci.com/documents/10199/93396227-d449-4229-9143-24a94dab122f}{\emph{CreditMetrics}}, J.P. Morgan \& Co., 2007, [Online]
\bibitem{bib:cva} D. Brigo, \emph{Counterparty Risk FAQ: Credit VaR, PFE, CVA, DVA, Closeout, Netting, Collateral, Re-hypothecation, WWR, Basel, Funding, CCDS and Margin Lending}, arXiv: 1111.1331, 2011
\end{thebibliography}
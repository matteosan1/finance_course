\chapter{VaR and Credit Risk}\label{var-and-credit-risk}

\section{Value at Risk}\label{value-at-risk}

The \emph{Value at Risk} (VaR) of a portfolio is usually used when it
is important to know to a certain confidence level how
much will be the maximum loss in the next $N$ days (time horizon).
It can be interpreted as the loss level over $N$ days that has a 
probability of only \((100 - X)\%\) of being exceeded.
By definition it is a function of two parameters:
the time horizon and the confidence level. 

Mathematically the VaR is the loss corresponding to the
\((100-X)\textrm{th}\) percentile of the portfolio 
change in value distribution over the next $N$ days 
(e.g. in Figure~\ref{fig:var_loss}
the graphical representation of the VaR assuming a normal
distribution for the changes of value is shown, with $N=1$ and $X=95$).

\begin{figure}[htb]
\centering
  \includegraphics[width=0.6\linewidth]{figures/95_var.png}
  \caption{Example of 95\% VaR on the distribution of changes in the value.}
  \label{fig:var_loss}
\end{figure}
    
VaR is useful to summarize all the information about the risk of a
portfolio in one single number, but this can also be considered its main
limitation as it implies too much simplification of such a complex task.

Concerning the time horizon parameter it is usually set to $N=1$ since
it is not easy to estimate market variables over periods longer than one
day. Nevertheless to generalize the VaR estimate it is assumed

\begin{equation}
\textrm{N-day VaR} = \textrm{1-day VaR}\times \sqrt{N}
\label{eq:var_horizon}
\end{equation}
This relation is true only if the portfolio changes of value over
the considered period of time have independent and identical normal
distributions with zero mean (otherwise it is just an approximation).

In the next Section the methods that can be used to estimate the VaR
will be reviewed.

\section{How to Estimate the VaR}
\label{how-to-estimate-the-var}

In the proposed examples historical series of Apple and Netflix
are used also we will assume to have a portfolio made of 
\textbf{60\% of AAPL and 40\% NFLX stocks}.
 
You can download the dateset with \texttt{ffn} as in the following code or to
check out \href{https://raw.githubusercontent.com/matteosan1/finance_course/develop/libro/input_files/historical_data.csv}{historical\_data.csv}.

\begin{codebox}
\begin{Verbatim}[commandchars=\\\{\}]
\PY{k+kn}{import} \PY{n+nn}{ffn}
	
\PY{n}{df} \PY{o}{=} \PY{n}{ffn}\PY{o}{.}\PY{n}{get}\PY{p}{(}\PY{l+s+s1}{\PYZsq{}}\PY{l+s+s1}{aapl,nflx}\PY{l+s+s1}{\PYZsq{}}\PY{p}{,} \PY{n}{start}\PY{o}{=}\PY{l+s+s1}{\PYZsq{}}\PY{l+s+s1}{2014\PYZhy{}01\PYZhy{}02}\PY{l+s+s1}{\PYZsq{}}\PY{p}{,} \PY{n}{end}\PY{o}{=}\PY{l+s+s1}{\PYZsq{}}\PY{l+s+s1}{2018\PYZhy{}03\PYZhy{}27}\PY{l+s+s1}{\PYZsq{}}\PY{p}{)}
\PY{n+nb}{print} \PY{p}{(}\PY{n}{df}\PY{o}{.}\PY{n}{head}\PY{p}{(}\PY{p}{)}\PY{p}{)}

                 aapl       nflx
Date
2014-01-02  17.598297  51.831429
2014-01-03  17.211735  51.871429
2014-01-06  17.305593  51.367142
2014-01-07  17.181829  48.500000
2014-01-08  17.290642  48.712856
\end{Verbatim}
\end{codebox}
\noindent
In the following we are going to add a new column with the daily return to the dataframe.

\begin{codebox}
\begin{Verbatim}[commandchars=\\\{\}]
\PY{k+kn}{import} \PY{n+nn}{pandas} \PY{k}{as} \PY{n+nn}{pd}
\PY{k+kn}{import} \PY{n+nn}{numpy} \PY{k}{as} \PY{n+nn}{np}
	
\PY{n}{w} \PY{o}{=} \PY{n}{np}\PY{o}{.}\PY{n}{array}\PY{p}{(}\PY{p}{[}\PY{l+m+mf}{0.6}\PY{p}{,} \PY{l+m+mf}{0.4}\PY{p}{]}\PY{p}{)}
	
\PY{c+c1}{\PYZsh{} uncomment if the file has been checked out}
\PY{c+c1}{\PYZsh{}df = pd.read\PYZus{}csv(\PYZdq{}historical\PYZus{}data.csv\PYZdq{}, index\PYZus{}col=\PYZsq{}Date\PYZsq{})}
	
\PY{n}{df}\PY{p}{[}\PY{l+s+s1}{\PYZsq{}}\PY{l+s+s1}{aapl\PYZus{}rets}\PY{l+s+s1}{\PYZsq{}}\PY{p}{]} \PY{o}{=} \PY{n}{df}\PY{p}{[}\PY{l+s+s1}{\PYZsq{}}\PY{l+s+s1}{aapl}\PY{l+s+s1}{\PYZsq{}}\PY{p}{]}\PY{o}{/}\PY{n}{df}\PY{p}{[}\PY{l+s+s1}{\PYZsq{}}\PY{l+s+s1}{aapl}\PY{l+s+s1}{\PYZsq{}}\PY{p}{]}\PY{o}{.}\PY{n}{shift}\PY{p}{(}\PY{l+m+mi}{1}\PY{p}{)} \PY{o}{\PYZhy{}} \PY{l+m+mi}{1}
\PY{n}{df}\PY{p}{[}\PY{l+s+s1}{\PYZsq{}}\PY{l+s+s1}{nflx\PYZus{}rets}\PY{l+s+s1}{\PYZsq{}}\PY{p}{]} \PY{o}{=} \PY{n}{df}\PY{p}{[}\PY{l+s+s1}{\PYZsq{}}\PY{l+s+s1}{nflx}\PY{l+s+s1}{\PYZsq{}}\PY{p}{]}\PY{o}{/}\PY{n}{df}\PY{p}{[}\PY{l+s+s1}{\PYZsq{}}\PY{l+s+s1}{nflx}\PY{l+s+s1}{\PYZsq{}}\PY{p}{]}\PY{o}{.}\PY{n}{shift}\PY{p}{(}\PY{l+m+mi}{1}\PY{p}{)} \PY{o}{\PYZhy{}} \PY{l+m+mi}{1}

                 aapl       nflx  aapl_rets  nflx_rets
Date
2014-01-02  17.598297  51.831429        NaN        NaN
2014-01-03  17.211735  51.871429  -0.021966   0.000772
2014-01-06  17.305593  51.367142   0.005453  -0.009722
2014-01-07  17.181829  48.500000  -0.007152  -0.055817
2014-01-08  17.290642  48.712856   0.006333   0.004389
\end{Verbatim}
\end{codebox}

\subsection{Historical Simulation}\label{historical-simulation}

In order to estimate the VaR from a historical series, we need to
collect the market variables "affecting" the portfolio over the last $N$ days 
(with $N$ quite large).

The variation over each day in our time interval will provide different
scenarios to be applied to today's market values.
For each scenario we can apply the historical variation and compute the distribution of portfolio change in value (\(\Delta P\)). 
The VaR estimate will be the $(100 - X)\%$ percentile of this distribution. 

Of course such kind of simulation relies on the assumption that past
behaviors are indicative of what might happen in the future, that's why it is important that out historical series was as large as possible.

As an example imagine a portfolio $P$ whose value depends only on two market variables (\(x_1(t), x_2(t)\)). From the historical series of these variables we can simulate the evolution of the portfolio as

\begin{equation}
P_i(t_n+1) = P\left(x_1(t_n)\cfrac{(x_1(t_i)-x_1(t_{i-1}))}{x_1(t_{i-1})}, x_2(t_n)\cfrac{(x_2(t_i)-x_2(t_{i-1}))}{x_2(t_{i-1})}\right)
\end{equation}

Essentially re-scaling the market variables according to the variation
between day \(i\) and \(i-1\) we can draw a distribution of possible
changes in the portfolio value \(P_i\) and then compute the VaR taking
the appropriate percentile.

Using the historical series seen above the procedure can be outlined as follows.
To compute the portfolio value the \emph{scalar product} between the invested 
amount ($w_i$) and the price of each portfolio component ($p_i$) is used. 
It is indicated with \texttt{.dot} function and defined as $\sum_{i} w_i \cdot p_i$.
For more information about scalar product, vector and matrices Chapter~\ref{sec:matrices}. 
Figure~\ref{fig:hist_var} shows the resulting VaR. 

\begin{codebox}
\begin{Verbatim}[commandchars=\\\{\}]
\PY{n}{rets} \PY{o}{=} \PY{p}{[}\PY{p}{]}
\PY{k}{for} \PY{n}{i} \PY{o+ow}{in} \PY{n+nb}{range}\PY{p}{(}\PY{l+m+mi}{1}\PY{p}{,} \PY{n+nb}{len}\PY{p}{(}\PY{n}{df}\PY{p}{)}\PY{p}{)}\PY{p}{:}
    \PY{n}{aapl_rets}\PY{o}{.}\PY{n}{append}\PY{p}{(}\PY{n}{w}\PY{p}{[}\PY{l+m+mi}{0}\PY{p}{]}\PY{o}{*}\PY{n}{df}\PY{o}{.}\PY{n}{iloc}\PY{p}{[}\PY{n}{i}\PY{p}{]}\PY{p}{[}\PY{l+s+s1}{\PYZsq{}}\PY{l+s+s1}{aapl_rets}\PY{l+s+s1}{\PYZsq{}}\PY{p}{]} \PY{o}{+} \PY{n}{w}\PY{p}{[}\PY{l+m+mi}{1}\PY{p}{]}\PY{o}{*}\PY{n}{df}\PY{o}{.}\PY{n}{loc}\PY{p}{[}\PY{n}{i}\PY{p}{]}\PY{p}{[}\PY{l+s+s1}{\PYZsq{}}\PY{l+s+s1}{nflx_rets}\PY{l+s+s1}{\PYZsq{}}\PY{p}{]}\PY{p}{)}
		
\PY{n}{price} \PY{o}{=} \PY{p}{[}\PY{n}{df}\PY{o}{.}\PY{n}{iloc}\PY{p}{[}\PY{o}{\PYZhy{}}\PY{l+m+mi}{1}\PY{p}{]}\PY{p}{[}\PY{l+s+s1}{\PYZsq{}}\PY{l+s+s1}{adj\PYZus{}close}\PY{l+s+s1}{\PYZsq{}}\PY{p}{]}\PY{p}{,} \PY{n}{df}\PY{o}{.}\PY{n}{iloc}\PY{p}{[}\PY{o}{\PYZhy{}}\PY{l+m+mi}{1}\PY{p}{]}\PY{p}{[}\PY{l+s+s1}{\PYZsq{}}\PY{l+s+s1}{adj\PYZus{}close}\PY{l+s+s1}{\PYZsq{}}\PY{p}{]}\PY{p}{]}
\PY{n}{portfolio\PYZus{}price} \PY{o}{=} \PY{n}{w}\PY{o}{.}\PY{n}{dot}\PY{p}{(}\PY{n}{price}\PY{p}{)}
\PY{n}{hist\PYZus{}var} \PY{o}{=} \PY{n}{portfolio\PYZus{}price}\PY{o}{*}\PY{n}{np}\PY{o}{.}\PY{n}{percentile}\PY{p}{(}\PY{n}{rets}\PY{p}{,} \PY{l+m+mi}{1}\PY{p}{)}
\PY{n+nb}{print} \PY{p}{(}\PY{l+s+s1}{\PYZsq{}}\PY{l+s+s1}{Historical VAR is }\PY{l+s+si}{\PYZob{}:.3f\PYZcb{}}\PY{l+s+s1}{\PYZsq{}}\PY{o}{.}\PY{n}{format}\PY{p}{(}\PY{n}{hist\PYZus{}var}\PY{p}{)}\PY{p}{)}

Historical VAR is -2.412
\end{Verbatim}
\end{codebox}

\begin{figure}[htb]
	\centering
	\includegraphics[width=0.7\textwidth]{figures/historical_var.png}
	\caption{Distribution of the changes of values estimated from historical data. 
		The red line shows the 99\% VaR.}
	\label{fig:hist_var}
\end{figure}

\subsection{Model Approach}\label{model-approach}

Our portfolio $P$ consists of different amounts $w_i$ invested on two assets. If with $\Delta x_i$ we denote the daily return of the $i^{th}$ asset the portfolio change in the value can be expressed as

\begin{equation}
\Delta P = \sum_{i=1}^n w_i \Delta x_i
\end{equation}

If we then assume that the asset variations are normally distributed
with zero mean (in this approach is typical to assume the expected change
in a market variable over the considered period equal to zero), \(\Delta P\) will
be also normally distributed (as a sum of normal distribution) with zero
mean.

To estimate the VaR we just need to compute the standard deviation of
$\Delta P$. In the general case with many different assets we define
$\sigma_i$ the daily volatility of the $i^{th}$ asset and with
$\rho_{ij}$ the correlation coefficient between the assets $i$ and $j$.
The variance of $\Delta P$ can then be expressed as

\begin{align}
\begin{split}
\sigma^2_P & = \sum_{i=1}^{n}\sum_{j=1}^{n}\rho_{ij}w_i w_j \sigma_i \sigma_j \\
& = \sum_{i=1}^{n} w_i^2 \sigma_i^2 + 2 \sum_{i=1}^{n}\sum_{j<i}^{n}\rho_{ij}w_i w_j \sigma_i \sigma_j 
\end{split}
\end{align}

As in the previous case if we are interested in a longer time horizon we
can use Eq.~\ref{eq:var_horizon}.

Once we have the variance of \(\Delta P\) it is easy to determine the appropriate percentile using the equations described in Appendix~\ref{transformation-to-standard-normal}.

\subsection{Monte Carlo Simulation}\label{monte-carlo-simulation}

A very useful alternative to previous approaches is using a Monte Carlo simulation to generate the probability distribution for the $\Delta P$. 

Imagine we need to compute the 1-day VaR for our example portfolio.
The simulation can be done either generating random returns from a distribution with mean and standard deviation obtained from the historical data of each stock, or by simulating the evolution of all the portfolio market variables in one day.

Let's start from the first case, computing mean and standard deviation of each historical dataset. We will then compute various simulated returns from a multivariate Gaussian with such means and variances. 
One useful aspect of this method is that in principle other distributions than Gaussian could be used.
Once we have the distribution of returns the VaR can be computed as usual, the result is shown in Fig.~\ref{fig:mc1_var}.

\begin{codebox}
\begin{Verbatim}[commandchars=\\\{\}]
\PY{k+kn}{from} \PY{n+nn}{scipy}\PY{n+nn}{.}\PY{n+nn}{stats} \PY{k}{import} \PY{n}{multivariate\PYZus{}normal}

\PY{n}{mean} \PY{o}{=} \PY{p}{[}\PY{n}{np}\PY{o}{.}\PY{n}{mean}\PY{p}{(}\PY{n}{df}\PY{p}{[}\PY{l+s+s1}{\PYZsq{}}\PY{l+s+s1}{aapl_rets}\PY{l+s+s1}{\PYZsq{}}\PY{p}{]}\PY{p}{)}\PY{p}{,} \PY{n}{np}\PY{o}{.}\PY{n}{mean}\PY{p}{(}\PY{n}{df}\PY{p}{[}\PY{l+s+s1}{\PYZsq{}}\PY{l+s+s1}{nflx_rets}\PY{l+s+s1}{\PYZsq{}}\PY{p}{]}\PY{p}{)}\PY{p}{]}
\PY{n}{cov} \PY{o}{=} \PY{n}{np}\PY{o}{.}\PY{n}{cov}\PY{p}{(}\PY{n}{df}\PY{p}{[}\PY{l+s+s1}{\PYZsq{}}\PY{l+s+s1}{aapl_rets}\PY{l+s+s1}{\PYZsq{}}\PY{p}{]}\PY{p}{[}\PY{l+m+mi}{1}\PY{p}{:}\PY{p}{]}\PY{p}{,} \PY{n}{df}\PY{p}{[}\PY{l+s+s1}{\PYZsq{}}\PY{l+s+s1}{rnflx_ets}\PY{l+s+s1}{\PYZsq{}}\PY{p}{]}\PY{p}{[}\PY{l+m+mi}{1}\PY{p}{:}\PY{o}{\PYZhy{}}\PY{l+m+mi}{1}\PY{p}{]}\PY{p}{)}		
\PY{n}{mvnorm} \PY{o}{=} \PY{n}{multivariate\PYZus{}normal}\PY{p}{(}\PY{n}{mean}\PY{o}{=}\PY{n}{mean}\PY{p}{,} \PY{n}{cov}\PY{o}{=}\PY{n}{cov}\PY{p}{)}
		
\PY{n}{np}\PY{o}{.}\PY{n}{random}\PY{o}{.}\PY{n}{seed}\PY{p}{(}\PY{l+m+mi}{1}\PY{p}{)}
\PY{n}{n\PYZus{}sims} \PY{o}{=} \PY{l+m+mi}{100000}
\PY{n}{sim\PYZus{}returns} \PY{o}{=} \PY{n}{mvnorm}\PY{o}{.}\PY{n}{rvs}\PY{p}{(}\PY{n}{n\PYZus{}sims}\PY{p}{)}
\PY{n}{p\PYZus{}returns} \PY{o}{=} \PY{p}{[}\PY{n}{w}\PY{o}{.}\PY{n}{dot}\PY{p}{(}\PY{n}{s}\PY{p}{)} \PY{k}{for} \PY{n}{s} \PY{o+ow}{in} \PY{n}{sim\PYZus{}returns}\PY{p}{]}
\PY{n}{mc\PYZus{}var} \PY{o}{=} \PY{n}{portfolio\PYZus{}price} \PY{o}{*} \PY{n}{np}\PY{o}{.}\PY{n}{percentile}\PY{p}{(}\PY{n}{p\PYZus{}returns}\PY{p}{,} \PY{l+m+mi}{1}\PY{p}{)}
\PY{n+nb}{print}\PY{p}{(}\PY{l+s+s1}{\PYZsq{}}\PY{l+s+s1}{Simulated VAR is }\PY{l+s+si}{\PYZob{}:.3f\PYZcb{}}\PY{l+s+s1}{\PYZsq{}}\PY{o}{.}\PY{n}{format}\PY{p}{(}\PY{n}{mc\PYZus{}var}\PY{p}{)}\PY{p}{)}

Simulated VAR is -2.197
\end{Verbatim}
\end{codebox}

\begin{figure}[htb]
	\centering
	\includegraphics[width=0.7\textwidth]{figures/sim1_var}
	\caption{Distribution of the changes of values estimated from simulated data. The red line shows the 99\% VaR.}
	\label{fig:mc1_var}
\end{figure}

This result can be compared with the VaR estimated with a simulation of the daily evolution of the stock price. We will use the log-normal evolution described in Section~\ref{derivation-of-log-normal-stochastic-differential-equation} where $\mu$ and $\sigma$ are the mean and variance estimated from the historical series. Figure~\ref{fig:mc2_var} shows the resulting distribution of the returns.

\begin{codebox}
\begin{Verbatim}[commandchars=\\\{\}]
\PY{k+kn}{from} \PY{n+nn}{numpy}\PY{n+nn}{.}\PY{n+nn}{random} \PY{k}{import} \PY{n}{normal}
\PY{k+kn}{from} \PY{n+nn}{numpy} \PY{k}{import} \PY{n}{exp}\PY{p}{,} \PY{n}{sqrt}
		
\PY{n}{T} \PY{o}{=} \PY{l+m+mi}{1}
\PY{n}{trials} \PY{o}{=} \PY{l+m+mi}{100000}
\PY{n}{dP} \PY{o}{=} \PY{p}{[}\PY{p}{]}
		
\PY{k}{for} \PY{n}{\PYZus{}} \PY{o+ow}{in} \PY{n+nb}{range}\PY{p}{(}\PY{n}{trials}\PY{p}{)}\PY{p}{:}
    \PY{n}{s} \PY{o}{=} \PY{l+m+mi}{0}
    \PY{k}{for} \PY{n}{i} \PY{o+ow}{in} \PY{n+nb}{range}\PY{p}{(}\PY{l+m+mi}{2}\PY{p}{)}\PY{p}{:}
        \PY{n}{s} \PY{o}{+}\PY{o}{=} \PY{n}{w}\PY{p}{[}\PY{n}{i}\PY{p}{]} \PY{o}{*} \PY{n}{price}\PY{p}{[}\PY{n}{i}\PY{p}{]} \PY{o}{*} \PY{n}{exp}\PY{p}{(}\PY{p}{(}\PY{n}{mean}\PY{p}{[}\PY{n}{i}\PY{p}{]} \PY{o}{\PYZhy{}} \PY{l+m+mf}{0.5} \PY{o}{*} \PY{n}{cov}\PY{p}{[}\PY{n}{i}\PY{p}{]}\PY{p}{[}\PY{n}{i}\PY{p}{]}\PY{p}{)} \PY{o}{*} \PY{n}{T} \PY{o}{+} 
             \PY{n}{sqrt}\PY{p}{(}\PY{n}{cov}\PY{p}{[}\PY{n}{i}\PY{p}{]}\PY{p}{[}\PY{n}{i}\PY{p}{]}\PY{p}{)} \PY{o}{*} \PY{n}{sqrt}\PY{p}{(}\PY{n}{T}\PY{p}{)} \PY{o}{*} \PY{n}{normal}\PY{p}{(}\PY{p}{)}\PY{p}{)}
        \PY{n}{dP}\PY{o}{.}\PY{n}{append}\PY{p}{(}\PY{n}{portfolio\PYZus{}price} \PY{o}{\PYZhy{}} \PY{n}{s}\PY{p}{)}
		
\PY{n}{mc\PYZus{}var2} \PY{o}{=} \PY{n}{np}\PY{o}{.}\PY{n}{percentile}\PY{p}{(}\PY{n}{dP}\PY{p}{,} \PY{l+m+mi}{1}\PY{p}{)}
\PY{n+nb}{print}\PY{p}{(}\PY{l+s+s1}{\PYZsq{}}\PY{l+s+s1}{Simulated VAR is }\PY{l+s+si}{\PYZob{}:.3f\PYZcb{}}\PY{l+s+s1}{\PYZsq{}}\PY{o}{.}\PY{n}{format}\PY{p}{(}\PY{n}{mc\PYZus{}var2}\PY{p}{)}\PY{p}{)}

Simulated VAR is -1.913
\end{Verbatim}
\end{codebox}

\begin{figure}[htb]
	\centering
	\includegraphics[width=0.7\textwidth]{figures/sim2_var}
	\caption{Distribution of the changes of values estimated from simulation of the evolution of the stock prices. The red line shows the 99\% VaR.}
	\label{fig:mc2_var}
\end{figure}

\subsection{Stress and Back Testing}\label{stress-testing-and-back-testing}

In addition to calculating the value at risk of a portfolio, it could be useful to 
check how it would behave under the most extreme moves seen in the last years.

This kind of test is called \emph{stress test} and it is done by extracting from the historical series, particular days with exceptional large variation of our market variables, in order to check how the portfolio value moves.
 
The idea is to take into account extreme events that can happen more frequently in reality than in a simulation (where usually Gaussian tails are assumed hence are quite small). For example a 5-standard deviation move is expected to happen once every 7000 years but in practice can be observed twice over 10 years.

\begin{attention}
\subsubsection{$n\sigma$ Event Likelihood}
Such event has a probability to occur equal to $n\sigma$ so the occurrence frequency 
can computed as 1 over $252\cdot\mathbb{P}(|x| \ge n\sigma$ (the 252 factor comes to the number of working days per year).

\begin{Verbatim}[commandchars=\\\{\}]
\PY{k+kn}{from} \PY{n+nn}{scipy}\PY{n+nn}{.}\PY{n+nn}{stats} \PY{k}{import} \PY{n}{norm}

\PY{n}{prob} \PY{o}{=} \PY{n}{norm}\PY{o}{.}\PY{n}{cdf}\PY{p}{(}\PY{o}{\PYZhy{}}\PY{l+m+mi}{5}\PY{p}{)} \PY{o}{*} \PY{l+m+mi}{2} \PY{c+c1}{\PYZsh{} e.g. consider +\PYZhy{} 5sigma movements}
\PY{n}{nyears} \PY{o}{=} \PY{l+m+mi}{1}\PY{o}{/}\PY{n}{prob}\PY{o}{/}\PY{l+m+mi}{252}
\PY{n+nb}{print} \PY{p}{(}\PY{n}{nyears}\PY{p}{)}

6921.737673091067
\end{Verbatim}
\noindent
So about seven thousand years as stated before.
\end{attention}

Another important check that could be done is the so-called \emph{back testing}
which consists of checking how well the VaR estimate would have
performed in the past. Basically it has to be tested how often the daily
loss exceeded the N-days X\% VaR just computed. If it happens on about
(100-X)\% of the times we can be confident that our estimate is correct.

\section{Credit VaR}\label{credit-var-cr-var}

%exposure at any given future time is the
%larger between zero and the market value of the portfolio of derivative
%positions with a counterparty that would be lost if the counterparty
%were to default with zero recovery at that time.

\emph{Credit VaR} is defined in the usual way Value at Risk measures are (i.e. as percentile of a loss distribution).
In this case we are concerned with the default risk associated to one or
multiple counter-parties in a specific portfolio instead of to the market risk.

To derive the loss distribution we need to consider the exposure $\textrm{EE}(\tau)$ defined as the sum of the discounted cash flows at the default date $\tau$.
The corresponding loss is then given by

\begin{equation}
L_{\tau, \hat{T}} = (1 - R) \cdot \textrm{EE}(\tau)
\end{equation}
where \(\hat{T}\) is the risk horizon and $L$ is non-zero only in
scenarios of early counter-party default. 

Given the above definitions we can express the Credit VaR as the 
q-quantile of $L_{\tau, \hat{T}}$.
With respect to the Value at Risk, the time horizon is usually set to one year and the percentile to $99.9^{th}$, so it returns the loss that is 
exceeded only in 1 case out of 1000. 

Credit VaR is actually either the difference of the percentile from the mean, or the percentile itself. There is more than one possible definition, anyway we  will use the latter.

%Consider your portfolio has a call option on equity with a final maturity of two years. To get the Credit-Var, roughly, you simulate the underlying equity up to one year, and obtain a number of scenarios for the underlying equity in one year. Also, you need to simulate the default scenarios up to one year, to know in each scenario whether the counter-parties have defaulted or not. 
%
%And then in each scenario at one year, if the counter-party
%has defaulted there will be a recovery value and all else will be lost.
%Otherwise, we price the call option over the remaining year using for
%example a Black Scholes formula. But this price is like taking the
%expected value of the call option payoff in two years, conditional on
%each scenario for the underlying equity in one year. Because this is
%pricing, this expected value will be taken under the pricing measure Q,
%not P. This gives the Black Scholes formula if the underlying equity
%follows a geometric brownian motion under Q.

\subsection{Credit VaR and MC Simulation}

Credit VaR can be calculated through a simulation of the basic financial
variables underlying the portfolio up to the risk horizon; the simulation must of course include possible defaults of the counter-parties. 

In each simulation of the basic financial variables the portfolio is priced obtaining a number of scenarios to draw the loss distribution. It is then straightforward to derive the Credit VaR from it.

Consider a portfolio of twenty zero coupon bonds each one with a default probability of 8\% and the same face value (\euro{100}). The recovery rate in case of default is $R=40\%$ and the risk free rate is 1\%.

\begin{codebox}
\begin{Verbatim}[commandchars=\\\{\}]
\PY{k+kn}{import} \PY{n+nn}{numpy} \PY{k}{as} \PY{n+nn}{np}
\PY{k+kn}{from} \PY{n+nn}{datetime} \PY{k}{import} \PY{n}{date}
\PY{k+kn}{from} \PY{n+nn}{dateutil}\PY{n+nn}{.}\PY{n+nn}{relativedelta} \PY{k}{import} \PY{n}{relativedelta}
\PY{k+kn}{from} \PY{n+nn}{finmarkets} \PY{k}{import} \PY{n}{DiscountCurve}\PY{p}{,} \PY{n}{CreditCurve}
\PY{k+kn}{from} \PY{n+nn}{scipy}\PY{n+nn}{.}\PY{n+nn}{stats} \PY{k}{import} \PY{n}{uniform}
		
\PY{n}{bonds} \PY{o}{=} \PY{l+m+mi}{20}
\PY{n}{S} \PY{o}{=} \PY{p}{[}\PY{l+m+mi}{1}\PY{o}{\PYZhy{}}\PY{l+m+mf}{0.08} \PY{k}{for} \PY{n}{\PYZus{}} \PY{o+ow}{in} \PY{n+nb}{range}\PY{p}{(}\PY{n}{bonds}\PY{p}{)}\PY{p}{]}
\PY{n}{FV} \PY{o}{=} \PY{p}{[}\PY{l+m+mi}{100} \PY{k}{for} \PY{n}{\PYZus{}} \PY{o+ow}{in} \PY{n+nb}{range}\PY{p}{(}\PY{l+m+mi}{20}\PY{p}{)}\PY{p}{]}
\PY{n}{R} \PY{o}{=} \PY{l+m+mf}{0.4}
\PY{n}{r} \PY{o}{=} \PY{l+m+mf}{0.01}
\PY{n}{obs\PYZus{}date} \PY{o}{=} \PY{n}{date}\PY{o}{.}\PY{n}{today}\PY{p}{(}\PY{p}{)}
		
\PY{n}{pillars} \PY{o}{=} \PY{p}{[}\PY{n}{obs\PYZus{}date}\PY{o}{+}\PY{n}{relativedelta}\PY{p}{(}\PY{n}{years}\PY{o}{=}\PY{n}{i}\PY{p}{)} \PY{k}{for} \PY{n}{i} \PY{o+ow}{in} \PY{n+nb}{range}\PY{p}{(}\PY{l+m+mi}{2}\PY{p}{)}\PY{p}{]}
\PY{n}{dfs} \PY{o}{=} \PY{p}{[}\PY{l+m+mi}{1}\PY{o}{/}\PY{p}{(}\PY{l+m+mi}{1}\PY{o}{+}\PY{n}{r}\PY{p}{)}\PY{o}{*}\PY{o}{*}\PY{n}{i} \PY{k}{for} \PY{n}{i} \PY{o+ow}{in} \PY{n+nb}{range}\PY{p}{(}\PY{l+m+mi}{2}\PY{p}{)}\PY{p}{]}
\PY{n}{dc} \PY{o}{=} \PY{n}{DiscountCurve}\PY{p}{(}\PY{n}{obs\PYZus{}date}\PY{p}{,} \PY{n}{pillars}\PY{p}{,} \PY{n}{dfs}\PY{p}{)}
\PY{n}{ccs} \PY{o}{=} \PY{p}{[}\PY{p}{]}
\PY{k}{for} \PY{n}{i} \PY{o+ow}{in} \PY{n+nb}{range}\PY{p}{(}\PY{n}{bonds}\PY{p}{)}\PY{p}{:}
    \PY{n}{ccs}\PY{o}{.}\PY{n}{append}\PY{p}{(}\PY{n}{CreditCurve}\PY{p}{(}\PY{n}{pillars}\PY{p}{,} \PY{p}{[}\PY{l+m+mi}{1}\PY{p}{,} \PY{n}{S}\PY{p}{[}\PY{n}{i}\PY{p}{]}\PY{p}{]}\PY{p}{)}\PY{p}{)}
		
\PY{n}{scenarios} \PY{o}{=} \PY{l+m+mi}{100000}
\PY{n}{losses} \PY{o}{=} \PY{p}{[}\PY{p}{]}
\PY{k}{for} \PY{n}{\PYZus{}} \PY{o+ow}{in} \PY{n+nb}{range}\PY{p}{(}\PY{n}{scenarios}\PY{p}{)}\PY{p}{:}
    \PY{n}{loss} \PY{o}{=} \PY{l+m+mi}{0}
    \PY{n}{unif} \PY{o}{=} \PY{n}{uniform}\PY{o}{.}\PY{n}{rvs}\PY{p}{(}\PY{n}{size}\PY{o}{=}\PY{n}{bonds}\PY{p}{)}
    \PY{k}{for} \PY{n}{i} \PY{o+ow}{in} \PY{n+nb}{range}\PY{p}{(}\PY{n}{bonds}\PY{p}{)}\PY{p}{:}
        \PY{k}{if} \PY{n}{unif}\PY{p}{[}\PY{n}{i}\PY{p}{]} \PY{o}{\PYZgt{}} \PY{n}{ccs}\PY{p}{[}\PY{n}{i}\PY{p}{]}\PY{o}{.}\PY{n}{ndp}\PY{p}{(}\PY{n}{pillars}\PY{p}{[}\PY{o}{\PYZhy{}}\PY{l+m+mi}{1}\PY{p}{]}\PY{p}{)}\PY{p}{:}
            \PY{n}{loss} \PY{o}{+}\PY{o}{=} \PY{p}{(}\PY{l+m+mi}{1} \PY{o}{\PYZhy{}} \PY{n}{R}\PY{p}{)}\PY{o}{*}\PY{n}{FV}\PY{p}{[}\PY{n}{i}\PY{p}{]}\PY{o}{*}\PY{n}{dc}\PY{o}{.}\PY{n}{df}\PY{p}{(}\PY{n}{pillars}\PY{p}{[}\PY{o}{\PYZhy{}}\PY{l+m+mi}{1}\PY{p}{]}\PY{p}{)} 
        \PY{n}{losses}\PY{o}{.}\PY{n}{append}\PY{p}{(}\PY{n}{loss}\PY{p}{)}
        
\PY{n+nb}{print} \PY{p}{(}\PY{n}{np}\PY{o}{.}\PY{n}{percentile}\PY{p}{(}\PY{n}{losses}\PY{p}{,} \PY{p}{[}\PY{l+m+mf}{99.9}\PY{p}{]}\PY{p}{)}\PY{p}{)}

[415.84158416]
\end{Verbatim}
\end{codebox}
\noindent
Figure~\ref{fig:credit_var} shows the loss distribution of our simple portfolio.

\begin{figure}[htb]
\centering
\includegraphics[width=0.7\textwidth]{figures/credit_var_zcb.png}
\caption{Distribution of losses in a portfolio made of twenty zero coupon bonds. The distribution is discrete since all the ZCB have the same face value and recovery rate. The red line indicates the Credit VaR.}
\label{fig:credit_var}
\end{figure}

\subsection{Credit VaR and One Factor Copula Model}
Consider a portfolio made of similar assets. As an approximation assume that the probability of default is the same for each counter-party and that the correlation between each pair is the same and equal to $\rho$. If we use the One Factor Copula model to describe the default correlations, Eq.~\ref{eq:gaussian_one_factor_copula}

\begin{equation}
Q^{\textrm{corr}}_M(T) = \Phi\Big(\cfrac{\Phi^{-1}[Q(T)]-M\sqrt{\rho}}{\sqrt{1-\rho}}\Big)
\label{eq:conditional_default_prob}
\end{equation}
where $\Phi$ is the cumulative distribution function of the standard normal.

Eq~\ref{eq:conditional_default_prob} gives the percentage of defaults by time $T$ given $M$. Indeed if there are $n$ counter-parties with the same default probability $Q^{\textrm{corr}}_M(T)$ the percentage of defaults at time $T$ is $Q^{\textrm{corr}}_M$ itself ($\textrm{\% of defaults} = \textrm{n\_defaults}/n = n\cdot Q^{\textrm{corr}}_M/n$).

Since $M$ is distributed according to a standard normal we can be $X\%$ certain that its value will be \emph{greater} than $\Phi^{-1}(1-X)=-\Phi^{-1}(X)$, where the last equality holds due to the symmetry of the Gaussian distribution (see Figure~\ref{fig:certain_for_X}).

\begin{figure}[htb]
	\centering
	\includegraphics[width=0.7\textwidth]{figures/certain_for_X.png}
	\caption{$X\%$ probability to get an higher value a threshold for a normally distributed random variable.}
	\label{fig:certain_for_X}
\end{figure} 

Once the time $T$ as been chosen the only random variable appearing in the conditional default probability expression of Eq.~\ref{eq:conditional_default_prob} is $M$, therefore we can be $X\%$ certain that the percentage of losses over $T$ years on a large portfolio will be \textbf{less} than $V(X,T)$ where
\[
V(X,T)= \Phi\Big(\cfrac{\Phi^{-1}[Q(T)]+\hat{m}\sqrt{\rho}}{\sqrt{1-\rho}}\Big) = \Phi\Big(\cfrac{\Phi^{-1}[Q(T)]+\Phi^{-1}(X)\sqrt{\rho}}{\sqrt{1-\rho}}\Big)
\]
When the confidence level is $X\%$ and the time horizon is $T$, a rough estimate of the Credit VaR is therefore $P(1-R)V(X,T)$, where $P$ is the size of the portfolio and $R$ is the recovery rate.

Suppose that a bank has a total of \euro{100} million of retail exposures. The 1-year probability of default averages to 2\% and the recovery rate averages to 60\%. The copula correlation parameter is estimated as 0.1.

\begin{codebox}
\begin{Verbatim}[commandchars=\\\{\}]
\PY{k+kn}{from} \PY{n+nn}{scipy}\PY{n+nn}{.}\PY{n+nn}{stats} \PY{k}{import} \PY{n}{norm}
\PY{k+kn}{from} \PY{n+nn}{math} \PY{k}{import} \PY{n}{sqrt}
		
\PY{n}{X} \PY{o}{=} \PY{l+m+mf}{0.999}
\PY{n}{rho} \PY{o}{=} \PY{l+m+mf}{0.1}
\PY{n}{R} \PY{o}{=} \PY{l+m+mf}{0.6}
\PY{n}{Q} \PY{o}{=} \PY{l+m+mf}{0.02}
\PY{n}{exposure} \PY{o}{=} \PY{l+m+mf}{100e6}
		
\PY{n}{num} \PY{o}{=} \PY{n}{norm}\PY{o}{.}\PY{n}{ppf}\PY{p}{(}\PY{n}{Q}\PY{p}{)} \PY{o}{+} \PY{n}{sqrt}\PY{p}{(}\PY{n}{rho}\PY{p}{)}\PY{o}{*}\PY{n}{norm}\PY{o}{.}\PY{n}{ppf}\PY{p}{(}\PY{n}{X}\PY{p}{)}
\PY{n}{den} \PY{o}{=} \PY{n}{sqrt}\PY{p}{(}\PY{l+m+mi}{1}\PY{o}{\PYZhy{}}\PY{n}{rho}\PY{p}{)}
\PY{n}{V} \PY{o}{=} \PY{n}{norm}\PY{o}{.}\PY{n}{cdf}\PY{p}{(}\PY{n}{num}\PY{o}{/}\PY{n}{den}\PY{p}{)}
		
\PY{n}{cr\PYZus{}var} \PY{o}{=} \PY{n}{exposure}\PY{o}{*}\PY{n}{V}\PY{o}{*}\PY{p}{(}\PY{l+m+mi}{1}\PY{o}{\PYZhy{}}\PY{n}{R}\PY{p}{)}
\PY{n+nb}{print} \PY{p}{(}\PY{l+s+s2}{\PYZdq{}}\PY{l+s+s2}{Cr\PYZhy{}VaR: }\PY{l+s+si}{\PYZob{}:.0f\PYZcb{}}\PY{l+s+s2}{\PYZdq{}}\PY{o}{.}\PY{n}{format}\PY{p}{(}\PY{n+nb}{round}\PY{p}{(}\PY{n}{cr\PYZus{}var}\PY{p}{,} \PY{o}{\PYZhy{}}\PY{l+m+mi}{4}\PY{p}{)}\PY{p}{)}\PY{p}{)}

Cr-VaR: 5130000
\end{Verbatim}
\end{codebox}
\noindent
The 1-year 99.9\% Credit VaR is therefore \euro{5.13} million.

\subsection{CreditMetrics}
Another popular approach to compute Credti VaR is \emph{CreditMetrics}. It involves estimating a probability distribution of credit losses by carrying out Monte Carlo simulations of the counter-party credit rating changes.

Imagine we would like to determine the probability distribution of losses over 1-year period. On each simulation, we are going to determine the credit rating of each counter-party using the estimated probability of migration between one rate to another (or to default). Since the portfolio value depends on its asset ratings we can determine the eventual losses. 

As an example consider Table~\ref{tab:credit_ratings}, it shows the percentage probability of a bond moving from one category to another during a 1-year period.

\begin{table}[htb]
	\centering
	\begin{tabular}{|l|c|c|c|c|c|c|c|c|}
	\hline
	Initial rating & AAA & AA & A & BBB & BB & B & CCC & default \\
	\hline
	\hline
	AAA & 90.81 & 8.33 & 0.68 & 0.06 & 0.08 & 0.02 & 0.01& 0.01 \\ 
	\hline
	AA & 0.70 & 90.65 & 7.79 & 0.64 & 0.06 & 0.13 & 0.02 & 0.01 \\ 
	\hline
	A & 0.09 & 2.27 & 91.05 & 5.52 & 0.74 & 0.26 & 0.01 & 0.06 \\ 
	\hline
	BBB & 0.02 & 0.33 & 5.95 & 85.93 & 5.30 & 1.17 & 1.12 & 0.18 \\
	\hline
	BB & 0.03 & 0.14 & 0.67 & 7.73 & 80.53 & 8.84 & 1.00 & 1.06 \\
	\hline
	B & 0.01 & 0.11 & 0.24 & 0.43 & 6.48 & 83.46 & 4.07 & 5.20 \\
	\hline
	CCC & 0.21 & 0 & 0.22 & 1.30 & 2.38 & 11.24 & 64.86 & 19.79 \\		
	\hline
\end{tabular}
\caption{Example of table with transition probabilities (in percent) between different credit rating categories.}
\label{tab:credit_ratings}
\end{table}

For a correct implementation of this technique credit rate changes cannot be assumed independent, hence a copula approach could be implemented also in this case.
%As an example suppose to simulate rating change of a portfolio of nine bonds with various ratings over 1-year period. The correlation between them is 0.2.
%
%\begin{codebox}
%\begin{Verbatim}[commandchars=\\\{\}]
%\PY{k+kn}{from} \PY{n+nn}{scipy}\PY{n+nn}{.}\PY{n+nn}{stats} \PY{k}{import} \PY{n}{multivariate\PYZus{}normal}\PY{p}{,} \PY{n}{norm}
%\PY{k+kn}{import} \PY{n+nn}{numpy}
%		
%\PY{c+c1}{\PYZsh{} AAA, AA, A, BBB, BB, B, CCC, Def}
%\PY{n}{table} \PY{o}{=} \PY{p}{[}\PY{p}{[}\PY{l+m+mf}{90.81}\PY{p}{,} \PY{l+m+mf}{8.33}\PY{p}{,} \PY{l+m+mf}{0.68}\PY{p}{,} \PY{l+m+mf}{0.06}\PY{p}{,} \PY{l+m+mf}{0.08}\PY{p}{,} \PY{l+m+mf}{0.02}\PY{p}{,} \PY{l+m+mf}{0.01}\PY{p}{,} \PY{l+m+mf}{0.01}\PY{p}{]}\PY{p}{,}
%         \PY{p}{[}\PY{l+m+mf}{0.70}\PY{p}{,} \PY{l+m+mf}{90.65}\PY{p}{,} \PY{l+m+mf}{7.79}\PY{p}{,} \PY{l+m+mf}{0.64}\PY{p}{,} \PY{l+m+mf}{0.06}\PY{p}{,} \PY{l+m+mf}{0.13}\PY{p}{,} \PY{l+m+mf}{0.02}\PY{p}{,} \PY{l+m+mf}{0.01}\PY{p}{]}\PY{p}{,}
%         \PY{p}{[}\PY{l+m+mf}{0.09}\PY{p}{,} \PY{l+m+mf}{2.27}\PY{p}{,} \PY{l+m+mf}{91.05}\PY{p}{,} \PY{l+m+mf}{5.52}\PY{p}{,} \PY{l+m+mf}{0.74}\PY{p}{,} \PY{l+m+mf}{0.26}\PY{p}{,} \PY{l+m+mf}{0.01}\PY{p}{,} \PY{l+m+mf}{0.06}\PY{p}{]}\PY{p}{,}
%         \PY{p}{[}\PY{l+m+mf}{0.02}\PY{p}{,} \PY{l+m+mf}{0.33}\PY{p}{,} \PY{l+m+mf}{5.95}\PY{p}{,} \PY{l+m+mf}{85.93}\PY{p}{,} \PY{l+m+mf}{5.30}\PY{p}{,} \PY{l+m+mf}{1.17}\PY{p}{,} \PY{l+m+mf}{1.12}\PY{p}{,} \PY{l+m+mf}{0.18}\PY{p}{]}\PY{p}{,}
%         \PY{p}{[}\PY{l+m+mf}{0.03}\PY{p}{,} \PY{l+m+mf}{0.14}\PY{p}{,} \PY{l+m+mf}{0.67}\PY{p}{,} \PY{l+m+mf}{7.73}\PY{p}{,} \PY{l+m+mf}{80.53}\PY{p}{,} \PY{l+m+mf}{8.84}\PY{p}{,} \PY{l+m+mf}{1.00}\PY{p}{,} \PY{l+m+mf}{1.06}\PY{p}{]}\PY{p}{,}
%         \PY{p}{[}\PY{l+m+mf}{0.01}\PY{p}{,} \PY{l+m+mf}{0.11}\PY{p}{,} \PY{l+m+mf}{0.24}\PY{p}{,} \PY{l+m+mf}{0.43}\PY{p}{,} \PY{l+m+mf}{6.48}\PY{p}{,} \PY{l+m+mf}{83.46}\PY{p}{,} \PY{l+m+mf}{4.07}\PY{p}{,} \PY{l+m+mf}{5.20}\PY{p}{]}\PY{p}{,}
%         \PY{p}{[}\PY{l+m+mf}{0.21}\PY{p}{,} \PY{l+m+mi}{0}\PY{p}{,} \PY{l+m+mf}{0.22}\PY{p}{,} \PY{l+m+mf}{1.30}\PY{p}{,} \PY{l+m+mf}{2.38}\PY{p}{,} \PY{l+m+mf}{11.24}\PY{p}{,} \PY{l+m+mf}{64.86}\PY{p}{,} \PY{l+m+mf}{19.79}\PY{p}{]}\PY{p}{]}
%				
%\PY{n}{t} \PY{o}{=} \PY{n}{numpy}\PY{o}{.}\PY{n}{array}\PY{p}{(}\PY{n}{table}\PY{p}{)}
%\PY{n}{table\PYZus{}gauss} \PY{o}{=} \PY{n}{norm}\PY{o}{.}\PY{n}{ppf}\PY{p}{(}\PY{n}{np}\PY{o}{.}\PY{n}{cumsum}\PY{p}{(}\PY{n}{t}\PY{o}{/}\PY{l+m+mf}{100.}\PY{p}{,} \PY{n}{axis}\PY{o}{=}\PY{l+m+mi}{1}\PY{p}{)}\PY{p}{)}
%\PY{n}{table\PYZus{}gauss}\PY{p}{[}\PY{p}{:}\PY{p}{,} \PY{o}{\PYZhy{}}\PY{l+m+mi}{1}\PY{p}{]} \PY{o}{=} \PY{n}{np}\PY{o}{.}\PY{n}{inf}
%		
%\PY{n}{N} \PY{o}{=} \PY{p}{[}\PY{l+m+mi}{100}\PY{p}{,} \PY{l+m+mi}{95}\PY{p}{,} \PY{l+m+mi}{92}\PY{p}{,} \PY{l+m+mi}{85}\PY{p}{,} \PY{l+m+mi}{80}\PY{p}{,} \PY{l+m+mi}{70}\PY{p}{,} \PY{l+m+mi}{60}\PY{p}{]}
%\PY{n}{portfolio} \PY{o}{=} \PY{p}{[}\PY{l+m+mi}{2}\PY{p}{,} \PY{l+m+mi}{3}\PY{p}{,} \PY{l+m+mi}{3}\PY{p}{,} \PY{l+m+mi}{4}\PY{p}{,} \PY{l+m+mi}{5}\PY{p}{,} \PY{l+m+mi}{6}\PY{p}{,} \PY{l+m+mi}{3}\PY{p}{,} \PY{l+m+mi}{4}\PY{p}{,} \PY{l+m+mi}{2}\PY{p}{]}
%\PY{n}{R} \PY{o}{=} \PY{l+m+mf}{0.4}
%		
%\PY{n}{p0} \PY{o}{=} \PY{l+m+mi}{0}
%\PY{k}{for} \PY{n}{i} \PY{o+ow}{in} \PY{n}{portfolio}\PY{p}{:}
%    \PY{n}{p0} \PY{o}{+}\PY{o}{=} \PY{n}{N}\PY{p}{[}\PY{n}{i}\PY{p}{]}
%		
%\PY{n}{numpy}\PY{o}{.}\PY{n}{random}\PY{o}{.}\PY{n}{seed}\PY{p}{(}\PY{l+m+mi}{1}\PY{p}{)}
%\PY{n}{mvnorm} \PY{o}{=} \PY{n}{multivariate\PYZus{}normal}\PY{p}{(}\PY{n}{mean}\PY{o}{=}\PY{p}{[}\PY{l+m+mi}{0} \PY{k}{for} \PY{n}{\PYZus{}} \PY{o+ow}{in} \PY{n+nb}{range}\PY{p}{(}\PY{l+m+mi}{9}\PY{p}{)}\PY{p}{]}\PY{p}{,}
%          \PY{n}{cov}\PY{o}{=}\PY{p}{[}\PY{p}{[}\PY{l+m+mi}{1}\PY{p}{,} \PY{l+m+mf}{0.2}\PY{p}{,} \PY{l+m+mf}{0.2}\PY{p}{,} \PY{l+m+mf}{0.2}\PY{p}{,} \PY{l+m+mf}{0.2}\PY{p}{,} \PY{l+m+mf}{0.2}\PY{p}{,} \PY{l+m+mf}{0.2}\PY{p}{,} \PY{l+m+mf}{0.2}\PY{p}{,} \PY{l+m+mf}{0.2}\PY{p}{]}\PY{p}{,}
%          \PY{p}{[}\PY{l+m+mf}{0.2}\PY{p}{,} \PY{l+m+mi}{1}\PY{p}{,} \PY{l+m+mf}{0.2}\PY{p}{,} \PY{l+m+mf}{0.2}\PY{p}{,} \PY{l+m+mf}{0.2}\PY{p}{,} \PY{l+m+mf}{0.2}\PY{p}{,} \PY{l+m+mf}{0.2}\PY{p}{,} \PY{l+m+mf}{0.2}\PY{p}{,} \PY{l+m+mf}{0.2}\PY{p}{]}\PY{p}{,}
%          \PY{p}{[}\PY{l+m+mf}{0.2}\PY{p}{,} \PY{l+m+mf}{0.2}\PY{p}{,} \PY{l+m+mi}{1}\PY{p}{,} \PY{l+m+mf}{0.2}\PY{p}{,} \PY{l+m+mf}{0.2}\PY{p}{,} \PY{l+m+mf}{0.2}\PY{p}{,} \PY{l+m+mf}{0.2}\PY{p}{,} \PY{l+m+mf}{0.2}\PY{p}{,} \PY{l+m+mf}{0.2}\PY{p}{]}\PY{p}{,}
%          \PY{p}{[}\PY{l+m+mf}{0.2}\PY{p}{,} \PY{l+m+mf}{0.2}\PY{p}{,} \PY{l+m+mf}{0.2}\PY{p}{,} \PY{l+m+mi}{1}\PY{p}{,} \PY{l+m+mf}{0.2}\PY{p}{,} \PY{l+m+mf}{0.2}\PY{p}{,} \PY{l+m+mf}{0.2}\PY{p}{,} \PY{l+m+mf}{0.2}\PY{p}{,} \PY{l+m+mf}{0.2}\PY{p}{]}\PY{p}{,}
%          \PY{p}{[}\PY{l+m+mf}{0.2}\PY{p}{,} \PY{l+m+mf}{0.2}\PY{p}{,} \PY{l+m+mf}{0.2}\PY{p}{,} \PY{l+m+mf}{0.2}\PY{p}{,} \PY{l+m+mi}{1}\PY{p}{,} \PY{l+m+mf}{0.2}\PY{p}{,} \PY{l+m+mf}{0.2}\PY{p}{,} \PY{l+m+mf}{0.2}\PY{p}{,} \PY{l+m+mf}{0.2}\PY{p}{]}\PY{p}{,}
%          \PY{p}{[}\PY{l+m+mf}{0.2}\PY{p}{,} \PY{l+m+mf}{0.2}\PY{p}{,} \PY{l+m+mf}{0.2}\PY{p}{,} \PY{l+m+mf}{0.2}\PY{p}{,} \PY{l+m+mf}{0.2}\PY{p}{,} \PY{l+m+mi}{1}\PY{p}{,} \PY{l+m+mf}{0.2}\PY{p}{,} \PY{l+m+mf}{0.2}\PY{p}{,} \PY{l+m+mf}{0.2}\PY{p}{]}\PY{p}{,}
%          \PY{p}{[}\PY{l+m+mf}{0.2}\PY{p}{,} \PY{l+m+mf}{0.2}\PY{p}{,} \PY{l+m+mf}{0.2}\PY{p}{,} \PY{l+m+mf}{0.2}\PY{p}{,} \PY{l+m+mf}{0.2}\PY{p}{,} \PY{l+m+mf}{0.2}\PY{p}{,} \PY{l+m+mi}{1}\PY{p}{,} \PY{l+m+mf}{0.2}\PY{p}{,} \PY{l+m+mf}{0.2}\PY{p}{]}\PY{p}{,}
%          \PY{p}{[}\PY{l+m+mf}{0.2}\PY{p}{,} \PY{l+m+mf}{0.2}\PY{p}{,} \PY{l+m+mf}{0.2}\PY{p}{,} \PY{l+m+mf}{0.2}\PY{p}{,} \PY{l+m+mf}{0.2}\PY{p}{,} \PY{l+m+mf}{0.2}\PY{p}{,} \PY{l+m+mf}{0.2}\PY{p}{,} \PY{l+m+mi}{1}\PY{p}{,} \PY{l+m+mf}{0.2}\PY{p}{]}\PY{p}{,}
%          \PY{p}{[}\PY{l+m+mf}{0.2}\PY{p}{,} \PY{l+m+mf}{0.2}\PY{p}{,} \PY{l+m+mf}{0.2}\PY{p}{,} \PY{l+m+mf}{0.2}\PY{p}{,} \PY{l+m+mf}{0.2}\PY{p}{,} \PY{l+m+mf}{0.2}\PY{p}{,} \PY{l+m+mf}{0.2}\PY{p}{,} \PY{l+m+mf}{0.2}\PY{p}{,} \PY{l+m+mi}{1}\PY{p}{]}\PY{p}{]}\PY{p}{)}
%		
%\PY{n}{trials} \PY{o}{=} \PY{l+m+mi}{1000000}
%\PY{n}{x\PYZus{}prob} \PY{o}{=} \PY{n}{mvnorm}\PY{o}{.}\PY{n}{rvs}\PY{p}{(}\PY{n}{size}\PY{o}{=}\PY{n}{trials}\PY{p}{)}
%		
%\PY{n}{dp} \PY{o}{=} \PY{p}{[}\PY{p}{]}
%\PY{k}{for} \PY{n}{x} \PY{o+ow}{in} \PY{n}{x\PYZus{}prob}\PY{p}{:}
%    \PY{n}{p} \PY{o}{=} \PY{l+m+mi}{0}
%    \PY{k}{for} \PY{n}{j} \PY{o+ow}{in} \PY{n+nb}{range}\PY{p}{(}\PY{n+nb}{len}\PY{p}{(}\PY{n}{portfolio}\PY{p}{)}\PY{p}{)}\PY{p}{:}
%        \PY{n}{ip} \PY{o}{=} \PY{l+m+mi}{0}
%        \PY{k}{while} \PY{n}{x}\PY{p}{[}\PY{n}{j}\PY{p}{]} \PY{o}{\PYZgt{}} \PY{n}{table\PYZus{}gauss}\PY{p}{[}\PY{n}{portfolio}\PY{p}{[}\PY{n}{j}\PY{p}{]}\PY{p}{,} \PY{n}{ip}\PY{p}{]}\PY{p}{:}
%            \PY{n}{ip} \PY{o}{+}\PY{o}{=} \PY{l+m+mi}{1}
%            \PY{k}{if} \PY{n}{ip} \PY{o}{==} \PY{l+m+mi}{7}\PY{p}{:}
%                \PY{n}{p} \PY{o}{+}\PY{o}{=} \PY{n}{N}\PY{p}{[}\PY{n}{portfolio}\PY{p}{[}\PY{n}{j}\PY{p}{]}\PY{p}{]}\PY{o}{*}\PY{p}{(}\PY{l+m+mi}{1}\PY{o}{\PYZhy{}}\PY{n}{R}\PY{p}{)}
%            \PY{k}{else}\PY{p}{:}
%                \PY{n}{p} \PY{o}{+}\PY{o}{=} \PY{n}{N}\PY{p}{[}\PY{n}{ip}\PY{p}{]}
%	
%        \PY{n}{r} \PY{o}{=} \PY{n+nb}{max}\PY{p}{(}\PY{l+m+mi}{0}\PY{p}{,} \PY{o}{\PYZhy{}}\PY{p}{(}\PY{n}{p} \PY{o}{\PYZhy{}} \PY{n}{p0}\PY{p}{)}\PY{p}{)}
%        \PY{k}{if} \PY{n}{r} \PY{o}{!=} \PY{l+m+mi}{0}\PY{p}{:}
%            \PY{n}{dp}\PY{o}{.}\PY{n}{append}\PY{p}{(}\PY{n}{r}\PY{p}{)}
%		
%\PY{n}{crvar} \PY{o}{=} \PY{n}{numpy}\PY{o}{.}\PY{n}{percentile}\PY{p}{(}\PY{n}{dp}\PY{p}{,} \PY{p}{[}\PY{l+m+mf}{99.9}\PY{p}{]}\PY{p}{)}
%\PY{n+nb}{print} \PY{p}{(}\PY{n}{crvar}\PY{p}{)}
%
%[124.]
%\end{Verbatim}
%\end{codebox}
%
%\begin{figure}[htb]
%	\centering
%	\includegraphics[width=0.7\textwidth]{figures/credit_metrics.png}
%	\caption{Distribution of the losses estimated from simulation of credit rating changes within 1 year. The red line shows the 99.9\% Cr-VaR.}
%\end{figure}

\section{Credit Valuation Adjustment}
\label{credit-valuation-adjustment}

Suppose you have a portfolio of derivatives with a counter-party. 
If the counter-party defaults and the present value of the portfolio at
default is positive to the surviving party, then it only
gets a recovery fraction of the portfolio value from the defaulted entity. 
If however the present value is negative to the surviving party,
it has to pay it in full to the liquidators of the defaulted entity.

This creates and asymmetry such that the deal value under counter-party risk is
the value without counter-party risk minus a positive adjustment, called
\emph{Credit Valuation Adjustment} (CVA).

It can be expressed in the following way:

\begin{equation}
\text{CVA} = (1-R) \int_0^T D(t) \cdot \textrm{EE}(t) dQ(t)
\label{eq:cva}
\end{equation}
where $T$ is the latest maturity in the portfolio, $D$ is the discount factor, EE is the expected exposure or \(\mathbb{E}[\text{max(0, NPV}_\text{portfolio})]\), and $dQ$ is the probability of default between $t$ and $t+dt$.

For an easier computation it is natural to discretize the above integral
and use a time grid going from 0 to the portfolio maturity:

\begin{equation}
\text{CVA} = (1-R) \sum_i^n D(t_i) \cdot \mathrm{EE}(t_i) Q(t_{i-1}, t_i)
\label{eq:cva_discrete}
\end{equation}

It is important to note that while Credit VaR measures the risk of losses faced due to the possible default of some counter-party, CVA measures the pricing component of this risk, i.e. the price adjustment of a product due to this risk.

\subsection{Debit Valuation Adjustment}

The adjustment seen from the point of view of our counter-party is positive, and is called Debit Valuation Adjustment, DVA. It is positive because the early default of the client itself would imply a discount on the client payment obligations, and this means a gain. So the client marks a positive adjustment over the risk free price by adding the positive amount called DVA. 

When both parties have the possibility to default, they
consistently include both defaults into the valuation. Hence
every party needs to include its own default besides the default of the
counter-party into the valuation. So they will mark a positive
CVA to be subtracted and a positive DVA to be added to the default
risk free price of the deal. The CVA of one party will be the DVA of
the other one and vice versa.

\[
\textrm{price}=\textrm{default risk free price + DVA - CVA}
\]
Now, since
\[
\textrm{default risk free price(A)} = - \textrm{default risk free price(A)}
\]
\[
\textrm{DVA(A)} = \textrm{CVA(B)}
\]
\[
\textrm{DVA(B)} = \textrm{CVA(A)}
\]
we get that eventually
\[
\textrm{price(A)} = -\textrm{price(B)}
\]
so that both parties agree on the price, or, we could say, there is money
conservation.

\subsection{CVA Computation}

The computation of the CVA is easily carried on with Monte Carlo simulation.
First simulate the development of your portfolio (its NPV) at each time point for each MC scenario. 
Then calculate the CVA using Eq.~\ref{eq:cva} or its discrete form in Eq.~\ref{eq:cva_discrete}.
Finally average the CVA of all the scenarios to get its estimate.

In case of zero coupon bonds the computation of the CVA can be simplified. Indeed in this case the exposure of the investor is equal to the face value of the bond, so it is enough to loop through each days from the observation date to the bond maturity and compute the CVA using Eq.~\ref{eq:cva_discrete}.

Imagine a 3-years zero coupon bond with a face value of $FV=$~\euro{100}. 
The bond issuer has the following default probabilities 10\%, 20\% and 30\% for 1, 2 and 3 years respectively and the recovery rate is 40\%. The risk free rate is instead 3\%. 

To compute the CVA we need to first define a discount curve and the credit curve corresponding to the issuer. Then we perform a daily loop to sum up all the contributions to the CVA and finally we set the price of the bond to the default-risk-free price minus the CVA.

\begin{codebox}
\begin{Verbatim}[commandchars=\\\{\}]
\PY{k+kn}{from} \PY{n+nn}{dateutil}\PY{n+nn}{.}\PY{n+nn}{relativedelta} \PY{k}{import} \PY{n}{relativedelta}
\PY{k+kn}{from} \PY{n+nn}{finmarkets} \PY{k}{import} \PY{n}{DiscountCurve}\PY{p}{,} \PY{n}{CreditCurve}
\PY{k+kn}{import} \PY{n+nn}{math}
		
\PY{n}{T} \PY{o}{=} \PY{l+m+mi}{3} 
\PY{n}{r} \PY{o}{=} \PY{l+m+mf}{0.03}
\PY{n}{R} \PY{o}{=} \PY{l+m+mf}{0.4}
\PY{n}{FV} \PY{o}{=} \PY{l+m+mi}{100}
		
\PY{n}{obs\PYZus{}date} \PY{o}{=} \PY{n}{date}\PY{o}{.}\PY{n}{today}\PY{p}{(}\PY{p}{)}
\PY{n}{pillars} \PY{o}{=} \PY{p}{[}\PY{n}{obs\PYZus{}date}\PY{o}{+}\PY{n}{relativedelta}\PY{p}{(}\PY{n}{years}\PY{o}{=}\PY{n}{i}\PY{p}{)} \PY{k}{for} \PY{n}{i} \PY{o+ow}{in} \PY{n+nb}{range}\PY{p}{(}\PY{n}{T}\PY{o}{+}\PY{l+m+mi}{1}\PY{p}{)}\PY{p}{]}
\PY{n}{dfs} \PY{o}{=} \PY{p}{[}\PY{n}{math}\PY{o}{.}\PY{n}{exp}\PY{p}{(}\PY{o}{\PYZhy{}}\PY{n}{r}\PY{o}{*}\PY{n}{i}\PY{p}{)} \PY{k}{for} \PY{n}{i} \PY{o+ow}{in} \PY{n+nb}{range}\PY{p}{(}\PY{n}{T}\PY{o}{+}\PY{l+m+mi}{1}\PY{p}{)}\PY{p}{]}
\PY{n}{dc} \PY{o}{=} \PY{n}{DiscountCurve}\PY{p}{(}\PY{n}{obs\PYZus{}date}\PY{p}{,} \PY{n}{pillars}\PY{p}{,} \PY{n}{dfs}\PY{p}{)}
\PY{n}{S} \PY{o}{=} \PY{p}{[}\PY{l+m+mi}{1}\PY{p}{,} \PY{l+m+mf}{0.9}\PY{p}{,} \PY{l+m+mf}{0.8}\PY{p}{,} \PY{l+m+mf}{0.7}\PY{p}{]}
\PY{n}{cc} \PY{o}{=} \PY{n}{CreditCurve}\PY{p}{(}\PY{n}{pillars}\PY{p}{,} \PY{n}{S}\PY{p}{)}
\PY{n}{PV} \PY{o}{=} \PY{n}{FV} \PY{o}{*} \PY{n}{math}\PY{o}{.}\PY{n}{exp}\PY{p}{(}\PY{o}{\PYZhy{}}\PY{n}{r}\PY{o}{*}\PY{n}{i}\PY{p}{)}
		
\PY{n}{cva} \PY{o}{=} \PY{l+m+mi}{0}
\PY{n}{d} \PY{o}{=} \PY{n}{obs\PYZus{}date}
\PY{k}{while} \PY{n}{d} \PY{o}{\PYZlt{}}\PY{o}{=} \PY{n}{pillars}\PY{p}{[}\PY{o}{\PYZhy{}}\PY{l+m+mi}{1}\PY{p}{]}\PY{p}{:}
    \PY{n}{cva} \PY{o}{+}\PY{o}{=} \PY{n}{dc}\PY{o}{.}\PY{n}{df}\PY{p}{(}\PY{n}{d}\PY{p}{)}\PY{o}{*}\PY{p}{(}\PY{n}{cc}\PY{o}{.}\PY{n}{ndp}\PY{p}{(}\PY{n}{d}\PY{p}{)} \PY{o}{\PYZhy{}} \PY{n}{cc}\PY{o}{.}\PY{n}{ndp}\PY{p}{(}\PY{n}{d}\PY{o}{+}\PY{n}{relativedelta}\PY{p}{(}\PY{n}{days}\PY{o}{=}\PY{l+m+mi}{1}\PY{p}{)}\PY{p}{)}\PY{p}{)}
    \PY{n}{d} \PY{o}{+}\PY{o}{=} \PY{n}{relativedelta}\PY{p}{(}\PY{n}{days}\PY{o}{=}\PY{l+m+mi}{1}\PY{p}{)}
		
\PY{n}{cva} \PY{o}{*}\PY{o}{=} \PY{p}{(}\PY{l+m+mi}{1}\PY{o}{\PYZhy{}}\PY{n}{R}\PY{p}{)} \PY{o}{*} \PY{n}{FV}
\PY{n+nb}{print} \PY{p}{(}\PY{l+s+s2}{\PYZdq{}}\PY{l+s+s2}{CVA: }\PY{l+s+si}{\PYZob{}:.2f\PYZcb{}}\PY{l+s+s2}{\PYZdq{}}\PY{o}{.}\PY{n}{format}\PY{p}{(}\PY{n}{cva}\PY{p}{)}\PY{p}{)}
\PY{n+nb}{print} \PY{p}{(}\PY{l+s+s2}{\PYZdq{}}\PY{l+s+s2}{Bond Price: }\PY{l+s+si}{\PYZob{}:.2f\PYZcb{}}\PY{l+s+s2}{\PYZdq{}}\PY{o}{.}\PY{n}{format}\PY{p}{(}\PY{n}{PV} \PY{o}{\PYZhy{}} \PY{n}{cva}\PY{p}{)}\PY{p}{)}

CVA: 17.21
Bond Price: 74.18
\end{Verbatim}
\end{codebox}

\begin{thebibliography}{9}
\bibitem{bib:var} J. C. Hull, \emph{Options, Futures and Other Derivatives, 7th Ed.}, Value at Risk (Ch. 20), Pearson Prentice Hall, 2009
\bibitem{bib:credit_var} J. C. Hull, \emph{Options, Futures and Other Derivatives, 7th Ed.}, Credit Risk (Ch. 22), Pearson Prentice Hall, 2009
\bibitem{bib:creditmetrics}RiskMetrics Group, \href{https://www.msci.com/documents/10199/93396227-d449-4229-9143-24a94dab122f}{\emph{CreditMetrics}}, J.P. Morgan \& Co., 2007, [Online]
\bibitem{bib:cva} D. Brigo, \emph{Counterparty Risk FAQ: Credit VaR, PFE, CVA, DVA, Closeout, Netting, Collateral, Re-hypothecation, WWR, Basel, Funding, CCDS and Margin Lending}, arXiv: 1111.1331, 2011
\end{thebibliography}
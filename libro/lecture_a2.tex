\chapter{Matrices}\label{app:matrices}

In mathematics, a matrix is a rectangular array of
numbers, symbols, or expressions, arranged in rows and columns. Matrices
are commonly written in box brackets. The horizontal and vertical lines
of entries in a matrix are called rows and columns, respectively. The
size of a matrix is defined by the number of rows and columns that it
contains. A matrix with \(m\) rows and \(n\) columns is called an
\(m\times n\) matrix or \(m\)-by-\(n\) matrix, while \(m\) and \(n\) are
called its dimensions. The dimensions of the following matrix are
\(2\times 3\) (read ``two by three''), because there are two rows and
three columns.

\[\begin{bmatrix}
1 & 2 & 3\\
4 & 5 & 6
\end{bmatrix}\]

The individual items (numbers, symbols or expressions) in a matrix are
called its elements or entries.

In \texttt{python} matrices can be represented as \texttt{numpy.array},
so the example above will become:

\begin{tcolorbox}[breakable, size=fbox, boxrule=1pt, pad at break*=1mm,colback=cellbackground, colframe=cellborder]
\begin{Verbatim}[commandchars=\\\{\}]
\PY{k+kn}{import} \PY{n+nn}{numpy} \PY{k}{as} \PY{n+nn}{np}

\PY{n+nb}{print} \PY{p}{(}\PY{n}{np}\PY{o}{.}\PY{n}{array}\PY{p}{(}\PY{p}{[}\PY{p}{[}\PY{l+m+mi}{1}\PY{p}{,}\PY{l+m+mi}{2}\PY{p}{,}\PY{l+m+mi}{3}\PY{p}{]}\PY{p}{,}\PY{p}{[}\PY{l+m+mi}{4}\PY{p}{,}\PY{l+m+mi}{5}\PY{p}{,}\PY{l+m+mi}{6}\PY{p}{]}\PY{p}{]}\PY{p}{)}\PY{p}{)}

[[1 2 3]
 [4 5 6]]
\end{Verbatim}
\end{tcolorbox}

    Essentially a \texttt{numpy.array} is a list of lists each one
representing a matrix row. Arrays can have any dimension so they can be
used to represent also vectors in \texttt{python}. There are two special
types of arrays \texttt{zeros} and \texttt{ones} whose name already
clarify their meaning:

    \begin{tcolorbox}[breakable, size=fbox, boxrule=1pt, pad at break*=1mm,colback=cellbackground, colframe=cellborder]
\begin{Verbatim}[commandchars=\\\{\}]
\PY{n}{A} \PY{o}{=} \PY{n}{np}\PY{o}{.}\PY{n}{zeros}\PY{p}{(}\PY{n}{shape}\PY{o}{=}\PY{p}{(}\PY{l+m+mi}{3}\PY{p}{,} \PY{l+m+mi}{3}\PY{p}{)}\PY{p}{)}
\PY{n}{B} \PY{o}{=} \PY{n}{np}\PY{o}{.}\PY{n}{ones}\PY{p}{(}\PY{n}{shape}\PY{o}{=}\PY{p}{(}\PY{l+m+mi}{4}\PY{p}{,} \PY{l+m+mi}{4}\PY{p}{)}\PY{p}{)}

\PY{n+nb}{print} \PY{p}{(}\PY{n}{A}\PY{p}{)}
\PY{n+nb}{print} \PY{p}{(}\PY{n}{B}\PY{p}{)}

[[0. 0. 0.]
 [0. 0. 0.]
 [0. 0. 0.]]

[[1. 1. 1. 1.]
 [1. 1. 1. 1.]
 [1. 1. 1. 1.]
 [1. 1. 1. 1.]]
    \end{Verbatim}
\end{tcolorbox}

\section{Transpose of a Matrix}
The transpose of a matrix $[A]$, denoted by $[A^{T}]$, may be constructed by writing the rows of $[A]$ as the columns of $[A^T]$
and the columns of $[A]$ as the rows of $[A^T]$.
Formally, the $i$-th row, $j$-th column element of $[A^T]$ is the $j$-th row, $i$-th column element of $[A]$:

\[[A^T]_{ij} = [A]_{ji}\]

If $[A]$ is an $m\times n$ matrix, then $[A^T]$ is an $n\times m$ matrix. 
As an example the transpose of:

\[
\begin{bmatrix}
1 & 5 & 3 \\
2 & -3 & 8
\end{bmatrix}
\quad \mathrm{is} \quad
\begin{bmatrix}
1 & 2 \\
5 & -3 \\
3  & 8
\end{bmatrix}
\]

\section{Operation with Matrices}
\subsection{Adding and Subtracting Matrices}\label{adding-and-subtracting-matrices}

We use matrices to list data or to represent systems. Because the
entries are numbers, we can perform operations on matrices. We add or
subtract matrices by adding or subtracting corresponding entries.

In order to do this, the entries must correspond. Therefore, addition
and subtraction of matrices is only possible when the matrices have the
same dimensions.\\
Adding matrices is very simple. Just add each element in the first
matrix to the corresponding element in the second matrix.

\[
\begin{bmatrix}
1 & 2 & 3 \\
4 & 5 & 6
\end{bmatrix}
+
\begin{bmatrix}
10 & 20 & 30\\
40 & 50 & 60
\end{bmatrix}
=
\begin{bmatrix}
11 & 22 & 33\\
44 & 55 & 66
\end{bmatrix}
\]

As you might guess, subtracting works much the same way except that you
subtract instead of adding.

\[
\begin{bmatrix}
10 & 20 & 30\\
40 & 50 & 60
\end{bmatrix}
-
\begin{bmatrix}
1 & 2 & 3 \\
4 & 5 & 6
\end{bmatrix}
=
\begin{bmatrix}
9 & 18 & 27\\
36 & 45 & 54
\end{bmatrix}
\]

Adding and subtracting \texttt{numpy.array} is as easy as that:

    \begin{tcolorbox}[breakable, size=fbox, boxrule=1pt, pad at break*=1mm,colback=cellbackground, colframe=cellborder]
\begin{Verbatim}[commandchars=\\\{\}]
\PY{n}{A} \PY{o}{=} \PY{n}{np}\PY{o}{.}\PY{n}{array}\PY{p}{(}\PY{p}{[}\PY{p}{[}\PY{l+m+mi}{1}\PY{p}{,}\PY{l+m+mi}{2}\PY{p}{,}\PY{l+m+mi}{3}\PY{p}{]}\PY{p}{,}\PY{p}{[}\PY{l+m+mi}{4}\PY{p}{,}\PY{l+m+mi}{5}\PY{p}{,}\PY{l+m+mi}{6}\PY{p}{]}\PY{p}{]}\PY{p}{)}
\PY{n}{B} \PY{o}{=} \PY{n}{np}\PY{o}{.}\PY{n}{array}\PY{p}{(}\PY{p}{[}\PY{p}{[}\PY{l+m+mi}{10}\PY{p}{,}\PY{l+m+mi}{20}\PY{p}{,}\PY{l+m+mi}{30}\PY{p}{]}\PY{p}{,}\PY{p}{[}\PY{l+m+mi}{40}\PY{p}{,}\PY{l+m+mi}{50}\PY{p}{,}\PY{l+m+mi}{60}\PY{p}{]}\PY{p}{]}\PY{p}{)}

\PY{n+nb}{print} \PY{p}{(}\PY{n}{A}\PY{o}{+}\PY{n}{B}\PY{p}{)}
\PY{n+nb}{print} \PY{p}{(}\PY{n}{B}\PY{o}{\PYZhy{}}\PY{n}{A}\PY{p}{)}

[[11 22 33]
 [44 55 66]]
 
[[ 9 18 27]
 [36 45 54]]
    \end{Verbatim}
\end{tcolorbox}

\subsection{Scalar Multiplication}\label{scalar-multiplication}

Multiplying a matrix by a scalar \(c\) means you add the matrix to
itself \(c\) times, or simply multiply each element by that constant.

\[
3 \cdot
\begin{bmatrix}
1 & 2 & 3 \\
4 & 5 & 6
\end{bmatrix}
=
\begin{bmatrix}
3 & 6 & 9 \\
12 & 15 & 18
\end{bmatrix}
\]

Scalar multiplication of \texttt{numpy.array} is:

    \begin{tcolorbox}[breakable, size=fbox, boxrule=1pt, pad at break*=1mm,colback=cellbackground, colframe=cellborder]
\begin{Verbatim}[commandchars=\\\{\}]
\PY{n}{c} \PY{o}{=} \PY{l+m+mi}{3}
\PY{n}{A} \PY{o}{=} \PY{n}{np}\PY{o}{.}\PY{n}{array}\PY{p}{(}\PY{p}{[}\PY{p}{[}\PY{l+m+mi}{1}\PY{p}{,}\PY{l+m+mi}{2}\PY{p}{,}\PY{l+m+mi}{3}\PY{p}{]}\PY{p}{,}\PY{p}{[}\PY{l+m+mi}{4}\PY{p}{,}\PY{l+m+mi}{5}\PY{p}{,}\PY{l+m+mi}{6}\PY{p}{]}\PY{p}{]}\PY{p}{)}

\PY{n+nb}{print} \PY{p}{(}\PY{n}{c}\PY{o}{*}\PY{n}{A}\PY{p}{)}

[[ 3  6  9]
 [12 15 18]]
    \end{Verbatim}
\end{tcolorbox}

\subsection{Matrix Multiplication}\label{matrix-multiplication}

Matrix multiplication is multiplying every element of each row of the
first matrix times every element of each column in the second matrix.
When multiplying matrices, the elements of the rows in the first matrix
are multiplied with corresponding columns in the second matrix. Each
entry of the resultant matrix is computed one at a time.

Let's see with an example: \[ 
\begin{bmatrix}
1 & 2 \\
3 & 4
\end{bmatrix}
\cdot
\begin{bmatrix}
5 & 6 \\
7 & 8
\end{bmatrix}
= ?
\]

First ask: do the number of columns in the first matrix equal the number
of rows in the second ? If so the product exists. Then start with
producing the product for the first row, first column element. Take the
first row of the first matrix and multiply by the first column of the
second like this:

\[ 
\begin{bmatrix}
1 & 2 \\
3 & 4
\end{bmatrix}
\cdot
\begin{bmatrix}
5 & 6 \\
7 & 8
\end{bmatrix}
=
\begin{bmatrix}
(1\cdot 5) + (2\cdot 7) & X \\
X & X
\end{bmatrix}
\]

Continue the pattern with the first row of the first matrix with the
second column of the second matrix:

\[ 
\begin{bmatrix}
1 & 2 \\
3 & 4
\end{bmatrix}
\cdot
\begin{bmatrix}
5 & 6 \\
7 & 8
\end{bmatrix}
=
\begin{bmatrix}
(1\cdot 5) + (2\cdot 7) & (1\cdot 6) + (2\cdot 8)  \\
X & X
\end{bmatrix}
\]

The do the same with the second row of the first matrix and you are
done:

\[ 
\begin{bmatrix}
1 & 2 \\
3 & 4
\end{bmatrix}
\cdot
\begin{bmatrix}
5 & 6 \\
7 & 8
\end{bmatrix}
=
\begin{bmatrix}
(1\cdot 5) + (2\cdot 7) & (1\cdot 6) + (2\cdot 8)  \\
(3\cdot 5) + (4\cdot 7) & (3\cdot 6) + (4\cdot 8) 
\end{bmatrix}
=
\begin{bmatrix}
19 & 22 \\
43 & 50 
\end{bmatrix}
\]

In \texttt{numpy} array multiplication can be done like this:

    \begin{tcolorbox}[breakable, size=fbox, boxrule=1pt, pad at break*=1mm,colback=cellbackground, colframe=cellborder]
\begin{Verbatim}[commandchars=\\\{\}]
\PY{n}{A} \PY{o}{=} \PY{n}{np}\PY{o}{.}\PY{n}{array}\PY{p}{(}\PY{p}{[}\PY{p}{[}\PY{l+m+mi}{1}\PY{p}{,}\PY{l+m+mi}{2}\PY{p}{]}\PY{p}{,}\PY{p}{[}\PY{l+m+mi}{3}\PY{p}{,}\PY{l+m+mi}{4}\PY{p}{]}\PY{p}{]}\PY{p}{)}
\PY{n}{B} \PY{o}{=} \PY{n}{np}\PY{o}{.}\PY{n}{array}\PY{p}{(}\PY{p}{[}\PY{p}{[}\PY{l+m+mi}{5}\PY{p}{,}\PY{l+m+mi}{6}\PY{p}{]}\PY{p}{,}\PY{p}{[}\PY{l+m+mi}{7}\PY{p}{,}\PY{l+m+mi}{8}\PY{p}{]}\PY{p}{]}\PY{p}{)}

\PY{n+nb}{print} \PY{p}{(}\PY{n}{np}\PY{o}{.}\PY{n}{dot}\PY{p}{(}\PY{n}{A}\PY{p}{,} \PY{n}{B}\PY{p}{)}\PY{p}{)}

[[19 22]
 [43 50]]
    \end{Verbatim}
\end{tcolorbox}

\section{The identity matrix}\label{the-identity-matrix}

\([I]\) is defined so that \([A][I]=[A]\), i.e.~it is the matrix version
of multiplying a number by one. What matrix has this property? A first
guess might be a matrix full of 1s, but that does not work:

\[
\begin{bmatrix}
1 & 2 \\
3 & 4
\end{bmatrix}
\begin{bmatrix}
1 & 1 \\
1 & 1
\end{bmatrix}
=
\begin{bmatrix}
3 & 3 \\
7 & 7
\end{bmatrix}
\]

So our initial guess was wrong ! The matrix that does work is a diagonal
stretch of 1s, with all other elements being 0:

\[
\begin{bmatrix}
1 & 2 \\
3 & 4
\end{bmatrix}
\begin{bmatrix}
1 & 0 \\
0 & 1
\end{bmatrix}
=
\begin{bmatrix}
1 & 2 \\
3 & 4
\end{bmatrix}
\]

So \([I] = 
\begin{bmatrix}
1 & 0 \\
0 & 1
\end{bmatrix}
\) is the identity matrix for \(2\times 2\) matrices.

The \texttt{numpy} equivalent of the identity matrix is given by
\texttt{numpy.identity(n)} with n the dimension of the matrix. So for
example:

    \begin{tcolorbox}[breakable, size=fbox, boxrule=1pt, pad at break*=1mm,colback=cellbackground, colframe=cellborder]
\begin{Verbatim}[commandchars=\\\{\}]
\PY{n}{A} \PY{o}{=} \PY{n}{np}\PY{o}{.}\PY{n}{array}\PY{p}{(}\PY{p}{[}\PY{p}{[}\PY{l+m+mi}{1}\PY{p}{,}\PY{l+m+mi}{2}\PY{p}{]}\PY{p}{,}\PY{p}{[}\PY{l+m+mi}{3}\PY{p}{,}\PY{l+m+mi}{4}\PY{p}{]}\PY{p}{]}\PY{p}{)}
\PY{n}{I} \PY{o}{=} \PY{n}{np}\PY{o}{.}\PY{n}{identity}\PY{p}{(}\PY{l+m+mi}{2}\PY{p}{)}

\PY{n+nb}{print} \PY{p}{(}\PY{n}{A}\PY{p}{)}
\PY{n+nb}{print} \PY{p}{(}\PY{p}{)}
\PY{n+nb}{print} \PY{p}{(}\PY{n}{np}\PY{o}{.}\PY{n}{dot}\PY{p}{(}\PY{n}{A}\PY{p}{,} \PY{n}{I}\PY{p}{)}\PY{p}{)}

[[1 2]
 [3 4]]

[[1. 2.]
 [3. 4.]]
    \end{Verbatim}
\end{tcolorbox}

\section{Matrix Equations}\label{matrix-equations}

Matrices can be used to compactly write and work with systems of
multiple linear equations. As we have learned in previous sections,
matrices can be manipulated in any way that a normal equation can be.
This is very helpful when we start to work with systems of equations. It
is helpful to understand how to organize matrices to solve these
systems.

\subsection{Writing a System of Equations with Matrices}\label{writing-a-system-of-equations-with-matrices}

It is possible to solve this system using the elimination or
substitution method, but it is also possible to do it with a matrix
operation. Before we start setting up the matrices, it is important to
do the following:

\begin{itemize}
\tightlist
\item
  make sure that all of the equations are written in a similar manner,
  meaning the variables need to all be in the same order;
\item
  make sure that one side of the equation is only variables and their
  coefficients, and the other side is just constants;
\end{itemize}

Using matrix multiplication, we may define a
system of equations with the same number of equations as variables as:

\[ [A][X] = [B]\]
where \([A]\) is the coefficient matrix, \([X]\) is the variable matrix, and
\([B]\) is the constant matrix. Given the system:

\[
\begin{cases}
x + 8y = 7 \\
2x -8y = -3
\end{cases}
\]

The corresponding matrices are then:

\[[A]=
\begin{bmatrix}
1 & 8\\
2 & -8
\end{bmatrix}
;\quad
[X]=
\begin{bmatrix}
x\\
y
\end{bmatrix}
;\quad
[B]=
\begin{bmatrix}
7\\
-3
\end{bmatrix}
\]

Thus to solve a system $[A][X]=[B]$, for $[X]$, multiply both sides byt the inverse of $[A]$ and 
we shall obtain the solution:

\[[A^{-1}][A][X]=[A^{-1}][B] \implies [X] = [A^{-1}][B] \]

Provided the inverse \([A^{-1}]\) exists, this formula will solve the
system. If the coefficient matrix is not invertible, the system could be
inconsistent and have no solution, or be dependent and have infinitely
many solutions.

 \section{The Inverse of a Matrix}\label{the-inverse-of-a-matrix}

The inverse of matrix \([A]\) is \([A^{-1}]\), and is defined by the
property:

\[ [A][A^{-1}]=[I] \]

Hence the matrix \([B]\) is the inverse of the matrix \([A]\) if when
multiplied together, \([A][B]\) gives the identity matrix. 

Using the definition let's try to find the inverse of:

\[
\begin{bmatrix}
3 & 4\\
5 & 6
\end{bmatrix}
\]

First, let the following be true:

\[
\begin{bmatrix}
3 & 4\\
5 & 6
\end{bmatrix}
\begin{bmatrix}
a & b\\
c & d
\end{bmatrix}
=
\begin{bmatrix}
1 & 0\\
0 & 1
\end{bmatrix}
\]

When multiplying this mystery matrix by our original matrix, the result is

\[
\begin{bmatrix}
3a+4c & 3b+4d\\
5a+6c & 5b+6d
\end{bmatrix}
=
\begin{bmatrix}
1 & 0\\
0 & 1
\end{bmatrix}
\]

For two matrices to be equal, every element in the left must equal its
corresponding element on the right. So, for these two matrices to equal
each other:

\[
\begin{cases}
3a+4c=1\\
3b+4d=0\\
5a+6c=0\\
5b+6d=1
\end{cases}
\]

Solving this simple system we get the following result:

\[
\begin{cases}
a=−3\\
b=2\\
c=2.5\\
d=−1.5
\end{cases}
\]

Having solved for the four variables, the result is the inverse

\[
\begin{bmatrix}
−3 & 2\\
2.5 & −1.5
\end{bmatrix}
\]

The quick check to be sure the result is correct is done in \texttt{python}.
The \texttt{linalg.inv()} function can be used to find the inverse of a
matrix:

    \begin{tcolorbox}[breakable, size=fbox, boxrule=1pt, pad at break*=1mm,colback=cellbackground, colframe=cellborder]
\begin{Verbatim}[commandchars=\\\{\}]
\PY{k+kn}{from} \PY{n+nn}{numpy}\PY{n+nn}{.}\PY{n+nn}{linalg} \PY{k}{import} \PY{n}{inv}

\PY{n}{A} \PY{o}{=} \PY{n}{np}\PY{o}{.}\PY{n}{array}\PY{p}{(}\PY{p}{[}\PY{p}{[}\PY{l+m+mi}{3}\PY{p}{,}\PY{l+m+mi}{4}\PY{p}{]}\PY{p}{,}\PY{p}{[}\PY{l+m+mi}{5}\PY{p}{,}\PY{l+m+mi}{6}\PY{p}{]}\PY{p}{]}\PY{p}{)}
\PY{n+nb}{print} \PY{p}{(}\PY{n}{inv}\PY{p}{(}\PY{n}{A}\PY{p}{)}\PY{p}{)}

[[-3.   2. ]
 [ 2.5 -1.5]]
    \end{Verbatim}
\end{tcolorbox}

\section{Solving Systems of Equations Using Matrix Inverses}\label{solving-systems-of-equations-using-matrix-inverses}

A system of equations can be readily solved using the concepts of the
inverse matrix and matrix multiplication.

Solve the following system of linear equations:

\[
\begin{cases}
x+2y-z=11\\
2x-y+3z=7\\
7x-3y-2z=2
\end{cases}
\]

Set up the three necessary matrices:

\[[A]=
\begin{bmatrix}
1 & 2 & -1 \\ 
2 & -1 & 3 \\
7 & -3 & -2
\end{bmatrix}
;\quad
[B]=
\begin{bmatrix}
11\\
7\\
2
\end{bmatrix}
;\quad
[X]=
\begin{bmatrix}
x\\
y \\ 
z
\end{bmatrix}
\]

Since to solve this system we have to find the inverse matrix of \([A]\)
and multiply it to \([B]\) we have all the ingredients to do it in
\texttt{python}:

    \begin{tcolorbox}[breakable, size=fbox, boxrule=1pt, pad at break*=1mm,colback=cellbackground, colframe=cellborder]
\begin{Verbatim}[commandchars=\\\{\}]
\PY{n}{A} \PY{o}{=} \PY{n}{np}\PY{o}{.}\PY{n}{array}\PY{p}{(}\PY{p}{[}\PY{p}{[}\PY{l+m+mi}{1}\PY{p}{,}\PY{l+m+mi}{2}\PY{p}{,}\PY{o}{\PYZhy{}}\PY{l+m+mi}{1}\PY{p}{]}\PY{p}{,}\PY{p}{[}\PY{l+m+mi}{2}\PY{p}{,}\PY{o}{\PYZhy{}}\PY{l+m+mi}{1}\PY{p}{,}\PY{l+m+mi}{3}\PY{p}{]}\PY{p}{,}\PY{p}{[}\PY{l+m+mi}{7}\PY{p}{,}\PY{o}{\PYZhy{}}\PY{l+m+mi}{3}\PY{p}{,}\PY{o}{\PYZhy{}}\PY{l+m+mi}{2}\PY{p}{]}\PY{p}{]}\PY{p}{)}
\PY{n}{B} \PY{o}{=} \PY{n}{np}\PY{o}{.}\PY{n}{array}\PY{p}{(}\PY{p}{[}\PY{l+m+mi}{11}\PY{p}{,}\PY{l+m+mi}{7}\PY{p}{,}\PY{l+m+mi}{2}\PY{p}{]}\PY{p}{)}

\PY{n}{A\PYZus{}inv} \PY{o}{=} \PY{n}{inv}\PY{p}{(}\PY{n}{A}\PY{p}{)}
\PY{n}{sol} \PY{o}{=} \PY{n}{np}\PY{o}{.}\PY{n}{dot}\PY{p}{(}\PY{n}{A\PYZus{}inv}\PY{p}{,} \PY{n}{B}\PY{p}{)}

\PY{n+nb}{print} \PY{p}{(}\PY{n}{sol}\PY{p}{)}

[3. 5. 2.]
    \end{Verbatim}
\end{tcolorbox}

    So the solution of the system is: \[
\begin{cases}
x=3\\
y=5\\
z=2
\end{cases}
\]

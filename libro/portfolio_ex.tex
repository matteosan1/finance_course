\chapter{Portfolio Optimization}
\label{ex-portfolio}

\begin{question}
Your are portfolio manager and one of your clients is asking to design her investment in order to reduce at the minimum the risk. Her position involves these shares: 'TMUS', 'TDS', 'ENTR', 'UEC', 'WM', 'CTRE', 'LDOS', 'COST', 'CHD', 'HI'.
Find the weights corresponding to each asset.

\noindent\textbf{Hint:} the historical series of the corresponding companies are store in \href{https://drive.google.com/file/d/1ARryI7zpNVqlpzPsrNS2rTH9Knj07JBV/view?usp=sharing}{share\_prices.csv}.

\end{question}

\cprotEnv \begin{solution}
\begin{ipython}
import pandas as pd
df = pd.read_csv("prova.csv", index_col='date')
daily_returns = df.pct_change()
returns = daily_returns.mean()*252
covariance = daily_returns.cov()*252

print (returns)
print (covariance)

CHD 	0.149289
COST 	0.165491
CTRE 	0.099866
ENTR 	0.611855
HI 		0.162094
LDOS 	0.226562
TDS 	0.081294
TMUS 	0.313158
UEC 	0.102048
WM 		0.189474
dtype: float64

          CHD     COST      CTRE      ENTR       HI     LDOS       TDS
CHD  0.024915 0.009337  0.008256  0.009681 0.008599 0.007354  0.011440
COST 0.009337 0.028864  0.007549  0.019100 0.010508 0.012134  0.011564
CTRE 0.008256 0.007549  0.098606 -0.037186 0.015380 0.014143  0.015634
ENTR 0.009681 0.019100 -0.037186  0.628810 0.001346 0.005644 -0.008187
HI   0.008599 0.010508  0.015380  0.001346 0.050175 0.018365  0.019070
LDOS 0.007354 0.012134  0.014143  0.005644 0.018365 0.065101  0.018032
TDS  0.011440 0.011564  0.015634 -0.008187 0.019070 0.018032  0.078266
TMUS 0.009991 0.008667  0.012142  0.020659 0.017274 0.015900  0.030212
UEC  0.009134 0.011193  0.002861  0.174809 0.034460 0.025498  0.024209
WM   0.007536 0.008773  0.010322  0.006651 0.011621 0.011010  0.011077

         TMUS      UEC       WM
CHD  0.009991 0.009134 0.007536
COST 0.008667 0.011193 0.008773
CTRE 0.012142 0.002861 0.010322
ENTR 0.020659 0.174809 0.006651
HI   0.017274 0.034460 0.011621
LDOS 0.015900 0.025498 0.011010
TDS  0.030212 0.024209 0.011077
TMUS 0.126895 0.024575 0.011563
UEC  0.024575 0.497870 0.014853
WM   0.011563 0.014853 0.019869

import numpy as np
from scipy.optimize import minimize

def sum_weights(w):
    return np.sum(w) - 1

def markowitz(w, cov):
    return np.sqrt(np.dot(w.T, np.dot(cov, w)))

num_assets = 10
constraints = ({'type': 'eq', 'fun': sum_weights},)
bounds = tuple((0, 1) for asset in range(num_assets))
weights = [1./num_assets for _ in range(num_assets)]
opts = minimize(markowitz, weights, args=(covariance,),
    bounds=bounds, constraints=constraints)

print (opts)

    fun: 0.1137665453329252
    jac: array([0.11396485, 0.11356963, 0.11363222, 0.11417477, 0.11384268,
                0.11409247, 0.11368632, 0.11387072, 0.12300402, 0.11367865])
message: 'Optimization terminated successfully.'
   nfev: 144
    nit: 12
   njev: 12
 status: 0
success: True
      x: array([2.84626544e-01, 1.87698071e-01, 3.77538201e-02, 
                1.90910627e-03, 4.80977699e-02, 4.07549907e-02, 
                8.50047814e-03, 1.50307431e-02, 4.44522891e-18, 
                3.75628477e-01])

print ("Portfolio composition")
for i, n in enumerate(df.columns):
    print ("{:5}: {:4.1f}%".format(n, opts.x[i]*100))
print ("Portfolio variance: {:.4f}".format(opts.fun**2))
print ("Expected Portfolio return: {:.3f}".format(opts.x.dot(returns)))

Portfolio composition
CHD  : 28.5%
COST : 18.8%
CTRE : 3.8%
ENTR : 0.2%
HI   : 4.8%
LDOS : 4.1%
TDS  : 0.9%
TMUS : 1.5%
UEC  : 0.0%
WM   : 37.6%
Portfolio variance: 0.0129
Expected Portfolio return: 0.172
\end{ipython}
\end{solution}

%\begin{question}[title={(Return Allocation Portfolio)}]
%After few months the same client of the previous question, contacted you again asking for larger returns. In particular she would like to increase it by 40\%.
%Find the new portfolio composition that makes your client happy.
%\end{question}
%\begin{solution}
%\begin{tcolorbox}[size=fbox, boxrule=1pt, colback=cellbackground, colframe=cellborder]
%\begin{Verbatim}[commandchars=\\\{\}]
%\PY{k}{def} \PY{n+nf}{efficient\PYZus{}frontier}\PY{p}{(}\PY{n}{w}\PY{p}{,} \PY{n}{asset\PYZus{}returns}\PY{p}{,} \PY{n}{target\PYZus{}return}\PY{p}{)}\PY{p}{:} 
%    \PY{n}{portfolio\PYZus{}return} \PY{o}{=} \PY{n}{np}\PY{o}{.}\PY{n}{sum}\PY{p}{(}\PY{n}{asset\PYZus{}returns} \PY{o}{*} \PY{n}{w}\PY{p}{)} 
%    \PY{k}{return} \PY{p}{(}\PY{n}{portfolio\PYZus{}return} \PY{o}{\PYZhy{}} \PY{n}{target\PYZus{}return}\PY{p}{)}
%		
%\PY{n}{results} \PY{o}{=} \PY{p}{[}\PY{p}{]}
%\PY{n}{bounds} \PY{o}{=} \PY{n+nb}{tuple}\PY{p}{(}\PY{p}{(}\PY{l+m+mi}{0}\PY{p}{,} \PY{l+m+mi}{1}\PY{p}{)} \PY{k}{for} \PY{n}{asset} \PY{o+ow}{in} \PY{n+nb}{range}\PY{p}{(}\PY{n}{num\PYZus{}assets}\PY{p}{)}\PY{p}{)}
%\PY{n}{target} \PY{o}{=} \PY{l+m+mf}{0.172}\PY{o}{*}\PY{l+m+mf}{1.4}
%\PY{n}{constraints} \PY{o}{=} \PY{p}{(}\PY{p}{\PYZob{}}\PY{l+s+s1}{\PYZsq{}}\PY{l+s+s1}{type}\PY{l+s+s1}{\PYZsq{}}\PY{p}{:} \PY{l+s+s1}{\PYZsq{}}\PY{l+s+s1}{eq}\PY{l+s+s1}{\PYZsq{}}\PY{p}{,} \PY{l+s+s1}{\PYZsq{}}\PY{l+s+s1}{fun}\PY{l+s+s1}{\PYZsq{}}\PY{p}{:} \PY{n}{efficient\PYZus{}frontier}\PY{p}{,} 
%                \PY{l+s+s1}{\PYZsq{}}\PY{l+s+s1}{args}\PY{l+s+s1}{\PYZsq{}}\PY{p}{:}\PY{p}{(}\PY{n}{returns}\PY{p}{,} \PY{n}{target}\PY{p}{)}\PY{p}{\PYZcb{}}\PY{p}{,}
%               \PY{p}{\PYZob{}}\PY{l+s+s1}{\PYZsq{}}\PY{l+s+s1}{type}\PY{l+s+s1}{\PYZsq{}}\PY{p}{:} \PY{l+s+s1}{\PYZsq{}}\PY{l+s+s1}{eq}\PY{l+s+s1}{\PYZsq{}}\PY{p}{,} \PY{l+s+s1}{\PYZsq{}}\PY{l+s+s1}{fun}\PY{l+s+s1}{\PYZsq{}}\PY{p}{:} \PY{n}{sum\PYZus{}weights}\PY{p}{\PYZcb{}}\PY{p}{)}
%\PY{n}{weights} \PY{o}{=} \PY{p}{[}\PY{l+m+mf}{1.}\PY{o}{/}\PY{n}{num\PYZus{}assets} \PY{k}{for} \PY{n}{\PYZus{}} \PY{o+ow}{in} \PY{n+nb}{range}\PY{p}{(}\PY{n}{num\PYZus{}assets}\PY{p}{)}\PY{p}{]}
%\PY{n}{opts} \PY{o}{=} \PY{n}{minimize}\PY{p}{(}\PY{n}{markowitz}\PY{p}{,} \PY{n}{weights}\PY{p}{,} \PY{n}{args}\PY{o}{=}\PY{p}{(}\PY{n}{covariance}\PY{p}{,}\PY{p}{)}\PY{p}{,}
%\PY{n}{bounds}\PY{o}{=}\PY{n}{bounds}\PY{p}{,} \PY{n}{constraints}\PY{o}{=}\PY{n}{constraints}\PY{p}{)} 
%		
%\PY{n}{p\PYZus{}variance} \PY{o}{=} \PY{n}{np}\PY{o}{.}\PY{n}{dot}\PY{p}{(}\PY{n}{opts}\PY{o}{.}\PY{n}{x}\PY{o}{.}\PY{n}{T}\PY{p}{,} \PY{n}{np}\PY{o}{.}\PY{n}{dot}\PY{p}{(}\PY{n}{covariance}\PY{p}{,} \PY{n}{opts}\PY{o}{.}\PY{n}{x}\PY{p}{)}\PY{p}{)}
%\PY{n}{ret} \PY{o}{=} \PY{n}{np}\PY{o}{.}\PY{n}{sum}\PY{p}{(}\PY{n}{returns}\PY{o}{*}\PY{n}{opts}\PY{o}{.}\PY{n}{x}\PY{p}{)}
%\PY{n+nb}{print} \PY{p}{(}\PY{l+s+s2}{\PYZdq{}}\PY{l+s+s2}{Portfolio variance: }\PY{l+s+si}{\PYZob{}:.3f\PYZcb{}}\PY{l+s+s2}{\PYZdq{}}\PY{o}{.}\PY{n}{format}\PY{p}{(}\PY{n}{variance}\PY{p}{)}\PY{p}{)}
%\PY{n+nb}{print} \PY{p}{(}\PY{l+s+s2}{\PYZdq{}}\PY{l+s+s2}{Portfolio return: }\PY{l+s+si}{\PYZob{}:.3f\PYZcb{}}\PY{l+s+s2}{\PYZdq{}}\PY{o}{.}\PY{n}{format}\PY{p}{(}\PY{n}{ret}\PY{p}{)}\PY{p}{)}
%
%Portfolio variance: 0.026
%Portfolio return: 0.241
%\end{Verbatim}
%\end{tcolorbox}	
%\end{solution}

\begin{question}
The client is not yet satisfied and decides to move to a risk parity portfolio. Compute the correct weights to have each asset contributing equally to the portfolio risk.
\end{question}

\cprotEnv \begin{solution}
\begin{ipython}
def risk_parity(w, cov):
    variance = np.dot(w.T, np.dot(cov, w))
    sum = 0 
    N = len(w)
    for i in range(N):
        sum += (w[i] - (variance/(N*np.dot(cov, w)[i])))**2
    return sum

args = (covariance,)
constraints = ({'type': 'eq', 'fun': sum_weights})
bounds = tuple((0, 1) for asset in range(num_assets))
weights = [1./num_assets for _ in range(num_assets)]
opts = minimize(risk_parity, weights, args=(covariance,),
    bounds=bounds, constraints=constraints)

print (opts)

    fun: 3.0597183058586087e-07
    jac: array([-5.36616836e-04,  1.70066372e-04, -3.68199937e-04, 
                 5.59974747e-03, -9.08102600e-04,  3.25565500e-04,  
                 3.17810572e-04, -6.52504960e-04,  1.32785256e-03, 
                -7.77000622e-05])
message: 'Optimization terminated successfully.'
   nfev: 163
    nit: 13
   njev: 13
 status: 0
success: True
      x: array([0.16075373, 0.14425109, 0.0998296 , 0.02449688, 0.10306514,
                0.09916455, 0.08835716, 0.0781992 , 0.04379689, 0.15808576])

sigma_i = []
for i in range(num_assets):
    std = np.sqrt(np.dot(opts.x.T, np.dot(covariance, opts.x)))
    a = opts.x[i]*np.dot(covariance, opts.x)[i]
    sigma_i.append(a/std)

for i in range(num_assets):
    print ("Risk contribution for asset {}: {:.3f}%"
        .format(i, sigma_i[i]/sum(sigma_i)*100))
        
Risk contribution for asset 0: 9.986%
Risk contribution for asset 1: 10.010%
Risk contribution for asset 2: 10.012%
Risk contribution for asset 3: 10.040%
Risk contribution for asset 4: 10.009%
Risk contribution for asset 5: 9.999%
Risk contribution for asset 6: 9.961%
Risk contribution for asset 7: 9.972%
Risk contribution for asset 8: 10.000%
Risk contribution for asset 9: 10.011%		
\end{ipython}
\end{solution}

\begin{question}
Stock A has a $\beta$ of 1.2 and Stock B has a $\beta$ of 0.6. Which of the following statements is true? 
\begin{enumerate}[label=\emph{\alph*})]
\item Stock A has more unsystematic risk than Stock B;
\item Stock B has more systematic risk than Stock A; 
\item if the risk-free rate and the market risk premium are both positive, Stock A has a higher expected return than Stock B according to the CAPM;
\item both \emph{a} and \emph{b} are true;
\item both \emph{b} and \emph{c} are true
\end{enumerate}

\end{question}

\begin{solution}
The correct answer is \emph{c} indeed according to CAPM formula
\[r_i = r_f + \beta_i (r_M - r_f)\]
if both $r_f$ and $(r_M - r_f)$ are positive then $r_A \gt r_B$.
\end{solution}	

\begin{question}
Consider a stock with a $\beta$ of 1.5. Which of the following statements is true?
 
\begin{enumerate}[label=\emph{\alph*}]
\item when the market goes down by 1.5\%, on average, the stock goes down by 1\%;
\item when the market goes up by 1.5\%, on average, the stock goes up by 1\%;
\item when the market goes up by 1\%, on average, the stock goes down by 1.5\%;
\item when the market goes down by 1\%, on average, the stock goes down by 1.5\%. 
\item both \emph{a} and \emph{b}.	
\end{enumerate}
\end{question}

\begin{solution}
The correct answer is \emph{d} since $\beta$ measures the slope of the regression line of the expected return of the stock vs the expected return of the market. So by definition it represents the variation of the stock expected return given the a unit variation of the expected return of the market.
\end{solution}	

\begin{question}
Suppose that the risk-free rate is 3\% and the market risk premium is 8\%. According to the CAPM, what is the required rate of return on a stock with a $\beta$ of 2 ?
\end{question}

\begin{solution}
Careful! The market risk premium is 8\%. This means that $r_M - r_f  = 8\%$. Plug this into the CAPM equation to get

\[r = r_f + \beta(r_M - r_f) = 3\% + 2\cdot(8\%) =19\%\]
\end{solution}

\begin{question}
You analyze the prospects of several companies and come to the following conclusions about the required return on each:

\begin{center}
\begin{tabular}{lc}
\textbf{Stock Required} & \textbf{Return} \\
Starbucks &18\% \\
Sears &8\% \\
Microsoft &16\% \\
Limited Brands &12\% \\
\end{tabular}
\end{center}

You decide to invest \$4000 in Starbucks, \$6000 in Sears, \$12000 in Microsoft, and \$3000 in Limited Brands. What is the required return on your portfolio?
\end{question}

\cprotEnv \begin{solution}
\begin{ipython}
P = 4000 + 6000 + 12000 + 3000
rp =(4000/P)*0.18 + (6000/P)*0.08 + (12000/P)*0.16 +(3000/P)*0.12
print ("Portfolio Return: {:.2f}%".format(rp*100))

Portfolio Return: 13.92%
\end{ipython}
\end{solution}	

\begin{question}
You have a portfolio that consists of 35\% Microsoft stock, 35\% Amazon stock, and 30\% GE stock. Microsoft has a $\beta$ of 1, Amazon has a $\beta$ of 3.0, and GE has a $\beta$ of 0.5. Treasury bills (the risk-free asset) currently offer a return of 4\%, and the expected return on the market is 11.5\%. What return should you expect on your portfolio according to the CAPM ? 
\end{question}

\cprotEnv \begin{solution}
You can work this problem two different ways.

\textbf{Method 1}: calculate the \(\beta\) of your portfolio and plug this \(\beta\) into the CAPM formula to get the required return of your portfolio.

\begin{ipython}
rm = 0.115
rf = 0.04
beta_p = 0.35 * 1 + 0.35 * 3 + 0.3 * 0.5
rp = rf + beta_p*(rm - rf)

print ("Portfolio beta: {:.2f}".format(beta_p))
print ("Portfolio return: {:.3f}%".format(rp*100))

Portfolio beta: 1.55
Portfolio return: 15.625%
\end{ipython}

\textbf{Method 2}: using the CAPM, calculate the required return on each individual stock. Then,calculate the weighted average of those required returns to get the required return of your portfolio.

\begin{ipython}
Er_msft = rf + 1*(rm-rf)
Er_amzn = rf + 3*(rm-rf)
Er_ge = rf + 0.5*(rm-rf)
rp2 = Er_msft * 0.35 + Er_amzn *0.35 + Er_ge * 0.3

print ("Er (MSFT): {:.2f}%".format(Er_msft*100))
print ("Er (AMZN): {:.2f}%".format(Er_amzn*100))
print ("Er (GE): {:.2f}%".format(Er_ge*100))
print ("Portfolio return: {:.3f}%".format(rp2*100))

Er (MSFT): 11.50%
Er (AMZN): 26.50%
Er (GE): 7.75%
Portfolio return: 15.625%
\end{ipython}
Notice that this is the same answer we got using method 1. When calculating the required return of a portfolio, it does not matter which way you do it. But method 1 is a little less work.
\end{solution}


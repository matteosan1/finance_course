\chapter{Machine Learning}\label{neural-network---practical-lesson-8}

In this Chapter we will see how machine learning techniques can be
successfully applied to solve financial problems. We will first do a
quick tour on the theory behind neural networks and then we will see an
example and two practical applications regarding regression and
classification issues.

This lecture just scratches the surface of the
machine learning topic which has seen a huge development in the latest
years leading to thousands of applications in many different fields.
    
\section{Neural Networks}\label{neural-networks}

Artificial Neural Networks (ANN or simply NN) are information processing
models that are developed by inspiring from the working principles of
human brain. Their most essential property is the ability of learning
from sample sets.

The basic unit of ANN architecture are neurons which are internally in
connection with other neurons.

\begin{figure}[htb]
	\centering
	\includegraphics[width=0.7\textwidth]{figures/neuron}
	\caption{Model of an artificial neuron.}
\end{figure}

A neuron consists of weights and real numbers. All
inputs injected into a neuron (\(x_i\)) are individually weighted , added together
(sometimes it is added also a bias \(w_0\)) and passed into the
activation function which produce the neuron output. There are many
different types of activation function but one of the simplest is the
\emph{step function} which returns just 0 or 1 according to the input
value (another is the \emph{sigmoid} which can be thought of as the
continuous version of the step function).

\begin{figure}[htb]
	\centering
	\includegraphics{figures/sigmoid.png}
	\caption{Sigmoidal activation function.}
\end{figure}

Other commonly used activation functions are \emph{Rectified Linear Unit}
(ReLU), \emph{Tan Hyperbolic} (tanh) and Identity function.

For an deeper discussion of activation functions see
\href{https://medium.com/the-theory-of-everything/understanding-activation-functions-in-neural-networks-9491262884e0}{this article}

\subsection{Training of a neuron}\label{training-of-a-neuron}

When teaching children how to recognize a bus, we just tell them,
showing an example: "This is a bus. That is not a bus." until they
learn the concept of what a bus is. Furthermore, if the child sees new
objects that she hasn't seen before, we could expect her to recognize
correctly whether the new object is a bus or not.

This is exactly the idea behind neurons. Similarly, inputs from a
\emph{training} set are presented to the neuron one after the other
together with the correct output and the neuron weights are modified
accordingly.

When an entire pass through all of the input training vectors is
completed the neuron has learn !

At this time, if an input vector \(\mathbf{P}\) (already in the training
set) is given to the neuron, it will output the correct value. If
\(\mathbf{P}\) is not in the training set, the network will respond with an
output similar to other training vectors close to \(\mathbf{P}\).

This kind of training is called \emph{supervised}.

Unfortunately using just a neuron is not too useful since it is not
possible to solve the interesting problems we would like to face with
just that simple architecture. The next step is then to put together
more neurons in \emph{layers}.

\subsection{Multilayered Neural Networks}\label{multi-layered-neural-networks}

\begin{figure}[htb]
	\centering
	\includegraphics[width=0.7\textwidth]{figures/multilayer.jpeg}
	\caption{A multi-layered neural network.}
\end{figure}

Each input from the \emph{input layer} is fed up to each node in the
next hidden layer, and from there to each node on the output layer. We
should note that there can be any number of nodes per layer and there
are usually multiple hidden layers to pass through before ultimately
reaching the output layer.

\subsection{Training a Multilayered Neural 		Network}\label{training-a-multilayered-neural-network}

The training of a multilayered NN follows these steps:

\begin{itemize}
	\tightlist
	\item
	present a training sample to the neural network (initialized with
	random weights);
	\item
	compute the network output obtained by calculating activation functions of each layer;
	\item
	calculate the error (loss) as the difference between the NN predicted
	output and the actual output;
	\item
	having calculated the error, readjust the weights of the network such
	that the error (difference) decreases;
	\item
	continue the process for all samples several times (epochs) until the
	weights are not changing too much (i.e. the process converged).
\end{itemize}

\begin{figure}[htb]
	\centering
	\includegraphics[width=0.7\textwidth]{figures/training_nn}
	\caption{Optimization of the weights done minimizing the loss function.}
\end{figure}

The NN error is computed by the \emph{loss function}. Different loss
functions will give different errors for the same prediction, and thus
have a considerable effect on the performance of the model. One of the
most widely used loss function is mean square error, which calculates
the square of difference between actual value and predicted value..

The error or loss is a function of the internal parameters of the model
i.e weights and bias. For accurate predictions, one needs to minimize
the calculated error. In a neural network, this is done using \emph{back propagation} (see \href{https://towardsdatascience.com/understanding-backpropagation-algorithm-7bb3aa2f95fd}{this	article}). The current error is typically propagated backwards to a previous layer, where it is used to modify the weights and bias in such a way that the error is minimized.

The weights are modified using a function called Optimization Function
(we will use \emph{Adam} as optimizator in the following but there are
more).

\begin{figure}[htb]
	\centering
	\includegraphics[width=0.7\textwidth]{figures/loss_function}
	\caption{Optimization of the weights done minimizing the loss function.}
\end{figure}

\subsection{Neural Network Design}\label{neural-network-design}

There is no rule to guide developer into the design of a neural network
in terms of number of layers and neuron per layer. The most common
strategy is a trail and error one where you finally pick up the solution
giving the best accuracy. In general a larger number of nodes is better
to catch highly structured data with a lot of feature although it may
require larger training sample to work correctly.

A common mistake to avoid is to \emph{overtrain} a NN. Overtraining is
what happens when the NN learns too well the training sample but its
performance degrade substantially in an independent testing sample.

So usually it is required to split the available sample in two parts
training and testing (e.g.~80\% and 20\%) and to use the former to
perform the training and the latter to cross-check the performance.
\textbf{Usually performance are \emph{measured} with the mean square
error computed between the truth of the sample and the NN predictions.}

Anyway as a rule of thumb a NN with just one hidden layer with a number
of neurons averaging the inputs and outputs is sufficient in most cases.
In the following we will use more complex networks just for
illustration, no strong attempt in optimizing the layout has been done
though.

\section{Function approximation}\label{function-approximation}

As a first practical example let's try to design an ANN which is capable
of learning the functional form underlying a set of data.

Let's generate a sample with \(x\) (input), \(f(x)\) (truth) pairs where
\(f(x) = x^3 +2\) and let's start to code the NN structure.

We start by importing the necessary modules. Then we generate the
training sample (i.e.~the \(x\), \(f(x)\) pairs) and apply a simple
transformation on the sample in order to have all the inputs and outputs
in the \([0, 1]\) range. This is usually done to provide the NN with
\emph{normalized} data, infact the NN can be fooled by large or very
small numbers giving unstable results.

\begin{tcolorbox}[breakable, size=fbox, boxrule=1pt, pad at break*=1mm,colback=cellbackground, colframe=cellborder]
\begin{Verbatim}[commandchars=\\\{\}]
\PY{k+kn}{from} \PY{n+nn}{finnn} \PY{k}{import} \PY{n}{FinNN}
\PY{k+kn}{from} \PY{n+nn}{numpy} \PY{k}{import} \PY{n}{arange}\PY{p}{,} \PY{n}{array}
		
\PY{c+c1}{\PYZsh{} define the dataset}
\PY{n}{x} \PY{o}{=} \PY{n}{array}\PY{p}{(}\PY{p}{[}\PY{n}{i} \PY{k}{for} \PY{n}{i} \PY{o+ow}{in} \PY{n}{arange}\PY{p}{(}\PY{o}{\PYZhy{}}\PY{l+m+mi}{2}\PY{p}{,} \PY{l+m+mi}{2}\PY{p}{,} \PY{l+m+mf}{0.005}\PY{p}{)}\PY{p}{]}\PY{p}{)}
		
\PY{n}{y} \PY{o}{=} \PY{n}{array}\PY{p}{(}\PY{p}{[}\PY{n}{i}\PY{o}{*}\PY{o}{*}\PY{l+m+mi}{3}\PY{o}{+}\PY{l+m+mi}{2} \PY{k}{for} \PY{n}{i} \PY{o+ow}{in} \PY{n}{x}\PY{p}{]}\PY{p}{)}
\PY{n+nb}{print}\PY{p}{(}\PY{l+s+s2}{\PYZdq{}}\PY{l+s+s2}{Distribution of original data }\PY{l+s+s2}{\PYZdq{}}\PY{p}{,} \PY{n}{x}\PY{o}{.}\PY{n}{min}\PY{p}{(}\PY{p}{)}\PY{p}{,} \PY{n}{x}\PY{o}{.}\PY{n}{max}\PY{p}{(}\PY{p}{)}\PY{p}{,} \PY{n}{y}\PY{o}{.}\PY{n}{min}\PY{p}{(}\PY{p}{)}\PY{p}{,} \PY{n}{y}\PY{o}{.}\PY{n}{max}\PY{p}{(}\PY{p}{)}\PY{p}{)}
		
\PY{n}{trainer} \PY{o}{=} \PY{n}{FinNN}\PY{p}{(}\PY{p}{)}
\PY{n}{trainer}\PY{o}{.}\PY{n}{setData}\PY{p}{(}\PY{n}{x}\PY{p}{,} \PY{n}{y}\PY{p}{)}
		
\PY{c+c1}{\PYZsh{}trainer.normalize()}
		
\PY{c+c1}{\PYZsh{} here you should see that x and y are between 0 and 1}
\PY{n+nb}{print}\PY{p}{(}\PY{l+s+s2}{\PYZdq{}}\PY{l+s+s2}{The same data after the normalization }\PY{l+s+s2}{\PYZdq{}}\PY{p}{,} \PY{n}{trainer}\PY{o}{.}\PY{n}{x}\PY{o}{.}\PY{n}{min}\PY{p}{(}\PY{p}{)}\PY{p}{,} 
\PY{n}{trainer}\PY{o}{.}\PY{n}{x}\PY{o}{.}\PY{n}{max}\PY{p}{(}\PY{p}{)}\PY{p}{,} \PY{n}{trainer}\PY{o}{.}\PY{n}{y}\PY{o}{.}\PY{n}{min}\PY{p}{(}\PY{p}{)}\PY{p}{,} \PY{n}{trainer}\PY{o}{.}\PY{n}{y}\PY{o}{.}\PY{n}{max}\PY{p}{(}\PY{p}{)}\PY{p}{)}

Distribution of original data  -2.0 1.9949999999999148 -6.0 9.940149874998983
The same data after the normalization  0.0 1.0 0.0 1.0
\end{Verbatim}
\end{tcolorbox}

Next we can define the structure of the neural network. There is no
predefined rule to decide the number of layers and nodes you need to go
by trial and error. Here the problem is quite simple so there is no need
to use a complecated NN.

In the end I have decided to use two layers with 10 nodes each and a
\emph{tanh} activation function. The input\_dim parameter has to be
set to 1 since we have just one single input, the \(x\) value.

\begin{tcolorbox}[breakable, size=fbox, boxrule=1pt, pad at break*=1mm,colback=cellbackground, colframe=cellborder]
\begin{Verbatim}[commandchars=\\\{\}]
\PY{c+c1}{\PYZsh{} design the neural network model}
\PY{n}{trainer}\PY{o}{.}\PY{n}{addInputLayer}\PY{p}{(}\PY{l+m+mi}{1}\PY{p}{,} \PY{l+m+mi}{15}\PY{p}{,} \PY{l+s+s1}{\PYZsq{}}\PY{l+s+s1}{tanh}\PY{l+s+s1}{\PYZsq{}}\PY{p}{)}
\PY{n}{trainer}\PY{o}{.}\PY{n}{addHiddenLayer}\PY{p}{(}\PY{l+m+mi}{5}\PY{p}{,} \PY{l+s+s1}{\PYZsq{}}\PY{l+s+s1}{tanh}\PY{l+s+s1}{\PYZsq{}}\PY{p}{)}
\PY{n}{trainer}\PY{o}{.}\PY{n}{addOutputLayer}\PY{p}{(}\PY{l+m+mi}{1}\PY{p}{)}
		
\PY{c+c1}{\PYZsh{} define the loss function (mean squared error) and optimization algorithm (Adam)}
\PY{n}{trainer}\PY{o}{.}\PY{n}{compileModel}\PY{p}{(}\PY{l+s+s1}{\PYZsq{}}\PY{l+s+s1}{mse}\PY{l+s+s1}{\PYZsq{}}\PY{p}{,} \PY{l+s+s1}{\PYZsq{}}\PY{l+s+s1}{adam}\PY{l+s+s1}{\PYZsq{}}\PY{p}{)}
		
\PY{c+c1}{\PYZsh{} fit the model on the training dataset}
\PY{c+c1}{\PYZsh{} using 500 epochs, a batch\PYZus{}size of 10}
\PY{n}{trainer}\PY{o}{.}\PY{n}{fit}\PY{p}{(}\PY{l+m+mi}{800}\PY{p}{,} \PY{l+m+mi}{100}\PY{p}{,} \PY{n}{verbose}\PY{o}{=}\PY{l+m+mi}{1}\PY{p}{)}
		
\PY{c+c1}{\PYZsh{} make predictions for the input data}
\PY{n}{trainer}\PY{o}{.}\PY{n}{fullPrediction}\PY{p}{(}\PY{p}{)}
		
\PY{c+c1}{\PYZsh{} invert previous transformation to get real data}
\PY{n}{trainer}\PY{o}{.}\PY{n}{reverseNormalization}\PY{p}{(}\PY{p}{)}

Epoch 1/800
799/799 [==============================] - 0s 127us/step - loss: 0.0919
Epoch 2/800
799/799 [==============================] - 0s 20us/step - loss: 0.0366
...
Epoch 799/800
799/799 [==============================] - 0s 15us/step - loss: 5.0436e-05
Epoch 800/800
799/799 [==============================] - 0s 20us/step - loss: 5.3768e-05
\end{Verbatim}
\end{tcolorbox}

After the training is completed we can evaluate how good it is. To do
this we can compute the residuals or the square root of the sum of the
squared difference between the true value and the one predicted by the
NN. We will also plot the true function and the predicted one in order
to have a graphical representation of the goodness of our training. To
have a numerical estimate of the agreement it has been computed also the
\emph{mean squared error} defined as:

\[\textrm{MSE} = \cfrac{\sum_{i=1}^n{\big(\cfrac{x_{i}^{pred} - x_i^{truth}}{x_i^{truth}}\big)^2}}{n}\]

A \emph{perfect} prediction would lead to \(\textrm{MSE}=0\) so the
lower this number the better the agreement.

\begin{tcolorbox}[breakable, size=fbox, boxrule=1pt, pad at break*=1mm,colback=cellbackground, colframe=cellborder]
\begin{Verbatim}[commandchars=\\\{\}]
\PY{k+kn}{from} \PY{n+nn}{sklearn}\PY{n+nn}{.}\PY{n+nn}{metrics} \PY{k}{import} \PY{n}{mean\PYZus{}squared\PYZus{}error}
		
\PY{c+c1}{\PYZsh{} report model error computing the mean squared error}
\PY{n+nb}{print}\PY{p}{(}\PY{l+s+s1}{\PYZsq{}}\PY{l+s+s1}{MSE: }\PY{l+s+si}{\PYZpc{}.3f}\PY{l+s+s1}{\PYZsq{}} \PY{o}{\PYZpc{}} \PY{n}{mean\PYZus{}squared\PYZus{}error}\PY{p}{(}\PY{n}{trainer}\PY{o}{.}\PY{n}{y}\PY{p}{,} \PY{n}{trainer}\PY{o}{.}\PY{n}{predictions}\PY{p}{)}\PY{p}{)}

MSE: 0.016
\end{Verbatim}
\end{tcolorbox}

To get an idea of what it is going on in the picture below are shown the
actual function we want to approximate and different predictions of our
NN obtained with four epoch numbers (5, 100, 800, 5000).

\begin{figure}[htb]
	\centering
	\includegraphics[width=0.85\textwidth]{figures/training_vs_epoch}
	\caption{Different function prediction as a function of the number of epochs used in the training.}
\end{figure}

It is clear how the agreement improves with higher number of epochs
which means that the NN has more opportunities to adapt the weights and
reduce the loss (or error or distance) to the target values. Even in the
case of 5000 epochs zooming in you could see discrepancies not visible
at the scale of the plot. Clearly increasing further the number of
epochs may lead to overfitting.

\subsection{Black-Scholes call
options}\label{black-scholes-call-options}

The first financial application of a NN concerns the pricing of european
call options: essentially we will create a neural network capable of
approximate the famous Black-Scholes pricing formula

\[P_\textrm{call} = F_\textrm{BS}(K, r, \sigma, ttm)\]

Like before we are going to generate the training sample this time made
of a grid of volatility-rate pairs \((\sigma, r)\) (for simplicity we
are going to set moneyness and time to maturity to 1). The truth values
are the price of a call computed using the pricing function in the
\(\tt{finmarkets.py}\) library with the corresponding inputs.

\begin{tcolorbox}[breakable, size=fbox, boxrule=1pt, pad at break*=1mm,colback=cellbackground, colframe=cellborder]
\begin{Verbatim}[commandchars=\\\{\}]
\PY{k+kn}{import} \PY{n+nn}{numpy} \PY{k}{as} \PY{n+nn}{np}
\PY{k+kn}{from} \PY{n+nn}{finmarkets} \PY{k}{import} \PY{n}{call}
		
\PY{n}{data} \PY{o}{=} \PY{p}{[}\PY{p}{]}
\PY{n}{rates} \PY{o}{=} \PY{n}{np}\PY{o}{.}\PY{n}{arange}\PY{p}{(}\PY{l+m+mf}{0.01}\PY{p}{,} \PY{l+m+mf}{0.11}\PY{p}{,} \PY{l+m+mf}{0.001}\PY{p}{)}
\PY{n}{sigmas} \PY{o}{=} \PY{n}{np}\PY{o}{.}\PY{n}{arange}\PY{p}{(}\PY{l+m+mf}{0.1}\PY{p}{,} \PY{l+m+mf}{0.6}\PY{p}{,} \PY{l+m+mf}{0.005}\PY{p}{)}
		
\PY{k}{for} \PY{n}{r} \PY{o+ow}{in} \PY{n}{rates}\PY{p}{:}
    \PY{k}{for} \PY{n}{sigma} \PY{o+ow}{in} \PY{n}{sigmas}\PY{p}{:}
        \PY{n}{call\PYZus{}price} \PY{o}{=} \PY{n}{call}\PY{p}{(}\PY{l+m+mi}{1}\PY{p}{,} \PY{n}{r}\PY{p}{,} \PY{n}{sigma}\PY{p}{,} \PY{l+m+mi}{1}\PY{p}{)}
        \PY{n}{data}\PY{o}{.}\PY{n}{append}\PY{p}{(}\PY{p}{[}\PY{n}{r}\PY{p}{,} \PY{n}{sigma}\PY{p}{,} \PY{n}{call\PYZus{}price}\PY{p}{]}\PY{p}{)}
		
\PY{c+c1}{\PYZsh{} we transform the list to a numpy array just because }
\PY{c+c1}{\PYZsh{} an array it is more convenient to use later}
\PY{n}{data} \PY{o}{=} \PY{n}{np}\PY{o}{.}\PY{n}{array}\PY{p}{(}\PY{n}{data}\PY{p}{)}
\end{Verbatim}
\end{tcolorbox}

Since it takes some time to generate data samples, it is always
advisable to save them in a file since we may need to load it many times
during the NN development. This can be done with \(\tt{pandas}\).

\begin{tcolorbox}[breakable, size=fbox, boxrule=1pt, pad at break*=1mm,colback=cellbackground, colframe=cellborder]
\begin{Verbatim}[commandchars=\\\{\}]
\PY{k+kn}{import} \PY{n+nn}{pandas} \PY{k}{as} \PY{n+nn}{pd}
		
\PY{n}{df} \PY{o}{=} \PY{n}{pd}\PY{o}{.}\PY{n}{DataFrame}\PY{p}{(}\PY{p}{)}
		
\PY{n}{df}\PY{p}{[}\PY{l+s+s1}{\PYZsq{}}\PY{l+s+s1}{vol}\PY{l+s+s1}{\PYZsq{}}\PY{p}{]} \PY{o}{=} \PY{n}{data}\PY{p}{[}\PY{p}{:}\PY{p}{,} \PY{l+m+mi}{1}\PY{p}{]}
\PY{n}{df}\PY{p}{[}\PY{l+s+s1}{\PYZsq{}}\PY{l+s+s1}{rate}\PY{l+s+s1}{\PYZsq{}}\PY{p}{]} \PY{o}{=} \PY{n}{data}\PY{p}{[}\PY{p}{:}\PY{p}{,} \PY{l+m+mi}{0}\PY{p}{]}
\PY{n}{df}\PY{p}{[}\PY{l+s+s1}{\PYZsq{}}\PY{l+s+s1}{price}\PY{l+s+s1}{\PYZsq{}}\PY{p}{]} \PY{o}{=} \PY{n}{data}\PY{p}{[}\PY{p}{:}\PY{p}{,} \PY{l+m+mi}{2}\PY{p}{]}
		
\PY{n}{df}\PY{o}{.}\PY{n}{to\PYZus{}csv}\PY{p}{(}\PY{l+s+s2}{\PYZdq{}}\PY{l+s+s2}{bs\PYZus{}training\PYZus{}sample.csv}\PY{l+s+s2}{\PYZdq{}}\PY{p}{)}
\end{Verbatim}
\end{tcolorbox}

Following the previous example we will use the \(\tt{FinNN}\) utility
class to develop the NN and also we will \emph{normalize} data to get
better results. \textbf{Beware that this time we have TWO input
parameters (rate and volatility)} and not just one.

Furthermore we will also split the generated sample into a training and
a testing part so that we could later check for overfitting by comparing
the performance in the two cases.

\begin{tcolorbox}[breakable, size=fbox, boxrule=1pt, pad at break*=1mm,colback=cellbackground, colframe=cellborder]
\begin{Verbatim}[commandchars=\\\{\}]
\PY{c+c1}{\PYZsh{} first load back data}
\PY{k+kn}{import} \PY{n+nn}{pandas} \PY{k}{as} \PY{n+nn}{pd}
\PY{k+kn}{from} \PY{n+nn}{finnn} \PY{k}{import} \PY{n}{FinNN}
		
\PY{n}{data} \PY{o}{=}  \PY{n}{pd}\PY{o}{.}\PY{n}{read\PYZus{}csv}\PY{p}{(}\PY{l+s+s2}{\PYZdq{}}\PY{l+s+s2}{bs\PYZus{}training\PYZus{}sample.csv}\PY{l+s+s2}{\PYZdq{}}\PY{p}{)}
		
\PY{n}{x} \PY{o}{=} \PY{n}{data}\PY{o}{.}\PY{n}{iloc}\PY{p}{[}\PY{p}{:}\PY{p}{,} \PY{l+m+mi}{1}\PY{p}{:}\PY{l+m+mi}{3}\PY{p}{]}\PY{o}{.}\PY{n}{values}
\PY{n}{y} \PY{o}{=} \PY{n}{data}\PY{o}{.}\PY{n}{iloc}\PY{p}{[}\PY{p}{:}\PY{p}{,} \PY{l+m+mi}{3}\PY{p}{]}\PY{o}{.}\PY{n}{values}

\PY{n}{trainer} \PY{o}{=} \PY{n}{FinNN}\PY{p}{(}\PY{p}{)}
		
\PY{n}{trainer}\PY{o}{.}\PY{n}{setData}\PY{p}{(}\PY{n}{x}\PY{p}{,} \PY{n}{y}\PY{p}{,} \PY{l+m+mf}{0.20}\PY{p}{)}
\PY{c+c1}{\PYZsh{} the last parameter tells the class to split the original}
\PY{c+c1}{\PYZsh{} sample in training (80\PYZpc{}) and testing parts (20\PYZpc{})}
\PY{n}{trainer}\PY{o}{.}\PY{n}{normalize}\PY{p}{(}\PY{p}{)}
		
\PY{c+c1}{\PYZsh{} define the NN architecture}
\PY{n}{trainer}\PY{o}{.}\PY{n}{addInputLayer}\PY{p}{(}\PY{l+m+mi}{2}\PY{p}{,} \PY{l+m+mi}{20}\PY{p}{,} \PY{l+s+s1}{\PYZsq{}}\PY{l+s+s1}{tanh}\PY{l+s+s1}{\PYZsq{}}\PY{p}{)}
\PY{n}{trainer}\PY{o}{.}\PY{n}{addHiddenLayer}\PY{p}{(}\PY{l+m+mi}{8}\PY{p}{,} \PY{l+s+s1}{\PYZsq{}}\PY{l+s+s1}{tanh}\PY{l+s+s1}{\PYZsq{}}\PY{p}{)}
\PY{n}{trainer}\PY{o}{.}\PY{n}{addOutputLayer}\PY{p}{(}\PY{l+m+mi}{1}\PY{p}{)}
		
\PY{c+c1}{\PYZsh{} define loss and optimizer algorithms}
\PY{n}{trainer}\PY{o}{.}\PY{n}{compileModel}\PY{p}{(}\PY{l+s+s1}{\PYZsq{}}\PY{l+s+s1}{mse}\PY{l+s+s1}{\PYZsq{}}\PY{p}{,} \PY{l+s+s1}{\PYZsq{}}\PY{l+s+s1}{adam}\PY{l+s+s1}{\PYZsq{}}\PY{p}{)}
		
\PY{c+c1}{\PYZsh{} run the training}
\PY{c+c1}{\PYZsh{} this time we are using many more epochs }
\PY{c+c1}{\PYZsh{} and a larger batch\PYZus{}size}
\PY{n}{trainer}\PY{o}{.}\PY{n}{fit}\PY{p}{(}\PY{l+m+mi}{2000}\PY{p}{,} \PY{l+m+mi}{500}\PY{p}{,} \PY{l+m+mi}{1}\PY{p}{)}
		
\PY{c+c1}{\PYZsh{} here we compare the performance }
\PY{c+c1}{\PYZsh{} on the training and test sample}
\PY{n}{trainer}\PY{o}{.}\PY{n}{evaluate}\PY{p}{(}\PY{p}{)}
		
\PY{c+c1}{\PYZsh{} when the training takes some time it is useful}
\PY{c+c1}{\PYZsh{} to save the model weights in a file to use it later on}
\PY{n}{trainer}\PY{o}{.}\PY{n}{saveModel}\PY{p}{(}\PY{p}{)}

Epoch 1/2000
8000/8000 [==============================] - 0s 15us/step - loss: 0.0101
Epoch 2/2000
8000/8000 [==============================] - 0s 4us/step - loss: 0.0039
...
Epoch 1999/2000
8000/8000 [==============================] - 0s 4us/step - loss: 8.5889e-07
Epoch 2000/2000
8000/8000 [==============================] - 0s 3us/step - loss: 2.1917e-06

8000/8000 [==============================] - 0s 16us/step
Training: 9.342097100670799e-07
2000/2000 [==============================] - 0s 14us/step
Test: 9.361866041217582e-07
\end{Verbatim}
\end{tcolorbox}

As you can see the training and test samples give roughly the same
MSE value so we are reasonably sure that there hasn't been
\emph{overfitting}. After the training is completed again we can
evaluate graphically how good it is, see Fig.~\ref{fig:vol_rate}. 

\begin{figure}[htb]
	\centering
	\includegraphics[width=0.7\textwidth]{figures/vol_rate}
	\caption{Mean squared error as a function of volatility and rate for our Black-Scholes function prediction.}
	\label{fig:vol_rate}
\end{figure}

We can also compare the prediction
in a practical case; let's say we want to know the price of a call (with
moneyness 1 and time to maturity 1 year) when the interest rate is 0.015
and the volatility 0.234

\begin{tcolorbox}[breakable, size=fbox, boxrule=1pt, pad at break*=1mm,colback=cellbackground, colframe=cellborder]
\begin{Verbatim}[commandchars=\\\{\}]
\PY{k+kn}{import} \PY{n+nn}{numpy} \PY{k}{as} \PY{n+nn}{np}
\PY{k+kn}{from} \PY{n+nn}{finmarkets} \PY{k}{import} \PY{n}{call}
		
\PY{c+c1}{\PYZsh{} here we load the trained model}
\PY{n}{trainer}\PY{o}{.}\PY{n}{loadModel}\PY{p}{(}\PY{l+s+s1}{\PYZsq{}}\PY{l+s+s1}{test}\PY{l+s+s1}{\PYZsq{}}\PY{p}{)}
		
\PY{c+c1}{\PYZsh{} this is our input vector}
\PY{n}{rv} \PY{o}{=} \PY{n}{np}\PY{o}{.}\PY{n}{array}\PY{p}{(}\PY{p}{[}\PY{p}{[}\PY{l+m+mf}{0.234}\PY{p}{,} \PY{l+m+mf}{0.015}\PY{p}{]}\PY{p}{]}\PY{p}{)}
\PY{c+c1}{\PYZsh{} here we compare the predection with the BS call price}
\PY{n+nb}{print} \PY{p}{(}\PY{l+s+s1}{\PYZsq{}}\PY{l+s+si}{\PYZob{}\PYZcb{}}\PY{l+s+s1}{ =\PYZgt{} }\PY{l+s+si}{\PYZob{}:.4f\PYZcb{}}\PY{l+s+s1}{ (expected }\PY{l+s+si}{\PYZob{}:.4f\PYZcb{}}\PY{l+s+s1}{)}\PY{l+s+s1}{\PYZsq{}}\PY{o}{.}\PY{n}{format}\PY{p}{(}\PY{n}{rv}\PY{o}{.}\PY{n}{tolist}\PY{p}{(}\PY{p}{)}\PY{p}{,} 
\PY{n}{trainer}\PY{o}{.}\PY{n}{predict}\PY{p}{(}\PY{n}{rv}\PY{p}{)}\PY{p}{[}\PY{l+m+mi}{0}\PY{p}{]}\PY{p}{[}\PY{l+m+mi}{0}\PY{p}{]}\PY{p}{,} 
\PY{n}{call}\PY{p}{(}\PY{l+m+mi}{1}\PY{p}{,} \PY{n}{rv}\PY{p}{[}\PY{l+m+mi}{0}\PY{p}{]}\PY{p}{[}\PY{l+m+mi}{1}\PY{p}{]}\PY{p}{,} \PY{n}{rv}\PY{p}{[}\PY{l+m+mi}{0}\PY{p}{]}\PY{p}{[}\PY{l+m+mi}{0}\PY{p}{]}\PY{p}{,} \PY{l+m+mi}{1}\PY{p}{)}\PY{p}{)}\PY{p}{)}

[[0.234, 0.015]] => 0.0999 (expected 0.1001)
\end{Verbatim}
\end{tcolorbox}

It is very import to remeber that a \textbf{NN cannot estrapolate}.
Indeed if you try to predict the price of a call from rate and
volatility outside the training \emph{phase space} (with values that
aren't in the intervals used in the training), say \(r = 0.22\) and
\(\sigma = 0.01\)\ldots{}

\begin{tcolorbox}[breakable, size=fbox, boxrule=1pt, pad at break*=1mm,colback=cellbackground, colframe=cellborder]
\begin{Verbatim}[commandchars=\\\{\}]
\PY{c+c1}{\PYZsh{} this is our input vector}
\PY{n}{rv} \PY{o}{=} \PY{n}{np}\PY{o}{.}\PY{n}{array}\PY{p}{(}\PY{p}{[}\PY{p}{[}\PY{l+m+mf}{0.01}\PY{p}{,} \PY{l+m+mf}{0.22}\PY{p}{]}\PY{p}{]}\PY{p}{)}
		
\PY{c+c1}{\PYZsh{} here we compare the predection with the BS call price                 }
\PY{n+nb}{print} \PY{p}{(}\PY{l+s+s1}{\PYZsq{}}\PY{l+s+si}{\PYZob{}\PYZcb{}}\PY{l+s+s1}{ =\PYZgt{} }\PY{l+s+si}{\PYZob{}:.4f\PYZcb{}}\PY{l+s+s1}{ (expected }\PY{l+s+si}{\PYZob{}:.4f\PYZcb{}}\PY{l+s+s1}{)}\PY{l+s+s1}{\PYZsq{}}\PY{o}{.}\PY{n}{format}\PY{p}{(}\PY{n}{rv}\PY{o}{.}\PY{n}{tolist}\PY{p}{(}\PY{p}{)}\PY{p}{,} 
\PY{n}{trainer}\PY{o}{.}\PY{n}{predict}\PY{p}{(}\PY{n}{rv}\PY{p}{)}\PY{p}{[}\PY{l+m+mi}{0}\PY{p}{]}\PY{p}{[}\PY{l+m+mi}{0}\PY{p}{]}\PY{p}{,} 
\PY{n}{call}\PY{p}{(}\PY{l+m+mi}{1}\PY{p}{,} \PY{n}{rv}\PY{p}{[}\PY{l+m+mi}{0}\PY{p}{]}\PY{p}{[}\PY{l+m+mi}{1}\PY{p}{]}\PY{p}{,} \PY{n}{rv}\PY{p}{[}\PY{l+m+mi}{0}\PY{p}{]}\PY{p}{[}\PY{l+m+mi}{0}\PY{p}{]}\PY{p}{,} \PY{l+m+mi}{1}\PY{p}{)}\PY{p}{)}\PY{p}{)}

[[0.01, 0.22]] => 0.1709 (expected 0.1975)
\end{Verbatim}
\end{tcolorbox}

\subsection{Model Calibration}\label{model-calibration}

The function approximation of a neural network can serve other scopes
rather than predicting prices. An very useful application is indeed
\emph{model calibration} which consists of deriving parameters of a
model directly from market values. This is especially convenient to
estimate parameters (e.g.~volatility) which are otherwise complicated to
compute.

Assume we need to estimate the implied volatility of a stock price in
real time. If in the market are available call options with out stock as
underlying we can exploit again the Black and Scholes formula. The idea
is in fact to train a NN where the input is a list of price, moneyness,
rate and time to maturity \((P_\textrm{call}, K, r, ttm)\) and the
target output is the volatility derived from the inversion of the call
option pricing formula

\[ \sigma = F^{-1}_\textrm{BS}(P_\textrm{call}, K, r, ttm)\]

We can than calibrate our model by predicting the stock volatility with
the trained NN using as input the market price of the option and its
characteristics.

\subsubsection{Historical vs Implied
		Volatility}\label{historical-vs.implied-volatility}

Historical volatility is the realized volatility of the underlying asset
over a previous time period. It is determined by measuring the standard
deviation of the underlying asset from the mean during that time period.

Standard deviation is a statistical measure of the variability of price
changes from the mean price change. This estimate differs from the
Black-Scholes method's implied volatility, as it is based on the actual
volatility of the underlying asset. However, using historical volatility
also has some drawbacks. Volatility shifts as markets go through
different regimes. Thus, historical volatility may not be an accurate
measure of future volatility. Implied volatility takes into account all
of the information used by market participants to determine prices in
the options market, instead of just past prices.

As an example we can reuse the training sample created before (again we
are going to set \(T=1\) and \(K=1\)). Clearly now \(\tt{x}\) will be
pairs of rate and price and \(\tt{y}\) the volatility.

\begin{tcolorbox}[breakable, size=fbox, boxrule=1pt, pad at break*=1mm,colback=cellbackground, colframe=cellborder]
\begin{Verbatim}[commandchars=\\\{\}]
\PY{k+kn}{import} \PY{n+nn}{pandas} \PY{k}{as} \PY{n+nn}{pd}
\PY{k+kn}{from} \PY{n+nn}{finnn} \PY{k}{import} \PY{n}{FinNN}
		
\PY{n}{data} \PY{o}{=}  \PY{n}{pd}\PY{o}{.}\PY{n}{read\PYZus{}csv}\PY{p}{(}\PY{l+s+s2}{\PYZdq{}}\PY{l+s+s2}{bs\PYZus{}training\PYZus{}sample.csv}\PY{l+s+s2}{\PYZdq{}}\PY{p}{)}
		
\PY{n}{x} \PY{o}{=} \PY{n}{data}\PY{o}{.}\PY{n}{iloc}\PY{p}{[}\PY{p}{:}\PY{p}{,} \PY{l+m+mi}{2}\PY{p}{:}\PY{l+m+mi}{4}\PY{p}{]}\PY{o}{.}\PY{n}{values}
\PY{n}{y} \PY{o}{=} \PY{n}{data}\PY{o}{.}\PY{n}{iloc}\PY{p}{[}\PY{p}{:}\PY{p}{,} \PY{l+m+mi}{1}\PY{p}{]}\PY{o}{.}\PY{n}{values}

\PY{n}{trainer} \PY{o}{=} \PY{n}{FinNN}\PY{p}{(}\PY{p}{)}
		
\PY{n}{trainer}\PY{o}{.}\PY{n}{setData}\PY{p}{(}\PY{n}{x}\PY{p}{,} \PY{n}{y}\PY{p}{,} \PY{l+m+mf}{0.20}\PY{p}{)}
\PY{n}{trainer}\PY{o}{.}\PY{n}{normalize}\PY{p}{(}\PY{p}{)}
		
\PY{n}{trainer}\PY{o}{.}\PY{n}{addInputLayer}\PY{p}{(}\PY{l+m+mi}{2}\PY{p}{,} \PY{l+m+mi}{20}\PY{p}{,} \PY{l+s+s1}{\PYZsq{}}\PY{l+s+s1}{tanh}\PY{l+s+s1}{\PYZsq{}}\PY{p}{)}
\PY{n}{trainer}\PY{o}{.}\PY{n}{addHiddenLayer}\PY{p}{(}\PY{l+m+mi}{8}\PY{p}{,} \PY{l+s+s1}{\PYZsq{}}\PY{l+s+s1}{tanh}\PY{l+s+s1}{\PYZsq{}}\PY{p}{)}
\PY{n}{trainer}\PY{o}{.}\PY{n}{addOutputLayer}\PY{p}{(}\PY{l+m+mi}{1}\PY{p}{)}
		
\PY{n}{trainer}\PY{o}{.}\PY{n}{compileModel}\PY{p}{(}\PY{l+s+s1}{\PYZsq{}}\PY{l+s+s1}{mse}\PY{l+s+s1}{\PYZsq{}}\PY{p}{,} \PY{l+s+s1}{\PYZsq{}}\PY{l+s+s1}{adam}\PY{l+s+s1}{\PYZsq{}}\PY{p}{)}
\PY{n}{trainer}\PY{o}{.}\PY{n}{fit}\PY{p}{(}\PY{l+m+mi}{2000}\PY{p}{,} \PY{l+m+mi}{500}\PY{p}{,} \PY{l+m+mi}{1}\PY{p}{)}
\PY{n}{trainer}\PY{o}{.}\PY{n}{evaluate}\PY{p}{(}\PY{p}{)}
\PY{n}{trainer}\PY{o}{.}\PY{n}{saveModel}\PY{p}{(}\PY{l+s+s2}{\PYZdq{}}\PY{l+s+s2}{calibration}\PY{l+s+s2}{\PYZdq{}}\PY{p}{)}

Epoch 1/2000
8000/8000 [==============================] - 0s 15us/step - loss: 1.7378
Epoch 2/2000
8000/8000 [==============================] - 0s 4us/step - loss: 0.7167
...
Epoch 2000/2000
8000/8000 [==============================] - 0s 4us/step - loss: 1.5793e-06

8000/8000 [==============================] - 0s 15us/step
Training: 1.9411823022892348e-06
2000/2000 [==============================] - 0s 14us/step
Test: 1.997178464080207e-06
\end{Verbatim}
\end{tcolorbox}

Provided our training includes the correct range of market prices of our
call we can quickly and easily estimate the volatility, for example if
the risk-free rate is 2\% and the current price is 0.15 (remember that
we are using the BS formula in terms of moneyness, strike divided by
underlying price).

\begin{tcolorbox}[breakable, size=fbox, boxrule=1pt, pad at break*=1mm,colback=cellbackground, colframe=cellborder]
\begin{Verbatim}[commandchars=\\\{\}]
\PY{k+kn}{import} \PY{n+nn}{numpy} \PY{k}{as} \PY{n+nn}{np}
\PY{k+kn}{from} \PY{n+nn}{finmarkets} \PY{k}{import} \PY{n}{call}
		
\PY{n}{trainer}\PY{o}{.}\PY{n}{loadModel}\PY{p}{(}\PY{l+s+s1}{\PYZsq{}}\PY{l+s+s1}{calibration}\PY{l+s+s1}{\PYZsq{}}\PY{p}{)}
		
\PY{n}{rv} \PY{o}{=} \PY{n}{np}\PY{o}{.}\PY{n}{array}\PY{p}{(}\PY{p}{[}\PY{p}{[}\PY{l+m+mf}{0.02}\PY{p}{,} \PY{l+m+mf}{0.15}\PY{p}{]}\PY{p}{]}\PY{p}{)}
		
\PY{n+nb}{print} \PY{p}{(}\PY{l+s+s1}{\PYZsq{}}\PY{l+s+si}{\PYZob{}\PYZcb{}}\PY{l+s+s1}{ =\PYZgt{} }\PY{l+s+si}{\PYZob{}:.4f\PYZcb{}}\PY{l+s+s1}{ (expected call price }\PY{l+s+si}{\PYZob{}:.4f\PYZcb{}}\PY{l+s+s1}{)}\PY{l+s+s1}{\PYZsq{}}\PY{o}{.}\PY{n}{format}\PY{p}{(}\PY{n}{rv}\PY{o}{.}\PY{n}{tolist}\PY{p}{(}\PY{p}{)}\PY{p}{,} 
\PY{n}{trainer}\PY{o}{.}\PY{n}{predict}\PY{p}{(}\PY{n}{rv}\PY{p}{)}\PY{p}{[}\PY{l+m+mi}{0}\PY{p}{]}\PY{p}{[}\PY{l+m+mi}{0}\PY{p}{]}\PY{p}{,} 
\PY{n}{call}\PY{p}{(}\PY{l+m+mi}{1}\PY{p}{,} \PY{l+m+mf}{0.02}\PY{p}{,} \PY{n}{trainer}\PY{o}{.}\PY{n}{predict}\PY{p}{(}\PY{n}{rv}\PY{p}{)}\PY{p}{[}\PY{l+m+mi}{0}\PY{p}{]}\PY{p}{[}\PY{l+m+mi}{0}\PY{p}{]}\PY{p}{,} \PY{l+m+mi}{1}\PY{p}{)}\PY{p}{)}\PY{p}{)}

[[0.02, 0.15]] => 0.3554 (expected call price 0.1498)
\end{Verbatim}
\end{tcolorbox}

\section{Neural net to recognize handwritten
		digits}\label{neural-net-to-recognize-handwritten-digits}

We don't usually appreciate how tough a problem our visual system solve,
maybe it is enough to consider that it involves 5 visual cortices
containing 140 million neurons each. However the difficulties of visual
pattern recognition become apparent if you attempt to write a computer
program to recognize digits like those in Fig.~\ref{fig:mnist}

\begin{figure}[htb]
	\centering
	\includegraphics[width=0.5\textwidth]{figures/mnist_100_digits}
	\caption{MNIST sample of handwritten digits.}
	\label{fig:mnist}
\end{figure}

Simple intuition about how we recognize shapes (e.g. a 9 has a loop at
the top, and a vertical stroke in the bottom right) turns out to be not
so simple to express algorithmically. When you try to make such rules
precise, you quickly get lost in a morass of exceptions and caveats and
special cases so that it seems hopeless.

Neural networks approach the problem in a different way. The idea is to
take a large number of handwritten digits and then develop a system
which can learn from those.

By increasing the number of training examples, the network can learn
more and more about handwriting, and so improve its accuracy. So while
it has been shown just 100 training digits above, we could certainly
build a better handwriting recognizer by using thousands or even
millions or billions of training examples (\textbf{as we have seen above	neural nets are not capable of extrapolating results, hence it won't
recognize a digit written in some strange way not included in the
training sample !!!}).

Let's try to implement a NN that is capable of recognizing handwritten
digits. To start we need to install another module, \(\tt{mnist}\) which
contains various predefined training samples.

Our program will be based on a slightly different kind of neural
network, one type specifically designed for image/pattern recognition,
the Convolutional Neural Network (CNN). We won't go in the details of
its implementation since it is outside the scope of these lectures but
it works essentially by applying on top of an image a series of filters
(\emph{convolutional layers}) that works as edge detectors. With them it
classifies the images according to their relevant features.

Convolutional layers prove very effective, and stacking them allows to
learn low-level features (e.g.~lines) and high-order or more abstract
features, like shapes or specific objects.

\begin{figure}[htb]
	\centering
	\includegraphics[width=1.\textwidth]{figures/edges.jpg}
	\caption{Edge detected by different layers of a convolutional neural network.}
\end{figure}

Another important difference with respect to the previous examples is
that in this case we are going to solve a classification problem
(contrary to before when we were trying to regress a sample or in other
word to approximate a function). Indeed our NN output won't be a single
number but rather a list containing the probabilities that an image
belong to class on the classes.

\begin{tcolorbox}[breakable, size=fbox, boxrule=1pt, pad at break*=1mm,colback=cellbackground, colframe=cellborder]
\begin{Verbatim}[commandchars=\\\{\}]
\PY{k+kn}{import} \PY{n+nn}{numpy} \PY{k}{as} \PY{n+nn}{np}\PY{o}{,} \PY{n+nn}{mnist}
\PY{k+kn}{from} \PY{n+nn}{finnn} \PY{k}{import} \PY{n}{FinNN}
		
\PY{c+c1}{\PYZsh{} load the training}
\PY{n}{train\PYZus{}images} \PY{o}{=} \PY{n}{mnist}\PY{o}{.}\PY{n}{train\PYZus{}images}\PY{p}{(}\PY{p}{)} \PY{c+c1}{\PYZsh{} the actual images}
\PY{n}{train\PYZus{}labels} \PY{o}{=} \PY{n}{mnist}\PY{o}{.}\PY{n}{train\PYZus{}labels}\PY{p}{(}\PY{p}{)} \PY{c+c1}{\PYZsh{} the truth (it is a 0, 1, 2...)}
		
\PY{c+c1}{\PYZsh{} 0 means do not split the sample in training and testing sets}
\PY{c+c1}{\PYZsh{} (MNIST has already dont it for us)}
\PY{c+c1}{\PYZsh{} the last parameter tells FinNN class that we are going to develop a CNN}
\PY{n}{trainer} \PY{o}{=} \PY{n}{FinNN}\PY{p}{(}\PY{l+s+s2}{\PYZdq{}}\PY{l+s+s2}{CNN2D}\PY{l+s+s2}{\PYZdq{}}\PY{p}{)}
\PY{n}{trainer}\PY{o}{.}\PY{n}{setData}\PY{p}{(}\PY{n}{train\PYZus{}images}\PY{p}{,} \PY{n}{train\PYZus{}labels}\PY{p}{)}
\PY{c+c1}{\PYZsh{}trainer.normalize()}

\PY{c+c1}{\PYZsh{} define our convolutional NN}
\PY{c+c1}{\PYZsh{} we decide to apply 8 filters to the images }
\PY{c+c1}{\PYZsh{} each with 3x3 pixels size}
\PY{c+c1}{\PYZsh{} the input images have 28x28 pixels size instead}
\PY{n}{trainer}\PY{o}{.}\PY{n}{addConv2DLayer}\PY{p}{(}\PY{l+m+mi}{8}\PY{p}{,} \PY{l+m+mi}{3}\PY{p}{,} \PY{p}{(}\PY{l+m+mi}{28}\PY{p}{,} \PY{l+m+mi}{28}\PY{p}{,} \PY{l+m+mi}{1}\PY{p}{)}\PY{p}{)}
		
\PY{c+c1}{\PYZsh{} the output is given by 10 neurons returning the }
\PY{c+c1}{\PYZsh{} probability that image is in each class.}
\PY{n}{trainer}\PY{o}{.}\PY{n}{addFlatten}\PY{p}{(}\PY{p}{)}
\PY{n}{trainer}\PY{o}{.}\PY{n}{addCNNOutputLayer}\PY{p}{(}\PY{l+m+mi}{10}\PY{p}{)}
		
\PY{n}{trainer}\PY{o}{.}\PY{n}{compileModel}\PY{p}{(}\PY{l+s+s1}{\PYZsq{}}\PY{l+s+s1}{categorical\PYZus{}crossentropy}\PY{l+s+s1}{\PYZsq{}}\PY{p}{,} \PY{l+s+s1}{\PYZsq{}}\PY{l+s+s1}{adam}\PY{l+s+s1}{\PYZsq{}}\PY{p}{)}
		
\PY{n}{trainer}\PY{o}{.}\PY{n}{fit}\PY{p}{(}\PY{l+m+mi}{5}\PY{p}{)}
		
\PY{n}{trainer}\PY{o}{.}\PY{n}{saveModel}\PY{p}{(}\PY{l+s+s1}{\PYZsq{}}\PY{l+s+s1}{digit\PYZus{}training}\PY{l+s+s1}{\PYZsq{}}\PY{p}{)}

Epoch 1/5
59999/59999 [==============================] - 35s 590us/step - loss: 2.5698
Epoch 2/5
59999/59999 [==============================] - 35s 588us/step - loss: 0.3765
Epoch 3/5
59999/59999 [==============================] - 36s 603us/step - loss: 0.2399
Epoch 4/5
51680/59999 [========================>{\ldots}] - ETA: 4s - loss: 0.1979
\end{Verbatim}
\end{tcolorbox}

\begin{tcolorbox}[breakable, size=fbox, boxrule=1pt, pad at break*=1mm, colframe=cellborder, colback=cream]
\subsubsection{For the Most Curious}\label{for-the-most-curious}

If you look closely to the \(\tt{finnn.py}\) module you will notice that
I have cheated when describing the CNN architecture. In particular I
have not mentioned the \(\tt{MaxPooling2}\) layer, so let's clarify its
feature.

Convolutional layers in a convolutional neural network systematically
apply learned filters to input images in order to create feature maps
that summarize the presence of those features in the input.

A limitation of the feature map output of convolutional layers is that
they record the precise position of features in the input. This means
that small movements in the position of the feature in the input image
will result in a different feature map. This can happen with
re-cropping, rotation, shifting, and other minor changes to the input
image.

Imagine a program that look for car plates in pictures taken by a speed
radar, cars won't be in the same position in the frame so there may be
differences in the classification of similar (but not equal) pictures.

A common approach to address this problem from signal processing is
called \emph{down sampling}. This is where a lower resolution version of
an input signal (e.g.~the picture) is created that still contains the
large or important structural elements, without the fine detail that may
not be as useful to the task.

Down sampling can be achieved using a pooling layer.

Pooling involves selecting a pooling operation, much like a filter to be
applied to feature maps. The size of the pooling operation or filter is
smaller than the size of the feature map; specifically, it is almost
always 2×2 pixels. This means that the pooling layer will always reduce
the size of each feature map by a factor of 2, e.g.~each dimension is
halved. For example, a pooling layer applied to a feature map of 6×6 (36
pixels) will result in an output pooled feature map of 3×3 (9 pixels).

The most common pooling operation are: * Average Pooling: calculate the
average value for each patch on the feature map; * Maximum Pooling (or
Max Pooling): calculate the maximum value for each patch of the feature
map.
\end{tcolorbox}

Now let's try to see how well our NN predicts \(\tt{mnist}\) testing
digits.

\begin{tcolorbox}[breakable, size=fbox, boxrule=1pt, pad at break*=1mm,colback=cellbackground, colframe=cellborder]
\begin{Verbatim}[commandchars=\\\{\}]
\PY{k+kn}{import} \PY{n+nn}{numpy} \PY{k}{as} \PY{n+nn}{np}
\PY{k+kn}{import} \PY{n+nn}{mnist}
		
\PY{n}{trainer}\PY{o}{.}\PY{n}{loadModel}\PY{p}{(}\PY{l+s+s1}{\PYZsq{}}\PY{l+s+s1}{digit\PYZus{}training}\PY{l+s+s1}{\PYZsq{}}\PY{p}{)}
		
\PY{c+c1}{\PYZsh{} testing with mnist test sample}
\PY{n}{test\PYZus{}images} \PY{o}{=} \PY{n}{mnist}\PY{o}{.}\PY{n}{test\PYZus{}images}\PY{p}{(}\PY{p}{)}
\PY{n}{test\PYZus{}labels} \PY{o}{=} \PY{n}{mnist}\PY{o}{.}\PY{n}{test\PYZus{}labels}\PY{p}{(}\PY{p}{)}
		
\PY{n}{test\PYZus{}images} \PY{o}{=} \PY{n}{np}\PY{o}{.}\PY{n}{expand\PYZus{}dims}\PY{p}{(}\PY{n}{test\PYZus{}images}\PY{p}{,} \PY{n}{axis}\PY{o}{=}\PY{l+m+mi}{3}\PY{p}{)}
		
\PY{n}{predictions} \PY{o}{=} \PY{n}{trainer}\PY{o}{.}\PY{n}{predict}\PY{p}{(}\PY{n}{test\PYZus{}images}\PY{p}{[}\PY{p}{:}\PY{l+m+mi}{10}\PY{p}{]}\PY{p}{)}
\PY{n+nb}{print} \PY{p}{(}\PY{l+s+s2}{\PYZdq{}}\PY{l+s+s2}{Tesing on MNIST digits...}\PY{l+s+s2}{\PYZdq{}}\PY{p}{)}
\PY{n+nb}{print}\PY{p}{(}\PY{l+s+s2}{\PYZdq{}}\PY{l+s+s2}{Predicted: }\PY{l+s+s2}{\PYZdq{}}\PY{p}{,} \PY{n}{np}\PY{o}{.}\PY{n}{argmax}\PY{p}{(}\PY{n}{predictions}\PY{p}{,} \PY{n}{axis}\PY{o}{=}\PY{l+m+mi}{1}\PY{p}{)}\PY{p}{)} 
\PY{n+nb}{print}\PY{p}{(}\PY{l+s+s2}{\PYZdq{}}\PY{l+s+s2}{Truth:}\PY{l+s+s2}{\PYZdq{}}\PY{p}{,} \PY{n}{test\PYZus{}labels}\PY{p}{[}\PY{p}{:}\PY{l+m+mi}{10}\PY{p}{]}\PY{p}{)}
		
\PY{c+c1}{\PYZsh{} this line returns the highest probability of the vector}
\PY{n+nb}{print}\PY{p}{(}\PY{l+s+s2}{\PYZdq{}}\PY{l+s+s2}{highest prob.:}\PY{l+s+s2}{\PYZdq{}}\PY{p}{,} \PY{p}{[}\PY{l+s+s2}{\PYZdq{}}\PY{l+s+si}{\PYZob{}:.6f\PYZcb{}}\PY{l+s+s2}{\PYZdq{}}\PY{o}{.}\PY{n}{format}\PY{p}{(}\PY{n}{p}\PY{p}{[}\PY{n}{np}\PY{o}{.}\PY{n}{argmax}\PY{p}{(}\PY{n}{p}\PY{p}{)}\PY{p}{]}\PY{p}{)} \PY{k}{for} \PY{n}{p} \PY{o+ow}{in} \PY{n}{predictions}\PY{p}{]}\PY{p}{)}

Tesing on MNIST digits{\ldots}
Predicted:  [7 2 1 0 4 1 4 9 5 9]
Truth: [7 2 1 0 4 1 4 9 5 9]
highest prob.: ['0.999999', '0.999928', '0.999889', '1.000000', '1.000000',
                '0.999999', '0.999977', '0.999983', '0.931519', '0.999535']
\end{Verbatim}
\end{tcolorbox}

Since the last but one digit has lower probability let's check the
returned list to see which other number have non-zero probability.

\begin{tcolorbox}[breakable, size=fbox, boxrule=1pt, pad at break*=1mm,colback=cellbackground, colframe=cellborder]
\begin{Verbatim}[commandchars=\\\{\}]
\PY{n+nb}{print}\PY{p}{(}\PY{l+s+s2}{\PYZdq{}}\PY{l+s+s2}{9th digit:}\PY{l+s+s2}{\PYZdq{}}\PY{p}{,} \PY{p}{[}\PY{l+s+s2}{\PYZdq{}}\PY{l+s+s2}{dig }\PY{l+s+si}{\PYZob{}\PYZcb{}}\PY{l+s+s2}{: }\PY{l+s+si}{\PYZob{}:.6f\PYZcb{}}\PY{l+s+s2}{\PYZdq{}}\PY{o}{.}\PY{n}{format}\PY{p}{(}\PY{n}{i}\PY{p}{,} \PY{n}{p}\PY{p}{)} 
                     \PY{k}{for} \PY{n}{i}\PY{p}{,} \PY{n}{p} \PY{o+ow}{in} \PY{n+nb}{enumerate}\PY{p}{(}\PY{n}{predictions}\PY{p}{[}\PY{l+m+mi}{8}\PY{p}{]}\PY{p}{)}\PY{p}{]}\PY{p}{)}

9th digit: ['dig 0: 0.000023', 'dig 1: 0.000000', 'dig 2: 0.000019', 
            'dig 3: 0.000000', 'dig 4: 0.000001', 'dig 5: 0.931519', 
            'dig 6: 0.047921', 'dig 7: 0.000000', 'dig 8: 0.020517', 
            'dig 9: 0.000000']
\end{Verbatim}
\end{tcolorbox}

So the second ranked digit is a 6 (which can be confused with a five if
the lower loop is almost closed). I am not sure how a 5 could be
confused with a 8 though.

To see how well our NN behaves with different kind of digits we will try
to check how it works with my calligraphy (as homework try to repeat the
exercise using your own digit following the instructions given below).

\begin{itemize}
	\tightlist
	\item
	Open \(\tt{paint}\) and create a 280x280 white square
	\item
	Change brush type and set the maximum size
	\item
	With the mouse draw a digit
	\item
	Finally save the file (e.g. five.png)
\end{itemize}

Before passing the image to the NN it has to be resized and this is done
with an ad-hoc function (\(\tt{transform\_image}\)) which is in the
\(\tt{digit\_converter.py}\) module.

\begin{tcolorbox}[breakable, size=fbox, boxrule=1pt, pad at break*=1mm,colback=cellbackground, colframe=cellborder]
\begin{Verbatim}[commandchars=\\\{\}]
\PY{k+kn}{from} \PY{n+nn}{digit\PYZus{}converter} \PY{k}{import} \PY{n}{transform\PYZus{}image}
		
\PY{n}{filenames} \PY{o}{=} \PY{p}{[}\PY{l+s+s1}{\PYZsq{}}\PY{l+s+s1}{four.png}\PY{l+s+s1}{\PYZsq{}}\PY{p}{,} \PY{l+s+s1}{\PYZsq{}}\PY{l+s+s1}{five.png}\PY{l+s+s1}{\PYZsq{}}\PY{p}{]}
		
\PY{k}{for} \PY{n}{f} \PY{o+ow}{in} \PY{n}{filenames}\PY{p}{:}
\PY{n}{test\PYZus{}images} \PY{o}{=} \PY{n}{np}\PY{o}{.}\PY{n}{array}\PY{p}{(}\PY{n}{transform\PYZus{}image}\PY{p}{(}\PY{n}{f}\PY{p}{)}\PY{p}{)}
\PY{n}{test\PYZus{}images} \PY{o}{=} \PY{n}{np}\PY{o}{.}\PY{n}{expand\PYZus{}dims}\PY{p}{(}\PY{n}{test\PYZus{}images}\PY{p}{,} \PY{n}{axis}\PY{o}{=}\PY{l+m+mi}{3}\PY{p}{)}
		
\PY{n}{predict} \PY{o}{=} \PY{n}{trainer}\PY{o}{.}\PY{n}{predict}\PY{p}{(}\PY{n}{test\PYZus{}images}\PY{p}{)}
\PY{n+nb}{print} \PY{p}{(}\PY{l+s+s2}{\PYZdq{}}\PY{l+s+se}{\PYZbs{}n}\PY{l+s+s2}{\PYZdq{}}\PY{p}{)}
\PY{n+nb}{print} \PY{p}{(}\PY{l+s+s2}{\PYZdq{}}\PY{l+s+s2}{Tesing on custom digits...}\PY{l+s+s2}{\PYZdq{}}\PY{p}{)}
\PY{n+nb}{print} \PY{p}{(}\PY{l+s+s2}{\PYZdq{}}\PY{l+s+s2}{Predicted: }\PY{l+s+s2}{\PYZdq{}}\PY{p}{,} \PY{n}{np}\PY{o}{.}\PY{n}{argmax}\PY{p}{(}\PY{n}{predict}\PY{p}{,} \PY{n}{axis}\PY{o}{=}\PY{l+m+mi}{1}\PY{p}{)}\PY{p}{)}
\PY{n+nb}{print}\PY{p}{(}\PY{l+s+s2}{\PYZdq{}}\PY{l+s+s2}{\PYZpc{}}\PY{l+s+s2}{:}\PY{l+s+s2}{\PYZdq{}}\PY{p}{,} \PY{p}{[}\PY{l+s+s2}{\PYZdq{}}\PY{l+s+si}{\PYZob{}:.3f\PYZcb{}}\PY{l+s+s2}{\PYZdq{}}\PY{o}{.}\PY{n}{format}\PY{p}{(}\PY{n}{p}\PY{p}{[}\PY{n}{np}\PY{o}{.}\PY{n}{argmax}\PY{p}{(}\PY{n}{p}\PY{p}{)}\PY{p}{]}\PY{p}{)} \PY{k}{for} \PY{n}{p} \PY{o+ow}{in} \PY{n}{predict}\PY{p}{]}\PY{p}{)}
	
Tesing on custom digits{\ldots}
Predicted:  [4]
\%: ['1.000']
		
Tesing on custom digits{\ldots}
Predicted:  [5]
\%: ['0.991']
\end{Verbatim}
\end{tcolorbox}

Those the images I have checked

\begin{figure}[htb]
	\centering
	\includegraphics[width=0.2\textwidth]{figures/four.png}
	\includegraphics[width=0.2\textwidth]{figures/five.png}
\end{figure}


\subsection{Model Calibration cont.}\label{model-calibration-cont.}

When the parameter(s) of our model we need to calibrate can be expressed
as a function of three variables the CNN can be used.

Consider again the Black and Scholes formula for the call options.
Assume you need to calibrate the rate \(r\) and the volatility \(\sigma\).
A convolutiona neural network can be trained fed with special images
which represents \(ttm, K\) and \(P_\textrm{call}\).

A black-white image indeed can be interpreted as a map where each pixel
is a pair (\(ttm, K)\) and the pixel color, an integer between 0 (black)
and 255 (white), represents \(P_\textrm{call}\). As in the previous
examples the neural network was classifying the pictures into digits, now
it will assign them to classes identified by \(r, \sigma\) pairs.

The creation of the training sample is a little more complicated now.
For convenience we will use also a new format to save data image,
\(\tt{json}\). This will be done through the corresponding module simply
using the functions \(\tt{dump}\) and \(\tt{load}\) to store and
retrieve data. The module \(\tt{PIL}\) (pillow) is instead used to
visualize the images.

\begin{tcolorbox}[breakable, size=fbox, boxrule=1pt, pad at break*=1mm,colback=cellbackground, colframe=cellborder]
\begin{Verbatim}[commandchars=\\\{\}]
\PY{k+kn}{import} \PY{n+nn}{numpy} \PY{k}{as} \PY{n+nn}{np}\PY{o}{,} \PY{n+nn}{json}
\PY{k+kn}{import} \PY{n+nn}{pandas} \PY{k}{as} \PY{n+nn}{pd}
		
\PY{n}{labels} \PY{o}{=} \PY{p}{[}\PY{p}{]}
\PY{n}{rates} \PY{o}{=} \PY{n}{np}\PY{o}{.}\PY{n}{arange}\PY{p}{(}\PY{l+m+mf}{0.01}\PY{p}{,} \PY{l+m+mf}{0.11}\PY{p}{,} \PY{l+m+mf}{0.001}\PY{p}{)}
\PY{n}{vols} \PY{o}{=} \PY{n}{np}\PY{o}{.}\PY{n}{arange}\PY{p}{(}\PY{l+m+mf}{0.1}\PY{p}{,} \PY{l+m+mf}{0.6}\PY{p}{,} \PY{l+m+mf}{0.005}\PY{p}{)}
\PY{k}{for} \PY{n}{i} \PY{o+ow}{in} \PY{n+nb}{range}\PY{p}{(}\PY{n+nb}{len}\PY{p}{(}\PY{n}{vols}\PY{p}{)}\PY{p}{)}\PY{p}{:}
    \PY{k}{for} \PY{n}{j} \PY{o+ow}{in} \PY{n+nb}{range}\PY{p}{(}\PY{n+nb}{len}\PY{p}{(}\PY{n}{rates}\PY{p}{)}\PY{p}{)}\PY{p}{:}
        \PY{n}{labels}\PY{o}{.}\PY{n}{append}\PY{p}{(}\PY{p}{(}\PY{n}{vols}\PY{p}{[}\PY{n}{i}\PY{p}{]}\PY{p}{,} \PY{n}{rates}\PY{p}{[}\PY{n}{j}\PY{p}{]}\PY{p}{)}\PY{p}{)} 
		
\PY{n}{json}\PY{o}{.}\PY{n}{dump}\PY{p}{(}\PY{n}{labels}\PY{p}{,} \PY{n+nb}{open}\PY{p}{(}\PY{l+s+s2}{\PYZdq{}}\PY{l+s+s2}{2d\PYZus{}label.json}\PY{l+s+s2}{\PYZdq{}}\PY{p}{,} \PY{l+s+s2}{\PYZdq{}}\PY{l+s+s2}{w}\PY{l+s+s2}{\PYZdq{}}\PY{p}{)}\PY{p}{)}
		
\PY{n}{k} \PY{o}{=} \PY{n}{np}\PY{o}{.}\PY{n}{arange}\PY{p}{(}\PY{l+m+mf}{0.8}\PY{p}{,} \PY{l+m+mf}{1.2}\PY{p}{,} \PY{p}{(}\PY{l+m+mf}{1.2}\PY{o}{\PYZhy{}}\PY{l+m+mf}{0.8}\PY{p}{)}\PY{o}{/}\PY{l+m+mi}{20}\PY{p}{)}
\PY{n}{ttm} \PY{o}{=} \PY{n}{np}\PY{o}{.}\PY{n}{arange}\PY{p}{(}\PY{l+m+mi}{1}\PY{p}{,} \PY{l+m+mi}{5}\PY{p}{,} \PY{l+m+mi}{4}\PY{o}{/}\PY{l+m+mi}{20}\PY{p}{)}
		
\PY{c+c1}{\PYZsh{} for each r, sigma pair}
\PY{c+c1}{\PYZsh{} generate a matrix of prices}
\PY{n}{maximum} \PY{o}{=} \PY{l+m+mi}{0}
\PY{n}{minimum} \PY{o}{=} \PY{n}{np}\PY{o}{.}\PY{n}{inf}
\PY{n}{prices} \PY{o}{=} \PY{p}{[}\PY{p}{]}
\PY{k}{for} \PY{n}{v} \PY{o+ow}{in} \PY{n}{vols}\PY{p}{:}
    \PY{k}{for} \PY{n}{r} \PY{o+ow}{in} \PY{n}{rates}\PY{p}{:}
        \PY{n}{price} \PY{o}{=}  \PY{n}{np}\PY{o}{.}\PY{n}{zeros}\PY{p}{(}\PY{n}{shape}\PY{o}{=}\PY{p}{(}\PY{l+m+mi}{20}\PY{p}{,} \PY{l+m+mi}{20}\PY{p}{)}\PY{p}{)}
        \PY{k}{for} \PY{n}{ik}\PY{p}{,} \PY{n}{kv} \PY{o+ow}{in} \PY{n+nb}{enumerate}\PY{p}{(}\PY{n}{k}\PY{p}{)}\PY{p}{:}
            \PY{k}{for} \PY{n}{it}\PY{p}{,} \PY{n}{t} \PY{o+ow}{in} \PY{n+nb}{enumerate}\PY{p}{(}\PY{n}{ttm}\PY{p}{)}\PY{p}{:}
                \PY{n}{price}\PY{p}{[}\PY{n}{ik}\PY{p}{,} \PY{n}{it}\PY{p}{]} \PY{o}{=} \PY{n}{call}\PY{p}{(}\PY{n}{kv}\PY{p}{,} \PY{n}{r}\PY{p}{,} \PY{n}{v}\PY{p}{,} \PY{n}{t}\PY{p}{)}
                \PY{n}{prices}\PY{o}{.}\PY{n}{append}\PY{p}{(}\PY{n}{price}\PY{p}{)}
\PY{c+c1}{\PYZsh{} max and min are saved to }
\PY{c+c1}{\PYZsh{} normalize our matrices}
\PY{n}{new\PYZus{}max} \PY{o}{=} \PY{n}{np}\PY{o}{.}\PY{n}{max}\PY{p}{(}\PY{n}{price}\PY{p}{)}
\PY{n}{new\PYZus{}min} \PY{o}{=} \PY{n}{np}\PY{o}{.}\PY{n}{min}\PY{p}{(}\PY{n}{price}\PY{p}{)}
\PY{k}{if} \PY{n}{new\PYZus{}max} \PY{o}{\PYZgt{}} \PY{n}{maximum}\PY{p}{:}
    \PY{n}{maximum} \PY{o}{=} \PY{n}{new\PYZus{}max}
\PY{k}{if} \PY{n}{new\PYZus{}min} \PY{o}{\PYZlt{}} \PY{n}{minimum}\PY{p}{:}
    \PY{n}{minimum} \PY{o}{=} \PY{n}{new\PYZus{}min}
		
\PY{c+c1}{\PYZsh{} normalize each matrix in the range 0, 255}
\PY{k}{for} \PY{n}{ip}\PY{p}{,} \PY{n}{p} \PY{o+ow}{in} \PY{n+nb}{enumerate}\PY{p}{(}\PY{n}{prices}\PY{p}{)}\PY{p}{:}
    \PY{n}{prices}\PY{p}{[}\PY{n}{ip}\PY{p}{]} \PY{o}{=} \PY{n}{np}\PY{o}{.}\PY{n}{interp}\PY{p}{(}\PY{n}{p}\PY{p}{,} \PY{p}{(}\PY{n}{minimum}\PY{p}{,} \PY{n}{maximum}\PY{p}{)}\PY{p}{,} \PY{p}{(}\PY{l+m+mi}{0}\PY{p}{,} \PY{l+m+mi}{255}\PY{p}{)}\PY{p}{)} 
		
\PY{n}{json}\PY{o}{.}\PY{n}{dump}\PY{p}{(}\PY{n}{prices}\PY{p}{,} \PY{n+nb}{open}\PY{p}{(}\PY{l+s+s2}{\PYZdq{}}\PY{l+s+s2}{2d.json}\PY{l+s+s2}{\PYZdq{}}\PY{p}{,} \PY{l+s+s2}{\PYZdq{}}\PY{l+s+s2}{w}\PY{l+s+s2}{\PYZdq{}}\PY{p}{)}\PY{p}{)}
\end{Verbatim}
\end{tcolorbox}

Below an example of the 20x20 images that have been created
\begin{figure}[htb]
\centering
\includegraphics[width=0.4\textwidth]{figures/2d_training_images}
\end{figure}

Then the training is similar to what has been done for the handwritten
digits.

\begin{tcolorbox}[breakable, size=fbox, boxrule=1pt, pad at break*=1mm,colback=cellbackground, colframe=cellborder]
\begin{Verbatim}[commandchars=\\\{\}]
\PY{k+kn}{import} \PY{n+nn}{numpy} \PY{k}{as} \PY{n+nn}{np}\PY{o}{,} \PY{n+nn}{json}
\PY{k+kn}{from} \PY{n+nn}{finnn} \PY{k}{import} \PY{n}{FinNN}
		
\PY{n}{labels} \PY{o}{=} \PY{n}{np}\PY{o}{.}\PY{n}{array}\PY{p}{(}\PY{n}{json}\PY{o}{.}\PY{n}{load}\PY{p}{(}\PY{n+nb}{open}\PY{p}{(}\PY{l+s+s2}{\PYZdq{}}\PY{l+s+s2}{2d\PYZus{}label.json}\PY{l+s+s2}{\PYZdq{}}\PY{p}{)}\PY{p}{)}\PY{p}{)}
\PY{n}{images} \PY{o}{=} \PY{n}{np}\PY{o}{.}\PY{n}{array}\PY{p}{(}\PY{n}{json}\PY{o}{.}\PY{n}{load}\PY{p}{(}\PY{n+nb}{open}\PY{p}{(}\PY{l+s+s2}{\PYZdq{}}\PY{l+s+s2}{2d.json}\PY{l+s+s2}{\PYZdq{}}\PY{p}{)}\PY{p}{)}\PY{p}{)}
\PY{n}{images} \PY{o}{=} \PY{p}{(}\PY{n}{images}\PY{o}{/}\PY{l+m+mf}{255.}\PY{p}{)} \PY{o}{\PYZhy{}} \PY{l+m+mf}{0.5}
		
\PY{n}{trainer} \PY{o}{=} \PY{n}{FinNN}\PY{p}{(}\PY{l+s+s2}{\PYZdq{}}\PY{l+s+s2}{CNN2D}\PY{l+s+s2}{\PYZdq{}}\PY{p}{)}
\PY{n}{trainer}\PY{o}{.}\PY{n}{setData}\PY{p}{(}\PY{n}{images}\PY{p}{,} \PY{n}{labels}\PY{p}{,} \PY{l+m+mf}{0.2}\PY{p}{)}
		
\PY{n}{trainer}\PY{o}{.}\PY{n}{addConv2DLayer}\PY{p}{(}\PY{l+m+mi}{8}\PY{p}{,} \PY{l+m+mi}{20}\PY{p}{,} \PY{n}{input\PYZus{}shape}\PY{o}{=}\PY{p}{(}\PY{l+m+mi}{20}\PY{p}{,} \PY{l+m+mi}{20}\PY{p}{,} \PY{l+m+mi}{1}\PY{p}{)}\PY{p}{,} \PY{n}{activation}\PY{o}{=}\PY{l+s+s1}{\PYZsq{}}\PY{l+s+s1}{relu}\PY{l+s+s1}{\PYZsq{}}\PY{p}{)}
\PY{n}{trainer}\PY{o}{.}\PY{n}{addFlatten}\PY{p}{(}\PY{p}{)}
\PY{n}{trainer}\PY{o}{.}\PY{n}{addHiddenLayer}\PY{p}{(}\PY{l+m+mi}{10}\PY{p}{,} \PY{n}{activation}\PY{o}{=}\PY{l+s+s1}{\PYZsq{}}\PY{l+s+s1}{relu}\PY{l+s+s1}{\PYZsq{}}\PY{p}{)}
\PY{n}{trainer}\PY{o}{.}\PY{n}{addOutputLayer}\PY{p}{(}\PY{l+m+mi}{2}\PY{p}{,} \PY{n}{activation}\PY{o}{=}\PY{l+s+s1}{\PYZsq{}}\PY{l+s+s1}{relu}\PY{l+s+s1}{\PYZsq{}}\PY{p}{,} \PY{n}{bias\PYZus{}initializer}\PY{o}{=}\PY{l+s+s1}{\PYZsq{}}\PY{l+s+s1}{random\PYZus{}uniform}\PY{l+s+s1}{\PYZsq{}}\PY{p}{)}
\PY{n}{trainer}\PY{o}{.}\PY{n}{compileModel}\PY{p}{(}\PY{l+s+s1}{\PYZsq{}}\PY{l+s+s1}{mse}\PY{l+s+s1}{\PYZsq{}}\PY{p}{,} \PY{l+s+s1}{\PYZsq{}}\PY{l+s+s1}{adam}\PY{l+s+s1}{\PYZsq{}}\PY{p}{,} \PY{n}{categorize}\PY{o}{=}\PY{k+kc}{False}\PY{p}{)}
		
\PY{n}{trainer}\PY{o}{.}\PY{n}{fit}\PY{p}{(}\PY{l+m+mi}{300}\PY{p}{,} \PY{l+m+mi}{32}\PY{p}{,} \PY{n}{verbose}\PY{o}{=}\PY{l+m+mi}{1}\PY{p}{)}
\PY{n}{trainer}\PY{o}{.}\PY{n}{saveModel}\PY{p}{(}\PY{l+s+s2}{\PYZdq{}}\PY{l+s+s2}{2d.b5}\PY{l+s+s2}{\PYZdq{}}\PY{p}{)}
		
\PY{n}{trainer}\PY{o}{.}\PY{n}{evaluate}\PY{p}{(}\PY{p}{)}

Epoch 1/300
8000/8000 [==============================] - 2s 306us/step - loss: 0.0084
Epoch 2/300
8000/8000 [==============================] - 1s 157us/step - loss: 0.0040
...
Epoch 299/300
8000/8000 [==============================] - 1s 139us/step - loss: 8.9151e-06
Epoch 300/300
8000/8000 [==============================] - 1s 139us/step - loss: 1.8579e-05

8000/8000 [==============================] - 0s 60us/step
Training: 8.98895842510683e-06
2000/2000 [==============================] - 0s 49us/step
Test: 9.32247715536505e-06
\end{Verbatim}
\end{tcolorbox}

At this point you should present to the trained CNN the prices of call
referring to the same underlying in the pictorial form shown before and
the in response it will give you the risk-free rate and the underlying
volatility.

\section{Technical Analysis}\label{technical-analysis}

In finance, \emph{technical analysis} is a security analysis discipline
for forecasting the direction of prices through the study of past market
data, primarily price and volume. Essentially the analyst looks for
particular patterns in the price time series that are \emph{known} to
develop in predictable ways to take profit of it.

\begin{figure}
	\centering
	\includegraphics[width=0.4\linewidth]{figures/H_and_s_top_new.jpg}
	
	\includegraphics[width=0.4\linewidth]{figures/Triangle-ascending.jpg}
	\caption{Examples of patterns in real time series.}
\end{figure}

As you may imagine we will try to develop a CNN (like in the handwriting
case) capable of classifying features in time series to be used in a
technical analysis (this is much faster than having somebody looking at
thousands of time series by eye\ldots{}).

I have generated myself the training set simulating 21600 time series
(1/3 with head and shoulder patter, 1/3 with triangle pattern and 1/3
with no pattern). \emph{To make the training easier the features have been exagerated.}

\begin{figure}
	\centering
	\includegraphics[width=0.6\linewidth]{figures/image_1.png}
	\caption{An example of training image with no pattern.}
\end{figure}

\begin{figure}
	\centering
	\includegraphics[width=0.6\linewidth]{figures/image_2.png}
	\caption{An example of training image with \emph{head and shoulder} pattern.}
\end{figure}

\begin{figure}
	\centering
	\includegraphics[width=0.6\linewidth]{figures/image_0.png}
	\caption{An example of training image with \emph{triangle} pattern.}
\end{figure}


\begin{tcolorbox}[breakable, size=fbox, boxrule=1pt, pad at break*=1mm,colback=cellbackground, colframe=cellborder]
\begin{Verbatim}[commandchars=\\\{\}]
\PY{k+kn}{from} \PY{n+nn}{finnn} \PY{k}{import} \PY{n}{FinNN}
\PY{k+kn}{import} \PY{n+nn}{numpy} \PY{k}{as} \PY{n+nn}{np}\PY{o}{,} \PY{n+nn}{json}
		
\PY{n}{train\PYZus{}labels} \PY{o}{=} \PY{n}{np}\PY{o}{.}\PY{n}{array}\PY{p}{(}\PY{n}{json}\PY{o}{.}\PY{n}{load}\PY{p}{(}\PY{n+nb}{open}\PY{p}{(}\PY{l+s+s2}{\PYZdq{}}\PY{l+s+s2}{training\PYZus{}techana\PYZus{}labels.json}\PY{l+s+s2}{\PYZdq{}}\PY{p}{)}\PY{p}{)}\PY{p}{)}
\PY{n}{train\PYZus{}images} \PY{o}{=} \PY{n}{np}\PY{o}{.}\PY{n}{array}\PY{p}{(}\PY{n}{json}\PY{o}{.}\PY{n}{load}\PY{p}{(}\PY{n+nb}{open}\PY{p}{(}\PY{l+s+s2}{\PYZdq{}}\PY{l+s+s2}{training\PYZus{}techana\PYZus{}images.json}\PY{l+s+s2}{\PYZdq{}}\PY{p}{)}\PY{p}{)}\PY{p}{)}
		
\PY{n}{trainer} \PY{o}{=} \PY{n}{FinNN}\PY{p}{(}\PY{l+s+s2}{\PYZdq{}}\PY{l+s+s2}{CNN1D}\PY{l+s+s2}{\PYZdq{}}\PY{p}{)}
\PY{n}{trainer}\PY{o}{.}\PY{n}{setData}\PY{p}{(}\PY{n}{train\PYZus{}images}\PY{p}{,} \PY{n}{train\PYZus{}labels}\PY{p}{,} \PY{l+m+mf}{0.2}\PY{p}{)}
\PY{n}{trainer}\PY{o}{.}\PY{n}{normalize}\PY{p}{(}\PY{p}{)}
		
\PY{n}{trainer}\PY{o}{.}\PY{n}{addConv1DInputLayer}\PY{p}{(}\PY{l+m+mi}{80}\PY{p}{,} \PY{l+m+mi}{20}\PY{p}{,} \PY{p}{(}\PY{l+m+mi}{101}\PY{p}{,} \PY{l+m+mi}{1}\PY{p}{)}\PY{p}{,} \PY{n}{activation}\PY{o}{=}\PY{l+s+s1}{\PYZsq{}}\PY{l+s+s1}{relu}\PY{l+s+s1}{\PYZsq{}}\PY{p}{)}
\PY{n}{trainer}\PY{o}{.}\PY{n}{addConv1DLayer}\PY{p}{(}\PY{l+m+mi}{80}\PY{p}{,} \PY{l+m+mi}{15}\PY{p}{,} \PY{n}{activation}\PY{o}{=}\PY{l+s+s1}{\PYZsq{}}\PY{l+s+s1}{relu}\PY{l+s+s1}{\PYZsq{}}\PY{p}{)}
\PY{n}{trainer}\PY{o}{.}\PY{n}{addMaxPooling1D}\PY{p}{(}\PY{l+m+mi}{3}\PY{p}{)}
\PY{n}{trainer}\PY{o}{.}\PY{n}{addConv1DLayer}\PY{p}{(}\PY{l+m+mi}{100}\PY{p}{,} \PY{l+m+mi}{10}\PY{p}{,} \PY{n}{activation}\PY{o}{=}\PY{l+s+s1}{\PYZsq{}}\PY{l+s+s1}{relu}\PY{l+s+s1}{\PYZsq{}}\PY{p}{)}
\PY{n}{trainer}\PY{o}{.}\PY{n}{addConv1DLayer}\PY{p}{(}\PY{l+m+mi}{100}\PY{p}{,} \PY{l+m+mi}{5}\PY{p}{,} \PY{n}{activation}\PY{o}{=}\PY{l+s+s1}{\PYZsq{}}\PY{l+s+s1}{relu}\PY{l+s+s1}{\PYZsq{}}\PY{p}{)}
\PY{n}{trainer}\PY{o}{.}\PY{n}{addGlobalAveragePooling1D}\PY{p}{(}\PY{p}{)}
\PY{n}{trainer}\PY{o}{.}\PY{n}{addDropout}\PY{p}{(}\PY{l+m+mf}{0.5}\PY{p}{)}
\PY{c+c1}{\PYZsh{}trainer.addFlatten()}
\PY{n}{trainer}\PY{o}{.}\PY{n}{addCNNOutputLayer}\PY{p}{(}\PY{l+m+mi}{3}\PY{p}{)}
		
\PY{n}{trainer}\PY{o}{.}\PY{n}{compileModel}\PY{p}{(}\PY{l+s+s1}{\PYZsq{}}\PY{l+s+s1}{categorical\PYZus{}crossentropy}\PY{l+s+s1}{\PYZsq{}}\PY{p}{,} \PY{l+s+s1}{\PYZsq{}}\PY{l+s+s1}{adam}\PY{l+s+s1}{\PYZsq{}}\PY{p}{)}
		
\PY{c+c1}{\PYZsh{} make the training}
\PY{n}{trainer}\PY{o}{.}\PY{n}{fit}\PY{p}{(}\PY{l+m+mi}{50}\PY{p}{,} \PY{l+m+mi}{5000}\PY{p}{)}
		
\PY{c+c1}{\PYZsh{}trainer.saveModel(\PYZsq{}techana.h5\PYZsq{})}
\end{Verbatim}
\end{tcolorbox}

\begin{tcolorbox}[breakable, size=fbox, boxrule=1pt, pad at break*=1mm,colback=cream, colframe=cellborder]
\subsubsection{Again for the Most		Curious}\label{again-for-the-most-curious}

Large neural nets trained on relatively small datasets can overfit the
training data.

This has the effect of the model learning the statistical noise in the
training data, which results in poor performance when the model is
evaluated on new data, e.g.~a test dataset.

One approach to reduce overfitting is to fit all possible different
neural networks on the same dataset and to average the predictions from
each model. This is not feasible in practice, and can be approximated
using a small collection of different models, called an ensemble. A
problem even with the ensemble approximation is that it requires
multiple models to be fit and stored, which can be a challenge if the
models are large, requiring days or weeks to train and tune.

\emph{Dropout} is a regularization method that approximates training a
large number of neural networks with different architectures in
parallel.

During training, some number of layer outputs are randomly ignored or
\emph{dropped out}. This has the effect of making the layer look-like
and be treated-like a layer with a different number of nodes and
connectivity to the prior layer. In effect, each update to a layer
during training is performed with a different ``view'' of the configured
layer.

Even if the it may seems counter intuitive (better training when
switching off nodes) indeed dropout breaks-up situations where network
layers co-adapt to correct mistakes from prior layers, in turn making
the model more robust.
\end{tcolorbox}

To test the performance I wanted to simulate a real case scenario where
the time series are analyzed in real-time in order to predict as soon as
possible a particular pattern and take advantage of the prediction.

To do so I have created a longer time series (i.e. more time points) and
passed as input to the CNN sliding time windows to simulate the
evolution of the time series. So if for example the time series is made
of 100 points, I presented to the network first the points between
\([0, 80]\), then \([1, 81]\), \([2, 82]\) and so on simulating new real
time data incoming. The goal was to check when the neural network was
capable of predicting the incoming pattern.

%\begin{center}
%	\adjustimage{max size={0.9\linewidth}{0.9\paperheight}}{output_46_0.png}
%\end{center}

\begin{tcolorbox}[breakable, size=fbox, boxrule=1pt, pad at break*=1mm,colback=cellbackground, colframe=cellborder]
\begin{Verbatim}[commandchars=\\\{\}]
\PY{k+kn}{import} \PY{n+nn}{numpy} \PY{k}{as} \PY{n+nn}{np}\PY{o}{,} \PY{n+nn}{json}
\PY{k+kn}{from} \PY{n+nn}{finnn} \PY{k}{import} \PY{n}{FinNN}
\PY{k+kn}{from} \PY{n+nn}{matplotlib} \PY{k}{import} \PY{n}{pyplot} \PY{k}{as} \PY{n}{plt}
		
\PY{n}{test\PYZus{}images} \PY{o}{=} \PY{n}{np}\PY{o}{.}\PY{n}{array}\PY{p}{(}\PY{n}{json}\PY{o}{.}\PY{n}{load}\PY{p}{(}\PY{n+nb}{open}\PY{p}{(}\PY{l+s+s2}{\PYZdq{}}\PY{l+s+s2}{testing\PYZus{}techana\PYZus{}frames.json}\PY{l+s+s2}{\PYZdq{}}\PY{p}{)}\PY{p}{)}\PY{p}{)}
		
\PY{n}{plt}\PY{o}{.}\PY{n}{figure}\PY{p}{(}\PY{n}{figsize}\PY{o}{=}\PY{p}{(}\PY{l+m+mi}{15}\PY{p}{,} \PY{l+m+mi}{6}\PY{p}{)}\PY{p}{)}
\PY{k}{for} \PY{n}{i} \PY{o+ow}{in} \PY{n+nb}{range}\PY{p}{(}\PY{n}{test\PYZus{}images}\PY{o}{.}\PY{n}{shape}\PY{p}{[}\PY{l+m+mi}{0}\PY{p}{]}\PY{p}{)}\PY{p}{:}
    \PY{n}{sub} \PY{o}{=} \PY{n}{plt}\PY{o}{.}\PY{n}{subplot}\PY{p}{(}\PY{l+m+mi}{2}\PY{p}{,} \PY{l+m+mi}{5}\PY{p}{,} \PY{n}{i}\PY{o}{+}\PY{l+m+mi}{1}\PY{p}{)}
    \PY{n}{sub}\PY{o}{.}\PY{n}{plot}\PY{p}{(}\PY{n}{test\PYZus{}images}\PY{p}{[}\PY{n}{i}\PY{p}{,} \PY{p}{:}\PY{p}{]}\PY{p}{)}
    \PY{n}{plt}\PY{o}{.}\PY{n}{show}\PY{p}{(}\PY{p}{)}
		
\PY{n}{trainer} \PY{o}{=} \PY{n}{FinNN}\PY{p}{(}\PY{l+s+s2}{\PYZdq{}}\PY{l+s+s2}{CNN1D}\PY{l+s+s2}{\PYZdq{}}\PY{p}{)}
\PY{n}{trainer}\PY{o}{.}\PY{n}{loadModel}\PY{p}{(}\PY{l+s+s2}{\PYZdq{}}\PY{l+s+s2}{techana}\PY{l+s+s2}{\PYZdq{}}\PY{p}{)}
\PY{n}{test\PYZus{}images} \PY{o}{=} \PY{n}{np}\PY{o}{.}\PY{n}{expand\PYZus{}dims}\PY{p}{(}\PY{n}{test\PYZus{}images}\PY{p}{,} \PY{n}{axis}\PY{o}{=}\PY{l+m+mi}{3}\PY{p}{)}
		
\PY{n}{predictions} \PY{o}{=} \PY{n}{trainer}\PY{o}{.}\PY{n}{predict}\PY{p}{(}\PY{n}{test\PYZus{}images}\PY{p}{)}
\PY{k}{for} \PY{n}{i} \PY{o+ow}{in} \PY{n+nb}{range}\PY{p}{(}\PY{n+nb}{len}\PY{p}{(}\PY{n}{predictions}\PY{p}{)}\PY{p}{)}\PY{p}{:}
    \PY{n+nb}{print} \PY{p}{(}\PY{n}{np}\PY{o}{.}\PY{n}{argmax}\PY{p}{(}\PY{n}{predictions}\PY{p}{[}\PY{n}{i}\PY{p}{]}\PY{p}{)}\PY{p}{,} \PY{p}{[}\PY{l+s+s2}{\PYZdq{}}\PY{l+s+si}{\PYZob{}:.2f\PYZcb{}}\PY{l+s+s2}{\PYZdq{}}\PY{o}{.}\PY{n}{format}\PY{p}{(}\PY{n}{p}\PY{p}{)} \PY{k}{for} \PY{n}{p} \PY{o+ow}{in} \PY{n}{predictions}\PY{p}{[}\PY{n}{i}\PY{p}{]}\PY{p}{]}\PY{p}{)}	

0 ['0.77', '0.00', '0.23']
0 ['0.71', '0.00', '0.29']
0 ['0.66', '0.00', '0.34']
0 ['0.82', '0.00', '0.18']
0 ['0.91', '0.00', '0.09']
2 ['0.40', '0.06', '0.54']
1 ['0.00', '1.00', '0.00']
1 ['0.00', '1.00', '0.00']
1 ['0.00', '1.00', '0.00']
1 ['0.00', '1.00', '0.00']
\end{Verbatim}
\end{tcolorbox}

So at the 6th sample the CNN start recognizing the \emph{head and shoulder} pattern in the price evolution.

Epoch 198/200
- 23s - loss: 0.0693 - accuracy: 0.9722
Epoch 199/200
- 23s - loss: 0.0737 - accuracy: 0.9701
Epoch 200/200
- 23s - loss: 0.0692 - accuracy: 0.9725


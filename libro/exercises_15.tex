\chapter{Machine Learning}\label{introduction-to-python---lesson-15}

\begin{Exercise}
In order to see how different parameter choices affect the training
(both looking at a plot like the one before and the the \(MSE\)) try to:

\begin{itemize}
\tightlist
\item
  reduce the number of points used in the training (change the step from
  0.1 to 1 or to 0.01 in \(\tt{x = arange(-50, 51, 0.1))}\), expect
  worse results with less points;
\item
  change the number of nodes per layer;
\item
  change the activation function from `sigmoid' to `relu';
\item
  change the number of epochs, this is the number of times the neural
  network will process the sample data to improve the training; setting
  verbose to 1 will show the progress with an estimate of the goodness
  of the training after each epoch; expect worse training with less
  epoch.
\end{itemize}
\end{Exercise}
\begin{Answer}
\end{Answer}

\begin{Exercise}
To see how well our NN behaves with different kind of digits we will try
to check how it works with my calligraphy (as homework try to repeat the
exercise using your own digit following the instructions given below).

\begin{itemize}
\tightlist
\item
  Open \texttt{paint} and create a 280x280 white square
\item
  Change brush type and set the maximum size
\item
  With the mouse draw a digit
\item
  Finally save the file (e.g.~five.png)
\end{itemize}

Before passing the image to the NN it has to be resized and this is done
with an ad-hoc function (\texttt{transform\_image}) which is in the
\texttt{digit\_converter.py} module.
\end{Exercise}
\begin{Answer}
\end{Answer}

\begin{Exercise}
Taking as example the pricing NN trained on call, try to price put
options.
\end{Exercise}
\begin{Answer}
\end{Answer}

Update \verb!finmarkets! module with \verb!CreditCurve! and \verb!CreditDefaultSwap! classes.
\end{Exercise}
\begin{Answer}
\end{Answer}

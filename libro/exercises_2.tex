\chapter{Data Containers}\label{introduction-to-python---lesson-2}

\begin{Exercise}
What is a dictionary in \(\tt{python}\) programming ? Create a dictionary, modify it and then print all its items.
\end{Exercise}

\begin{Answer}
A dictionary is a container that maps a key (any object) to a value (any
object), contrary to lists which map an integer (the index) to a value
(any object).

\begin{tcolorbox}[size=fbox, boxrule=1pt, colback=cellbackground, colframe=cellborder]
\begin{Verbatim}[commandchars=\\\{\}]
\PY{n}{dictionary} \PY{o}{=} \PY{p}{\PYZob{}}\PY{l+s+s2}{\PYZdq{}}\PY{l+s+s2}{calculus}\PY{l+s+s2}{\PYZdq{}}\PY{p}{:}\PY{l+m+mi}{28}\PY{p}{,} \PY{l+s+s2}{\PYZdq{}}\PY{l+s+s2}{physics}\PY{l+s+s2}{\PYZdq{}}\PY{p}{:}\PY{l+m+mi}{30}\PY{p}{,} \PY{l+s+s2}{\PYZdq{}}\PY{l+s+s2}{chemistry}\PY{l+s+s2}{\PYZdq{}}\PY{p}{:}\PY{l+m+mi}{25}\PY{p}{\PYZcb{}}

\PY{n}{dictionary}\PY{p}{[}\PY{l+s+s2}{\PYZdq{}}\PY{l+s+s2}{laboratory}\PY{l+s+s2}{\PYZdq{}}\PY{p}{]} \PY{o}{=} \PY{l+m+mi}{27}
\PY{n}{dictionary}\PY{p}{[}\PY{l+s+s2}{\PYZdq{}}\PY{l+s+s2}{chemistry}\PY{l+s+s2}{\PYZdq{}}\PY{p}{]} \PY{o}{=} \PY{l+m+mi}{24}

\PY{n+nb}{print} \PY{p}{(}\PY{l+s+s2}{\PYZdq{}}\PY{l+s+s2}{Exam}\PY{l+s+se}{\PYZbs{}t}\PY{l+s+se}{\PYZbs{}t}\PY{l+s+s2}{Vote}\PY{l+s+s2}{\PYZdq{}}\PY{p}{)}
\PY{k}{for} \PY{n}{k}\PY{p}{,} \PY{n}{v} \PY{o+ow}{in} \PY{n}{dictionary}\PY{o}{.}\PY{n}{items}\PY{p}{(}\PY{p}{)}\PY{p}{:}
    \PY{n+nb}{print} \PY{p}{(}\PY{l+s+s2}{\PYZdq{}}\PY{l+s+si}{\PYZob{}\PYZcb{}}\PY{l+s+s2}{:}\PY{l+s+se}{\PYZbs{}t}\PY{l+s+si}{\PYZob{}\PYZcb{}}\PY{l+s+s2}{\PYZdq{}}\PY{o}{.}\PY{n}{format}\PY{p}{(}\PY{n}{k}\PY{p}{,} \PY{n}{v}\PY{p}{)}\PY{p}{)}

Exam            Vote
calculus:       28
physics:        30
chemistry:      24
laboratory:     27
\end{Verbatim}
\end{tcolorbox}
\end{Answer}

\begin{Exercise}
Write code which, given the following list

\begin{Shaded}
\begin{Highlighting}[]
\NormalTok{input_list }\OperatorTok{=}\NormalTok{ [}\DecValTok{3}\NormalTok{, }\DecValTok{5}\NormalTok{, }\DecValTok{2}\NormalTok{, }\DecValTok{1}\NormalTok{, }\DecValTok{13}\NormalTok{, }\DecValTok{5}\NormalTok{, }\DecValTok{5}\NormalTok{, }\DecValTok{1}\NormalTok{, }\DecValTok{3}\NormalTok{, }\DecValTok{4}\NormalTok{]}
\end{Highlighting}
\end{Shaded}
prints out the indices of every occurrence of
\begin{Shaded}
\begin{Highlighting}[]
\NormalTok{y }\OperatorTok{=} \DecValTok{5}
\end{Highlighting}
\end{Shaded}
\end{Exercise}

\begin{Answer}
\begin{tcolorbox}[size=fbox, boxrule=1pt, colback=cellbackground, colframe=cellborder]
\begin{Verbatim}[commandchars=\\\{\}]
\PY{n}{l} \PY{o}{=} \PY{p}{[}\PY{l+m+mi}{3}\PY{p}{,} \PY{l+m+mi}{5}\PY{p}{,} \PY{l+m+mi}{2}\PY{p}{,} \PY{l+m+mi}{1}\PY{p}{,} \PY{l+m+mi}{13}\PY{p}{,} \PY{l+m+mi}{5}\PY{p}{,} \PY{l+m+mi}{5}\PY{p}{,} \PY{l+m+mi}{1}\PY{p}{,} \PY{l+m+mi}{3}\PY{p}{,} \PY{l+m+mi}{4}\PY{p}{]}

\PY{k}{for} \PY{n}{i} \PY{o+ow}{in} \PY{n+nb}{range}\PY{p}{(}\PY{n+nb}{len}\PY{p}{(}\PY{n}{l}\PY{p}{)}\PY{p}{)}\PY{p}{:}
    \PY{k}{if} \PY{n}{l}\PY{p}{[}\PY{n}{i}\PY{p}{]} \PY{o}{==} \PY{l+m+mi}{5}\PY{p}{:}
        \PY{n+nb}{print} \PY{p}{(}\PY{n}{i}\PY{p}{)}

1
5
6
\end{Verbatim}
\end{tcolorbox}

Note that lists already have a way to get the occurrences of an item: \texttt{l.count(5)} would have done the job.
\end{Answer}

\begin{Exercise}
Write a \texttt{python} program to convert a list of tuples into a dictionary where the keys are the first elements of each tuples and the values the second.
Input:
\begin{Shaded}
\begin{Highlighting}[]
\NormalTok{l }\OperatorTok{=}\NormalTok{ [(}\StringTok{"x"}\NormalTok{, }\DecValTok{1}\NormalTok{), (}\StringTok{"x"}\NormalTok{, }\DecValTok{2}\NormalTok{), (}\StringTok{"x"}\NormalTok{, }\DecValTok{3}\NormalTok{), (}\StringTok{"y"}\NormalTok{, }\DecValTok{1}\NormalTok{), (}\StringTok{"y"}\NormalTok{, }\DecValTok{2}\NormalTok{), (}\StringTok{"z"}\NormalTok{, }\DecValTok{1}\NormalTok{)]}
\end{Highlighting}
\end{Shaded}
\end{Exercise}

\begin{Answer}
\begin{tcolorbox}[size=fbox, boxrule=1pt, colback=cellbackground, colframe=cellborder]
\begin{Verbatim}[commandchars=\\\{\}]
\PY{n}{l} \PY{o}{=} \PY{p}{[}\PY{p}{(}\PY{l+s+s2}{\PYZdq{}}\PY{l+s+s2}{x}\PY{l+s+s2}{\PYZdq{}}\PY{p}{,} \PY{l+m+mi}{1}\PY{p}{)}\PY{p}{,} \PY{p}{(}\PY{l+s+s2}{\PYZdq{}}\PY{l+s+s2}{x}\PY{l+s+s2}{\PYZdq{}}\PY{p}{,} \PY{l+m+mi}{2}\PY{p}{)}\PY{p}{,} \PY{p}{(}\PY{l+s+s2}{\PYZdq{}}\PY{l+s+s2}{x}\PY{l+s+s2}{\PYZdq{}}\PY{p}{,} \PY{l+m+mi}{3}\PY{p}{)}\PY{p}{,} \PY{p}{(}\PY{l+s+s2}{\PYZdq{}}\PY{l+s+s2}{y}\PY{l+s+s2}{\PYZdq{}}\PY{p}{,} \PY{l+m+mi}{1}\PY{p}{)}\PY{p}{,} \PY{p}{(}\PY{l+s+s2}{\PYZdq{}}\PY{l+s+s2}{y}\PY{l+s+s2}{\PYZdq{}}\PY{p}{,} \PY{l+m+mi}{2}\PY{p}{)}\PY{p}{,} \PY{p}{(}\PY{l+s+s2}{\PYZdq{}}\PY{l+s+s2}{z}\PY{l+s+s2}{\PYZdq{}}\PY{p}{,} \PY{l+m+mi}{1}\PY{p}{)}\PY{p}{]}

\PY{n}{d} \PY{o}{=} \PY{p}{\PYZob{}}\PY{p}{\PYZcb{}}
\PY{k}{for} \PY{n}{item} \PY{o+ow}{in} \PY{n}{l}\PY{p}{:}
    \PY{n}{d}\PY{p}{[}\PY{n}{item}\PY{p}{[}\PY{l+m+mi}{0}\PY{p}{]}\PY{p}{]} \PY{o}{=} \PY{n}{item}\PY{p}{[}\PY{l+m+mi}{1}\PY{p}{]}
    
\PY{n+nb}{print} \PY{p}{(}\PY{n}{d}\PY{p}{)}

\{'x': 3, 'y': 2, 'z': 1\}
\end{Verbatim}
\end{tcolorbox}

Note that there is just one occurrence of the key $\tt{x}$ and $\tt{y}$ because keys has to be unique and setting the same key to a different value simply overwrite the existing entry.
\end{Answer}

\begin{Exercise}
Write a \texttt{python} program to replace the last value of each tuples in a list.
Input:
\begin{Shaded}
\begin{Highlighting}[]
\NormalTok{l }\OperatorTok{=}\NormalTok{ [(}\DecValTok{10}\NormalTok{, }\DecValTok{20}\NormalTok{, }\DecValTok{40}\NormalTok{), (}\DecValTok{40}\NormalTok{, }\DecValTok{50}\NormalTok{, }\DecValTok{60}\NormalTok{), (}\DecValTok{70}\NormalTok{, }\DecValTok{80}\NormalTok{, }\DecValTok{90}\NormalTok{)]}
\end{Highlighting}
\end{Shaded}
\end{Exercise}

\begin{Answer}
\begin{tcolorbox}[size=fbox, boxrule=1pt, colback=cellbackground, colframe=cellborder]
\begin{Verbatim}[commandchars=\\\{\}]
\PY{n}{l} \PY{o}{=} \PY{p}{[}\PY{p}{(}\PY{l+m+mi}{10}\PY{p}{,} \PY{l+m+mi}{20}\PY{p}{,} \PY{l+m+mi}{40}\PY{p}{)}\PY{p}{,} \PY{p}{(}\PY{l+m+mi}{40}\PY{p}{,} \PY{l+m+mi}{50}\PY{p}{,} \PY{l+m+mi}{60}\PY{p}{)}\PY{p}{,} \PY{p}{(}\PY{l+m+mi}{70}\PY{p}{,} \PY{l+m+mi}{80}\PY{p}{,} \PY{l+m+mi}{90}\PY{p}{)}\PY{p}{]}

\PY{k}{for} \PY{n}{i} \PY{o+ow}{in} \PY{n+nb}{range}\PY{p}{(}\PY{n+nb}{len}\PY{p}{(}\PY{n}{l}\PY{p}{)}\PY{p}{)}\PY{p}{:}
    \PY{n}{new\PYZus{}tuple} \PY{o}{=} \PY{n}{l}\PY{p}{[}\PY{n}{i}\PY{p}{]}\PY{p}{[}\PY{l+m+mi}{0}\PY{p}{:}\PY{l+m+mi}{2}\PY{p}{]} \PY{o}{+} \PY{p}{(}\PY{n}{l}\PY{p}{[}\PY{n}{i}\PY{p}{]}\PY{p}{[}\PY{l+m+mi}{2}\PY{p}{]} \PY{o}{+} \PY{l+m+mi}{10}\PY{p}{,}\PY{p}{)}
    \PY{n}{l}\PY{p}{[}\PY{n}{i}\PY{p}{]} \PY{o}{=} \PY{n}{new\PYZus{}tuple}
    
\PY{n+nb}{print} \PY{p}{(}\PY{n}{l}\PY{p}{)}

[(10, 20, 50), (40, 50, 70), (70, 80, 100)]
\end{Verbatim}
\end{tcolorbox} 
\end{Answer}

\begin{Exercise}
Write a \texttt{python} program to count the elements in a list until an element is a tuple.
Input:
\begin{Shaded}
\begin{Highlighting}[]
\NormalTok{\{[}\DecValTok{1}\NormalTok{, }\DecValTok{5}\NormalTok{, 'a}\StringTok{'}\NormalTok{, (}\DecValTok{1}\NormalTok{,}\DecValTok{2}\NormalTok{), \{'test}\StringTok{'}\NormalTok{:}\DecValTok{1}\NormalTok{\}]\}}
\end{Highlighting}
\end{Shaded}
\end{Exercise}

\begin{Answer}
\begin{tcolorbox}[size=fbox, boxrule=1pt, colback=cellbackground, colframe=cellborder]
\begin{Verbatim}[commandchars=\\\{\}]
\PY{n}{l} \PY{o}{=} \PY{p}{[}\PY{l+m+mi}{1}\PY{p}{,} \PY{l+m+mi}{5}\PY{p}{,} \PY{l+s+s2}{\PYZdq{}}\PY{l+s+s2}{a}\PY{l+s+s2}{\PYZdq{}}\PY{p}{,} \PY{p}{(}\PY{l+m+mi}{1}\PY{p}{,}\PY{l+m+mi}{2}\PY{p}{)}\PY{p}{,} \PY{p}{\PYZob{}}\PY{l+s+s2}{\PYZdq{}}\PY{l+s+s2}{test}\PY{l+s+s2}{\PYZdq{}}\PY{p}{:}\PY{l+m+mi}{1}\PY{p}{\PYZcb{}}\PY{p}{]}

\PY{n}{number\PYZus{}of\PYZus{}items} \PY{o}{=} \PY{l+m+mi}{0}
\PY{k}{for} \PY{n}{item} \PY{o+ow}{in} \PY{n}{l}\PY{p}{:}
    \PY{k}{if} \PY{n+nb}{type}\PY{p}{(}\PY{n}{item}\PY{p}{)} \PY{o}{!=} \PY{n+nb}{tuple}\PY{p}{:}
        \PY{n}{number\PYZus{}of\PYZus{}items} \PY{o}{=} \PY{n}{number\PYZus{}of\PYZus{}items} \PY{o}{+} \PY{l+m+mi}{1}
    \PY{k}{else}\PY{p}{:}
        \PY{k}{break}
        
\PY{n+nb}{print} \PY{p}{(}\PY{l+s+s2}{\PYZdq{}}\PY{l+s+s2}{There are }\PY{l+s+si}{\PYZob{}\PYZcb{}}\PY{l+s+s2}{ items before a tuple.}\PY{l+s+s2}{\PYZdq{}}\PY{o}{.}\PY{n}{format}\PY{p}{(}\PY{n}{number\PYZus{}of\PYZus{}items}\PY{p}{)}\PY{p}{)}

There are 3 items before a tuple.
\end{Verbatim}
\end{tcolorbox}
\end{Answer}

\begin{Exercise}
Write a \texttt{python} script to concatenate following dictionaries to create a new single one.
Input:
\begin{Shaded}
\begin{Highlighting}[]
\NormalTok{dic1}\OperatorTok{=}\NormalTok{\{}\DecValTok{1}\NormalTok{:}\DecValTok{10}\NormalTok{, }\DecValTok{2}\NormalTok{:}\DecValTok{20}\NormalTok{\}}
\NormalTok{dic2}\OperatorTok{=}\NormalTok{\{}\DecValTok{3}\NormalTok{:}\DecValTok{30}\NormalTok{, }\DecValTok{4}\NormalTok{:}\DecValTok{40}\NormalTok{\}}
\NormalTok{dic3}\OperatorTok{=}\NormalTok{\{}\DecValTok{5}\NormalTok{:}\DecValTok{50}\NormalTok{, }\DecValTok{6}\NormalTok{:}\DecValTok{60}\NormalTok{\}}
\end{Highlighting}
\end{Shaded}
\end{Exercise}

\begin{Answer}
\begin{tcolorbox}[size=fbox, boxrule=1pt, colback=cellbackground, colframe=cellborder]
\begin{Verbatim}[commandchars=\\\{\}]
\PY{n}{dic1}\PY{o}{=}\PY{p}{\PYZob{}}\PY{l+m+mi}{1}\PY{p}{:}\PY{l+m+mi}{10}\PY{p}{,} \PY{l+m+mi}{2}\PY{p}{:}\PY{l+m+mi}{20}\PY{p}{\PYZcb{}}
\PY{n}{dic2}\PY{o}{=}\PY{p}{\PYZob{}}\PY{l+m+mi}{3}\PY{p}{:}\PY{l+m+mi}{30}\PY{p}{,} \PY{l+m+mi}{4}\PY{p}{:}\PY{l+m+mi}{40}\PY{p}{\PYZcb{}}
\PY{n}{dic3}\PY{o}{=}\PY{p}{\PYZob{}}\PY{l+m+mi}{5}\PY{p}{:}\PY{l+m+mi}{50}\PY{p}{,}\PY{l+m+mi}{6}\PY{p}{:}\PY{l+m+mi}{60}\PY{p}{\PYZcb{}}

\PY{n}{dic\PYZus{}tot} \PY{o}{=} \PY{n+nb}{dict}\PY{p}{(}\PY{p}{)}
\PY{n}{dic\PYZus{}tot}\PY{o}{.}\PY{n}{update}\PY{p}{(}\PY{n}{dic1}\PY{p}{)}
\PY{n}{dic\PYZus{}tot}\PY{o}{.}\PY{n}{update}\PY{p}{(}\PY{n}{dic2}\PY{p}{)}
\PY{n}{dic\PYZus{}tot}\PY{o}{.}\PY{n}{update}\PY{p}{(}\PY{n}{dic3}\PY{p}{)}

\PY{n+nb}{print} \PY{p}{(}\PY{n}{dic\PYZus{}tot}\PY{p}{)}

\{1: 10, 2: 20, 3: 30, 4: 40, 5: 50, 6: 60\}
\end{Verbatim}
\end{tcolorbox}
\end{Answer}

\begin{Exercise}
Write a \texttt{python} script to check whether a given key already exists in a dictionary.
\end{Exercise}

\begin{Answer}
\begin{tcolorbox}[size=fbox, boxrule=1pt, colback=cellbackground, colframe=cellborder]
\begin{Verbatim}[commandchars=\\\{\}]
\PY{n}{dic} \PY{o}{=} \PY{p}{\PYZob{}}\PY{l+s+s2}{\PYZdq{}}\PY{l+s+s2}{a}\PY{l+s+s2}{\PYZdq{}}\PY{p}{:}\PY{l+m+mi}{1}\PY{p}{,} \PY{l+s+s2}{\PYZdq{}}\PY{l+s+s2}{b}\PY{l+s+s2}{\PYZdq{}}\PY{p}{:}\PY{l+m+mi}{2}\PY{p}{,} \PY{l+s+s2}{\PYZdq{}}\PY{l+s+s2}{c}\PY{l+s+s2}{\PYZdq{}}\PY{p}{:}\PY{l+m+mi}{3}\PY{p}{\PYZcb{}}

\PY{n+nb}{print} \PY{p}{(}\PY{l+s+s2}{\PYZdq{}}\PY{l+s+s2}{z}\PY{l+s+s2}{\PYZdq{}} \PY{o+ow}{in} \PY{n}{dic}\PY{p}{)}
\PY{n+nb}{print} \PY{p}{(}\PY{l+s+s2}{\PYZdq{}}\PY{l+s+s2}{a}\PY{l+s+s2}{\PYZdq{}} \PY{o+ow}{in} \PY{n}{dic}\PY{p}{)}

False
True
\end{Verbatim}
\end{tcolorbox}
\end{Answer}

\begin{Exercise}
Write a \texttt{python} program to combine two dictionary adding values for common keys.
Input:
\begin{Shaded}
\begin{Highlighting}[]
\NormalTok{d1 }\OperatorTok{=}\NormalTok{ \{}\StringTok{'a'}\NormalTok{: }\DecValTok{100}\NormalTok{, }\StringTok{'b'}\NormalTok{: }\DecValTok{200}\NormalTok{, }\StringTok{'c'}\NormalTok{:}\DecValTok{300}\NormalTok{\}}
\NormalTok{d2 }\OperatorTok{=}\NormalTok{ \{}\StringTok{'a'}\NormalTok{: }\DecValTok{300}\NormalTok{, }\StringTok{'b'}\NormalTok{: }\DecValTok{200}\NormalTok{, }\StringTok{'d'}\NormalTok{:}\DecValTok{400}\NormalTok{\}}
\end{Highlighting}
\end{Shaded}
\end{Exercise}

\begin{Answer}
\begin{tcolorbox}[size=fbox, boxrule=1pt, colback=cellbackground, colframe=cellborder]
\begin{Verbatim}[commandchars=\\\{\}]
\PY{n}{d1} \PY{o}{=} \PY{p}{\PYZob{}}\PY{l+s+s1}{\PYZsq{}}\PY{l+s+s1}{a}\PY{l+s+s1}{\PYZsq{}}\PY{p}{:} \PY{l+m+mi}{100}\PY{p}{,} \PY{l+s+s1}{\PYZsq{}}\PY{l+s+s1}{b}\PY{l+s+s1}{\PYZsq{}}\PY{p}{:} \PY{l+m+mi}{200}\PY{p}{,} \PY{l+s+s1}{\PYZsq{}}\PY{l+s+s1}{c}\PY{l+s+s1}{\PYZsq{}}\PY{p}{:}\PY{l+m+mi}{300}\PY{p}{\PYZcb{}}
\PY{n}{d2} \PY{o}{=} \PY{p}{\PYZob{}}\PY{l+s+s1}{\PYZsq{}}\PY{l+s+s1}{a}\PY{l+s+s1}{\PYZsq{}}\PY{p}{:} \PY{l+m+mi}{300}\PY{p}{,} \PY{l+s+s1}{\PYZsq{}}\PY{l+s+s1}{b}\PY{l+s+s1}{\PYZsq{}}\PY{p}{:} \PY{l+m+mi}{200}\PY{p}{,} \PY{l+s+s1}{\PYZsq{}}\PY{l+s+s1}{d}\PY{l+s+s1}{\PYZsq{}}\PY{p}{:}\PY{l+m+mi}{400}\PY{p}{\PYZcb{}}

\PY{n}{d} \PY{o}{=} \PY{p}{\PYZob{}}\PY{p}{\PYZcb{}}
\PY{n}{d}\PY{o}{.}\PY{n}{update}\PY{p}{(}\PY{n}{d1}\PY{p}{)}

\PY{k}{for} \PY{n}{k} \PY{o+ow}{in} \PY{n}{d2}\PY{o}{.}\PY{n}{keys}\PY{p}{(}\PY{p}{)}\PY{p}{:}
    \PY{k}{if} \PY{n}{k} \PY{o+ow}{in} \PY{n}{d}\PY{p}{:}
        \PY{n}{d}\PY{p}{[}\PY{n}{k}\PY{p}{]} \PY{o}{=} \PY{n}{d}\PY{p}{[}\PY{n}{k}\PY{p}{]} \PY{o}{+} \PY{n}{d2}\PY{p}{[}\PY{n}{k}\PY{p}{]}
    \PY{k}{else}\PY{p}{:}
        \PY{n}{d}\PY{p}{[}\PY{n}{k}\PY{p}{]} \PY{o}{=} \PY{n}{d2}\PY{p}{[}\PY{n}{k}\PY{p}{]}

\PY{n+nb}{print} \PY{p}{(}\PY{n}{d}\PY{p}{)}

\{'a': 400, 'b': 400, 'c': 300, 'd': 400\}
\end{Verbatim}
\end{tcolorbox}
\end{Answer}

\begin{Exercise}
Given the following dictionary mapping currencies to 2-year zero coupon
bond prices, build another dictionary mapping the same currencies to the
corresponding annualized interest rates.

\begin{Shaded}
\begin{Highlighting}[]
\NormalTok{d }\OperatorTok{=}\NormalTok{ \{}
\StringTok{'EUR'}\NormalTok{: }\FloatTok{0.98}\NormalTok{,}
\StringTok{'CHF'}\NormalTok{: }\FloatTok{1.005}\NormalTok{,}
\StringTok{'USD'}\NormalTok{: }\FloatTok{0.985}\NormalTok{,}
\StringTok{'GBP'}\NormalTok{: }\FloatTok{0.97}
\NormalTok{\}}
\end{Highlighting}
\end{Shaded}
\end{Exercise}

\begin{Answer}
The price of a n-years zero coupon bond is:

\[ P = \frac{M}{(1+r)^{n}} = M\cdot D \]

where $M$ is the value of the bond at the maturity, $r$ is the
risk-free rate and $n$ is the number of years until maturity.

Hence:

\[ D = \cfrac{1}{(1+r)^{n}} \implies r = \Big(\cfrac{1}{D}\Big)^{\cfrac{1}{n}} - 1\]

\begin{tcolorbox}[size=fbox, boxrule=1pt, colback=cellbackground, colframe=cellborder]
\begin{Verbatim}[commandchars=\\\{\}]
\PY{k+kn}{from} \PY{n+nn}{math} \PY{k}{import} \PY{n}{exp}

\PY{c+c1}{\PYZsh{} initialize an empty dictionary in which to store result}
\PY{n}{rates} \PY{o}{=} \PY{p}{\PYZob{}}\PY{p}{\PYZcb{}}

\PY{n}{maturity} \PY{o}{=} \PY{l+m+mi}{2}
\PY{n}{discount\PYZus{}factors} \PY{o}{=} \PY{p}{\PYZob{}}
    \PY{l+s+s1}{\PYZsq{}}\PY{l+s+s1}{EUR}\PY{l+s+s1}{\PYZsq{}}\PY{p}{:} \PY{l+m+mf}{0.98}\PY{p}{,}
    \PY{l+s+s1}{\PYZsq{}}\PY{l+s+s1}{CHF}\PY{l+s+s1}{\PYZsq{}}\PY{p}{:} \PY{l+m+mf}{1.005}\PY{p}{,}
    \PY{l+s+s1}{\PYZsq{}}\PY{l+s+s1}{USD}\PY{l+s+s1}{\PYZsq{}}\PY{p}{:} \PY{l+m+mf}{0.985}\PY{p}{,}
    \PY{l+s+s1}{\PYZsq{}}\PY{l+s+s1}{GBP}\PY{l+s+s1}{\PYZsq{}}\PY{p}{:} \PY{l+m+mf}{0.97}
\PY{p}{\PYZcb{}}

\PY{c+c1}{\PYZsh{} loop over the input dictionary to get the currencies}
\PY{k}{for} \PY{n}{currency}\PY{p}{,} \PY{n}{df} \PY{o+ow}{in} \PY{n}{discount\PYZus{}factors}\PY{o}{.}\PY{n}{items}\PY{p}{(}\PY{p}{)}\PY{p}{:}
    \PY{c+c1}{\PYZsh{} calculate the rate and store it in the output dictionary}
    \PY{n}{rates}\PY{p}{[}\PY{n}{currency}\PY{p}{]} \PY{o}{=} \PY{n+nb}{pow}\PY{p}{(}\PY{l+m+mi}{1}\PY{o}{/}\PY{n}{df}\PY{p}{,} \PY{l+m+mi}{1}\PY{o}{/}\PY{n}{maturity}\PY{p}{)} \PY{o}{\PYZhy{}} \PY{l+m+mi}{1}
    
\PY{k}{for} \PY{n}{r} \PY{o+ow}{in} \PY{n}{rates}\PY{o}{.}\PY{n}{items}\PY{p}{(}\PY{p}{)}\PY{p}{:}    
    \PY{n+nb}{print} \PY{p}{(}\PY{n}{r}\PY{p}{)}

('EUR', 0.010152544552210818)
('CHF', -0.002490663892367073)
('USD', 0.007585443719756668)
('GBP', 0.015346165133619083)
\end{Verbatim}
\end{tcolorbox}
\end{Answer}

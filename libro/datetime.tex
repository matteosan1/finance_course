\chapter{Date and Time}
\label{sec:datetime}

In this Chapter we take a little break and concentrate on a topic that it is not usually covered 
in a \texttt{python} review for beginners. However given its importance for financial computation 
the next paragraphs will be devoted to a close look up on the $\tt{datetime}$ module~\cite{bib:datetime}, whose usage 
will help in manipulating dates.

\section{Dates}\label{dates}

As said dates are not usually included in a standard \texttt{python} tutorial, however since they 
are pretty essential for finance we are going to cover this topic in some detail. 
In \texttt{python} the date utilities mainly lives in the \texttt{datetime} module. Briefly we are 
also going to show \texttt{relativedelta} from the \texttt{dateutil} module, which allows to 
add/subtract days/months/years to dates, in other words to make operations on them.

In this first example today's date is defined and with $\tt{relativedelta}$ two more dates are created 
adding two months and subtracting three days to the first one.

\begin{codebox}
\begin{Verbatim}[commandchars=\\\{\}]
\PY{k+kn}{from} \PY{n+nn}{datetime} \PY{k}{import} \PY{n}{date}\PY{p}{,} \PY{n}{datetime}
\PY{k+kn}{from} \PY{n+nn}{dateutil}\PY{n+nn}{.}\PY{n+nn}{relativedelta} \PY{k}{import} \PY{n}{relativedelta}

\PY{n}{date1} \PY{o}{=} \PY{n}{date}\PY{o}{.}\PY{n}{today}\PY{p}{(}\PY{p}{)}
\PY{n+nb}{print} \PY{p}{(}\PY{n}{date1}\PY{p}{)}
\PY{n}{date2} \PY{o}{=} \PY{n}{date}\PY{o}{.}\PY{n}{today}\PY{p}{(}\PY{p}{)} \PY{o}{+} \PY{n}{relativedelta}\PY{p}{(}\PY{n}{months}\PY{o}{=}\PY{l+m+mi}{2}\PY{p}{)}
\PY{n+nb}{print} \PY{p}{(}\PY{n}{date2}\PY{p}{)}
\PY{n}{date3} \PY{o}{=} \PY{n}{date}\PY{o}{.}\PY{n}{today}\PY{p}{(}\PY{p}{)} \PY{o}{\PYZhy{}} \PY{n}{relativedelta}\PY{p}{(}\PY{n}{days}\PY{o}{=}\PY{l+m+mi}{3}\PY{p}{)}
\PY{n+nb}{print} \PY{p}{(}\PY{n}{date3}\PY{p}{)}

2020-08-03
2020-10-03
2020-07-31
\end{Verbatim}
\end{codebox}

There is another way of computing a new date: a one day delta is stored in a variable 
and today's date is moved by three days multiplying the defined delta by three.

\begin{codebox}
\begin{Verbatim}[commandchars=\\\{\}]
\PY{n}{one\PYZus{}day} \PY{o}{=} \PY{n}{relativedelta}\PY{p}{(}\PY{n}{days}\PY{o}{=}\PY{l+m+mi}{1}\PY{p}{)}
\PY{n}{date}\PY{o}{.}\PY{n}{today}\PY{p}{(}\PY{p}{)} \PY{o}{\PYZhy{}} \PY{l+m+mi}{3} \PY{o}{*} \PY{n}{one\PYZus{}day}

datetime.date(2020, 7, 31)
\end{Verbatim}
\end{codebox}

Next, given two dates their difference is computed and expressed in number of days.

\begin{codebox}
\begin{Verbatim}[commandchars=\\\{\}]
\PY{n}{date1} \PY{o}{=} \PY{n}{date}\PY{p}{(}\PY{l+m+mi}{2019}\PY{p}{,} \PY{l+m+mi}{7}\PY{p}{,} \PY{l+m+mi}{2}\PY{p}{)}
\PY{n}{date2} \PY{o}{=} \PY{n}{date}\PY{p}{(}\PY{l+m+mi}{2019}\PY{p}{,} \PY{l+m+mi}{8}\PY{p}{,} \PY{l+m+mi}{16}\PY{p}{)}
\PY{p}{(}\PY{n}{date2} \PY{o}{\PYZhy{}} \PY{n}{date1}\PY{p}{)}\PY{o}{.}\PY{n}{days}

45
\end{Verbatim}
\end{codebox}

Dates can be converted to and from strings and a large variety of formats can be specified in this conversions. 
The format is determined by a string in which each character starting with \% represent an element 
of the date, e.g. \%Y year, \%d day, \%s seconds\ldots

Below dates to string conversion:

\begin{codebox}
\begin{Verbatim}[commandchars=\\\{\}]
\PY{n}{date1} \PY{o}{=} \PY{n}{date}\PY{p}{(}\PY{l+m+mi}{2019}\PY{p}{,} \PY{l+m+mi}{7}\PY{p}{,} \PY{l+m+mi}{2}\PY{p}{)}
\PY{n}{date1}\PY{o}{.}\PY{n}{strftime}\PY{p}{(}\PY{l+s+s2}{\PYZdq{}}\PY{l+s+s2}{\PYZpc{}}\PY{l+s+s2}{Y\PYZhy{}}\PY{l+s+s2}{\PYZpc{}}\PY{l+s+s2}{b\PYZhy{}}\PY{l+s+si}{\PYZpc{}d}\PY{l+s+s2}{ (}\PY{l+s+si}{\PYZpc{}a}\PY{l+s+s2}{)}\PY{l+s+s2}{\PYZdq{}}\PY{p}{)} \PY{c+c1}{\PYZsh{} dates can formatted in many ways}
                                \PY{c+c1}{\PYZsh{} check the docs for more details}

'2019-Jul-02 (Tue)'
\end{Verbatim}
\end{codebox}

And here, a string is converted to \texttt{datetime} object:

\begin{codebox}
\begin{Verbatim}[commandchars=\\\{\}]
\PY{c+c1}{\PYZsh{} a string can be converted to dates too}
\PY{n}{datetime}\PY{o}{.}\PY{n}{strptime}\PY{p}{(}\PY{l+s+s1}{\PYZsq{}}\PY{l+s+s1}{25 Aug 2019}\PY{l+s+s1}{\PYZsq{}}\PY{p}{,} \PY{l+s+s2}{\PYZdq{}}\PY{l+s+si}{\PYZpc{}d}\PY{l+s+s2}{ }\PY{l+s+s2}{\PYZpc{}}\PY{l+s+s2}{b }\PY{l+s+s2}{\PYZpc{}}\PY{l+s+s2}{Y}\PY{l+s+s2}{\PYZdq{}}\PY{p}{)}\PY{o}{.}\PY{n}{date}\PY{p}{(}\PY{p}{)}

datetime.date(2019, 8, 25)
\end{Verbatim}
\end{codebox}

Finally a last example showing how to get the week-day from a date:

\begin{codebox}
\begin{Verbatim}[commandchars=\\\{\}]
\PY{n}{date1}\PY{o}{.}\PY{n}{weekday}\PY{p}{(}\PY{p}{)} \PY{c+c1}{\PYZsh{} 0 = Monday, ..., 6 = Sunday}

1
\end{Verbatim}
\end{codebox}

\section{Exercises}
\begin{question}
\label{ex:dateex}
Write code that:

\begin{itemize}
\item print the day of the week of your birthday;
\item print the weekday of your birthdays for the next 120 years.
\end{itemize}
\end{question}

\cprotEnv\begin{solution}
\begin{ipython}
import datetime

birthday = datetime.date(1974, 10, 20)
print (birthday.weekday()) # remember it starts form 0
\end{ipython}
\begin{ioutput}
6
\end{ioutput}
\begin{ipython}
from dateutil.relativedelta import relativedelta

for i in range(120):
    print ((birthday + relativedelta(years=i)).weekday())
\end{ipython}
\begin{ioutput}
6
0
2
3
4
5
0
1
2
3
4
...
\end{ioutput}
\end{solution}

\begin{question}
Write code to determine whether a given year is a leap year and test it with 1800, 1987 and 2020.

\noindent\textbf{Hint:} a leap year is divisible by 4, by 100 and by 400.
\end{question}

\cprotEnv\begin{solution}
\begin{ipython}
years = [1800, 1987, 2020]

for y in years:
    if y % 400 == 0:
        print ("{} is a leap year".format(y))
	elif y % 100 == 0:
        print ("{} is a leap year".format(y)) 
    elif y % 4 == 0:
        print ("{} is a leap year".format(y))
    else:
        print ("{} is NOT a leap year".format(y))
\end{ipython}
\begin{ioutput}
1800 is NOT a leap year
1987 is NOT a leap year
2020 is a leap year        
\end{ioutput}  
\end{solution}

\begin{question}
Write code to print next five days starting from today.
\end{question}

\cprotEnv\begin{solution}
\begin{ipython}
d = datetime.date.today()
for i in range(1, 6):
     print (d + relativedelta(days=i))
\end{ipython}
\begin{ioutput}
2020-08-04
2020-08-05
2020-08-06
2020-08-07
2020-08-08
\end{ioutput}
\end{solution}

\begin{question}
Build again dates as in Exercise~\ref{ex:dateex} (i.e. the weekday of your birthdays for the next 120 years) and count how many of your birthdays is a Monday, Tuesday, \ldots{} , Sunday until 120 years of age. Print out the result using a dictionary. (expected output something like: \texttt{\{6:\ 10,\ 0:\ 10,\ 2:\ 9,\ 3:\ 10,\ 4:\ 10,\ 5:\ 10,\ 1:\ 9\}})
\end{question}

\cprotEnv\begin{solution}
\begin{ipython}
import datetime
from dateutil.relativedelta import relativedelta

birthday = datetime.date(1974, 10, 20)
d = {}
for i in range(120):
    wd = (birthday + relativedelta(years=i)).weekday() 
    if wd in d.keys():
        d[wd] += 1
    else:
        d[wd] = 1
        
print (d)
\end{ipython}
\begin{ioutput}
{6: 17, 0: 18, 2: 17, 3: 17, 4: 17, 5: 17, 1: 17}
\end{ioutput}
\end{solution}

\begin{question}
Write an algorithm which takes in input a start date and a number of months, and returns a list of dates with \textbf{annual} frequency from the start date to the ending of the period after the specified number of months.

For example
\begin{itemize}
\item 2019-11-10 start date 12 months \(\rightarrow\) 2019-11-10, 2020-11-10
\item 2019-11-10 start date 24 months \(\rightarrow\) 2019-11-10, 2020-11-10, 2021-11-10
\end{itemize}

Note that if the number of months is not a multiple of 12, the last period should simply be shorter than 12 months. For example:

\begin{itemize}
\item 2019-11-10 start date 9 months \(\rightarrow\) 2019-11-10, 2020-08-10
\item 2019-11-10 start date 15 months \(\rightarrow\) 2019-11-10, 2020-11-10, 2021-02-10
\end{itemize}

%\begin{figure}
%  %\centering
%\includegraphics[width=0.8\linewidth]{time_flow.png}
%\end{figure}

%Here's some skeleton code to help you get started:
%
%\begin{Shaded}
%\begin{Highlighting}[]
%\ImportTok{from}\NormalTok{ dateutil }\ImportTok{import}\NormalTok{ relativedelta}
%
%\KeywordTok{def}\NormalTok{ generate_swap_dates(start_date, n_months):}
%\NormalTok{    dates }\OperatorTok{=}\NormalTok{ []}
%    \CommentTok{# your code here which adds all the relevant dates to the dates list}
%    \ControlFlowTok{return}\NormalTok{ dates}
%\end{Highlighting}
%\end{Shaded}
%
%\begin{Shaded}
%\begin{Highlighting}[]
%\CommentTok{# some tests to check if the function is working correctly}
%\ImportTok{from}\NormalTok{ datetime }\ImportTok{import}\NormalTok{ date}
%
%\ControlFlowTok{assert}\NormalTok{ generate_swap_dates(date(}\DecValTok{2019}\NormalTok{, }\DecValTok{11}\NormalTok{, }\DecValTok{10}\NormalTok{), }\DecValTok{12}\NormalTok{) }\OperatorTok{==}\NormalTok{ [date(}\DecValTok{2019}\NormalTok{, }\DecValTok{11}\NormalTok{, }\DecValTok{10}\NormalTok{), }
%\NormalTok{                                                       date(}\DecValTok{2020}\NormalTok{, }\DecValTok{11}\NormalTok{, }\DecValTok{10}\NormalTok{)]}
%\ControlFlowTok{assert}\NormalTok{ generate_swap_dates(date(}\DecValTok{2019}\NormalTok{, }\DecValTok{11}\NormalTok{, }\DecValTok{10}\NormalTok{), }\DecValTok{24}\NormalTok{) }\OperatorTok{==}\NormalTok{ [date(}\DecValTok{2019}\NormalTok{, }\DecValTok{11}\NormalTok{, }\DecValTok{10}\NormalTok{), }
%\NormalTok{                                                       date(}\DecValTok{2020}\NormalTok{, }\DecValTok{11}\NormalTok{, }\DecValTok{10}\NormalTok{), }
%\NormalTok{                                                       date(}\DecValTok{2021}\NormalTok{, }\DecValTok{11}\NormalTok{, }\DecValTok{10}\NormalTok{)]}
%
%\ControlFlowTok{assert}\NormalTok{ generate_swap_dates(date(}\DecValTok{2019}\NormalTok{, }\DecValTok{11}\NormalTok{, }\DecValTok{10}\NormalTok{), }\DecValTok{9}\NormalTok{) }\OperatorTok{==}\NormalTok{ [date(}\DecValTok{2019}\NormalTok{, }\DecValTok{11}\NormalTok{, }\DecValTok{10}\NormalTok{), }
%\NormalTok{                                                      date(}\DecValTok{2020}\NormalTok{, }\DecValTok{8}\NormalTok{, }\DecValTok{10}\NormalTok{)]}
%\ControlFlowTok{assert}\NormalTok{ generate_swap_dates(date(}\DecValTok{2019}\NormalTok{, }\DecValTok{11}\NormalTok{, }\DecValTok{10}\NormalTok{), }\DecValTok{15}\NormalTok{) }\OperatorTok{==}\NormalTok{ [date(}\DecValTok{2019}\NormalTok{, }\DecValTok{11}\NormalTok{, }\DecValTok{10}\NormalTok{), }
%\NormalTok{                                                       date(}\DecValTok{2020}\NormalTok{, }\DecValTok{11}\NormalTok{, }\DecValTok{10}\NormalTok{), }
%\NormalTok{                                                       date(}\DecValTok{2021}\NormalTok{, }\DecValTok{2}\NormalTok{, }\DecValTok{10}\NormalTok{)]}
%\end{Highlighting}
%\end{Shaded}
\end{question}

\cprotEnv\begin{solution}
\begin{ipython}
from datetime import date
from dateutil.relativedelta import relativedelta

start_dates} = date(2019, 11, 10)
n_months = 15
dates = []
for i in range(0, n_months, 12):
    dates.append(start_date + relativedelta(months=i))
dates.append(start_date + relativedelta(months=n_months))

print(dates)
\end{ipython}
\begin{ioutput}
[date(2019, 11, 10), date(2020, 11, 10), date(2021, 2, 10)]
\end{ioutput}
\end{solution}

%\begin{solution}
%\begin{Verbatim}[commandchars=\\\{\}]
%\PY{k+kn}{from} \PY{n+nn}{finmarkets} \PY{k}{import} \PY{n}{generate\PYZus{}swap\PYZus{}dates}
%\PY{k+kn}{from} \PY{n+nn}{datetime} \PY{k}{import} \PY{n}{date}
%\PY{k+kn}{from} \PY{n+nn}{dateutil}\PY{n+nn}{.}\PY{n+nn}{relativedelta} \PY{k}{import} \PY{n}{relativedelta}
%
%\PY{k}{def} \PY{n+nf}{generate\PYZus{}swap\PYZus{}dates}\PY{p}{(}\PY{n}{start\PYZus{}date}\PY{p}{,} \PY{n}{n\PYZus{}months}\PY{p}{)}\PY{p}{:}
%    \PY{n}{dates} \PY{o}{=} \PY{p}{[}\PY{p}{]}
%    \PY{k}{for} \PY{n}{i} \PY{o+ow}{in} \PY{n+nb}{range}\PY{p}{(}\PY{l+m+mi}{0}\PY{p}{,} \PY{n}{n\PYZus{}months}\PY{p}{,} \PY{l+m+mi}{12}\PY{p}{)}\PY{p}{:}
%        \PY{n}{dates}\PY{o}{.}\PY{n}{append}\PY{p}{(}\PY{n}{start\PYZus{}date} \PY{o}{+} \PY{n}{relativedelta}\PY{p}{(}\PY{n}{months}\PY{o}{=}\PY{n}{i}\PY{p}{)}\PY{p}{)}
%    \PY{n}{dates}\PY{o}{.}\PY{n}{append}\PY{p}{(}\PY{n}{start\PYZus{}date} \PY{o}{+} \PY{n}{relativedelta}\PY{p}{(}\PY{n}{months}\PY{o}{=}\PY{n}{n\PYZus{}months}\PY{p}{)}\PY{p}{)}
%    
%    \PY{k}{return} \PY{n}{dates}
%
%
%\PY{k}{assert} \PY{n}{generate\PYZus{}swap\PYZus{}dates}\PY{p}{(}\PY{n}{date}\PY{p}{(}\PY{l+m+mi}{2019}\PY{p}{,} \PY{l+m+mi}{11}\PY{p}{,} \PY{l+m+mi}{10}\PY{p}{)}\PY{p}{,} \PY{l+m+mi}{12}\PY{p}{)} \PY{o}{==} \PY{p}{[}\PY{n}{date}\PY{p}{(}\PY{l+m+mi}{2019}\PY{p}{,} \PY{l+m+mi}{11}\PY{p}{,} \PY{l+m+mi}{10}\PY{p}{)}\PY{p}{,} 
%                                                       \PY{n}{date}\PY{p}{(}\PY{l+m+mi}{2020}\PY{p}{,} \PY{l+m+mi}{11}\PY{p}{,} \PY{l+m+mi}{10}\PY{p}{)}\PY{p}{]}
%\PY{k}{assert} \PY{n}{generate\PYZus{}swap\PYZus{}dates}\PY{p}{(}\PY{n}{date}\PY{p}{(}\PY{l+m+mi}{2019}\PY{p}{,} \PY{l+m+mi}{11}\PY{p}{,} \PY{l+m+mi}{10}\PY{p}{)}\PY{p}{,} \PY{l+m+mi}{24}\PY{p}{)} \PY{o}{==} \PY{p}{[}\PY{n}{date}\PY{p}{(}\PY{l+m+mi}{2019}\PY{p}{,} \PY{l+m+mi}{11}\PY{p}{,} \PY{l+m+mi}{10}\PY{p}{)}\PY{p}{,} 
%                                                       \PY{n}{date}\PY{p}{(}\PY{l+m+mi}{2020}\PY{p}{,} \PY{l+m+mi}{11}\PY{p}{,} \PY{l+m+mi}{10}\PY{p}{)}\PY{p}{,} 
%                                                       \PY{n}{date}\PY{p}{(}\PY{l+m+mi}{2021}\PY{p}{,} \PY{l+m+mi}{11}\PY{p}{,} \PY{l+m+mi}{10}\PY{p}{)}\PY{p}{]}
%
%\PY{k}{assert} \PY{n}{generate\PYZus{}swap\PYZus{}dates}\PY{p}{(}\PY{n}{date}\PY{p}{(}\PY{l+m+mi}{2019}\PY{p}{,} \PY{l+m+mi}{11}\PY{p}{,} \PY{l+m+mi}{10}\PY{p}{)}\PY{p}{,} \PY{l+m+mi}{9}\PY{p}{)} \PY{o}{==} \PY{p}{[}\PY{n}{date}\PY{p}{(}\PY{l+m+mi}{2019}\PY{p}{,} \PY{l+m+mi}{11}\PY{p}{,} \PY{l+m+mi}{10}\PY{p}{)}\PY{p}{,} 
%                                                      \PY{n}{date}\PY{p}{(}\PY{l+m+mi}{2020}\PY{p}{,} \PY{l+m+mi}{8}\PY{p}{,} \PY{l+m+mi}{10}\PY{p}{)}\PY{p}{]}
%\PY{k}{assert} \PY{n}{generate\PYZus{}swap\PYZus{}dates}\PY{p}{(}\PY{n}{date}\PY{p}{(}\PY{l+m+mi}{2019}\PY{p}{,} \PY{l+m+mi}{11}\PY{p}{,} \PY{l+m+mi}{10}\PY{p}{)}\PY{p}{,} \PY{l+m+mi}{15}\PY{p}{)} \PY{o}{==} \PY{p}{[}\PY{n}{date}\PY{p}{(}\PY{l+m+mi}{2019}\PY{p}{,} \PY{l+m+mi}{11}\PY{p}{,} \PY{l+m+mi}{10}\PY{p}{)}\PY{p}{,} 
%                                                               \PY{n}{date}\PY{p}{(}\PY{l+m+mi}{2020}\PY{p}{,} \PY{l+m+mi}{11}\PY{p}{,} \PY{l+m+mi}{10}\PY{p}{)}\PY{p}{,} 
%                                                               \PY{n}{date}\PY{p}{(}\PY{l+m+mi}{2021}\PY{p}{,} \PY{l+m+mi}{2}\PY{p}{,} \PY{l+m+mi}{10}\PY{p}{)}\PY{p}{]}
%\end{Verbatim}
%\end{solution}


\begin{thebibliography}{9}
\bibitem{bib:datetime} \href{https://realpython.com/python-datetime/}{\emph{Using Python datetime to Work With Dates and Times}}, Real Python [Online]
\end{thebibliography}
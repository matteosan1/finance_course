\chapter{Interest Rate Swaps and Swaptions}\label{interest-rate-swaps-and-swaptions}

\subsection{Interest Rate Swaps}\label{interest-rate-swaps}

Interest rate swaps (IRS) consist of a floating leg and a fixed leg. The
contract parameters are:

\begin{itemize}
\tightlist
\item
  start date \(d_0\);
\item
  notional \(N\);
\item
  fixed rate \(K\);
\item
  floating rate tenor (months);
\item
  maturity (years).
\end{itemize}

The floating leg pays the reference LIBOR fixing at a frequency equal to
the tenor of the floating rate - so for example an IRS on a 3-month
LIBOR will pay a floating coupon every three months, an IRS on 6-month
EURIBOR pays the floating coupon every six months and so on.

The fixed leg pays a predetermined cash flow at annual frequency,
regardless of the tenor of the underlying floating rate. For simplicity
we will only consider swaps with maturities which are multiples of 1
year.

    Before going into the deatils of the valuation of IRSs, we need to
modify the \texttt{generate\_swap\_dates} function in our
\texttt{finmarkets} module to generate the payment dates for both the
fixed and floating legs, as follows:

    \begin{tcolorbox}[breakable, size=fbox, boxrule=1pt, pad at break*=1mm,colback=cellbackground, colframe=cellborder]
\begin{Verbatim}[commandchars=\\\{\}]
\PY{k+kn}{from} \PY{n+nn}{datetime} \PY{k}{import} \PY{n}{date}
\PY{k+kn}{from} \PY{n+nn}{dateutil}\PY{n+nn}{.}\PY{n+nn}{relativedelta} \PY{k}{import} \PY{n}{relativedelta}
    
\PY{k}{def} \PY{n+nf}{generate\PYZus{}swap\PYZus{}dates}\PY{p}{(}\PY{n}{start\PYZus{}date}\PY{p}{,} \PY{n}{n\PYZus{}months}\PY{p}{,} \PY{n}{tenor\PYZus{}months}\PY{o}{=}\PY{l+m+mi}{12}\PY{p}{)}\PY{p}{:}
    \PY{n}{dates} \PY{o}{=} \PY{p}{[}\PY{p}{]}
    \PY{k}{for} \PY{n}{n} \PY{o+ow}{in} \PY{n+nb}{range}\PY{p}{(}\PY{l+m+mi}{0}\PY{p}{,} \PY{n}{n\PYZus{}months}\PY{p}{,} \PY{n}{tenor\PYZus{}months}\PY{p}{)}\PY{p}{:}
        \PY{n}{dates}\PY{o}{.}\PY{n}{append}\PY{p}{(}\PY{n}{start\PYZus{}date} \PY{o}{+} \PY{n}{relativedelta}\PY{p}{(}\PY{n}{months}\PY{o}{=}\PY{n}{n}\PY{p}{)}\PY{p}{)}
    \PY{n}{dates}\PY{o}{.}\PY{n}{append}\PY{p}{(}\PY{n}{start\PYZus{}date} \PY{o}{+} \PY{n}{relativedelta}\PY{p}{(}\PY{n}{months}\PY{o}{=}\PY{n}{n\PYZus{}months}\PY{p}{)}\PY{p}{)}
    \PY{k}{return} \PY{n}{dates}

\PY{n}{generate\PYZus{}swap\PYZus{}dates}\PY{p}{(}\PY{n}{date}\PY{o}{.}\PY{n}{today}\PY{p}{(}\PY{p}{)}\PY{p}{,} \PY{l+m+mi}{16}\PY{p}{,} \PY{l+m+mi}{3}\PY{p}{)}

[datetime.date(2020, 10, 15),
 datetime.date(2021, 1, 15),
 datetime.date(2021, 4, 15),
 datetime.date(2021, 7, 15),
 datetime.date(2021, 10, 15),
 datetime.date(2022, 1, 15),
 datetime.date(2022, 2, 15)]
\end{Verbatim}
\end{tcolorbox}
        
    Using this function and the contract parameters we will be able to
determine a sequence of payment dates for each of the two legs.

    Let \(d_0=d_0^{\mathrm{fixed}},...,d_p^{\mathrm{fixed}}\) be the fixed
leg payment dates and
\(d_0=d_0^{\mathrm{float}},...,d_p^{\mathrm{float}}\) be the floating
leg payment dates, and let's use the following notation:

\begin{itemize}
\tightlist
\item
  \(d\) the pricing date;
\item
  \(D(d, d')\) the discount factor observed in date \(d\) for the value
  date \(d'\);
\item
  \(F(d, d', d'')\) the forward rate observed in date \(d\) for the
  period \([d', d'']\). The rate tenor is \(\tau = d'' - d'\).
\end{itemize}

\subsection{IRS Valuation}\label{irs-valuation}

The NPV of the fixed leg is calculated as follows:

\[\mathrm{NPV}_{\mathrm{fixed}}(d; K) = N\cdot K\cdot\sum_{i=1}^{n}D(d, d_{i}^{\mathrm{fixed}})\]
while the NPV of the floating leg is calculated as follows:

\[\mathrm{NPV}_{\mathrm{float}}(d) = N\cdot\sum_{i=1}^{m}F(d, d_{j-1}^{\mathrm{float}}, d_{j}^{\mathrm{float}}) \cdot \frac{d_{j}^{\mathrm{float}}-d_{j-1}^{\mathrm{float}}}{360}
\cdot D(d, d_{i}^{\mathrm{float}})\]

Therefore the NPV of the swap (seen from the point of view of the
counter-party which receives the floating leg) is

\[\mathrm{NPV}(d; K) = \mathrm{NPV}_{\mathrm{float}}(d) - \mathrm{NPV}_{\mathrm{fixed}}(d;K)\]

    For reasons which will become apparent later, it's actually more
convenient to express the NPV of an IRS as a function of the fair value
fixed rate \(S\) of the IRS, also known as the \textbf{swap rate}. \(S\)
is the value of K which makes \(\mathrm{NPV}(d)=0\).

On the basis of the previous expressions, we can easiy calculate \(S\)
as:

\[\mathrm{NPV}_{\mathrm{fixed}}(d;S) = \mathrm{NPV}_{\mathrm{float}}(d)\]
\[N\cdot S\cdot\sum_{i=1}^{n}D(d, d_{i}^{\mathrm{fixed}}) = N\cdot\sum_{i=1}^{m}F(d, d_{j-1}^{\mathrm{float}}, d_{j}^{\mathrm{float}}) \cdot \frac{d_{j}^{\mathrm{float}}-d_{j-1}^{\mathrm{float}}}{360} \cdot D(d, d_{i}^{\mathrm{float}})\]
\[S=\frac{\sum_{i=1}^{m}F(d, d_{j-1}^{\mathrm{float}}, d_{j}^{\mathrm{float}}) \cdot \frac{d_{j}^{\mathrm{float}}-d_{j-1}^{\mathrm{float}}}{360}
\cdot D(d, d_{i}^{\mathrm{float}})}{\sum_{i=1}^{n}D(d, d_i^{\mathrm{fixed}})} \]

    Once we have calculated \(S\), we can express the \(\mathrm{NPV}\) of an
IRS as follows:

\begin{align}&\mathrm{NPV}(d; K) = \mathrm{NPV}_{\mathrm{float}}(d) - \mathrm{NPV}_{\mathrm{fixed}}(d; K) = & \\ \\ &= \underbrace{\mathrm{NPV}_{\mathrm{float}}(d) - \mathrm{NPV}_{\mathrm{fixed}}(d; S)}_{\mathrm{=\;0}} + \mathrm{NPV}_{\mathrm{fixed}}(d;S) - \mathrm{NPV}_{\mathrm{fixed}}(d;K) & \\ & = N\cdot(S-K)\cdot\underbrace{\sum_{i=1}^{n}D(d, d_{i}^{\mathrm{fixed}})}_{\mathrm{'annuity'}}\end{align}

    For convenience the relevant inputs that will be used later (libor and discount curve definitions) have been saved in the files \href{https://drive.google.com/file/d/1dm5oZnZKmJM6UrV0L32OcqD5Tzs9SI9A/view?usp=sharing}{libor.xlsx} and \href{https://drive.google.com/file/d/14R22r7m-6VpQ_P79D3qHdK0QN_mOQ_UB/view?usp=sharing}{discount\_curve.xlsx} respectively.

    \begin{tcolorbox}[breakable, size=fbox, boxrule=1pt, pad at break*=1mm,colback=cellbackground, colframe=cellborder]
\begin{Verbatim}[commandchars=\\\{\}]
\PY{k+kn}{import} \PY{n+nn}{pandas} \PY{k}{as} \PY{n+nn}{pd}
\PY{k+kn}{from} \PY{n+nn}{datetime} \PY{k}{import} \PY{n}{date}
\PY{k+kn}{from} \PY{n+nn}{finmarkets} \PY{k}{import} \PY{n}{DiscountCurve}\PY{p}{,} \PY{n}{ForwardRateCurve}

\PY{n}{pricing\PYZus{}date} \PY{o}{=} \PY{n}{date}\PY{p}{(}\PY{l+m+mi}{2019}\PY{p}{,} \PY{l+m+mi}{11}\PY{p}{,} \PY{l+m+mi}{23}\PY{p}{)}
\PY{n}{start\PYZus{}date} \PY{o}{=} \PY{n}{date}\PY{p}{(}\PY{l+m+mi}{2021}\PY{p}{,} \PY{l+m+mi}{11}\PY{p}{,} \PY{l+m+mi}{23}\PY{p}{)}
\PY{n}{exercise\PYZus{}date} \PY{o}{=} \PY{n}{date}\PY{p}{(}\PY{l+m+mi}{2020}\PY{p}{,} \PY{l+m+mi}{11}\PY{p}{,} \PY{l+m+mi}{23}\PY{p}{)}

\PY{n}{discount\PYZus{}data} \PY{o}{=} \PY{n}{pd}\PY{o}{.}\PY{n}{read\PYZus{}excel}\PY{p}{(}\PY{l+s+s1}{\PYZsq{}}\PY{l+s+s1}{discount\PYZus{}curve.xlsx}\PY{l+s+s1}{\PYZsq{}}\PY{p}{)}
\PY{n}{libor\PYZus{}data} \PY{o}{=} \PY{n}{pd}\PY{o}{.}\PY{n}{read\PYZus{}excel}\PY{p}{(}\PY{l+s+s1}{\PYZsq{}}\PY{l+s+s1}{libor.xlsx}\PY{l+s+s1}{\PYZsq{}}\PY{p}{)}

\PY{n}{dc} \PY{o}{=} \PY{n}{DiscountCurve}\PY{p}{(}\PY{n}{pricing\PYZus{}date}\PY{p}{,} 
                   \PY{n}{discount\PYZus{}data}\PY{p}{[}\PY{l+s+s1}{\PYZsq{}}\PY{l+s+s1}{pillar}\PY{l+s+s1}{\PYZsq{}}\PY{p}{]}\PY{o}{.}\PY{n}{dt}\PY{o}{.}\PY{n}{date}\PY{o}{.}\PY{n}{tolist}\PY{p}{(}\PY{p}{)}\PY{p}{,}
                   \PY{n}{discount\PYZus{}data}\PY{p}{[}\PY{l+s+s1}{\PYZsq{}}\PY{l+s+s1}{discount\PYZus{}factor}\PY{l+s+s1}{\PYZsq{}}\PY{p}{]}\PY{o}{.}\PY{n}{tolist}\PY{p}{(}\PY{p}{)}\PY{p}{)}

\PY{n}{fr} \PY{o}{=} \PY{n}{ForwardRateCurve}\PY{p}{(}\PY{n}{libor\PYZus{}data}\PY{p}{[}\PY{l+s+s1}{\PYZsq{}}\PY{l+s+s1}{date}\PY{l+s+s1}{\PYZsq{}}\PY{p}{]}\PY{o}{.}\PY{n}{dt}\PY{o}{.}\PY{n}{date}\PY{o}{.}\PY{n}{tolist}\PY{p}{(}\PY{p}{)}\PY{p}{,}
                      \PY{n}{libor\PYZus{}data}\PY{p}{[}\PY{l+s+s1}{\PYZsq{}}\PY{l+s+s1}{rate}\PY{l+s+s1}{\PYZsq{}}\PY{p}{]}\PY{o}{.}\PY{n}{tolist}\PY{p}{(}\PY{p}{)}\PY{p}{)}

\PY{n+nb}{print}\PY{p}{(}\PY{n}{dc}\PY{o}{.}\PY{n}{df}\PY{p}{(}\PY{n}{date}\PY{p}{(}\PY{l+m+mi}{2020}\PY{p}{,} \PY{l+m+mi}{1}\PY{p}{,} \PY{l+m+mi}{1}\PY{p}{)}\PY{p}{)}\PY{p}{)}
\PY{n+nb}{print} \PY{p}{(}\PY{n}{fr}\PY{o}{.}\PY{n}{forward\PYZus{}rate}\PY{p}{(}\PY{n}{date}\PY{p}{(}\PY{l+m+mi}{2020}\PY{p}{,} \PY{l+m+mi}{1}\PY{p}{,} \PY{l+m+mi}{1}\PY{p}{)}\PY{p}{)}\PY{p}{)}

1.0003778376026249
0.01000266393442623
    \end{Verbatim}
\end{tcolorbox}

    Now we can implement an \texttt{InterestRateSwap} class to valuate IRS
contracts.

    \begin{tcolorbox}[breakable, size=fbox, boxrule=1pt, pad at break*=1mm,colback=cellbackground, colframe=cellborder]
\begin{Verbatim}[commandchars=\\\{\}]
\PY{k}{class} \PY{n+nc}{InterestRateSwap}\PY{p}{:}
    
    \PY{k}{def} \PY{n+nf}{\PYZus{}\PYZus{}init\PYZus{}\PYZus{}}\PY{p}{(}\PY{n+nb+bp}{self}\PY{p}{,} \PY{n}{start\PYZus{}date}\PY{p}{,} \PY{n}{notional}\PY{p}{,} 
                 \PY{n}{fixed\PYZus{}rate}\PY{p}{,} \PY{n}{tenor\PYZus{}months}\PY{p}{,} 
                 \PY{n}{maturity\PYZus{}years}\PY{p}{)}\PY{p}{:}
        \PY{n+nb+bp}{self}\PY{o}{.}\PY{n}{notional} \PY{o}{=} \PY{n}{notional}
        \PY{n+nb+bp}{self}\PY{o}{.}\PY{n}{fixed\PYZus{}rate} \PY{o}{=} \PY{n}{fixed\PYZus{}rate}
        \PY{n+nb+bp}{self}\PY{o}{.}\PY{n}{fixed\PYZus{}leg\PYZus{}dates} \PY{o}{=} \PYZbs{}
            \PY{n}{generate\PYZus{}swap\PYZus{}dates}\PY{p}{(}\PY{n}{start\PYZus{}date}\PY{p}{,} \PY{l+m+mi}{12} \PY{o}{*} \PY{n}{maturity\PYZus{}years}\PY{p}{)}
        \PY{n+nb+bp}{self}\PY{o}{.}\PY{n}{floating\PYZus{}leg\PYZus{}dates} \PY{o}{=} \PYZbs{}
            \PY{n}{generate\PYZus{}swap\PYZus{}dates}\PY{p}{(}\PY{n}{start\PYZus{}date}\PY{p}{,} \PY{l+m+mi}{12} \PY{o}{*} \PY{n}{maturity\PYZus{}years}\PY{p}{,}
                                \PY{n}{tenor\PYZus{}months}\PY{p}{)}
        
    \PY{k}{def} \PY{n+nf}{annuity}\PY{p}{(}\PY{n+nb+bp}{self}\PY{p}{,} \PY{n}{discount\PYZus{}curve}\PY{p}{)}\PY{p}{:}
        \PY{n}{a} \PY{o}{=} \PY{l+m+mi}{0}
        \PY{k}{for} \PY{n}{i} \PY{o+ow}{in} \PY{n+nb}{range}\PY{p}{(}\PY{l+m+mi}{1}\PY{p}{,} \PY{n+nb}{len}\PY{p}{(}\PY{n+nb+bp}{self}\PY{o}{.}\PY{n}{fixed\PYZus{}leg\PYZus{}dates}\PY{p}{)}\PY{p}{)}\PY{p}{:}
            \PY{n}{a} \PY{o}{+}\PY{o}{=} \PY{n}{discount\PYZus{}curve}\PY{o}{.}\PY{n}{df}\PY{p}{(}\PY{n+nb+bp}{self}\PY{o}{.}\PY{n}{fixed\PYZus{}leg\PYZus{}dates}\PY{p}{[}\PY{n}{i}\PY{p}{]}\PY{p}{)}
        \PY{k}{return} \PY{n}{a}

    \PY{k}{def} \PY{n+nf}{swap\PYZus{}rate}\PY{p}{(}\PY{n+nb+bp}{self}\PY{p}{,} \PY{n}{discount\PYZus{}curve}\PY{p}{,} \PY{n}{libor\PYZus{}curve}\PY{p}{)}\PY{p}{:}
        \PY{n}{s} \PY{o}{=} \PY{l+m+mi}{0}
        \PY{k}{for} \PY{n}{j} \PY{o+ow}{in} \PY{n+nb}{range}\PY{p}{(}\PY{l+m+mi}{1}\PY{p}{,} \PY{n+nb}{len}\PY{p}{(}\PY{n+nb+bp}{self}\PY{o}{.}\PY{n}{floating\PYZus{}leg\PYZus{}dates}\PY{p}{)}\PY{p}{)}\PY{p}{:}
            \PY{n}{F} \PY{o}{=} \PY{n}{libor\PYZus{}curve}\PY{o}{.}\PY{n}{forward\PYZus{}rate}\PY{p}{(}\PY{n+nb+bp}{self}\PY{o}{.}\PY{n}{floating\PYZus{}leg\PYZus{}dates}\PY{p}{[}\PY{n}{j}\PY{o}{\PYZhy{}}\PY{l+m+mi}{1}\PY{p}{]}\PY{p}{)}
            \PY{n}{tau} \PY{o}{=} \PY{p}{(}\PY{n+nb+bp}{self}\PY{o}{.}\PY{n}{floating\PYZus{}leg\PYZus{}dates}\PY{p}{[}\PY{n}{j}\PY{p}{]} \PY{o}{\PYZhy{}} \PYZbs{}
                   \PY{n+nb+bp}{self}\PY{o}{.}\PY{n}{floating\PYZus{}leg\PYZus{}dates}\PY{p}{[}\PY{n}{j}\PY{o}{\PYZhy{}}\PY{l+m+mi}{1}\PY{p}{]}\PY{p}{)}\PY{o}{.}\PY{n}{days} \PY{o}{/} \PY{l+m+mi}{360}
            \PY{n}{P} \PY{o}{=} \PY{n}{discount\PYZus{}curve}\PY{o}{.}\PY{n}{df}\PY{p}{(}\PY{n+nb+bp}{self}\PY{o}{.}\PY{n}{floating\PYZus{}leg\PYZus{}dates}\PY{p}{[}\PY{n}{j}\PY{p}{]}\PY{p}{)}
            \PY{n}{s} \PY{o}{+}\PY{o}{=} \PY{n}{F} \PY{o}{*} \PY{n}{tau} \PY{o}{*} \PY{n}{P}
        \PY{k}{return} \PY{n}{s} \PY{o}{/} \PY{n+nb+bp}{self}\PY{o}{.}\PY{n}{annuity}\PY{p}{(}\PY{n}{discount\PYZus{}curve}\PY{p}{)}
        
    \PY{k}{def} \PY{n+nf}{npv}\PY{p}{(}\PY{n+nb+bp}{self}\PY{p}{,} \PY{n}{discount\PYZus{}curve}\PY{p}{,} \PY{n}{libor\PYZus{}curve}\PY{p}{)}\PY{p}{:}
        \PY{n}{S} \PY{o}{=} \PY{n+nb+bp}{self}\PY{o}{.}\PY{n}{swap\PYZus{}rate}\PY{p}{(}\PY{n}{discount\PYZus{}curve}\PY{p}{,} \PY{n}{libor\PYZus{}curve}\PY{p}{)}
        \PY{n}{A} \PY{o}{=} \PY{n+nb+bp}{self}\PY{o}{.}\PY{n}{annuity}\PY{p}{(}\PY{n}{discount\PYZus{}curve}\PY{p}{)}
        \PY{k}{return} \PY{n+nb+bp}{self}\PY{o}{.}\PY{n}{notional} \PY{o}{*} \PY{p}{(}\PY{n}{S} \PY{o}{\PYZhy{}} \PY{n+nb+bp}{self}\PY{o}{.}\PY{n}{fixed\PYZus{}rate}\PY{p}{)} \PY{o}{*} \PY{n}{A}
\end{Verbatim}
\end{tcolorbox}

    Let's test our class instantiating an IRS with 1M notional, fixed rate
of 5\%, 6 month tenor and a maturity of 4 years; discount and libor
curves are the same as before.

    \begin{tcolorbox}[breakable, size=fbox, boxrule=1pt, pad at break*=1mm,colback=cellbackground, colframe=cellborder]
\begin{Verbatim}[commandchars=\\\{\}]
\PY{n}{pricing\PYZus{}date} \PY{o}{=} \PY{n}{date}\PY{p}{(}\PY{l+m+mi}{2019}\PY{p}{,} \PY{l+m+mi}{11}\PY{p}{,} \PY{l+m+mi}{23}\PY{p}{)}
\PY{n}{irs} \PY{o}{=} \PY{n}{InterestRateSwap}\PY{p}{(}\PY{n}{pricing\PYZus{}date}\PY{p}{,} \PY{l+m+mf}{1e6}\PY{p}{,} \PY{l+m+mf}{0.05}\PY{p}{,} \PY{l+m+mi}{6}\PY{p}{,} \PY{l+m+mi}{4}\PY{p}{)}
\PY{n+nb}{print} \PY{p}{(}\PY{l+s+s2}{\PYZdq{}}\PY{l+s+si}{\PYZob{}:.2f\PYZcb{}}\PY{l+s+s2}{ EUR}\PY{l+s+s2}{\PYZdq{}}\PY{o}{.}\PY{n}{format}\PY{p}{(}\PY{n}{irs}\PY{o}{.}\PY{n}{npv}\PY{p}{(}\PY{n}{dc}\PY{p}{,} \PY{n}{fr}\PY{p}{)}\PY{p}{)}\PY{p}{)}

-160130.58 EUR
    \end{Verbatim}
\end{tcolorbox}

\textbf{Can you guess what could be the swap rate given that the npv is negative
?}

(Remember that we are looking at this contracts from the point of view
of the receiver of the floating leg\ldots{})

    \begin{tcolorbox}[breakable, size=fbox, boxrule=1pt, pad at break*=1mm,colback=cellbackground, colframe=cellborder]
\begin{Verbatim}[commandchars=\\\{\}]
\PY{n+nb}{print} \PY{p}{(}\PY{l+s+s2}{\PYZdq{}}\PY{l+s+si}{\PYZob{}\PYZcb{}}\PY{l+s+s2}{\PYZdq{}}\PY{o}{.}\PY{n}{format}\PY{p}{(}\PY{n}{irs}\PY{o}{.}\PY{n}{swap\PYZus{}rate}\PY{p}{(}\PY{n}{dc}\PY{p}{,} \PY{n}{fr}\PY{p}{)}\PY{p}{)}\PY{p}{)}

0.010254255993254186
    \end{Verbatim}
    \end{tcolorbox}
    
    To check if the we have computed correctly the swap rate we can
instanciate a new IRS with fixed rate equal to the just calculated swap
rate and print its NPV, it should come very close to 0.

    \begin{tcolorbox}[breakable, size=fbox, boxrule=1pt, pad at break*=1mm,colback=cellbackground, colframe=cellborder]
\begin{Verbatim}[commandchars=\\\{\}]
\PY{n}{irs2} \PY{o}{=} \PY{n}{InterestRateSwap}\PY{p}{(}\PY{n}{pricing\PYZus{}date}\PY{p}{,} \PY{l+m+mf}{1e6}\PY{p}{,} \PY{l+m+mf}{0.01025425}\PY{p}{,} \PY{l+m+mi}{6}\PY{p}{,} \PY{l+m+mi}{4}\PY{p}{)}
\PY{n+nb}{print} \PY{p}{(}\PY{l+s+s2}{\PYZdq{}}\PY{l+s+si}{\PYZob{}:.2f\PYZcb{}}\PY{l+s+s2}{ EUR}\PY{l+s+s2}{\PYZdq{}}\PY{o}{.}\PY{n}{format}\PY{p}{(}\PY{n}{irs2}\PY{o}{.}\PY{n}{npv}\PY{p}{(}\PY{n}{dc}\PY{p}{,} \PY{n}{fr}\PY{p}{)}\PY{p}{)}\PY{p}{)}

0.02 EUR
\end{Verbatim}
    \end{tcolorbox}
    
\section{Swaptions}\label{interest-rate-swaptions}

Swaptions are the equivalent of European options for the interest rate
markets. They give the option holder the right but not the obligation,
at the exercise date \(d_{ex}\), to enter into an Interest Rate Swap at
a pre-determined fixed rate.

Clearly the option holder will only choose to do this if the NPV of the
underlying swap at \(d_{ex}\) is positive - looking at the expression
for the NPV of the IRS in terms of the swap rate \(S\) therefore, we can
see that the payoff of the swaption is

\[N\cdot \mathrm{max}(0, S(d_{\mathrm{ex}}) - K)\cdot\sum D(d_{\mathrm{ex}}, d_i^{\mathrm{fixed}})\]

The key issue is now to estimate \(S(d_{\mathrm{ex}})\) in order to
evaluate the payoff of a swaption. This will be shown with two
alternative approaches.

\subsection{Evaluation through Black-Scholes Formula}\label{evaluation-through-black-scholes-formula}

In this case, to evaluate the NPV of this payoff, we'll use a
generalization of the Black-Scholes-Merton formula applied to swaptions:

\[\mathrm{NPV} = N\cdot A\cdot [S \mathcal{N}(d_+) - K\mathcal{N}(d_-)]\]
where \(\mathcal{N}\) represent a normal distribution

\[d_{\pm} = \frac{\mathrm{log}(\frac{S}{K}) \pm \frac{1}{2}\sigma^{2}T}{\sigma\sqrt{T}}\qquad(\sigma~\textrm{is the volatility of the swap rate})\\\]
\[A =\sum_{i=1}^{p}D(d, d_{i}^{\mathrm{fixed}})\qquad\mathrm{(annuity})\]

As an example let's consider a swaption whose underlying 6M-IRS has a
notional of 1M, fixed rate of 1\%, and a maturity of 4 years. In
addition we assume a volatility associated to the swap rate of about
7\%.

\begin{tcolorbox}[breakable, size=fbox, boxrule=1pt, pad at break*=1mm,colback=cellbackground, colframe=cellborder]
\begin{Verbatim}[commandchars=\\\{\}]
\PY{k+kn}{from} \PY{n+nn}{math} \PY{k}{import} \PY{n}{log}
\PY{k+kn}{from} \PY{n+nn}{scipy}\PY{n+nn}{.}\PY{n+nn}{stats} \PY{k}{import} \PY{n}{norm} 
\PY{k+kn}{from} \PY{n+nn}{dateutil}\PY{n+nn}{.}\PY{n+nn}{relativedelta} \PY{k}{import} \PY{n}{relativedelta}

\PY{k}{def} \PY{n+nf}{npvSwaptionBS}\PY{p}{(}\PY{n}{irs}\PY{p}{,} \PY{n}{sigma}\PY{p}{,} 
                  \PY{n}{pricing\PYZus{}date}\PY{p}{,}
                  \PY{n}{exercise\PYZus{}date}\PY{p}{,} 
                  \PY{n}{discount\PYZus{}curve}\PY{p}{,} \PY{n}{libor\PYZus{}curve}\PY{p}{)}\PY{p}{:}
    \PY{n}{T} \PY{o}{=} \PY{p}{(}\PY{n}{exercise\PYZus{}date} \PY{o}{\PYZhy{}} \PY{n}{pricing\PYZus{}date}\PY{p}{)}\PY{o}{.}\PY{n}{days} \PY{o}{/} \PY{l+m+mi}{365}
    \PY{n}{A} \PY{o}{=} \PY{n}{irs}\PY{o}{.}\PY{n}{annuity}\PY{p}{(}\PY{n}{discount\PYZus{}curve}\PY{p}{)}
    \PY{n}{S} \PY{o}{=} \PY{n}{irs}\PY{o}{.}\PY{n}{swap\PYZus{}rate}\PY{p}{(}\PY{n}{discount\PYZus{}curve}\PY{p}{,} \PY{n}{libor\PYZus{}curve}\PY{p}{)}
    \PY{n}{K} \PY{o}{=} \PY{n}{irs}\PY{o}{.}\PY{n}{fixed\PYZus{}rate}
    \PY{n}{N} \PY{o}{=} \PY{n}{irs}\PY{o}{.}\PY{n}{notional}
    
    \PY{n}{d\PYZus{}plus} \PY{o}{=} \PY{p}{(}\PY{n}{log}\PY{p}{(}\PY{n}{S}\PY{o}{/}\PY{n}{K}\PY{p}{)} \PY{o}{+} \PY{l+m+mf}{0.5} \PY{o}{*} \PY{n}{sigma}\PY{o}{*}\PY{o}{*}\PY{l+m+mi}{2} \PY{o}{*} \PY{n}{T}\PY{p}{)} \PY{o}{/} \PY{p}{(}\PY{n}{sigma} \PY{o}{*} \PY{n}{T}\PY{o}{*}\PY{o}{*}\PY{l+m+mf}{0.5}\PY{p}{)}
    \PY{n}{d\PYZus{}minus} \PY{o}{=} \PY{p}{(}\PY{n}{log}\PY{p}{(}\PY{n}{S}\PY{o}{/}\PY{n}{K}\PY{p}{)} \PY{o}{\PYZhy{}} \PY{l+m+mf}{0.5} \PY{o}{*} \PY{n}{sigma}\PY{o}{*}\PY{o}{*}\PY{l+m+mi}{2} \PY{o}{*} \PY{n}{T}\PY{p}{)} \PY{o}{/} \PY{p}{(}\PY{n}{sigma} \PY{o}{*} \PY{n}{T}\PY{o}{*}\PY{o}{*}\PY{l+m+mf}{0.5}\PY{p}{)}
    \PY{k}{return} \PY{n}{irs}\PY{o}{.}\PY{n}{notional} \PY{o}{*} \PY{n}{A} \PY{o}{*} \PY{p}{(}\PY{n}{S} \PY{o}{*} \PY{n}{norm}\PY{o}{.}\PY{n}{cdf}\PY{p}{(}\PY{n}{d\PYZus{}plus}\PY{p}{)} \PY{o}{\PYZhy{}} \PY{n}{K} \PY{o}{*} \PY{n}{norm}\PY{o}{.}\PY{n}{cdf}\PY{p}{(}\PY{n}{d\PYZus{}minus}\PY{p}{)}\PY{p}{)}

\PY{n}{sigma} \PY{o}{=} \PY{l+m+mf}{0.07}
\PY{n}{irs} \PY{o}{=} \PY{n}{InterestRateSwap}\PY{p}{(}\PY{n}{pricing\PYZus{}date}\PY{p}{,} \PY{l+m+mf}{1e6}\PY{p}{,} \PY{l+m+mf}{0.01}\PY{p}{,} \PY{l+m+mi}{6}\PY{p}{,} \PY{l+m+mi}{4}\PY{p}{)}
\PY{n}{exercise\PYZus{}date} \PY{o}{=} \PY{n}{start\PYZus{}date} \PY{o}{+} \PY{n}{relativedelta}\PY{p}{(}\PY{n}{years}\PY{o}{=}\PY{l+m+mi}{4}\PY{p}{)}

\PY{n}{npv} \PY{o}{=} \PY{n}{npvSwaptionBS}\PY{p}{(}\PY{n}{irs}\PY{p}{,} \PY{n}{sigma}\PY{p}{,} \PY{n}{pricing\PYZus{}date}\PY{p}{,} 
                    \PY{n}{exercise\PYZus{}date}\PY{p}{,} \PY{n}{dc}\PY{p}{,} \PY{n}{fr}\PY{p}{)}
\PY{n+nb}{print}\PY{p}{(}\PY{l+s+s2}{\PYZdq{}}\PY{l+s+s2}{Swaption NPV with BS: }\PY{l+s+si}{\PYZob{}:.3f\PYZcb{}}\PY{l+s+s2}{ EUR}\PY{l+s+s2}{\PYZdq{}}\PY{o}{.}\PY{n}{format}\PY{p}{(}\PY{n}{npv}\PY{p}{)}\PY{p}{)}

Swaption NPV with BS: 3330.741 EUR
    \end{Verbatim}
\end{tcolorbox}

\paragraph{Evaluation through Monte-Carlo Simulation}\label{evaluation-through-monte-carlo-simulation}

In this second case we start from the current swap rate \(S(d)\)
evaluated at the pricing date \(d\), and assume that it follows a
log-normal stochastic process, i.e.~its distribution at
\(d_{\mathrm{ex}}\) (exercise date) is
\(S(d_{\mathrm{ex}}) = S(d)\mathrm{exp}(-\frac{1}{2}\sigma^{2}T+\sigma\sqrt{T}\epsilon)\)
where \(\epsilon\approx\mathcal{N}(0,1)\). Notice that it is assumed
that the \emph{drift} rate in the evolution of the swap rate is zero.
Given that the discounted payoff is given by:

\[N\cdot \mathrm{max}(0, S(d_{\mathrm{ex}}) - K)\cdot\sum D(d_{\mathrm{ex}}, d_i^{\mathrm{fixed}})\]
to perform the simulation we can:

\begin{itemize}
\tightlist
\item
  sample the normal distribution \(\mathcal{N}(0, 1)\) to calculate a
  large number of scenarios for \(S(d_{\mathrm{ex}})\);
\item
  evaluate the underlying swap's NPV at the expiry date, and
  consequently the swaption's payoff, for each scenario;
\item
  take the average of these values to get the final estimate.
\end{itemize}

    \begin{tcolorbox}[breakable, size=fbox, boxrule=1pt, pad at break*=1mm,colback=cellbackground, colframe=cellborder]
\begin{Verbatim}[commandchars=\\\{\}]
\PY{k+kn}{import} \PY{n+nn}{numpy} \PY{k}{as} \PY{n+nn}{np}
\PY{k+kn}{from} \PY{n+nn}{math} \PY{k}{import} \PY{n}{exp}\PY{p}{,} \PY{n}{sqrt}
\PY{k+kn}{from} \PY{n+nn}{numpy}\PY{n+nn}{.}\PY{n+nn}{random} \PY{k}{import} \PY{n}{normal}\PY{p}{,} \PY{n}{seed}

\PY{c+c1}{\PYZsh{} define the number of Monte Carlo scenarios}
\PY{n}{n\PYZus{}scenarios} \PY{o}{=} \PY{l+m+mi}{50000}
\PY{n}{discounted\PYZus{}payoffs} \PY{o}{=} \PY{p}{[}\PY{p}{]}
\PY{n}{seed}\PY{p}{(}\PY{l+m+mi}{1}\PY{p}{)}

\PY{n}{T} \PY{o}{=} \PY{p}{(}\PY{n}{exercise\PYZus{}date} \PY{o}{\PYZhy{}} \PY{n}{pricing\PYZus{}date}\PY{p}{)}\PY{o}{.}\PY{n}{days} \PY{o}{/} \PY{l+m+mi}{365}
\PY{n}{A} \PY{o}{=} \PY{n}{irs}\PY{o}{.}\PY{n}{annuity}\PY{p}{(}\PY{n}{dc}\PY{p}{)}
\PY{n}{S} \PY{o}{=} \PY{n}{irs}\PY{o}{.}\PY{n}{swap\PYZus{}rate}\PY{p}{(}\PY{n}{dc}\PY{p}{,} \PY{n}{fr}\PY{p}{)}
    
\PY{k}{for} \PY{n}{i\PYZus{}scenario} \PY{o+ow}{in} \PY{n+nb}{range}\PY{p}{(}\PY{n}{n\PYZus{}scenarios}\PY{p}{)}\PY{p}{:}
    \PY{n}{S\PYZus{}simulated} \PY{o}{=} \PY{n}{S} \PY{o}{*} \PY{n}{exp}\PY{p}{(}\PY{o}{\PYZhy{}}\PY{l+m+mf}{0.5} \PY{o}{*} \PY{n}{sigma} \PY{o}{*} \PY{n}{sigma} \PY{o}{*} \PY{n}{T} \PY{o}{+}
                          \PY{n}{sigma} \PY{o}{*} \PY{n}{sqrt}\PY{p}{(}\PY{n}{T}\PY{p}{)} \PY{o}{*} \PY{n}{normal}\PY{p}{(}\PY{p}{)}\PY{p}{)}
    
    \PY{c+c1}{\PYZsh{} calculate the swap NPV in this scenario}
    \PY{n}{swap\PYZus{}npv} \PY{o}{=} \PY{n}{irs}\PY{o}{.}\PY{n}{notional} \PY{o}{*} \PY{p}{(}\PY{n}{S\PYZus{}simulated} \PY{o}{\PYZhy{}} \PY{n}{irs}\PY{o}{.}\PY{n}{fixed\PYZus{}rate}\PY{p}{)} \PY{o}{*} \PY{n}{A}
    
    \PY{c+c1}{\PYZsh{} add the discounted payoff of the swaption, in this scenario, to the list}
    \PY{n}{discounted\PYZus{}payoffs}\PY{o}{.}\PY{n}{append}\PY{p}{(}\PY{n+nb}{max}\PY{p}{(}\PY{l+m+mi}{0}\PY{p}{,} \PY{n}{swap\PYZus{}npv}\PY{p}{)}\PY{p}{)}
    
\PY{c+c1}{\PYZsh{} calculate the NPV of the swaption }
\PY{c+c1}{\PYZsh{} by taking the average of the discounted }
\PY{c+c1}{\PYZsh{} payoffs across all the scenarios}
\PY{n}{npv\PYZus{}mc} \PY{o}{=} \PY{n}{np}\PY{o}{.}\PY{n}{mean}\PY{p}{(}\PY{n}{discounted\PYZus{}payoffs}\PY{p}{)}
    
\PY{c+c1}{\PYZsh{} calculate the MC error estimate as}
\PY{c+c1}{\PYZsh{} 99\PYZpc{} confidence interval}
\PY{n}{npv\PYZus{}error} \PY{o}{=} \PY{l+m+mf}{2.57} \PY{o}{*} \PY{n}{np}\PY{o}{.}\PY{n}{std}\PY{p}{(}\PY{n}{discounted\PYZus{}payoffs}\PY{p}{)}\PY{o}{/}\PY{n}{sqrt}\PY{p}{(}\PY{n}{n\PYZus{}scenarios}\PY{p}{)}

\PY{n+nb}{print}\PY{p}{(}\PY{l+s+s2}{\PYZdq{}}\PY{l+s+s2}{Swaption NPV: }\PY{l+s+si}{\PYZob{}:.2f\PYZcb{}}\PY{l+s+s2}{ EUR (+/- }\PY{l+s+si}{\PYZob{}:.2f\PYZcb{}}\PY{l+s+s2}{ EUR with 99}\PY{l+s+si}{\PYZpc{} confidence)}\PY{l+s+s2}{\PYZdq{}}\PYZbs{}
      \PY{o}{.}\PY{n}{format}\PY{p}{(}\PY{n}{npv\PYZus{}mc}\PY{p}{,} \PY{n}{npv\PYZus{}error}\PY{p}{)}\PY{p}{)}

Swaption NPV: 3351.42 EUR (+/- 56.66 EUR with 99\% confidence)
    \end{Verbatim}
\end{tcolorbox}

    Note that this is not \emph{strictly speaking} the correct way of
calculating the swaption NPV, the reason being that one should calculate
the swap NPV at the expiry date of the swaption, apply the payoff
function max(0, \ldots{}) and \emph{then} discount from the expiry date
to today.

However, it's simpler to calculate it as above and it doesn't make any
difference for the result, since

\[ DF\cdot \mathrm{max}(0, \mathrm{SwapNPVAtExpiry}) = \mathrm{max}(0, DF \cdot\mathrm{SwapNPVAtExpiry}) \]

    The NPV calculated via the Black-Scholes-Merton formula falls within the
confidence interval produced by the Monte Carlo simulation, so we can
assert that the two methods are in agreement:

\begin{itemize}
\tightlist
\item
  Swaption NPV (BS): \euro{3330.74}
\item
  Swaption NPV (MC): \euro{3351.42}
\end{itemize}

%\subsubsection{Exercise 6.3}\label{exercise-6.3}}
%
%Using the function \texttt{normal} of \texttt{numpy.random} simulate the
%price of a stock which evolves according to a log-normal stochastic
%process with a daily rate of return \(\mu=0.1\) and a volatility
%\(\sigma=0.15\) for 30 days.
%
%Also plot the price. Try to play with \(\mu\) and \(\sigma\) to see how
%the plot changes.

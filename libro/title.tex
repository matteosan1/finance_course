\usepackage[T1]{fontenc}
\usepackage{lmodern}
\usepackage{url}
\usepackage[svgnames]{xcolor}
\ifpdf
\usepackage{pdfcolmk}
\fi
%% check if using xelatex rather than pdflatex
\ifxetex
\usepackage{fontspec}
\fi
\usepackage{graphicx}
%%\usepackage{hyperref}
%% drawing package
\usepackage{tikz}
%% for dingbats
\usepackage{pifont}
\providecommand{\HUGE}{\Huge}% if not using memoir
\newlength{\drop}% for my convenience
%% specify the Webomints family
\newcommand*{\wb}[1]{\fontsize{#1}{#2}\usefont{U}{webo}{xl}{n}}
%% select a (FontSite) font by its font family ID
\newcommand*{\FSfont}[1]{\fontencoding{T1}\fontfamily{#1}\selectfont}
%% if you don’t have the FontSite fonts either \renewcommand*{\FSfont}[1]{}
%% or use your own choice of family.
%% select a (TeX Font) font by its font family ID
\newcommand*{\TXfont}[1]{\fontencoding{T1}\fontfamily{#1}\selectfont}
%% Generic publisher’s logo
\newcommand*{\plogo}{\fbox{$\mathcal{PL}$}}
45
%% Some shades
\defincolor{Dark}{gray}{0.2}
\defincolor{MedDark}{gray}{0.4}
\defincolor{Medium}{gray}{0.6}
\defincolor{Light}{gray}{0.8}
%%%% Additional font series macros
\makeatletter
%%%% light series
%% e.g., kernel doc, section s: line 12 or thereabouts
\DeclareRobustCommand\ltseries
{\not@math@alphabet\ltseries\relax
\fontseries\ltdefault\selectfont}
%% e.g., kernel doc, section t: line 32 or thereabouts
\newcommand{\ltdefault}{l}
%% e.g., kernel doc, section v: line 19 or thereabouts
\DeclareTextFontCommand{\textlt}{\ltseries}
% heavy(bold) series
\DeclareRobustCommand\hbseries
{\not@math@alphabet\hbseries\relax
\fontseries\hbdefault\selectfont}
\newcommand{\hbdefault}{hb}
\DeclareTextFontCommand{\texthb}{\hbseries}
\makeatother
\begin{document}
\pagestyle{empty}
\titleX
\clearpage

Some of the original title pages used color. In such cases I have tried to match the
colors using palette provided by the svgnames option to the xcolor package.
In other cases I have used the graphicx package to scale or rotate elements of
the design.
You may, of course, want to add or subtract material from the examples, such as
including a publication date or removing the publisher, or use a dicerent font (or fonts).
The design is in your hands.
I used slightly dicerent coding than given below to display the graphic examples
in order to produce slightly smaller displays than the following macros will provide. If
you are interested look in the preamble of the source file titlepages.tex.

A.1 titleJT (graphic on page 4)
This is based on the title page of Jan Tschichold: Typographer [McL75]. The book was
designed by Herbert Spencer and Christine Charlton. The text type is 12/14 Monotype
Garamond 156 with chapter headings in 24 point Sabon Semi-bold. Page size is 57pc
by 57pc.
I used Garamond for the printed example.
\newcommand*{\titleJT}{\begingroup% Jan Tschichold: typographer
\FSfont{5gm} % FontSite Garamond
\drop = 0.08\textheight
\vspace*{\drop}
\hspace*{0.3\textwidth}
{\Large The Author}\\[2\drop]
\hspace*{0.3\textwidth}{\Huge\itshape The Big Book of}\par
{\raggedleft\Huge\itshape Conundrums\par}
\vfill
\hspace*{0.3\textwidth}{\Large \plogo}\\[0.5\baselineskip]
\hspace*{0.3\textwidth}{\Large The Publisher}
\vspace*{\drop}
\endgroup}



A.2 titleTH (graphic on page 5)
This is based on the title page of Thames and Hudson’s Manual of Typography by
Ruari McLean [McL80]. The main element of the title was set using a red font with
everything else in normal black. If you use this you need the xcolor package. The
page size is 38 by 57pc.
\newcommand*{\titleTH}{\begingroup% T&H Typography
\raggedleft
\vspace*{\baselineskip}
{\Large The Author}\\[0.167\textheight]
{\bfseries The Big Book of}\\[\baselineskip]
{\textcolor{Red}{\Huge CONUNDRUMS}}\\[\baselineskip]
{\small With 123 illustrations}\par
\vfill
{\Large The Publisher \plogo}\par
\vspace*{3\baselineskip}
\endgroup}

A.7 titleAT (graphic on page 10)
This is based on the title page from Anatomy of a Typeface [Law90]. In the original
both the title, in a decorative font, and the publisher’s logo were in red. You will need
the xcolor package for this style. The text was set in Galliard, designed by Matthew
Carter. The page size is 36 by 54pc.
I used Mona Lisa for the main title and Bergamo (Bembo) for the remainder in the
printed example.
\newcommand*{\titleAT}{\begingroup% Anatomy of a Typeface
\FSfont{5bp} % FontSite Bergamo (Bembo)
\drop=0.1\textheight
\vspace*{\drop}
\rule{\textwidth}{1pt}\par
\vspace{2pt}\vspace{-\baselineskip}
\rule{\textwidth}{0.4pt}\par
\vspace{0.5\drop}
\centering
\textcolor{Red}{
{\FSfont{5ml} % FontSite Mona Lisa
\Huge THE BOOK}\\[0.5\baselineskip]
{\FSfont{5ml}
\Large OF}\\[0.75\baselineskip]
{\FSfont{5ml}
\Huge CONUNDRUMS}}\par
\vspace{0.25\drop}
\rule{0.3\textwidth}{0.4pt}\par
\vspace{\drop}
{\Large \scshape The Author}\par
\vfill
{\large \textcolor{Red}{\plogo}}\\[0.5\baselineskip]
{\large\scshape the publisher}\par
\vspace*{\drop}
\endgroup}

A.9 titleAM (graphic on page 12)
This is based on a title page designed by Will Carter for a limited edition of Samual
Taylor Coleridge’s The Rime of the Ancient Mariner from The Design of Books [Wil93].
The book was typeset using Bembo.
I used Bergamo (Bembo) for the printed example.
\newcommand*{\titleAM}{\begingroup% Ancient Mariner, AW p. 151
\FSfont{5bo} % FontSite Bergamo (Bembo)
\drop = 0.12\textheight
\centering
\vspace*{\drop}
{\large The Author}\\[\baselineskip]
{\Huge THE BIG BOOK}\\[\baselineskip]
{\Large OF}\\[\baselineskip]
{\Huge CONUNDRUMS}\\[\baselineskip]
{\scshape with ten engravings}\\
{\scshape and with a foreword by}\\
{\large\scshape an other}\\[\drop]
{\plogo}\\[0.5\baselineskip]
{\small\scshape the publisher}\par
{\small\scshape year}\par
51
\vfill\null
\endgroup}

A.14 titleCC (graphic on page 17)
A title page for a work that consists of two or more volumes. It is based on a Survey
of Cambridge [RCHM59] which was published in 2 Parts, plus a set of maps. The page
size is 50 by 63.5pc. The main element of the title was set using a red font, everything
else was as normal. If you try this you need to use the xcolor package.
\newcommand*{\titleCC}{\begingroup% City of Cambridge
\drop=0.1\textheight
\vspace*{\drop}
\centering
{\Large\itshape THE BIG BOOK OF}\\[0.5\drop]
{\textcolor{Red}{\HUGE\bfseries CONUNDRUMS}}\par
\vspace{\drop}
{\LARGE\itshape VOLUME 1: SOCIAL AND MORAL}\par
\vfill
{\Large THE AUTHOR}\par
\vfill
{\plogo}\\[0.5\baselineskip]
54
{\itshape THE PUBLISHER}\par
{\scshape year}\par
%\vfill
\vspace*{\drop}
\endgroup}

A.22 titleTMB (graphic on page 25)
A somewhat old fashioned title page style which, however, suits the period of the book
— Three Men in a Boat [Jer64], first published in 1889. The edition that I have has a
page size of 31 by 51pc.
I had to do a lot of hand adjustments to the spacing to get a reasonable look for the
example.
\newcommand*{\titleTMB}{\begingroup% Three Men in a Boat
\drop=0.1\textheight
\centering
\settowidth{\unitlength}{\LARGE THE BOOK OF CONUNDRUMS}
\vspace*{\baselineskip}
{\large\scshape the author}\\[\baselineskip]
\rule{\unitlength}{1.6pt}\vspace*{-\baselineskip}\vspace*{2pt}
\rule{\unitlength}{0.4pt}\\[\baselineskip]
{\LARGE THE BOOK OF CONUNDRUMS}\\[\baselineskip]
59
{\itshape puzzles for the mind}\\[0.2\baselineskip]
\rule{\unitlength}{0.4pt}\vspace*{-\baselineskip}\vspace{3.2pt}
\rule{\unitlength}{1.6pt}\\[\baselineskip]
{\large\scshape drawings by the artist}\par
\vfill
{\large\scshape the publisher}\\[\baselineskip]
{\small\scshape year}\par
\vspace*{\drop}
\endgroup}

A.30 titleJA (graphic on page 33)
This example is based on the title page of Jost Amman’s Cuts of Craft-Workers [Amm08]
published by the Incline Press.2 The cuts — wood cuts of 16th Century craftsmen —
2A more affordable version is published by Dover in 1976 with the title The Book of
Trades (ISBN 0-486-22886-X).
64
were introduced by Veronica Speedwell who, I believe, also designed the book. The
page size is 30 by 47pc.
I used Abbot Old Style for the main title and Belwe for the remainder.
\newcommand*{\titleJA}{\begingroup% Jost Amman
\FSfont{5bl}% FontSite Belwe
\begin{center}
\drop=0.2\textheight
\vspace*{0.5\drop}
\Large THE AUTHOR’S \\[\baselineskip]
{\FSfont{5at}% FontSite Abbot Old Style
\huge\textcolor{Red}{Some Conundrums}}\\[2\baselineskip]
\large \textit{with an intruction by} \\
SOMEONE ELSE\par
\vfill
YEAR \\
{\color{Red} \rule{\txtwidth}{0.4pt}\vspace*{-\baselineskip}\vspace{3pt}
\rule{\txtwidth}{0.4pt}} \\[\baselineskip]
\Large THE PUBLISHER
\end{center}
\vspace*{\drop}
\mbox{}
\endgroup}

A.33 titlePP (graphic on page 36)
This example is based on the title page of Printing Poetry [Bur80] written and designed
by Clicord and printed letterpress. The page size is 41 by 64.5pc.
66
The original was composed in Monotype Italian Old Style but not having that I have
used Jenson Recut (Centaur) instead
\newcommand*{\titlePP}{\begingroup% Printing Poetry
\FSfont{5jr}% FontSite Jenson Recut (Centaur)
\drop=0.1\textheight
\vspace*{\drop}
\begin{raggedleft}
{\HUGE PUZZLING}\\[\baselineskip]
{A WORKBOOK FOR RESOLUTIONS}\\[1.1\baselineskip]
{\HUGE CONUNDRUMS} \\[\baselineskip]
{\Large BY THE AUTHOR}\par
\end{raggedleft}
\vfill
\begin{center}
{\large THE PUBLISHER YEAR}
\end{center}
\vspace*{\drop}
%\mbox{}
\endgroup}

A.36 titleGWP (graphic on page 39)
An example based on The Complete Works of the Gawain-Poet [Gaw65], designed by
Adrian for the University of Chicago Press. The page size is 40 by 57pc.
I used Cipollini for the main title and Garamond for the remainder. In the center of
the original titlepage there is a woodcut of a knight on horseback; I have used a large
‘?’ instead.
\newcommand*{\titleGWP}{\begingroup% Gawain Poet
\FSfont{5gm}% FontSite Garamond
\drop=0.1\textheight
\vspace*{\drop}
\begin{center}
\FSfont{5ci}% FontSite Cipollini
\Huge\color{Red}
THE COMPLETE WORKS OF THE CONUNDRUM POSER
\end{center}
\vfill
\begin{center}
{\normalfont\fontsize{256pt}{256pt}\selectfont ?}
\end{center}
\vfill
\begin{center}
\large
{\itshape In a Readable Version with a Critical Introduction \\
by The Critic \\
\normalsize
Woodcuts by The Artist} \\
{\color{Red} \rule{\txtwidth}{0.4pt}}
THE PUBLISHER
\end{center}
\vspace*{0.5\drop}
\endgroup}























\documentclass[letterpaper]{article}
\usepackage{amsmath}
\usepackage{tikz}
\usepackage{epigraph}
\usepackage{lipsum}

\renewcommand\epigraphflush{flushright}
\renewcommand\epigraphsize{\normalsize}
\setlength\epigraphwidth{0.7\textwidth}

\definecolor{titlepagecolor}{cmyk}{1,.60,0,.40}

\DeclareFixedFont{\titlefont}{T1}{ppl}{b}{it}{0.5in}

\makeatletter                       
\def\printauthor{%                  
    {\large \@author}}              
\makeatother
\author{%
    Author 1 name \\
    Department name \\
    \texttt{email1@example.com}\vspace{20pt} \\
    Author 2 name \\
    Department name \\
    \texttt{email2@example.com}
    }

% The following code is borrowed from: https://tex.stackexchange.com/a/86310/10898

\newcommand\titlepagedecoration{%
\begin{tikzpicture}[remember picture,overlay,shorten >= -10pt]

\coordinate (aux1) at ([yshift=-15pt]current page.north east);
\coordinate (aux2) at ([yshift=-410pt]current page.north east);
\coordinate (aux3) at ([xshift=-4.5cm]current page.north east);
\coordinate (aux4) at ([yshift=-150pt]current page.north east);

\begin{scope}[titlepagecolor!40,line width=12pt,rounded corners=12pt]
\draw
  (aux1) -- coordinate (a)
  ++(225:5) --
  ++(-45:5.1) coordinate (b);
\draw[shorten <= -10pt]
  (aux3) --
  (a) --
  (aux1);
\draw[opacity=0.6,titlepagecolor,shorten <= -10pt]
  (b) --
  ++(225:2.2) --
  ++(-45:2.2);
\end{scope}
\draw[titlepagecolor,line width=8pt,rounded corners=8pt,shorten <= -10pt]
  (aux4) --
  ++(225:0.8) --
  ++(-45:0.8);
\begin{scope}[titlepagecolor!70,line width=6pt,rounded corners=8pt]
\draw[shorten <= -10pt]
  (aux2) --
  ++(225:3) coordinate[pos=0.45] (c) --
  ++(-45:3.1);
\draw
  (aux2) --
  (c) --
  ++(135:2.5) --
  ++(45:2.5) --
  ++(-45:2.5) coordinate[pos=0.3] (d);   
\draw 
  (d) -- +(45:1);
\end{scope}
\end{tikzpicture}%
}

\begin{document}
\begin{titlepage}

\noindent
\titlefont Hardy's Theorem\par
\epigraph{Pure mathematics is on the whole distinctly more useful than applied. For what is useful above all is technique, and mathematical technique is taught mainly through pure mathematics.}%
{\textit{London 1941}\\ \textsc{G. H. Hardy}}
\null\vfill
\vspace*{1cm}
\noindent
\hfill
\begin{minipage}{0.35\linewidth}
    \begin{flushright}
        \printauthor
    \end{flushright}
\end{minipage}
%
\begin{minipage}{0.02\linewidth}
    \rule{1pt}{125pt}
\end{minipage}
\titlepagedecoration
\end{titlepage}
\lipsum[1-2]
\end{document}




\documentclass{article}
\usepackage{tikz,fouriernc}
\usetikzlibrary{calc}
\definecolor{lime}{RGB}{0,0,128}
\begin{document}
    \begin{titlepage}
        \begin{tikzpicture}[overlay,remember picture]
            \fill[lime!10] (current page.south east) rectangle (current page.north west);
            \fill[lime,even odd rule] (current page.south west) rectangle ([xshift=1.5cm]current page.north west) ($(current page.north west)!.5!(current page.south west)$) arc(180:270:3) ($(current page.north west)!.5!(current page.south west)$) arc(270:450:3.5) ([yshift=7cm]$(current page.north west)!.5!(current page.south west)$)  arc(180:90:3);
%           \draw[lime!10,double=lime!10,double distance=2mm] (current page.south west) --+ (2,2) --+ (-2,6) --+ (2,8) --+ (-2,9) --+ (2,13) --+ (-2,18) --+ (2,30);
                \draw[lime,thin] (13,-1) node[lime,above left] {\Huge\it Teacher} --+ (0,3) node[lime,below right] {\Huge Q11};
            \node[yshift=-5cm,xslant=1,yscale=1.6,lime!40,opacity=.2] at (9.5,-6.5) {\scalebox{6}{Mathematik}};
            \node[yshift=-5cm,lime] at (8,-7) {\scalebox{6}{Mathematik}};
                \draw[lime,fill=lime!20] ([xshift=-1cm]current page.north east) -- ([yshift=-1cm]current page.north east) -- ([yshift=-2cm]current page.north east) -- ([xshift=-2cm]current page.north east);
            \path ([xshift=-1cm]current page.north east) -- ([yshift=-1cm]current page.north east) node[midway,sloped,below=.1cm,lime] {My Name};
        \end{tikzpicture}
    \end{titlepage}
\end{document}



\documentclass{amsart}
\usepackage[margin=1in]{geometry}
\usepackage{afterpage}
\usepackage{tikz}
\usetikzlibrary{fadings}

\begin{document}

\DeclareFixedFont{\titlefont}{T1}{ppl}{b}{}{0.7in}
\DeclareFixedFont{\subtitlefont}{T1}{ppl}{b}{}{0.4in}
\afterpage{\restoregeometry}
\newgeometry{left=1in, right=1in,top=1in, bottom=0in}
\definecolor{mytan}{HTML}{F6D5A8}
\pagecolor{mytan}\afterpage{\nopagecolor}

\thispagestyle{empty}
\begin{flushright}
\titlefont Algebraic Geometry\\
\subtitlefont UT Austin, Spring 2016
\end{flushright}
\vfill
\begin{center}
\begin{tikzpicture}
\node[scope fading=north, inner sep=0pt, outer sep=0pt]{
 \makebox[\textwidth]{\includegraphics[width=\paperwidth]{../the-great-wave-off-kanagawa}}
};
\end{tikzpicture}
\end{center}

% for additional pages after the title, uncomment the following line first:
% \clearpage

\end{document}

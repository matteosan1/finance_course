\chapter{Interest Rate Swap and Swaptions}\label{introduction-to-python---lesson-8}

\section{Exercises}

\begin{Exercise}[title={(Forward rate curve class)}]
Write a ForwardRateCurve class (for EURIBOR/LIBOR rate curve) which
doesn't compute discount factors but only interplatates forward rates;
then add it to the \texttt{finmarkets} module (this function is used to
define the Libor curve needed throughout this lesson).
\end{Exercise}
\begin{Answer}
In this case it is enough to write a new \texttt{class} that has three
attributes: a today date, a set of pillar\_dates and the corresponding
rates. There will be just a single method \texttt{forward\_rate} which
returns the corresponding interpolated rate.

\begin{tcolorbox}[size=fbox, boxrule=1pt, colback=cellbackground, colframe=cellborder]
\begin{Verbatim}[commandchars=\\\{\}]
\PY{k+kn}{import} \PY{n+nn}{numpy}
        
\PY{c+c1}{\PYZsh{} an EURIBOR or LIBOR rate curve}
\PY{c+c1}{\PYZsh{} doesn\PYZsq{}t calculate discount factors, only interpolates forward rates}
\PY{k}{class} \PY{n+nc}{ForwardRateCurve}\PY{p}{(}\PY{n+nb}{object}\PY{p}{)}\PY{p}{:}
   
   \PY{c+c1}{\PYZsh{} the special \PYZus{}\PYZus{}init\PYZus{}\PYZus{} method defines how to}
   \PY{c+c1}{\PYZsh{} construct instances of the class}
   \PY{k}{def} \PY{n+nf}{\PYZus{}\PYZus{}init\PYZus{}\PYZus{}}\PY{p}{(}\PY{n+nb+bp}{self}\PY{p}{,} \PY{n}{pillar\PYZus{}dates}\PY{p}{,} \PY{n}{rates}\PY{p}{)}\PY{p}{:}
       
       \PY{c+c1}{\PYZsh{} we just store the arguments as attributes of the instance}
       \PY{n+nb+bp}{self}\PY{o}{.}\PY{n}{today} \PY{o}{=} \PY{n}{pillar\PYZus{}dates}\PY{p}{[}\PY{l+m+mi}{0}\PY{p}{]}
       \PY{n+nb+bp}{self}\PY{o}{.}\PY{n}{rates} \PY{o}{=} \PY{n}{rates}
       
       \PY{n+nb+bp}{self}\PY{o}{.}\PY{n}{pillar\PYZus{}days} \PY{o}{=} \PY{p}{[}
           \PY{p}{(}\PY{n}{pillar\PYZus{}date} \PY{o}{\PYZhy{}} \PY{n+nb+bp}{self}\PY{o}{.}\PY{n}{today}\PY{p}{)}\PY{o}{.}\PY{n}{days}
           \PY{k}{for} \PY{n}{pillar\PYZus{}date} \PY{o+ow}{in} \PY{n}{pillar\PYZus{}dates}
       \PY{p}{]}
       
       
   \PY{c+c1}{\PYZsh{} interpolates the forward rates stored in the instance}
   \PY{k}{def} \PY{n+nf}{forward\PYZus{}rate}\PY{p}{(}\PY{n+nb+bp}{self}\PY{p}{,} \PY{n}{d}\PY{p}{)}\PY{p}{:}
       \PY{n}{d\PYZus{}days} \PY{o}{=} \PY{p}{(}\PY{n}{d} \PY{o}{\PYZhy{}} \PY{n+nb+bp}{self}\PY{o}{.}\PY{n}{today}\PY{p}{)}\PY{o}{.}\PY{n}{days}
       \PY{k}{return} \PY{n}{numpy}\PY{o}{.}\PY{n}{interp}\PY{p}{(}\PY{n}{d\PYZus{}days}\PY{p}{,} \PY{n+nb+bp}{self}\PY{o}{.}\PY{n}{pillar\PYZus{}days}\PY{p}{,} \PY{n+nb+bp}{self}\PY{o}{.}\PY{n}{rates}\PY{p}{)}
\end{Verbatim}

\end{tcolorbox}
\end{Answer}

\begin{Exercise}[title={(Monte Carlo Simulation I)}]
Using the function \texttt{randint} of the module \texttt{random} make a
Monte Carlo simulation of rolling three dices to check the probability
of getting the same values on the three of them.

From the probability theory you should expect:

\[P_{d1=d2=d3} = \frac{1}{6}\cdot\frac{1}{6}\cdot\frac{1}{6}\cdot 6 = \frac{1}{36} = 0.0278\]
\end{Exercise}
\begin{Answer}
\begin{tcolorbox}[size=fbox, boxrule=1pt, colback=cellbackground, colframe=cellborder]
\begin{Verbatim}[commandchars=\\\{\}]
\PY{k+kn}{from} \PY{n+nn}{random} \PY{k}{import} \PY{n}{seed}\PY{p}{,} \PY{n}{randint}
        
\PY{n}{seed}\PY{p}{(}\PY{l+m+mi}{1}\PY{p}{)}
        
\PY{n}{trials} \PY{o}{=} \PY{l+m+mi}{10000000}
\PY{n}{success} \PY{o}{=} \PY{l+m+mi}{0}
\PY{k}{for} \PY{n}{\PYZus{}} \PY{o+ow}{in} \PY{n+nb}{range}\PY{p}{(}\PY{n}{trials}\PY{p}{)}\PY{p}{:}{6}\PY{p}{)}\PY{p}{,} \PY{n}{randint}\PY{p}{(}\PY{l+m+mi}{1}\PY{p}{,} \PY{l+m+mi}{6}\PY{p}{)}\PY{p}{,} \PY{n}{randint}\PY{p}{(}\PY{l+m+mi}{1}\PY{p}{,} \PY{l+m+mi}{6}\PY{p}{)}    
    \PY{k}{if} \PY{n}{d1} \PY{o}{==} \PY{n}{d2} \PY{o+ow}{and} \PY{n}{d2} \PY{o}{==} \PY{n}{d3}\PY{p}{:}
        \PY{n}{success} \PY{o}{+}\PY{o}{=} \PY{l+m+mi}{1}
    
\PY{n+nb}{print} \PY{p}{(}\PY{l+s+s2}{\PYZdq{}}\PY{l+s+s2}{The probability to get three equal dice is }\PY{l+s+si}{\PYZob{}:.4f\PYZcb{}}\PY{l+s+s2}{\PYZdq{}}\PY{o}{.}\PY{n}{format}\PY{p}{(}\PY{n}{success}\PY{o}{/}\PY{n}{trials}\PY{p}{)}\PY{p}{)}
        
The probability to get three equal dice is 0.0278
\end{Verbatim}
\end{tcolorbox}
\end{Answer}

\begin{Exercise}[title={(Analytic bootstrapping)}]
Using the function \texttt{normal} of \texttt{numpy.random} simulate the
price of a stock which evolves according to a log-normal stochastic
process with a daily rate of return \(\mu=0.1\) and a volatility
\(\sigma=0.15\) for 30 days.

Also plot the price. Try to play with \(\mu\) and \(\sigma\) to see how
the plot changes.
\end{Exercise}
\begin{Answer}
\begin{tcolorbox}[size=fbox, boxrule=1pt, colback=cellbackground, colframe=cellborder]
\begin{Verbatim}[commandchars=\\\{\}]
\PY{k+kn}{from} \PY{n+nn}{numpy}\PY{n+nn}{.}\PY{n+nn}{random} \PY{k}{import} \PY{n}{normal}\PY{p}{,} \PY{n}{seed}
 \PY{k+kn}{from} \PY{n+nn}{matplotlib} \PY{k}{import} \PY{n}{pyplot} \PY{k}{as} \PY{n}{plt}
 \PY{k+kn}{import} \PY{n+nn}{math}
 
 \PY{n}{S} \PY{o}{=} \PY{l+m+mi}{100}
 \PY{n}{mu} \PY{o}{=} \PY{l+m+mf}{0.1}
 \PY{n}{sigma} \PY{o}{=} \PY{l+m+mf}{0.15}
 \PY{n}{T} \PY{o}{=} \PY{l+m+mi}{1}
 
 \PY{n}{seed}\PY{p}{(}\PY{l+m+mi}{1}\PY{p}{)}
 \PY{n}{historical\PYZus{}series} \PY{o}{=} \PY{p}{[}\PY{n}{S}\PY{p}{]}
 \PY{k}{for} \PY{n}{i} \PY{o+ow}{in} \PY{n+nb}{range}\PY{p}{(}\PY{l+m+mi}{30}\PY{p}{)}\PY{p}{:}
     \PY{n}{S} \PY{o}{=} \PY{n}{S} \PY{o}{*} \PY{n}{math}\PY{o}{.}\PY{n}{exp}\PY{p}{(}\PY{p}{(}\PY{n}{mu} \PY{o}{\PYZhy{}} \PY{l+m+mf}{0.5} \PY{o}{*} \PY{n}{sigma} \PY{o}{*} \PY{n}{sigma}\PY{p}{)} \PY{o}{*} \PY{n}{T} \PY{o}{+} \PY{n}{sigma} \PY{o}{*} \PY{n}{math}\PY{o}{.}\PY{n}{sqrt}\PY{p}{(}\PY{n}{T}\PY{p}{)} \PY{o}{*} \PY{n}{normal}\PY{p}{(}\PY{p}{)}\PY{p}{)}
     \PY{n}{historical\PYZus{}series}\PY{o}{.}\PY{n}{append}\PY{p}{(}\PY{n}{S}\PY{p}{)}
     
 \PY{n}{plt}\PY{o}{.}\PY{n}{plot}\PY{p}{(}\PY{n+nb}{range}\PY{p}{(}\PY{l+m+mi}{31}\PY{p}{)}\PY{p}{,} \PY{n}{historical\PYZus{}series}\PY{p}{)}
 \PY{n}{plt}\PY{o}{.}\PY{n}{xlabel}\PY{p}{(}\PY{l+s+s2}{\PYZdq{}}\PY{l+s+s2}{days}\PY{l+s+s2}{\PYZdq{}}\PY{p}{)}
 \PY{n}{plt}\PY{o}{.}\PY{n}{ylabel}\PY{p}{(}\PY{l+s+s2}{\PYZdq{}}\PY{l+s+s2}{Price of stock X}\PY{l+s+s2}{\PYZdq{}}\PY{p}{)}
 \PY{n}{plt}\PY{o}{.}\PY{n}{show}\PY{p}{(}\PY{p}{)}
\end{Verbatim}
\end{tcolorbox}

\begin{center}
\adjustimage{max size={0.9\linewidth}{0.9\paperheight}}{lesson6_solutions_files/lesson6_solutions_5_0.png}
\end{center}
{ \hspace*{\fill} \\}
\end{Answer}

\begin{Exercise}[title={(Analytic bootstrapping)}]
Suppouse that the Libor Forward rates are those defined here in
(\href{https://repl.it/@MatteoSani/support6}{curve\_data.py}).
Determine the value of an option to pay a fixed rate of 4\% and receives
LIBOR on a 5 year swap starting in 1 year. Assume the notional is 100
EUR, the exercise date is on October, 30th 2020 and the swap rate
volatility is 15\%.
\end{Exercise}
\begin{Answer}
\begin{tcolorbox}[size=fbox, boxrule=1pt, colback=cellbackground, colframe=cellborder]
\begin{Verbatim}[commandchars=\\\{\}]
\PY{k+kn}{from} \PY{n+nn}{finmarkets} \PY{k}{import} \PY{n}{InterestRateSwap}
\PY{k+kn}{from} \PY{n+nn}{datetime} \PY{k}{import} \PY{n}{date}
\PY{k+kn}{from} \PY{n+nn}{dateutil}\PY{n+nn}{.}\PY{n+nn}{relativedelta} \PY{k}{import} \PY{n}{relativedelta}
\PY{k+kn}{from} \PY{n+nn}{curve\PYZus{}data} \PY{k}{import} \PY{n}{discount\PYZus{}curve}\PY{p}{,} \PY{n}{libor\PYZus{}curve}
\PY{k+kn}{from} \PY{n+nn}{scipy}\PY{n+nn}{.}\PY{n+nn}{stats} \PY{k}{import} \PY{n}{norm}
\PY{k+kn}{import} \PY{n+nn}{math}

\PY{n}{pricing\PYZus{}date} \PY{o}{=} \PY{n}{date}\PY{o}{.}\PY{n}{today}\PY{p}{(}\PY{p}{)}
\PY{n}{start\PYZus{}date} \PY{o}{=} \PY{n}{date}\PY{o}{.}\PY{n}{today}\PY{p}{(}\PY{p}{)} \PY{o}{+} \PY{n}{relativedelta}\PY{p}{(}\PY{n}{years}\PY{o}{=}\PY{l+m+mi}{1}\PY{p}{)}
\PY{n}{exercise\PYZus{}date} \PY{o}{=} \PY{n}{date}\PY{p}{(}\PY{l+m+mi}{2020}\PY{p}{,} \PY{l+m+mi}{10}\PY{p}{,} \PY{l+m+mi}{30}\PY{p}{)}

\PY{n}{irs} \PY{o}{=} \PY{n}{InterestRateSwap}\PY{p}{(}\PY{n}{start\PYZus{}date}\PY{p}{,} \PY{l+m+mi}{100}\PY{p}{,} \PY{l+m+mf}{0.04}\PY{p}{,} \PY{l+m+mi}{12}\PY{p}{,} \PY{l+m+mi}{5}\PY{p}{)}
\PY{n}{sigma} \PY{o}{=} \PY{l+m+mf}{0.15}

\PY{n}{A} \PY{o}{=} \PY{n}{irs}\PY{o}{.}\PY{n}{annuity}\PY{p}{(}\PY{n}{discount\PYZus{}curve}\PY{p}{)}
\PY{n}{S} \PY{o}{=} \PY{n}{irs}\PY{o}{.}\PY{n}{swap\PYZus{}rate}\PY{p}{(}\PY{n}{discount\PYZus{}curve}\PY{p}{,} \PY{n}{libor\PYZus{}curve}\PY{p}{)}
\PY{n}{T} \PY{o}{=} \PY{p}{(}\PY{n}{exercise\PYZus{}date} \PY{o}{\PYZhy{}} \PY{n}{pricing\PYZus{}date}\PY{p}{)}\PY{o}{.}\PY{n}{days} \PY{o}{/} \PY{l+m+mi}{365}
\PY{n}{d1} \PY{o}{=} \PY{p}{(}\PY{n}{math}\PY{o}{.}\PY{n}{log}\PY{p}{(}\PY{n}{S}\PY{o}{/}\PY{n}{irs}\PY{o}{.}\PY{n}{fixed\PYZus{}rate}\PY{p}{)} \PY{o}{+} \PY{l+m+mf}{0.5} \PY{o}{*} \PY{n}{sigma}\PY{o}{*}\PY{o}{*}\PY{l+m+mi}{2} \PY{o}{*} \PY{n}{T}\PY{p}{)} \PY{o}{/} \PY{p}{(}\PY{n}{sigma} \PY{o}{*} \PY{n}{T}\PY{o}{*}\PY{o}{*}\PY{l+m+mf}{0.5}\PY{p}{)}
\PY{n}{d2} \PY{o}{=} \PY{p}{(}\PY{n}{math}\PY{o}{.}\PY{n}{log}\PY{p}{(}\PY{n}{S}\PY{o}{/}\PY{n}{irs}\PY{o}{.}\PY{n}{fixed\PYZus{}rate}\PY{p}{)} \PY{o}{\PYZhy{}} \PY{l+m+mf}{0.5} \PY{o}{*} \PY{n}{sigma}\PY{o}{*}\PY{o}{*}\PY{l+m+mi}{2} \PY{o}{*} \PY{n}{T}\PY{p}{)} \PY{o}{/} \PY{p}{(}\PY{n}{sigma} \PY{o}{*} \PY{n}{T}\PY{o}{*}\PY{o}{*}\PY{l+m+mf}{0.5}\PY{p}{)}
\PY{n}{npv} \PY{o}{=} \PY{n}{irs}\PY{o}{.}\PY{n}{notional} \PY{o}{*} \PY{n}{A} \PY{o}{*} \PY{p}{(}\PY{n}{S} \PY{o}{*} \PY{n}{norm}\PY{o}{.}\PY{n}{cdf}\PY{p}{(}\PY{n}{d1}\PY{p}{)} \PY{o}{\PYZhy{}} \PY{n}{irs}\PY{o}{.}\PY{n}{fixed\PYZus{}rate} \PY{o}{*} \PY{n}{norm}\PY{o}{.}\PY{n}{cdf}\PY{p}{(}\PY{n}{d2}\PY{p}{)}\PY{p}{)}

\PY{n+nb}{print}\PY{p}{(}\PY{l+s+s2}{\PYZdq{}}\PY{l+s+s2}{Swaption NPV: }\PY{l+s+si}{\PYZob{}:.3f\PYZcb{}}\PY{l+s+s2}{ EUR}\PY{l+s+s2}{\PYZdq{}}\PY{o}{.}\PY{n}{format}\PY{p}{(}\PY{n}{npv}\PY{p}{)}\PY{p}{)}

Swaption NPV: 13.587 EUR

\end{Verbatim}
\end{tcolorbox}
\end{Answer}

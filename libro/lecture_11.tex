\chapter{Interest Rate Swaps and Swaptions}\label{interest-rate-swaps-and-swaptions}

In the previous Chapters we introduced a particular type of swap, the Overnight Index Swap, here we describes a more general Interest Rate Swap and see how it can underlying a swaption, the analogous of the European options for interest rate market.

\section{Interest Rate Swaps}\label{interest-rate-swaps}

Interest rate swaps (IRS) consist of two legs a floating and a fixed. The
contract parameters are:

\begin{itemize}
\tightlist
\item
  start date \(d_0\);
\item
  notional \(N\);
\item
  fixed rate \(K\);
\item
  floating rate tenor (months);
\item
  maturity (years).
\end{itemize}

The floating leg pays the reference LIBOR fixing at a frequency equal to
the tenor of the floating rate, so for example an IRS on a 3-month
LIBOR will pay a floating coupon every three months, an IRS on 6-month
EURIBOR pays the floating coupon every six months and so on.

The fixed leg pays a predetermined cash flow at annual frequency,
regardless of the tenor of the underlying floating rate (for simplicity we will only consider swaps with maturities which are multiples of 1 year).

Before going into the details of the valuation of IRSs, we need to
modify the \texttt{generate\_swap\_dates} function in
\texttt{finmarkets} module to generate the payment dates for both the
fixed and floating legs. 
The modification consists of the addition of a new input
parameter, the tenor, which was previously set to 12 months.

\begin{tcolorbox}[breakable, size=fbox, boxrule=1pt, pad at break*=1mm,colback=cellbackground, colframe=cellborder]
\begin{Verbatim}[commandchars=\\\{\}]
\PY{k+kn}{from} \PY{n+nn}{datetime} \PY{k}{import} \PY{n}{date}
\PY{k+kn}{from} \PY{n+nn}{dateutil}\PY{n+nn}{.}\PY{n+nn}{relativedelta} \PY{k}{import} \PY{n}{relativedelta}
    
\PY{k}{def} \PY{n+nf}{generate\PYZus{}swap\PYZus{}dates}\PY{p}{(}\PY{n}{start\PYZus{}date}\PY{p}{,} \PY{n}{n\PYZus{}months}\PY{p}{,} \PY{n}{tenor\PYZus{}months}\PY{o}{=}\PY{l+m+mi}{12}\PY{p}{)}\PY{p}{:}
    \PY{n}{dates} \PY{o}{=} \PY{p}{[}\PY{p}{]}
    \PY{k}{for} \PY{n}{n} \PY{o+ow}{in} \PY{n+nb}{range}\PY{p}{(}\PY{l+m+mi}{0}\PY{p}{,} \PY{n}{n\PYZus{}months}\PY{p}{,} \PY{n}{tenor\PYZus{}months}\PY{p}{)}\PY{p}{:}
        \PY{n}{dates}\PY{o}{.}\PY{n}{append}\PY{p}{(}\PY{n}{start\PYZus{}date} \PY{o}{+} \PY{n}{relativedelta}\PY{p}{(}\PY{n}{months}\PY{o}{=}\PY{n}{n}\PY{p}{)}\PY{p}{)}
    \PY{n}{dates}\PY{o}{.}\PY{n}{append}\PY{p}{(}\PY{n}{start\PYZus{}date} \PY{o}{+} \PY{n}{relativedelta}\PY{p}{(}\PY{n}{months}\PY{o}{=}\PY{n}{n\PYZus{}months}\PY{p}{)}\PY{p}{)}
    \PY{k}{return} \PY{n}{dates}

\PY{n}{generate\PYZus{}swap\PYZus{}dates}\PY{p}{(}\PY{n}{date}\PY{o}{.}\PY{n}{today}\PY{p}{(}\PY{p}{)}\PY{p}{,} \PY{l+m+mi}{16}\PY{p}{,} \PY{l+m+mi}{3}\PY{p}{)}

[datetime.date(2020, 10, 15),
 datetime.date(2021, 1, 15),
 datetime.date(2021, 4, 15),
 datetime.date(2021, 7, 15),
 datetime.date(2021, 10, 15),
 datetime.date(2022, 1, 15),
 datetime.date(2022, 2, 15)]
\end{Verbatim}
\end{tcolorbox}
        
Using this function and the contract parameters we will be able to
determine a sequence of payment dates for each leg of the IRS.

\subsection{IRS Valuation}\label{irs-valuation}

Let \(d_0=d_0^{\mathrm{fixed}},...,d_p^{\mathrm{fixed}}\) be the fixed
leg payment dates and
\(d_0=d_0^{\mathrm{float}},...,d_p^{\mathrm{float}}\) be the floating
leg payment dates, and let's use the following notation:

\begin{itemize}
\tightlist
\item
  \(d\) the pricing date;
\item
  \(D(d, d')\) the discount factor observed in date \(d\) for the value
  date \(d'\);
\item
  \(F(d, d', d'')\) the forward rate observed in date \(d\) for the
  period \([d', d'']\); 
  \item the rate tenor is \(\tau = d'' - d'\).
\end{itemize}
The NPV of the fixed leg is calculated as follows:

\begin{equation}
\mathrm{NPV}_{\mathrm{fixed}}(d, K) = N\cdot K\cdot\sum_{i=1}^{n}D(d, d_{i}^{\mathrm{fixed}})\end{equation}
while the NPV of the floating leg is calculated as follows:

\begin{equation}\mathrm{NPV}_{\mathrm{float}}(d) = N\cdot\sum_{i=1}^{m}F(d, d_{j-1}^{\mathrm{float}}, d_{j}^{\mathrm{float}}) \cdot \frac{d_{j}^{\mathrm{float}}-d_{j-1}^{\mathrm{float}}}{360}
\cdot D(d, d_{i}^{\mathrm{float}})\end{equation}

Therefore the NPV of the swap (seen from the point of view of the
counter-party which receives the floating leg) is

\begin{equation}\mathrm{NPV}(d, K) = \mathrm{NPV}_{\mathrm{float}}(d) - \mathrm{NPV}_{\mathrm{fixed}}(d, K)\end{equation}

For reasons which will become apparent later, it's actually more
convenient to express the NPV of an IRS as a function of the fair value
fixed rate \(S\) of the IRS, also known as the \textbf{swap rate}, which
is the value of K which makes \(\mathrm{NPV}(d)=0\).
On the basis of the previous expressions, we can easily calculate \(S\)
as

\begin{equation}
\begin{gathered}
\mathrm{NPV}_{\mathrm{fixed}}(d, S) = \mathrm{NPV}_{\mathrm{float}}(d)\\
N\cdot S\cdot\sum_{i=1}^{n}D(d, d_{i}^{\mathrm{fixed}}) = N\cdot\sum_{i=1}^{m}F(d, d_{j-1}^{\mathrm{float}}, d_{j}^{\mathrm{float}}) \cdot \frac{d_{j}^{\mathrm{float}}-d_{j-1}^{\mathrm{float}}}{360} \cdot D(d, d_{i}^{\mathrm{float}})\\
S=\frac{\sum_{i=1}^{m}F(d, d_{j-1}^{\mathrm{float}}, d_{j}^{\mathrm{float}}) \cdot \frac{d_{j}^{\mathrm{float}}-d_{j-1}^{\mathrm{float}}}{360}
\cdot D(d, d_{i}^{\mathrm{float}})}{\sum_{i=1}^{n}D(d, d_i^{\mathrm{fixed}})}
\end{gathered}
\end{equation}

Once we have calculated \(S\), we can express the \(\mathrm{NPV}\) of an
IRS as follows:

\begin{equation}
\begin{split}&\mathrm{NPV}(d, K) = \mathrm{NPV}_{\mathrm{float}}(d) - \mathrm{NPV}_{\mathrm{fixed}}(d, K) = \\ 
&= \underbrace{\mathrm{NPV}_{\mathrm{float}}(d) - \mathrm{NPV}_{\mathrm{fixed}}(d, S)}_{\mathrm{=\;0}} + \mathrm{NPV}_{\mathrm{fixed}}(d, S) - \mathrm{NPV}_{\mathrm{fixed}}(d, K) \\ 
& = N\cdot(S-K)\cdot\underbrace{\sum_{i=1}^{n}D(d, d_{i}^{\mathrm{fixed}})}_{\mathrm{'annuity'}}
\end{split}
\end{equation}

For convenience the relevant inputs that will be used later (LIBOR and discount curve definitions) have been saved in the files \href{https://drive.google.com/file/d/1dm5oZnZKmJM6UrV0L32OcqD5Tzs9SI9A/view?usp=sharing}{libor\_curve.xlsx} and \href{https://drive.google.com/file/d/14R22r7m-6VpQ_P79D3qHdK0QN_mOQ_UB/view?usp=sharing}{discount\_curve.xlsx} respectively.

\begin{tcolorbox}[breakable, size=fbox, boxrule=1pt, pad at break*=1mm,colback=cellbackground, colframe=cellborder]
\begin{Verbatim}[commandchars=\\\{\}]
\PY{k+kn}{import} \PY{n+nn}{pandas} \PY{k}{as} \PY{n+nn}{pd}
\PY{k+kn}{from} \PY{n+nn}{datetime} \PY{k}{import} \PY{n}{date}
\PY{k+kn}{from} \PY{n+nn}{finmarkets} \PY{k}{import} \PY{n}{DiscountCurve}\PY{p}{,} \PY{n}{ForwardRateCurve}

\PY{n}{observation\PYZus{}date} \PY{o}{=} \PY{n}{date.today()}
\PY{n}{discount\PYZus{}data} \PY{o}{=} \PY{n}{pd}\PY{o}{.}\PY{n}{read\PYZus{}excel}\PY{p}{(}\PY{l+s+s1}{\PYZsq{}}\PY{l+s+s1}{discount\PYZus{}curve.xlsx}\PY{l+s+s1}{\PYZsq{}}\PY{p}{)}
\PY{n}{libor\PYZus{}data} \PY{o}{=} \PY{n}{pd}\PY{o}{.}\PY{n}{read\PYZus{}excel}\PY{p}{(}\PY{l+s+s1}{\PYZsq{}}\PY{l+s+s1}{libor.xlsx}\PY{l+s+s1}{\PYZsq{}}\PY{p}{)}

\PY{n}{dc} \PY{o}{=} \PY{n}{DiscountCurve}\PY{p}{(}\PY{n}{pricing\PYZus{}date}\PY{p}{,} 
                   \PY{n}{discount\PYZus{}data}\PY{p}{[}\PY{l+s+s1}{\PYZsq{}}\PY{l+s+s1}{pillar}\PY{l+s+s1}{\PYZsq{}}\PY{p}{]}\PY{o}{.}\PY{n}{dt}\PY{o}{.}\PY{n}{date}\PY{o}{.}\PY{n}{tolist}\PY{p}{(}\PY{p}{)}\PY{p}{,}
                   \PY{n}{discount\PYZus{}data}\PY{p}{[}\PY{l+s+s1}{\PYZsq{}}\PY{l+s+s1}{discount\PYZus{}factor}\PY{l+s+s1}{\PYZsq{}}\PY{p}{]}\PY{o}{.}\PY{n}{tolist}\PY{p}{(}\PY{p}{)}\PY{p}{)}

\PY{n}{fr} \PY{o}{=} \PY{n}{ForwardRateCurve}\PY{p}{(}\PY{n}{libor\PYZus{}data}\PY{p}{[}\PY{l+s+s1}{\PYZsq{}}\PY{l+s+s1}{date}\PY{l+s+s1}{\PYZsq{}}\PY{p}{]}\PY{o}{.}\PY{n}{dt}\PY{o}{.}\PY{n}{date}\PY{o}{.}\PY{n}{tolist}\PY{p}{(}\PY{p}{)}\PY{p}{,}
                      \PY{n}{libor\PYZus{}data}\PY{p}{[}\PY{l+s+s1}{\PYZsq{}}\PY{l+s+s1}{rate}\PY{l+s+s1}{\PYZsq{}}\PY{p}{]}\PY{o}{.}\PY{n}{tolist}\PY{p}{(}\PY{p}{)}\PY{p}{)}

\PY{n+nb}{print}\PY{p}{(}\PY{n}{dc}\PY{o}{.}\PY{n}{df}\PY{p}{(}\PY{n}{date}\PY{p}{(}\PY{l+m+mi}{2021}\PY{p}{,} \PY{l+m+mi}{1}\PY{p}{,} \PY{l+m+mi}{1}\PY{p}{)}\PY{p}{)}\PY{p}{)}
\PY{n+nb}{print} \PY{p}{(}\PY{n}{fr}\PY{o}{.}\PY{n}{forward\PYZus{}rate}\PY{p}{(}\PY{n}{date}\PY{p}{(}\PY{l+m+mi}{2021}\PY{p}{,} \PY{l+m+mi}{1}\PY{p}{,} \PY{l+m+mi}{1}\PY{p}{)}\PY{p}{)}\PY{p}{)}

1.0041959227522805
0.060712328767123284
\end{Verbatim}
\end{tcolorbox}

Now we can implement the \texttt{InterestRateSwap} class to valuate IRS
contracts.

\begin{tcolorbox}[breakable, size=fbox, boxrule=1pt, pad at break*=1mm,colback=cellbackground, colframe=cellborder]
\begin{Verbatim}[commandchars=\\\{\}]
\PY{k}{class} \PY{n+nc}{InterestRateSwap}\PY{p}{:}    
    \PY{k}{def} \PY{n+nf}{\PYZus{}\PYZus{}init\PYZus{}\PYZus{}}\PY{p}{(}\PY{n+nb+bp}{self}\PY{p}{,} \PY{n}{start\PYZus{}date}\PY{p}{,} \PY{n}{notional}\PY{p}{,} 
                 \PY{n}{fixed\PYZus{}rate}\PY{p}{,} \PY{n}{tenor\PYZus{}months}\PY{p}{,} 
                 \PY{n}{maturity\PYZus{}years}\PY{p}{)}\PY{p}{:}
        \PY{n+nb+bp}{self}\PY{o}{.}\PY{n}{notional} \PY{o}{=} \PY{n}{notional}
        \PY{n+nb+bp}{self}\PY{o}{.}\PY{n}{fixed\PYZus{}rate} \PY{o}{=} \PY{n}{fixed\PYZus{}rate}
        \PY{n+nb+bp}{self}\PY{o}{.}\PY{n}{fixed\PYZus{}leg\PYZus{}dates} \PY{o}{=} \PYZbs{}
            \PY{n}{generate\PYZus{}swap\PYZus{}dates}\PY{p}{(}\PY{n}{start\PYZus{}date}\PY{p}{,} \PY{l+m+mi}{12} \PY{o}{*} \PY{n}{maturity\PYZus{}years}\PY{p}{)}
        \PY{n+nb+bp}{self}\PY{o}{.}\PY{n}{floating\PYZus{}leg\PYZus{}dates} \PY{o}{=} \PYZbs{}
            \PY{n}{generate\PYZus{}swap\PYZus{}dates}\PY{p}{(}\PY{n}{start\PYZus{}date}\PY{p}{,} \PY{l+m+mi}{12} \PY{o}{*} \PY{n}{maturity\PYZus{}years}\PY{p}{,}
                                \PY{n}{tenor\PYZus{}months}\PY{p}{)}
        
    \PY{k}{def} \PY{n+nf}{annuity}\PY{p}{(}\PY{n+nb+bp}{self}\PY{p}{,} \PY{n}{discount\PYZus{}curve}\PY{p}{)}\PY{p}{:}
        \PY{n}{a} \PY{o}{=} \PY{l+m+mi}{0}
        \PY{k}{for} \PY{n}{i} \PY{o+ow}{in} \PY{n+nb}{range}\PY{p}{(}\PY{l+m+mi}{1}\PY{p}{,} \PY{n+nb}{len}\PY{p}{(}\PY{n+nb+bp}{self}\PY{o}{.}\PY{n}{fixed\PYZus{}leg\PYZus{}dates}\PY{p}{)}\PY{p}{)}\PY{p}{:}
            \PY{n}{a} \PY{o}{+}\PY{o}{=} \PY{n}{discount\PYZus{}curve}\PY{o}{.}\PY{n}{df}\PY{p}{(}\PY{n+nb+bp}{self}\PY{o}{.}\PY{n}{fixed\PYZus{}leg\PYZus{}dates}\PY{p}{[}\PY{n}{i}\PY{p}{]}\PY{p}{)}
        \PY{k}{return} \PY{n}{a}

    \PY{k}{def} \PY{n+nf}{swap\PYZus{}rate}\PY{p}{(}\PY{n+nb+bp}{self}\PY{p}{,} \PY{n}{discount\PYZus{}curve}\PY{p}{,} \PY{n}{libor\PYZus{}curve}\PY{p}{)}\PY{p}{:}
        \PY{n}{s} \PY{o}{=} \PY{l+m+mi}{0}
        \PY{k}{for} \PY{n}{j} \PY{o+ow}{in} \PY{n+nb}{range}\PY{p}{(}\PY{l+m+mi}{1}\PY{p}{,} \PY{n+nb}{len}\PY{p}{(}\PY{n+nb+bp}{self}\PY{o}{.}\PY{n}{floating\PYZus{}leg\PYZus{}dates}\PY{p}{)}\PY{p}{)}\PY{p}{:}
            \PY{n}{F} \PY{o}{=} \PY{n}{libor\PYZus{}curve}\PY{o}{.}\PY{n}{forward\PYZus{}rate}\PY{p}{(}\PY{n+nb+bp}{self}\PY{o}{.}\PY{n}{floating\PYZus{}leg\PYZus{}dates}\PY{p}{[}\PY{n}{j}\PY{o}{\PYZhy{}}\PY{l+m+mi}{1}\PY{p}{]}\PY{p}{)}
            \PY{n}{tau} \PY{o}{=} \PY{p}{(}\PY{n+nb+bp}{self}\PY{o}{.}\PY{n}{floating\PYZus{}leg\PYZus{}dates}\PY{p}{[}\PY{n}{j}\PY{p}{]} \PY{o}{\PYZhy{}} \PYZbs{}
                   \PY{n+nb+bp}{self}\PY{o}{.}\PY{n}{floating\PYZus{}leg\PYZus{}dates}\PY{p}{[}\PY{n}{j}\PY{o}{\PYZhy{}}\PY{l+m+mi}{1}\PY{p}{]}\PY{p}{)}\PY{o}{.}\PY{n}{days} \PY{o}{/} \PY{l+m+mi}{360}
            \PY{n}{P} \PY{o}{=} \PY{n}{discount\PYZus{}curve}\PY{o}{.}\PY{n}{df}\PY{p}{(}\PY{n+nb+bp}{self}\PY{o}{.}\PY{n}{floating\PYZus{}leg\PYZus{}dates}\PY{p}{[}\PY{n}{j}\PY{p}{]}\PY{p}{)}
            \PY{n}{s} \PY{o}{+}\PY{o}{=} \PY{n}{F} \PY{o}{*} \PY{n}{tau} \PY{o}{*} \PY{n}{P}
        \PY{k}{return} \PY{n}{s} \PY{o}{/} \PY{n+nb+bp}{self}\PY{o}{.}\PY{n}{annuity}\PY{p}{(}\PY{n}{discount\PYZus{}curve}\PY{p}{)}
        
    \PY{k}{def} \PY{n+nf}{npv}\PY{p}{(}\PY{n+nb+bp}{self}\PY{p}{,} \PY{n}{discount\PYZus{}curve}\PY{p}{,} \PY{n}{libor\PYZus{}curve}\PY{p}{)}\PY{p}{:}
        \PY{n}{S} \PY{o}{=} \PY{n+nb+bp}{self}\PY{o}{.}\PY{n}{swap\PYZus{}rate}\PY{p}{(}\PY{n}{discount\PYZus{}curve}\PY{p}{,} \PY{n}{libor\PYZus{}curve}\PY{p}{)}
        \PY{n}{A} \PY{o}{=} \PY{n+nb+bp}{self}\PY{o}{.}\PY{n}{annuity}\PY{p}{(}\PY{n}{discount\PYZus{}curve}\PY{p}{)}
        \PY{k}{return} \PY{n+nb+bp}{self}\PY{o}{.}\PY{n}{notional} \PY{o}{*} \PY{p}{(}\PY{n}{S} \PY{o}{\PYZhy{}} \PY{n+nb+bp}{self}\PY{o}{.}\PY{n}{fixed\PYZus{}rate}\PY{p}{)} \PY{o}{*} \PY{n}{A}
\end{Verbatim}
\end{tcolorbox}

Let's test our class instantiating an IRS with 1M notional, fixed rate
of 5\%, 6 month tenor and a maturity of 4 years; discount and LIBOR
curves are the same as before.

\begin{tcolorbox}[breakable, size=fbox, boxrule=1pt, pad at break*=1mm,colback=cellbackground, colframe=cellborder]
\begin{Verbatim}[commandchars=\\\{\}]
\PY{n}{start\PYZus{}date} \PY{o}{=} \PY{n}{date.today() + relativedelta(months=1)}
\PY{n}{irs} \PY{o}{=} \PY{n}{InterestRateSwap}\PY{p}{(}\PY{n}{start\PYZus{}date}\PY{p}{,} \PY{l+m+mf}{1e6}\PY{p}{,} \PY{l+m+mf}{0.05}\PY{p}{,} \PY{l+m+mi}{6}\PY{p}{,} \PY{l+m+mi}{4}\PY{p}{)}
\PY{n+nb}{print} \PY{p}{(}\PY{l+s+s2}{\PYZdq{}}\PY{l+s+si}{\PYZob{}:.2f\PYZcb{}}\PY{l+s+s2}{ EUR}\PY{l+s+s2}{\PYZdq{}}\PY{o}{.}\PY{n}{format}\PY{p}{(}\PY{n}{irs}\PY{o}{.}\PY{n}{npv}\PY{p}{(}\PY{n}{dc}\PY{p}{,} \PY{n}{fr}\PY{p}{)}\PY{p}{)}\PY{p}{)}

22453.12 EUR
\end{Verbatim}
\end{tcolorbox}

\textbf{Can you guess what could be the \textbf{swap rate} given that the NPV obtained above ?}
Since we are looking at this contracts from the point of view
of the receiver of the floating leg and that the swap rate is the rate that makes the total NPV equal to 0, the NPV of the fixed leg has to be increased to balance the value of the floating leg so the swap rate will be higher than the fixed rate of the IRS.

\begin{tcolorbox}[breakable, size=fbox, boxrule=1pt, pad at break*=1mm,colback=cellbackground, colframe=cellborder]
\begin{Verbatim}[commandchars=\\\{\}]
\PY{n+nb}{print} \PY{p}{(}\PY{l+s+s2}{\PYZdq{}}\PY{l+s+si}{\PYZob{}\PYZcb{}}\PY{l+s+s2}{\PYZdq{}}\PY{o}{.}\PY{n}{format}\PY{p}{(}\PY{n}{irs}\PY{o}{.}\PY{n}{swap\PYZus{}rate}\PY{p}{(}\PY{n}{dc}\PY{p}{,} \PY{n}{fr}\PY{p}{)}\PY{p}{)}\PY{p}{)}

0.054458656972619764
\end{Verbatim}
\end{tcolorbox}
    
To check if we have computed correctly the swap rate we can
instantiate a new IRS with fixed rate equal to the just calculated swap
rate and print its NPV, it should come very close to 0.

\begin{tcolorbox}[breakable, size=fbox, boxrule=1pt, pad at break*=1mm,colback=cellbackground, colframe=cellborder]
\begin{Verbatim}[commandchars=\\\{\}]
\PY{n}{irs2} \PY{o}{=} \PY{n}{InterestRateSwap}\PY{p}{(}\PY{n}{start\PYZus{}date}\PY{p}{,} \PY{l+m+mf}{1e6}\PY{p}{,} \PY{l+m+mf}{0.054458656972619764}\PY{p}{,} \PY{l+m+mi}{6}\PY{p}{,} \PY{l+m+mi}{4}\PY{p}{)}
\PY{n+nb}{print} \PY{p}{(}\PY{l+s+s2}{\PYZdq{}}\PY{l+s+si}{\PYZob{}:.2f\PYZcb{}}\PY{l+s+s2}{ EUR}\PY{l+s+s2}{\PYZdq{}}\PY{o}{.}\PY{n}{format}\PY{p}{(}\PY{n}{irs2}\PY{o}{.}\PY{n}{npv}\PY{p}{(}\PY{n}{dc}\PY{p}{,} \PY{n}{fr}\PY{p}{)}\PY{p}{)}\PY{p}{)}

0.0 EUR
\end{Verbatim}
\end{tcolorbox}
    
\section{Inheritance Again}
Now that we have introduced two kind of swaps we can try to make an alternative implementation of their classes, this time using inheritance.

The base (or parent) class will be \texttt{GenericSwap} and it will implement just the constructor taking in input the basic data characterizing a swap: notional, maturity, tenor and rate of the fixed leg. We will slightly modify the implementation of the \texttt{OvernightIndexSwap} class since now the payment dates are computed directly in the constructor of \texttt{GenericSwap}.

\begin{tcolorbox}[breakable, size=fbox, boxrule=1pt, pad at break*=1mm,colback=cellbackground, colframe=cellborder]
\begin{Verbatim}[commandchars=\\\{\}]
\PY{k}{class} \PY{n+nc}{GenericSwap}\PY{p}{:}
    \PY{k}{def} \PY{n+nf}{\PYZus{}\PYZus{}init\PYZus{}\PYZus{}}\PY{p}{(}\PY{n+nb+bp}{self}\PY{p}{,} \PY{n}{start\PYZus{}date}\PY{p}{,} \PY{n}{notional}\PY{p}{,} 
                 \PY{n}{fixed\PYZus{}rate}\PY{p}{,} \PY{n}{maturity\PYZus{}years}\PY{p}{,} \PY{n}{tenor\PYZus{}months}\PY{o}{=}\PY{l+m+mi}{12}\PY{p}{)}\PY{p}{:}
        \PY{n+nb+bp}{self}\PY{o}{.}\PY{n}{notional} \PY{o}{=} \PY{n}{notional}
        \PY{n+nb+bp}{self}\PY{o}{.}\PY{n}{fixed\PYZus{}rate} \PY{o}{=} \PY{n}{fixed\PYZus{}rate}
        \PY{n+nb+bp}{self}\PY{o}{.}\PY{n}{fixed\PYZus{}leg\PYZus{}dates} \PY{o}{=} \PYZbs{}
            \PY{n}{generate\PYZus{}swap\PYZus{}dates}\PY{p}{(}\PY{n}{start\PYZus{}date}\PY{p}{,} \PY{l+m+mi}{12} \PY{o}{*} \PY{n}{maturity\PYZus{}years}\PY{p}{)}
        \PY{n+nb+bp}{self}\PY{o}{.}\PY{n}{floating\PYZus{}leg\PYZus{}dates} \PY{o}{=} \PYZbs{}
            \PY{n}{generate\PYZus{}swap\PYZus{}dates}\PY{p}{(}\PY{n}{start\PYZus{}date}\PY{p}{,} \PY{l+m+mi}{12} \PY{o}{*} \PY{n}{maturity\PYZus{}years}\PY{p}{,} \PY{n}{tenor\PYZus{}months}\PY{p}{)}
        \PY{n+nb+bp}{self}\PY{o}{.}\PY{n}{maturity} \PY{o}{=} \PY{n}{maturity\PYZus{}years}
	
    \PY{k}{def} \PY{n+nf}{npv}\PY{p}{(}\PY{n+nb+bp}{self}\PY{p}{)}\PY{p}{:}
        \PY{n+nb}{print} \PY{p}{(}\PY{l+s+s2}{\PYZdq{}}\PY{l+s+si}{\PYZob{}\PYZcb{}}\PY{l+s+s2}{ doesn}\PY{l+s+s2}{\PYZsq{}}\PY{l+s+s2}{t implement npv method.}\PY{l+s+s2}{\PYZdq{}}\PY{o}{.}\PY{n}{format}\PY{p}{(}\PY{n+nb+bp}{self}\PY{o}{.}\PY{n+nv+vm}{\PYZus{}\PYZus{}name\PYZus{}\PYZus{}}\PY{p}{)}\PY{p}{)}
	
\PY{k}{class} \PY{n+nc}{OvernightIndexSwap}\PY{p}{(}\PY{n}{GenericSwap}\PY{p}{)}\PY{p}{:}
    \PY{k}{def} \PY{n+nf}{npv\PYZus{}floating\PYZus{}leg}\PY{p}{(}\PY{n+nb+bp}{self}\PY{p}{,} \PY{n}{discount\PYZus{}curve}\PY{p}{)}\PY{p}{:}
        \PY{k}{return} \PY{n+nb+bp}{self}\PY{o}{.}\PY{n}{notional} \PY{o}{*} \PY{p}{(}\PY{n}{discount\PYZus{}curve}\PY{o}{.}\PY{n}{df}\PY{p}{(}\PY{n+nb+bp}{self}\PY{o}{.}\PY{n}{floating\PYZus{}leg\PYZus{}dates}\PY{p}{[}\PY{l+m+mi}{0}\PY{p}{]}\PY{p}{)} \PY{o}{\PYZhy{}}
                                \PY{n}{discount\PYZus{}curve}\PY{o}{.}\PY{n}{df}\PY{p}{(}\PY{n+nb+bp}{self}\PY{o}{.}\PY{n}{floating\PYZus{}leg\PYZus{}dates}\PY{p}{[}\PY{o}{\PYZhy{}}\PY{l+m+mi}{1}\PY{p}{]}\PY{p}{)}\PY{p}{)}
	
    \PY{k}{def} \PY{n+nf}{npv\PYZus{}fixed\PYZus{}leg}\PY{p}{(}\PY{n+nb+bp}{self}\PY{p}{,} \PY{n}{discount\PYZus{}curve}\PY{p}{)}\PY{p}{:}
        \PY{n}{npv} \PY{o}{=} \PY{l+m+mi}{0}
        \PY{k}{for} \PY{n}{i} \PY{o+ow}{in} \PY{n+nb}{range}\PY{p}{(}\PY{l+m+mi}{1}\PY{p}{,} \PY{n+nb}{len}\PY{p}{(}\PY{n+nb+bp}{self}\PY{o}{.}\PY{n}{fixed\PYZus{}leg\PYZus{}dates}\PY{p}{)}\PY{p}{)}\PY{p}{:}
            \PY{n}{start\PYZus{}date} \PY{o}{=} \PY{n+nb+bp}{self}\PY{o}{.}\PY{n}{fixed\PYZus{}leg\PYZus{}dates}\PY{p}{[}\PY{n}{i}\PY{o}{\PYZhy{}}\PY{l+m+mi}{1}\PY{p}{]}
            \PY{n}{end\PYZus{}date} \PY{o}{=} \PY{n+nb+bp}{self}\PY{o}{.}\PY{n}{fixed\PYZus{}leg\PYZus{}dates}\PY{p}{[}\PY{n}{i}\PY{p}{]}
            \PY{n}{tau} \PY{o}{=} \PY{p}{(}\PY{n}{end\PYZus{}date} \PY{o}{\PYZhy{}} \PY{n}{start\PYZus{}date}\PY{p}{)}\PY{o}{.}\PY{n}{days} \PY{o}{/} \PY{l+m+mi}{360}
            \PY{n}{df} \PY{o}{=} \PY{n}{discount\PYZus{}curve}\PY{o}{.}\PY{n}{df}\PY{p}{(}\PY{n}{end\PYZus{}date}\PY{p}{)}
            \PY{n}{npv} \PY{o}{+=} \PY{n}{df} \PY{o}{*} \PY{n}{tau}
        \PY{k}{return} \PY{n+nb+bp}{self}\PY{o}{.}\PY{n}{notional} \PY{o}{*} \PY{n+nb+bp}{self}\PY{o}{.}\PY{n}{fixed\PYZus{}rate} \PY{o}{*} \PY{n}{npv}
	
    \PY{k}{def} \PY{n+nf}{npv}\PY{p}{(}\PY{n+nb+bp}{self}\PY{p}{,} \PY{n}{discount\PYZus{}curve}\PY{p}{)}\PY{p}{:}
        \PY{n}{float\PYZus{}npv} \PY{o}{=} \PY{n+nb+bp}{self}\PY{o}{.}\PY{n}{npv\PYZus{}floating\PYZus{}leg}\PY{p}{(}\PY{n}{discount\PYZus{}curve}\PY{p}{)}
        \PY{n}{fixed\PYZus{}npv} \PY{o}{=} \PY{n+nb+bp}{self}\PY{o}{.}\PY{n}{npv\PYZus{}fixed\PYZus{}leg}\PY{p}{(}\PY{n}{discount\PYZus{}curve}\PY{p}{)}
        \PY{k}{return} \PY{n}{float\PYZus{}npv} \PY{o}{\PYZhy{}} \PY{n}{fixed\PYZus{}npv}
	
\PY{k}{class} \PY{n+nc}{InterestRateSwap}\PY{p}{(}\PY{n}{GenericSwap}\PY{p}{)}\PY{p}{:}        
    \PY{k}{def} \PY{n+nf}{annuity}\PY{p}{(}\PY{n+nb+bp}{self}\PY{p}{,} \PY{n}{discount\PYZus{}curve}\PY{p}{)}\PY{p}{:}
        \PY{n}{a} \PY{o}{=} \PY{l+m+mi}{0}
        \PY{k}{for} \PY{n}{i} \PY{o+ow}{in} \PY{n+nb}{range}\PY{p}{(}\PY{l+m+mi}{1}\PY{p}{,} \PY{n+nb}{len}\PY{p}{(}\PY{n+nb+bp}{self}\PY{o}{.}\PY{n}{fixed\PYZus{}leg\PYZus{}dates}\PY{p}{)}\PY{p}{)}\PY{p}{:}
            \PY{n}{a} \PY{o}{+}\PY{o}{=} \PY{n}{discount\PYZus{}curve}\PY{o}{.}\PY{n}{df}\PY{p}{(}\PY{n+nb+bp}{self}\PY{o}{.}\PY{n}{fixed\PYZus{}leg\PYZus{}dates}\PY{p}{[}\PY{n}{i}\PY{p}{]}\PY{p}{)}
        \PY{k}{return} \PY{n}{a}
	
    \PY{k}{def} \PY{n+nf}{swap\PYZus{}rate}\PY{p}{(}\PY{n+nb+bp}{self}\PY{p}{,} \PY{n}{discount\PYZus{}curve}\PY{p}{,} \PY{n}{libor\PYZus{}curve}\PY{p}{)}\PY{p}{:}
        \PY{n}{s} \PY{o}{=} \PY{l+m+mi}{0}
        \PY{k}{for} \PY{n}{j} \PY{o+ow}{in} \PY{n+nb}{range}\PY{p}{(}\PY{l+m+mi}{1}\PY{p}{,} \PY{n+nb}{len}\PY{p}{(}\PY{n+nb+bp}{self}\PY{o}{.}\PY{n}{floating\PYZus{}leg\PYZus{}dates}\PY{p}{)}\PY{p}{)}\PY{p}{:}
            \PY{n}{F} \PY{o}{=} \PY{n}{libor\PYZus{}curve}\PY{o}{.}\PY{n}{forward\PYZus{}rate}\PY{p}{(}\PY{n+nb+bp}{self}\PY{o}{.}\PY{n}{floating\PYZus{}leg\PYZus{}dates}\PY{p}{[}\PY{n}{j}\PY{o}{\PYZhy{}}\PY{l+m+mi}{1}\PY{p}{]}\PY{p}{)}
            \PY{n}{tau} \PY{o}{=} \PY{p}{(}\PY{n+nb+bp}{self}\PY{o}{.}\PY{n}{floating\PYZus{}leg\PYZus{}dates}\PY{p}{[}\PY{n}{j}\PY{p}{]} \PY{o}{\PYZhy{}} \PYZbs{}
            \PY{n+nb+bp}{self}\PY{o}{.}\PY{n}{floating\PYZus{}leg\PYZus{}dates}\PY{p}{[}\PY{n}{j}\PY{o}{\PYZhy{}}\PY{l+m+mi}{1}\PY{p}{]}\PY{p}{)}\PY{o}{.}\PY{n}{days} \PY{o}{/} \PY{l+m+mi}{360}
            \PY{n}{P} \PY{o}{=} \PY{n}{discount\PYZus{}curve}\PY{o}{.}\PY{n}{df}\PY{p}{(}\PY{n+nb+bp}{self}\PY{o}{.}\PY{n}{floating\PYZus{}leg\PYZus{}dates}\PY{p}{[}\PY{n}{j}\PY{p}{]}\PY{p}{)}
            \PY{n}{s} \PY{o}{+}\PY{o}{=} \PY{n}{F} \PY{o}{*} \PY{n}{tau} \PY{o}{*} \PY{n}{P}
        \PY{k}{return} \PY{n}{s} \PY{o}{/} \PY{n+nb+bp}{self}\PY{o}{.}\PY{n}{annuity}\PY{p}{(}\PY{n}{discount\PYZus{}curve}\PY{p}{)}
	
    \PY{k}{def} \PY{n+nf}{npv}\PY{p}{(}\PY{n+nb+bp}{self}\PY{p}{,} \PY{n}{discount\PYZus{}curve}\PY{p}{,} \PY{n}{libor\PYZus{}curve}\PY{p}{)}\PY{p}{:}
        \PY{n}{S} \PY{o}{=} \PY{n+nb+bp}{self}\PY{o}{.}\PY{n}{swap\PYZus{}rate}\PY{p}{(}\PY{n}{discount\PYZus{}curve}\PY{p}{,} \PY{n}{libor\PYZus{}curve}\PY{p}{)}
        \PY{n}{A} \PY{o}{=} \PY{n+nb+bp}{self}\PY{o}{.}\PY{n}{annuity}\PY{p}{(}\PY{n}{discount\PYZus{}curve}\PY{p}{)}
        \PY{k}{return} \PY{n+nb+bp}{self}\PY{o}{.}\PY{n}{notional} \PY{o}{*} \PY{p}{(}\PY{n}{S} \PY{o}{\PYZhy{}} \PY{n+nb+bp}{self}\PY{o}{.}\PY{n}{fixed\PYZus{}rate}\PY{p}{)} \PY{o}{*} \PY{n}{A}
\end{Verbatim}
\end{tcolorbox}

This is just an example. Actually may be an overkill to use inheritance here, since there is not much of code to share between the classes (the implementation of the NPV calculation is different in each of them).
Anyway this is a practical application to show how it works.

\section{Swaptions}\label{interest-rate-swaptions}

Swaptions are the equivalent of European options for the interest rate
markets. They give the option holder the right but not the obligation,
at the exercise date \(d_{ex}\), to enter into an Interest Rate Swap at
a pre-determined fixed rate.

Clearly the option holder will only choose to do this if the NPV of the
underlying swap at \(d_{ex}\) is positive. Looking at the expression
for the NPV of the IRS in terms of the swap rate \(S\) therefore, we can
see that the payoff of the swaption is

\begin{equation}\mathrm{payoff} = N\cdot \mathrm{max}(0, S(d_{\mathrm{ex}}) - K)\cdot\sum D(d, d_i^{\mathrm{fixed}})
\label{eq:swaption_payoff}
\end{equation}

In order to evaluate the payoff of a swaption then the key issue is to estimate \(S(d_{\mathrm{ex}})\). This will be shown with two alternative approaches.

\subsection{Evaluation through Black-Scholes Formula}\label{evaluation-through-black-scholes-formula}

In this case, to evaluate the NPV of this payoff, we'll use a
generalization of the Black-Scholes-Merton formula applied to swaptions:

\begin{equation}\mathrm{payoff} = N\cdot A\cdot [S \Phi(d_+) - K\Phi(d_-)]\end{equation}
where \(\Phi\) represents the cumulative distribution function of the normal distribution

\begin{equation}
\begin{gathered}d_{\pm} = \frac{\mathrm{log}(\frac{S}{K}) \pm \frac{1}{2}\sigma^{2}T}{\sigma\sqrt{T}}\qquad(\sigma~\textrm{is the volatility of the swap rate})\\
A =\sum_{i=1}^{p}D(d, d_{i}^{\mathrm{fixed}})\qquad\mathrm{(annuity})
\end{gathered}
\end{equation}

As an example let's consider a swaption whose underlying 6M-IRS has a
notional of 1M, fixed rate of 1\%, and a maturity of 4 years. In
addition we assume a volatility associated to the swap rate of about
7\%.

\begin{tcolorbox}[breakable, size=fbox, boxrule=1pt, pad at break*=1mm,colback=cellbackground, colframe=cellborder]
\begin{Verbatim}[commandchars=\\\{\}]
\PY{k+kn}{from} \PY{n+nn}{math} \PY{k}{import} \PY{n}{log}
\PY{k+kn}{from} \PY{n+nn}{scipy}\PY{n+nn}{.}\PY{n+nn}{stats} \PY{k}{import} \PY{n}{norm} 
\PY{k+kn}{from} \PY{n+nn}{dateutil}\PY{n+nn}{.}\PY{n+nn}{relativedelta} \PY{k}{import} \PY{n}{relativedelta}

\PY{k}{def} \PY{n+nf}{npvSwaptionBS}\PY{p}{(}\PY{n}{irs}\PY{p}{,} \PY{n}{sigma}\PY{p}{, } 
                  \PY{n}{discount\PYZus{}curve}\PY{p}{,} \PY{n}{libor\PYZus{}curve, T}\PY{p}{)}\PY{p}{:}
    \PY{n}{A} \PY{o}{=} \PY{n}{irs}\PY{o}{.}\PY{n}{annuity}\PY{p}{(}\PY{n}{discount\PYZus{}curve}\PY{p}{)}
    \PY{n}{S} \PY{o}{=} \PY{n}{irs}\PY{o}{.}\PY{n}{swap\PYZus{}rate}\PY{p}{(}\PY{n}{discount\PYZus{}curve}\PY{p}{,} \PY{n}{libor\PYZus{}curve}\PY{p}{)}
    \PY{n}{K} \PY{o}{=} \PY{n}{irs}\PY{o}{.}\PY{n}{fixed\PYZus{}rate}
    \PY{n}{N} \PY{o}{=} \PY{n}{irs}\PY{o}{.}\PY{n}{notional}
    
    \PY{n}{d\PYZus{}plus} \PY{o}{=} \PY{p}{(}\PY{n}{log}\PY{p}{(}\PY{n}{S}\PY{o}{/}\PY{n}{K}\PY{p}{)} \PY{o}{+} \PY{l+m+mf}{0.5} \PY{o}{*} \PY{n}{sigma}\PY{o}{*}\PY{o}{*}\PY{l+m+mi}{2} \PY{o}{*} \PY{n}{T}\PY{p}{)} \PY{o}{/} \PY{p}{(}\PY{n}{sigma} \PY{o}{*} \PY{n}{T}\PY{o}{*}\PY{o}{*}\PY{l+m+mf}{0.5}\PY{p}{)}
    \PY{n}{d\PYZus{}minus} \PY{o}{=} \PY{p}{(}\PY{n}{log}\PY{p}{(}\PY{n}{S}\PY{o}{/}\PY{n}{K}\PY{p}{)} \PY{o}{\PYZhy{}} \PY{l+m+mf}{0.5} \PY{o}{*} \PY{n}{sigma}\PY{o}{*}\PY{o}{*}\PY{l+m+mi}{2} \PY{o}{*} \PY{n}{T}\PY{p}{)} \PY{o}{/} \PY{p}{(}\PY{n}{sigma} \PY{o}{*} \PY{n}{T}\PY{o}{*}\PY{o}{*}\PY{l+m+mf}{0.5}\PY{p}{)}
    \PY{k}{return} \PY{n}{irs}\PY{o}{.}\PY{n}{notional} \PY{o}{*} \PY{n}{A} \PY{o}{*} \PY{p}{(}\PY{n}{S} \PY{o}{*} \PY{n}{norm}\PY{o}{.}\PY{n}{cdf}\PY{p}{(}\PY{n}{d\PYZus{}plus}\PY{p}{)} \PY{o}{\PYZhy{}} \PY{n}{K} \PY{o}{*} \PY{n}{norm}\PY{o}{.}\PY{n}{cdf}\PY{p}{(}\PY{n}{d\PYZus{}minus}\PY{p}{)}\PY{p}{)}

\PY{n}{sigma} \PY{o}{=} \PY{l+m+mf}{0.07}
\PY{n}{irs} \PY{o}{=} \PY{n}{InterestRateSwap}\PY{p}{(}\PY{n}{start\PYZus{}date}\PY{p}{,} \PY{l+m+mf}{1e6}\PY{p}{,} \PY{l+m+mf}{0.01}\PY{p}{,} \PY{l+m+mi}{6}\PY{p}{,} \PY{l+m+mi}{4}\PY{p}{)}
\PY{n}{exercise\PYZus{}date} \PY{o}{=} \PY{n}{start\PYZus{}date} 
\PY{n}{T} \PY{o}{=} \PY{p}{(}\PY{n}{exercise\PYZus{}date} \PY{o}{\PYZhy{}} \PY{n}{observation\PYZus{}date}\PY{p}{)}\PY{o}{.}\PY{n}{days} \PY{o}{/} \PY{l+m+mi}{365}
\PY{n}{npv} \PY{o}{=} \PY{n}{npvSwaptionBS}\PY{p}{(}\PY{n}{irs}\PY{p}{,} \PY{n}{sigma}\PY{p}{,} \PY{n}{dc}\PY{p}{,} \PY{n}{fr, T}\PY{p}{)}

\PY{n+nb}{print}\PY{p}{(}\PY{l+s+s2}{\PYZdq{}}\PY{l+s+s2}{Swaption NPV with BS: }\PY{l+s+si}{\PYZob{}:.3f\PYZcb{}}\PY{l+s+s2}{ EUR}\PY{l+s+s2}{\PYZdq{}}\PY{o}{.}\PY{n}{format}\PY{p}{(}\PY{n}{npv}\PY{p}{)}\PY{p}{)}

Swaption NPV with BS: 223887.09 EUR
\end{Verbatim}
\end{tcolorbox}

\subsection{Evaluation through Monte-Carlo Simulation}\label{evaluation-through-monte-carlo-simulation}

In this second case we start from the current swap rate \(S(d)\)
evaluated at the pricing date \(d\), and assume that it follows a
log-normal stochastic process, i.e. its distribution at
\(d_{\mathrm{ex}}\) (exercise date) is
\(S(d_{\mathrm{ex}}) = S(d)\mathrm{exp}(-\frac{1}{2}\sigma^{2}T+\sigma\sqrt{T}\epsilon)\)
where \(\epsilon\approx\mathcal{N}(0,1)\) (notice that it is assumed
that the \emph{drift} rate in the evolution of the swap rate is zero).
Given that the discounted payoff is given by Eq.~\ref{eq:swaption_payoff}
to perform the simulation we can:

\begin{itemize}
\tightlist
\item
  sample the normal distribution \(\mathcal{N}(0, 1)\) to calculate a
  large number of scenarios for \(S(d_{\mathrm{ex}})\);
\item
  evaluate the underlying swap's NPV at the expiry date, and
  consequently the swaption's payoff, for each scenario;
\item
  take the average of these values to get the final estimate.
\end{itemize}

\begin{tcolorbox}[breakable, size=fbox, boxrule=1pt, pad at break*=1mm,colback=cellbackground, colframe=cellborder]
\begin{Verbatim}[commandchars=\\\{\}]
\PY{k+kn}{import} \PY{n+nn}{numpy} \PY{k}{as} \PY{n+nn}{np}
\PY{k+kn}{from} \PY{n+nn}{math} \PY{k}{import} \PY{n}{exp}\PY{p}{,} \PY{n}{sqrt}
\PY{k+kn}{from} \PY{n+nn}{numpy}\PY{n+nn}{.}\PY{n+nn}{random} \PY{k}{import} \PY{n}{normal}\PY{p}{,} \PY{n}{seed}

\PY{n}{n\PYZus{}scenarios} \PY{o}{=} \PY{l+m+mi}{100000}
\PY{n}{discounted\PYZus{}payoffs} \PY{o}{=} \PY{p}{[}\PY{p}{]}
\PY{n}{seed}\PY{p}{(}\PY{l+m+mi}{1}\PY{p}{)}

\PY{n}{T} \PY{o}{=} \PY{p}{(}\PY{n}{exercise\PYZus{}date} \PY{o}{\PYZhy{}} \PY{n}{observation\PYZus{}date}\PY{p}{)}\PY{o}{.}\PY{n}{days} \PY{o}{/} \PY{l+m+mi}{365}
\PY{n}{A} \PY{o}{=} \PY{n}{irs}\PY{o}{.}\PY{n}{annuity}\PY{p}{(}\PY{n}{dc}\PY{p}{)}
\PY{n}{S0} \PY{o}{=} \PY{n}{irs}\PY{o}{.}\PY{n}{swap\PYZus{}rate}\PY{p}{(}\PY{n}{dc}\PY{p}{,} \PY{n}{fr}\PY{p}{)}
    
\PY{k}{for} \PY{n}{i\PYZus{}scenario} \PY{o+ow}{in} \PY{n+nb}{range}\PY{p}{(}\PY{n}{n\PYZus{}scenarios}\PY{p}{)}\PY{p}{:}
    \PY{n}{S} \PY{o}{=} \PY{n}{S0} \PY{o}{*} \PY{n}{exp}\PY{p}{(}\PY{o}{\PYZhy{}}\PY{l+m+mf}{0.5} \PY{o}{*} \PY{n}{sigma} \PY{o}{*} \PY{n}{sigma} \PY{o}{*} \PY{n}{T} \PY{o}{+}
                          \PY{n}{sigma} \PY{o}{*} \PY{n}{sqrt}\PY{p}{(}\PY{n}{T}\PY{p}{)} \PY{o}{*} \PY{n}{normal}\PY{p}{(}\PY{p}{)}\PY{p}{)}
    
    \PY{n}{swap\PYZus{}npv} \PY{o}{=} \PY{n}{irs}\PY{o}{.}\PY{n}{notional} \PY{o}{*} \PY{n+nb}{max}\PY{p}{(}\PY{n+nb}{0}\PY{n}{,S} \PY{o}{\PYZhy{}} \PY{n}{irs}\PY{o}{.}\PY{n}{fixed\PYZus{}rate}\PY{p}{)} \PY{o}{*} \PY{n}{A}
    
    \PY{n}{discounted\PYZus{}payoffs}\PY{o}{.}\PY{n}{append}\PY{p}{(}\PY{n+nb}{max}\PY{p}{(}\PY{l+m+mi}{0}\PY{p}{,} \PY{n}{swap\PYZus{}npv}\PY{p}{)}\PY{p}{)}
    
\PY{n}{npv\PYZus{}mc} \PY{o}{=} \PY{n}{np}\PY{o}{.}\PY{n}{mean}\PY{p}{(}\PY{n}{discounted\PYZus{}payoffs}\PY{p}{)}
\PY{n+nb}{print}\PY{p}{(}\PY{l+s+s2}{\PYZdq{}}\PY{l+s+s2}{Swaption NPV: }\PY{l+s+si}{\PYZob{}:.2f\PYZcb{}}\PY{l+s+s2}{ EUR}\PY{l+s+s2}{\PYZdq{}}\PY{o}{.}\PY{n}{format}\PY{p}{(}\PY{n}{npv\PYZus{}mc}\PY{p}{)}\PY{p}{)}

Swaption NPV: 223915.50 EUR
\end{Verbatim}
\end{tcolorbox}

%Note that this is not \emph{strictly speaking} the correct way of
%calculating the swaption NPV, the reason being that one should calculate
%the swap NPV at the expiry date of the swaption, apply the payoff
%function max(0, \ldots{}) and \emph{then} discount from the expiry date
%to today.
%
%However, it's simpler to calculate it as above and it doesn't make any
%difference for the result, since
%\[ DF\cdot \mathrm{max}(0, \mathrm{SwapNPVAtExpiry}) = \mathrm{max}(0, DF \cdot\mathrm{SwapNPVAtExpiry}) \]

\subsection{Confidence Interval of MC Simulation}

Using the confidence interval we can check whether the Monte Carlo estimate of the swaption payoff is in agreement with what computed using the Black-Scholes formula.
Let's then calculate the 95\% confidence level for the swaption simulation:

\begin{tcolorbox}[breakable, size=fbox, boxrule=1pt, pad at break*=1mm,colback=cellbackground, colframe=cellborder]
\begin{Verbatim}[commandchars=\\\{\}]
\PY{n}{npv\PYZus{}error} \PY{o}{=} \PY{l+m+mf}{1.96} \PY{o}{*} \PY{n}{np}\PY{o}{.}\PY{n}{std}\PY{p}{(}\PY{n}{discounted\PYZus{}payoffs}\PY{p}{)}\PY{o}{/}\PY{n}{sqrt}\PY{p}{(}\PY{n}{n\PYZus{}scenarios}\PY{p}{)}
				
\PY{n+nb}{print}\PY{p}{(}\PY{l+s+s2}{\PYZdq{}}\PY{l+s+s2}{Swaption NPV: }\PY{l+s+si}{\PYZob{}:.2f\PYZcb{}}\PY{l+s+s2}{ EUR (+/- }\PY{l+s+si}{\PYZob{}:.2f\PYZcb{}}\PY{l+s+s2}{ EUR with 95}\PY{l+s+si}{\PYZpc{} c}\PY{l+s+s2}{onfidence)}\PY{l+s+s2}{\PYZdq{}}\PYZbs{}
      \PY{o}{.}\PY{n}{format}\PY{p}{(}\PY{n}{npv\PYZus{}mc}\PY{p}{,} \PY{n}{npv\PYZus{}error}\PY{p}{)}\PY{p}{)}

Swaption NPV: 223915.50 EUR (+/- 34.62 EUR with 95\% confidence)
\end{Verbatim}
\end{tcolorbox}

The NPV calculated via the Black-Scholes formula
falls well within the confidence interval produced by the Monte Carlo
simulation:

\begin{itemize}
\tightlist
\item
  Swaption NPV (BS): \euro{223887}
\item
  Swaption NPV (MC): \euro{223915} $\pm$ 34
\end{itemize}
so we can assert that the two estimates are in agreement at the 95\% confidence level.

\chapter{Short Rate Models}

The volume of traded interest rate derivatives in both the OTC and exchange-traded 
markets have been increasing sharply over the past decades. As the number of new 
and simultaneously exotic interest rate products virtually exploded, 
it has been a key challenge to find good and primarily robust procedures for 
pricing and also hedging these products. Interest rate derivatives are more 
difficult to value than equity derivatives, which are often priced by an ordinary 
or some slight variation of the Black-Scholes model. 
The following points give some reasons for this:
- Interest rates do typically not follow a GBM as stock prices normally do.
- In most cases, we have to describe the development and general behavior of the entire term structure in order to get reasonable prices.
- Volatilities can no longer assumed to be constant since they highly depend on particular points of the yield curve. This makes intuitively sense if we consider the volatility of a bond price, which becomes smaller and smaller as the duration decreases and its maturity is approaching.
- Interest rates are no longer used for discounting purposes only but they additionally define the payoffs from the derivatives.

The need to go beyond Black’s closed-form formula in order to be able to 
value more sophisticated products has led to the development of several 
methods for simulating stochastic
interest rate models. More precisely, these models describe the entire 
term structure instead of a single final value. The ultimate purpose of 
interest rate modeling is to acquire a profound understanding of interest 
rate behavior. We would like to explore the dynamics of interest rates by 
fitting a model to available interest rate data. Furthermore, we are 
interested in the way that interest rates and corresponding derivative 
prices eventually relate together, both to understand them and last 
but not least to be able to price interest rate products correctly.

6.1 Types of interest rate models
Numerous different models have been introduced over the past years, 
all with the ultimate purpose to capture the uncertain dynamics of 
interest rates. In general, we can classify these models by the 
following two different criteria.

Number of stochastic factors
In the simplest models, the so called one factor models, we describe the 
dynamics of the short rate r(t) (Section 6.2) or the forward 
rate f (t, T ) (Section 6.3) by just one stochastic factor, thus there 
is one source of uncertainty only. As a matter of fact, bond prices, 
for instance, depend not only on one single but rather on the entire 
term structure of interest rates. Therefore, one factor models are 
essentially assuming that the dynamics of the whole term structure is 
captured by the short or forward rate, and its future evolution. Nevertheless, 
one factor models are not as restrictive as it might appear. They imply 
that all rates move in the same direction over any short time interval, 
but they do not assume that they move by the same amount [56, p.650].

Yet, it is sometimes difficult or even infeasible to find a more or less 
realistic model based on one factor only. A principal component analysis, 
conducted by Rebonato [92, p.263], indicated that a large proportion of 
the variance of interest rates of different maturities may be satisfactorily 
explained by including two or three orthogonal factors. The first relevant 
principal component explained the average level of the yield curve, the 
second the curve’s slope, and the third its curvature. These three factors 
eventually accounted together for approximately 95 to 99\% of the observed 
term structure variability.
On the other hand, a multifactor model means that we have to deal with an 
increased complexity of the model as we will learn in the course of the 
following sections. Generally, in a q-factor model, where q denotes an 
arbitrary, usually small number of factors, the price changes of 
these q bonds completely explain the changes in the prices of all 
other existing bonds. Moreover, with additional factors the required 
number of sample paths to achieve a particular level of accuracy in the 
simulation grows with the power of the number of factors. 
Therefore, the choice of the number of factors in a model should be made with care.

Equilibrium or no-arbitrage models
The starting point for equilibrium models is usually given by 
assumptions about how the economy works and about economic 
variables themselves. Then they attempt to derive a process 
for the short rate r(t) and explore what the defined process 
for r may imply for bond and option prices [56, p.650]. Thus, 
the key idea behind this approach is to build an economically 
sound model, whose output are interest rates evolved as a consequence 
of market equilibrium [15, p.126]. It may happen that the initial 
theoretical set of bond prices is not equal to the prices observed 
on the market. This might result in possible arbitrage opportunities. 
This is exactly the point where equilibrium models have often been criticized. 
They do not automatically fit today’s term structure of interest rates. 
The current term structure is included as an output rather than an input. 
This eventually gives rise to inconsistencies with the actual term structure. 
Therefore, in practice there is little confidence in the bond option 
prices when the applied model is not even able to price the underlying 
bond correctly.
The no-arbitrage model is an alternative idea which is trying to build models that match the currently observed term structure exactly, that means the current term structure is used as an input. Thus, these models also provide us with underlying bond prices that are correct. However, in order to fit today’s term structure, the model parameters have to be calibrated to prevailing market data. The calibration process is, amongst others, responsible that a model eventually yields consistent instrument prices. Nonetheless, calibrating a model correctly is somehow burdensome and takes significant additional computing effort [77, p.93].
6.2 Short rate models
t
Short rate models have been introduced in order to be able to explore the development of the dynamics of an instantaneous continuously compounded short rate r(t). In this section the focus lies on simulating some of these simple yet important models. Assume that an investment of 1 in a money market account at time t, earning an interest rate of r(u) at time u, grows to a value of
􏰉􏰑T 􏰊
V (T ) = exp r(u)du (6.1)
at time T [46, 60, p.108, p.176f]. Therefore, the arbitrage-free price of a derivative instrument at time t that yields a payoff of Y (T) at time T is the expected value of Y (T)/V (T) [20, p.52]:
􏰋􏰉􏰑T􏰊􏰌
Et exp − r(u)du Y(T) , (6.2)
t
where Et stands for the time t-conditional expectation. Even though (6.1) is now a stochastic
quantity, it remains the numeraire for risk-neutral pricing. Particularly, the price of a zero
coupon bond at time t that pays 1 at time T is defined, given that V (T) = 1, as 􏰋􏰉􏰑T 􏰊􏰌
P (t, T ) = Et exp − r(u)du . (6.3) t
47

6 Term structure modeling
Equation (6.3) guarantees that w􏰎e are always able to calculate bond prices P whenever we can
􏰃T􏰄
define the distribution of exp − r(u)du in terms of a certain dynamic structure for r(t) [20,
t
p.51].
Short rate models generally owe their popularity to both their high degree of tractability and
their flexibility [60, p.177]. For many models one can find explicit solutions for bond prices and for bond option prices. They may be applied to price simple instruments in closed form or sometimes by deterministic numerical methods. However, there are still some extensions of the basic models which require Monte Carlo simulation for the computation of expressions of the form (6.2) [46, p.108].
In particular, the stochastic evolution of short rate models is identified by the following gen- eralized SDE [20, 25, 77, 92, p.52f, p.51f, p.91, p.181]:
dr(t) = (θ(t) − α(t)r(t)) dt + σ(t)r(t)γ dW (t). (6.4)
This equation denotes a general Gaussian Markov process with θ, α and σ all deterministic functions of time. Most of the particular short rate models can be expressed by (6.4). Notice that this process does not follow a GBM anymore. Interest rates, unlike stock prices, appear to be pulled back so some long-run average interest rate level over time. This phenomenon is well-known as mean reversion. The process in (6.4) does exhibit mean reversion where the parameter α measures the mean reversion speed towards a long-term interest rate level θ/α = b. Additionally, γ is a measure of the degree to which the volatility σ of the short rate depends on the current level of r(t). Obviously, a higher γ simultaneously means that the volatility reacts sensitively [77, p.91]. Note that in short rate models, in contrast to other models we will look at in Sections 6.3, 7 and 8, the dynamics of all interest rates are totally determined by the dynamics of the overnight rate.
Vasicek
One of the earliest one factor stochastic equilibrium models of the term structure was developed and proposed by Vasicek [102] in 1977. This classical model describes the short rate through a so called Ornstein-Uhlenbeck process. An unspecified short-term interest rate serves as a basis for the model. Furthermore, the Vasicek model corresponds to the choices α(t) ≡ α, θ(t)/α(t) ≡ b and σ(t) ≡ σ constant while γ = 0 [25, p.52]. Thus, the risk adjusted dynamics in the Vasicek model of r(t) are given by the following SDE [20, 25, 102, p.53, p.52, p.185]:
dr(t) = α(b − r(t))dt + σdW (t). (6.5) In order to be able to simulate this process, we require the solution for the SDE (6.5). The
solution for any 0 < u < t [25, 46, p.53, p.109]
􏰑t 􏰑t
r(t) = r(u)e−α(t−u) + α e−α(t−s)b(s)ds + σ e−α(t−s)dW (s) (6.6) uu
48

6 Term structure modeling
can be found through an application of Itˆo’s formula. Consequently, for a given r(u) the value r(t) is normally distributed with mean [46, 102, p.109f, p.186]
and variance
􏰑 t
σ2 􏰃
􏰄
r(u)e−α(t−u) + μ(u, t), with μ(u, t) ≡ α
􏰑t u
e−α(t−s)b(s)ds,
σ2(u, t) ≡ σ2
e−2α(t−s)ds =
1 − e−2α(t−u)
2α
In a next step we are interested in how we may eventually simulate a short rate process by
Vasicek’s model. The simulation of r at the discrete-time steps 0 = t0 < t1 < · · · < tn can be carried out by [46, p.110]
r(t )=r(t)e−α(ti+1−ti)+μ(t,t )+σ(t,t )Z , (6.7) i+1 i i i+1 r i i+1 i+1
where we repeatedly draw Zi, . . . , Zn independent variates from φ(0, 1). Note that Equation (6.7) allows an exact simulation. Exact in the sense that the distribution of the simulated short rates r(t1),...,r(tn) is precisely that of the original Vasicek process at times t1,...,tn for the identical value of r(0) [46, p.110].
The main advantage of the Vasicek model is its analytical tractability what makes the model relatively easy to understand, implement and eventually apply. However, there are also some shortcomings compared to rival models. Firstly, under the risk-neutral measure the short rate r is normally distributed. Thus there is a positive probability that r becomes negative. Secondly, the Vasicek model does not leave complete flexibility in simulating any specific shapes of the term structure. In particular, possible shapes are upward-sloping, downward-sloping or slightly humped only. Furthermore, since the parameter γ is assumed to be equal to zero, the volatility of changes is forced to be both constant and independent of r. This assumption is restrictive with respect to the incorporation of different volatility structures [16, 20, 56, 60, p.168f, p.58ff, p.651f, p.181].
Cox, Ingersoll and Ross
Cox, Ingersoll and Ross [31] proposed a new class of equilibrium processes, generally referred to as CIR models, which include a square-root diffusion term. The risk adjusted dynamics of the short term rate in a CIR model are defined by the SDE [31, p.391]
dr(t) = α(b − r(t))dt + σ􏰔r(t)dW (t) (6.8)
and correspond to the choices α(t) ≡ α, θ(t)/α(t) ≡ b and σ(t) ≡ σ constant equal to the Vasicek model, whereas now γ = 1/2 [25, p.53]. In this general equilibrium framework holds if the change in production opportunities is assumed to follow a process of the form (6.8), then the short rate does as well [46, p.120]. The CIR model has been very successful and often applied for many years and thus established some kind of a benchmark. The reason for this lies in its analytical tractability and the fact that the instantaneous short rates remain always positive.
49
u
.

6 Term structure modeling
Contrary to the Vasicek model, the square-root diffusion term σ􏰔r(t) drops off to zero while r(t) approaches the origin and this prevents r(t) from becoming negative [20, 25, p.64, p.52]. As in the Vasicek model, the term structure in a CIR model may take the forms upward- and downward-sloping as well as slightly humped. The level of the entire term structure, yet not the general shape, at time t is fully determined by the value of r(t) [56, p.653].
Since there are still explicit derivative pricing formulae available, though harder to find than in the Vasicek model, numerical approaches as simulation techniques are not the center stage and therefore the CIR model is not amplified in this paper.
Ho-Lee
Ho and Lee [55] did pioneer work by proposing the first no-arbitrage model consistent with the initial term structure. Allowing for time-dependent drift parameters typically makes short rate models consistent with a prevailing set of bond prices. In the continuous-time Ho-Lee model the stochastic process followed by the short rate is [55, 60, p.1020, p.184]
dr(t) = θ(t)dt + σdW (t). (6.9)
Equation (6.9) corresponds to an Ornstein-Uhlenbeck process with a time dependent drift pa- rameter. This enables the model to be consistent with current market data. θ(t) directly defines the average direction that r moves at time t. The fact that the SDE (6.9) does not incorporate mean reversion is clearly a shortcoming of this approach [56, p.654f]. Similar to the Vasicek model, for every t there is a positive probability that the interest rates become negative. On the other hand, the main advantage of the Ho-Lee model is that the market price of risk is irrelevant and does not have to be estimated in order to determine the risk-adjusted process required to price interest rate derivatives. The risk-neutral process is automatically defined by fitting the model to the current term structure [54, 56, 77, p.11ff, p.654f, p.93].
Applying Itˆo’s lemma, the SDE (6.9) can be solved explicitly for bond prices and short rates. There is an explicit formula available in the Ho-Lee model and thus the term structures are defined as [46, 60, p.111, p.184]
􏰑t 0
Given this equation, only a small step brings us to the simulation of the Ho-Lee model. Sampling Z1, Z2, . . . from a standard normal distribution and inserting into (6.10) gives us a possible term structure generated by Ho-Lee.
Hull-White
The unsatisfactory fitting of the observed term structure of interest rates implied by the Vasicek model brought Hull and White in their paper [57] to propose a new model that basically combines the advantages of the Vasicek and the Ho-Lee model. Therefore, in literature the Hull-White
50
r(t) = r(0) +
θ(s)ds + σdW (t). (6.10)

model is also referred to as Extended Vasicek model. The dynamics of the short rates in the Hull-White model are given by the SDE [20, 57, p.72, p.577]
dr(t) = α (b(t) − r(t)) dt + σdW (t). (6.11)
Firstly, the Hull-White model makes the parameter b time-dependent. Thereby enough degrees of freedom are available to fit the model to the initial term structure and the term structure of spot or forward rate volatilities. Thus the Hull-White model falls into the category of no- arbitrage models and is therefore consistent with the current market data [16, 57, p.173, p.577f]. Secondly, it can be characterized as the Ho-Lee model with mean reversion at rate α. The model is to the same extent analytically tractable as the Ho-Lee model. The mean reversion level, i.e. the function b(t), can be directly calculated from the initial term structure [56, p.656].
For an extensive discussion of this model the interested reader is referred either to the initial paper of Hull and White [57] or to the Chapter 3.3 of the book written by Brigo and Mercurio [20, p.71-80].
Table 5, according to Brigo and Mercurio [20, p.57], gives a review of some established instan- taneous short rate models. Even though this list is not exhaustive, it includes most fundamental approaches in order to model a term structure of short rates. Note that some of these models, Hull-White, for instance, have been further developed and extended to multifactor models.
51

6.3 Forward rate models
6 Term structure modeling
Forward rate models describe the arbitrage-free dynamics of the term structure of interest rates through the evolution of forward rates f (t, T ). The distinguishing feature of these models is that they explicitly describe the evolution of the full term structure, contrary to the previously discussed short rate models, which do only provide a description of the dynamics of the short rate r(t) [46, p.149]. Instantaneous short rate models, as we have seen before, are mostly easy to implement and are usually able to price many standard and also nonstandard interest rate derivatives consistently [56, p.679].
However, to legitimize the application of forward rate models, there are the two major limi- tations of short rate models that are generally overcome by using different types of forward rate models:
- In short rate models the current value of all term structure quantities is solely determined by the current value of the short rate. Thus, the term structure is entirely summarized by today’s short rate. In multifactor models the complexity of the yield curve dynamics is subsumed into the current values of a finite, but usually small number of underlying factors [46, 56, p.149, p.679f]. The new generation of correlation-dependent instruments, however, requires models that describe the state of the world by the full term structure and not necessarily by a finite number of factors, which are often not able to deal with this new dimension of complexity [92, p.311].
- Short rate models do not provide us with a complete freedom in the choice of the corre- sponding volatility structure. In a forward rate framework the forward rate dynamics are completely specified through their instantaneous volatility structure. In contrast, in short rate models the volatility of the short rate alone is not sufficient to fully characterize the relevant interest rate model [20, 56, 92, p.183, p.679, p.311f].
Forward rate models can basically be divided into two classes, namely models based on con- tinuous and simple rates, respectively. The by far most popular continuous forward rate model is the Heath, Jarrow and Morton framework. It was one of the first forward rate models pro- posed and simultaneously establishes a basis for models based on simple rates. These models are therefore closely related to the Heath, Jarrow and Morton approach and called LIBOR market models. The seemingly minor shift from continuous to simple rates has surprisingly far-reaching practical and theoretical implications [46, p.165f], as we will see later on. As in the application of both models, Heath–Jarrow–Morton as well as LIBOR market model, Monte Carlo tech- niques play a decisive role, we will give these two approaches a fundamental treatment and will implement them in Section 7 and 8, respectively.
52

7 Heath, Jarrow and Morton
In 1992 Heath, Jarrow and Morton (hereafter, HJM) published the important paper Bond pricing and the term structure of interest rates [53] describing an alternative framework for modeling the term structure of interest rates. Under the risk-neutral measure, they derived a generalized formula for the drifts of instantaneous forward rates in terms of the volatilities of the forward rates. One of the key insights of this framework is that the HJM model is completely defined by specifying the volatilities of forward rates. Consequently, by using the observed term structure of forward rates as an input one can easily match a HJM model with the current market discount bond prices [6, p.310]
Basically, there are two common formulations of the model. Firstly, the price based and secondly the forward based approach. The former technique is based on the dynamics of discount bonds, i.e. it takes them as the fundamental building block. The latter directly obtains the no- arbitrage SDE which is obeyed by the forward rates [92, p.314]. For a rigorous proof of the equivalence of these two approaches it is referred to the paper of Carverhill [27]. Note, though, that, historically, the pioneering results of the HJM were first achieved in the forward rates context.
The virtue of the HJM theory lies in the fact that within such a framework virtually any (exogenous term structure) interest rate model can be derived. Even market models (see Section 8), for instance, have evolved starting from the instantaneous forward rate dynamics of the HJM approach. However, the disadvantage is that there are only a few volatility structures which lead to a corresponding short rate process which is indeed Markovian. This means that models of this form will, except for some wise choices of volatility specifications, which we will look at in Section 7.2, generally be path-dependent and thus non-Markovian processes [6, 20, p.310, p.184]. A short rate process can be characterized as being Markovian if its future evolution is in no way affected, in a stochastic sense, by its past realizations. If a process is indeed not influenced by its past, then the short rate process is, to use Carverhill’s [26] apt oxymoron, randomly determined [92, p.347f].
In general, prices of fixed-income instruments, e.g. bonds, do not depend on the values of a few individual factors but rather on the entire history of the forward rate process. This fact makes computation exceptionally difficult and often only possible by simulation since the nodes of non-recombining, or bushy trees grow exponentially and therefore the approximating lattice will literally explode already after a small number of steps [6, p.310]. As a result, Monte Carlo simulation is the computational tool of choice in the HJM setting.
7.1 Framework
As mentioned before, the HJM characterizes the dynamics of the entire forward rate curve f (t, T ), whereas 0 ≤ t ≤ T ≤ τ . The origin of any consistent implementation of the HJM approach is the current yield curve, which is obviously based on market data. This yield curve can be described either by the collection of discount bonds P(0,T) or by the instantaneous
53

7 Heath, Jarrow and Morton
forward rates f (t, T ). The forward rate at time t for date T > t is defined by [46, 53, 92, p.150, p.79, p.314]
∂lnP(t,T) ∂T
By solving the differential equation of Expression (7.1) the relationship between forward rates, spot rates and bond prices become obvious. In particular, spot rates, r(t,T), and bond prices, P(t,T), are written in terms of forward rates, f(t,T), as
and
r(t, T ) = P(t,T) = exp −
f(t,T) = −
. (7.1)
1􏰑T T−tt
f (t, s)ds 􏰊
f(t,s)ds ,
(7.2) (7.3)
􏰉 􏰑T t
respectively, for all T ∈ [0,τ], t ∈ [0,T] [6, 53, 92, p.310, p.80, p.314]. Note that the spot rate at time t, r(t), is exactly the same as the instantaneous forward rate at time t for date t, i.e., r(t) = f (t, t) [53, p.80].
In the HJM setting, the dynamics of the instantaneous forward rate curve are specified, under a given measure, through the SDE [46, 53, 60, p.150, p.80f, p.200]
df(t,T) = α(t,T,ω)dt+σ(t,T,ω)dW(t), (7.4)
where ω is a vector containing the past and present values of interest rates as well as bond prices at time t in the sample space Ω. The process W is a standard d-dimensional Brownian motion, whereas d is the number of factors or volatility curves which describe the forward rates. Thus, the HJM approach can be constructed as a one factor as well as a multifactor model.
Due to the absence of arbitrage, requested by the no-arbitrage conditio􏰎n, asset prices must
be martingales when divided by the numeraire, which in this case is exp 􏰃 0t r(s)ds􏰄. However,
forward rates and asset prices are by no means the same. Therefore, the restrictions imposed
on the dynamics in (7.4) with the purpose to avoid any arbitrage opportunities are different.
Nevertheless, asset prices, particularly bonds, serve as a starting point􏰎in order to find these 􏰃t􏰄
restrictions. To ensure that the discounted bond prices P (t, T ) exp − r(s)ds are positive
0
martingales, Heath et al. [53, p.81f] derived the dynamics of the form dP(t,T)
P(t,T) =r(t)dt+v(t,T,ω)dW(t), (7.5)
where 0 ≤ t ≤ T ≤ τ [46, 60, p.151, p.200]. The volatilities of the bonds v(t, T, ω) may either be functions of observed bond prices or, equivalently, of current forward rates since (7.1) indicates that they are one-to-one related to each other. Applying Itˆo’s formula yet again, Heath et al. showed [53, p.80ff] that forward rate volatilities may be derived from bond price volatilities and we therefore must have
∂ ∂T
σ(t,T,ω) = − 54
v(t,T,ω)

7 Heath, Jarrow and Morton
and 􏰑T
v(t, T, ω) = − σ(t, s)ds + constant, (7.6)
t
simultaneously [46, p.152]. Notice, however, that P (t, T ) approaches 1 as t → T and therefore we must have v(T,T,ω) = 0 because bond’s price volatility declines to zero at maturity [92, p.318]. Thus the constant in Equation (7.6) is zero which allows us to rewrite the expression for α as [20, 46, 60, p.186, p.152f, p.200f]
􏰑T 􏰏N 􏰑T
α(t, T, ω) = σ(t, T, ω) σ(t, s)ds = σi(t, T, ω) σi(t, s)ds. (7.7)
t i=1 t
This expression represents the risk-neutral drift imposed by the no-arbitage condition. Addition- ally, (7.7) reveals a key result of the HJM research activities. The assumption of arbitrage-free dynamics has led to a specific relationship between the drift and the volatility. Namely that the drift of the dynamics in (7.4) is completely determined by the volatility structure of the instan- taneous forward rates. Thus we know that the HJM drifts themselves are simply a function of the volatilities, either of forward rates, or of discount bonds. If we substitute (7.7) into (7.4) then the new HJM dynamics are [46, 53, 92, p.152, p.89f, p.318ff]
􏰉􏰑T􏰊
df(t,T)= σ(t,T,ω) σ(t,s)ds dt+σ(t,T,ω)dW(t). (7.8)
t
This SDE defines the no-arbitrage dynamics of the forward rate curve under the risk-neutral measure and is simultaneously the central insight of the HJM approach. Recall why the model eventually fulfills the required no-arbitrage condition. While deriving the forward rate restriction (7.7) and the corresponding dynamics (7.8), a proper specification of the drift was chosen in order to guarantee the absence of arbitrage, or more precisely, to make the discounted bond prices martingales. However, integrating (7.8) leads to the dynamics of f(t,T)
􏰑t􏰑T􏰑t
f(t,T) = f(0,T)+ σ(u,T,ω) σ(u,s)dsdu+ σ(s,T,ω)dW(s), (7.9)
0u0
which are completely determined once the vector volatility function σ is specified [20, 53, p.186, p.89f]. Only a small step takes us now from the dynamics of the forward rates to the dynamics of the zero coupon bond price P(t,T):
􏰋 􏰉􏰑T 􏰊 􏰌
dP(t,T)=P(t,T) r(t)dt− σ(t,s)ds dW(t) , (7.10)
t
where r(t) denotes the instantaneous short term interest rate at time t. The dynamics of r(t) are given by [20, 53, p.186, p.90]
􏰑t􏰑t􏰑t
r(t) = f (t, t) = f (0, t) + σ(u, t, ω) σ(u, s)ds du + σ(s, t, ω)dW (s). (7.11)
0u0
55

7 Heath, Jarrow and Morton
The decisive discrepancy between HJM and previously discussed instantaneous short rate models has become apparent. We know that the HJM drift is determined once the volatility is finally specified. In contrast, the derivation of the short rate models disclosed that their dynamics and thus their drift parameters could be completely specified independent of any diffusion coefficients. Even though no volatility structures were taken into account, there was still no arbitrage introduced. A wise choice of the drift parameters is indeed crucial for calibrating short rate models to the observed bond prices. Provided that the initial forward rate curve is chosen in a way that the consistency, implied by (7.1), is guaranteed, the HJM model, however, is automatically calibrated to prevailing bond prices. To conclude, while calibrating a HJM model the initial forward curve conditions rather than individual parameters are essential. With regard to the calibrating process of a HJM framework itself, the main effort lies in choosing σ properly to firstly match bond prices and eventually the market prices of interest rate derivatives [46, 60, 92, p.153, p.200, p.312f].
7.2 Volatility functions
The proceeding section has already foreshadowed that the consequences of the choice of volatility specification for the implementation and simulation of a HJM model are enormous. The short rate process (7.11) is, as mentioned before, not a Markov process in general. However, there are suitable specifications of the volatility function σ(t, T, ω) for which r(t) is yet Markovian [20, 60, p.186f, p.203f]. For an extensive treatment of Markovian short rate processes it is referred to either the paper of Carverhill [26] or the Chapter 16 of Rebonato [92, p.347-359]. In the general case when σ(t,T,ω) is not Markovian, we may encounter major computational problems when discretizing the Dynamics (7.11) for the pricing of even simple derivatives. However, several standard Markovian functional forms for σ(t,T,ω) have been explored. Two of them, both single factor models (d = 1), are listed and shortly explained below [60, p.203]:
- Constant σ: σ(t, T, ω) ≡ σ.
In this one factor model with constant volatility each increment dW(t) moves the entire forward curve, i.e. all existing points on the curve, by an equal amount of σdW(t). Thus, the forward curve is constrained to move only in parallel [46, p.153]. For this particularly simple specification, bond prices are assumed to be lognormal and all forward rates are normal and exhibit exactly the same volatility. If these distributional assumptions are fulfilled, the according models are qualified as Gaussian [92, p.317f]. They are analytically well-tractable but also unrealistic to some extent, what we will see later on. If we insert σ into Equation (7.7) the resulting drift of a HJM model with constant σ has the form [46, p.153]
􏰑 T
α(t, T, ω) = σ σ ds = σ2(T − t).
t
Solving (7.8) with constant σ leads to the dynamics of the forward rates f(t,T)=f(0,T)+ 12σ2[T2 −(T −t)2]+σW(t),
56

7 Heath, Jarrow and Morton
which is simultaneously the continuous-time equivalent of the Ho-Lee model [60, 92, p.202, p.318].
Flesaker [40] conducted an empirical test about the constant volatility version of the HJM model for Eurodollar futures and futures options. He found that this general version of the HJM model is unable to explain cross-sectional pricing pattern of futures options and that it exhibits systematic biases, why the model is eventually soundly rejected. Moreover, he observed that the model tends to overvalue short-term options relative to long-term options. Accordingly, he clearly motivated further empirical studies of HJM models where the volatility is allowed to vary over time rather than artificially kept constant.
-Exponentialσ: σ(t,T,ω)=σexp(−λ(T−t)),forsomeconstantsσ,λ>0.
This diffusion term σ(t,T,ω) has a greater impact on forward rates for short maturities than on forward rates for long maturities [46, p.154]. Due to this fact the HJM model in- cluding an exponential σ is eventually completely equal to the general short rate dynamics proposed by Hull and White [57], see (6.11). This makes clear that one can establish a one-to-one equivalence between the HJM one factor model and the general formulation of the Gaussian one factor instantaneous short rate model of Hull and White [20, p.187].
This volatility specification keeps its analytical tractability but is able to perform slightly better than the HJM model with constant σ. Yet, the results are still unsatisfactory and therefore the HJM model including exponential volatility is also rejected [60, p.203].
Asides from many standard approaches with the goal of finding suitable specifications of σ(t,T,ω), which make the short rate process indeed Markovian, there have been developed some more sophisticated techniques, too. They basically exploit that, even though a short rate process might not be Markovian, there may yet exist a higher-dimensional Markov process that possesses the instantaneous short rate as one of its components [20, p.188]. Ritchken and Sankarasubramanian [95], for instance, proposed a model for simply capturing the path dependence of the short rate r(t) through a single statistic. In particular, they found a few necessary and sufficient conditions which are imposed on the volatility structure of the forward rates. These conditions enable them to control the short rate’s path dependence and make them analytically tractable [20, 60, p.189, p.205].
A second model, which was proposed by Mercurio and Moraleda [78], derives an interest rate model within the HJM framework which explicitly assumes a humped volatility structure in the instantaneous forward rate dynamics and thus is also analytically tractable. They motivate their assumption of a humped volatility structure by the fact that forward rates themselves commonly exhibit humped volatility functions that are generally implied by simple market quotes. They have shown that their model consequently outperforms similar existing models, which do not explicitly contain such a humped shape in volatilities [20, 78, p.191f, p.213].
57

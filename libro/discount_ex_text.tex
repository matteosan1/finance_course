\begin{question}
\texttt{Python} has a useful command called \texttt{assert} which can be used for checking that a given condition is satisfied, and raising an error if the condition is not satisfied.

The following line does not cause an error, in fact it does nothing since 1 is lower than 2, hence the condition is met.

\lstinline[language=iPython]|assert 1 < 2|

\noindent
This causes an error (the condition is evaluated to false). 

\lstinline[language=iPython]|assert 1 > 2|

\noindent
\texttt{assert} can take a second argument with a message to display in case of failure.

\lstinline[language=iPython]|assert 1 > 2, "Two is greater than one"|

\noindent
Now takes the \texttt{df} function from the Chapter on Discounting and modify it by adding some assertions to check that:

\begin{itemize}
\item the pillar date list contains at least 2 elements;
\item the pillar date list has the same length as the discount factor one;
\item the first pillar date is equal to the today's date;
\item the value date (first argument \texttt{d}) is greater or equal to the first pillar date and also less than or equal to the last pillar date.
\end{itemize}

Then try using the function with some invalid data to make sure that your assertions are correctly checking the desired conditions
\end{question}

\cprotEnv\begin{solution}
\begin{ipython}
# import modules and objects that we need
from datetime import date
import numpy
import math

today_date = date(2017, 10, 1)
pillar_dates = [date(2017, 10, 1),
date(2018, 10, 1),
date(2019, 10, 1)]
discount_factors = [1.0, 0.95, 0.8]

def df(d, observation_date, pillar_dates, discount_factors):
    ############## CHECKS ################
    assert len(pillar_dates) >= 2, " need at least 2 pillar dates"
    assert len(pillar_dates) == len(discount_factors), \
        "number of pillar dates should be equal to \
         the number of pillar discount factors"
    assert observation_date == pillar_dates[0], \
        "first pillar date should be the observation date"
    assert pillar_dates[0] <= d <= pillar_dates[-1], \
        "Invalid value date %s" % (d)
    ############## END OF CHECKS ################
    log_discount_factors = []
    for discount_factor in discount_factors:
        log_discount_factors.append(math.log(discount_factor))
    pillar_days = []
    for pillar_date in pillar_dates:
        pillar_days.append((pillar_date - observation_date).days)
    d_days = (d - observation_date).days
    interpolated_log_discount_factor = \
        numpy.interp(d_days, pillar_days, log_discount_factors)
    return math.exp(interpolated_log_discount_factor)

df(date(2019, 1, 1), today_date, pillar_dates, discount_factors)
\end{ipython}
\begin{ioutput}
0.9097285910181567
\end{ioutput}
\end{solution}

\cprotEnv\begin{question}
Mario wants to lease a car for 3 years. The car dealer has given him two payment options to choose from:

\begin{table}[htbp]
\centering
\begin{tabular}{l|c|c|c|c|c|c|c}
Year & 0 & 0.5 & 1 & 1.5 & 2 & 2.5 & 3 \\
\hline
Option 1 &	3000.00	& 500.00 & 500.00 &	500.00 & 500.00 & 500.00 & 500.00 \\
\hline
Option 2 &	5000.00	& 350.00 & 350.00 &	350.00 & 350.00 & 350.00 & 350.00 \\
\end{tabular}
\end{table}

Which payment option should he choose? (Assume an annual discount rate of 10\%)
\end{question}

\cprotEnv\begin{solution}
In order to decide which payment to choose from, he would need to calculate the PV of the payment stream, and look at which costs him the least. 

\begin{gather*}
DF_n = 1/(1 + r)^n \\
PV_n = [\textrm{payment in year n}] / DF_n \\
Total PV = \sum PV_n
\end{gather*}

\begin{ipython}
def discount_factor(year):
    return 1/(1 + 0.1)**year

opt1 = {0:3000, 0.5:500, 1:500, 1.5:500, 2:500, 2.5:500, 3:500}
opt2 = {0:5000, 0.5:350, 1:350, 1.5:350, 2:350, 2.5:350, 3:350}

npv1 = sum([discount_factor(k)*v for k,v in opt1.items()])
npv2 = sum([discount_factor(k)*v for k,v in opt2.items()])

print ("Option1: {:.1f}".format(npv1))
print ("Option2: {:.1f}".format(npv2))
\end{ipython}
\begin{ioutput}
Option1: 5547.5
Option2: 6783.3
\end{ioutput}
Hence he should choose option 1 as it costs him less than option 2.
\end{solution}

\begin{question}
Tots and Toys, Inc. offers Maria an investment plan that requires 10 yearly payments of 1000 dollars, starting from today. The plan promises an annual return of 8\% on her investment. The money will be available to her at the end of 10 years (no withdrawls are allowed before that time.) 
What amount will she get at the end of 10 years?
\end{question}

\cprotEnv\begin{solution}
The question requires you to find the future value (FV) of the stream of payments. The rate given is 8\%.
In order to find the FV, you need to multiply each amount by its respective FV factor, and then sum the results.

\begin{ipython}
def fv_factor(year):
    return (1 + 0.08)**year

fv = 0
for year in range(11):
    fv += 1000*fv_factor(year)

print ("future value: {:.1f}".format(fv))
\end{ipython}
\begin{ioutput}
future value: 16645.5
\end{ioutput}
Amount at the end of 10 years is 16645.5 dollars.
\end{solution}

%\begin{question}
%Copy into the file \texttt{finmarkets.py} the function used to compute Black Scholes formula used in Ex.~\ref{ex:BS2}. This is another utility for our financial library. Then repeat Ex.~\ref{ex:BS2} now using the version of the Black and Scholes formula in the \texttt{finmarkets} module.
%\end{question}
%
%\begin{solution}
%\begin{tcolorbox}[size=fbox, boxrule=1pt, colback=cellbackground, colframe=cellborder]
%\begin{Verbatim}[commandchars=\\\{\}]
%\PY{k}{import} \PY{n}{finmarkets}
%        
%\PY{n}{s} \PY{o}{=} \PY{l+m+mi}{800}
%\PY{c+c1}{\PYZsh{} strikes expressed as \PYZpc{} of spot price}
%\PY{n}{moneyness} \PY{o}{=} \PY{p}{[} \PY{l+m+mf}{0.5}\PY{p}{,} \PY{l+m+mf}{0.75}\PY{p}{,} \PY{l+m+mf}{0.825}\PY{p}{,} \PYZbs{}
%             \PY{l+m+mf}{1.0}\PY{p}{,} \PY{l+m+mf}{1.125}\PY{p}{,} \PY{l+m+mf}{1.25}\PY{p}{,} \PY{l+m+mf}{1.5} \PY{p}{]}
%\PY{n}{vol} \PY{o}{=} \PY{l+m+mf}{0.3}
%\PY{n}{ttm} \PY{o}{=} \PY{l+m+mf}{0.75}
%\PY{n}{r} \PY{o}{=} \PY{l+m+mf}{0.005}
%
%\PY{n}{result} \PY{o}{=} \PY{p}{\PYZob{}}\PY{p}{\PYZcb{}}
%\PY{k}{for} \PY{n}{m} \PY{o+ow}{in} \PY{n}{moneyness}\PY{p}{:}
%    \PY{n}{result}\PY{p}{[}\PY{n}{s}\PY{o}{*}\PY{n}{m}\PY{p}{]} \PY{o}{=} \PY{n}{finmarkets.call}\PY{p}{(}\PY{n}{s}\PY{p}{,} \PY{n}{m}\PY{o}{*}\PY{n}{s}\PY{p}{,} \PY{n}{r}\PY{p}{,} \PY{n}{vol}\PY{p}{,} \PY{n}{ttm}\PY{p}{)}
%\PY{n}{result}
%
%\{400.0: 401.66074527896365,
%  600.0: 213.9883852521275,
%  660.0: 166.85957363897393,
%  800.0: 84.03697017660357,
%  900.0: 47.61880394696229,
%  1000.0: 25.632722952585738,
%  1200.0: 6.655275227771156\}
%\end{Verbatim}
%\end{tcolorbox}
%\end{solution}

%\begin{question}
%Following the steps outlined in Chapter Discount Factors, implement a \texttt{DiscountCurve} class and add it to \texttt{finmarkets} module. The class should have as attributes the pillar dates and the corresponding discount factors and two methods, one to interpolate discount factors and another to calculate forward rates.
%Finally using that class compute the forward 6M LIBOR coupon using the curves given below in pre and post 2008 crisis way.
%
%\textbf{Input:}
%\begin{Shaded}
%\begin{Highlighting}[]
%\NormalTok{observation_date }\OperatorTok{=}\NormalTok{ date (}\DecValTok{2020}\NormalTok{, }\DecValTok{1}\NormalTok{, }\DecValTok{1}\NormalTok{)}
%\NormalTok{t1 }\OperatorTok{=}\NormalTok{ date(}\DecValTok{2020}\NormalTok{,}\DecValTok{4}\NormalTok{, }\DecValTok{1}\NormalTok{)}
%\NormalTok{t2 }\OperatorTok{=}\NormalTok{ date(}\DecValTok{2020}\NormalTok{, }\DecValTok{10}\NormalTok{, }\DecValTok{1}\NormalTok{)}
%
%\CommentTok{# for EONIA}
%\NormalTok{pillar_dates_eonia }\OperatorTok{=}\NormalTok{ [date(}\DecValTok{2020}\NormalTok{ , }\DecValTok{1}\NormalTok{ ,}\DecValTok{1}\NormalTok{), }
%\NormalTok{                      date(}\DecValTok{2021}\NormalTok{, }\DecValTok{1}\NormalTok{, }\DecValTok{1}\NormalTok{), }
%\NormalTok{                      date(}\DecValTok{2022}\NormalTok{, }\DecValTok{10}\NormalTok{ ,}\DecValTok{1}\NormalTok{)]}
%\NormalTok{discount_factors_eonia }\OperatorTok{=}\NormalTok{ [}\FloatTok{1.0}\NormalTok{, }\FloatTok{0.97}\NormalTok{, }\FloatTok{0.72}\NormalTok{]}
%
%\CommentTok{# for LIBOR 6M}
%\NormalTok{pillar_dates_libor }\OperatorTok{=}\NormalTok{ [date(}\DecValTok{2020}\NormalTok{, }\DecValTok{1}\NormalTok{ ,}\DecValTok{1}\NormalTok{), }
%\NormalTok{                      date(}\DecValTok{2020}\NormalTok{, }\DecValTok{6}\NormalTok{, }\DecValTok{1}\NormalTok{), }
%\NormalTok{                      date(}\DecValTok{2020}\NormalTok{, }\DecValTok{12}\NormalTok{ ,}\DecValTok{1}\NormalTok{)]}
%\NormalTok{discount_factors_libor }\OperatorTok{=}\NormalTok{ [}\FloatTok{1.0}\NormalTok{, }\FloatTok{0.95}\NormalTok{, }\FloatTok{0.90}\NormalTok{]}
%\end{Highlighting}
%\end{Shaded}
%\end{question}
%
%\begin{solution}
%\begin{tcolorbox}[size=fbox, boxrule=1pt, pad at break*=1mm,colback=cellbackground, colframe=cellborder]
%\begin{Verbatim}[commandchars=\\\{\}]
%\PY{k+kn}{import} \PY{n+nn}{math}
%\PY{k+kn}{import} \PY{n+nn}{numpy}
%\PY{k+kn}{from} \PY{n+nn}{datetime} \PY{k}{import} \PY{n}{date}
%
%\PY{k}{class} \PY{n+nc}{DiscountCurve}\PY{p}{:}
%
%    \PY{k}{def} \PY{n+nf}{\PYZus{}\PYZus{}init\PYZus{}\PYZus{}}\PY{p}{(}\PY{n+nb+bp}{self}\PY{p}{,} \PY{n}{today}\PY{p}{,} \PY{n}{pillar\PYZus{}dates}\PY{p}{,} \PY{n}{discount\PYZus{}factors}\PY{p}{)}\PY{p}{:}
%        \PY{n+nb+bp}{self}\PY{o}{.}\PY{n}{today} \PY{o}{=} \PY{n}{today}
%        \PY{n+nb+bp}{self}\PY{o}{.}\PY{n}{pillar\PYZus{}dates} \PY{o}{=} \PY{n}{pillar\PYZus{}dates}
%        \PY{n+nb+bp}{self}\PY{o}{.}\PY{n}{discount\PYZus{}factors} \PY{o}{=} \PY{n}{discount\PYZus{}factors}
%
%    \PY{k}{def} \PY{n+nf}{df}\PY{p}{(}\PY{n+nb+bp}{self}\PY{p}{,} \PY{n}{d}\PY{p}{)}\PY{p}{:}
%        \PY{n}{log\PYZus{}discount\PYZus{}factors} \PY{o}{=} \PYZbs{}
%          \PY{p}{[}\PY{n}{math}\PY{o}{.}\PY{n}{log}\PY{p}{(}\PY{n}{discount\PYZus{}factor}\PY{p}{)} 
%           \PY{k}{for} \PY{n}{discount\PYZus{}factor} \PY{o+ow}{in} \PY{n+nb+bp}{self}\PY{o}{.}\PY{n}{discount\PYZus{}factors}\PY{p}{]}
%        \PY{n}{pillar\PYZus{}days} \PY{o}{=} \PY{p}{[}\PY{p}{(}\PY{n}{pillar\PYZus{}date} \PY{o}{\PYZhy{}} \PY{n+nb+bp}{self}\PY{o}{.}\PY{n}{today}\PY{p}{)}\PY{o}{.}\PY{n}{days} 
%                       \PY{k}{for} \PY{n}{pillar\PYZus{}date} \PY{o+ow}{in} \PY{n+nb+bp}{self}\PY{o}{.}\PY{n}{pillar\PYZus{}dates}\PY{p}{]}
%        \PY{n}{d\PYZus{}days} \PY{o}{=} \PY{p}{(}\PY{n}{d} \PY{o}{\PYZhy{}} \PY{n+nb+bp}{self}\PY{o}{.}\PY{n}{today}\PY{p}{)}\PY{o}{.}\PY{n}{days}
%        \PY{n}{interpolated\PYZus{}log\PYZus{}discount\PYZus{}factor} \PY{o}{=} \PYZbs{}
%            \PY{n}{numpy}\PY{o}{.}\PY{n}{interp}\PY{p}{(}\PY{n}{d\PYZus{}days}\PY{p}{,} \PY{n}{pillar\PYZus{}days}\PY{p}{,} \PY{n}{log\PYZus{}discount\PYZus{}factors}\PY{p}{)}
%        \PY{k}{return} \PY{n}{math}\PY{o}{.}\PY{n}{exp}\PY{p}{(}\PY{n}{interpolated\PYZus{}log\PYZus{}discount\PYZus{}factor}\PY{p}{)}
%
%    \PY{k}{def} \PY{n+nf}{forward\PYZus{}rate}\PY{p}{(}\PY{n+nb+bp}{self}\PY{p}{,} \PY{n}{d1}\PY{p}{,} \PY{n}{d2}\PY{p}{)}\PY{p}{:}
%        \PY{k}{return} \PY{p}{(}\PY{n+nb+bp}{self}\PY{o}{.}\PY{n}{df}\PY{p}{(}\PY{n}{d1}\PY{p}{)} \PY{o}{/} \PY{n+nb+bp}{self}\PY{o}{.}\PY{n}{df}\PY{p}{(}\PY{n}{d2}\PY{p}{)} \PY{o}{\PYZhy{}} \PY{l+m+mf}{1.0}\PY{p}{)} \PY{o}{*} \PYZbs{}
%                \PY{p}{(}\PY{l+m+mf}{365.0} \PY{o}{/} \PY{p}{(}\PY{p}{(}\PY{n}{d2} \PY{o}{\PYZhy{}} \PY{n}{d1}\PY{p}{)}\PY{o}{.}\PY{n}{days}\PY{p}{)}\PY{p}{)}
%\end{Verbatim}
%\end{tcolorbox}
%
%\begin{tcolorbox}[breakable, size=fbox, boxrule=1pt, pad at break*=1mm,colback=cellbackground, colframe=cellborder]
%\begin{Verbatim}[commandchars=\\\{\}]
%\PY{k+kn}{from} \PY{n+nn}{finmarkets} \PY{k}{import} \PY{n}{DiscountCurve}
%
%\PY{n}{observation\PYZus{}date} \PY{o}{=} \PY{n}{date} \PY{p}{(}\PY{l+m+mi}{2020}\PY{p}{,} \PY{l+m+mi}{1}\PY{p}{,} \PY{l+m+mi}{1}\PY{p}{)}
%\PY{n}{t1} \PY{o}{=} \PY{n}{date}\PY{p}{(}\PY{l+m+mi}{2020}\PY{p}{,}\PY{l+m+mi}{4}\PY{p}{,} \PY{l+m+mi}{1}\PY{p}{)}
%\PY{n}{t2} \PY{o}{=} \PY{n}{date}\PY{p}{(}\PY{l+m+mi}{2020}\PY{p}{,} \PY{l+m+mi}{10}\PY{p}{,} \PY{l+m+mi}{1}\PY{p}{)}
%
%\PY{c+c1}{\PYZsh{} for EONIA}
%\PY{n}{pillar\PYZus{}dates\PYZus{}eonia} \PY{o}{=} \PY{p}{[}\PY{n}{date}\PY{p}{(}\PY{l+m+mi}{2020} \PY{p}{,} \PY{l+m+mi}{1} \PY{p}{,}\PY{l+m+mi}{1}\PY{p}{)}\PY{p}{,} 
%                      \PY{n}{date}\PY{p}{(}\PY{l+m+mi}{2021}\PY{p}{,} \PY{l+m+mi}{1}\PY{p}{,} \PY{l+m+mi}{1}\PY{p}{)}\PY{p}{,} 
%                      \PY{n}{date}\PY{p}{(}\PY{l+m+mi}{2022}\PY{p}{,} \PY{l+m+mi}{10} \PY{p}{,}\PY{l+m+mi}{1}\PY{p}{)}\PY{p}{]}
%\PY{n}{discount\PYZus{}factors\PYZus{}eonia} \PY{o}{=} \PY{p}{[}\PY{l+m+mf}{1.0}\PY{p}{,} \PY{l+m+mf}{0.97}\PY{p}{,} \PY{l+m+mf}{0.72}\PY{p}{]}
%
%\PY{c+c1}{\PYZsh{} for LIBOR 6M}
%\PY{n}{pillar\PYZus{}dates\PYZus{}libor} \PY{o}{=} \PY{p}{[}\PY{n}{date}\PY{p}{(}\PY{l+m+mi}{2020}\PY{p}{,} \PY{l+m+mi}{1} \PY{p}{,}\PY{l+m+mi}{1}\PY{p}{)}\PY{p}{,} 
%                      \PY{n}{date}\PY{p}{(}\PY{l+m+mi}{2020}\PY{p}{,} \PY{l+m+mi}{6}\PY{p}{,} \PY{l+m+mi}{1}\PY{p}{)}\PY{p}{,} 
%                      \PY{n}{date}\PY{p}{(}\PY{l+m+mi}{2020}\PY{p}{,} \PY{l+m+mi}{12} \PY{p}{,}\PY{l+m+mi}{1}\PY{p}{)}\PY{p}{]}
%\PY{n}{discount\PYZus{}factors\PYZus{}libor} \PY{o}{=} \PY{p}{[}\PY{l+m+mf}{1.0}\PY{p}{,} \PY{l+m+mf}{0.95}\PY{p}{,} \PY{l+m+mf}{0.90}\PY{p}{]}
%
%
%\PY{n}{eonia\PYZus{}curve} \PY{o}{=} \PY{n}{DiscountCurve}\PY{p}{(}\PY{n}{observation\PYZus{}date}\PY{p}{,} 
%                            \PY{n}{pillar\PYZus{}dates\PYZus{}eonia}\PY{p}{,} 
%                            \PY{n}{discount\PYZus{}factors\PYZus{}eonia}\PY{p}{)}
%\PY{n}{libor\PYZus{}curve} \PY{o}{=} \PY{n}{DiscountCurve}\PY{p}{(}\PY{n}{observation\PYZus{}date}\PY{p}{,} 
%                            \PY{n}{pillar\PYZus{}dates\PYZus{}libor}\PY{p}{,} 
%                            \PY{n}{discount\PYZus{}factors\PYZus{}libor}\PY{p}{)}
%
%
%\PY{n}{npv} \PY{o}{=} \PY{n}{eonia\PYZus{}curve}\PY{o}{.}\PY{n}{df}\PY{p}{(}\PY{n}{t1}\PY{p}{)} \PY{o}{*} \PY{n}{libor\PYZus{}curve}\PY{o}{.}\PY{n}{forward\PYZus{}rate}\PY{p}{(}\PY{n}{t1}\PY{p}{,} \PY{n}{t2}\PY{p}{)}
%
%\PY{c+c1}{\PYZsh{} Compute it in the pre\PYZhy{}2008 way}
%\PY{n}{npv\PYZus{}pre2008} \PY{o}{=} \PY{n}{libor\PYZus{}curve}\PY{o}{.}\PY{n}{df}\PY{p}{(}\PY{n}{t1}\PY{p}{)} \PY{o}{*} \PY{n}{libor\PYZus{}curve}\PY{o}{.}\PY{n}{forward\PYZus{}rate}\PY{p}{(}\PY{n}{t1}\PY{p}{,} \PY{n}{t2}\PY{p}{)}
%
%\PY{n+nb}{print} \PY{p}{(}\PY{l+s+s2}{\PYZdq{}}\PY{l+s+s2}{NPV post 2008:}\PY{l+s+s2}{\PYZdq{}}\PY{p}{,} \PY{n}{npv}\PY{p}{)}
%\PY{n+nb}{print} \PY{p}{(}\PY{l+s+s2}{\PYZdq{}}\PY{l+s+s2}{NPV pre 2008:}\PY{l+s+s2}{\PYZdq{}}\PY{p}{,} \PY{n}{npv\PYZus{}pre2008}\PY{p}{)}
%
%NPV post 2008: 0.11533243116069992
%NPV pre 2008: 0.11269481011359303
%\end{Verbatim}
%\end{tcolorbox}
%\end{solution}
%
%\begin{question}
%Write a ForwardRateCurve class (for EURIBOR/LIBOR rate curve) which
%doesn't compute discount factors but only interplatates forward rates;
%then add it to the \texttt{finmarkets} module (this function is used to
%define the LIBOR curve needed throughout future lessons).
%\end{question}
%
%\begin{solution}
%In this case it is enough to write a new \texttt{class} that has three
%attributes: a today date, a set of pillar\_dates and the corresponding
%rates. There will be just a single method \texttt{forward\_rate} which
%returns the corresponding interpolated rate.
%
%\begin{tcolorbox}[size=fbox, boxrule=1pt, colback=cellbackground, colframe=cellborder]
%\begin{Verbatim}[commandchars=\\\{\}]
%\PY{k+kn}{import} \PY{n+nn}{numpy}
%        
%\PY{c+c1}{\PYZsh{} an EURIBOR or LIBOR rate curve}
%\PY{c+c1}{\PYZsh{} doesn\PYZsq{}t calculate discount factors, only interpolates forward rates}
%\PY{k}{class} \PY{n+nc}{ForwardRateCurve}\PY{p}{(}\PY{n+nb}{object}\PY{p}{)}\PY{p}{:}
%   
%   \PY{c+c1}{\PYZsh{} the special \PYZus{}\PYZus{}init\PYZus{}\PYZus{} method defines how to}
%   \PY{c+c1}{\PYZsh{} construct instances of the class}
%   \PY{k}{def} \PY{n+nf}{\PYZus{}\PYZus{}init\PYZus{}\PYZus{}}\PY{p}{(}\PY{n+nb+bp}{self}\PY{p}{,} \PY{n}{pillar\PYZus{}dates}\PY{p}{,} \PY{n}{rates}\PY{p}{)}\PY{p}{:}
%       
%       \PY{c+c1}{\PYZsh{} we just store the arguments as attributes of the instance}
%       \PY{n+nb+bp}{self}\PY{o}{.}\PY{n}{today} \PY{o}{=} \PY{n}{pillar\PYZus{}dates}\PY{p}{[}\PY{l+m+mi}{0}\PY{p}{]}
%       \PY{n+nb+bp}{self}\PY{o}{.}\PY{n}{rates} \PY{o}{=} \PY{n}{rates}
%       
%       \PY{n+nb+bp}{self}\PY{o}{.}\PY{n}{pillar\PYZus{}days} \PY{o}{=} \PY{p}{[}
%           \PY{p}{(}\PY{n}{pillar\PYZus{}date} \PY{o}{\PYZhy{}} \PY{n+nb+bp}{self}\PY{o}{.}\PY{n}{today}\PY{p}{)}\PY{o}{.}\PY{n}{days}
%           \PY{k}{for} \PY{n}{pillar\PYZus{}date} \PY{o+ow}{in} \PY{n}{pillar\PYZus{}dates}
%       \PY{p}{]}
%       
%       
%   \PY{c+c1}{\PYZsh{} interpolates the forward rates stored in the instance}
%   \PY{k}{def} \PY{n+nf}{forward\PYZus{}rate}\PY{p}{(}\PY{n+nb+bp}{self}\PY{p}{,} \PY{n}{d}\PY{p}{)}\PY{p}{:}
%       \PY{n}{d\PYZus{}days} \PY{o}{=} \PY{p}{(}\PY{n}{d} \PY{o}{\PYZhy{}} \PY{n+nb+bp}{self}\PY{o}{.}\PY{n}{today}\PY{p}{)}\PY{o}{.}\PY{n}{days}
%       \PY{k}{return} \PY{n}{numpy}\PY{o}{.}\PY{n}{interp}\PY{p}{(}\PY{n}{d\PYZus{}days}\PY{p}{,} \PY{n+nb+bp}{self}\PY{o}{.}\PY{n}{pillar\PYZus{}days}\PY{p}{,} \PY{n+nb+bp}{self}\PY{o}{.}\PY{n}{rates}\PY{p}{)}
%\end{Verbatim}
%\end{tcolorbox}
%\end{solution}


  

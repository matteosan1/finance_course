\chapter{Modeling Correlation between Risks}

In credit derivative valuation and credit risk management, one of the
fundamentally important issues is the estimate of default
probabilities and their correlations. For this, generally speaking,
there are two ways: using historical default data or using mathematical
models.

Historical default data has played an important role in the estimation
of default probabilities. However, because default events are rare,
there is very limited default data available. Moreover, historical data
reflects the historical default pattern only and it may not be a proper
indicator of the future. This makes the estimate of default
probabilities from historical data difficult and inexact. To use this
same data to estimate default correlations is even more difficult and
more inexact.

The market trend now is towards more and more to the use of mathematical
models that don't rely on historical default data. In
Chapter~\ref{credit_default_swaps} we have seen how it is possible to derive default probabilities from market data, here we will see how the copula can be used to model their correlations. 

\section{Credit Derivatives}\label{credit-derivatives}
Here we are going also to introduce two kind of credit derivatives.

\subsection{Basket Default Swaps}\label{basket-default-swaps}

A basket default swap is a credit derivative on a portfolio of reference
entities. The simplest basket default swaps are first-to-default,
second-to-default, or nth-to-default swaps. 

This kind of contracts are very similar to normal CDS except for the protection they offer.
With respect to a basket of reference entities, a first-to-default swap provides insurance for only the first default, a second-to-default swap provides insurance
for only the second default, and a nth-to-default swap provides insurance
for only the nth default. 

For example, in the last case, the
seller does not make a payment to the protection buyer for
the first n-1 defaulted reference entities, and makes a payment only for the
nth defaulted reference entity. Once there has been this payment the swap terminates.

\subsection{Collateralized Debt Obligation}\label{collateralized-debt-obligation}








 Because the equity tranche is
extremely risky, the sponsor of a CDO holds the equity tranche and the
SPV sells other tranches to investors.



A collateralized debt obligation (CDO) is a type of structured asset-backed security (ABS). 
A CDO can be thought of as a promise to pay investors in a prescribed sequence, based on the cash flow the CDO collects from the pool of bonds or other assets it owns. Distinctively, CDO credit risk is typically assessed based on a probability of default (PD) derived from ratings on those bonds or assets.

The CDO is "sliced" into "tranches", which "catch" the cash flow of interest and principal payments in sequence based on seniority.
In the literature, tranches of a CDO are classified as
subordinate/equity tranche, mezzanine tranches, and senior tranches
according to their levels.

If some loans default and the cash collected by the CDO is insufficient to pay all of its investors, those in the lowest, most "junior" tranches suffer losses first. The last to lose payment from default are the safest, most senior tranches. Consequently, coupon payments (and interest rates) vary by tranche with the safest/most senior tranches receiving the lowest rates and the lowest tranches receiving the highest rates to compensate for higher default risk.

Each tranche has an attachment percentage and a detachment
percentage. When the cumulative percentage loss of the portfolio reaches
the attachment percentage, investors in the tranche start to lose their
principal, and when the cumulative percentage loss of principal reaches
the detachment percentage, the investors in the tranche lose all their
principal and no further loss can occur to them.

A common analogy compares the cash flow from the CDO's portfolio of securities (say mortgage payments from mortgage-backed bonds) to water flowing into cups of the investors where senior tranches were filled first and overflowing cash flowed to junior tranches, then equity tranches. If a large portion of the mortgages enter default, there is insufficient cash flow to fill all these cups and equity tranche investors face the losses first.

A CDO can be initiated by one or more of the following entities, which are referred to as the
\emph{sponsor}: banks, non-bank financial institutions, and asset management companies. 

The sponsor of a CDO creates a company so-called the \emph{special purpose vehicle} (SPV).
It is the SPV, rather than the parent investment bank, that issue the CDOs and pay interest to investors. As CDOs developed, some sponsors repackaged tranches into yet another iteration, known as "CDO-Squared", "CDOs of CDOs" or "synthetic CDOs".

In the early 2000s, the debt underpinning CDOs was generally diversified, but by 2006–2007, when the CDO market grew to hundreds of billions of dollars, this had changed. CDO collateral became dominated by high risk tranches recycled from other asset-backed securities, whose assets were usually subprime mortgages. These CDOs have been called \emph{the engine that powered the mortgage supply chain} for subprime mortgages, and are credited with giving lenders greater incentive to make subprime loans, leading to the 2007-2009 subprime mortgage crisis

\section{Correlated Defaults}\label{correlated-defaults}
The cost of protection in nth-to-default CDS or a tranche of CDOs is critically dependent
on default correlation. 
Suppose that a basket of 100 reference entities is used to define a 5-year
nth-to-default CDS and that each reference entity has a probability of 2\% of defaulting during the next
5 years. When the default correlation between the reference entities is zero the binomial distribution (see Appendix~\ref{binomial-distribution}) 
shows that the probability of one or more defaults during 5 years is 86.74\% and the probability of ten or more defaults is 0.0034\%.

\begin{tcolorbox}[breakable, size=fbox, boxrule=1pt, pad at break*=1mm,colback=cellbackground, colframe=cellborder]
\begin{Verbatim}[commandchars=\\\{\}]
\PY{k+kn}{from} \PY{n+nn}{scipy}\PY{n+nn}{.}\PY{n+nn}{stats} \PY{k}{import} \PY{n}{binom}
	
\PY{n}{b} \PY{o}{=} \PY{n}{binom}\PY{p}{(}\PY{l+m+mi}{100}\PY{p}{,} \PY{l+m+mf}{0.02}\PY{p}{)}
\PY{n+nb}{print}\PY{p}{(}\PY{l+s+s2}{\PYZdq{}}\PY{l+s+s2}{P(\PYZgt{}=1) : }\PY{l+s+si}{\PYZob{}\PYZcb{}}\PY{l+s+s2}{\PYZdq{}}\PY{o}{.}\PY{n}{format}\PY{p}{(}\PY{l+m+mi}{1} \PY{o}{\PYZhy{}} \PY{n}{b}\PY{o}{.}\PY{n}{cdf}\PY{p}{(}\PY{l+m+mi}{0}\PY{p}{)}\PY{p}{)}\PY{p}{)}
\PY{n+nb}{print}\PY{p}{(}\PY{l+s+s2}{\PYZdq{}}\PY{l+s+s2}{P(\PYZgt{}=10): }\PY{l+s+si}{\PYZob{}\PYZcb{}}\PY{l+s+s2}{\PYZdq{}}\PY{o}{.}\PY{n}{format}\PY{p}{(}\PY{l+m+mi}{1} \PY{o}{\PYZhy{}} \PY{n}{b}\PY{o}{.}\PY{n}{cdf}\PY{p}{(}\PY{l+m+mi}{9}\PY{p}{)}\PY{p}{)}\PY{p}{)}
	
P(>=1) : 0.8673804441052471
P(>=10): 3.441680604299169e-05
\end{Verbatim}
\end{tcolorbox}

A first-to-default is therefore quite valuable whereas a tenth-to-default CDS is worth almost nothing.

As the default correlation increases the probability of one or more defaults declines and the probability of ten or more defaults increases. In the limit where the default correlation is perfect the probability of one or more defaults equals the probability of ten or more defaults and it is 2\%. This is because in this extreme situation the reference entities are essentially the same : either they all default (with 2\% probability) or none of them default (with 98\% probability).

\subsection{Calculating nth-to-default probability}

\subsubsection{Independent Defaults}\label{independent-defaults}

If the default times of the names of a basket are independent,
first-to-default, nth-to-default, all-to-default probabilities can be
calculated through multiplication and integration of the default
probability curves of the basket components.

As an example, consider the second-to-default probability of a 4-name
basket. Let \(\tau_i\) be the default time of name \(i\) and \(F_i(t)\)
its distribution. Then the probability that name 1 defaults second in
the basket before time \(t\) is:

\begin{equation}
\begin{split}
&\mathbb{P}((\tau_2\lt\tau_1)\cap (\tau_1\lt t)\cap (\tau_1\lt\tau_3)\cap (\tau_1\lt\tau_4)) +\\
&\mathbb{P}((\tau_3\lt\tau_1)\cap (\tau_1\lt t)\cap (\tau_1\lt\tau_2)\cap (\tau_1\lt\tau_4)) =\\
&\int_0^t{F_2 (s)\cdot (1-F_3 (s)) \cdot (1-F_4 (s))~dF_1(s)} +  \int_0^t{F_3 (s)\cdot (1-F_2 (s)) \cdot (1-F_4 (s))~dF_1(s)}
\end{split}
\label{eq:indep_default}
\end{equation}

Suppose the default probabilities of three companies, $A$, $B$ and $C$ are
given as in the following table (in each interval are linear):

\begin{center}
	\begin{tabular}{|c|c|c|c|}
		time in years & A & B & C \\
		\hline
		0 & 0 & 0 & 0 \\
		1 & 0.022032 & 0.0317 & 0.035 \\
		2 & 0.046242 & 0.0655 & 0.075 \\
		3 & 0.07266 & 0.1022 & 0.121 \\
		4 & 0.101233 & 0.142 & 0.153 \\
		5 & 0.131885 & 0.1752 & 0.205 \\
	\end{tabular}
\end{center}
and suppose that the default events of the three companies are
independent. The integral in Eq.~\ref{eq:indep_default} can be solved by substitution:

\[ \int_{x_0}^{x_1}{(1-F_B(x))(1-F_C(x))dF_A(x)}\]

Setting \(t=m_A x + q_A\) it becomes:

\[ \int_{m_A x_0 + q_A}^{m_A x_1 + q_A}{(1-F_B(x(t)))(1-F_C(x(t)))dt}~~~~~~\Big(\textrm{with}~x(t) = \cfrac{t -q_A}{m_A}\Big) \]
and similarly for company $B$ and $C$.

To convert it into \texttt{python} we can use \texttt{scipy.integrate.quad} to
perform the integral and \texttt{numpy.interp} to determine the
intermediate default probabilities.

\begin{tcolorbox}[breakable, size=fbox, boxrule=1pt, pad at break*=1mm,colback=cellbackground, colframe=cellborder]
\begin{Verbatim}[commandchars=\\\{\}]
\PY{k+kn}{from} \PY{n+nn}{scipy}\PY{n+nn}{.}\PY{n+nn}{integrate} \PY{k}{import} \PY{n}{quad}
\PY{k+kn}{from} \PY{n+nn}{numpy} \PY{k}{import} \PY{n}{interp}
	
\PY{n}{default\PYZus{}rates} \PY{o}{=} \PY{p}{\PYZob{}}\PY{l+s+s2}{\PYZdq{}}\PY{l+s+s2}{A}\PY{l+s+s2}{\PYZdq{}}\PY{p}{:}\PY{p}{(}\PY{l+m+mi}{0}\PY{p}{,} \PY{l+m+mf}{0.022032}\PY{p}{,} \PY{l+m+mf}{0.046242}\PY{p}{,} \PY{l+m+mf}{0.07266}\PY{p}{,} \PY{l+m+mf}{0.101233}\PY{p}{,} \PY{l+m+mf}{0.131885}\PY{p}{)}\PY{p}{,}
                \PY{l+s+s2}{\PYZdq{}}\PY{l+s+s2}{B}\PY{l+s+s2}{\PYZdq{}}\PY{p}{:}\PY{p}{(}\PY{l+m+mi}{0}\PY{p}{,} \PY{l+m+mf}{0.0317}\PY{p}{,} \PY{l+m+mf}{0.0655}\PY{p}{,} \PY{l+m+mf}{0.1022}\PY{p}{,} \PY{l+m+mf}{0.142}\PY{p}{,} \PY{l+m+mf}{0.1752}\PY{p}{)}\PY{p}{,}
                \PY{l+s+s2}{\PYZdq{}}\PY{l+s+s2}{C}\PY{l+s+s2}{\PYZdq{}}\PY{p}{:}\PY{p}{(}\PY{l+m+mi}{0}\PY{p}{,} \PY{l+m+mf}{0.035}\PY{p}{,} \PY{l+m+mf}{0.075}\PY{p}{,} \PY{l+m+mf}{0.121}\PY{p}{,} \PY{l+m+mf}{0.153}\PY{p}{,} \PY{l+m+mf}{0.205}\PY{p}{)}\PY{p}{\PYZcb{}}
	
\PY{k}{def} \PY{n+nf}{func}\PY{p}{(}\PY{n}{x}\PY{p}{,} \PY{n}{default}\PY{p}{,} \PY{n}{companies}\PY{p}{,} \PY{n}{t}\PY{p}{)}\PY{p}{:}
    \PY{n}{m} \PY{o}{=} \PY{n}{default}\PY{p}{[}\PY{n}{companies}\PY{p}{[}\PY{l+m+mi}{0}\PY{p}{]}\PY{p}{]}\PY{p}{[}\PY{n}{t}\PY{p}{]} \PY{o}{\PYZhy{}} \PY{n}{default}\PY{p}{[}\PY{n}{companies}\PY{p}{[}\PY{l+m+mi}{0}\PY{p}{]}\PY{p}{]}\PY{p}{[}\PY{n}{t}\PY{o}{\PYZhy{}}\PY{l+m+mi}{1}\PY{p}{]}
    \PY{n}{q} \PY{o}{=} \PY{n}{default}\PY{p}{[}\PY{n}{companies}\PY{p}{[}\PY{l+m+mi}{0}\PY{p}{]}\PY{p}{]}\PY{p}{[}\PY{n}{t}\PY{o}{\PYZhy{}}\PY{l+m+mi}{1}\PY{p}{]} \PY{o}{\PYZhy{}} \PY{n}{m} \PY{o}{*} \PY{p}{(}\PY{n}{t}\PY{o}{\PYZhy{}}\PY{l+m+mi}{1}\PY{p}{)}
    \PY{n}{t} \PY{o}{=} \PY{p}{(}\PY{n}{x}\PY{o}{\PYZhy{}}\PY{n}{q}\PY{p}{)}\PY{o}{/}\PY{n}{m}
    \PY{n}{F2} \PY{o}{=} \PY{l+m+mi}{1} \PY{o}{\PYZhy{}} \PY{n}{interp}\PY{p}{(}\PY{n}{t}\PY{p}{,} \PY{n+nb}{range}\PY{p}{(}\PY{n+nb}{len}\PY{p}{(}\PY{n}{default}\PY{p}{[}\PY{n}{companies}\PY{p}{[}\PY{l+m+mi}{1}\PY{p}{]}\PY{p}{]}\PY{p}{)}\PY{p}{)}\PY{p}{,} \PY{n}{default}\PY{p}{[}\PY{n}{companies}\PY{p}{[}\PY{l+m+mi}{1}\PY{p}{]}\PY{p}{]}\PY{p}{)}
    \PY{n}{F3} \PY{o}{=} \PY{l+m+mi}{1} \PY{o}{\PYZhy{}} \PY{n}{interp}\PY{p}{(}\PY{n}{t}\PY{p}{,} \PY{n+nb}{range}\PY{p}{(}\PY{n+nb}{len}\PY{p}{(}\PY{n}{default}\PY{p}{[}\PY{n}{companies}\PY{p}{[}\PY{l+m+mi}{2}\PY{p}{]}\PY{p}{]}\PY{p}{)}\PY{p}{)}\PY{p}{,} \PY{n}{default}\PY{p}{[}\PY{n}{companies}\PY{p}{[}\PY{l+m+mi}{2}\PY{p}{]}\PY{p}{]}\PY{p}{)}
    \PY{k}{return} \PY{n}{F2}\PY{o}{*}\PY{n}{F3}
	
\PY{k}{def} \PY{n+nf}{integral}\PY{p}{(}\PY{n}{default}\PY{p}{,} \PY{n}{companies}\PY{p}{,} \PY{n}{t}\PY{p}{)}\PY{p}{:}
    \PY{k}{return} \PY{n}{quad}\PY{p}{(}\PY{n}{func}\PY{p}{,} \PY{l+m+mi}{0}\PY{p}{,} \PY{n}{default}\PY{p}{[}\PY{n}{companies}\PY{p}{[}\PY{l+m+mi}{0}\PY{p}{]}\PY{p}{]}\PY{p}{[}\PY{n}{t}\PY{p}{]}\PY{p}{,} 

\PY{n}{args}\PY{o}{=}\PY{p}{(}\PY{n}{default}\PY{p}{,} \PY{n}{companies}\PY{p}{,} \PY{n}{t}\PY{p}{)}\PY{p}{)}\PY{p}{[}\PY{l+m+mi}{0}\PY{p}{]}
\PY{k}{for} \PY{n}{companies} \PY{o+ow}{in} \PY{p}{[}\PY{p}{(}\PY{l+s+s2}{\PYZdq{}}\PY{l+s+s2}{A}\PY{l+s+s2}{\PYZdq{}}\PY{p}{,} \PY{l+s+s2}{\PYZdq{}}\PY{l+s+s2}{B}\PY{l+s+s2}{\PYZdq{}}\PY{p}{,} \PY{l+s+s2}{\PYZdq{}}\PY{l+s+s2}{C}\PY{l+s+s2}{\PYZdq{}}\PY{p}{)}\PY{p}{,} \PY{p}{(}\PY{l+s+s2}{\PYZdq{}}\PY{l+s+s2}{B}\PY{l+s+s2}{\PYZdq{}}\PY{p}{,} \PY{l+s+s2}{\PYZdq{}}\PY{l+s+s2}{A}\PY{l+s+s2}{\PYZdq{}}\PY{p}{,} \PY{l+s+s2}{\PYZdq{}}\PY{l+s+s2}{C}\PY{l+s+s2}{\PYZdq{}}\PY{p}{)}\PY{p}{,} \PY{p}{(}\PY{l+s+s2}{\PYZdq{}}\PY{l+s+s2}{C}\PY{l+s+s2}{\PYZdq{}}\PY{p}{,} \PY{l+s+s2}{\PYZdq{}}\PY{l+s+s2}{A}\PY{l+s+s2}{\PYZdq{}}\PY{p}{,} \PY{l+s+s2}{\PYZdq{}}\PY{l+s+s2}{B}\PY{l+s+s2}{\PYZdq{}}\PY{p}{)}\PY{p}{]}\PY{p}{:}
    \PY{n}{prob} \PY{o}{=} \PY{l+m+mi}{0}
    \PY{k}{for} \PY{n}{t} \PY{o+ow}{in} \PY{n+nb}{range}\PY{p}{(}\PY{l+m+mi}{1}\PY{p}{,} \PY{l+m+mi}{6}\PY{p}{)}\PY{p}{:}
    \PY{n}{prob} \PY{o}{=} \PY{n}{integral}\PY{p}{(}\PY{n}{default\PYZus{}rates}\PY{p}{,} \PY{n}{companies}\PY{p}{,} \PY{n}{t}\PY{p}{)}
    \PY{n+nb}{print} \PY{p}{(}\PY{l+s+s2}{\PYZdq{}}\PY{l+s+s2}{P(1st def) at time (}\PY{l+s+si}{\PYZob{}\PYZcb{}}\PY{l+s+s2}{) for company }\PY{l+s+si}{\PYZob{}\PYZcb{}}\PY{l+s+s2}{: }\PY{l+s+si}{\PYZob{}:.5f\PYZcb{}}\PY{l+s+s2}{\PYZdq{}}\PY{o}{.}\PY{n}{format}\PY{p}{(}\PY{n}{t}\PY{p}{,} 
    \PY{n}{companies}\PY{p}{[}\PY{l+m+mi}{0}\PY{p}{]}\PY{p}{,} \PY{n}{prob}\PY{p}{)}\PY{p}{)}

First to default prob at time (1) for company A: 0.02131
First to default prob at time (2) for company A: 0.04301
First to default prob at time (3) for company A: 0.06460
First to default prob at time (4) for company A: 0.08573
First to default prob at time (5) for company A: 0.10606
First to default prob at time (1) for company B: 0.03080
First to default prob at time (2) for company B: 0.06160
...
\end{Verbatim}
\end{tcolorbox}

\subsubsection{Correlated Defaults}\label{correlated-defaults}
When the default probabilities of the companies are correlated the copula approach can be used like in the example shown in Section~\ref{generate-correlated-distributions}.

Suppose to have to estimate the default probabilities for the next 5 years for 6 companies. 
The copula default correlation between each company is 0.2 and the cumulative probability of default during the next 1,2,3,4 5 years is 1\%, 3\%, 6\%, 10\%, 13\% respectively for each company.

When a Gaussian copula is used in order to simulate the defaults we need to sample from a multivariate normal distribution a vector $\mathbf{x}$, transform then each $x_i$ into the corresponding default probability $p_i$.

Let's check the 3th-to-default probabilities for each year.
\begin{tcolorbox}[breakable, size=fbox, boxrule=1pt, pad at break*=1mm,colback=cellbackground, colframe=cellborder]
\begin{Verbatim}[commandchars=\\\{\}]
\PY{k+kn}{from} \PY{n+nn}{scipy}\PY{n+nn}{.}\PY{n+nn}{stats} \PY{k}{import} \PY{n}{multivariate\PYZus{}normal}
	
\PY{n}{p\PYZus{}default} \PY{o}{=} \PY{p}{[}\PY{l+m+mi}{0}\PY{p}{,} \PY{l+m+mf}{0.01}\PY{p}{,} \PY{l+m+mf}{0.03}\PY{p}{,} \PY{l+m+mf}{0.06}\PY{p}{,} \PY{l+m+mf}{0.10}\PY{p}{,} \PY{l+m+mf}{0.13}\PY{p}{]}
	
\PY{n}{mvnorm} \PY{o}{=} \PY{n}{multivariate\PYZus{}normal}\PY{p}{(}\PY{n}{mean}\PY{o}{=}\PY{p}{[}\PY{l+m+mi}{0}\PY{p}{]}\PY{o}{*}\PY{l+m+mi}{6}\PY{p}{,}
                             \PY{n}{cov} \PY{o}{=} \PY{p}{[}\PY{p}{[}\PY{l+m+mi}{1}\PY{p}{,} \PY{l+m+mf}{0.2}\PY{p}{,} \PY{l+m+mf}{0.2}\PY{p}{,} \PY{l+m+mf}{0.2}\PY{p}{,} \PY{l+m+mf}{0.2}\PY{p}{,} \PY{l+m+mf}{0.2}\PY{p}{]}\PY{p}{,}
                                    \PY{p}{[}\PY{l+m+mf}{0.2}\PY{p}{,} \PY{l+m+mi}{1}\PY{p}{,} \PY{l+m+mf}{0.2}\PY{p}{,} \PY{l+m+mf}{0.2}\PY{p}{,} \PY{l+m+mf}{0.2}\PY{p}{,} \PY{l+m+mf}{0.2}\PY{p}{]}\PY{p}{,}
                                    \PY{p}{[}\PY{l+m+mf}{0.2}\PY{p}{,} \PY{l+m+mf}{0.2}\PY{p}{,} \PY{l+m+mi}{1}\PY{p}{,} \PY{l+m+mf}{0.2}\PY{p}{,} \PY{l+m+mf}{0.2}\PY{p}{,} \PY{l+m+mf}{0.2}\PY{p}{]}\PY{p}{,}
                                    \PY{p}{[}\PY{l+m+mf}{0.2}\PY{p}{,} \PY{l+m+mf}{0.2}\PY{p}{,} \PY{l+m+mf}{0.2}\PY{p}{,} \PY{l+m+mi}{1}\PY{p}{,} \PY{l+m+mf}{0.2}\PY{p}{,} \PY{l+m+mf}{0.2}\PY{p}{]}\PY{p}{,}
                                    \PY{p}{[}\PY{l+m+mf}{0.2}\PY{p}{,} \PY{l+m+mf}{0.2}\PY{p}{,} \PY{l+m+mf}{0.2}\PY{p}{,} \PY{l+m+mf}{0.2}\PY{p}{,} \PY{l+m+mi}{1}\PY{p}{,} \PY{l+m+mf}{0.2}\PY{p}{]}\PY{p}{,}
                                    \PY{p}{[}\PY{l+m+mf}{0.2}\PY{p}{,} \PY{l+m+mf}{0.2}\PY{p}{,} \PY{l+m+mf}{0.2}\PY{p}{,} \PY{l+m+mf}{0.2}\PY{p}{,} \PY{l+m+mf}{0.2}\PY{p}{,} \PY{l+m+mi}{1}\PY{p}{]}\PY{p}{]}\PY{p}{)}
	
\PY{n}{trials} \PY{o}{=} \PY{l+m+mi}{100000}
\PY{n}{result} \PY{o}{=} \PY{p}{[}\PY{l+m+mf}{0.}\PY{p}{,} \PY{l+m+mf}{0.}\PY{p}{,} \PY{l+m+mf}{0.}\PY{p}{,} \PY{l+m+mf}{0.}\PY{p}{,} \PY{l+m+mf}{0.}\PY{p}{,} \PY{l+m+mf}{0.}\PY{p}{]}
\PY{n}{x} \PY{o}{=} \PY{n}{mvnorm}\PY{o}{.}\PY{n}{rvs}\PY{p}{(}\PY{n}{size}\PY{o}{=}\PY{n}{trials}\PY{p}{)}
	
\PY{k}{for} \PY{n}{n} \PY{o+ow}{in} \PY{n+nb}{range}\PY{p}{(}\PY{n+nb}{len}\PY{p}{(}\PY{n}{x}\PY{p}{)}\PY{p}{)}\PY{p}{:}
    \PY{n}{p} \PY{o}{=} \PY{n+nb}{sorted}\PY{p}{(}\PY{n}{norm}\PY{o}{.}\PY{n}{cdf}\PY{p}{(}\PY{n}{x}\PY{p}{[}\PY{n}{n}\PY{p}{]}\PY{p}{)}\PY{p}{)}
    \PY{k}{for} \PY{n}{i} \PY{o+ow}{in} \PY{n+nb}{range}\PY{p}{(}\PY{l+m+mi}{1}\PY{p}{,} \PY{n+nb}{len}\PY{p}{(}\PY{n}{p\PYZus{}default}\PY{p}{)}\PY{p}{)}\PY{p}{:}
        \PY{k}{if} \PY{n}{p\PYZus{}default}\PY{p}{[}\PY{n}{i}\PY{o}{\PYZhy{}}\PY{l+m+mi}{1}\PY{p}{]} \PY{o}{\PYZlt{}}\PY{o}{=} \PY{n}{p}\PY{p}{[}\PY{l+m+mi}{2}\PY{p}{]} \PY{o}{\PYZlt{}}\PY{o}{=} \PY{n}{p\PYZus{}default}\PY{p}{[}\PY{n}{i}\PY{p}{]}\PY{p}{:}
            \PY{n}{result}\PY{p}{[}\PY{n}{i}\PY{p}{]} \PY{o}{+}\PY{o}{=} \PY{l+m+mi}{1}
	
\PY{n+nb}{print} \PY{p}{(}\PY{l+s+s2}{\PYZdq{}}\PY{l+s+s2}{3rd\PYZhy{}to\PYZhy{}default probabilies}\PY{l+s+s2}{\PYZdq{}}\PY{p}{)}
\PY{k}{for} \PY{n}{i} \PY{o+ow}{in} \PY{n+nb}{range}\PY{p}{(}\PY{n+nb}{len}\PY{p}{(}\PY{n}{p\PYZus{}default}\PY{p}{)}\PY{p}{)}\PY{p}{:}
    \PY{n+nb}{print} \PY{p}{(}\PY{l+s+s2}{\PYZdq{}}\PY{l+s+si}{\PYZob{}\PYZcb{}}\PY{l+s+s2}{: }\PY{l+s+si}{\PYZob{}:.4f\PYZcb{}}\PY{l+s+s2}{\PYZdq{}}\PY{o}{.}\PY{n}{format}\PY{p}{(}\PY{n}{i}\PY{p}{,} \PY{n}{result}\PY{p}{[}\PY{n}{i}\PY{p}{]}\PY{o}{/}\PY{n}{trials}\PY{p}{)}\PY{p}{)}

3rd-to-default probabilies
0: 0.0000
1: 0.0003
2: 0.0033
3: 0.0109
4: 0.0250
5: 0.0267
\end{Verbatim}
\end{tcolorbox}

\section{Gaussian Copula Model for Time to Default}\label{standard-market-model}
While there are several types of copula function models, the first
introduced was the one-factor Gaussian copula model. This model has,
above all, the advantage that can be solved semi-analytically.

Consider a portfolio of \(N\) companies and assume that the marginal
probabilities of default are known for each company. Define:

\begin{itemize}
	\tightlist
	\item
	\(t_i\), the time of default of the \(i\)th company:
	\item
	\(Q_i(t)\), the cumulative probability that company \(i\) will default
	before time \(t\); that is, the probability that \(t_i \le t\);
	\item
	\(S_i(t) = 1 – Q_i(t)\), the probability that company \(i\) will
	survive beyond time \(t\); that is, the probability that \(t_i > t\).
\end{itemize}

To generate a one-factor model for the \(t_i\) we define random
variables \(X_i\) \((1\le i \le N)\)

\[X_i = a_i M + \sqrt{1-a_i^2 Z_i},\qquad i = 1, 2,\ldots, n\]
where \(M\) and the \(Z_i\) are independent zero-mean unit-variance
distributions and \(–1 \le a_i \lt 1\).

The previous equation defines a correlation structure between the
\(X_i\) dependent on a single common factor \(M\) and an idiosincratic term $Z_i$. The correlation
between \(X_i\) and \(X_j\) is \(a_i a_j\).

Let \(F_i\) be the cumulative distribution of \(X_i\). Under the copula
model the \(X_i\) are mapped to the \(t_i\) using a
\emph{percentile-to-percentile} transformation. The five-percentile
point in the probability distribution for \(X_i\) is transformed to the
five-percentile point in the probability distribution of \(t_i\) and so
on.

In general the point \(X_i = x\) is transformed to \(t_i = t\) where
\(t = Q_i^{–1}[F_i(x)]\).

Let \(H\) be the cumulative distribution of the \(Z_i\). It follows from
the previous equation that

\[\mathbb{P}(X_i < x|M) = H\left(\cfrac{x-a_i M}{\sqrt{1-a_i^2}}\right)\]

When \(x = F_i^{–1}[Q_i(t)]\),
\(\mathbb{P}(t_i < t) = \mathbb{P}(x_i < x)\). Hence

\[\mathbb{P}(t_i < t|M) = H\left\{\cfrac{F_i^{–1}[Q_i(t)]-a_i M}{\sqrt{1-a_i^2}}\right\}\]

The conditional probability that the ith bond will \textbf{survive}
beyond time \(T\) is therefore

\[S_i(T|M) = 1 - H\left\{\cfrac{F_i^{–1}[Q_i(t)]-a_i M}{\sqrt{1-a_i^2}}\right\}\]

Although in principle any distribution could be used for \(M\)'s and the
\(Z\)'s (provided they have zero mean and unit variance), one common
choice is to let them be standard normal distributions (resulting in a
Gaussian copula).

Different choices of distributions result in different copula models,
and in different natures of the default dependence. For example, copulas
where the \(M\)'s have heavy tails generate models where there is a
greater likelihood of a clustering of early defaults for several
companies.

For simplicity, the following two assumptions are made:

\begin{itemize}
	\tightlist
	\item
	all the companies have the same default intensity, i.e,
	\(\lambda_i = \lambda\);
	\item
	the pairwise default correlations are the same, i.e \(a_i = a\).
\end{itemize}

The second assumption means that the contribution of the market
component is the same for all the companies and the correlation between
any two companies is constant, \(\rho = a^2\).

Under these assumptions, given the market situation \(M = m\), all the
companies have the same cumulative default probability
\(DP_{t|M}=\mathbb{P}(t_i < t|M)\). Moreover, for a given value of the
market component \(M\), the defaults are mutually independent for all
the underlying companies. Letting \(N_{t|m}\) be the total defaults that
have occurred by time \(t\) conditional on the market condition
\(M = m\), then \(N_{t|m}\) follows a binomial distribution, and

\[DP(N_{t|m} = j) = \cfrac{n!}{j!(n-j)!}DP^j_{t|m}(1-DP_{t|m})^{n-j},\qquad  j=0, 1, 2,\ldots,n\]
The probability that there will be exactly \(j\) defaults by time \(t\)
is
\begin{equation}
DP(N_{t} = j) = \int_{-\infty}^{\infty}{DP(N_{t|m} = j)f_M(m)dm}
\label{eq:gaussian_quadrature}
\end{equation}
where \(f_M(m)\) is the probability density function (PDF) of the random
variable \(M\).

This model is also called \emph{Market Standard Model}.

\subsubsection{Gaussian Quadrature}\label{gaussian-quadrature}
The integral in Eq.~\ref{eq:gaussian_quadrature} can be quite
complicated depending on the distribution of \(f_M\).

The Gaussian Quadrature is a technique that allows to approximate that
integral with a discrete weighted sum with weights determined by the
function \(f_M\). Assuming \(f\) is Gaussian we can write

\[\int_{-\infty}^{+\infty}\cfrac{1}{\sqrt{2\pi}}e^{-F^{2}/2}g(F)dF\approx\sum_{k=1}^{k=N}w_k g(F_k)\]

As \(N\) increases the accuracy of the approximation increases, but usually
\(N=60\) is sufficient.

\href{https://drive.google.com/file/d/1Ic20cgVx4dpDG4W_pIHR4EXq9ens1G-H/view?usp=sharing}{Here} a dedicated class, \(\tt{GaussianQuadrature}\) is
available. Through the method \(\tt{M}\) returns the appropriate list of
weights \(w_k\) and values \(F_k\) to compute the intergral (it is
necessary to download also \href{https://drive.google.com/file/d/1zpQ0ubbEzniJb9usMWeeMCc0DSc2MGsC/view?usp=sharing}{this file} with all the values).

\begin{tcolorbox}[breakable, size=fbox, boxrule=1pt, pad at break*=1mm,colback=cellbackground, colframe=cellborder]
\begin{Verbatim}[commandchars=\\\{\}]
\PY{k+kn}{from} \PY{n+nn}{finmarkets} \PY{k}{import} \PY{n}{GaussianQuadrature}
	
\PY{n}{gq} \PY{o}{=} \PY{n}{GaussianQuadrature}\PY{p}{(}\PY{p}{)}
\PY{n+nb}{print} \PY{p}{(}\PY{l+s+s2}{\PYZdq{}}\PY{l+s+s2}{values: }\PY{l+s+s2}{\PYZdq{}}\PY{p}{,} \PY{n}{gq}\PY{o}{.}\PY{n}{M}\PY{p}{(}\PY{l+m+mi}{60}\PY{p}{)}\PY{p}{[}\PY{l+m+mi}{0}\PY{p}{]}\PY{p}{)}
\PY{n+nb}{print} \PY{p}{(}\PY{l+s+s2}{\PYZdq{}}\PY{l+s+s2}{weights: }\PY{l+s+s2}{\PYZdq{}}\PY{p}{,} \PY{n}{gq}\PY{o}{.}\PY{n}{M}\PY{p}{(}\PY{l+m+mi}{60}\PY{p}{)}\PY{p}{[}\PY{l+m+mi}{1}\PY{p}{]}\PY{p}{)}

values:  [14.36715008, 13.46459111, 12.71717121, 12.04990464, 11.43418345,
10.85527694, 10.30435118, 9.775583711, 9.264879565, 8.769218063, 8.286287662,
7.814266859, 7.351685308, 6.897331785, 6.450190946, 6.009398508, 5.574208647,
5.143969727, 4.718105868, 4.296102681, 3.877496046, 3.461863141, 3.048815156,
2.637991292, 2.229053735, 1.821683385, 1.415576157, 1.010439725, 0.605990602,
0.201951448, -0.201951448, -0.605990602, -1.010439725, -1.415576157,
-1.821683385, -2.229053735, -2.637991292, -3.048815156, -3.461863141,
-3.877496046, -4.296102681, -4.718105868, -5.143969727, -5.574208647,
-6.009398508, -6.450190946, -6.897331785, -7.351685308, -7.814266859,
-8.286287662, -8.769218063, -9.264879565, -9.775583711, -10.30435118,
-10.85527694, -11.43418345, -12.04990464, -12.71717121, -13.46459111,
-14.36715008]
weights:  [6.26018e-46, 1.37648e-40, 2.12791e-36, 7.51816e-33, 9.67908e-30,
5.8078e-27, 1.88764e-24, 3.67432e-22, 4.6002e-20, 3.90602e-18, 2.34277e-16,
1.02492e-14, 3.35604e-13, 8.40054e-12, 1.63579e-10, 2.51449e-09, 3.08925e-08,
3.06553e-07, 2.47921e-06, 1.64672e-05, 9.04268e-05, 0.000412859, 0.001574836,
0.005039442, 0.013575119, 0.030871873, 0.059408466, 0.096914318, 0.134203158,
0.157890214, 0.157890214, 0.134203158, 0.096914318, 0.059408466, 0.030871873,
0.013575119, 0.005039442, 0.001574836, 0.000412859, 9.04268e-05, 1.64672e-05,
2.47921e-06, 3.06553e-07, 3.08925e-08, 2.51449e-09, 1.63579e-10, 8.40054e-12,
3.35604e-13, 1.02492e-14, 2.34277e-16, 3.90602e-18, 4.6002e-20, 3.67432e-22,
1.88764e-24, 5.8078e-27, 9.67908e-30, 7.51816e-33, 2.12791e-36, 1.37648e-40,
6.26018e-46]
\end{Verbatim}
\end{tcolorbox}

\section{Basket CDS Valuation under Market Standard Model}\label{basket-cds-valuation-under-market-standard-model}
We now present some numerical results for an nth to default basket.
We assume that the principals and expected recovery rates are the same
for all underlying reference assets. The valuation procedure is similar
to that for a regular CDS where there is only one reference entity.

In a regular CDS indeed the valuation is based on the probability that a
default occurred between times \(t1\) and \(t2\). Here instead the
valuation will be based on the probability that the nth default was
between times \(t1\) and \(t2\).
The buyer of protection makes quarterly payments at a
specified rate until the nth default occurs or the end of life
of the contract is reached.

In the event of the nth default occurring, the seller pays
\(N\cdot(1-R)\). The contract can be valued by calculating the expected
present value of payments and the expected present value of payoffs in a
risk-neutral world.

Consider a 5-year nth to default CDS on a basket of ten
reference entities in the situation where the copula correlation is 0.3
and the expected recovery rate, \(R\), is \(40\%\). The term structure
of interest rates is assumed to be flat at 5\%. The default
probabilities for the ten entities are generated by Poisson processes
with constant default intensities (hazard rates), \(\lambda_i\),
\((1 \le i \le 10)\) so that

\[ DP(t) = 1 - e^{-\lambda t} \]

\begin{tcolorbox}[breakable, size=fbox, boxrule=1pt, pad at break*=1mm,colback=cellbackground, colframe=cellborder]
\begin{Verbatim}[commandchars=\\\{\}]
\PY{k+kn}{from} \PY{n+nn}{finmarkets} \PY{k}{import} \PY{n}{DiscountCurve}\PY{p}{,} \PY{n}{CreditCurve}\PY{p}{,} \PY{n}{CreditDefaultSwap}
\PY{k+kn}{from} \PY{n+nn}{finmarkets} \PY{k}{import} \PY{n}{GaussianQuadrature}
\PY{k+kn}{from} \PY{n+nn}{datetime} \PY{k}{import} \PY{n}{date}
\PY{k+kn}{from} \PY{n+nn}{dateutil}\PY{n+nn}{.}\PY{n+nn}{relativedelta} \PY{k}{import} \PY{n}{relativedelta}
\PY{k+kn}{from} \PY{n+nn}{scipy}\PY{n+nn}{.}\PY{n+nn}{stats} \PY{k}{import} \PY{n}{norm}\PY{p}{,} \PY{n}{binom}
\PY{k+kn}{from} \PY{n+nn}{math} \PY{k}{import} \PY{n}{sqrt}\PY{p}{,} \PY{n}{exp}
	
\PY{n}{n\PYZus{}cds} \PY{o}{=} \PY{l+m+mi}{10}
\PY{n}{rho} \PY{o}{=} \PY{l+m+mf}{0.3}
\PY{n}{l} \PY{o}{=} \PY{l+m+mf}{0.01}
\PY{n}{pillar\PYZus{}dates} \PY{o}{=} \PY{p}{[}\PY{p}{]}
\PY{n}{df} \PY{o}{=} \PY{p}{[}\PY{p}{]}
\PY{n}{observation\PYZus{}date} \PY{o}{=} \PY{n}{date}\PY{o}{.}\PY{n}{today}\PY{p}{(}\PY{p}{)}
	
\PY{k}{for} \PY{n}{i} \PY{o+ow}{in} \PY{n+nb}{range}\PY{p}{(}\PY{l+m+mi}{6}\PY{p}{)}\PY{p}{:}
\PY{n}{pillar\PYZus{}dates}\PY{o}{.}\PY{n}{append}\PY{p}{(}\PY{n}{observation\PYZus{}date} \PY{o}{+} \PY{n}{relativedelta}\PY{p}{(}\PY{n}{years}\PY{o}{=}\PY{n}{i}\PY{p}{)}\PY{p}{)}
\PY{n}{df}\PY{o}{.}\PY{n}{append}\PY{p}{(}\PY{l+m+mi}{1}\PY{o}{/}\PY{p}{(}\PY{l+m+mi}{1}\PY{o}{+}\PY{l+m+mf}{0.05}\PY{o}{*}\PY{n}{i}\PY{p}{)}\PY{p}{)}
\PY{n}{dc} \PY{o}{=} \PY{n}{DiscountCurve}\PY{p}{(}\PY{n}{observation\PYZus{}date}\PY{p}{,} \PY{n}{pillar\PYZus{}dates}\PY{p}{,} \PY{n}{df}\PY{p}{)}
	
\PY{n}{gq} \PY{o}{=} \PY{n}{GaussianQuadrature}\PY{p}{(}\PY{p}{)}
\PY{n}{values}\PY{p}{,} \PY{n}{weights} \PY{o}{=} \PY{n}{gq}\PY{o}{.}\PY{n}{M}\PY{p}{(}\PY{l+m+mi}{60}\PY{p}{)}
	
\PY{n}{Q} \PY{o}{=} \PY{p}{[}\PY{l+m+mi}{1}\PY{o}{\PYZhy{}}\PY{n}{exp}\PY{p}{(}\PY{o}{\PYZhy{}}\PY{p}{(}\PY{n}{l}\PY{o}{*}\PY{n}{t}\PY{p}{)}\PY{p}{)} \PY{k}{for} \PY{n}{t} \PY{o+ow}{in} \PY{n+nb}{range}\PY{p}{(}\PY{l+m+mi}{6}\PY{p}{)}\PY{p}{]}
\PY{n}{cds} \PY{o}{=} \PY{n}{CreditDefaultSwap}\PY{p}{(}\PY{l+m+mi}{1}\PY{p}{,} \PY{n}{observation\PYZus{}date}\PY{p}{,} \PY{l+m+mf}{0.01}\PY{p}{,} \PY{l+m+mi}{5}\PY{p}{)}
	
\PY{n}{ndefault} \PY{o}{=} \PY{l+m+mi}{3}
\PY{n}{S} \PY{o}{=} \PY{p}{[}\PY{p}{]}
\PY{k}{for} \PY{n}{j} \PY{o+ow}{in} \PY{n+nb}{range}\PY{p}{(}\PY{n+nb}{len}\PY{p}{(}\PY{n}{values}\PY{p}{)}\PY{p}{)}\PY{p}{:}
    \PY{n}{temp} \PY{o}{=} \PY{p}{[}\PY{p}{]}
    \PY{k}{for} \PY{n}{i} \PY{o+ow}{in} \PY{n+nb}{range}\PY{p}{(}\PY{l+m+mi}{6}\PY{p}{)}\PY{p}{:}
        \PY{n}{P} \PY{o}{=} \PY{n}{norm}\PY{o}{.}\PY{n}{cdf}\PY{p}{(}\PY{p}{(}\PY{n}{norm}\PY{o}{.}\PY{n}{ppf}\PY{p}{(}\PY{n}{Q}\PY{p}{[}\PY{n}{i}\PY{p}{]}\PY{p}{)} \PY{o}{\PYZhy{}} \PY{n}{sqrt}\PY{p}{(}\PY{n}{rho}\PY{p}{)}\PY{o}{*}\PY{n}{values}\PY{p}{[}\PY{n}{j}\PY{p}{]}\PY{p}{)}\PY{o}{/}
        \PY{p}{(}\PY{n}{sqrt}\PY{p}{(}\PY{l+m+mi}{1}\PY{o}{\PYZhy{}}\PY{n}{rho}\PY{p}{)}\PY{p}{)}\PY{p}{)}
        \PY{n}{b} \PY{o}{=} \PY{n}{binom}\PY{p}{(}\PY{n}{n\PYZus{}cds}\PY{p}{,} \PY{n}{P}\PY{p}{)}
        \PY{n}{temp}\PY{o}{.}\PY{n}{append}\PY{p}{(}\PY{l+m+mi}{1} \PY{o}{\PYZhy{}} \PY{p}{(}\PY{n}{b}\PY{o}{.}\PY{n}{cdf}\PY{p}{(}\PY{n}{n\PYZus{}cds}\PY{p}{)}\PY{o}{\PYZhy{}}\PY{n}{b}\PY{o}{.}\PY{n}{cdf}\PY{p}{(}\PY{n}{ndefault}\PY{o}{\PYZhy{}}\PY{l+m+mi}{1}\PY{p}{)}\PY{p}{)}\PY{p}{)}
    \PY{n}{S}\PY{o}{.}\PY{n}{append}\PY{p}{(}\PY{n}{temp}\PY{p}{)}
	
\PY{n}{s} \PY{o}{=} \PY{l+m+mi}{0}
\PY{k}{for} \PY{n}{j} \PY{o+ow}{in} \PY{n+nb}{range}\PY{p}{(}\PY{n+nb}{len}\PY{p}{(}\PY{n}{values}\PY{p}{)}\PY{p}{)}\PY{p}{:}
    \PY{n}{s} \PY{o}{+}\PY{o}{=} \PY{n}{weights}\PY{p}{[}\PY{n}{j}\PY{p}{]} \PY{o}{*} \PY{n}{cds}\PY{o}{.}\PY{n}{breakevenRate}\PY{p}{(}\PY{n}{dc}\PY{p}{,} \PY{n}{CreditCurve}\PY{p}{(}\PY{n}{pillar\PYZus{}dates}\PY{p}{,} \PY{n}{S}\PY{p}{[}\PY{n}{j}\PY{p}{]}\PY{p}{)}\PY{p}{)}
\PY{n+nb}{print} \PY{p}{(}\PY{n}{s}\PY{p}{)}

0.0017795634956118353
\end{Verbatim}
\end{tcolorbox}

\section{CDO Valuation}
Suppose that the payment date on a CDO tranche are at times $\tau_i$. Define $\mathbb{E}_j$ the expected 
tranche principal at time $\tau$ and $D(\tau)$ the discount factor at time $\tau$. Suppose also that the spread
on a particular tranche (i.e. the number of basis point paid for protection on the remaining tranche principal) is $s$. 

The present value of the expected regular spread payments on the CDO is given by
\begin{equation}
s\cdot \sum_{j=1}^{m}(\tau_j - \tau_{j-1})\mathbb{E}_{j}D(\tau_j)
\label{eq:A}
\end{equation}
The expected loss between times $\tau_{j-1}$ and $\tau_j$ is $\mathbb{E}_{j-1}-\mathbb{E}_j$. For simplicity assume
the loss occurs only at the midpoint of the time interval, so the present value of the expected payoffs on the CDO tranche is
\begin{equation}
\sum_{j=1}^{m}(\mathbb{E}_{j-1}-\mathbb{E}_j)D\left(\frac{\tau_{j-1}+\tau_j}{2}\right)
\label{eq:C}
\end{equation}
The accrual payment due on the losses is finally given by
\begin{equation}
s\cdot\sum_{j=1}^{m}\frac{1}{2}(\tau_j - \tau_{j-1})(\mathbb{E}_{j-1}-\mathbb{E}_j)D(\frac{\tau_{j-1}+\tau_j}{2})
\label{eq:B}
\end{equation}

The value of the tranche, valued from the point of view of the protection buyer is $C-sA-sB$. The breakeven spread 
on the tranche occurs when the present value of the payments equals the present value of the payoffs so

\[ s = \cfrac{C}{A+B}\]

Suppose that the tranche under consideration covers losses on the portfolio between $\alpha_L$ and $\alpha_H$ and
define

\[n_L = \cfrac{\alpha_L n}{1-R}\qquad \mathrm{and}\qquad n_H = \cfrac{\alpha_H n}{1-R}\]
where $R$ is the recovery rate. Finally define $m(x)$ as the smallest integer greater than $x$.
By definition the tranche principal stays $N$ while the number of defaults $k$ is less than $m(n_L)$, it 
is zero when the number of default is greater or equal to $m(n_H)$, otherwise is

\[\cfrac{\alpha_H -k(1-R)/n}{\alpha_H - \alpha_L}\]

The expected tranche principal at time $\tau_j$ conditional of the value of the factor $M$ is
\begin{equation}
\mathbb{E}_j(M) = \sum_{k=0}^{m(n_L)-1}\mathbb{P}(k, \tau_j|M) + \sum_{k=m(n_L)}^{m(n_H)-1}\mathbb{P}(k, \tau_j|M) \cfrac{\alpha_H -k(1-R)/n}{\alpha_H - \alpha_L}
\label{eq:E}
\end{equation}

To compute the breakeven spread it is necessary to substitute Eq.~\ref{eq:E} into Eq.~\ref{eq:A},~\ref{eq:B} and~\ref{eq:C}
and we need to integrate the result over the variable $M$ (remember that has a standard normal distribution). 
The integration is quite complicated and is best accomplished with a technique called \emph{Gaussian quadrature} which
exploits the approximation

\[\int_{-\infty}^{\infty}{\cfrac{1}{\sqrt{2}}e^{-M^{2}/2}g(M)dM} \approx \sum_{k=1}^{k=L}w_k g(F_k)\]
as $L$ increases, accuracy increases.

\section{Extensions of the One Factor Copula Model}


\chapter{Modeling Correlation between Risks}


In credit derivative valuation and credit risk management, one of the
fundamentally important issues is the estimation of default
probabilities and their correlations. For this, generally speaking,
there are two ways: using historical default data or using mathematical
models.

Historical default data has played an important role in the estimation
of default probabilities. However, because default events are rare,
there is very limited default data available. Moreover, historical data
reflects the historical default pattern only and it may not be a proper
indicator of the future. This makes the estimation of default
probabilities from historical data difficult and inexact. To use this
same data to estimate default correlations is even more difficult and
more inexact.

The market trend now is towards more and more to the use of mathematical
models that don't rely on historical default data. In
Chapter~\ref{credit_default_swaps} we have seen how it is possible to derive default probabilities from market data,
here we will see how the copula can be used to model their correlations. 

Before going into the details of the application of the copula let's introduce two kind of credit derivatives.

\subsection{Basket Default Swaps}\label{basket-default-swaps}

A basket default swap is a credit derivative on a portfolio of reference
entities. The simplest basket default swaps are first-to-default swaps,
second-to-default swaps, and nth-to-default swaps. 

With respect to a basket of reference entities, a first-to-default swap provides insurance for only the first default, 
a second-to-default swap provides insurance
for only the second default, an nth-to-default swap provides insurance
for only the nth default. 

For example, in an nth-to-default swap, the
protection seller does not make a payment to the protection buyer for
the first n-defaulted reference entities, and makes a payment only for the
nth defaulted reference entity. Once there has been this payment the swap terminates.

\subsection{Collateralized Debt Obligation}\label{collateralized-debt-obligation}

A collateralized debt obligation (CDO) is a security backed by a
diversified pool of one or more kinds of debt obligations such as bonds,
loans, credit default swaps or structured products (mortgage-backed
securities, asset-backed securities, and even other CDOs). 

A CDO can be
initiated by one or more of the following: banks, non-bank financial
institutions, and asset management companies, which is referred to as the
sponsor. 

The sponsor of a CDO creates a company so-called the special
purpose vehicle (SPV). The SPV works as an independent entity. In this
way, CDO investors are isolated from the credit risk of the sponsor.
Moreover, the SPV is responsible for the administration. It obtains
the credit risk exposure by purchasing debt obligations (bonds or
residential and commercial loans) or selling CDSs; it transfers the
credit risk by issuing debt obligations (tranches/credit-linked notes).
The investors in the tranches of a CDO have the ultimate credit risk
exposure to the underlying reference entities. 

The SPV issues four kinds
of tranches. Each tranche has an attachment percentage and a detachment
percentage. When the cumulative percentage loss of the portfolio reaches
the attachment percentage, investors in the tranche start to lose their
principal, and when the cumulative percentage loss of principal reaches
the detachment percentage, the investors in the tranche lose all their
principal and no further loss can occur to them.

In the literature, tranches of a CDO are classified as
subordinate/equity tranche, mezzanine tranches, and senior tranches
according to their levels. Because the equity tranche is
extremely risky, the sponsor of a CDO holds the equity tranche and the
SPV sells other tranches to investors.


\section{CDO Valuation}
Suppose that the payment dates of a CDO tranche are at times $d_i$. Define $\mathbb{E}_j$ the expected 
tranche value at time $d$ and $D(d)$ the discount factor. Suppose also that the spread
on a particular tranche (i.e. the number of basis point paid for protection on the remaining tranche principal) is $S$. 

The present value of the expected regular spread payments on the CDO is given by
\begin{equation}
\mathrm{NPV}_{\mathrm{spread}} = S\cdot \sum_{j=1}^{m}(d_j - d_{j-1})\mathbb{E}_{j}D(d_j)
\label{eq:A}
\end{equation}
The expected loss between times $d_{j-1}$ and $d_j$ is $\mathbb{E}_{j-1}-\mathbb{E}_j$. For simplicity assume
the loss occurs only at the midpoint of the time interval, so the present value of the expected payoffs on the CDO tranche is
\begin{equation}
\mathrm{NPV}_{\mathrm{payoff}}=\sum_{j=1}^{m}(\mathbb{E}_{j-1}-\mathbb{E}_j)D\left(\frac{d_{j-1}+d_j}{2}\right)
\label{eq:C}
\end{equation}
The accrual payment due to the losses is finally given by
\begin{equation}
\mathrm{NPV}_{\mathrm{accrual}} = S\cdot\sum_{j=1}^{m}\frac{1}{2}(d_j - d_{j-1})(\mathbb{E}_{j-1}-\mathbb{E}_j)D(\frac{d_{j-1}+d_j}{2})
\label{eq:B}
\end{equation}

The value of the tranche, valued from the point of view of the protection buyer is $C-sA-sB$. The breakeven spread 
on the tranche occurs when the present value of the payments equals the present value of the payoffs so

\[ s = \cfrac{C}{A+B}\]

Suppose that the tranche under consideration covers losses on the portfolio between $\alpha_L$ and $\alpha_H$ and
define

\[n_L = \cfrac{\alpha_L n}{1-R}\qquad \mathrm{and}\qquad n_H = \cfrac{\alpha_H n}{1-R}\]
where $R$ is the recovery rate. Finally define $m(x)$ as the smallest integer greater than $x$.
By definition the tranche principal stays $N$ while the number of defaults $k$ is less than $m(n_L)$, it 
is zero when the number of default is greater or equal to $m(n_H)$, otherwise is

\[\cfrac{\alpha_H -k(1-R)/n}{\alpha_H - \alpha_L}\]

The expected tranche principal at time $\tau_j$ conditional of the value of the factor $M$ is
\begin{equation}
\mathbb{E}_j(M) = \sum_{k=0}^{m(n_L)-1}\mathbb{P}(k, \tau_j|M) + \sum_{k=m(n_L)}^{m(n_H)-1}\mathbb{P}(k, \tau_j|M) \cfrac{\alpha_H -k(1-R)/n}{\alpha_H - \alpha_L}
\label{eq:E}
\end{equation}

To compute the breakeven spread it is necessary to substitute Eq.~\ref{eq:E} into Eq.~\ref{eq:A},~\ref{eq:B} and~\ref{eq:C}
and we need to integrate the result over the variable $M$ (remember that has a standard normal distribution). 
The integration is quite complicated and is best accomplished with a technique called \emph{Gaussian quadrature} which
exploits the approximation

\[\int_{-\infty}^{\infty}{\cfrac{1}{\sqrt{2}}e^{-M^{2}/2}g(M)dM} \approx \sum_{k=1}^{k=L}w_k g(F_k)\]
as $L$ increases, accuracy increases.

\section{Basket CDS Valuation}

\section{Extensions of the One Factor Copula Model}


\chapter{Discount Factors and Forward Rates}\label{introduction-to-python---lesson-6}

\begin{Exercise}
Python has a useful command called \texttt{assert} which can be used for
checking that a given condition is satisfied, and raising an error if
the condition is not satisfied.

The following line does not cause an error, in fact it does nothing

\begin{Shaded}
\begin{Highlighting}[]
\ControlFlowTok{assert} \DecValTok{1} \OperatorTok{<} \DecValTok{2}
\end{Highlighting}
\end{Shaded}

This causes an error

\begin{Shaded}
\begin{Highlighting}[]
\ControlFlowTok{assert} \DecValTok{1} \OperatorTok{>} \DecValTok{2}
\end{Highlighting}
\end{Shaded}

\texttt{assert} can take a second argument with a message to display in case of failure.

\begin{Shaded}
\begin{Highlighting}[]
\ControlFlowTok{assert} \DecValTok{1} \OperatorTok{>} \DecValTok{2}\NormalTok{, }\StringTok{"Two is bigger than one"}
\end{Highlighting}
\end{Shaded}

Take the \texttt{df} function from lesson 3 and modify it by adding some assertions to check that:

\begin{itemize}
\tightlist
\item
  the pillar date list contains at least 2 elements;
\item
  the pillar date list is the same length as the discount factor list;
\item
  the first pillar date is equal to the today date;
\item
  the value date argument `d' is greater or equal to the first pillar
  date and also less than or equal to the last pillar date.
\end{itemize}

Then try using the function with some invalid data to make sure that your as sertions are correctly checking the desired conditions
\end{Exercise}

\begin{Answer}
\begin{tcolorbox}[size=fbox, boxrule=1pt, colback=cellbackground, colframe=cellborder]
\begin{Verbatim}[commandchars=\\\{\}]
\PY{c+c1}{\PYZsh{} import modules and objects that we need}
\PY{k+kn}{from} \PY{n+nn}{datetime} \PY{k}{import} \PY{n}{date}
\PY{k+kn}{import} \PY{n+nn}{numpy}
\PY{k+kn}{import} \PY{n+nn}{math}

\PY{c+c1}{\PYZsh{} define the input data}
\PY{n}{today\PYZus{}date} \PY{o}{=} \PY{n}{date}\PY{p}{(}\PY{l+m+mi}{2017}\PY{p}{,} \PY{l+m+mi}{10}\PY{p}{,} \PY{l+m+mi}{1}\PY{p}{)}
\PY{n}{pillar\PYZus{}dates} \PY{o}{=} \PY{p}{[}\PY{n}{date}\PY{p}{(}\PY{l+m+mi}{2017}\PY{p}{,} \PY{l+m+mi}{10}\PY{p}{,} \PY{l+m+mi}{1}\PY{p}{)}\PY{p}{,} 
                \PY{n}{date}\PY{p}{(}\PY{l+m+mi}{2018}\PY{p}{,} \PY{l+m+mi}{10}\PY{p}{,} \PY{l+m+mi}{1}\PY{p}{)}\PY{p}{,} 
                \PY{n}{date}\PY{p}{(}\PY{l+m+mi}{2019}\PY{p}{,} \PY{l+m+mi}{10}\PY{p}{,} \PY{l+m+mi}{1}\PY{p}{)}\PY{p}{]}
\PY{n}{discount\PYZus{}factors} \PY{o}{=} \PY{p}{[}\PY{l+m+mf}{1.0}\PY{p}{,} \PY{l+m+mf}{0.95}\PY{p}{,} \PY{l+m+mf}{0.8}\PY{p}{]}

\PY{c+c1}{\PYZsh{} define the df function}
\PY{k}{def} \PY{n+nf}{df}\PY{p}{(}\PY{n}{t}\PY{p}{)}\PY{p}{:}
    \PY{c+c1}{\PYZsh{}\PYZsh{}\PYZsh{}\PYZsh{}\PYZsh{}\PYZsh{}\PYZsh{}\PYZsh{}\PYZsh{}\PYZsh{}\PYZsh{}\PYZsh{}\PYZsh{}\PYZsh{} CHECKS \PYZsh{}\PYZsh{}\PYZsh{}\PYZsh{}\PYZsh{}\PYZsh{}\PYZsh{}\PYZsh{}\PYZsh{}\PYZsh{}\PYZsh{}\PYZsh{}\PYZsh{}\PYZsh{}\PYZsh{}\PYZsh{}}
    \PY{c+c1}{\PYZsh{} Check that there are at least 2 pillar dates}
    \PY{k}{assert} \PY{n+nb}{len}\PY{p}{(}\PY{n}{pillar\PYZus{}dates}\PY{p}{)} \PY{o}{\PYZgt{}}\PY{o}{=} \PY{l+m+mi}{2}\PY{p}{,} \PY{l+s+s2}{\PYZdq{}}\PY{l+s+s2}{ need at least 2 pillar dates}\PY{l+s+s2}{\PYZdq{}}
    
    \PY{c+c1}{\PYZsh{} Check that the number of pillar dates }
    \PY{c+c1}{\PYZsh{}is equal to the number of pillar discount factors}
    \PY{k}{assert} \PY{n+nb}{len}\PY{p}{(}\PY{n}{pillar\PYZus{}dates}\PY{p}{)} \PY{o}{==} \PY{n+nb}{len}\PY{p}{(}\PY{n}{discount\PYZus{}factors}\PY{p}{)}\PY{p}{,} \PYZbs{}
        \PY{l+s+s2}{\PYZdq{}}\PY{l+s+s2}{number of pillar dates should be equal to }\PY{l+s+se}{\PYZbs{}}
\PY{l+s+s2}{        the number of pillar discount factors}\PY{l+s+s2}{\PYZdq{}}
    
    \PY{c+c1}{\PYZsh{} Check that the first pillar date is the today date}
    \PY{k}{assert} \PY{n}{today\PYZus{}date} \PY{o}{==} \PY{n}{pillar\PYZus{}dates}\PY{p}{[}\PY{l+m+mi}{0}\PY{p}{]}\PY{p}{,} \PYZbs{}
        \PY{l+s+s2}{\PYZdq{}}\PY{l+s+s2}{first pillar date should be the today date}\PY{l+s+s2}{\PYZdq{}}
    
    \PY{c+c1}{\PYZsh{} Check that the value date argument is between }
    \PY{c+c1}{\PYZsh{}the first and last pillar dates}
    \PY{k}{assert} \PY{n}{pillar\PYZus{}dates}\PY{p}{[}\PY{l+m+mi}{0}\PY{p}{]} \PY{o}{\PYZlt{}}\PY{o}{=} \PY{n}{t} \PY{o}{\PYZlt{}}\PY{o}{=} \PY{n}{pillar\PYZus{}dates}\PY{p}{[}\PY{o}{\PYZhy{}}\PY{l+m+mi}{1}\PY{p}{]}\PY{p}{,} \PYZbs{}
        \PY{l+s+s2}{\PYZdq{}}\PY{l+s+s2}{Invalid value date }\PY{l+s+si}{\PYZpc{}s}\PY{l+s+s2}{\PYZdq{}} \PY{o}{\PYZpc{}} \PY{p}{(}\PY{n}{d}\PY{p}{)}
    \PY{c+c1}{\PYZsh{}\PYZsh{}\PYZsh{}\PYZsh{}\PYZsh{}\PYZsh{}\PYZsh{}\PYZsh{}\PYZsh{}\PYZsh{}\PYZsh{}\PYZsh{}\PYZsh{}\PYZsh{} END OF CHECKS \PYZsh{}\PYZsh{}\PYZsh{}\PYZsh{}\PYZsh{}\PYZsh{}\PYZsh{}\PYZsh{}\PYZsh{}\PYZsh{}\PYZsh{}\PYZsh{}\PYZsh{}\PYZsh{}\PYZsh{}\PYZsh{}}
    
    \PY{n}{log\PYZus{}discount\PYZus{}factors} \PY{o}{=} \PY{p}{[}\PY{p}{]}
    \PY{k}{for} \PY{n}{discount\PYZus{}factor} \PY{o+ow}{in} \PY{n}{discount\PYZus{}factors}\PY{p}{:}
        \PY{n}{log\PYZus{}discount\PYZus{}factors}\PY{o}{.}\PY{n}{append}\PY{p}{(}\PY{n}{math}\PY{o}{.}\PY{n}{log}\PY{p}{(}\PY{n}{discount\PYZus{}factor}\PY{p}{)}\PY{p}{)}
    
    \PY{n}{pillar\PYZus{}days} \PY{o}{=} \PY{p}{[}\PY{p}{]}
    \PY{k}{for} \PY{n}{pillar\PYZus{}date} \PY{o+ow}{in} \PY{n}{pillar\PYZus{}dates}\PY{p}{:}
        \PY{n}{pillar\PYZus{}days}\PY{o}{.}\PY{n}{append}\PY{p}{(}\PY{p}{(}\PY{n}{pillar\PYZus{}date} \PY{o}{\PYZhy{}} \PY{n}{today\PYZus{}date}\PY{p}{)}\PY{o}{.}\PY{n}{days}\PY{p}{)}
    
    \PY{n}{t\PYZus{}days} \PY{o}{=} \PY{p}{(}\PY{n}{t} \PY{o}{\PYZhy{}} \PY{n}{today\PYZus{}date}\PY{p}{)}\PY{o}{.}\PY{n}{days}
    
    \PY{n}{interpolated\PYZus{}log\PYZus{}discount\PYZus{}factor} \PY{o}{=} \PYZbs{}
        \PY{n}{numpy}\PY{o}{.}\PY{n}{interp}\PY{p}{(}\PY{n}{t\PYZus{}days}\PY{p}{,} \PY{n}{pillar\PYZus{}days}\PY{p}{,} \PY{n}{log\PYZus{}discount\PYZus{}factors}\PY{p}{)}
    
    \PY{k}{return} \PY{n}{math}\PY{o}{.}\PY{n}{exp}\PY{p}{(}\PY{n}{interpolated\PYZus{}log\PYZus{}discount\PYZus{}factor}\PY{p}{)}

\PY{n}{df}\PY{p}{(}\PY{n}{date}\PY{p}{(}\PY{l+m+mi}{2019}\PY{p}{,} \PY{l+m+mi}{1}\PY{p}{,} \PY{l+m+mi}{1}\PY{p}{)}\PY{p}{)}

0.9097285910181567
\end{Verbatim}
\end{tcolorbox} 
\end{Answer}

\begin{Exercise}[label={ex:BS3}]
Create a file \texttt{finmarket.py} and copy into it the function used to compute Black Scholes formula used in Exercise~\ref{ex:BS2}. This is the intial implementation of our financial library. Repeat Exercise~\ref{ex:BS2} now using the version of the Black and Scholes formula in the finmarket module.
\end{Exercise}

\begin{Answer}
\begin{tcolorbox}[size=fbox, boxrule=1pt, colback=cellbackground, colframe=cellborder]
\begin{Verbatim}[commandchars=\\\{\}]
\PY{k}{import} \PY{n}{finmarket}
        
\PY{n}{s} \PY{o}{=} \PY{l+m+mi}{800}
\PY{c+c1}{\PYZsh{} strikes expressed as \PYZpc{} of spot price}
\PY{n}{moneyness} \PY{o}{=} \PY{p}{[} \PY{l+m+mf}{0.5}\PY{p}{,} \PY{l+m+mf}{0.75}\PY{p}{,} \PY{l+m+mf}{0.825}\PY{p}{,} \PYZbs{}
             \PY{l+m+mf}{1.0}\PY{p}{,} \PY{l+m+mf}{1.125}\PY{p}{,} \PY{l+m+mf}{1.25}\PY{p}{,} \PY{l+m+mf}{1.5} \PY{p}{]}
\PY{n}{vol} \PY{o}{=} \PY{l+m+mf}{0.3}
\PY{n}{ttm} \PY{o}{=} \PY{l+m+mf}{0.75}
\PY{n}{r} \PY{o}{=} \PY{l+m+mf}{0.005}

\PY{n}{result} \PY{o}{=} \PY{p}{\PYZob{}}\PY{p}{\PYZcb{}}
\PY{k}{for} \PY{n}{m} \PY{o+ow}{in} \PY{n}{moneyness}\PY{p}{:}
    \PY{n}{result}\PY{p}{[}\PY{n}{s}\PY{o}{*}\PY{n}{m}\PY{p}{]} \PY{o}{=} \PY{n}{call}\PY{p}{(}\PY{n}{s}\PY{p}{,} \PY{n}{m}\PY{o}{*}\PY{n}{s}\PY{p}{,} \PY{n}{r}\PY{p}{,} \PY{n}{vol}\PY{p}{,} \PY{n}{ttm}\PY{p}{)}
\PY{n}{result}

\{400.0: 401.66074527896365,
  600.0: 213.9883852521275,
  660.0: 166.85957363897393,
  800.0: 84.03697017660357,
  900.0: 47.61880394696229,
  1000.0: 25.632722952585738,
  1200.0: 6.655275227771156\}
\end{Verbatim}
\end{tcolorbox}
\end{Answer}

\begin{Exercise}
Add to \texttt{finmarket} module the \texttt{DiscountCurve} class developed during the lesson. Using the libor and eonia curves defined in the examples compute the forward 6m libor coupon.
\end{Exercise}

\begin{Answer}
\begin{tcolorbox}[size=fbox, boxrule=1pt, colback=cellbackground, colframe=cellborder]
\begin{Verbatim}[commandchars=\\\{\}]
\PY{n}{npv} \PY{o}{=} \PY{n}{eonia\PYZus{}curve}\PY{o}{.}\PY{n}{df}\PY{p}{(}\PY{n}{date}\PY{p}{(}\PY{l+m+mi}{2020}\PY{p}{,} \PY{l+m+mi}{4}\PY{p}{,} \PY{l+m+mi}{1}\PY{p}{)}\PY{p}{)} \PY{o}{*} \PYZbs{}
      \PY{n}{libor\PYZus{}curve}\PY{o}{.}\PY{n}{forward\PYZus{}rate}\PY{p}{(}\PY{n}{date}\PY{p}{(}\PY{l+m+mi}{2020}\PY{p}{,}\PY{l+m+mi}{4}\PY{p}{,} \PY{l+m+mi}{1}\PY{p}{)}\PY{p}{,} 
                                \PY{n}{date}\PY{p}{(}\PY{l+m+mi}{2020}\PY{p}{,} \PY{l+m+mi}{10}\PY{p}{,} \PY{l+m+mi}{1}\PY{p}{)}\PY{p}{)}

\PY{c+c1}{\PYZsh{} Compute it in the pre\PYZhy{}2008 way}
\PY{n}{npv\PYZus{}pre\PYZus{}2008} \PY{o}{=} \PY{n}{libor\PYZus{}curve}\PY{o}{.}\PY{n}{df}\PY{p}{(}\PY{n}{date}\PY{p}{(}\PY{l+m+mi}{2020}\PY{p}{,} \PY{l+m+mi}{4}\PY{p}{,} \PY{l+m+mi}{1}\PY{p}{)}\PY{p}{)} \PY{o}{*} \PYZbs{}
               \PY{n}{libor\PYZus{}curve}\PY{o}{.}\PY{n}{forward\PYZus{}rate}\PY{p}{(}\PY{n}{date}\PY{p}{(}\PY{l+m+mi}{2020}\PY{p}{,} \PY{l+m+mi}{4}\PY{p}{,} \PY{l+m+mi}{1}\PY{p}{)}\PY{p}{,} 
                                         \PY{n}{date}\PY{p}{(}\PY{l+m+mi}{2020}\PY{p}{,} \PY{l+m+mi}{10}\PY{p}{,} \PY{l+m+mi}{1}\PY{p}{)}\PY{p}{)}
\PY{n+nb}{print} \PY{p}{(}\PY{n}{npv}\PY{p}{)}
\PY{n+nb}{print} \PY{p}{(}\PY{n}{npv\PYZus{}pre\PYZus{}2008}\PY{p}{)}

0.37932346377238657
0.38139410902305737
\end{Verbatim}
\end{tcolorbox}
\end{Answer}




  

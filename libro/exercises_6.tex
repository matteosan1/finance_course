\chapter{Interpolation, Discount Factors and Forward Rates}\label{introduction-to-python---lesson-6}

\begin{Exercise}
Python has a useful command called \texttt{assert} which can be used for
checking that a given condition is satisfied, and raising an error if
the condition is not satisfied.

The following line does not cause an error, in fact it does nothing

\begin{Shaded}
\begin{Highlighting}[]
\ControlFlowTok{assert} \DecValTok{1} \OperatorTok{<} \DecValTok{2}
\end{Highlighting}
\end{Shaded}

This causes an error

\begin{Shaded}
\begin{Highlighting}[]
\ControlFlowTok{assert} \DecValTok{1} \OperatorTok{>} \DecValTok{2}
\end{Highlighting}
\end{Shaded}

\texttt{assert} can take a second argument with a message to display in case of failure.

\begin{Shaded}
\begin{Highlighting}[]
\ControlFlowTok{assert} \DecValTok{1} \OperatorTok{>} \DecValTok{2}\NormalTok{, }\StringTok{"Two is bigger than one"}
\end{Highlighting}
\end{Shaded}

Take the \texttt{df} function from Chapter 6 of the Lecture Notes and modify it by adding some assertions to check that:

\begin{itemize}
\tightlist
\item
  the pillar date list contains at least 2 elements;
\item
  the pillar date list is the same length as the discount factor list;
\item
  the first pillar date is equal to the today date;
\item
  the value date argument `d' is greater or equal to the first pillar
  date and also less than or equal to the last pillar date.
\end{itemize}

Then try using the function with some invalid data to make sure that your as sertions are correctly checking the desired conditions
\end{Exercise}

\begin{Answer}
\begin{tcolorbox}[size=fbox, boxrule=1pt, colback=cellbackground, colframe=cellborder]
\begin{Verbatim}[commandchars=\\\{\}]
\PY{c+c1}{\PYZsh{} import modules and objects that we need}
\PY{k+kn}{from} \PY{n+nn}{datetime} \PY{k}{import} \PY{n}{date}
\PY{k+kn}{import} \PY{n+nn}{numpy}
\PY{k+kn}{import} \PY{n+nn}{math}

\PY{n}{today\PYZus{}date} \PY{o}{=} \PY{n}{date}\PY{p}{(}\PY{l+m+mi}{2017}\PY{p}{,} \PY{l+m+mi}{10}\PY{p}{,} \PY{l+m+mi}{1}\PY{p}{)}
\PY{n}{pillar\PYZus{}dates} \PY{o}{=} \PY{p}{[}\PY{n}{date}\PY{p}{(}\PY{l+m+mi}{2017}\PY{p}{,} \PY{l+m+mi}{10}\PY{p}{,} \PY{l+m+mi}{1}\PY{p}{)}\PY{p}{,} 
                \PY{n}{date}\PY{p}{(}\PY{l+m+mi}{2018}\PY{p}{,} \PY{l+m+mi}{10}\PY{p}{,} \PY{l+m+mi}{1}\PY{p}{)}\PY{p}{,} 
                \PY{n}{date}\PY{p}{(}\PY{l+m+mi}{2019}\PY{p}{,} \PY{l+m+mi}{10}\PY{p}{,} \PY{l+m+mi}{1}\PY{p}{)}\PY{p}{]}
\PY{n}{discount\PYZus{}factors} \PY{o}{=} \PY{p}{[}\PY{l+m+mf}{1.0}\PY{p}{,} \PY{l+m+mf}{0.95}\PY{p}{,} \PY{l+m+mf}{0.8}\PY{p}{]}

\PY{k}{def} \PY{n+nf}{df}\PY{p}{(}\PY{n}{t, pillar_dates, discount_factors}\PY{p}{)}\PY{p}{:}
    \PY{c+c1}{\PYZsh{}\PYZsh{}\PYZsh{}\PYZsh{}\PYZsh{}\PYZsh{}\PYZsh{}\PYZsh{}\PYZsh{}\PYZsh{}\PYZsh{}\PYZsh{}\PYZsh{}\PYZsh{} CHECKS \PYZsh{}\PYZsh{}\PYZsh{}\PYZsh{}\PYZsh{}\PYZsh{}\PYZsh{}\PYZsh{}\PYZsh{}\PYZsh{}\PYZsh{}\PYZsh{}\PYZsh{}\PYZsh{}\PYZsh{}\PYZsh{}}
    \PY{k}{assert} \PY{n+nb}{len}\PY{p}{(}\PY{n}{pillar\PYZus{}dates}\PY{p}{)} \PY{o}{\PYZgt{}}\PY{o}{=} \PY{l+m+mi}{2}\PY{p}{,} \PY{l+s+s2}{\PYZdq{}}\PY{l+s+s2}{ need at least 2 pillar dates}\PY{l+s+s2}{\PYZdq{}}
    
    \PY{k}{assert} \PY{n+nb}{len}\PY{p}{(}\PY{n}{pillar\PYZus{}dates}\PY{p}{)} \PY{o}{==} \PY{n+nb}{len}\PY{p}{(}\PY{n}{discount\PYZus{}factors}\PY{p}{)}\PY{p}{,} \PYZbs{}
        \PY{l+s+s2}{\PYZdq{}}\PY{l+s+s2}{number of pillar dates should be equal to }\PY{l+s+se}{\PYZbs{}}
\PY{l+s+s2}{        the number of pillar discount factors}\PY{l+s+s2}{\PYZdq{}}
    
    \PY{k}{assert} \PY{n}{today\PYZus{}date} \PY{o}{==} \PY{n}{pillar\PYZus{}dates}\PY{p}{[}\PY{l+m+mi}{0}\PY{p}{]}\PY{p}{,} \PYZbs{}
        \PY{l+s+s2}{\PYZdq{}}\PY{l+s+s2}{first pillar date should be the today date}\PY{l+s+s2}{\PYZdq{}}
    
    \PY{k}{assert} \PY{n}{pillar\PYZus{}dates}\PY{p}{[}\PY{l+m+mi}{0}\PY{p}{]} \PY{o}{\PYZlt{}}\PY{o}{=} \PY{n}{t} \PY{o}{\PYZlt{}}\PY{o}{=} \PY{n}{pillar\PYZus{}dates}\PY{p}{[}\PY{o}{\PYZhy{}}\PY{l+m+mi}{1}\PY{p}{]}\PY{p}{,} \PYZbs{}
        \PY{l+s+s2}{\PYZdq{}}\PY{l+s+s2}{Invalid value date }\PY{l+s+si}{\PYZpc{}s}\PY{l+s+s2}{\PYZdq{}} \PY{o}{\PYZpc{}} \PY{p}{(}\PY{n}{d}\PY{p}{)}
    \PY{c+c1}{\PYZsh{}\PYZsh{}\PYZsh{}\PYZsh{}\PYZsh{}\PYZsh{}\PYZsh{}\PYZsh{}\PYZsh{}\PYZsh{}\PYZsh{}\PYZsh{}\PYZsh{}\PYZsh{} END OF CHECKS \PYZsh{}\PYZsh{}\PYZsh{}\PYZsh{}\PYZsh{}\PYZsh{}\PYZsh{}\PYZsh{}\PYZsh{}\PYZsh{}\PYZsh{}\PYZsh{}\PYZsh{}\PYZsh{}\PYZsh{}\PYZsh{}}
    
    \PY{n}{log\PYZus{}discount\PYZus{}factors} \PY{o}{=} \PY{p}{[}\PY{p}{]}
    \PY{k}{for} \PY{n}{discount\PYZus{}factor} \PY{o+ow}{in} \PY{n}{discount\PYZus{}factors}\PY{p}{:}
        \PY{n}{log\PYZus{}discount\PYZus{}factors}\PY{o}{.}\PY{n}{append}\PY{p}{(}\PY{n}{math}\PY{o}{.}\PY{n}{log}\PY{p}{(}\PY{n}{discount\PYZus{}factor}\PY{p}{)}\PY{p}{)}
    
    \PY{n}{pillar\PYZus{}days} \PY{o}{=} \PY{p}{[}\PY{p}{]}
    \PY{k}{for} \PY{n}{pillar\PYZus{}date} \PY{o+ow}{in} \PY{n}{pillar\PYZus{}dates}\PY{p}{:}
        \PY{n}{pillar\PYZus{}days}\PY{o}{.}\PY{n}{append}\PY{p}{(}\PY{p}{(}\PY{n}{pillar\PYZus{}date} \PY{o}{\PYZhy{}} \PY{n}{today\PYZus{}date}\PY{p}{)}\PY{o}{.}\PY{n}{days}\PY{p}{)}
    
    \PY{n}{t\PYZus{}days} \PY{o}{=} \PY{p}{(}\PY{n}{t} \PY{o}{\PYZhy{}} \PY{n}{today\PYZus{}date}\PY{p}{)}\PY{o}{.}\PY{n}{days}
    
    \PY{n}{interpolated\PYZus{}log\PYZus{}discount\PYZus{}factor} \PY{o}{=} \PYZbs{}
        \PY{n}{numpy}\PY{o}{.}\PY{n}{interp}\PY{p}{(}\PY{n}{t\PYZus{}days}\PY{p}{,} \PY{n}{pillar\PYZus{}days}\PY{p}{,} \PY{n}{log\PYZus{}discount\PYZus{}factors}\PY{p}{)}
    
    \PY{k}{return} \PY{n}{math}\PY{o}{.}\PY{n}{exp}\PY{p}{(}\PY{n}{interpolated\PYZus{}log\PYZus{}discount\PYZus{}factor}\PY{p}{)}

\PY{n}{df}\PY{p}{(}\PY{n}{date}\PY{p}{(}\PY{l+m+mi}{2019}\PY{p}{,} \PY{l+m+mi}{1}\PY{p}{,} \PY{l+m+mi}{1}\PY{p}{), pillar_dates, discount_factors}\PY{p}{)}

0.9097285910181567
\end{Verbatim}
\end{tcolorbox} 
\end{Answer}

\begin{Exercise}[label={ex:BS3}]
Copy into the file \texttt{finmarkets.py} the function used to compute Black Scholes formula used in Exercise~\ref{ex:BS2}. This is the implementation of our financial library. Repeat Exercise~\ref{ex:BS2} now using the version of the Black and Scholes formula in the finmarket module.
\end{Exercise}

\begin{Answer}
\begin{tcolorbox}[size=fbox, boxrule=1pt, colback=cellbackground, colframe=cellborder]
\begin{Verbatim}[commandchars=\\\{\}]
\PY{k}{import} \PY{n}{finmarket}
        
\PY{n}{s} \PY{o}{=} \PY{l+m+mi}{800}
\PY{c+c1}{\PYZsh{} strikes expressed as \PYZpc{} of spot price}
\PY{n}{moneyness} \PY{o}{=} \PY{p}{[} \PY{l+m+mf}{0.5}\PY{p}{,} \PY{l+m+mf}{0.75}\PY{p}{,} \PY{l+m+mf}{0.825}\PY{p}{,} \PYZbs{}
             \PY{l+m+mf}{1.0}\PY{p}{,} \PY{l+m+mf}{1.125}\PY{p}{,} \PY{l+m+mf}{1.25}\PY{p}{,} \PY{l+m+mf}{1.5} \PY{p}{]}
\PY{n}{vol} \PY{o}{=} \PY{l+m+mf}{0.3}
\PY{n}{ttm} \PY{o}{=} \PY{l+m+mf}{0.75}
\PY{n}{r} \PY{o}{=} \PY{l+m+mf}{0.005}

\PY{n}{result} \PY{o}{=} \PY{p}{\PYZob{}}\PY{p}{\PYZcb{}}
\PY{k}{for} \PY{n}{m} \PY{o+ow}{in} \PY{n}{moneyness}\PY{p}{:}
    \PY{n}{result}\PY{p}{[}\PY{n}{s}\PY{o}{*}\PY{n}{m}\PY{p}{]} \PY{o}{=} \PY{n}{call}\PY{p}{(}\PY{n}{s}\PY{p}{,} \PY{n}{m}\PY{o}{*}\PY{n}{s}\PY{p}{,} \PY{n}{r}\PY{p}{,} \PY{n}{vol}\PY{p}{,} \PY{n}{ttm}\PY{p}{)}
\PY{n}{result}

\{400.0: 401.66074527896365,
  600.0: 213.9883852521275,
  660.0: 166.85957363897393,
  800.0: 84.03697017660357,
  900.0: 47.61880394696229,
  1000.0: 25.632722952585738,
  1200.0: 6.655275227771156\}
\end{Verbatim}
\end{tcolorbox}
\end{Answer}

\begin{Exercise}
  Following the steps outlined in Chapter 6 of the Lecture Notes, implement a \texttt{DiscountCurve} class and add it to \texttt{finmarket} module. The class should have as attributes the pillar dates and the corresponding discount factors and two methods, one to interpolate discount factors and another to calculate forward rates.
  Finally using that class compute the forward 6m libor coupon using dummy curves in pre and post 2008 crisis way.
\end{Exercise}

\begin{Answer}
\begin{tcolorbox}[size=fbox, boxrule=1pt, pad at break*=1mm,colback=cellbackground, colframe=cellborder]
\begin{Verbatim}[commandchars=\\\{\}]
\PY{k+kn}{import} \PY{n+nn}{math}
\PY{k+kn}{import} \PY{n+nn}{numpy}
\PY{k+kn}{from} \PY{n+nn}{datetime} \PY{k}{import} \PY{n}{date}

\PY{k}{class} \PY{n+nc}{DiscountCurve}\PY{p}{:}

    \PY{k}{def} \PY{n+nf}{\PYZus{}\PYZus{}init\PYZus{}\PYZus{}}\PY{p}{(}\PY{n+nb+bp}{self}\PY{p}{,} \PY{n}{today}\PY{p}{,} \PY{n}{pillar\PYZus{}dates}\PY{p}{,} \PY{n}{discount\PYZus{}factors}\PY{p}{)}\PY{p}{:}
        \PY{n+nb+bp}{self}\PY{o}{.}\PY{n}{today} \PY{o}{=} \PY{n}{today}
        \PY{n+nb+bp}{self}\PY{o}{.}\PY{n}{pillar\PYZus{}dates} \PY{o}{=} \PY{n}{pillar\PYZus{}dates}
        \PY{n+nb+bp}{self}\PY{o}{.}\PY{n}{discount\PYZus{}factors} \PY{o}{=} \PY{n}{discount\PYZus{}factors}

    \PY{k}{def} \PY{n+nf}{df}\PY{p}{(}\PY{n+nb+bp}{self}\PY{p}{,} \PY{n}{d}\PY{p}{)}\PY{p}{:}
        \PY{n}{log\PYZus{}discount\PYZus{}factors} \PY{o}{=} \PYZbs{}
          \PY{p}{[}\PY{n}{math}\PY{o}{.}\PY{n}{log}\PY{p}{(}\PY{n}{discount\PYZus{}factor}\PY{p}{)} 
           \PY{k}{for} \PY{n}{discount\PYZus{}factor} \PY{o+ow}{in} \PY{n+nb+bp}{self}\PY{o}{.}\PY{n}{discount\PYZus{}factors}\PY{p}{]}
        \PY{n}{pillar\PYZus{}days} \PY{o}{=} \PY{p}{[}\PY{p}{(}\PY{n}{pillar\PYZus{}date} \PY{o}{\PYZhy{}} \PY{n+nb+bp}{self}\PY{o}{.}\PY{n}{today}\PY{p}{)}\PY{o}{.}\PY{n}{days} 
                       \PY{k}{for} \PY{n}{pillar\PYZus{}date} \PY{o+ow}{in} \PY{n+nb+bp}{self}\PY{o}{.}\PY{n}{pillar\PYZus{}dates}\PY{p}{]}
        \PY{n}{d\PYZus{}days} \PY{o}{=} \PY{p}{(}\PY{n}{d} \PY{o}{\PYZhy{}} \PY{n+nb+bp}{self}\PY{o}{.}\PY{n}{today}\PY{p}{)}\PY{o}{.}\PY{n}{days}
        \PY{n}{interpolated\PYZus{}log\PYZus{}discount\PYZus{}factor} \PY{o}{=} \PYZbs{}
            \PY{n}{numpy}\PY{o}{.}\PY{n}{interp}\PY{p}{(}\PY{n}{d\PYZus{}days}\PY{p}{,} \PY{n}{pillar\PYZus{}days}\PY{p}{,} \PY{n}{log\PYZus{}discount\PYZus{}factors}\PY{p}{)}
        \PY{k}{return} \PY{n}{math}\PY{o}{.}\PY{n}{exp}\PY{p}{(}\PY{n}{interpolated\PYZus{}log\PYZus{}discount\PYZus{}factor}\PY{p}{)}

    \PY{k}{def} \PY{n+nf}{forward\PYZus{}rate}\PY{p}{(}\PY{n+nb+bp}{self}\PY{p}{,} \PY{n}{d1}\PY{p}{,} \PY{n}{d2}\PY{p}{)}\PY{p}{:}
        \PY{k}{return} \PY{p}{(}\PY{n+nb+bp}{self}\PY{o}{.}\PY{n}{df}\PY{p}{(}\PY{n}{d1}\PY{p}{)} \PY{o}{/} \PY{n+nb+bp}{self}\PY{o}{.}\PY{n}{df}\PY{p}{(}\PY{n}{d2}\PY{p}{)} \PY{o}{\PYZhy{}} \PY{l+m+mf}{1.0}\PY{p}{)} \PY{o}{*} \PYZbs{}
                \PY{p}{(}\PY{l+m+mf}{365.0} \PY{o}{/} \PY{p}{(}\PY{p}{(}\PY{n}{d2} \PY{o}{\PYZhy{}} \PY{n}{d1}\PY{p}{)}\PY{o}{.}\PY{n}{days}\PY{p}{)}\PY{p}{)}

\PY{n}{npv} \PY{o}{=} \PY{n}{eonia\PYZus{}curve}\PY{o}{.}\PY{n}{df}\PY{p}{(}\PY{n}{date}\PY{p}{(}\PY{l+m+mi}{2020}\PY{p}{,} \PY{l+m+mi}{4}\PY{p}{,} \PY{l+m+mi}{1}\PY{p}{)}\PY{p}{)} \PY{o}{*} \PYZbs{}
      \PY{n}{libor\PYZus{}curve}\PY{o}{.}\PY{n}{forward\PYZus{}rate}\PY{p}{(}\PY{n}{date}\PY{p}{(}\PY{l+m+mi}{2020}\PY{p}{,}\PY{l+m+mi}{4}\PY{p}{,} \PY{l+m+mi}{1}\PY{p}{)}\PY{p}{,} 
                                \PY{n}{date}\PY{p}{(}\PY{l+m+mi}{2020}\PY{p}{,} \PY{l+m+mi}{10}\PY{p}{,} \PY{l+m+mi}{1}\PY{p}{)}\PY{p}{)}

\PY{c+c1}{\PYZsh{} Compute it in the pre\PYZhy{}2008 way}
\PY{n}{npv\PYZus{}pre\PYZus{}2008} \PY{o}{=} \PY{n}{libor\PYZus{}curve}\PY{o}{.}\PY{n}{df}\PY{p}{(}\PY{n}{date}\PY{p}{(}\PY{l+m+mi}{2020}\PY{p}{,} \PY{l+m+mi}{4}\PY{p}{,} \PY{l+m+mi}{1}\PY{p}{)}\PY{p}{)} \PY{o}{*} \PYZbs{}
               \PY{n}{libor\PYZus{}curve}\PY{o}{.}\PY{n}{forward\PYZus{}rate}\PY{p}{(}\PY{n}{date}\PY{p}{(}\PY{l+m+mi}{2020}\PY{p}{,} \PY{l+m+mi}{4}\PY{p}{,} \PY{l+m+mi}{1}\PY{p}{)}\PY{p}{,} 
                                         \PY{n}{date}\PY{p}{(}\PY{l+m+mi}{2020}\PY{p}{,} \PY{l+m+mi}{10}\PY{p}{,} \PY{l+m+mi}{1}\PY{p}{)}\PY{p}{)}
\PY{n+nb}{print} \PY{p}{(}\PY{n}{npv}\PY{p}{)}
\PY{n+nb}{print} \PY{p}{(}\PY{n}{npv\PYZus{}pre\PYZus{}2008}\PY{p}{)}

0.37932346377238657
0.38139410902305737
\end{Verbatim}
\end{tcolorbox}
\end{Answer}

\begin{Exercise}[title={(Forward rate curve class)}]
Write a ForwardRateCurve class (for EURIBOR/LIBOR rate curve) which
doesn't compute discount factors but only interplatates forward rates;
then add it to the \texttt{finmarkets} module (this function is used to
define the LIBOR curve needed throughout future lessons).
\end{Exercise}

\begin{Answer}
In this case it is enough to write a new \texttt{class} that has three
attributes: a today date, a set of pillar\_dates and the corresponding
rates. There will be just a single method \texttt{forward\_rate} which
returns the corresponding interpolated rate.

\begin{tcolorbox}[size=fbox, boxrule=1pt, colback=cellbackground, colframe=cellborder]
\begin{Verbatim}[commandchars=\\\{\}]
\PY{k+kn}{import} \PY{n+nn}{numpy}
        
\PY{c+c1}{\PYZsh{} an EURIBOR or LIBOR rate curve}
\PY{c+c1}{\PYZsh{} doesn\PYZsq{}t calculate discount factors, only interpolates forward rates}
\PY{k}{class} \PY{n+nc}{ForwardRateCurve}\PY{p}{(}\PY{n+nb}{object}\PY{p}{)}\PY{p}{:}
   
   \PY{c+c1}{\PYZsh{} the special \PYZus{}\PYZus{}init\PYZus{}\PYZus{} method defines how to}
   \PY{c+c1}{\PYZsh{} construct instances of the class}
   \PY{k}{def} \PY{n+nf}{\PYZus{}\PYZus{}init\PYZus{}\PYZus{}}\PY{p}{(}\PY{n+nb+bp}{self}\PY{p}{,} \PY{n}{pillar\PYZus{}dates}\PY{p}{,} \PY{n}{rates}\PY{p}{)}\PY{p}{:}
       
       \PY{c+c1}{\PYZsh{} we just store the arguments as attributes of the instance}
       \PY{n+nb+bp}{self}\PY{o}{.}\PY{n}{today} \PY{o}{=} \PY{n}{pillar\PYZus{}dates}\PY{p}{[}\PY{l+m+mi}{0}\PY{p}{]}
       \PY{n+nb+bp}{self}\PY{o}{.}\PY{n}{rates} \PY{o}{=} \PY{n}{rates}
       
       \PY{n+nb+bp}{self}\PY{o}{.}\PY{n}{pillar\PYZus{}days} \PY{o}{=} \PY{p}{[}
           \PY{p}{(}\PY{n}{pillar\PYZus{}date} \PY{o}{\PYZhy{}} \PY{n+nb+bp}{self}\PY{o}{.}\PY{n}{today}\PY{p}{)}\PY{o}{.}\PY{n}{days}
           \PY{k}{for} \PY{n}{pillar\PYZus{}date} \PY{o+ow}{in} \PY{n}{pillar\PYZus{}dates}
       \PY{p}{]}
       
       
   \PY{c+c1}{\PYZsh{} interpolates the forward rates stored in the instance}
   \PY{k}{def} \PY{n+nf}{forward\PYZus{}rate}\PY{p}{(}\PY{n+nb+bp}{self}\PY{p}{,} \PY{n}{d}\PY{p}{)}\PY{p}{:}
       \PY{n}{d\PYZus{}days} \PY{o}{=} \PY{p}{(}\PY{n}{d} \PY{o}{\PYZhy{}} \PY{n+nb+bp}{self}\PY{o}{.}\PY{n}{today}\PY{p}{)}\PY{o}{.}\PY{n}{days}
       \PY{k}{return} \PY{n}{numpy}\PY{o}{.}\PY{n}{interp}\PY{p}{(}\PY{n}{d\PYZus{}days}\PY{p}{,} \PY{n+nb+bp}{self}\PY{o}{.}\PY{n}{pillar\PYZus{}days}\PY{p}{,} \PY{n+nb+bp}{self}\PY{o}{.}\PY{n}{rates}\PY{p}{)}
\end{Verbatim}
\end{tcolorbox}
\end{Answer}


  

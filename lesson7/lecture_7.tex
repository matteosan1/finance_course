\documentclass[11pt]{article}

    \usepackage[breakable]{tcolorbox}
    \usepackage{parskip} % Stop auto-indenting (to mimic markdown behaviour)
    
    \usepackage{iftex}
    \ifPDFTeX
    	\usepackage[T1]{fontenc}
    	\usepackage{mathpazo}
    \else
    	\usepackage{fontspec}
    \fi

    % Basic figure setup, for now with no caption control since it's done
    % automatically by Pandoc (which extracts ![](path) syntax from Markdown).
    \usepackage{graphicx}
    % Maintain compatibility with old templates. Remove in nbconvert 6.0
    \let\Oldincludegraphics\includegraphics
    % Ensure that by default, figures have no caption (until we provide a
    % proper Figure object with a Caption API and a way to capture that
    % in the conversion process - todo).
    \usepackage{caption}
    \DeclareCaptionFormat{nocaption}{}
    \captionsetup{format=nocaption,aboveskip=0pt,belowskip=0pt}

    \usepackage[Export]{adjustbox} % Used to constrain images to a maximum size
    \adjustboxset{max size={0.9\linewidth}{0.9\paperheight}}
    \usepackage{float}
    \floatplacement{figure}{H} % forces figures to be placed at the correct location
    \usepackage{xcolor} % Allow colors to be defined
    \usepackage{enumerate} % Needed for markdown enumerations to work
    \usepackage{geometry} % Used to adjust the document margins
    \usepackage{amsmath} % Equations
    \usepackage{amssymb} % Equations
    \usepackage{textcomp} % defines textquotesingle
    % Hack from http://tex.stackexchange.com/a/47451/13684:
    \AtBeginDocument{%
        \def\PYZsq{\textquotesingle}% Upright quotes in Pygmentized code
    }
    \usepackage{upquote} % Upright quotes for verbatim code
    \usepackage{eurosym} % defines \euro
    \usepackage[mathletters]{ucs} % Extended unicode (utf-8) support
    \usepackage{fancyvrb} % verbatim replacement that allows latex
    \usepackage{grffile} % extends the file name processing of package graphics 
                         % to support a larger range
    \makeatletter % fix for grffile with XeLaTeX
    \def\Gread@@xetex#1{%
      \IfFileExists{"\Gin@base".bb}%
      {\Gread@eps{\Gin@base.bb}}%
      {\Gread@@xetex@aux#1}%
    }
    \makeatother

    % The hyperref package gives us a pdf with properly built
    % internal navigation ('pdf bookmarks' for the table of contents,
    % internal cross-reference links, web links for URLs, etc.)
    \usepackage{hyperref}
    % The default LaTeX title has an obnoxious amount of whitespace. By default,
    % titling removes some of it. It also provides customization options.
    \usepackage{titling}
    \usepackage{longtable} % longtable support required by pandoc >1.10
    \usepackage{booktabs}  % table support for pandoc > 1.12.2
    \usepackage[inline]{enumitem} % IRkernel/repr support (it uses the enumerate* environment)
    \usepackage[normalem]{ulem} % ulem is needed to support strikethroughs (\sout)
                                % normalem makes italics be italics, not underlines
    \usepackage{mathrsfs}
    

    
    % Colors for the hyperref package
    \definecolor{urlcolor}{rgb}{0,.145,.698}
    \definecolor{linkcolor}{rgb}{.71,0.21,0.01}
    \definecolor{citecolor}{rgb}{.12,.54,.11}

    % ANSI colors
    \definecolor{ansi-black}{HTML}{3E424D}
    \definecolor{ansi-black-intense}{HTML}{282C36}
    \definecolor{ansi-red}{HTML}{E75C58}
    \definecolor{ansi-red-intense}{HTML}{B22B31}
    \definecolor{ansi-green}{HTML}{00A250}
    \definecolor{ansi-green-intense}{HTML}{007427}
    \definecolor{ansi-yellow}{HTML}{DDB62B}
    \definecolor{ansi-yellow-intense}{HTML}{B27D12}
    \definecolor{ansi-blue}{HTML}{208FFB}
    \definecolor{ansi-blue-intense}{HTML}{0065CA}
    \definecolor{ansi-magenta}{HTML}{D160C4}
    \definecolor{ansi-magenta-intense}{HTML}{A03196}
    \definecolor{ansi-cyan}{HTML}{60C6C8}
    \definecolor{ansi-cyan-intense}{HTML}{258F8F}
    \definecolor{ansi-white}{HTML}{C5C1B4}
    \definecolor{ansi-white-intense}{HTML}{A1A6B2}
    \definecolor{ansi-default-inverse-fg}{HTML}{FFFFFF}
    \definecolor{ansi-default-inverse-bg}{HTML}{000000}

    % commands and environments needed by pandoc snippets
    % extracted from the output of `pandoc -s`
    \providecommand{\tightlist}{%
      \setlength{\itemsep}{0pt}\setlength{\parskip}{0pt}}
    \DefineVerbatimEnvironment{Highlighting}{Verbatim}{commandchars=\\\{\}}
    % Add ',fontsize=\small' for more characters per line
    \newenvironment{Shaded}{}{}
    \newcommand{\KeywordTok}[1]{\textcolor[rgb]{0.00,0.44,0.13}{\textbf{{#1}}}}
    \newcommand{\DataTypeTok}[1]{\textcolor[rgb]{0.56,0.13,0.00}{{#1}}}
    \newcommand{\DecValTok}[1]{\textcolor[rgb]{0.25,0.63,0.44}{{#1}}}
    \newcommand{\BaseNTok}[1]{\textcolor[rgb]{0.25,0.63,0.44}{{#1}}}
    \newcommand{\FloatTok}[1]{\textcolor[rgb]{0.25,0.63,0.44}{{#1}}}
    \newcommand{\CharTok}[1]{\textcolor[rgb]{0.25,0.44,0.63}{{#1}}}
    \newcommand{\StringTok}[1]{\textcolor[rgb]{0.25,0.44,0.63}{{#1}}}
    \newcommand{\CommentTok}[1]{\textcolor[rgb]{0.38,0.63,0.69}{\textit{{#1}}}}
    \newcommand{\OtherTok}[1]{\textcolor[rgb]{0.00,0.44,0.13}{{#1}}}
    \newcommand{\AlertTok}[1]{\textcolor[rgb]{1.00,0.00,0.00}{\textbf{{#1}}}}
    \newcommand{\FunctionTok}[1]{\textcolor[rgb]{0.02,0.16,0.49}{{#1}}}
    \newcommand{\RegionMarkerTok}[1]{{#1}}
    \newcommand{\ErrorTok}[1]{\textcolor[rgb]{1.00,0.00,0.00}{\textbf{{#1}}}}
    \newcommand{\NormalTok}[1]{{#1}}
    
    % Additional commands for more recent versions of Pandoc
    \newcommand{\ConstantTok}[1]{\textcolor[rgb]{0.53,0.00,0.00}{{#1}}}
    \newcommand{\SpecialCharTok}[1]{\textcolor[rgb]{0.25,0.44,0.63}{{#1}}}
    \newcommand{\VerbatimStringTok}[1]{\textcolor[rgb]{0.25,0.44,0.63}{{#1}}}
    \newcommand{\SpecialStringTok}[1]{\textcolor[rgb]{0.73,0.40,0.53}{{#1}}}
    \newcommand{\ImportTok}[1]{{#1}}
    \newcommand{\DocumentationTok}[1]{\textcolor[rgb]{0.73,0.13,0.13}{\textit{{#1}}}}
    \newcommand{\AnnotationTok}[1]{\textcolor[rgb]{0.38,0.63,0.69}{\textbf{\textit{{#1}}}}}
    \newcommand{\CommentVarTok}[1]{\textcolor[rgb]{0.38,0.63,0.69}{\textbf{\textit{{#1}}}}}
    \newcommand{\VariableTok}[1]{\textcolor[rgb]{0.10,0.09,0.49}{{#1}}}
    \newcommand{\ControlFlowTok}[1]{\textcolor[rgb]{0.00,0.44,0.13}{\textbf{{#1}}}}
    \newcommand{\OperatorTok}[1]{\textcolor[rgb]{0.40,0.40,0.40}{{#1}}}
    \newcommand{\BuiltInTok}[1]{{#1}}
    \newcommand{\ExtensionTok}[1]{{#1}}
    \newcommand{\PreprocessorTok}[1]{\textcolor[rgb]{0.74,0.48,0.00}{{#1}}}
    \newcommand{\AttributeTok}[1]{\textcolor[rgb]{0.49,0.56,0.16}{{#1}}}
    \newcommand{\InformationTok}[1]{\textcolor[rgb]{0.38,0.63,0.69}{\textbf{\textit{{#1}}}}}
    \newcommand{\WarningTok}[1]{\textcolor[rgb]{0.38,0.63,0.69}{\textbf{\textit{{#1}}}}}
    
    
    % Define a nice break command that doesn't care if a line doesn't already
    % exist.
    \def\br{\hspace*{\fill} \\* }
    % Math Jax compatibility definitions
    \def\gt{>}
    \def\lt{<}
    \let\Oldtex\TeX
    \let\Oldlatex\LaTeX
    \renewcommand{\TeX}{\textrm{\Oldtex}}
    \renewcommand{\LaTeX}{\textrm{\Oldlatex}}
    % Document parameters
    % Document title
    \title{lecture\_7}
    
    
    
    
    
% Pygments definitions
\makeatletter
\def\PY@reset{\let\PY@it=\relax \let\PY@bf=\relax%
    \let\PY@ul=\relax \let\PY@tc=\relax%
    \let\PY@bc=\relax \let\PY@ff=\relax}
\def\PY@tok#1{\csname PY@tok@#1\endcsname}
\def\PY@toks#1+{\ifx\relax#1\empty\else%
    \PY@tok{#1}\expandafter\PY@toks\fi}
\def\PY@do#1{\PY@bc{\PY@tc{\PY@ul{%
    \PY@it{\PY@bf{\PY@ff{#1}}}}}}}
\def\PY#1#2{\PY@reset\PY@toks#1+\relax+\PY@do{#2}}

\expandafter\def\csname PY@tok@w\endcsname{\def\PY@tc##1{\textcolor[rgb]{0.73,0.73,0.73}{##1}}}
\expandafter\def\csname PY@tok@c\endcsname{\let\PY@it=\textit\def\PY@tc##1{\textcolor[rgb]{0.25,0.50,0.50}{##1}}}
\expandafter\def\csname PY@tok@cp\endcsname{\def\PY@tc##1{\textcolor[rgb]{0.74,0.48,0.00}{##1}}}
\expandafter\def\csname PY@tok@k\endcsname{\let\PY@bf=\textbf\def\PY@tc##1{\textcolor[rgb]{0.00,0.50,0.00}{##1}}}
\expandafter\def\csname PY@tok@kp\endcsname{\def\PY@tc##1{\textcolor[rgb]{0.00,0.50,0.00}{##1}}}
\expandafter\def\csname PY@tok@kt\endcsname{\def\PY@tc##1{\textcolor[rgb]{0.69,0.00,0.25}{##1}}}
\expandafter\def\csname PY@tok@o\endcsname{\def\PY@tc##1{\textcolor[rgb]{0.40,0.40,0.40}{##1}}}
\expandafter\def\csname PY@tok@ow\endcsname{\let\PY@bf=\textbf\def\PY@tc##1{\textcolor[rgb]{0.67,0.13,1.00}{##1}}}
\expandafter\def\csname PY@tok@nb\endcsname{\def\PY@tc##1{\textcolor[rgb]{0.00,0.50,0.00}{##1}}}
\expandafter\def\csname PY@tok@nf\endcsname{\def\PY@tc##1{\textcolor[rgb]{0.00,0.00,1.00}{##1}}}
\expandafter\def\csname PY@tok@nc\endcsname{\let\PY@bf=\textbf\def\PY@tc##1{\textcolor[rgb]{0.00,0.00,1.00}{##1}}}
\expandafter\def\csname PY@tok@nn\endcsname{\let\PY@bf=\textbf\def\PY@tc##1{\textcolor[rgb]{0.00,0.00,1.00}{##1}}}
\expandafter\def\csname PY@tok@ne\endcsname{\let\PY@bf=\textbf\def\PY@tc##1{\textcolor[rgb]{0.82,0.25,0.23}{##1}}}
\expandafter\def\csname PY@tok@nv\endcsname{\def\PY@tc##1{\textcolor[rgb]{0.10,0.09,0.49}{##1}}}
\expandafter\def\csname PY@tok@no\endcsname{\def\PY@tc##1{\textcolor[rgb]{0.53,0.00,0.00}{##1}}}
\expandafter\def\csname PY@tok@nl\endcsname{\def\PY@tc##1{\textcolor[rgb]{0.63,0.63,0.00}{##1}}}
\expandafter\def\csname PY@tok@ni\endcsname{\let\PY@bf=\textbf\def\PY@tc##1{\textcolor[rgb]{0.60,0.60,0.60}{##1}}}
\expandafter\def\csname PY@tok@na\endcsname{\def\PY@tc##1{\textcolor[rgb]{0.49,0.56,0.16}{##1}}}
\expandafter\def\csname PY@tok@nt\endcsname{\let\PY@bf=\textbf\def\PY@tc##1{\textcolor[rgb]{0.00,0.50,0.00}{##1}}}
\expandafter\def\csname PY@tok@nd\endcsname{\def\PY@tc##1{\textcolor[rgb]{0.67,0.13,1.00}{##1}}}
\expandafter\def\csname PY@tok@s\endcsname{\def\PY@tc##1{\textcolor[rgb]{0.73,0.13,0.13}{##1}}}
\expandafter\def\csname PY@tok@sd\endcsname{\let\PY@it=\textit\def\PY@tc##1{\textcolor[rgb]{0.73,0.13,0.13}{##1}}}
\expandafter\def\csname PY@tok@si\endcsname{\let\PY@bf=\textbf\def\PY@tc##1{\textcolor[rgb]{0.73,0.40,0.53}{##1}}}
\expandafter\def\csname PY@tok@se\endcsname{\let\PY@bf=\textbf\def\PY@tc##1{\textcolor[rgb]{0.73,0.40,0.13}{##1}}}
\expandafter\def\csname PY@tok@sr\endcsname{\def\PY@tc##1{\textcolor[rgb]{0.73,0.40,0.53}{##1}}}
\expandafter\def\csname PY@tok@ss\endcsname{\def\PY@tc##1{\textcolor[rgb]{0.10,0.09,0.49}{##1}}}
\expandafter\def\csname PY@tok@sx\endcsname{\def\PY@tc##1{\textcolor[rgb]{0.00,0.50,0.00}{##1}}}
\expandafter\def\csname PY@tok@m\endcsname{\def\PY@tc##1{\textcolor[rgb]{0.40,0.40,0.40}{##1}}}
\expandafter\def\csname PY@tok@gh\endcsname{\let\PY@bf=\textbf\def\PY@tc##1{\textcolor[rgb]{0.00,0.00,0.50}{##1}}}
\expandafter\def\csname PY@tok@gu\endcsname{\let\PY@bf=\textbf\def\PY@tc##1{\textcolor[rgb]{0.50,0.00,0.50}{##1}}}
\expandafter\def\csname PY@tok@gd\endcsname{\def\PY@tc##1{\textcolor[rgb]{0.63,0.00,0.00}{##1}}}
\expandafter\def\csname PY@tok@gi\endcsname{\def\PY@tc##1{\textcolor[rgb]{0.00,0.63,0.00}{##1}}}
\expandafter\def\csname PY@tok@gr\endcsname{\def\PY@tc##1{\textcolor[rgb]{1.00,0.00,0.00}{##1}}}
\expandafter\def\csname PY@tok@ge\endcsname{\let\PY@it=\textit}
\expandafter\def\csname PY@tok@gs\endcsname{\let\PY@bf=\textbf}
\expandafter\def\csname PY@tok@gp\endcsname{\let\PY@bf=\textbf\def\PY@tc##1{\textcolor[rgb]{0.00,0.00,0.50}{##1}}}
\expandafter\def\csname PY@tok@go\endcsname{\def\PY@tc##1{\textcolor[rgb]{0.53,0.53,0.53}{##1}}}
\expandafter\def\csname PY@tok@gt\endcsname{\def\PY@tc##1{\textcolor[rgb]{0.00,0.27,0.87}{##1}}}
\expandafter\def\csname PY@tok@err\endcsname{\def\PY@bc##1{\setlength{\fboxsep}{0pt}\fcolorbox[rgb]{1.00,0.00,0.00}{1,1,1}{\strut ##1}}}
\expandafter\def\csname PY@tok@kc\endcsname{\let\PY@bf=\textbf\def\PY@tc##1{\textcolor[rgb]{0.00,0.50,0.00}{##1}}}
\expandafter\def\csname PY@tok@kd\endcsname{\let\PY@bf=\textbf\def\PY@tc##1{\textcolor[rgb]{0.00,0.50,0.00}{##1}}}
\expandafter\def\csname PY@tok@kn\endcsname{\let\PY@bf=\textbf\def\PY@tc##1{\textcolor[rgb]{0.00,0.50,0.00}{##1}}}
\expandafter\def\csname PY@tok@kr\endcsname{\let\PY@bf=\textbf\def\PY@tc##1{\textcolor[rgb]{0.00,0.50,0.00}{##1}}}
\expandafter\def\csname PY@tok@bp\endcsname{\def\PY@tc##1{\textcolor[rgb]{0.00,0.50,0.00}{##1}}}
\expandafter\def\csname PY@tok@fm\endcsname{\def\PY@tc##1{\textcolor[rgb]{0.00,0.00,1.00}{##1}}}
\expandafter\def\csname PY@tok@vc\endcsname{\def\PY@tc##1{\textcolor[rgb]{0.10,0.09,0.49}{##1}}}
\expandafter\def\csname PY@tok@vg\endcsname{\def\PY@tc##1{\textcolor[rgb]{0.10,0.09,0.49}{##1}}}
\expandafter\def\csname PY@tok@vi\endcsname{\def\PY@tc##1{\textcolor[rgb]{0.10,0.09,0.49}{##1}}}
\expandafter\def\csname PY@tok@vm\endcsname{\def\PY@tc##1{\textcolor[rgb]{0.10,0.09,0.49}{##1}}}
\expandafter\def\csname PY@tok@sa\endcsname{\def\PY@tc##1{\textcolor[rgb]{0.73,0.13,0.13}{##1}}}
\expandafter\def\csname PY@tok@sb\endcsname{\def\PY@tc##1{\textcolor[rgb]{0.73,0.13,0.13}{##1}}}
\expandafter\def\csname PY@tok@sc\endcsname{\def\PY@tc##1{\textcolor[rgb]{0.73,0.13,0.13}{##1}}}
\expandafter\def\csname PY@tok@dl\endcsname{\def\PY@tc##1{\textcolor[rgb]{0.73,0.13,0.13}{##1}}}
\expandafter\def\csname PY@tok@s2\endcsname{\def\PY@tc##1{\textcolor[rgb]{0.73,0.13,0.13}{##1}}}
\expandafter\def\csname PY@tok@sh\endcsname{\def\PY@tc##1{\textcolor[rgb]{0.73,0.13,0.13}{##1}}}
\expandafter\def\csname PY@tok@s1\endcsname{\def\PY@tc##1{\textcolor[rgb]{0.73,0.13,0.13}{##1}}}
\expandafter\def\csname PY@tok@mb\endcsname{\def\PY@tc##1{\textcolor[rgb]{0.40,0.40,0.40}{##1}}}
\expandafter\def\csname PY@tok@mf\endcsname{\def\PY@tc##1{\textcolor[rgb]{0.40,0.40,0.40}{##1}}}
\expandafter\def\csname PY@tok@mh\endcsname{\def\PY@tc##1{\textcolor[rgb]{0.40,0.40,0.40}{##1}}}
\expandafter\def\csname PY@tok@mi\endcsname{\def\PY@tc##1{\textcolor[rgb]{0.40,0.40,0.40}{##1}}}
\expandafter\def\csname PY@tok@il\endcsname{\def\PY@tc##1{\textcolor[rgb]{0.40,0.40,0.40}{##1}}}
\expandafter\def\csname PY@tok@mo\endcsname{\def\PY@tc##1{\textcolor[rgb]{0.40,0.40,0.40}{##1}}}
\expandafter\def\csname PY@tok@ch\endcsname{\let\PY@it=\textit\def\PY@tc##1{\textcolor[rgb]{0.25,0.50,0.50}{##1}}}
\expandafter\def\csname PY@tok@cm\endcsname{\let\PY@it=\textit\def\PY@tc##1{\textcolor[rgb]{0.25,0.50,0.50}{##1}}}
\expandafter\def\csname PY@tok@cpf\endcsname{\let\PY@it=\textit\def\PY@tc##1{\textcolor[rgb]{0.25,0.50,0.50}{##1}}}
\expandafter\def\csname PY@tok@c1\endcsname{\let\PY@it=\textit\def\PY@tc##1{\textcolor[rgb]{0.25,0.50,0.50}{##1}}}
\expandafter\def\csname PY@tok@cs\endcsname{\let\PY@it=\textit\def\PY@tc##1{\textcolor[rgb]{0.25,0.50,0.50}{##1}}}

\def\PYZbs{\char`\\}
\def\PYZus{\char`\_}
\def\PYZob{\char`\{}
\def\PYZcb{\char`\}}
\def\PYZca{\char`\^}
\def\PYZam{\char`\&}
\def\PYZlt{\char`\<}
\def\PYZgt{\char`\>}
\def\PYZsh{\char`\#}
\def\PYZpc{\char`\%}
\def\PYZdl{\char`\$}
\def\PYZhy{\char`\-}
\def\PYZsq{\char`\'}
\def\PYZdq{\char`\"}
\def\PYZti{\char`\~}
% for compatibility with earlier versions
\def\PYZat{@}
\def\PYZlb{[}
\def\PYZrb{]}
\makeatother


    % For linebreaks inside Verbatim environment from package fancyvrb. 
    \makeatletter
        \newbox\Wrappedcontinuationbox 
        \newbox\Wrappedvisiblespacebox 
        \newcommand*\Wrappedvisiblespace {\textcolor{red}{\textvisiblespace}} 
        \newcommand*\Wrappedcontinuationsymbol {\textcolor{red}{\llap{\tiny$\m@th\hookrightarrow$}}} 
        \newcommand*\Wrappedcontinuationindent {3ex } 
        \newcommand*\Wrappedafterbreak {\kern\Wrappedcontinuationindent\copy\Wrappedcontinuationbox} 
        % Take advantage of the already applied Pygments mark-up to insert 
        % potential linebreaks for TeX processing. 
        %        {, <, #, %, $, ' and ": go to next line. 
        %        _, }, ^, &, >, - and ~: stay at end of broken line. 
        % Use of \textquotesingle for straight quote. 
        \newcommand*\Wrappedbreaksatspecials {% 
            \def\PYGZus{\discretionary{\char`\_}{\Wrappedafterbreak}{\char`\_}}% 
            \def\PYGZob{\discretionary{}{\Wrappedafterbreak\char`\{}{\char`\{}}% 
            \def\PYGZcb{\discretionary{\char`\}}{\Wrappedafterbreak}{\char`\}}}% 
            \def\PYGZca{\discretionary{\char`\^}{\Wrappedafterbreak}{\char`\^}}% 
            \def\PYGZam{\discretionary{\char`\&}{\Wrappedafterbreak}{\char`\&}}% 
            \def\PYGZlt{\discretionary{}{\Wrappedafterbreak\char`\<}{\char`\<}}% 
            \def\PYGZgt{\discretionary{\char`\>}{\Wrappedafterbreak}{\char`\>}}% 
            \def\PYGZsh{\discretionary{}{\Wrappedafterbreak\char`\#}{\char`\#}}% 
            \def\PYGZpc{\discretionary{}{\Wrappedafterbreak\char`\%}{\char`\%}}% 
            \def\PYGZdl{\discretionary{}{\Wrappedafterbreak\char`\$}{\char`\$}}% 
            \def\PYGZhy{\discretionary{\char`\-}{\Wrappedafterbreak}{\char`\-}}% 
            \def\PYGZsq{\discretionary{}{\Wrappedafterbreak\textquotesingle}{\textquotesingle}}% 
            \def\PYGZdq{\discretionary{}{\Wrappedafterbreak\char`\"}{\char`\"}}% 
            \def\PYGZti{\discretionary{\char`\~}{\Wrappedafterbreak}{\char`\~}}% 
        } 
        % Some characters . , ; ? ! / are not pygmentized. 
        % This macro makes them "active" and they will insert potential linebreaks 
        \newcommand*\Wrappedbreaksatpunct {% 
            \lccode`\~`\.\lowercase{\def~}{\discretionary{\hbox{\char`\.}}{\Wrappedafterbreak}{\hbox{\char`\.}}}% 
            \lccode`\~`\,\lowercase{\def~}{\discretionary{\hbox{\char`\,}}{\Wrappedafterbreak}{\hbox{\char`\,}}}% 
            \lccode`\~`\;\lowercase{\def~}{\discretionary{\hbox{\char`\;}}{\Wrappedafterbreak}{\hbox{\char`\;}}}% 
            \lccode`\~`\:\lowercase{\def~}{\discretionary{\hbox{\char`\:}}{\Wrappedafterbreak}{\hbox{\char`\:}}}% 
            \lccode`\~`\?\lowercase{\def~}{\discretionary{\hbox{\char`\?}}{\Wrappedafterbreak}{\hbox{\char`\?}}}% 
            \lccode`\~`\!\lowercase{\def~}{\discretionary{\hbox{\char`\!}}{\Wrappedafterbreak}{\hbox{\char`\!}}}% 
            \lccode`\~`\/\lowercase{\def~}{\discretionary{\hbox{\char`\/}}{\Wrappedafterbreak}{\hbox{\char`\/}}}% 
            \catcode`\.\active
            \catcode`\,\active 
            \catcode`\;\active
            \catcode`\:\active
            \catcode`\?\active
            \catcode`\!\active
            \catcode`\/\active 
            \lccode`\~`\~ 	
        }
    \makeatother

    \let\OriginalVerbatim=\Verbatim
    \makeatletter
    \renewcommand{\Verbatim}[1][1]{%
        %\parskip\z@skip
        \sbox\Wrappedcontinuationbox {\Wrappedcontinuationsymbol}%
        \sbox\Wrappedvisiblespacebox {\FV@SetupFont\Wrappedvisiblespace}%
        \def\FancyVerbFormatLine ##1{\hsize\linewidth
            \vtop{\raggedright\hyphenpenalty\z@\exhyphenpenalty\z@
                \doublehyphendemerits\z@\finalhyphendemerits\z@
                \strut ##1\strut}%
        }%
        % If the linebreak is at a space, the latter will be displayed as visible
        % space at end of first line, and a continuation symbol starts next line.
        % Stretch/shrink are however usually zero for typewriter font.
        \def\FV@Space {%
            \nobreak\hskip\z@ plus\fontdimen3\font minus\fontdimen4\font
            \discretionary{\copy\Wrappedvisiblespacebox}{\Wrappedafterbreak}
            {\kern\fontdimen2\font}%
        }%
        
        % Allow breaks at special characters using \PYG... macros.
        \Wrappedbreaksatspecials
        % Breaks at punctuation characters . , ; ? ! and / need catcode=\active 	
        \OriginalVerbatim[#1,codes*=\Wrappedbreaksatpunct]%
    }
    \makeatother

    % Exact colors from NB
    \definecolor{incolor}{HTML}{303F9F}
    \definecolor{outcolor}{HTML}{D84315}
    \definecolor{cellborder}{HTML}{CFCFCF}
    \definecolor{cellbackground}{HTML}{F7F7F7}
    
    % prompt
    \makeatletter
    \newcommand{\boxspacing}{\kern\kvtcb@left@rule\kern\kvtcb@boxsep}
    \makeatother
    \newcommand{\prompt}[4]{
        \ttfamily\llap{{\color{#2}[#3]:\hspace{3pt}#4}}\vspace{-\baselineskip}
    }
    

    
    % Prevent overflowing lines due to hard-to-break entities
    \sloppy 
    % Setup hyperref package
    \hypersetup{
      breaklinks=true,  % so long urls are correctly broken across lines
      colorlinks=true,
      urlcolor=urlcolor,
      linkcolor=linkcolor,
      citecolor=citecolor,
      }
    % Slightly bigger margins than the latex defaults
    
    \geometry{verbose,tmargin=1in,bmargin=1in,lmargin=1in,rmargin=1in}
    
    

\begin{document}
    
    \maketitle
    
    

    
    \hypertarget{credit-default-swaps}{%
\section{Credit Default Swaps}\label{credit-default-swaps}}

\hypertarget{credit-curves}{%
\subsection{Credit curves}\label{credit-curves}}

Just like a discount curve is a way of representing the underlying
interest rates implicit in the market quotes of a collection of
real-world interest rate products, \textbf{credit curves} are a way of
representing the data implied by credit default swaps.

\textbf{Credit default swaps} (\textbf{CDS}) are instruments whose value
depends on the likelihood that a given company (the curve's
\textbf{issuer}) will suffer a credit event over a given period.

A \textbf{credit event} can be a default, the failure to make payments,
the issuer entering into bankruptcy proceedings, or the occurence of
other legal events. The exact definition of what constitutes a credit
event depends on a series of factors and is usually defined in some kind
of ISDA (International Swaps and Derivatives Association) master
agreement.

In any case, we will generically call a credit event a \emph{default},
and talk about \textbf{non-default probabilities} (\textbf{NDP}),
i.e.~the probability that the issuer will not suffer a credit event
before a given date.

NDPs are the equivalent for credit curves of discount factors for
discount curves. Just like discount curves, credit curves are built by
specifying a pricing/observation date, a sequence of pillar dates and a
sequence of NDPs. We will then implement a \(\tt{CreditCurve}\) class
that provides a method which interpolates between the pillars of the
curve to return the NDP at an arbitrary value date between the pricing
date and the last pillar date.

In addition, we'll also write a method which returns the \textbf{hazard
rate} at an arbitrary value date. The hazard rate is the credit curve
equivalent of the short rate or overnight rate for discount curves. It
represents the instantaneous probability of the issuer defaulting
conditioned on it not having defaulted until that moment. In practice we
will calculate it numerically, and therefore it'll be the (annualized)
conditional probability of the issuer defaulting between the value date
and the day after.

Discount Curve

Credit Curve

Represents underlying rates implicit in market quotes of IR products

Represents default probability implied by credit default swaps

discount factors

non-default probabilities

short rate

hazard rate

    \begin{tcolorbox}[breakable, size=fbox, boxrule=1pt, pad at break*=1mm,colback=cellbackground, colframe=cellborder]
\prompt{In}{incolor}{ }{\boxspacing}
\begin{Verbatim}[commandchars=\\\{\}]

\end{Verbatim}
\end{tcolorbox}

    \hypertarget{hazard-rate}{%
\subsubsection{Hazard Rate}\label{hazard-rate}}

Hazard rate is often called a \emph{conditional failure rate} since it's
expression is a direct application of the conditional probability
concept.

Conditional probability answers to the question ``how should you update
probabilities of events when there is additional information available
?''. To derive the general formula let's start with an example.

A fair die is rolled. Let \(A\) be the event that the outcome is an odd
number (\(A={1,3,5}\)). Also let \(B\) be the event that the outcome is
less than or equal to \(3\) (\(B={1,2,3}\)). What is the probability of
\(A\) (\(P(A)\)) ? What is the probability of \(A\) given \(B\)
(\(P(A|B)\)) ?

Being a simple example we can compute the result by hand:

\[P(A) = \frac{|A|}{|S|} = \frac{|\{1,3,5\}|}{6} = \frac{1}{2}\qquad\textrm{(where S is the entire sample space)}\]

Now let's find the conditional probability of \(A\) given that \(B\)
occurred. If we know \(B\) has occurred, the outcome must be among
\(\{1,2,3\}\). For \(A\) to also happen the outcome must be in
\(A\cap B = \{1,3\}\). Since all die rolls are equally likely, we argue
that \(P(A|B)\) must be equal to

\[P(A|B) = \frac{|A\cap B|}{|B|} = \frac{2}{3}\]

To generalize our example we can rewrite the calculation by dividing the
numerator and denominator by the entire space of the events \(|S|\)
hence:

\[P(A|B) = \cfrac{|A\cap B|}{|B|} = \cfrac{\cfrac{|A\cap B|}{|S|}}{\cfrac{|B|}{|S|}} = \cfrac{P(A\cap B)}{P(B)}\]

In formula if the non-default probability is indicated by \(N\) and the
hazard rate by \(\lambda\):

\[\lambda = -\cfrac{dN}{dt}\cfrac{1}{N(t_0, t_1)}\]

where the minus sign derives from the fact that \(N\) is a \textbf{non}
default probability while the hazard rate is defined in terms of the
probability of default.

Conversly given the hazard rate the non-default probability can be
determined as:

\[\lambda = -\cfrac{1}{dt}\cdot\cfrac{dN}{N} = -\cfrac{d(\textrm{log}N)}{dt}\]

\[N(t_0, t) = e^{-\int_{t_0}^{t}\lambda dt}\]

    \begin{tcolorbox}[breakable, size=fbox, boxrule=1pt, pad at break*=1mm,colback=cellbackground, colframe=cellborder]
\prompt{In}{incolor}{ }{\boxspacing}
\begin{Verbatim}[commandchars=\\\{\}]
\PY{k+kn}{import} \PY{n+nn}{math}\PY{o}{,} \PY{n+nn}{numpy}
\PY{k+kn}{from} \PY{n+nn}{dateutil}\PY{n+nn}{.}\PY{n+nn}{relativedelta} \PY{k}{import} \PY{n}{relativedelta}

\PY{k}{class} \PY{n+nc}{CreditCurve}\PY{p}{(}\PY{n+nb}{object}\PY{p}{)}\PY{p}{:}
    
    \PY{k}{def} \PY{n+nf}{\PYZus{}\PYZus{}init\PYZus{}\PYZus{}}\PY{p}{(}\PY{n+nb+bp}{self}\PY{p}{,} \PY{n}{pillar\PYZus{}dates}\PY{p}{,} \PY{n}{pillar\PYZus{}ndps}\PY{p}{)}\PY{p}{:}
        \PY{n+nb+bp}{self}\PY{o}{.}\PY{n}{pillar\PYZus{}dates} \PY{o}{=} \PY{n}{pillar\PYZus{}dates}
        
        \PY{n+nb+bp}{self}\PY{o}{.}\PY{n}{pillar\PYZus{}days} \PY{o}{=} \PY{p}{[}
            \PY{p}{(}\PY{n}{pd} \PY{o}{\PYZhy{}} \PY{n}{pillar\PYZus{}dates}\PY{p}{[}\PY{l+m+mi}{0}\PY{p}{]}\PY{p}{)}\PY{o}{.}\PY{n}{days}
            \PY{k}{for} \PY{n}{pd} \PY{o+ow}{in} \PY{n}{pillar\PYZus{}dates}
        \PY{p}{]}
        
        \PY{n+nb+bp}{self}\PY{o}{.}\PY{n}{log\PYZus{}ndps} \PY{o}{=} \PY{p}{[}
            \PY{n}{math}\PY{o}{.}\PY{n}{log}\PY{p}{(}\PY{n}{ndp}\PY{p}{)}
            \PY{k}{for} \PY{n}{ndp} \PY{o+ow}{in} \PY{n}{pillar\PYZus{}ndps}
        \PY{p}{]}
        
    \PY{k}{def} \PY{n+nf}{ndp}\PY{p}{(}\PY{n+nb+bp}{self}\PY{p}{,} \PY{n}{value\PYZus{}date}\PY{p}{)}\PY{p}{:}
        \PY{n}{value\PYZus{}days} \PY{o}{=} \PY{p}{(}\PY{n}{value\PYZus{}date} \PY{o}{\PYZhy{}} \PY{n+nb+bp}{self}\PY{o}{.}\PY{n}{pillar\PYZus{}dates}\PY{p}{[}\PY{l+m+mi}{0}\PY{p}{]}\PY{p}{)}\PY{o}{.}\PY{n}{days}
        \PY{k}{return} \PY{n}{math}\PY{o}{.}\PY{n}{exp}\PY{p}{(}
            \PY{n}{numpy}\PY{o}{.}\PY{n}{interp}\PY{p}{(}\PY{n}{value\PYZus{}days}\PY{p}{,}
                         \PY{n+nb+bp}{self}\PY{o}{.}\PY{n}{pillar\PYZus{}days}\PY{p}{,}
                         \PY{n+nb+bp}{self}\PY{o}{.}\PY{n}{log\PYZus{}ndps}\PY{p}{)}\PY{p}{)}
    
    \PY{k}{def} \PY{n+nf}{hazard}\PY{p}{(}\PY{n+nb+bp}{self}\PY{p}{,} \PY{n}{value\PYZus{}date}\PY{p}{)}\PY{p}{:}
        \PY{n}{ndp\PYZus{}1} \PY{o}{=} \PY{n+nb+bp}{self}\PY{o}{.}\PY{n}{ndp}\PY{p}{(}\PY{n}{value\PYZus{}date}\PY{p}{)}
        \PY{n}{ndp\PYZus{}2} \PY{o}{=} \PY{n+nb+bp}{self}\PY{o}{.}\PY{n}{ndp}\PY{p}{(}\PY{n}{value\PYZus{}date} \PY{o}{+} \PY{n}{relativedelta}\PY{p}{(}\PY{n}{days}\PY{o}{=}\PY{l+m+mi}{1}\PY{p}{)}\PY{p}{)}
        \PY{n}{delta\PYZus{}t} \PY{o}{=} \PY{l+m+mf}{1.0} \PY{o}{/} \PY{l+m+mf}{365.0}
        \PY{n}{h} \PY{o}{=} \PY{o}{\PYZhy{}}\PY{l+m+mf}{1.0} \PY{o}{/} \PY{n}{ndp\PYZus{}1} \PY{o}{*} \PY{p}{(}\PY{n}{ndp\PYZus{}2} \PY{o}{\PYZhy{}} \PY{n}{ndp\PYZus{}1}\PY{p}{)} \PY{o}{/} \PY{n}{delta\PYZus{}t}
        \PY{k}{return} \PY{n}{h}
\end{Verbatim}
\end{tcolorbox}

    As usual we test the newly developed class with some dummy data.

    \begin{tcolorbox}[breakable, size=fbox, boxrule=1pt, pad at break*=1mm,colback=cellbackground, colframe=cellborder]
\prompt{In}{incolor}{ }{\boxspacing}
\begin{Verbatim}[commandchars=\\\{\}]
\PY{k+kn}{from} \PY{n+nn}{datetime} \PY{k}{import} \PY{n}{date}

\PY{n}{pricing\PYZus{}date} \PY{o}{=} \PY{n}{date}\PY{o}{.}\PY{n}{today}\PY{p}{(}\PY{p}{)}

\PY{n}{cc} \PY{o}{=} \PY{n}{CreditCurve}\PY{p}{(}
    \PY{p}{[}\PY{n}{pricing\PYZus{}date}\PY{p}{,} \PY{n}{pricing\PYZus{}date} \PY{o}{+} \PY{n}{relativedelta}\PY{p}{(}\PY{n}{years}\PY{o}{=}\PY{l+m+mi}{2}\PY{p}{)}\PY{p}{]}\PY{p}{,}
    \PY{p}{[}\PY{l+m+mf}{1.0}\PY{p}{,} \PY{l+m+mf}{0.8}\PY{p}{]}
\PY{p}{)}
\end{Verbatim}
\end{tcolorbox}

    \begin{tcolorbox}[breakable, size=fbox, boxrule=1pt, pad at break*=1mm,colback=cellbackground, colframe=cellborder]
\prompt{In}{incolor}{ }{\boxspacing}
\begin{Verbatim}[commandchars=\\\{\}]
\PY{n}{cc}\PY{o}{.}\PY{n}{ndp}\PY{p}{(}\PY{n}{pricing\PYZus{}date} \PY{o}{+} \PY{n}{relativedelta}\PY{p}{(}\PY{n}{years}\PY{o}{=}\PY{l+m+mi}{1}\PY{p}{)}\PY{p}{)}
\end{Verbatim}
\end{tcolorbox}

    \begin{tcolorbox}[breakable, size=fbox, boxrule=1pt, pad at break*=1mm,colback=cellbackground, colframe=cellborder]
\prompt{In}{incolor}{ }{\boxspacing}
\begin{Verbatim}[commandchars=\\\{\}]
\PY{n}{cc}\PY{o}{.}\PY{n}{hazard}\PY{p}{(}\PY{n}{pricing\PYZus{}date} \PY{o}{+} \PY{n}{relativedelta}\PY{p}{(}\PY{n}{years}\PY{o}{=}\PY{l+m+mi}{1}\PY{p}{)}\PY{p}{)}
\end{Verbatim}
\end{tcolorbox}

    \hypertarget{credit-deafult-swaps}{%
\subsection{Credit Deafult Swaps}\label{credit-deafult-swaps}}

Once we have implemented a \(\tt{CreditCurve}\) class which allows us to
interpolate pillar non-default probabilities (NDPs), and also to
calculate the hazard rate at arbitrary dates, we can use it to price
\textbf{credit default swaps} (CDSs).

CDSs are made up of two legs:

\begin{itemize}
\tightlist
\item
  the \emph{default} leg: which pays \(LGD = 1 - R\), known as the
  \textbf{loss given default}, if and when the credit event occurs. A
  conventional value for the recovery parameter \(R\) is 40\%;
\item
  the \emph{premium} leg: which pays a \emph{spread} \(S\) every m
  months until the credit event occurs.
\end{itemize}

\hypertarget{premium-leg}{%
\subsubsection{Premium leg}\label{premium-leg}}

Let's start with the premium leg, which is easier to calculate. We will
use the following notation:

\begin{itemize}
\tightlist
\item
  \(t\) today's date;
\item
  \(t_0\) the start date of the CDS (could be different from \(t\));
\item
  \(t_1, ..., t_n\) the payment dates of the premium leg, which occur at
  a m-month frequency (we assume that \(t_n\) is the end date of the
  CDS);
\item
  \(D(t')\) the discount factor between \(t\) and \(t'\);
\item
  \(N(t')\) the non-default probability between \(t\) and \(t'\);
\item
  \(\tau\) the random variable representing the date of the credit
  event.
\end{itemize}

At each payment date \(t_i\), a flow \(S\) is paid if and only if the
credit event has not occurred before that date. Therefore the NPV of the
each flow is

\[\mathbb{E}\left[ S \times D(t_i) \times \mathbb{1}(\tau > t_i) \right] = S \cdot D(t_i) \cdot N(t_i)\]
therefore the NPV of the premium leg is simply the sum, over the payment
dates, of these terms:

\[\textrm{NPV}_{premium} = \sum_{i=1}^{n} S \cdot D(t_i) \cdot N(t_i)\]

\#\#~Default leg

The LGD \((1-R)\) is paid out on the same date on which the credit event
occurs, i.e.~it can potentially be paid out on any date between \(t_0\)
and \(t_n\). Mathematically, therefore, the NPV of the premium leg can
be expressed as follows:

\[ \mathbb{E} \left[(1-R) \times D(\tau) \times \mathbb{1}(\tau \leq t_n) \right] \]

Using the law of total probability, we can break this down into the sum
of ``daily NPVs'' calculated as a function of the daily ``forward''
default probabilities \(\mathbb{P}\):

\[
\begin{align*}
\mathbb{E}\left[(1-R) \times D(\tau) \times \mathbb{1}(\tau \leq t_n) \right]
&= \sum_{t'=t_0}^{t_n} \mathbb{E}[ (1-R) \times D(\tau) | \tau = t'] \mathbb{P}[ \tau = t' ] \\
&= (1-R) \sum_{t'=t_0}^{t_n} D(t') \left( \mathbb{P}[ \tau \geq t' ] - \mathbb{P}[ \tau \geq t'+1 ] \right) \\
&= (1-R) \sum_{t'=t_0}^{t_n} D(t') \left( N(t') - N(t'+1) \right)
\end{align*}
\]

where the last step holds since
\(\mathbb{P}[\tau\geq d^{'}] = 1 - \mathbb{P}[\tau < d^{'}] = 1 - (1-N(d^{'})) = N(d^{'})\).

    \begin{tcolorbox}[breakable, size=fbox, boxrule=1pt, pad at break*=1mm,colback=cellbackground, colframe=cellborder]
\prompt{In}{incolor}{ }{\boxspacing}
\begin{Verbatim}[commandchars=\\\{\}]
\PY{k+kn}{from} \PY{n+nn}{finmarkets} \PY{k}{import} \PY{n}{generate\PYZus{}swap\PYZus{}dates}

\PY{k}{class} \PY{n+nc}{CreditDefaultSwap}\PY{p}{:}
    
    \PY{k}{def} \PY{n+nf}{\PYZus{}\PYZus{}init\PYZus{}\PYZus{}}\PY{p}{(}\PY{n+nb+bp}{self}\PY{p}{,} \PY{n}{notional}\PY{p}{,} \PY{n}{start\PYZus{}date}\PY{p}{,} \PY{n}{fixed\PYZus{}spread}\PY{p}{,} 
                 \PY{n}{maturity}\PY{p}{,} \PY{n}{tenor}\PY{o}{=}\PY{l+m+mi}{3}\PY{p}{,} \PY{n}{recovery}\PY{o}{=}\PY{l+m+mf}{0.4}\PY{p}{)}\PY{p}{:}
        \PY{n+nb+bp}{self}\PY{o}{.}\PY{n}{notional} \PY{o}{=} \PY{n}{notional}
        \PY{n+nb+bp}{self}\PY{o}{.}\PY{n}{payment\PYZus{}dates} \PY{o}{=} \PY{n}{generate\PYZus{}swap\PYZus{}dates}\PY{p}{(}\PY{n}{start\PYZus{}date}\PY{p}{,} \PY{n}{maturity}\PY{o}{*}\PY{l+m+mi}{12}\PY{p}{,} \PY{n}{tenor}\PY{p}{)}
        \PY{n+nb+bp}{self}\PY{o}{.}\PY{n}{fixed\PYZus{}spread} \PY{o}{=} \PY{n}{fixed\PYZus{}spread}
        \PY{n+nb+bp}{self}\PY{o}{.}\PY{n}{recovery} \PY{o}{=} \PY{n}{recovery}
    
    \PY{k}{def} \PY{n+nf}{premium\PYZus{}leg\PYZus{}npv}\PY{p}{(}\PY{n+nb+bp}{self}\PY{p}{,} \PY{n}{discount\PYZus{}curve}\PY{p}{,} \PY{n}{credit\PYZus{}curve}\PY{p}{)}\PY{p}{:}
        \PY{n}{npv} \PY{o}{=} \PY{l+m+mi}{0}
        \PY{k}{for} \PY{n}{i} \PY{o+ow}{in} \PY{n+nb}{range}\PY{p}{(}\PY{l+m+mi}{1}\PY{p}{,} \PY{n+nb}{len}\PY{p}{(}\PY{n+nb+bp}{self}\PY{o}{.}\PY{n}{payment\PYZus{}dates}\PY{p}{)}\PY{p}{)}\PY{p}{:}
            \PY{n}{npv} \PY{o}{+}\PY{o}{=} \PY{p}{(}
                \PY{n+nb+bp}{self}\PY{o}{.}\PY{n}{fixed\PYZus{}spread} \PY{o}{*}
                \PY{n}{discount\PYZus{}curve}\PY{o}{.}\PY{n}{df}\PY{p}{(}\PY{n+nb+bp}{self}\PY{o}{.}\PY{n}{payment\PYZus{}dates}\PY{p}{[}\PY{n}{i}\PY{p}{]}\PY{p}{)} \PY{o}{*}
                \PY{n}{credit\PYZus{}curve}\PY{o}{.}\PY{n}{ndp}\PY{p}{(}\PY{n+nb+bp}{self}\PY{o}{.}\PY{n}{payment\PYZus{}dates}\PY{p}{[}\PY{n}{i}\PY{p}{]}\PY{p}{)}
            \PY{p}{)}
        \PY{k}{return} \PY{n}{npv} \PY{o}{*} \PY{n+nb+bp}{self}\PY{o}{.}\PY{n}{notional}
    
    \PY{k}{def} \PY{n+nf}{default\PYZus{}leg\PYZus{}npv}\PY{p}{(}\PY{n+nb+bp}{self}\PY{p}{,} \PY{n}{discount\PYZus{}curve}\PY{p}{,} \PY{n}{credit\PYZus{}curve}\PY{p}{)}\PY{p}{:}
        \PY{n}{npv} \PY{o}{=} \PY{l+m+mi}{0}
        \PY{n}{d} \PY{o}{=} \PY{n+nb+bp}{self}\PY{o}{.}\PY{n}{payment\PYZus{}dates}\PY{p}{[}\PY{l+m+mi}{0}\PY{p}{]}
        \PY{k}{while} \PY{n}{d} \PY{o}{\PYZlt{}}\PY{o}{=} \PY{n+nb+bp}{self}\PY{o}{.}\PY{n}{payment\PYZus{}dates}\PY{p}{[}\PY{o}{\PYZhy{}}\PY{l+m+mi}{1}\PY{p}{]}\PY{p}{:}
            \PY{n}{npv} \PY{o}{+}\PY{o}{=} \PY{n}{discount\PYZus{}curve}\PY{o}{.}\PY{n}{df}\PY{p}{(}\PY{n}{d}\PY{p}{)} \PY{o}{*} \PY{p}{(}
                \PY{n}{credit\PYZus{}curve}\PY{o}{.}\PY{n}{ndp}\PY{p}{(}\PY{n}{d}\PY{p}{)} \PY{o}{\PYZhy{}}
                \PY{n}{credit\PYZus{}curve}\PY{o}{.}\PY{n}{ndp}\PY{p}{(}\PY{n}{d} \PY{o}{+} \PY{n}{relativedelta}\PY{p}{(}\PY{n}{days}\PY{o}{=}\PY{l+m+mi}{1}\PY{p}{)}\PY{p}{)}
            \PY{p}{)}
            \PY{n}{d} \PY{o}{+}\PY{o}{=} \PY{n}{relativedelta}\PY{p}{(}\PY{n}{days}\PY{o}{=}\PY{l+m+mi}{1}\PY{p}{)}
        \PY{k}{return} \PY{n}{npv} \PY{o}{*} \PY{n+nb+bp}{self}\PY{o}{.}\PY{n}{notional} \PY{o}{*} \PY{p}{(}\PY{l+m+mi}{1} \PY{o}{\PYZhy{}} \PY{n+nb+bp}{self}\PY{o}{.}\PY{n}{recovery}\PY{p}{)}
    
    \PY{k}{def} \PY{n+nf}{npv}\PY{p}{(}\PY{n+nb+bp}{self}\PY{p}{,} \PY{n}{discount\PYZus{}curve}\PY{p}{,} \PY{n}{credit\PYZus{}curve}\PY{p}{)}\PY{p}{:}
        \PY{k}{return} \PY{n+nb+bp}{self}\PY{o}{.}\PY{n}{default\PYZus{}leg\PYZus{}npv}\PY{p}{(}\PY{n}{discount\PYZus{}curve}\PY{p}{,} \PY{n}{credit\PYZus{}curve}\PY{p}{)} \PY{o}{\PYZhy{}} \PYZbs{}
               \PY{n+nb+bp}{self}\PY{o}{.}\PY{n}{premium\PYZus{}leg\PYZus{}npv}\PY{p}{(}\PY{n}{discount\PYZus{}curve}\PY{p}{,} \PY{n}{credit\PYZus{}curve}\PY{p}{)}
\end{Verbatim}
\end{tcolorbox}

    Below a simple test of the class.

    \begin{tcolorbox}[breakable, size=fbox, boxrule=1pt, pad at break*=1mm,colback=cellbackground, colframe=cellborder]
\prompt{In}{incolor}{ }{\boxspacing}
\begin{Verbatim}[commandchars=\\\{\}]
\PY{n}{credit\PYZus{}curve} \PY{o}{=} \PY{n}{CreditCurve}\PY{p}{(}\PY{p}{[}\PY{n}{pricing\PYZus{}date}\PY{p}{,} \PY{n}{pricing\PYZus{}date} \PY{o}{+} \PY{n}{relativedelta}\PY{p}{(}\PY{n}{months}\PY{o}{=}\PY{l+m+mi}{36}\PY{p}{)}\PY{p}{]}\PY{p}{,} 
                           \PY{p}{[}\PY{l+m+mf}{1.0}\PY{p}{,} \PY{l+m+mf}{0.7}\PY{p}{]}\PY{p}{)}

\PY{n}{cds} \PY{o}{=} \PY{n}{CreditDefaultSwap}\PY{p}{(}\PY{l+m+mf}{1e6}\PY{p}{,} \PY{n}{pricing\PYZus{}date}\PY{p}{,} \PY{l+m+mi}{3}\PY{p}{,} \PY{l+m+mf}{0.03}\PY{p}{)}
\PY{n}{cds}\PY{o}{.}\PY{n}{premium\PYZus{}leg\PYZus{}npv}\PY{p}{(}\PY{n}{discount\PYZus{}curve}\PY{p}{,} \PY{n}{credit\PYZus{}curve}\PY{p}{)}
\end{Verbatim}
\end{tcolorbox}

    \begin{tcolorbox}[breakable, size=fbox, boxrule=1pt, pad at break*=1mm,colback=cellbackground, colframe=cellborder]
\prompt{In}{incolor}{ }{\boxspacing}
\begin{Verbatim}[commandchars=\\\{\}]
\PY{n}{cds}\PY{o}{.}\PY{n}{default\PYZus{}leg\PYZus{}npv}\PY{p}{(}\PY{n}{discount\PYZus{}curve}\PY{p}{,} \PY{n}{credit\PYZus{}curve}\PY{p}{)}
\end{Verbatim}
\end{tcolorbox}

    \begin{tcolorbox}[breakable, size=fbox, boxrule=1pt, pad at break*=1mm,colback=cellbackground, colframe=cellborder]
\prompt{In}{incolor}{ }{\boxspacing}
\begin{Verbatim}[commandchars=\\\{\}]
\PY{n}{cds}\PY{o}{.}\PY{n}{npv}\PY{p}{(}\PY{n}{discount\PYZus{}curve}\PY{p}{,} \PY{n}{credit\PYZus{}curve}\PY{p}{)}
\end{Verbatim}
\end{tcolorbox}

    \hypertarget{estimate-default-probabilities-from-cds}{%
\subsection{Estimate Default Probabilities from
CDS}\label{estimate-default-probabilities-from-cds}}

Pretty much like the discount curves could be derived from swap market
quotes, we can estimate default probabilities (hence credit curves) from
CDS quotes. Following the same steps outlined in
Chapter\textasciitilde{}\ref{swaps-and-bootstrapping---practical-lesson-5}
in fact, through a boostrapping algorithm, we can determine default
probabilites at discrete dates to fill our curve.

Although the derivation is left as an exercise to the reader here we
list the necessary steps:

\begin{itemize}
\tightlist
\item
  collect market quotes for a number of CDS with different maturities;
\item
  create the corresponding CDS objects;
\item
  define a \(\tt{CreditCurve}\) whose pillars are the CDS maturity dates
  and with a set of unknown default probabilities;
\item
  define an objective function to minimize the sum of the squared CDS's
  NPVs;
\item
  set the non-default probabilities to an initial value and define their
  range of variability between \([0, 1]\) since they are probabilities
  and fix ``today's'' probability to 1 since there hasn't been any
  default;
\item
  run the minimization.
\end{itemize}

    \begin{tcolorbox}[breakable, size=fbox, boxrule=1pt, pad at break*=1mm,colback=cellbackground, colframe=cellborder]
\prompt{In}{incolor}{2}{\boxspacing}
\begin{Verbatim}[commandchars=\\\{\}]
\PY{k+kn}{import} \PY{n+nn}{finmarkets}
\PY{n+nb}{dir}\PY{p}{(}\PY{n}{finmarkets}\PY{p}{)}
\end{Verbatim}
\end{tcolorbox}

            \begin{tcolorbox}[breakable, size=fbox, boxrule=.5pt, pad at break*=1mm, opacityfill=0]
\prompt{Out}{outcolor}{2}{\boxspacing}
\begin{Verbatim}[commandchars=\\\{\}]
['CreditCurve',
 'CreditDefaultSwap',
 'DiscountCurve',
 'ForwardRateCurve',
 'InterestRateSwap',
 'InterestRateSwaption',
 'OvernightIndexSwap',
 '\_\_builtins\_\_',
 '\_\_cached\_\_',
 '\_\_doc\_\_',
 '\_\_file\_\_',
 '\_\_loader\_\_',
 '\_\_name\_\_',
 '\_\_package\_\_',
 '\_\_spec\_\_',
 'date',
 'generate\_swap\_dates',
 'math',
 'numpy',
 'relativedelta']
\end{Verbatim}
\end{tcolorbox}
        
    \begin{tcolorbox}[breakable, size=fbox, boxrule=1pt, pad at break*=1mm,colback=cellbackground, colframe=cellborder]
\prompt{In}{incolor}{3}{\boxspacing}
\begin{Verbatim}[commandchars=\\\{\}]
\PY{n}{help}\PY{p}{(}\PY{n}{finmarkets}\PY{p}{)}
\end{Verbatim}
\end{tcolorbox}

    \begin{Verbatim}[commandchars=\\\{\}]
Help on module finmarkets:

NAME
    finmarkets

CLASSES
    builtins.object
        CreditCurve
        CreditDefaultSwap
        DiscountCurve
        ForwardRateCurve
        InterestRateSwap
        InterestRateSwaption
        OvernightIndexSwap

    class CreditCurve(builtins.object)
     |  CreditCurve(pillar\_dates, pillar\_ndps)
     |
     |  CreditCurve: a class to manage credit curves.
     |
     |  Attributes:
     |  -----------
     |  pillar\_date: list of datetime.date
     |      List of dates that forms the pillars of the curve.
     |  ndps: list of floats
     |      List of non-default probabilities.
     |
     |  Methods defined here:
     |
     |  \_\_init\_\_(self, pillar\_dates, pillar\_ndps)
     |      Initialize self.  See help(type(self)) for accurate signature.
     |
     |  hazard(self, value\_date)
     |      hazard: compute the annualized hazard rate.
     |
     |      Params:
     |      -------
     |      value\_date: datetime.date
     |          The date at which the hazard rate is computed.
     |
     |  ndp(self, value\_date)
     |      npd: method to interpolate non-default probability at arbitrary
dates.
     |
     |      Params:
     |      -------
     |      value\_date: datatime.date
     |          The date of the interpolation.
     |
     |  ----------------------------------------------------------------------
     |  Data descriptors defined here:
     |
     |  \_\_dict\_\_
     |      dictionary for instance variables (if defined)
     |
     |  \_\_weakref\_\_
     |      list of weak references to the object (if defined)

    class CreditDefaultSwap(builtins.object)
     |  CreditDefaultSwap(notional, start\_date, fixed\_spread, maturity, tenor=3,
recovery=0.4)
     |
     |  CreditDefaultSwap: a class to valuate Credit Default Swaps
     |
     |  Attributes:
     |  -----------
     |  notional: float
     |      Notional of the swap.
     |  start\_date: datetime.date
     |      Starting date of the contract.
     |  fixed\_spread: float
     |      The spread associated to the premium leg.
     |  maturity: int
     |      Maturity of the swap in years.
     |  tenor: int
     |      Tenor of the premium leg in months, default is 3.
     |  recovery: float
     |      Recovery parameter in case of default, default value is 40\%
     |
     |  Methods defined here:
     |
     |  \_\_init\_\_(self, notional, start\_date, fixed\_spread, maturity, tenor=3,
recovery=0.4)
     |      Initialize self.  See help(type(self)) for accurate signature.
     |
     |  default\_leg\_npv(self, discount\_curve, credit\_curve)
     |      default\_leg\_npv: valuate the default leg.
     |
     |      Params:
     |      -------
     |      discount\_curve: DiscountCurve
     |          The curve to discount the NPV.
     |      credit\_curve: CreditCurve
     |          The curve to extract the default probabilities.
     |
     |  npv(self, discount\_curve, credit\_curve)
     |      npv: valuate the CDS.
     |
     |      Params:
     |      -------
     |      discount\_curve: DiscountCurve
     |          The curve to discount the NPV.
     |      credit\_curve: CreditCurve
     |          The curve to extract the default probabilities.
     |
     |  premium\_leg\_npv(self, discount\_curve, credit\_curve)
     |      premium\_leg\_npv: valuate the premium leg.
     |
     |      Params:
     |      -------
     |      discount\_curve: DiscountCurve
     |          The curve to discount the NPV.
     |      credit\_curve: CreditCurve
     |          The curve to extract the default probabilities.
     |
     |  ----------------------------------------------------------------------
     |  Data descriptors defined here:
     |
     |  \_\_dict\_\_
     |      dictionary for instance variables (if defined)
     |
     |  \_\_weakref\_\_
     |      list of weak references to the object (if defined)

    class DiscountCurve(builtins.object)
     |  DiscountCurve(today, pillar\_dates, discount\_factors)
     |
     |  DiscountCurve: class that manage discount curves.
     |
     |  Attributes:
     |  -----------
     |  today: datetime.date
     |      Observation date
     |  pillar\_dates: list of datetime.date
     |      List of pillars of the discount curve.
     |  discount\_factors: list of float
     |      List of the actual discount factors.
     |
     |  Methods defined here:
     |
     |  \_\_init\_\_(self, today, pillar\_dates, discount\_factors)
     |      Initialize self.  See help(type(self)) for accurate signature.
     |
     |  df(self, d)
     |      df: method to get interpolated discoutn factor at `d`.
     |
     |      Params:
     |      -------
     |      d: datetime.date
     |          The actual date at which we would like the interpolated discount
factor.
     |
     |  forward\_rate(self, d1, d2)
     |      forward\_rate: computes the forward rate referred to the period
d2-d1.
     |
     |      Params:
     |      -------
     |      d1: datetime.date
     |      d2: datetime.date
     |
     |  ----------------------------------------------------------------------
     |  Data descriptors defined here:
     |
     |  \_\_dict\_\_
     |      dictionary for instance variables (if defined)
     |
     |  \_\_weakref\_\_
     |      list of weak references to the object (if defined)

    class ForwardRateCurve(builtins.object)
     |  ForwardRateCurve(pillar\_dates, pillar\_rates)
     |
     |  ForwardRateCurve: container for a forward rate curve.
     |
     |  Attributes:
     |  -----------
     |  pillar\_dates: list of datetime.date
     |      List of pillars of the forward rate curve.
     |  pillar\_rates: list of rates
     |      List of rates of the forward curve.
     |
     |  Methods defined here:
     |
     |  \_\_init\_\_(self, pillar\_dates, pillar\_rates)
     |      Initialize self.  See help(type(self)) for accurate signature.
     |
     |  forward\_rate(self, d)
     |      forward\_rate: compute the forward rate at time d by interpolation.
     |
     |      Params:
     |      -------
     |      d: datetime.date
     |          The date for the interpolation.
     |
     |  ----------------------------------------------------------------------
     |  Data descriptors defined here:
     |
     |  \_\_dict\_\_
     |      dictionary for instance variables (if defined)
     |
     |  \_\_weakref\_\_
     |      list of weak references to the object (if defined)

    class InterestRateSwap(builtins.object)
     |  InterestRateSwap(start\_date, notional, fixed\_rate, tenor\_months,
maturity\_years)
     |
     |  InterestRateSwap: a class to valuate Interest Rate Swaps
     |
     |  Attributes:
     |  -----------
     |  start\_date: datetime.date
     |      Starting date of the contract.
     |  notional: float
     |      Notional of the swap.
     |  fixed\_rate: float
     |      Rate of the fixed leg of the swap.
     |  tenor\_months: int
     |      Tenor of the contract in months.
     |  maturity\_years: int
     |      Maturity of the swap in years.
     |
     |  Methods defined here:
     |
     |  \_\_init\_\_(self, start\_date, notional, fixed\_rate, tenor\_months,
maturity\_years)
     |      Initialize self.  See help(type(self)) for accurate signature.
     |
     |  annuity(self, discount\_curve)
     |      annuity: compute the annuity.
     |
     |      Params:
     |      -------
     |      discount\_curve: DiscountCurve
     |          Discount curve object used for the annuity.
     |
     |  npv(self, discount\_curve, libor\_curve)
     |      npv: compute the npv of the IRS.
     |
     |      Params:
     |      -------
     |      discount\_curve: DiscountCurve
     |          Discount curve object used for swap rate calculation.
     |      libor\_curve: ForwardRateCurve
     |          Libor curve object used for swap rate calculation.
     |
     |  swap\_rate(self, discount\_curve, libor\_curve)
     |      swap\_rate: compute the swap rate of the IRS.
     |
     |      Params:
     |      -------
     |      discount\_curve: DiscountCurve
     |          Discount curve object used for swap rate calculation.
     |      libor\_curve: ForwardRateCurve
     |          Libor curve object used for swap rate calculation.
     |
     |  ----------------------------------------------------------------------
     |  Data descriptors defined here:
     |
     |  \_\_dict\_\_
     |      dictionary for instance variables (if defined)
     |
     |  \_\_weakref\_\_
     |      list of weak references to the object (if defined)

    class InterestRateSwaption(builtins.object)
     |  InterestRateSwaption(exercise\_date, irs)
     |
     |  InterestRateSwaption: class to manage swaptions.
     |
     |  Attributes:
     |  -----------
     |  exercise\_date: datetime.date
     |      The exercise date of the swaptions.
     |  irs: InterestRateSwap
     |      The IRS underlying the swaptions.
     |
     |  Methods defined here:
     |
     |  \_\_init\_\_(self, exercise\_date, irs)
     |      Initialize self.  See help(type(self)) for accurate signature.
     |
     |  npv\_bs(self, discount\_curve, libor\_curve, sigma)
     |      npv\_bs: estimate the swaption NPV using Black-Scholes formula.
     |
     |      Params:
     |      -------
     |      discount\_curve: DiscountCurve
     |          The curve to discount the npv.
     |      libor\_curve: ForwardRateCurve
     |          The libor curve to compute the swap rate.
     |      simga: float
     |          The volatility of the swap rate.
     |
     |  npv\_mc(self, discount\_curve, libor\_curve, sigma, n\_scenarios=10000)
     |      npv\_bs: estimate the swaption NPV with Monte Carlo Simulation.
     |
     |      Params:
     |      -------
     |      discount\_curve: DiscountCurve
     |          The curve to discount the npv.
     |      libor\_curve: ForwardRateCurve
     |          The libor curve to compute the swap rate.
     |      simga: float
     |          The volatility of the swap rate.
     |      n\_scenarios: int
     |          Number of Monte Carlo experiment to simulate.
     |
     |  ----------------------------------------------------------------------
     |  Data descriptors defined here:
     |
     |  \_\_dict\_\_
     |      dictionary for instance variables (if defined)
     |
     |  \_\_weakref\_\_
     |      list of weak references to the object (if defined)

    class OvernightIndexSwap(builtins.object)
     |  OvernightIndexSwap(notional, payment\_dates, fixed\_rate)
     |
     |  OvernightIndexSwap: a class to valuate Overnight Index Swaps
     |
     |  Attributes:
     |  -----------
     |  notional: float
     |      Notional of the swap.
     |  payment\_dates: list of datetime.date
     |      List of payment dates of the swap.
     |  fixed\_rate: float
     |      Rate of the fixed leg of the swap.
     |
     |  Methods defined here:
     |
     |  \_\_init\_\_(self, notional, payment\_dates, fixed\_rate)
     |      Initialize self.  See help(type(self)) for accurate signature.
     |
     |  fair\_value\_strike(self, discount\_curve)
     |      fair\_value\_strike: compute the fair value strike of the OIS.
     |
     |      Params:
     |      -------
     |      discount\_curve: DiscountCurve
     |          Discount curve object used for npv calculation.
     |
     |  npv(self, discount\_curve)
     |      npv: computes the total npv of the swap.
     |
     |      Params:
     |      -------
     |      discount\_curve: DiscountCurve
     |          Discount curve object used for npv calculation.
     |
     |  npv\_fixed\_leg(self, discount\_curve)
     |      npv\_fixed\_leg: computes the fixed leg npv.
     |
     |      Params:
     |      -------
     |      discount\_curve: DiscountCurve
     |          Discount curve object used for npv calculation.
     |
     |  npv\_floating\_leg(self, discount\_curve)
     |      npv\_floating\_leg: computes the floating leg npv.
     |
     |      Params:
     |      -------
     |      discount\_curve: DiscountCurve
     |          Discount curve object used for npv calculation.
     |
     |  ----------------------------------------------------------------------
     |  Data descriptors defined here:
     |
     |  \_\_dict\_\_
     |      dictionary for instance variables (if defined)
     |
     |  \_\_weakref\_\_
     |      list of weak references to the object (if defined)

FUNCTIONS
    generate\_swap\_dates(start\_date, n\_months, tenor\_months=12)
        generate\_swap\_dates: computes a set of dates given starting date and
length in months.
                             The tenor is by construction 12 months.

        Params:
        -------
        start\_date: datetime.date
            The start date of the set of dates.
        n\_months: int
            Number of months that define the length of the list of dates.
        tenor\_months: int
            Set the tenor of the list of dates, by default it is 12 months.

FILE
    /Users/sani/finance\_course/lesson7/finmarkets.py


    \end{Verbatim}

    \begin{tcolorbox}[breakable, size=fbox, boxrule=1pt, pad at break*=1mm,colback=cellbackground, colframe=cellborder]
\prompt{In}{incolor}{ }{\boxspacing}
\begin{Verbatim}[commandchars=\\\{\}]

\end{Verbatim}
\end{tcolorbox}


    % Add a bibliography block to the postdoc
    
    
    
\end{document}

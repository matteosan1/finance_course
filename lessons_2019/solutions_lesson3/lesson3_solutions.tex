
% Default to the notebook output style

    


% Inherit from the specified cell style.




    
\documentclass[11pt]{article}

    
    
    \usepackage[T1]{fontenc}
    % Nicer default font (+ math font) than Computer Modern for most use cases
    \usepackage{mathpazo}

    % Basic figure setup, for now with no caption control since it's done
    % automatically by Pandoc (which extracts ![](path) syntax from Markdown).
    \usepackage{graphicx}
    % We will generate all images so they have a width \maxwidth. This means
    % that they will get their normal width if they fit onto the page, but
    % are scaled down if they would overflow the margins.
    \makeatletter
    \def\maxwidth{\ifdim\Gin@nat@width>\linewidth\linewidth
    \else\Gin@nat@width\fi}
    \makeatother
    \let\Oldincludegraphics\includegraphics
    % Set max figure width to be 80% of text width, for now hardcoded.
    \renewcommand{\includegraphics}[1]{\Oldincludegraphics[width=.8\maxwidth]{#1}}
    % Ensure that by default, figures have no caption (until we provide a
    % proper Figure object with a Caption API and a way to capture that
    % in the conversion process - todo).
    \usepackage{caption}
    \DeclareCaptionLabelFormat{nolabel}{}
    \captionsetup{labelformat=nolabel}

    \usepackage{adjustbox} % Used to constrain images to a maximum size 
    \usepackage{xcolor} % Allow colors to be defined
    \usepackage{enumerate} % Needed for markdown enumerations to work
    \usepackage{geometry} % Used to adjust the document margins
    \usepackage{amsmath} % Equations
    \usepackage{amssymb} % Equations
    \usepackage{textcomp} % defines textquotesingle
    % Hack from http://tex.stackexchange.com/a/47451/13684:
    \AtBeginDocument{%
        \def\PYZsq{\textquotesingle}% Upright quotes in Pygmentized code
    }
    \usepackage{upquote} % Upright quotes for verbatim code
    \usepackage{eurosym} % defines \euro
    \usepackage[mathletters]{ucs} % Extended unicode (utf-8) support
    \usepackage[utf8x]{inputenc} % Allow utf-8 characters in the tex document
    \usepackage{fancyvrb} % verbatim replacement that allows latex
    \usepackage{grffile} % extends the file name processing of package graphics 
                         % to support a larger range 
    % The hyperref package gives us a pdf with properly built
    % internal navigation ('pdf bookmarks' for the table of contents,
    % internal cross-reference links, web links for URLs, etc.)
    \usepackage{hyperref}
    \usepackage{longtable} % longtable support required by pandoc >1.10
    \usepackage{booktabs}  % table support for pandoc > 1.12.2
    \usepackage[inline]{enumitem} % IRkernel/repr support (it uses the enumerate* environment)
    \usepackage[normalem]{ulem} % ulem is needed to support strikethroughs (\sout)
                                % normalem makes italics be italics, not underlines
    \usepackage{mathrsfs}
    

    
    
    % Colors for the hyperref package
    \definecolor{urlcolor}{rgb}{0,.145,.698}
    \definecolor{linkcolor}{rgb}{.71,0.21,0.01}
    \definecolor{citecolor}{rgb}{.12,.54,.11}

    % ANSI colors
    \definecolor{ansi-black}{HTML}{3E424D}
    \definecolor{ansi-black-intense}{HTML}{282C36}
    \definecolor{ansi-red}{HTML}{E75C58}
    \definecolor{ansi-red-intense}{HTML}{B22B31}
    \definecolor{ansi-green}{HTML}{00A250}
    \definecolor{ansi-green-intense}{HTML}{007427}
    \definecolor{ansi-yellow}{HTML}{DDB62B}
    \definecolor{ansi-yellow-intense}{HTML}{B27D12}
    \definecolor{ansi-blue}{HTML}{208FFB}
    \definecolor{ansi-blue-intense}{HTML}{0065CA}
    \definecolor{ansi-magenta}{HTML}{D160C4}
    \definecolor{ansi-magenta-intense}{HTML}{A03196}
    \definecolor{ansi-cyan}{HTML}{60C6C8}
    \definecolor{ansi-cyan-intense}{HTML}{258F8F}
    \definecolor{ansi-white}{HTML}{C5C1B4}
    \definecolor{ansi-white-intense}{HTML}{A1A6B2}
    \definecolor{ansi-default-inverse-fg}{HTML}{FFFFFF}
    \definecolor{ansi-default-inverse-bg}{HTML}{000000}

    % commands and environments needed by pandoc snippets
    % extracted from the output of `pandoc -s`
    \providecommand{\tightlist}{%
      \setlength{\itemsep}{0pt}\setlength{\parskip}{0pt}}
    \DefineVerbatimEnvironment{Highlighting}{Verbatim}{commandchars=\\\{\}}
    % Add ',fontsize=\small' for more characters per line
    \newenvironment{Shaded}{}{}
    \newcommand{\KeywordTok}[1]{\textcolor[rgb]{0.00,0.44,0.13}{\textbf{{#1}}}}
    \newcommand{\DataTypeTok}[1]{\textcolor[rgb]{0.56,0.13,0.00}{{#1}}}
    \newcommand{\DecValTok}[1]{\textcolor[rgb]{0.25,0.63,0.44}{{#1}}}
    \newcommand{\BaseNTok}[1]{\textcolor[rgb]{0.25,0.63,0.44}{{#1}}}
    \newcommand{\FloatTok}[1]{\textcolor[rgb]{0.25,0.63,0.44}{{#1}}}
    \newcommand{\CharTok}[1]{\textcolor[rgb]{0.25,0.44,0.63}{{#1}}}
    \newcommand{\StringTok}[1]{\textcolor[rgb]{0.25,0.44,0.63}{{#1}}}
    \newcommand{\CommentTok}[1]{\textcolor[rgb]{0.38,0.63,0.69}{\textit{{#1}}}}
    \newcommand{\OtherTok}[1]{\textcolor[rgb]{0.00,0.44,0.13}{{#1}}}
    \newcommand{\AlertTok}[1]{\textcolor[rgb]{1.00,0.00,0.00}{\textbf{{#1}}}}
    \newcommand{\FunctionTok}[1]{\textcolor[rgb]{0.02,0.16,0.49}{{#1}}}
    \newcommand{\RegionMarkerTok}[1]{{#1}}
    \newcommand{\ErrorTok}[1]{\textcolor[rgb]{1.00,0.00,0.00}{\textbf{{#1}}}}
    \newcommand{\NormalTok}[1]{{#1}}
    
    % Additional commands for more recent versions of Pandoc
    \newcommand{\ConstantTok}[1]{\textcolor[rgb]{0.53,0.00,0.00}{{#1}}}
    \newcommand{\SpecialCharTok}[1]{\textcolor[rgb]{0.25,0.44,0.63}{{#1}}}
    \newcommand{\VerbatimStringTok}[1]{\textcolor[rgb]{0.25,0.44,0.63}{{#1}}}
    \newcommand{\SpecialStringTok}[1]{\textcolor[rgb]{0.73,0.40,0.53}{{#1}}}
    \newcommand{\ImportTok}[1]{{#1}}
    \newcommand{\DocumentationTok}[1]{\textcolor[rgb]{0.73,0.13,0.13}{\textit{{#1}}}}
    \newcommand{\AnnotationTok}[1]{\textcolor[rgb]{0.38,0.63,0.69}{\textbf{\textit{{#1}}}}}
    \newcommand{\CommentVarTok}[1]{\textcolor[rgb]{0.38,0.63,0.69}{\textbf{\textit{{#1}}}}}
    \newcommand{\VariableTok}[1]{\textcolor[rgb]{0.10,0.09,0.49}{{#1}}}
    \newcommand{\ControlFlowTok}[1]{\textcolor[rgb]{0.00,0.44,0.13}{\textbf{{#1}}}}
    \newcommand{\OperatorTok}[1]{\textcolor[rgb]{0.40,0.40,0.40}{{#1}}}
    \newcommand{\BuiltInTok}[1]{{#1}}
    \newcommand{\ExtensionTok}[1]{{#1}}
    \newcommand{\PreprocessorTok}[1]{\textcolor[rgb]{0.74,0.48,0.00}{{#1}}}
    \newcommand{\AttributeTok}[1]{\textcolor[rgb]{0.49,0.56,0.16}{{#1}}}
    \newcommand{\InformationTok}[1]{\textcolor[rgb]{0.38,0.63,0.69}{\textbf{\textit{{#1}}}}}
    \newcommand{\WarningTok}[1]{\textcolor[rgb]{0.38,0.63,0.69}{\textbf{\textit{{#1}}}}}
    
    
    % Define a nice break command that doesn't care if a line doesn't already
    % exist.
    \def\br{\hspace*{\fill} \\* }
    % Math Jax compatibility definitions
    \def\gt{>}
    \def\lt{<}
    \let\Oldtex\TeX
    \let\Oldlatex\LaTeX
    \renewcommand{\TeX}{\textrm{\Oldtex}}
    \renewcommand{\LaTeX}{\textrm{\Oldlatex}}
    % Document parameters
    % Document title
    \title{Solutions - Practical Lesson 3}
    \author{Matteo Sani \\ \href{mailto:matteosan1@gmail.com}{matteosan1@gmail.com}}
        
    

    % Pygments definitions
    
\makeatletter
\def\PY@reset{\let\PY@it=\relax \let\PY@bf=\relax%
    \let\PY@ul=\relax \let\PY@tc=\relax%
    \let\PY@bc=\relax \let\PY@ff=\relax}
\def\PY@tok#1{\csname PY@tok@#1\endcsname}
\def\PY@toks#1+{\ifx\relax#1\empty\else%
    \PY@tok{#1}\expandafter\PY@toks\fi}
\def\PY@do#1{\PY@bc{\PY@tc{\PY@ul{%
    \PY@it{\PY@bf{\PY@ff{#1}}}}}}}
\def\PY#1#2{\PY@reset\PY@toks#1+\relax+\PY@do{#2}}

\expandafter\def\csname PY@tok@w\endcsname{\def\PY@tc##1{\textcolor[rgb]{0.73,0.73,0.73}{##1}}}
\expandafter\def\csname PY@tok@c\endcsname{\let\PY@it=\textit\def\PY@tc##1{\textcolor[rgb]{0.25,0.50,0.50}{##1}}}
\expandafter\def\csname PY@tok@cp\endcsname{\def\PY@tc##1{\textcolor[rgb]{0.74,0.48,0.00}{##1}}}
\expandafter\def\csname PY@tok@k\endcsname{\let\PY@bf=\textbf\def\PY@tc##1{\textcolor[rgb]{0.00,0.50,0.00}{##1}}}
\expandafter\def\csname PY@tok@kp\endcsname{\def\PY@tc##1{\textcolor[rgb]{0.00,0.50,0.00}{##1}}}
\expandafter\def\csname PY@tok@kt\endcsname{\def\PY@tc##1{\textcolor[rgb]{0.69,0.00,0.25}{##1}}}
\expandafter\def\csname PY@tok@o\endcsname{\def\PY@tc##1{\textcolor[rgb]{0.40,0.40,0.40}{##1}}}
\expandafter\def\csname PY@tok@ow\endcsname{\let\PY@bf=\textbf\def\PY@tc##1{\textcolor[rgb]{0.67,0.13,1.00}{##1}}}
\expandafter\def\csname PY@tok@nb\endcsname{\def\PY@tc##1{\textcolor[rgb]{0.00,0.50,0.00}{##1}}}
\expandafter\def\csname PY@tok@nf\endcsname{\def\PY@tc##1{\textcolor[rgb]{0.00,0.00,1.00}{##1}}}
\expandafter\def\csname PY@tok@nc\endcsname{\let\PY@bf=\textbf\def\PY@tc##1{\textcolor[rgb]{0.00,0.00,1.00}{##1}}}
\expandafter\def\csname PY@tok@nn\endcsname{\let\PY@bf=\textbf\def\PY@tc##1{\textcolor[rgb]{0.00,0.00,1.00}{##1}}}
\expandafter\def\csname PY@tok@ne\endcsname{\let\PY@bf=\textbf\def\PY@tc##1{\textcolor[rgb]{0.82,0.25,0.23}{##1}}}
\expandafter\def\csname PY@tok@nv\endcsname{\def\PY@tc##1{\textcolor[rgb]{0.10,0.09,0.49}{##1}}}
\expandafter\def\csname PY@tok@no\endcsname{\def\PY@tc##1{\textcolor[rgb]{0.53,0.00,0.00}{##1}}}
\expandafter\def\csname PY@tok@nl\endcsname{\def\PY@tc##1{\textcolor[rgb]{0.63,0.63,0.00}{##1}}}
\expandafter\def\csname PY@tok@ni\endcsname{\let\PY@bf=\textbf\def\PY@tc##1{\textcolor[rgb]{0.60,0.60,0.60}{##1}}}
\expandafter\def\csname PY@tok@na\endcsname{\def\PY@tc##1{\textcolor[rgb]{0.49,0.56,0.16}{##1}}}
\expandafter\def\csname PY@tok@nt\endcsname{\let\PY@bf=\textbf\def\PY@tc##1{\textcolor[rgb]{0.00,0.50,0.00}{##1}}}
\expandafter\def\csname PY@tok@nd\endcsname{\def\PY@tc##1{\textcolor[rgb]{0.67,0.13,1.00}{##1}}}
\expandafter\def\csname PY@tok@s\endcsname{\def\PY@tc##1{\textcolor[rgb]{0.73,0.13,0.13}{##1}}}
\expandafter\def\csname PY@tok@sd\endcsname{\let\PY@it=\textit\def\PY@tc##1{\textcolor[rgb]{0.73,0.13,0.13}{##1}}}
\expandafter\def\csname PY@tok@si\endcsname{\let\PY@bf=\textbf\def\PY@tc##1{\textcolor[rgb]{0.73,0.40,0.53}{##1}}}
\expandafter\def\csname PY@tok@se\endcsname{\let\PY@bf=\textbf\def\PY@tc##1{\textcolor[rgb]{0.73,0.40,0.13}{##1}}}
\expandafter\def\csname PY@tok@sr\endcsname{\def\PY@tc##1{\textcolor[rgb]{0.73,0.40,0.53}{##1}}}
\expandafter\def\csname PY@tok@ss\endcsname{\def\PY@tc##1{\textcolor[rgb]{0.10,0.09,0.49}{##1}}}
\expandafter\def\csname PY@tok@sx\endcsname{\def\PY@tc##1{\textcolor[rgb]{0.00,0.50,0.00}{##1}}}
\expandafter\def\csname PY@tok@m\endcsname{\def\PY@tc##1{\textcolor[rgb]{0.40,0.40,0.40}{##1}}}
\expandafter\def\csname PY@tok@gh\endcsname{\let\PY@bf=\textbf\def\PY@tc##1{\textcolor[rgb]{0.00,0.00,0.50}{##1}}}
\expandafter\def\csname PY@tok@gu\endcsname{\let\PY@bf=\textbf\def\PY@tc##1{\textcolor[rgb]{0.50,0.00,0.50}{##1}}}
\expandafter\def\csname PY@tok@gd\endcsname{\def\PY@tc##1{\textcolor[rgb]{0.63,0.00,0.00}{##1}}}
\expandafter\def\csname PY@tok@gi\endcsname{\def\PY@tc##1{\textcolor[rgb]{0.00,0.63,0.00}{##1}}}
\expandafter\def\csname PY@tok@gr\endcsname{\def\PY@tc##1{\textcolor[rgb]{1.00,0.00,0.00}{##1}}}
\expandafter\def\csname PY@tok@ge\endcsname{\let\PY@it=\textit}
\expandafter\def\csname PY@tok@gs\endcsname{\let\PY@bf=\textbf}
\expandafter\def\csname PY@tok@gp\endcsname{\let\PY@bf=\textbf\def\PY@tc##1{\textcolor[rgb]{0.00,0.00,0.50}{##1}}}
\expandafter\def\csname PY@tok@go\endcsname{\def\PY@tc##1{\textcolor[rgb]{0.53,0.53,0.53}{##1}}}
\expandafter\def\csname PY@tok@gt\endcsname{\def\PY@tc##1{\textcolor[rgb]{0.00,0.27,0.87}{##1}}}
\expandafter\def\csname PY@tok@err\endcsname{\def\PY@bc##1{\setlength{\fboxsep}{0pt}\fcolorbox[rgb]{1.00,0.00,0.00}{1,1,1}{\strut ##1}}}
\expandafter\def\csname PY@tok@kc\endcsname{\let\PY@bf=\textbf\def\PY@tc##1{\textcolor[rgb]{0.00,0.50,0.00}{##1}}}
\expandafter\def\csname PY@tok@kd\endcsname{\let\PY@bf=\textbf\def\PY@tc##1{\textcolor[rgb]{0.00,0.50,0.00}{##1}}}
\expandafter\def\csname PY@tok@kn\endcsname{\let\PY@bf=\textbf\def\PY@tc##1{\textcolor[rgb]{0.00,0.50,0.00}{##1}}}
\expandafter\def\csname PY@tok@kr\endcsname{\let\PY@bf=\textbf\def\PY@tc##1{\textcolor[rgb]{0.00,0.50,0.00}{##1}}}
\expandafter\def\csname PY@tok@bp\endcsname{\def\PY@tc##1{\textcolor[rgb]{0.00,0.50,0.00}{##1}}}
\expandafter\def\csname PY@tok@fm\endcsname{\def\PY@tc##1{\textcolor[rgb]{0.00,0.00,1.00}{##1}}}
\expandafter\def\csname PY@tok@vc\endcsname{\def\PY@tc##1{\textcolor[rgb]{0.10,0.09,0.49}{##1}}}
\expandafter\def\csname PY@tok@vg\endcsname{\def\PY@tc##1{\textcolor[rgb]{0.10,0.09,0.49}{##1}}}
\expandafter\def\csname PY@tok@vi\endcsname{\def\PY@tc##1{\textcolor[rgb]{0.10,0.09,0.49}{##1}}}
\expandafter\def\csname PY@tok@vm\endcsname{\def\PY@tc##1{\textcolor[rgb]{0.10,0.09,0.49}{##1}}}
\expandafter\def\csname PY@tok@sa\endcsname{\def\PY@tc##1{\textcolor[rgb]{0.73,0.13,0.13}{##1}}}
\expandafter\def\csname PY@tok@sb\endcsname{\def\PY@tc##1{\textcolor[rgb]{0.73,0.13,0.13}{##1}}}
\expandafter\def\csname PY@tok@sc\endcsname{\def\PY@tc##1{\textcolor[rgb]{0.73,0.13,0.13}{##1}}}
\expandafter\def\csname PY@tok@dl\endcsname{\def\PY@tc##1{\textcolor[rgb]{0.73,0.13,0.13}{##1}}}
\expandafter\def\csname PY@tok@s2\endcsname{\def\PY@tc##1{\textcolor[rgb]{0.73,0.13,0.13}{##1}}}
\expandafter\def\csname PY@tok@sh\endcsname{\def\PY@tc##1{\textcolor[rgb]{0.73,0.13,0.13}{##1}}}
\expandafter\def\csname PY@tok@s1\endcsname{\def\PY@tc##1{\textcolor[rgb]{0.73,0.13,0.13}{##1}}}
\expandafter\def\csname PY@tok@mb\endcsname{\def\PY@tc##1{\textcolor[rgb]{0.40,0.40,0.40}{##1}}}
\expandafter\def\csname PY@tok@mf\endcsname{\def\PY@tc##1{\textcolor[rgb]{0.40,0.40,0.40}{##1}}}
\expandafter\def\csname PY@tok@mh\endcsname{\def\PY@tc##1{\textcolor[rgb]{0.40,0.40,0.40}{##1}}}
\expandafter\def\csname PY@tok@mi\endcsname{\def\PY@tc##1{\textcolor[rgb]{0.40,0.40,0.40}{##1}}}
\expandafter\def\csname PY@tok@il\endcsname{\def\PY@tc##1{\textcolor[rgb]{0.40,0.40,0.40}{##1}}}
\expandafter\def\csname PY@tok@mo\endcsname{\def\PY@tc##1{\textcolor[rgb]{0.40,0.40,0.40}{##1}}}
\expandafter\def\csname PY@tok@ch\endcsname{\let\PY@it=\textit\def\PY@tc##1{\textcolor[rgb]{0.25,0.50,0.50}{##1}}}
\expandafter\def\csname PY@tok@cm\endcsname{\let\PY@it=\textit\def\PY@tc##1{\textcolor[rgb]{0.25,0.50,0.50}{##1}}}
\expandafter\def\csname PY@tok@cpf\endcsname{\let\PY@it=\textit\def\PY@tc##1{\textcolor[rgb]{0.25,0.50,0.50}{##1}}}
\expandafter\def\csname PY@tok@c1\endcsname{\let\PY@it=\textit\def\PY@tc##1{\textcolor[rgb]{0.25,0.50,0.50}{##1}}}
\expandafter\def\csname PY@tok@cs\endcsname{\let\PY@it=\textit\def\PY@tc##1{\textcolor[rgb]{0.25,0.50,0.50}{##1}}}

\def\PYZbs{\char`\\}
\def\PYZus{\char`\_}
\def\PYZob{\char`\{}
\def\PYZcb{\char`\}}
\def\PYZca{\char`\^}
\def\PYZam{\char`\&}
\def\PYZlt{\char`\<}
\def\PYZgt{\char`\>}
\def\PYZsh{\char`\#}
\def\PYZpc{\char`\%}
\def\PYZdl{\char`\$}
\def\PYZhy{\char`\-}
\def\PYZsq{\char`\'}
\def\PYZdq{\char`\"}
\def\PYZti{\char`\~}
% for compatibility with earlier versions
\def\PYZat{@}
\def\PYZlb{[}
\def\PYZrb{]}
\makeatother


    % Exact colors from NB
    \definecolor{incolor}{rgb}{0.0, 0.0, 0.5}
    \definecolor{outcolor}{rgb}{0.545, 0.0, 0.0}



    
    % Prevent overflowing lines due to hard-to-break entities
    \sloppy 
    % Setup hyperref package
    \hypersetup{
      breaklinks=true,  % so long urls are correctly broken across lines
      colorlinks=true,
      urlcolor=urlcolor,
      linkcolor=linkcolor,
      citecolor=citecolor,
      }
    % Slightly bigger margins than the latex defaults
    
    \geometry{verbose,tmargin=1in,bmargin=1in,lmargin=1in,rmargin=1in}
    
    

    \begin{document}
    
    
    \maketitle
    
    

    
    \hypertarget{clarifications}{%
\section{Clarifications}\label{clarifications}}

\hypertarget{pycharm-and-debugging}{%
\subsection{PyCharm and debugging}\label{pycharm-and-debugging}}

I've prepared few slides to show you how to use the debugger (running
line by line a program) with PyCharm. We won't probably have time to do
it here, please if you are interested take a look at home and in case
let me know if you have troubles.

\hypertarget{section}{%
\subsection{\texorpdfstring{Backslash ``\textbackslash''
}{ }}\label{section}}

As you may have noticed in many programs (especially in the pdf of the
lessons) I have used a \textbackslash~inside the lines of code. The \textbackslash~symbol is used to
tell python that an instruction is not finished but continues in the
line below and it is used to fit the code in the width of the pdf and
hopefully to make it more readable.

\begin{Shaded}
\begin{Highlighting}[]
\CommentTok{# this line}
\NormalTok{log_discount_factors }\OperatorTok{=}\NormalTok{ [log(discount_factor) }\ControlFlowTok{for}\NormalTok{ discount_factor }\KeywordTok{in}\NormalTok{ discount_factors]}

\CommentTok{# is equivalent to this one }
\NormalTok{log_discount_factors }\OperatorTok{=} \OperatorTok{\textbackslash{}}
\NormalTok{    [log(discount_factor) }\ControlFlowTok{for}\NormalTok{ discount_factor }\KeywordTok{in}\NormalTok{ discount_factors]}
\end{Highlighting}
\end{Shaded}

\hypertarget{functions-and-return-value}{%
\subsection{\texorpdfstring{Functions and \texttt{return}
Value}{Functions and return Value}}\label{functions-and-return-value}}

As we have seen in Practical Lesson 2 functions can return any kind of
objects (numbers, strings, lists, complex objects\ldots{}) but this is
not mandatory i.e.~you can write a function \textbf{without} a
\texttt{return} statement.

\begin{Shaded}
\begin{Highlighting}[]
\KeywordTok{def}\NormalTok{ printing(mystring):}
    \BuiltInTok{print}\NormalTok{ (myString)}
\end{Highlighting}
\end{Shaded}

This is a dummy example to show that a function can just print something
to screen. In addition the syntax of the \texttt{return} is different
from \texttt{Visual\ Basic}, the returned object needn't have to have
the same name as the function (and actually it is better not to to avoid
confusion):

\begin{Shaded}
\begin{Highlighting}[]
\CommentTok{# advise AGAINST using this style}
\KeywordTok{def}\NormalTok{ pippo(a):}
\NormalTok{    pippo }\OperatorTok{=}\NormalTok{ a}
    \ControlFlowTok{return}\NormalTok{ pippo}
\end{Highlighting}
\end{Shaded}

\hypertarget{assert}{%
\subsection{\texorpdfstring{\texttt{assert}}{assert}}\label{assert}}

Forgot to tell you that \texttt{assert} can take a second argument with
a message to display in case of failure.

\begin{Shaded}
\begin{Highlighting}[]
\ControlFlowTok{assert} \DecValTok{1} \OperatorTok{>} \DecValTok{2}\NormalTok{, }\StringTok{"Two is bigger than one"}
\end{Highlighting}
\end{Shaded}

\hypertarget{import}{%
\subsection{\texorpdfstring{\texttt{import}}{import}}\label{import}}

Whenever you import one python file into another the code in the global
scope of the imported file is run:

\begin{Shaded}
\begin{Highlighting}[]
\CommentTok{# file A.py}
\KeywordTok{def}\NormalTok{ functionA(sumUpTo):}
\NormalTok{    result }\OperatorTok{=} \DecValTok{0}
    \ControlFlowTok{for}\NormalTok{ i }\KeywordTok{in} \BuiltInTok{range}\NormalTok{(sumUpTo}\OperatorTok{+}\DecValTok{1}\NormalTok{):}
\NormalTok{        result }\OperatorTok{+=}\NormalTok{ i}
    \ControlFlowTok{return}\NormalTok{ result}

\CommentTok{# this is a test for functionA}
\ControlFlowTok{assert}\NormalTok{ functionA(}\DecValTok{5}\NormalTok{) }\OperatorTok{==} \DecValTok{15}
\BuiltInTok{print}\NormalTok{ (}\StringTok{"Test is OK !"}\NormalTok{)}

\OperatorTok{--------------------------------}
\CommentTok{# file B.py}
\ImportTok{from}\NormalTok{ A }\ImportTok{import}\NormalTok{ functionA}

\BuiltInTok{print}\NormalTok{ (functionA(}\DecValTok{12}\NormalTok{))}
\end{Highlighting}
\end{Shaded}

In this case when you run file B.py you will see on the screen ``Test is
OK !'' since during the import of A.py whatever is in the global scope
(i.e.~the code not indented) is run.

In order to avoid this behaviour you can \emph{protect} the code in A.py
using the following syntax:

\begin{Shaded}
\begin{Highlighting}[]
\CommentTok{# file A.py}
\KeywordTok{def}\NormalTok{ functionA(sumUpTo):}
\NormalTok{    result }\OperatorTok{=} \DecValTok{0}
    \ControlFlowTok{for}\NormalTok{ i }\KeywordTok{in} \BuiltInTok{range}\NormalTok{(sumUpTo}\OperatorTok{+}\DecValTok{1}\NormalTok{):}
\NormalTok{        result }\OperatorTok{+=}\NormalTok{ i}
    \ControlFlowTok{return}\NormalTok{ result}

\CommentTok{# this is a test for functionA}
\ControlFlowTok{if} \VariableTok{__name__} \OperatorTok{==} \StringTok{"__main__"}\NormalTok{:}
    \ControlFlowTok{assert}\NormalTok{ functionA(}\DecValTok{5}\NormalTok{) }\OperatorTok{==} \DecValTok{15}
    \BuiltInTok{print}\NormalTok{ (}\StringTok{"Test is OK !"}\NormalTok{)}

\OperatorTok{--------------------------------}
\CommentTok{# file B.py}
\ImportTok{from}\NormalTok{ A }\ImportTok{import}\NormalTok{ functionA}

\BuiltInTok{print}\NormalTok{ (functionA(}\DecValTok{12}\NormalTok{))}
\end{Highlighting}
\end{Shaded}

Basically the special variable \texttt{\_\_name\_\_} takes the value
\texttt{\_\_main\_\_} if and only if you are running the directly file
A.py. Otherwise, importing A.py, it has a different value and the if
block is not executed.

    \hypertarget{solutions---practical-lession-3}{%
\section{Solutions
3}\label{solutions---practical-lession-3}}

\hypertarget{exercises}{%
\subsection{Exercises}\label{exercises}}

\hypertarget{exercise-3.1}{%
\subsubsection{Exercise 3.1}\label{exercise-3.1}}

Take the code for the Black-Scholes formula from exercise 2.3 and wrap
it in a function. Then, use this function to calculate the prices of
calls with various strikes, using the following data.

\begin{Shaded}
\begin{Highlighting}[]
\NormalTok{s }\OperatorTok{=} \DecValTok{800}
\CommentTok{# strikes expressed as % of spot price}
\NormalTok{moneyness }\OperatorTok{=}\NormalTok{ [ }\FloatTok{0.5}\NormalTok{, }\FloatTok{0.75}\NormalTok{, }\FloatTok{0.825}\NormalTok{, }\FloatTok{1.0}\NormalTok{, }\FloatTok{1.125}\NormalTok{, }\FloatTok{1.25}\NormalTok{, }\FloatTok{1.5}\NormalTok{ ] }
\NormalTok{vol }\OperatorTok{=} \FloatTok{0.3}
\NormalTok{ttm }\OperatorTok{=} \FloatTok{0.75}
\NormalTok{r }\OperatorTok{=} \FloatTok{0.005}
\end{Highlighting}
\end{Shaded}

The output should be a dictionary mapping strikes to call prices.

\textbf{Solution:}

    \begin{Verbatim}[commandchars=\\\{\}]
{\color{incolor}In [{\color{incolor}1}]:} \PY{k+kn}{from} \PY{n+nn}{math} \PY{k}{import} \PY{n}{log}\PY{p}{,} \PY{n}{exp}\PY{p}{,} \PY{n}{sqrt}
        \PY{k+kn}{from} \PY{n+nn}{scipy}\PY{n+nn}{.}\PY{n+nn}{stats} \PY{k}{import} \PY{n}{norm}
        
        \PY{k}{def} \PY{n+nf}{d1}\PY{p}{(}\PY{n}{S\PYZus{}t}\PY{p}{,} \PY{n}{K}\PY{p}{,} \PY{n}{r}\PY{p}{,} \PY{n}{vol}\PY{p}{,} \PY{n}{ttm}\PY{p}{)}\PY{p}{:}
            \PY{n}{num} \PY{o}{=} \PY{n}{log}\PY{p}{(}\PY{n}{S\PYZus{}t}\PY{o}{/}\PY{n}{K}\PY{p}{)} \PY{o}{+} \PY{p}{(}\PY{n}{r} \PY{o}{+} \PY{l+m+mf}{0.5}\PY{o}{*}\PY{n+nb}{pow}\PY{p}{(}\PY{n}{vol}\PY{p}{,} \PY{l+m+mi}{2}\PY{p}{)}\PY{p}{)} \PY{o}{*} \PY{n}{ttm}
            \PY{n}{den} \PY{o}{=} \PY{n}{vol} \PY{o}{*} \PY{n}{sqrt}\PY{p}{(}\PY{n}{ttm}\PY{p}{)}
            \PY{k}{return} \PY{n}{num}\PY{o}{/}\PY{n}{den}
        
        \PY{k}{def} \PY{n+nf}{d2}\PY{p}{(}\PY{n}{S\PYZus{}t}\PY{p}{,} \PY{n}{K}\PY{p}{,} \PY{n}{r}\PY{p}{,} \PY{n}{vol}\PY{p}{,} \PY{n}{ttm}\PY{p}{)}\PY{p}{:}
            \PY{k}{return} \PY{n}{d1}\PY{p}{(}\PY{n}{S\PYZus{}t}\PY{p}{,} \PY{n}{K}\PY{p}{,} \PY{n}{r}\PY{p}{,} \PY{n}{vol}\PY{p}{,} \PY{n}{ttm}\PY{p}{)} \PY{o}{\PYZhy{}} \PY{n}{vol} \PY{o}{*} \PY{n}{sqrt}\PY{p}{(}\PY{n}{ttm}\PY{p}{)}
        
        \PY{k}{def} \PY{n+nf}{call}\PY{p}{(}\PY{n}{S\PYZus{}t}\PY{p}{,} \PY{n}{K}\PY{p}{,} \PY{n}{r}\PY{p}{,} \PY{n}{vol}\PY{p}{,} \PY{n}{ttm}\PY{p}{)}\PY{p}{:}
            \PY{k}{return} \PY{n}{S\PYZus{}t} \PY{o}{*} \PY{n}{norm}\PY{o}{.}\PY{n}{cdf}\PY{p}{(}\PY{n}{d1}\PY{p}{(}\PY{n}{S\PYZus{}t}\PY{p}{,} \PY{n}{K}\PY{p}{,} \PY{n}{r}\PY{p}{,} \PY{n}{vol}\PY{p}{,} \PY{n}{ttm}\PY{p}{)}\PY{p}{)} \PYZbs{}
               \PY{o}{\PYZhy{}} \PY{n}{K} \PY{o}{*} \PY{n}{exp}\PY{p}{(}\PY{o}{\PYZhy{}}\PY{n}{r} \PY{o}{*} \PY{n}{ttm}\PY{p}{)} \PY{o}{*} \PY{n}{norm}\PY{o}{.}\PY{n}{cdf}\PY{p}{(}\PY{n}{d2}\PY{p}{(}\PY{n}{S\PYZus{}t}\PY{p}{,} \PY{n}{K}\PY{p}{,} \PY{n}{r}\PY{p}{,} \PY{n}{vol}\PY{p}{,} \PY{n}{ttm}\PY{p}{)}\PY{p}{)}
        
        \PY{n}{s} \PY{o}{=} \PY{l+m+mi}{800}
        \PY{c+c1}{\PYZsh{} strikes expressed as \PYZpc{} of spot price}
        \PY{n}{moneyness} \PY{o}{=} \PY{p}{[} \PY{l+m+mf}{0.5}\PY{p}{,} \PY{l+m+mf}{0.75}\PY{p}{,} \PY{l+m+mf}{0.825}\PY{p}{,} \PYZbs{}
                     \PY{l+m+mf}{1.0}\PY{p}{,} \PY{l+m+mf}{1.125}\PY{p}{,} \PY{l+m+mf}{1.25}\PY{p}{,} \PY{l+m+mf}{1.5} \PY{p}{]}
        \PY{n}{vol} \PY{o}{=} \PY{l+m+mf}{0.3}
        \PY{n}{ttm} \PY{o}{=} \PY{l+m+mf}{0.75}
        \PY{n}{r} \PY{o}{=} \PY{l+m+mf}{0.005}
        
        \PY{n}{result} \PY{o}{=} \PY{p}{\PYZob{}}\PY{p}{\PYZcb{}}
        \PY{k}{for} \PY{n}{m} \PY{o+ow}{in} \PY{n}{moneyness}\PY{p}{:}
            \PY{n}{result}\PY{p}{[}\PY{n}{s}\PY{o}{*}\PY{n}{m}\PY{p}{]} \PY{o}{=} \PY{n}{call}\PY{p}{(}\PY{n}{s}\PY{p}{,} \PY{n}{m}\PY{o}{*}\PY{n}{s}\PY{p}{,} \PY{n}{r}\PY{p}{,} \PY{n}{vol}\PY{p}{,} \PY{n}{ttm}\PY{p}{)}
        \PY{n}{result}
\end{Verbatim}

\begin{Verbatim}[commandchars=\\\{\}]
{\color{outcolor}Out[{\color{outcolor}1}]:} \{400.0: 401.66074527896365,
         600.0: 213.9883852521275,
         660.0: 166.85957363897393,
         800.0: 84.03697017660357,
         900.0: 47.61880394696229,
         1000.0: 25.632722952585738,
         1200.0: 6.655275227771156\}
\end{Verbatim}
            
    \hypertarget{exercise-3.2}{%
\subsubsection{Exercise 3.2}\label{exercise-3.2}}

Python has a useful command called \texttt{assert} which can be used for
checking that a given condition is satisfied, and raising an error if
the condition is not satisfied.

The following line does not cause an error, in fact it does nothing

\begin{Shaded}
\begin{Highlighting}[]
\ControlFlowTok{assert} \DecValTok{1} \OperatorTok{<} \DecValTok{2}
\end{Highlighting}
\end{Shaded}

This causes an error

\begin{Shaded}
\begin{Highlighting}[]
\ControlFlowTok{assert} \DecValTok{1} \OperatorTok{>} \DecValTok{2}
\end{Highlighting}
\end{Shaded}

\texttt{assert} can take a second argument with a message to display in
case of failure.

\begin{Shaded}
\begin{Highlighting}[]
\ControlFlowTok{assert} \DecValTok{1} \OperatorTok{>} \DecValTok{2}\NormalTok{, }\StringTok{"Two is bigger than one"}
\end{Highlighting}
\end{Shaded}

Take the \texttt{df} function from lesson 3 and modify it by adding some
assertions to check that:

\begin{itemize}
\tightlist
\item
  the pillar date list contains at least 2 elements;
\item
  the pillar date list is the same length as the discount factor list;
\item
  the first pillar date is equal to the today date;
\item
  the value date argument `d' is greater or equal to the first pillar
  date and also less than or equal to the last pillar date.
\end{itemize}

Then try using the function with some invalid data to make sure that
your as sertions are correctly checking the desired conditions

\textbf{Solution:}

    \begin{Verbatim}[commandchars=\\\{\}]
{\color{incolor}In [{\color{incolor}2}]:} \PY{c+c1}{\PYZsh{} import modules and objects that we need}
        \PY{k+kn}{from} \PY{n+nn}{datetime} \PY{k}{import} \PY{n}{date}
        \PY{k+kn}{import} \PY{n+nn}{numpy}
        \PY{k+kn}{import} \PY{n+nn}{math}
        
        \PY{c+c1}{\PYZsh{} define the input data}
        \PY{n}{today\PYZus{}date} \PY{o}{=} \PY{n}{date}\PY{p}{(}\PY{l+m+mi}{2017}\PY{p}{,} \PY{l+m+mi}{10}\PY{p}{,} \PY{l+m+mi}{1}\PY{p}{)}
        \PY{n}{pillar\PYZus{}dates} \PY{o}{=} \PY{p}{[}\PY{n}{date}\PY{p}{(}\PY{l+m+mi}{2017}\PY{p}{,} \PY{l+m+mi}{10}\PY{p}{,} \PY{l+m+mi}{1}\PY{p}{)}\PY{p}{,} 
                        \PY{n}{date}\PY{p}{(}\PY{l+m+mi}{2018}\PY{p}{,} \PY{l+m+mi}{10}\PY{p}{,} \PY{l+m+mi}{1}\PY{p}{)}\PY{p}{,} 
                        \PY{n}{date}\PY{p}{(}\PY{l+m+mi}{2019}\PY{p}{,} \PY{l+m+mi}{10}\PY{p}{,} \PY{l+m+mi}{1}\PY{p}{)}\PY{p}{]}
        \PY{n}{discount\PYZus{}factors} \PY{o}{=} \PY{p}{[}\PY{l+m+mf}{1.0}\PY{p}{,} \PY{l+m+mf}{0.95}\PY{p}{,} \PY{l+m+mf}{0.8}\PY{p}{]}
        
        \PY{c+c1}{\PYZsh{} define the df function}
        \PY{k}{def} \PY{n+nf}{df}\PY{p}{(}\PY{n}{t}\PY{p}{)}\PY{p}{:}
            \PY{c+c1}{\PYZsh{}\PYZsh{}\PYZsh{}\PYZsh{}\PYZsh{}\PYZsh{}\PYZsh{}\PYZsh{}\PYZsh{}\PYZsh{}\PYZsh{}\PYZsh{}\PYZsh{}\PYZsh{} CHECKS \PYZsh{}\PYZsh{}\PYZsh{}\PYZsh{}\PYZsh{}\PYZsh{}\PYZsh{}\PYZsh{}\PYZsh{}\PYZsh{}\PYZsh{}\PYZsh{}\PYZsh{}\PYZsh{}\PYZsh{}\PYZsh{}}
            \PY{c+c1}{\PYZsh{} Check that there are at least 2 pillar dates}
            \PY{k}{assert} \PY{n+nb}{len}\PY{p}{(}\PY{n}{pillar\PYZus{}dates}\PY{p}{)} \PY{o}{\PYZgt{}}\PY{o}{=} \PY{l+m+mi}{2}\PY{p}{,} \PY{l+s+s2}{\PYZdq{}}\PY{l+s+s2}{ need at least 2 pillar dates}\PY{l+s+s2}{\PYZdq{}}
            
            \PY{c+c1}{\PYZsh{} Check that the number of pillar dates }
            \PY{c+c1}{\PYZsh{}is equal to the number of pillar discount factors}
            \PY{k}{assert} \PY{n+nb}{len}\PY{p}{(}\PY{n}{pillar\PYZus{}dates}\PY{p}{)} \PY{o}{==} \PY{n+nb}{len}\PY{p}{(}\PY{n}{discount\PYZus{}factors}\PY{p}{)}\PY{p}{,} \PYZbs{}
                \PY{l+s+s2}{\PYZdq{}}\PY{l+s+s2}{number of pillar dates should be equal to }\PY{l+s+se}{\PYZbs{}}
        \PY{l+s+s2}{        the number of pillar discount factors}\PY{l+s+s2}{\PYZdq{}}
            
            \PY{c+c1}{\PYZsh{} Check that the first pillar date is the today date}
            \PY{k}{assert} \PY{n}{today\PYZus{}date} \PY{o}{==} \PY{n}{pillar\PYZus{}dates}\PY{p}{[}\PY{l+m+mi}{0}\PY{p}{]}\PY{p}{,} \PYZbs{}
                \PY{l+s+s2}{\PYZdq{}}\PY{l+s+s2}{first pillar date should be the today date}\PY{l+s+s2}{\PYZdq{}}
            
            \PY{c+c1}{\PYZsh{} Check that the value date argument is between }
            \PY{c+c1}{\PYZsh{}the first and last pillar dates}
            \PY{k}{assert} \PY{n}{pillar\PYZus{}dates}\PY{p}{[}\PY{l+m+mi}{0}\PY{p}{]} \PY{o}{\PYZlt{}}\PY{o}{=} \PY{n}{t} \PY{o}{\PYZlt{}}\PY{o}{=} \PY{n}{pillar\PYZus{}dates}\PY{p}{[}\PY{o}{\PYZhy{}}\PY{l+m+mi}{1}\PY{p}{]}\PY{p}{,} \PYZbs{}
                \PY{l+s+s2}{\PYZdq{}}\PY{l+s+s2}{Invalid value date }\PY{l+s+si}{\PYZpc{}s}\PY{l+s+s2}{\PYZdq{}} \PY{o}{\PYZpc{}} \PY{p}{(}\PY{n}{d}\PY{p}{)}
            \PY{c+c1}{\PYZsh{}\PYZsh{}\PYZsh{}\PYZsh{}\PYZsh{}\PYZsh{}\PYZsh{}\PYZsh{}\PYZsh{}\PYZsh{}\PYZsh{}\PYZsh{}\PYZsh{}\PYZsh{} END OF CHECKS \PYZsh{}\PYZsh{}\PYZsh{}\PYZsh{}\PYZsh{}\PYZsh{}\PYZsh{}\PYZsh{}\PYZsh{}\PYZsh{}\PYZsh{}\PYZsh{}\PYZsh{}\PYZsh{}\PYZsh{}\PYZsh{}}
            
            \PY{n}{log\PYZus{}discount\PYZus{}factors} \PY{o}{=} \PY{p}{[}\PY{p}{]}
            \PY{k}{for} \PY{n}{discount\PYZus{}factor} \PY{o+ow}{in} \PY{n}{discount\PYZus{}factors}\PY{p}{:}
                \PY{n}{log\PYZus{}discount\PYZus{}factors}\PY{o}{.}\PY{n}{append}\PY{p}{(}\PY{n}{math}\PY{o}{.}\PY{n}{log}\PY{p}{(}\PY{n}{discount\PYZus{}factor}\PY{p}{)}\PY{p}{)}
            
            \PY{n}{pillar\PYZus{}days} \PY{o}{=} \PY{p}{[}\PY{p}{]}
            \PY{k}{for} \PY{n}{pillar\PYZus{}date} \PY{o+ow}{in} \PY{n}{pillar\PYZus{}dates}\PY{p}{:}
                \PY{n}{pillar\PYZus{}days}\PY{o}{.}\PY{n}{append}\PY{p}{(}\PY{p}{(}\PY{n}{pillar\PYZus{}date} \PY{o}{\PYZhy{}} \PY{n}{today\PYZus{}date}\PY{p}{)}\PY{o}{.}\PY{n}{days}\PY{p}{)}
            
            \PY{n}{t\PYZus{}days} \PY{o}{=} \PY{p}{(}\PY{n}{t} \PY{o}{\PYZhy{}} \PY{n}{today\PYZus{}date}\PY{p}{)}\PY{o}{.}\PY{n}{days}
            
            \PY{n}{interpolated\PYZus{}log\PYZus{}discount\PYZus{}factor} \PY{o}{=} \PYZbs{}
                \PY{n}{numpy}\PY{o}{.}\PY{n}{interp}\PY{p}{(}\PY{n}{t\PYZus{}days}\PY{p}{,} \PY{n}{pillar\PYZus{}days}\PY{p}{,} \PY{n}{log\PYZus{}discount\PYZus{}factors}\PY{p}{)}
            
            \PY{k}{return} \PY{n}{math}\PY{o}{.}\PY{n}{exp}\PY{p}{(}\PY{n}{interpolated\PYZus{}log\PYZus{}discount\PYZus{}factor}\PY{p}{)}
        
        \PY{n}{df}\PY{p}{(}\PY{n}{date}\PY{p}{(}\PY{l+m+mi}{2019}\PY{p}{,} \PY{l+m+mi}{1}\PY{p}{,} \PY{l+m+mi}{1}\PY{p}{)}\PY{p}{)}
\end{Verbatim}

\begin{Verbatim}[commandchars=\\\{\}]
{\color{outcolor}Out[{\color{outcolor}2}]:} 0.9097285910181567
\end{Verbatim}
            
    \hypertarget{exercise-3.3}{%
\subsubsection{Exercise 3.3}\label{exercise-3.3}}

Python comes with a module called \texttt{matplotlib} which can be used
for plotting graphs and charts. In particular, we can use a sub-module
called \texttt{pyplot} which provides slightly easier-to-use interface
for plotting interactively. Use this function to plot the call prices
from exercise 3.1

\textbf{Solution:}

    \begin{Verbatim}[commandchars=\\\{\}]
{\color{incolor}In [{\color{incolor}4}]:} \PY{k+kn}{from} \PY{n+nn}{matplotlib} \PY{k}{import} \PY{n}{pyplot}
        
        \PY{n}{strikes} \PY{o}{=} \PY{p}{[}\PY{n}{s}\PY{o}{*}\PY{n}{m} \PY{k}{for} \PY{n}{m} \PY{o+ow}{in} \PY{n}{moneyness}\PY{p}{]}
        \PY{n}{call\PYZus{}prices} \PY{o}{=} \PY{p}{[}\PY{n}{call}\PY{p}{(}\PY{n}{s}\PY{p}{,} \PY{n}{k}\PY{p}{,} \PY{n}{r}\PY{p}{,} \PY{n}{vol}\PY{p}{,} \PY{n}{ttm}\PY{p}{)} \PY{k}{for} \PY{n}{k} \PY{o+ow}{in} \PY{n}{strikes}\PY{p}{]}
        \PY{n}{pyplot}\PY{o}{.}\PY{n}{plot}\PY{p}{(}\PY{n}{strikes}\PY{p}{,} \PY{n}{call\PYZus{}prices}\PY{p}{,} \PY{n}{marker}\PY{o}{=}\PY{l+s+s1}{\PYZsq{}}\PY{l+s+s1}{o}\PY{l+s+s1}{\PYZsq{}}\PY{p}{)}
        \PY{c+c1}{\PYZsh{} the next line saves the picture in a file (it is then shown in repl.it)}
        \PY{c+c1}{\PYZsh{}pyplot.savefig(\PYZsq{}graph4.png\PYZsq{}) }
\end{Verbatim}

\begin{Verbatim}[commandchars=\\\{\}]
{\color{outcolor}Out[{\color{outcolor}4}]:} [<matplotlib.lines.Line2D at 0x7f5362ff89e8>]
\end{Verbatim}
            
    \begin{center}
    \adjustimage{max size={0.9\linewidth}{0.9\paperheight}}{lesson3_solutions_files/lesson3_solutions_6_1.png}
    \end{center}
    { \hspace*{\fill} \\}
    

    % Add a bibliography block to the postdoc
    
    
    
    \end{document}

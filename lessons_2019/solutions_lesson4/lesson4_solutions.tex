
% Default to the notebook output style

    


% Inherit from the specified cell style.




    
\documentclass[11pt]{article}

    
    
    \usepackage[T1]{fontenc}
    % Nicer default font (+ math font) than Computer Modern for most use cases
    \usepackage{mathpazo}

    % Basic figure setup, for now with no caption control since it's done
    % automatically by Pandoc (which extracts ![](path) syntax from Markdown).
    \usepackage{graphicx}
    % We will generate all images so they have a width \maxwidth. This means
    % that they will get their normal width if they fit onto the page, but
    % are scaled down if they would overflow the margins.
    \makeatletter
    \def\maxwidth{\ifdim\Gin@nat@width>\linewidth\linewidth
    \else\Gin@nat@width\fi}
    \makeatother
    \let\Oldincludegraphics\includegraphics
    % Set max figure width to be 80% of text width, for now hardcoded.
    \renewcommand{\includegraphics}[1]{\Oldincludegraphics[width=.8\maxwidth]{#1}}
    % Ensure that by default, figures have no caption (until we provide a
    % proper Figure object with a Caption API and a way to capture that
    % in the conversion process - todo).
    \usepackage{caption}
    \DeclareCaptionLabelFormat{nolabel}{}
    \captionsetup{labelformat=nolabel}

    \usepackage{adjustbox} % Used to constrain images to a maximum size 
    \usepackage{xcolor} % Allow colors to be defined
    \usepackage{enumerate} % Needed for markdown enumerations to work
    \usepackage{geometry} % Used to adjust the document margins
    \usepackage{amsmath} % Equations
    \usepackage{amssymb} % Equations
    \usepackage{textcomp} % defines textquotesingle
    % Hack from http://tex.stackexchange.com/a/47451/13684:
    \AtBeginDocument{%
        \def\PYZsq{\textquotesingle}% Upright quotes in Pygmentized code
    }
    \usepackage{upquote} % Upright quotes for verbatim code
    \usepackage{eurosym} % defines \euro
    \usepackage[mathletters]{ucs} % Extended unicode (utf-8) support
    \usepackage[utf8x]{inputenc} % Allow utf-8 characters in the tex document
    \usepackage{fancyvrb} % verbatim replacement that allows latex
    \usepackage{grffile} % extends the file name processing of package graphics 
                         % to support a larger range 
    % The hyperref package gives us a pdf with properly built
    % internal navigation ('pdf bookmarks' for the table of contents,
    % internal cross-reference links, web links for URLs, etc.)
    \usepackage{hyperref}
    \usepackage{longtable} % longtable support required by pandoc >1.10
    \usepackage{booktabs}  % table support for pandoc > 1.12.2
    \usepackage[inline]{enumitem} % IRkernel/repr support (it uses the enumerate* environment)
    \usepackage[normalem]{ulem} % ulem is needed to support strikethroughs (\sout)
                                % normalem makes italics be italics, not underlines
    \usepackage{mathrsfs}
    

    
    
    % Colors for the hyperref package
    \definecolor{urlcolor}{rgb}{0,.145,.698}
    \definecolor{linkcolor}{rgb}{.71,0.21,0.01}
    \definecolor{citecolor}{rgb}{.12,.54,.11}

    % ANSI colors
    \definecolor{ansi-black}{HTML}{3E424D}
    \definecolor{ansi-black-intense}{HTML}{282C36}
    \definecolor{ansi-red}{HTML}{E75C58}
    \definecolor{ansi-red-intense}{HTML}{B22B31}
    \definecolor{ansi-green}{HTML}{00A250}
    \definecolor{ansi-green-intense}{HTML}{007427}
    \definecolor{ansi-yellow}{HTML}{DDB62B}
    \definecolor{ansi-yellow-intense}{HTML}{B27D12}
    \definecolor{ansi-blue}{HTML}{208FFB}
    \definecolor{ansi-blue-intense}{HTML}{0065CA}
    \definecolor{ansi-magenta}{HTML}{D160C4}
    \definecolor{ansi-magenta-intense}{HTML}{A03196}
    \definecolor{ansi-cyan}{HTML}{60C6C8}
    \definecolor{ansi-cyan-intense}{HTML}{258F8F}
    \definecolor{ansi-white}{HTML}{C5C1B4}
    \definecolor{ansi-white-intense}{HTML}{A1A6B2}
    \definecolor{ansi-default-inverse-fg}{HTML}{FFFFFF}
    \definecolor{ansi-default-inverse-bg}{HTML}{000000}

    % commands and environments needed by pandoc snippets
    % extracted from the output of `pandoc -s`
    \providecommand{\tightlist}{%
      \setlength{\itemsep}{0pt}\setlength{\parskip}{0pt}}
    \DefineVerbatimEnvironment{Highlighting}{Verbatim}{commandchars=\\\{\}}
    % Add ',fontsize=\small' for more characters per line
    \newenvironment{Shaded}{}{}
    \newcommand{\KeywordTok}[1]{\textcolor[rgb]{0.00,0.44,0.13}{\textbf{{#1}}}}
    \newcommand{\DataTypeTok}[1]{\textcolor[rgb]{0.56,0.13,0.00}{{#1}}}
    \newcommand{\DecValTok}[1]{\textcolor[rgb]{0.25,0.63,0.44}{{#1}}}
    \newcommand{\BaseNTok}[1]{\textcolor[rgb]{0.25,0.63,0.44}{{#1}}}
    \newcommand{\FloatTok}[1]{\textcolor[rgb]{0.25,0.63,0.44}{{#1}}}
    \newcommand{\CharTok}[1]{\textcolor[rgb]{0.25,0.44,0.63}{{#1}}}
    \newcommand{\StringTok}[1]{\textcolor[rgb]{0.25,0.44,0.63}{{#1}}}
    \newcommand{\CommentTok}[1]{\textcolor[rgb]{0.38,0.63,0.69}{\textit{{#1}}}}
    \newcommand{\OtherTok}[1]{\textcolor[rgb]{0.00,0.44,0.13}{{#1}}}
    \newcommand{\AlertTok}[1]{\textcolor[rgb]{1.00,0.00,0.00}{\textbf{{#1}}}}
    \newcommand{\FunctionTok}[1]{\textcolor[rgb]{0.02,0.16,0.49}{{#1}}}
    \newcommand{\RegionMarkerTok}[1]{{#1}}
    \newcommand{\ErrorTok}[1]{\textcolor[rgb]{1.00,0.00,0.00}{\textbf{{#1}}}}
    \newcommand{\NormalTok}[1]{{#1}}
    
    % Additional commands for more recent versions of Pandoc
    \newcommand{\ConstantTok}[1]{\textcolor[rgb]{0.53,0.00,0.00}{{#1}}}
    \newcommand{\SpecialCharTok}[1]{\textcolor[rgb]{0.25,0.44,0.63}{{#1}}}
    \newcommand{\VerbatimStringTok}[1]{\textcolor[rgb]{0.25,0.44,0.63}{{#1}}}
    \newcommand{\SpecialStringTok}[1]{\textcolor[rgb]{0.73,0.40,0.53}{{#1}}}
    \newcommand{\ImportTok}[1]{{#1}}
    \newcommand{\DocumentationTok}[1]{\textcolor[rgb]{0.73,0.13,0.13}{\textit{{#1}}}}
    \newcommand{\AnnotationTok}[1]{\textcolor[rgb]{0.38,0.63,0.69}{\textbf{\textit{{#1}}}}}
    \newcommand{\CommentVarTok}[1]{\textcolor[rgb]{0.38,0.63,0.69}{\textbf{\textit{{#1}}}}}
    \newcommand{\VariableTok}[1]{\textcolor[rgb]{0.10,0.09,0.49}{{#1}}}
    \newcommand{\ControlFlowTok}[1]{\textcolor[rgb]{0.00,0.44,0.13}{\textbf{{#1}}}}
    \newcommand{\OperatorTok}[1]{\textcolor[rgb]{0.40,0.40,0.40}{{#1}}}
    \newcommand{\BuiltInTok}[1]{{#1}}
    \newcommand{\ExtensionTok}[1]{{#1}}
    \newcommand{\PreprocessorTok}[1]{\textcolor[rgb]{0.74,0.48,0.00}{{#1}}}
    \newcommand{\AttributeTok}[1]{\textcolor[rgb]{0.49,0.56,0.16}{{#1}}}
    \newcommand{\InformationTok}[1]{\textcolor[rgb]{0.38,0.63,0.69}{\textbf{\textit{{#1}}}}}
    \newcommand{\WarningTok}[1]{\textcolor[rgb]{0.38,0.63,0.69}{\textbf{\textit{{#1}}}}}
    
    
    % Define a nice break command that doesn't care if a line doesn't already
    % exist.
    \def\br{\hspace*{\fill} \\* }
    % Math Jax compatibility definitions
    \def\gt{>}
    \def\lt{<}
    \let\Oldtex\TeX
    \let\Oldlatex\LaTeX
    \renewcommand{\TeX}{\textrm{\Oldtex}}
    \renewcommand{\LaTeX}{\textrm{\Oldlatex}}
    % Document parameters
    % Document title
    \title{Solutions - Practical Lesson 4}
    \author{Matteo Sani \\\href{mailto:matteosan1@gmail.com}{matteosan1@gmail.com}}
    
    
    
    
    

    % Pygments definitions
    
\makeatletter
\def\PY@reset{\let\PY@it=\relax \let\PY@bf=\relax%
    \let\PY@ul=\relax \let\PY@tc=\relax%
    \let\PY@bc=\relax \let\PY@ff=\relax}
\def\PY@tok#1{\csname PY@tok@#1\endcsname}
\def\PY@toks#1+{\ifx\relax#1\empty\else%
    \PY@tok{#1}\expandafter\PY@toks\fi}
\def\PY@do#1{\PY@bc{\PY@tc{\PY@ul{%
    \PY@it{\PY@bf{\PY@ff{#1}}}}}}}
\def\PY#1#2{\PY@reset\PY@toks#1+\relax+\PY@do{#2}}

\expandafter\def\csname PY@tok@w\endcsname{\def\PY@tc##1{\textcolor[rgb]{0.73,0.73,0.73}{##1}}}
\expandafter\def\csname PY@tok@c\endcsname{\let\PY@it=\textit\def\PY@tc##1{\textcolor[rgb]{0.25,0.50,0.50}{##1}}}
\expandafter\def\csname PY@tok@cp\endcsname{\def\PY@tc##1{\textcolor[rgb]{0.74,0.48,0.00}{##1}}}
\expandafter\def\csname PY@tok@k\endcsname{\let\PY@bf=\textbf\def\PY@tc##1{\textcolor[rgb]{0.00,0.50,0.00}{##1}}}
\expandafter\def\csname PY@tok@kp\endcsname{\def\PY@tc##1{\textcolor[rgb]{0.00,0.50,0.00}{##1}}}
\expandafter\def\csname PY@tok@kt\endcsname{\def\PY@tc##1{\textcolor[rgb]{0.69,0.00,0.25}{##1}}}
\expandafter\def\csname PY@tok@o\endcsname{\def\PY@tc##1{\textcolor[rgb]{0.40,0.40,0.40}{##1}}}
\expandafter\def\csname PY@tok@ow\endcsname{\let\PY@bf=\textbf\def\PY@tc##1{\textcolor[rgb]{0.67,0.13,1.00}{##1}}}
\expandafter\def\csname PY@tok@nb\endcsname{\def\PY@tc##1{\textcolor[rgb]{0.00,0.50,0.00}{##1}}}
\expandafter\def\csname PY@tok@nf\endcsname{\def\PY@tc##1{\textcolor[rgb]{0.00,0.00,1.00}{##1}}}
\expandafter\def\csname PY@tok@nc\endcsname{\let\PY@bf=\textbf\def\PY@tc##1{\textcolor[rgb]{0.00,0.00,1.00}{##1}}}
\expandafter\def\csname PY@tok@nn\endcsname{\let\PY@bf=\textbf\def\PY@tc##1{\textcolor[rgb]{0.00,0.00,1.00}{##1}}}
\expandafter\def\csname PY@tok@ne\endcsname{\let\PY@bf=\textbf\def\PY@tc##1{\textcolor[rgb]{0.82,0.25,0.23}{##1}}}
\expandafter\def\csname PY@tok@nv\endcsname{\def\PY@tc##1{\textcolor[rgb]{0.10,0.09,0.49}{##1}}}
\expandafter\def\csname PY@tok@no\endcsname{\def\PY@tc##1{\textcolor[rgb]{0.53,0.00,0.00}{##1}}}
\expandafter\def\csname PY@tok@nl\endcsname{\def\PY@tc##1{\textcolor[rgb]{0.63,0.63,0.00}{##1}}}
\expandafter\def\csname PY@tok@ni\endcsname{\let\PY@bf=\textbf\def\PY@tc##1{\textcolor[rgb]{0.60,0.60,0.60}{##1}}}
\expandafter\def\csname PY@tok@na\endcsname{\def\PY@tc##1{\textcolor[rgb]{0.49,0.56,0.16}{##1}}}
\expandafter\def\csname PY@tok@nt\endcsname{\let\PY@bf=\textbf\def\PY@tc##1{\textcolor[rgb]{0.00,0.50,0.00}{##1}}}
\expandafter\def\csname PY@tok@nd\endcsname{\def\PY@tc##1{\textcolor[rgb]{0.67,0.13,1.00}{##1}}}
\expandafter\def\csname PY@tok@s\endcsname{\def\PY@tc##1{\textcolor[rgb]{0.73,0.13,0.13}{##1}}}
\expandafter\def\csname PY@tok@sd\endcsname{\let\PY@it=\textit\def\PY@tc##1{\textcolor[rgb]{0.73,0.13,0.13}{##1}}}
\expandafter\def\csname PY@tok@si\endcsname{\let\PY@bf=\textbf\def\PY@tc##1{\textcolor[rgb]{0.73,0.40,0.53}{##1}}}
\expandafter\def\csname PY@tok@se\endcsname{\let\PY@bf=\textbf\def\PY@tc##1{\textcolor[rgb]{0.73,0.40,0.13}{##1}}}
\expandafter\def\csname PY@tok@sr\endcsname{\def\PY@tc##1{\textcolor[rgb]{0.73,0.40,0.53}{##1}}}
\expandafter\def\csname PY@tok@ss\endcsname{\def\PY@tc##1{\textcolor[rgb]{0.10,0.09,0.49}{##1}}}
\expandafter\def\csname PY@tok@sx\endcsname{\def\PY@tc##1{\textcolor[rgb]{0.00,0.50,0.00}{##1}}}
\expandafter\def\csname PY@tok@m\endcsname{\def\PY@tc##1{\textcolor[rgb]{0.40,0.40,0.40}{##1}}}
\expandafter\def\csname PY@tok@gh\endcsname{\let\PY@bf=\textbf\def\PY@tc##1{\textcolor[rgb]{0.00,0.00,0.50}{##1}}}
\expandafter\def\csname PY@tok@gu\endcsname{\let\PY@bf=\textbf\def\PY@tc##1{\textcolor[rgb]{0.50,0.00,0.50}{##1}}}
\expandafter\def\csname PY@tok@gd\endcsname{\def\PY@tc##1{\textcolor[rgb]{0.63,0.00,0.00}{##1}}}
\expandafter\def\csname PY@tok@gi\endcsname{\def\PY@tc##1{\textcolor[rgb]{0.00,0.63,0.00}{##1}}}
\expandafter\def\csname PY@tok@gr\endcsname{\def\PY@tc##1{\textcolor[rgb]{1.00,0.00,0.00}{##1}}}
\expandafter\def\csname PY@tok@ge\endcsname{\let\PY@it=\textit}
\expandafter\def\csname PY@tok@gs\endcsname{\let\PY@bf=\textbf}
\expandafter\def\csname PY@tok@gp\endcsname{\let\PY@bf=\textbf\def\PY@tc##1{\textcolor[rgb]{0.00,0.00,0.50}{##1}}}
\expandafter\def\csname PY@tok@go\endcsname{\def\PY@tc##1{\textcolor[rgb]{0.53,0.53,0.53}{##1}}}
\expandafter\def\csname PY@tok@gt\endcsname{\def\PY@tc##1{\textcolor[rgb]{0.00,0.27,0.87}{##1}}}
\expandafter\def\csname PY@tok@err\endcsname{\def\PY@bc##1{\setlength{\fboxsep}{0pt}\fcolorbox[rgb]{1.00,0.00,0.00}{1,1,1}{\strut ##1}}}
\expandafter\def\csname PY@tok@kc\endcsname{\let\PY@bf=\textbf\def\PY@tc##1{\textcolor[rgb]{0.00,0.50,0.00}{##1}}}
\expandafter\def\csname PY@tok@kd\endcsname{\let\PY@bf=\textbf\def\PY@tc##1{\textcolor[rgb]{0.00,0.50,0.00}{##1}}}
\expandafter\def\csname PY@tok@kn\endcsname{\let\PY@bf=\textbf\def\PY@tc##1{\textcolor[rgb]{0.00,0.50,0.00}{##1}}}
\expandafter\def\csname PY@tok@kr\endcsname{\let\PY@bf=\textbf\def\PY@tc##1{\textcolor[rgb]{0.00,0.50,0.00}{##1}}}
\expandafter\def\csname PY@tok@bp\endcsname{\def\PY@tc##1{\textcolor[rgb]{0.00,0.50,0.00}{##1}}}
\expandafter\def\csname PY@tok@fm\endcsname{\def\PY@tc##1{\textcolor[rgb]{0.00,0.00,1.00}{##1}}}
\expandafter\def\csname PY@tok@vc\endcsname{\def\PY@tc##1{\textcolor[rgb]{0.10,0.09,0.49}{##1}}}
\expandafter\def\csname PY@tok@vg\endcsname{\def\PY@tc##1{\textcolor[rgb]{0.10,0.09,0.49}{##1}}}
\expandafter\def\csname PY@tok@vi\endcsname{\def\PY@tc##1{\textcolor[rgb]{0.10,0.09,0.49}{##1}}}
\expandafter\def\csname PY@tok@vm\endcsname{\def\PY@tc##1{\textcolor[rgb]{0.10,0.09,0.49}{##1}}}
\expandafter\def\csname PY@tok@sa\endcsname{\def\PY@tc##1{\textcolor[rgb]{0.73,0.13,0.13}{##1}}}
\expandafter\def\csname PY@tok@sb\endcsname{\def\PY@tc##1{\textcolor[rgb]{0.73,0.13,0.13}{##1}}}
\expandafter\def\csname PY@tok@sc\endcsname{\def\PY@tc##1{\textcolor[rgb]{0.73,0.13,0.13}{##1}}}
\expandafter\def\csname PY@tok@dl\endcsname{\def\PY@tc##1{\textcolor[rgb]{0.73,0.13,0.13}{##1}}}
\expandafter\def\csname PY@tok@s2\endcsname{\def\PY@tc##1{\textcolor[rgb]{0.73,0.13,0.13}{##1}}}
\expandafter\def\csname PY@tok@sh\endcsname{\def\PY@tc##1{\textcolor[rgb]{0.73,0.13,0.13}{##1}}}
\expandafter\def\csname PY@tok@s1\endcsname{\def\PY@tc##1{\textcolor[rgb]{0.73,0.13,0.13}{##1}}}
\expandafter\def\csname PY@tok@mb\endcsname{\def\PY@tc##1{\textcolor[rgb]{0.40,0.40,0.40}{##1}}}
\expandafter\def\csname PY@tok@mf\endcsname{\def\PY@tc##1{\textcolor[rgb]{0.40,0.40,0.40}{##1}}}
\expandafter\def\csname PY@tok@mh\endcsname{\def\PY@tc##1{\textcolor[rgb]{0.40,0.40,0.40}{##1}}}
\expandafter\def\csname PY@tok@mi\endcsname{\def\PY@tc##1{\textcolor[rgb]{0.40,0.40,0.40}{##1}}}
\expandafter\def\csname PY@tok@il\endcsname{\def\PY@tc##1{\textcolor[rgb]{0.40,0.40,0.40}{##1}}}
\expandafter\def\csname PY@tok@mo\endcsname{\def\PY@tc##1{\textcolor[rgb]{0.40,0.40,0.40}{##1}}}
\expandafter\def\csname PY@tok@ch\endcsname{\let\PY@it=\textit\def\PY@tc##1{\textcolor[rgb]{0.25,0.50,0.50}{##1}}}
\expandafter\def\csname PY@tok@cm\endcsname{\let\PY@it=\textit\def\PY@tc##1{\textcolor[rgb]{0.25,0.50,0.50}{##1}}}
\expandafter\def\csname PY@tok@cpf\endcsname{\let\PY@it=\textit\def\PY@tc##1{\textcolor[rgb]{0.25,0.50,0.50}{##1}}}
\expandafter\def\csname PY@tok@c1\endcsname{\let\PY@it=\textit\def\PY@tc##1{\textcolor[rgb]{0.25,0.50,0.50}{##1}}}
\expandafter\def\csname PY@tok@cs\endcsname{\let\PY@it=\textit\def\PY@tc##1{\textcolor[rgb]{0.25,0.50,0.50}{##1}}}

\def\PYZbs{\char`\\}
\def\PYZus{\char`\_}
\def\PYZob{\char`\{}
\def\PYZcb{\char`\}}
\def\PYZca{\char`\^}
\def\PYZam{\char`\&}
\def\PYZlt{\char`\<}
\def\PYZgt{\char`\>}
\def\PYZsh{\char`\#}
\def\PYZpc{\char`\%}
\def\PYZdl{\char`\$}
\def\PYZhy{\char`\-}
\def\PYZsq{\char`\'}
\def\PYZdq{\char`\"}
\def\PYZti{\char`\~}
% for compatibility with earlier versions
\def\PYZat{@}
\def\PYZlb{[}
\def\PYZrb{]}
\makeatother


    % Exact colors from NB
    \definecolor{incolor}{rgb}{0.0, 0.0, 0.5}
    \definecolor{outcolor}{rgb}{0.545, 0.0, 0.0}



    
    % Prevent overflowing lines due to hard-to-break entities
    \sloppy 
    % Setup hyperref package
    \hypersetup{
      breaklinks=true,  % so long urls are correctly broken across lines
      colorlinks=true,
      urlcolor=urlcolor,
      linkcolor=linkcolor,
      citecolor=citecolor,
      }
    % Slightly bigger margins than the latex defaults
    
    \geometry{verbose,tmargin=1in,bmargin=1in,lmargin=1in,rmargin=1in}
    
    

    \begin{document}
    
    
    \maketitle
    
    

    
    \hypertarget{solutions---practical-lesson-4}{%
\section{Solutions}\label{solutions---practical-lesson-4}}

\hypertarget{exercises}{%
\subsection{Exercises}\label{exercises}}

\hypertarget{exercise-4.1}{%
\subsubsection{Exercise 4.1}\label{exercise-4.1}}

Write two classes, \texttt{Circle} and \texttt{Rectangle} that given the
radius and height, width respectively allow to compute area and
perimeter of the two shapes. Test them with the following:

\begin{Shaded}
\begin{Highlighting}[]
\NormalTok{a_circle }\OperatorTok{=}\NormalTok{ Circle(}\DecValTok{5}\NormalTok{)}
\BuiltInTok{print}\NormalTok{ (}\StringTok{"My circle has an area of }\SpecialCharTok{\{\}}\StringTok{ m**2"}\NormalTok{.}\BuiltInTok{format}\NormalTok{(a_circle.area()))}

\NormalTok{a_rectangle }\OperatorTok{=}\NormalTok{ Rectangle(}\DecValTok{3}\NormalTok{, }\DecValTok{6}\NormalTok{)}
\BuiltInTok{print}\NormalTok{ (}\StringTok{"My rectangle has a perimeter of }\SpecialCharTok{\{\}}\StringTok{ m and an area of }\SpecialCharTok{\{\}}\StringTok{ m**2"}\NormalTok{ \textbackslash{}}
\NormalTok{    .}\BuiltInTok{format}\NormalTok{(a_rectangle.perimeter(), a_rectangle.area()))}
\end{Highlighting}
\end{Shaded}

    \begin{Verbatim}[commandchars=\\\{\}]
{\color{incolor}In [{\color{incolor}1}]:} \PY{k+kn}{from} \PY{n+nn}{math} \PY{k}{import} \PY{n}{pi}
        
        \PY{k}{class} \PY{n+nc}{Circle}\PY{p}{:}
            \PY{k}{def} \PY{n+nf}{\PYZus{}\PYZus{}init\PYZus{}\PYZus{}}\PY{p}{(}\PY{n+nb+bp}{self}\PY{p}{,} \PY{n}{radius}\PY{p}{)}\PY{p}{:}
                \PY{n+nb+bp}{self}\PY{o}{.}\PY{n}{radius} \PY{o}{=} \PY{n}{radius}
        
            \PY{k}{def} \PY{n+nf}{area}\PY{p}{(}\PY{n+nb+bp}{self}\PY{p}{)}\PY{p}{:}
                \PY{k}{return} \PY{n}{pi}\PY{o}{*}\PY{n+nb+bp}{self}\PY{o}{.}\PY{n}{radius}\PY{o}{*}\PY{o}{*}\PY{l+m+mi}{2}
        
        \PY{k}{class} \PY{n+nc}{Rectangle}\PY{p}{:}
            \PY{k}{def} \PY{n+nf}{\PYZus{}\PYZus{}init\PYZus{}\PYZus{}}\PY{p}{(}\PY{n+nb+bp}{self}\PY{p}{,} \PY{n}{width}\PY{p}{,} \PY{n}{height}\PY{p}{)}\PY{p}{:}
                \PY{n+nb+bp}{self}\PY{o}{.}\PY{n}{height} \PY{o}{=} \PY{n}{height}
                \PY{n+nb+bp}{self}\PY{o}{.}\PY{n}{width} \PY{o}{=} \PY{n}{width}
        
            \PY{k}{def} \PY{n+nf}{area}\PY{p}{(}\PY{n+nb+bp}{self}\PY{p}{)}\PY{p}{:}
                \PY{k}{return} \PY{n+nb+bp}{self}\PY{o}{.}\PY{n}{width}\PY{o}{*}\PY{n+nb+bp}{self}\PY{o}{.}\PY{n}{height}
        
            \PY{k}{def} \PY{n+nf}{perimeter}\PY{p}{(}\PY{n+nb+bp}{self}\PY{p}{)}\PY{p}{:}
                \PY{k}{return} \PY{n+nb+bp}{self}\PY{o}{.}\PY{n}{width}\PY{o}{*}\PY{l+m+mi}{2} \PY{o}{+} \PY{n+nb+bp}{self}\PY{o}{.}\PY{n}{height}\PY{o}{*}\PY{l+m+mi}{2}
        
        \PY{n}{circle} \PY{o}{=} \PY{n}{Circle}\PY{p}{(}\PY{l+m+mi}{5}\PY{p}{)}
        \PY{n+nb}{print} \PY{p}{(}\PY{l+s+s2}{\PYZdq{}}\PY{l+s+s2}{My circle area is }\PY{l+s+si}{\PYZob{}:.1f\PYZcb{}}\PY{l+s+s2}{ m**2}\PY{l+s+s2}{\PYZdq{}}\PY{o}{.}\PY{n}{format}\PY{p}{(}\PY{n}{circle}\PY{o}{.}\PY{n}{area}\PY{p}{(}\PY{p}{)}\PY{p}{)}\PY{p}{)}
        
        \PY{n}{rect} \PY{o}{=} \PY{n}{Rectangle}\PY{p}{(}\PY{l+m+mi}{3}\PY{p}{,} \PY{l+m+mi}{6}\PY{p}{)}
        \PY{n+nb}{print} \PY{p}{(}\PY{l+s+s2}{\PYZdq{}}\PY{l+s+s2}{My rect area is }\PY{l+s+si}{\PYZob{}:.1f\PYZcb{}}\PY{l+s+s2}{ m**2 and the }\PY{l+s+s2}{\PYZdq{}}\PYZbs{}
               \PY{l+s+s2}{\PYZdq{}}\PY{l+s+s2}{perimeter is }\PY{l+s+si}{\PYZob{}\PYZcb{}}\PY{l+s+s2}{ m}\PY{l+s+s2}{\PYZdq{}}\PY{o}{.}\PY{n}{format}\PY{p}{(}\PY{n}{rect}\PY{o}{.}\PY{n}{area}\PY{p}{(}\PY{p}{)}\PY{p}{,} \PY{n}{rect}\PY{o}{.}\PY{n}{perimeter}\PY{p}{(}\PY{p}{)}\PY{p}{)}\PY{p}{)}
\end{Verbatim}

    \begin{Verbatim}[commandchars=\\\{\}]
My circle area is 78.5 m**2
My rect area is 18.0 m**2 and the perimeter is 18 m

    \end{Verbatim}

    \hypertarget{exercise-4.2}{%
\subsubsection{Exercise 4.2}\label{exercise-4.2}}

Define a class \texttt{Songs}, its \texttt{\_\_init\_\_} should take as
input a dictionary (\texttt{lyrics} that contains lyrics line by line).
Define a method, \texttt{sing\_me\_a\_song} that prints each element of
the lyrics in his own line. Also test it with the follwing input.

\begin{Shaded}
\begin{Highlighting}[]
\NormalTok{lyrics }\OperatorTok{=}\NormalTok{ \{}\StringTok{"Wonderwall"}\NormalTok{:[}\StringTok{"Today is gonna be the day"}\NormalTok{,}
                        \StringTok{"That they're gonna throw it back to you"}\NormalTok{,}
                        \StringTok{"By now you should've somehow"}\NormalTok{, }\StringTok{"..."}\NormalTok{], }
          \StringTok{"Vado al massimo"}\NormalTok{:  [}\StringTok{"Voglio veder come va a finire"}\NormalTok{,}
                        \StringTok{"Andando al massimo senza frenare"}
                        \StringTok{"Voglio vedere se davvero poi"}\NormalTok{,}
                        \StringTok{"Si va a finir male"}\NormalTok{, }\StringTok{"..."}\NormalTok{]\}}
\end{Highlighting}
\end{Shaded}

    \begin{Verbatim}[commandchars=\\\{\}]
{\color{incolor}In [{\color{incolor}2}]:} \PY{k}{class} \PY{n+nc}{Songs}\PY{p}{:}
            \PY{k}{def} \PY{n+nf}{\PYZus{}\PYZus{}init\PYZus{}\PYZus{}}\PY{p}{(}\PY{n+nb+bp}{self}\PY{p}{,} \PY{n}{lyrics}\PY{p}{)}\PY{p}{:}
                \PY{n+nb+bp}{self}\PY{o}{.}\PY{n}{lyrics} \PY{o}{=} \PY{n}{lyrics}
        
            \PY{k}{def} \PY{n+nf}{sing\PYZus{}me\PYZus{}a\PYZus{}song}\PY{p}{(}\PY{n+nb+bp}{self}\PY{p}{,} \PY{n}{title}\PY{p}{)}\PY{p}{:}
                \PY{n}{song} \PY{o}{=} \PY{n+nb+bp}{self}\PY{o}{.}\PY{n}{lyrics}\PY{p}{[}\PY{n}{title}\PY{p}{]}
                \PY{n+nb}{print} \PY{p}{(}\PY{l+s+s2}{\PYZdq{}}\PY{l+s+s2}{Title: }\PY{l+s+si}{\PYZob{}\PYZcb{}}\PY{l+s+s2}{\PYZdq{}}\PY{o}{.}\PY{n}{format}\PY{p}{(}\PY{n}{title}\PY{p}{)}\PY{p}{)}
                \PY{n+nb}{print} \PY{p}{(}\PY{l+s+s2}{\PYZdq{}}\PY{l+s+s2}{********************}\PY{l+s+s2}{\PYZdq{}}\PY{p}{)}
                \PY{k}{for} \PY{n}{line} \PY{o+ow}{in} \PY{n}{song}\PY{p}{:}
                    \PY{n+nb}{print} \PY{p}{(}\PY{n}{line}\PY{p}{)}
        
        \PY{n}{lyrics} \PY{o}{=} \PY{p}{\PYZob{}}\PY{l+s+s2}{\PYZdq{}}\PY{l+s+s2}{Wonderwall}\PY{l+s+s2}{\PYZdq{}}\PY{p}{:}\PY{p}{[}\PY{l+s+s2}{\PYZdq{}}\PY{l+s+s2}{Today is gonna be the day}\PY{l+s+s2}{\PYZdq{}}\PY{p}{,}
                                \PY{l+s+s2}{\PYZdq{}}\PY{l+s+s2}{That they}\PY{l+s+s2}{\PYZsq{}}\PY{l+s+s2}{re gonna throw it back to you}\PY{l+s+s2}{\PYZdq{}}\PY{p}{,}
                                \PY{l+s+s2}{\PYZdq{}}\PY{l+s+s2}{By now you should}\PY{l+s+s2}{\PYZsq{}}\PY{l+s+s2}{ve somehow}\PY{l+s+s2}{\PYZdq{}}\PY{p}{,} \PY{l+s+s2}{\PYZdq{}}\PY{l+s+s2}{...}\PY{l+s+s2}{\PYZdq{}}\PY{p}{]}\PY{p}{,} 
                  \PY{l+s+s2}{\PYZdq{}}\PY{l+s+s2}{Vado al massimo}\PY{l+s+s2}{\PYZdq{}}\PY{p}{:}  \PY{p}{[}\PY{l+s+s2}{\PYZdq{}}\PY{l+s+s2}{Voglio veder come va a finire}\PY{l+s+s2}{\PYZdq{}}\PY{p}{,}
                                \PY{l+s+s2}{\PYZdq{}}\PY{l+s+s2}{Andando al massimo senza frenare}\PY{l+s+s2}{\PYZdq{}}
                                \PY{l+s+s2}{\PYZdq{}}\PY{l+s+s2}{Voglio vedere se davvero poi}\PY{l+s+s2}{\PYZdq{}}\PY{p}{,}
                                \PY{l+s+s2}{\PYZdq{}}\PY{l+s+s2}{Si va a finir male}\PY{l+s+s2}{\PYZdq{}}\PY{p}{,} \PY{l+s+s2}{\PYZdq{}}\PY{l+s+s2}{...}\PY{l+s+s2}{\PYZdq{}}\PY{p}{]}\PY{p}{\PYZcb{}}
        
        \PY{n}{songs} \PY{o}{=} \PY{n}{Songs}\PY{p}{(}\PY{n}{lyrics}\PY{p}{)}
        \PY{n}{songs}\PY{o}{.}\PY{n}{sing\PYZus{}me\PYZus{}a\PYZus{}song}\PY{p}{(}\PY{l+s+s2}{\PYZdq{}}\PY{l+s+s2}{Wonderwall}\PY{l+s+s2}{\PYZdq{}}\PY{p}{)}
\end{Verbatim}

    \begin{Verbatim}[commandchars=\\\{\}]
Title: Wonderwall
********************
Today is gonna be the day
That they're gonna throw it back to you
By now you should've somehow
{\ldots}

    \end{Verbatim}

    \hypertarget{exercise-4.3}{%
\subsubsection{Exercise 4.3}\label{exercise-4.3}}

Define a Point2D class that represent a point in a plane. Its
\texttt{\_\_init\_\_} method should accept the point coordinates
\texttt{x} and \texttt{y}. Write a method \texttt{distanceTo} that
compute the distance of the point to another passed as input. Test the
class by printing the distance of the point \(P=(4, 5)\) to the origin
\(P=(0,0)\) and to \(P=(3,4)\).

    \begin{Verbatim}[commandchars=\\\{\}]
{\color{incolor}In [{\color{incolor}3}]:} \PY{k+kn}{from} \PY{n+nn}{math} \PY{k}{import} \PY{n}{sqrt}
        
        \PY{k}{class} \PY{n+nc}{Point2D}\PY{p}{:}
            \PY{k}{def} \PY{n+nf}{\PYZus{}\PYZus{}init\PYZus{}\PYZus{}}\PY{p}{(}\PY{n+nb+bp}{self}\PY{p}{,} \PY{n}{x}\PY{p}{,} \PY{n}{y}\PY{p}{)}\PY{p}{:}
                \PY{n+nb+bp}{self}\PY{o}{.}\PY{n}{x} \PY{o}{=} \PY{n}{x}
                \PY{n+nb+bp}{self}\PY{o}{.}\PY{n}{y} \PY{o}{=} \PY{n}{y}
        
            \PY{k}{def} \PY{n+nf}{distanceTo}\PY{p}{(}\PY{n+nb+bp}{self}\PY{p}{,} \PY{n}{x}\PY{p}{,} \PY{n}{y}\PY{p}{)}\PY{p}{:}
                \PY{n}{dist} \PY{o}{=} \PY{n}{sqrt}\PY{p}{(}\PY{p}{(}\PY{n+nb+bp}{self}\PY{o}{.}\PY{n}{x}\PY{o}{\PYZhy{}}\PY{n}{x}\PY{p}{)}\PY{o}{*}\PY{o}{*}\PY{l+m+mi}{2} \PY{o}{+} \PY{p}{(}\PY{n+nb+bp}{self}\PY{o}{.}\PY{n}{y} \PY{o}{\PYZhy{}} \PY{n}{y}\PY{p}{)}\PY{o}{*}\PY{o}{*}\PY{l+m+mi}{2}\PY{p}{)}
                \PY{k}{return} \PY{n}{dist}
        
            \PY{k}{def} \PY{n+nf}{distanceTo\PYZus{}v2}\PY{p}{(}\PY{n+nb+bp}{self}\PY{p}{,} \PY{n}{p}\PY{p}{)}\PY{p}{:}
                \PY{n}{dist} \PY{o}{=} \PY{n}{sqrt}\PY{p}{(}\PY{p}{(}\PY{n+nb+bp}{self}\PY{o}{.}\PY{n}{x}\PY{o}{\PYZhy{}}\PY{n}{p}\PY{p}{[}\PY{l+m+mi}{0}\PY{p}{]}\PY{p}{)}\PY{o}{*}\PY{o}{*}\PY{l+m+mi}{2} \PY{o}{+} \PY{p}{(}\PY{n+nb+bp}{self}\PY{o}{.}\PY{n}{y} \PY{o}{\PYZhy{}} \PY{n}{p}\PY{p}{[}\PY{l+m+mi}{1}\PY{p}{]}\PY{p}{)}\PY{o}{*}\PY{o}{*}\PY{l+m+mi}{2}\PY{p}{)}
                \PY{k}{return} \PY{n}{dist}
        
            \PY{k}{def} \PY{n+nf}{distanceTo\PYZus{}v3}\PY{p}{(}\PY{n+nb+bp}{self}\PY{p}{,} \PY{n}{p}\PY{p}{)}\PY{p}{:}
                \PY{n}{dist} \PY{o}{=} \PY{n}{sqrt}\PY{p}{(}\PY{p}{(}\PY{n+nb+bp}{self}\PY{o}{.}\PY{n}{x}\PY{o}{\PYZhy{}}\PY{n}{p}\PY{o}{.}\PY{n}{x}\PY{p}{)}\PY{o}{*}\PY{o}{*}\PY{l+m+mi}{2} \PY{o}{+} \PY{p}{(}\PY{n+nb+bp}{self}\PY{o}{.}\PY{n}{y} \PY{o}{\PYZhy{}} \PY{n}{p}\PY{o}{.}\PY{n}{y}\PY{p}{)}\PY{o}{*}\PY{o}{*}\PY{l+m+mi}{2}\PY{p}{)}
                \PY{k}{return} \PY{n}{dist}
        
        
        \PY{n}{point} \PY{o}{=} \PY{n}{Point2D}\PY{p}{(}\PY{l+m+mi}{4}\PY{p}{,} \PY{l+m+mi}{5}\PY{p}{)}
        \PY{n}{p0} \PY{o}{=} \PY{p}{(}\PY{l+m+mi}{0}\PY{p}{,} \PY{l+m+mi}{0}\PY{p}{)}
        \PY{n}{point0} \PY{o}{=} \PY{n}{Point2D}\PY{p}{(}\PY{l+m+mi}{0}\PY{p}{,} \PY{l+m+mi}{0}\PY{p}{)}
        \PY{n+nb}{print} \PY{p}{(}\PY{l+s+s2}{\PYZdq{}}\PY{l+s+s2}{distance to p0: }\PY{l+s+si}{\PYZob{}:.2f\PYZcb{}}\PY{l+s+s2}{\PYZdq{}}\PY{o}{.}\PY{n}{format}\PY{p}{(}\PY{n}{point}\PY{o}{.}\PY{n}{distanceTo}\PY{p}{(}\PY{n}{p0}\PY{p}{[}\PY{l+m+mi}{0}\PY{p}{]}\PY{p}{,} \PY{n}{p0}\PY{p}{[}\PY{l+m+mi}{1}\PY{p}{]}\PY{p}{)}\PY{p}{)}\PY{p}{)}
        \PY{n+nb}{print} \PY{p}{(}\PY{l+s+s2}{\PYZdq{}}\PY{l+s+s2}{distance\PYZus{}v2 to p0: }\PY{l+s+si}{\PYZob{}:.2f\PYZcb{}}\PY{l+s+s2}{\PYZdq{}}\PY{o}{.}\PY{n}{format}\PY{p}{(}\PY{n}{point}\PY{o}{.}\PY{n}{distanceTo\PYZus{}v2}\PY{p}{(}\PY{n}{p0}\PY{p}{)}\PY{p}{)}\PY{p}{)}
        \PY{n+nb}{print} \PY{p}{(}\PY{l+s+s2}{\PYZdq{}}\PY{l+s+s2}{distance\PYZus{}v3 to p0: }\PY{l+s+si}{\PYZob{}:.2f\PYZcb{}}\PY{l+s+s2}{\PYZdq{}}\PY{o}{.}\PY{n}{format}\PY{p}{(}\PY{n}{point}\PY{o}{.}\PY{n}{distanceTo\PYZus{}v3}\PY{p}{(}\PY{n}{point0}\PY{p}{)}\PY{p}{)}\PY{p}{)}
        
        \PY{n}{p1} \PY{o}{=} \PY{p}{(}\PY{l+m+mi}{3}\PY{p}{,} \PY{l+m+mi}{4}\PY{p}{)}
        \PY{n}{point1} \PY{o}{=} \PY{n}{Point2D}\PY{p}{(}\PY{l+m+mi}{3}\PY{p}{,} \PY{l+m+mi}{4}\PY{p}{)}
        \PY{n+nb}{print} \PY{p}{(}\PY{l+s+s2}{\PYZdq{}}\PY{l+s+s2}{distance to p1: }\PY{l+s+si}{\PYZob{}:.2f\PYZcb{}}\PY{l+s+s2}{\PYZdq{}}\PY{o}{.}\PY{n}{format}\PY{p}{(}\PY{n}{point}\PY{o}{.}\PY{n}{distanceTo}\PY{p}{(}\PY{n}{p1}\PY{p}{[}\PY{l+m+mi}{0}\PY{p}{]}\PY{p}{,} \PY{n}{p1}\PY{p}{[}\PY{l+m+mi}{1}\PY{p}{]}\PY{p}{)}\PY{p}{)}\PY{p}{)}
        \PY{n+nb}{print} \PY{p}{(}\PY{l+s+s2}{\PYZdq{}}\PY{l+s+s2}{distance\PYZus{}v2 to p1: }\PY{l+s+si}{\PYZob{}:.2f\PYZcb{}}\PY{l+s+s2}{\PYZdq{}}\PY{o}{.}\PY{n}{format}\PY{p}{(}\PY{n}{point}\PY{o}{.}\PY{n}{distanceTo\PYZus{}v2}\PY{p}{(}\PY{n}{p1}\PY{p}{)}\PY{p}{)}\PY{p}{)}
        \PY{n+nb}{print} \PY{p}{(}\PY{l+s+s2}{\PYZdq{}}\PY{l+s+s2}{distance\PYZus{}v3 to p1: }\PY{l+s+si}{\PYZob{}:.2f\PYZcb{}}\PY{l+s+s2}{\PYZdq{}}\PY{o}{.}\PY{n}{format}\PY{p}{(}\PY{n}{point}\PY{o}{.}\PY{n}{distanceTo\PYZus{}v3}\PY{p}{(}\PY{n}{point1}\PY{p}{)}\PY{p}{)}\PY{p}{)}
\end{Verbatim}

    \begin{Verbatim}[commandchars=\\\{\}]
distance to p0: 6.40
distance\_v2 to p0: 6.40
distance\_v3 to p0: 6.40
distance to p1: 1.41
distance\_v2 to p1: 1.41
distance\_v3 to p1: 1.41

    \end{Verbatim}

    \hypertarget{exercise-4.4}{%
\subsubsection{Exercise 4.4}\label{exercise-4.4}}

So now that we have an idea of what a class is, try to write a
\texttt{DiscountCurve} class which contains the pillar dates and pillar
discount factors as attributes and which has methods for calculating the
discount factor and forward rate at arbitrary dates.

\hypertarget{solution}{%
\paragraph{Solution:}\label{solution}}

    \begin{Verbatim}[commandchars=\\\{\}]
{\color{incolor}In [{\color{incolor}4}]:} \PY{k+kn}{import} \PY{n+nn}{math}
        \PY{k+kn}{import} \PY{n+nn}{numpy}
        \PY{k+kn}{from} \PY{n+nn}{datetime} \PY{k}{import} \PY{n}{date}
        
        \PY{k}{class} \PY{n+nc}{DiscountCurve}\PY{p}{:}
        
            \PY{c+c1}{\PYZsh{} the special \PYZus{}\PYZus{}init\PYZus{}\PYZus{} method defines }
            \PY{c+c1}{\PYZsh{} how to construct instances of the class}
            \PY{k}{def} \PY{n+nf}{\PYZus{}\PYZus{}init\PYZus{}\PYZus{}}\PY{p}{(}\PY{n+nb+bp}{self}\PY{p}{,} \PY{n}{today}\PY{p}{,} \PY{n}{pillar\PYZus{}dates}\PY{p}{,} \PY{n}{discount\PYZus{}factors}\PY{p}{)}\PY{p}{:}
                \PY{c+c1}{\PYZsh{} we just store the arguments as attributes of the instance}
                \PY{n+nb+bp}{self}\PY{o}{.}\PY{n}{today} \PY{o}{=} \PY{n}{today}
                \PY{n+nb+bp}{self}\PY{o}{.}\PY{n}{pillar\PYZus{}dates} \PY{o}{=} \PY{n}{pillar\PYZus{}dates}
                \PY{n+nb+bp}{self}\PY{o}{.}\PY{n}{discount\PYZus{}factors} \PY{o}{=} \PY{n}{discount\PYZus{}factors}
        
            \PY{c+c1}{\PYZsh{} calculates a discount factor at an arbitrary }
            \PY{c+c1}{\PYZsh{}value date using the data stored in the instance}
            \PY{k}{def} \PY{n+nf}{df}\PY{p}{(}\PY{n+nb+bp}{self}\PY{p}{,} \PY{n}{d}\PY{p}{)}\PY{p}{:}
                \PY{c+c1}{\PYZsh{} these remain local variables, }
                \PY{c+c1}{\PYZsh{} i.e. they are only available within the function.}
                \PY{c+c1}{\PYZsh{} to read (or write) instance attributes, }
                \PY{c+c1}{\PYZsh{} you always need to use the self. syntax}
                \PY{n}{log\PYZus{}discount\PYZus{}factors} \PY{o}{=} \PYZbs{}
                  \PY{p}{[}\PY{n}{math}\PY{o}{.}\PY{n}{log}\PY{p}{(}\PY{n}{discount\PYZus{}factor}\PY{p}{)} 
                   \PY{k}{for} \PY{n}{discount\PYZus{}factor} \PY{o+ow}{in} \PY{n+nb+bp}{self}\PY{o}{.}\PY{n}{discount\PYZus{}factors}\PY{p}{]}
                \PY{n}{pillar\PYZus{}days} \PY{o}{=} \PY{p}{[}\PY{p}{(}\PY{n}{pillar\PYZus{}date} \PY{o}{\PYZhy{}} \PY{n+nb+bp}{self}\PY{o}{.}\PY{n}{today}\PY{p}{)}\PY{o}{.}\PY{n}{days} 
                               \PY{k}{for} \PY{n}{pillar\PYZus{}date} \PY{o+ow}{in} \PY{n+nb+bp}{self}\PY{o}{.}\PY{n}{pillar\PYZus{}dates}\PY{p}{]}
                \PY{n}{d\PYZus{}days} \PY{o}{=} \PY{p}{(}\PY{n}{d} \PY{o}{\PYZhy{}} \PY{n+nb+bp}{self}\PY{o}{.}\PY{n}{today}\PY{p}{)}\PY{o}{.}\PY{n}{days}
                \PY{n}{interpolated\PYZus{}log\PYZus{}discount\PYZus{}factor} \PY{o}{=} \PYZbs{}
                    \PY{n}{numpy}\PY{o}{.}\PY{n}{interp}\PY{p}{(}\PY{n}{d\PYZus{}days}\PY{p}{,} \PY{n}{pillar\PYZus{}days}\PY{p}{,} \PY{n}{log\PYZus{}discount\PYZus{}factors}\PY{p}{)}
                \PY{k}{return} \PY{n}{math}\PY{o}{.}\PY{n}{exp}\PY{p}{(}\PY{n}{interpolated\PYZus{}log\PYZus{}discount\PYZus{}factor}\PY{p}{)}
        
            \PY{c+c1}{\PYZsh{} calculates a forward libor rate based on the discount }
            \PY{c+c1}{\PYZsh{} curve data stored in the instance}
            \PY{k}{def} \PY{n+nf}{forward\PYZus{}rate}\PY{p}{(}\PY{n+nb+bp}{self}\PY{p}{,} \PY{n}{d1}\PY{p}{,} \PY{n}{d2}\PY{p}{)}\PY{p}{:}
                \PY{c+c1}{\PYZsh{} we use the df method of the current instance to calculate}
                \PY{c+c1}{\PYZsh{} the forward rate}
                \PY{k}{return} \PY{p}{(}\PY{n+nb+bp}{self}\PY{o}{.}\PY{n}{df}\PY{p}{(}\PY{n}{d1}\PY{p}{)} \PY{o}{/} \PY{n+nb+bp}{self}\PY{o}{.}\PY{n}{df}\PY{p}{(}\PY{n}{d2}\PY{p}{)} \PY{o}{\PYZhy{}} \PY{l+m+mf}{1.0}\PY{p}{)} \PY{o}{*} \PYZbs{}
                        \PY{p}{(}\PY{l+m+mf}{365.0} \PY{o}{/} \PY{p}{(}\PY{p}{(}\PY{n}{d2} \PY{o}{\PYZhy{}} \PY{n}{d1}\PY{p}{)}\PY{o}{.}\PY{n}{days}\PY{p}{)}\PY{p}{)}
\end{Verbatim}

    \begin{Verbatim}[commandchars=\\\{\}]
{\color{incolor}In [{\color{incolor}5}]:} \PY{c+c1}{\PYZsh{} build the EONIA curve object}
        \PY{c+c1}{\PYZsh{} n.b. here we use the \PYZsq{}parameter=argument\PYZsq{} syntax }
        \PY{c+c1}{\PYZsh{} (today=..., pillar\PYZus{}dates=...)}
        \PY{c+c1}{\PYZsh{} just so it\PYZsq{}s really clear what we\PYZsq{}re doing - it\PYZsq{}s not necessary, }
        \PY{c+c1}{\PYZsh{} it\PYZsq{}s only for clarity}
        \PY{n}{eonia\PYZus{}curve} \PY{o}{=} \PY{n}{DiscountCurve}\PY{p}{(}\PY{n}{today}\PY{o}{=}\PY{n}{date}\PY{p}{(}\PY{l+m+mi}{2019}\PY{p}{,} \PY{l+m+mi}{10}\PY{p}{,} \PY{l+m+mi}{1}\PY{p}{)}\PY{p}{,}
                                    \PY{n}{pillar\PYZus{}dates}\PY{o}{=}\PY{p}{[}\PY{n}{date}\PY{p}{(}\PY{l+m+mi}{2019}\PY{p}{,} \PY{l+m+mi}{10}\PY{p}{,} \PY{l+m+mi}{1}\PY{p}{)}\PY{p}{,} 
                                                  \PY{n}{date}\PY{p}{(}\PY{l+m+mi}{2020}\PY{p}{,} \PY{l+m+mi}{10}\PY{p}{,} \PY{l+m+mi}{1}\PY{p}{)}\PY{p}{,} 
                                                  \PY{n}{date}\PY{p}{(}\PY{l+m+mi}{2021}\PY{p}{,} \PY{l+m+mi}{10}\PY{p}{,} \PY{l+m+mi}{1}\PY{p}{)}\PY{p}{]}\PY{p}{,}
                                    \PY{n}{discount\PYZus{}factors}\PY{o}{=}\PY{p}{[}\PY{l+m+mf}{1.0}\PY{p}{,} \PY{l+m+mf}{0.95}\PY{p}{,} \PY{l+m+mf}{0.8}\PY{p}{]}\PY{p}{)}
        
        \PY{c+c1}{\PYZsh{} build the Libor curve object}
        \PY{n}{libor\PYZus{}curve} \PY{o}{=} \PY{n}{DiscountCurve}\PY{p}{(}\PY{n}{today}\PY{o}{=}\PY{n}{date}\PY{p}{(}\PY{l+m+mi}{2019}\PY{p}{,} \PY{l+m+mi}{10}\PY{p}{,} \PY{l+m+mi}{1}\PY{p}{)}\PY{p}{,}
                                    \PY{n}{pillar\PYZus{}dates}\PY{o}{=}\PY{p}{[}\PY{n}{date}\PY{p}{(}\PY{l+m+mi}{2019}\PY{p}{,} \PY{l+m+mi}{10}\PY{p}{,} \PY{l+m+mi}{1}\PY{p}{)}\PY{p}{,} 
                                                  \PY{n}{date}\PY{p}{(}\PY{l+m+mi}{2020}\PY{p}{,} \PY{l+m+mi}{4}\PY{p}{,} \PY{l+m+mi}{1}\PY{p}{)}\PY{p}{,} 
                                                  \PY{n}{date}\PY{p}{(}\PY{l+m+mi}{2020}\PY{p}{,} \PY{l+m+mi}{10}\PY{p}{,} \PY{l+m+mi}{1}\PY{p}{)}\PY{p}{]}\PY{p}{,}
                                    \PY{n}{discount\PYZus{}factors}\PY{o}{=}\PY{p}{[}\PY{l+m+mf}{1.0}\PY{p}{,} \PY{l+m+mf}{0.98}\PY{p}{,} \PY{l+m+mf}{0.82}\PY{p}{]}\PY{p}{)}
        
        \PY{c+c1}{\PYZsh{} Let\PYZsq{}s compute the discount factor of the two curves}
        \PY{c+c1}{\PYZsh{} on the 2020\PYZhy{}5\PYZhy{}1}
        \PY{n+nb}{print} \PY{p}{(}\PY{n}{eonia\PYZus{}curve}\PY{o}{.}\PY{n}{df}\PY{p}{(}\PY{n}{date}\PY{p}{(}\PY{l+m+mi}{2020}\PY{p}{,} \PY{l+m+mi}{5}\PY{p}{,} \PY{l+m+mi}{1}\PY{p}{)}\PY{p}{)}\PY{p}{)}
        \PY{n+nb}{print} \PY{p}{(}\PY{n}{libor\PYZus{}curve}\PY{o}{.}\PY{n}{df}\PY{p}{(}\PY{n}{date}\PY{p}{(}\PY{l+m+mi}{2020}\PY{p}{,} \PY{l+m+mi}{5}\PY{p}{,} \PY{l+m+mi}{1}\PY{p}{)}\PY{p}{)}\PY{p}{)}
\end{Verbatim}

    \begin{Verbatim}[commandchars=\\\{\}]
0.9705901255781632
0.9517777485424973

    \end{Verbatim}

    \begin{Verbatim}[commandchars=\\\{\}]
{\color{incolor}In [{\color{incolor}6}]:} \PY{c+c1}{\PYZsh{} Let\PYZsq{}s compute now the 6m forward rate at 1\PYZhy{}4\PYZhy{}2020}
        \PY{n+nb}{print} \PY{p}{(}\PY{n}{eonia\PYZus{}curve}\PY{o}{.}\PY{n}{forward\PYZus{}rate}\PY{p}{(}\PY{n}{date}\PY{p}{(}\PY{l+m+mi}{2019}\PY{p}{,} \PY{l+m+mi}{10}\PY{p}{,} \PY{l+m+mi}{1}\PY{p}{)}\PY{p}{,} 
                                         \PY{n}{date}\PY{p}{(}\PY{l+m+mi}{2020}\PY{p}{,} \PY{l+m+mi}{10}\PY{p}{,} \PY{l+m+mi}{1}\PY{p}{)}\PY{p}{)}\PY{p}{)}
        
        \PY{n+nb}{print} \PY{p}{(}\PY{n}{libor\PYZus{}curve}\PY{o}{.}\PY{n}{forward\PYZus{}rate}\PY{p}{(}\PY{n}{date}\PY{p}{(}\PY{l+m+mi}{2020}\PY{p}{,} \PY{l+m+mi}{4}\PY{p}{,} \PY{l+m+mi}{1}\PY{p}{)}\PY{p}{,} 
                                         \PY{n}{date}\PY{p}{(}\PY{l+m+mi}{2020}\PY{p}{,} \PY{l+m+mi}{10}\PY{p}{,} \PY{l+m+mi}{1}\PY{p}{)}\PY{p}{)}\PY{p}{)}
\end{Verbatim}

    \begin{Verbatim}[commandchars=\\\{\}]
0.05248777681909687
0.3891776622684259

    \end{Verbatim}

    \begin{Verbatim}[commandchars=\\\{\}]
{\color{incolor}In [{\color{incolor}7}]:} \PY{c+c1}{\PYZsh{} Compute the NPV of the 6m forward libor coupon}
        \PY{n}{npv} \PY{o}{=} \PY{n}{eonia\PYZus{}curve}\PY{o}{.}\PY{n}{df}\PY{p}{(}\PY{n}{date}\PY{p}{(}\PY{l+m+mi}{2020}\PY{p}{,} \PY{l+m+mi}{4}\PY{p}{,} \PY{l+m+mi}{1}\PY{p}{)}\PY{p}{)} \PY{o}{*} \PYZbs{}
              \PY{n}{libor\PYZus{}curve}\PY{o}{.}\PY{n}{forward\PYZus{}rate}\PY{p}{(}\PY{n}{date}\PY{p}{(}\PY{l+m+mi}{2020}\PY{p}{,}\PY{l+m+mi}{4}\PY{p}{,} \PY{l+m+mi}{1}\PY{p}{)}\PY{p}{,} 
                                        \PY{n}{date}\PY{p}{(}\PY{l+m+mi}{2020}\PY{p}{,} \PY{l+m+mi}{10}\PY{p}{,} \PY{l+m+mi}{1}\PY{p}{)}\PY{p}{)}
        
        \PY{c+c1}{\PYZsh{} Compute it in the pre\PYZhy{}2008 way}
        \PY{n}{npv\PYZus{}pre\PYZus{}2008} \PY{o}{=} \PY{n}{libor\PYZus{}curve}\PY{o}{.}\PY{n}{df}\PY{p}{(}\PY{n}{date}\PY{p}{(}\PY{l+m+mi}{2020}\PY{p}{,} \PY{l+m+mi}{4}\PY{p}{,} \PY{l+m+mi}{1}\PY{p}{)}\PY{p}{)} \PY{o}{*} \PYZbs{}
                       \PY{n}{libor\PYZus{}curve}\PY{o}{.}\PY{n}{forward\PYZus{}rate}\PY{p}{(}\PY{n}{date}\PY{p}{(}\PY{l+m+mi}{2020}\PY{p}{,} \PY{l+m+mi}{4}\PY{p}{,} \PY{l+m+mi}{1}\PY{p}{)}\PY{p}{,} 
                                                 \PY{n}{date}\PY{p}{(}\PY{l+m+mi}{2020}\PY{p}{,} \PY{l+m+mi}{10}\PY{p}{,} \PY{l+m+mi}{1}\PY{p}{)}\PY{p}{)}
        \PY{n+nb}{print} \PY{p}{(}\PY{n}{npv}\PY{p}{)}
        \PY{n+nb}{print} \PY{p}{(}\PY{n}{npv\PYZus{}pre\PYZus{}2008}\PY{p}{)}
\end{Verbatim}

    \begin{Verbatim}[commandchars=\\\{\}]
0.37932346377238657
0.38139410902305737

    \end{Verbatim}


    % Add a bibliography block to the postdoc
    
    
    
    \end{document}


% Default to the notebook output style

    


% Inherit from the specified cell style.




    
\documentclass[11pt]{article}

    
    
    \usepackage[T1]{fontenc}
    % Nicer default font (+ math font) than Computer Modern for most use cases
    \usepackage{mathpazo}

    % Basic figure setup, for now with no caption control since it's done
    % automatically by Pandoc (which extracts ![](path) syntax from Markdown).
    \usepackage{graphicx}
    % We will generate all images so they have a width \maxwidth. This means
    % that they will get their normal width if they fit onto the page, but
    % are scaled down if they would overflow the margins.
    \makeatletter
    \def\maxwidth{\ifdim\Gin@nat@width>\linewidth\linewidth
    \else\Gin@nat@width\fi}
    \makeatother
    \let\Oldincludegraphics\includegraphics
    % Set max figure width to be 80% of text width, for now hardcoded.
    \renewcommand{\includegraphics}[1]{\Oldincludegraphics[width=.8\maxwidth]{#1}}
    % Ensure that by default, figures have no caption (until we provide a
    % proper Figure object with a Caption API and a way to capture that
    % in the conversion process - todo).
    \usepackage{caption}
    \DeclareCaptionLabelFormat{nolabel}{}
    \captionsetup{labelformat=nolabel}

    \usepackage{adjustbox} % Used to constrain images to a maximum size 
    \usepackage{xcolor} % Allow colors to be defined
    \usepackage{enumerate} % Needed for markdown enumerations to work
    \usepackage{geometry} % Used to adjust the document margins
    \usepackage{amsmath} % Equations
    \usepackage{amssymb} % Equations
    \usepackage{textcomp} % defines textquotesingle
    % Hack from http://tex.stackexchange.com/a/47451/13684:
    \AtBeginDocument{%
        \def\PYZsq{\textquotesingle}% Upright quotes in Pygmentized code
    }
    \usepackage{upquote} % Upright quotes for verbatim code
    \usepackage{eurosym} % defines \euro
    \usepackage[mathletters]{ucs} % Extended unicode (utf-8) support
    \usepackage[utf8x]{inputenc} % Allow utf-8 characters in the tex document
    \usepackage{fancyvrb} % verbatim replacement that allows latex
    \usepackage{grffile} % extends the file name processing of package graphics 
                         % to support a larger range 
    % The hyperref package gives us a pdf with properly built
    % internal navigation ('pdf bookmarks' for the table of contents,
    % internal cross-reference links, web links for URLs, etc.)
    \usepackage{hyperref}
    \usepackage{longtable} % longtable support required by pandoc >1.10
    \usepackage{booktabs}  % table support for pandoc > 1.12.2
    \usepackage[inline]{enumitem} % IRkernel/repr support (it uses the enumerate* environment)
    \usepackage[normalem]{ulem} % ulem is needed to support strikethroughs (\sout)
                                % normalem makes italics be italics, not underlines
    \usepackage{mathrsfs}
    

    
    
    % Colors for the hyperref package
    \definecolor{urlcolor}{rgb}{0,.145,.698}
    \definecolor{linkcolor}{rgb}{.71,0.21,0.01}
    \definecolor{citecolor}{rgb}{.12,.54,.11}

    % ANSI colors
    \definecolor{ansi-black}{HTML}{3E424D}
    \definecolor{ansi-black-intense}{HTML}{282C36}
    \definecolor{ansi-red}{HTML}{E75C58}
    \definecolor{ansi-red-intense}{HTML}{B22B31}
    \definecolor{ansi-green}{HTML}{00A250}
    \definecolor{ansi-green-intense}{HTML}{007427}
    \definecolor{ansi-yellow}{HTML}{DDB62B}
    \definecolor{ansi-yellow-intense}{HTML}{B27D12}
    \definecolor{ansi-blue}{HTML}{208FFB}
    \definecolor{ansi-blue-intense}{HTML}{0065CA}
    \definecolor{ansi-magenta}{HTML}{D160C4}
    \definecolor{ansi-magenta-intense}{HTML}{A03196}
    \definecolor{ansi-cyan}{HTML}{60C6C8}
    \definecolor{ansi-cyan-intense}{HTML}{258F8F}
    \definecolor{ansi-white}{HTML}{C5C1B4}
    \definecolor{ansi-white-intense}{HTML}{A1A6B2}
    \definecolor{ansi-default-inverse-fg}{HTML}{FFFFFF}
    \definecolor{ansi-default-inverse-bg}{HTML}{000000}

    % commands and environments needed by pandoc snippets
    % extracted from the output of `pandoc -s`
    \providecommand{\tightlist}{%
      \setlength{\itemsep}{0pt}\setlength{\parskip}{0pt}}
    \DefineVerbatimEnvironment{Highlighting}{Verbatim}{commandchars=\\\{\}}
    % Add ',fontsize=\small' for more characters per line
    \newenvironment{Shaded}{}{}
    \newcommand{\KeywordTok}[1]{\textcolor[rgb]{0.00,0.44,0.13}{\textbf{{#1}}}}
    \newcommand{\DataTypeTok}[1]{\textcolor[rgb]{0.56,0.13,0.00}{{#1}}}
    \newcommand{\DecValTok}[1]{\textcolor[rgb]{0.25,0.63,0.44}{{#1}}}
    \newcommand{\BaseNTok}[1]{\textcolor[rgb]{0.25,0.63,0.44}{{#1}}}
    \newcommand{\FloatTok}[1]{\textcolor[rgb]{0.25,0.63,0.44}{{#1}}}
    \newcommand{\CharTok}[1]{\textcolor[rgb]{0.25,0.44,0.63}{{#1}}}
    \newcommand{\StringTok}[1]{\textcolor[rgb]{0.25,0.44,0.63}{{#1}}}
    \newcommand{\CommentTok}[1]{\textcolor[rgb]{0.38,0.63,0.69}{\textit{{#1}}}}
    \newcommand{\OtherTok}[1]{\textcolor[rgb]{0.00,0.44,0.13}{{#1}}}
    \newcommand{\AlertTok}[1]{\textcolor[rgb]{1.00,0.00,0.00}{\textbf{{#1}}}}
    \newcommand{\FunctionTok}[1]{\textcolor[rgb]{0.02,0.16,0.49}{{#1}}}
    \newcommand{\RegionMarkerTok}[1]{{#1}}
    \newcommand{\ErrorTok}[1]{\textcolor[rgb]{1.00,0.00,0.00}{\textbf{{#1}}}}
    \newcommand{\NormalTok}[1]{{#1}}
    
    % Additional commands for more recent versions of Pandoc
    \newcommand{\ConstantTok}[1]{\textcolor[rgb]{0.53,0.00,0.00}{{#1}}}
    \newcommand{\SpecialCharTok}[1]{\textcolor[rgb]{0.25,0.44,0.63}{{#1}}}
    \newcommand{\VerbatimStringTok}[1]{\textcolor[rgb]{0.25,0.44,0.63}{{#1}}}
    \newcommand{\SpecialStringTok}[1]{\textcolor[rgb]{0.73,0.40,0.53}{{#1}}}
    \newcommand{\ImportTok}[1]{{#1}}
    \newcommand{\DocumentationTok}[1]{\textcolor[rgb]{0.73,0.13,0.13}{\textit{{#1}}}}
    \newcommand{\AnnotationTok}[1]{\textcolor[rgb]{0.38,0.63,0.69}{\textbf{\textit{{#1}}}}}
    \newcommand{\CommentVarTok}[1]{\textcolor[rgb]{0.38,0.63,0.69}{\textbf{\textit{{#1}}}}}
    \newcommand{\VariableTok}[1]{\textcolor[rgb]{0.10,0.09,0.49}{{#1}}}
    \newcommand{\ControlFlowTok}[1]{\textcolor[rgb]{0.00,0.44,0.13}{\textbf{{#1}}}}
    \newcommand{\OperatorTok}[1]{\textcolor[rgb]{0.40,0.40,0.40}{{#1}}}
    \newcommand{\BuiltInTok}[1]{{#1}}
    \newcommand{\ExtensionTok}[1]{{#1}}
    \newcommand{\PreprocessorTok}[1]{\textcolor[rgb]{0.74,0.48,0.00}{{#1}}}
    \newcommand{\AttributeTok}[1]{\textcolor[rgb]{0.49,0.56,0.16}{{#1}}}
    \newcommand{\InformationTok}[1]{\textcolor[rgb]{0.38,0.63,0.69}{\textbf{\textit{{#1}}}}}
    \newcommand{\WarningTok}[1]{\textcolor[rgb]{0.38,0.63,0.69}{\textbf{\textit{{#1}}}}}
    
    
    % Define a nice break command that doesn't care if a line doesn't already
    % exist.
    \def\br{\hspace*{\fill} \\* }
    % Math Jax compatibility definitions
    \def\gt{>}
    \def\lt{<}
    \let\Oldtex\TeX
    \let\Oldlatex\LaTeX
    \renewcommand{\TeX}{\textrm{\Oldtex}}
    \renewcommand{\LaTeX}{\textrm{\Oldlatex}}
    % Document parameters
    % Document title
    \title{Solutions - Practical Lesson 2}
    \author {Matteo Sani \\ \href{mailto:matteosan1@gmail.com}{matteosan1@gmail.com}}
    

    % Pygments definitions
    
\makeatletter
\def\PY@reset{\let\PY@it=\relax \let\PY@bf=\relax%
    \let\PY@ul=\relax \let\PY@tc=\relax%
    \let\PY@bc=\relax \let\PY@ff=\relax}
\def\PY@tok#1{\csname PY@tok@#1\endcsname}
\def\PY@toks#1+{\ifx\relax#1\empty\else%
    \PY@tok{#1}\expandafter\PY@toks\fi}
\def\PY@do#1{\PY@bc{\PY@tc{\PY@ul{%
    \PY@it{\PY@bf{\PY@ff{#1}}}}}}}
\def\PY#1#2{\PY@reset\PY@toks#1+\relax+\PY@do{#2}}

\expandafter\def\csname PY@tok@w\endcsname{\def\PY@tc##1{\textcolor[rgb]{0.73,0.73,0.73}{##1}}}
\expandafter\def\csname PY@tok@c\endcsname{\let\PY@it=\textit\def\PY@tc##1{\textcolor[rgb]{0.25,0.50,0.50}{##1}}}
\expandafter\def\csname PY@tok@cp\endcsname{\def\PY@tc##1{\textcolor[rgb]{0.74,0.48,0.00}{##1}}}
\expandafter\def\csname PY@tok@k\endcsname{\let\PY@bf=\textbf\def\PY@tc##1{\textcolor[rgb]{0.00,0.50,0.00}{##1}}}
\expandafter\def\csname PY@tok@kp\endcsname{\def\PY@tc##1{\textcolor[rgb]{0.00,0.50,0.00}{##1}}}
\expandafter\def\csname PY@tok@kt\endcsname{\def\PY@tc##1{\textcolor[rgb]{0.69,0.00,0.25}{##1}}}
\expandafter\def\csname PY@tok@o\endcsname{\def\PY@tc##1{\textcolor[rgb]{0.40,0.40,0.40}{##1}}}
\expandafter\def\csname PY@tok@ow\endcsname{\let\PY@bf=\textbf\def\PY@tc##1{\textcolor[rgb]{0.67,0.13,1.00}{##1}}}
\expandafter\def\csname PY@tok@nb\endcsname{\def\PY@tc##1{\textcolor[rgb]{0.00,0.50,0.00}{##1}}}
\expandafter\def\csname PY@tok@nf\endcsname{\def\PY@tc##1{\textcolor[rgb]{0.00,0.00,1.00}{##1}}}
\expandafter\def\csname PY@tok@nc\endcsname{\let\PY@bf=\textbf\def\PY@tc##1{\textcolor[rgb]{0.00,0.00,1.00}{##1}}}
\expandafter\def\csname PY@tok@nn\endcsname{\let\PY@bf=\textbf\def\PY@tc##1{\textcolor[rgb]{0.00,0.00,1.00}{##1}}}
\expandafter\def\csname PY@tok@ne\endcsname{\let\PY@bf=\textbf\def\PY@tc##1{\textcolor[rgb]{0.82,0.25,0.23}{##1}}}
\expandafter\def\csname PY@tok@nv\endcsname{\def\PY@tc##1{\textcolor[rgb]{0.10,0.09,0.49}{##1}}}
\expandafter\def\csname PY@tok@no\endcsname{\def\PY@tc##1{\textcolor[rgb]{0.53,0.00,0.00}{##1}}}
\expandafter\def\csname PY@tok@nl\endcsname{\def\PY@tc##1{\textcolor[rgb]{0.63,0.63,0.00}{##1}}}
\expandafter\def\csname PY@tok@ni\endcsname{\let\PY@bf=\textbf\def\PY@tc##1{\textcolor[rgb]{0.60,0.60,0.60}{##1}}}
\expandafter\def\csname PY@tok@na\endcsname{\def\PY@tc##1{\textcolor[rgb]{0.49,0.56,0.16}{##1}}}
\expandafter\def\csname PY@tok@nt\endcsname{\let\PY@bf=\textbf\def\PY@tc##1{\textcolor[rgb]{0.00,0.50,0.00}{##1}}}
\expandafter\def\csname PY@tok@nd\endcsname{\def\PY@tc##1{\textcolor[rgb]{0.67,0.13,1.00}{##1}}}
\expandafter\def\csname PY@tok@s\endcsname{\def\PY@tc##1{\textcolor[rgb]{0.73,0.13,0.13}{##1}}}
\expandafter\def\csname PY@tok@sd\endcsname{\let\PY@it=\textit\def\PY@tc##1{\textcolor[rgb]{0.73,0.13,0.13}{##1}}}
\expandafter\def\csname PY@tok@si\endcsname{\let\PY@bf=\textbf\def\PY@tc##1{\textcolor[rgb]{0.73,0.40,0.53}{##1}}}
\expandafter\def\csname PY@tok@se\endcsname{\let\PY@bf=\textbf\def\PY@tc##1{\textcolor[rgb]{0.73,0.40,0.13}{##1}}}
\expandafter\def\csname PY@tok@sr\endcsname{\def\PY@tc##1{\textcolor[rgb]{0.73,0.40,0.53}{##1}}}
\expandafter\def\csname PY@tok@ss\endcsname{\def\PY@tc##1{\textcolor[rgb]{0.10,0.09,0.49}{##1}}}
\expandafter\def\csname PY@tok@sx\endcsname{\def\PY@tc##1{\textcolor[rgb]{0.00,0.50,0.00}{##1}}}
\expandafter\def\csname PY@tok@m\endcsname{\def\PY@tc##1{\textcolor[rgb]{0.40,0.40,0.40}{##1}}}
\expandafter\def\csname PY@tok@gh\endcsname{\let\PY@bf=\textbf\def\PY@tc##1{\textcolor[rgb]{0.00,0.00,0.50}{##1}}}
\expandafter\def\csname PY@tok@gu\endcsname{\let\PY@bf=\textbf\def\PY@tc##1{\textcolor[rgb]{0.50,0.00,0.50}{##1}}}
\expandafter\def\csname PY@tok@gd\endcsname{\def\PY@tc##1{\textcolor[rgb]{0.63,0.00,0.00}{##1}}}
\expandafter\def\csname PY@tok@gi\endcsname{\def\PY@tc##1{\textcolor[rgb]{0.00,0.63,0.00}{##1}}}
\expandafter\def\csname PY@tok@gr\endcsname{\def\PY@tc##1{\textcolor[rgb]{1.00,0.00,0.00}{##1}}}
\expandafter\def\csname PY@tok@ge\endcsname{\let\PY@it=\textit}
\expandafter\def\csname PY@tok@gs\endcsname{\let\PY@bf=\textbf}
\expandafter\def\csname PY@tok@gp\endcsname{\let\PY@bf=\textbf\def\PY@tc##1{\textcolor[rgb]{0.00,0.00,0.50}{##1}}}
\expandafter\def\csname PY@tok@go\endcsname{\def\PY@tc##1{\textcolor[rgb]{0.53,0.53,0.53}{##1}}}
\expandafter\def\csname PY@tok@gt\endcsname{\def\PY@tc##1{\textcolor[rgb]{0.00,0.27,0.87}{##1}}}
\expandafter\def\csname PY@tok@err\endcsname{\def\PY@bc##1{\setlength{\fboxsep}{0pt}\fcolorbox[rgb]{1.00,0.00,0.00}{1,1,1}{\strut ##1}}}
\expandafter\def\csname PY@tok@kc\endcsname{\let\PY@bf=\textbf\def\PY@tc##1{\textcolor[rgb]{0.00,0.50,0.00}{##1}}}
\expandafter\def\csname PY@tok@kd\endcsname{\let\PY@bf=\textbf\def\PY@tc##1{\textcolor[rgb]{0.00,0.50,0.00}{##1}}}
\expandafter\def\csname PY@tok@kn\endcsname{\let\PY@bf=\textbf\def\PY@tc##1{\textcolor[rgb]{0.00,0.50,0.00}{##1}}}
\expandafter\def\csname PY@tok@kr\endcsname{\let\PY@bf=\textbf\def\PY@tc##1{\textcolor[rgb]{0.00,0.50,0.00}{##1}}}
\expandafter\def\csname PY@tok@bp\endcsname{\def\PY@tc##1{\textcolor[rgb]{0.00,0.50,0.00}{##1}}}
\expandafter\def\csname PY@tok@fm\endcsname{\def\PY@tc##1{\textcolor[rgb]{0.00,0.00,1.00}{##1}}}
\expandafter\def\csname PY@tok@vc\endcsname{\def\PY@tc##1{\textcolor[rgb]{0.10,0.09,0.49}{##1}}}
\expandafter\def\csname PY@tok@vg\endcsname{\def\PY@tc##1{\textcolor[rgb]{0.10,0.09,0.49}{##1}}}
\expandafter\def\csname PY@tok@vi\endcsname{\def\PY@tc##1{\textcolor[rgb]{0.10,0.09,0.49}{##1}}}
\expandafter\def\csname PY@tok@vm\endcsname{\def\PY@tc##1{\textcolor[rgb]{0.10,0.09,0.49}{##1}}}
\expandafter\def\csname PY@tok@sa\endcsname{\def\PY@tc##1{\textcolor[rgb]{0.73,0.13,0.13}{##1}}}
\expandafter\def\csname PY@tok@sb\endcsname{\def\PY@tc##1{\textcolor[rgb]{0.73,0.13,0.13}{##1}}}
\expandafter\def\csname PY@tok@sc\endcsname{\def\PY@tc##1{\textcolor[rgb]{0.73,0.13,0.13}{##1}}}
\expandafter\def\csname PY@tok@dl\endcsname{\def\PY@tc##1{\textcolor[rgb]{0.73,0.13,0.13}{##1}}}
\expandafter\def\csname PY@tok@s2\endcsname{\def\PY@tc##1{\textcolor[rgb]{0.73,0.13,0.13}{##1}}}
\expandafter\def\csname PY@tok@sh\endcsname{\def\PY@tc##1{\textcolor[rgb]{0.73,0.13,0.13}{##1}}}
\expandafter\def\csname PY@tok@s1\endcsname{\def\PY@tc##1{\textcolor[rgb]{0.73,0.13,0.13}{##1}}}
\expandafter\def\csname PY@tok@mb\endcsname{\def\PY@tc##1{\textcolor[rgb]{0.40,0.40,0.40}{##1}}}
\expandafter\def\csname PY@tok@mf\endcsname{\def\PY@tc##1{\textcolor[rgb]{0.40,0.40,0.40}{##1}}}
\expandafter\def\csname PY@tok@mh\endcsname{\def\PY@tc##1{\textcolor[rgb]{0.40,0.40,0.40}{##1}}}
\expandafter\def\csname PY@tok@mi\endcsname{\def\PY@tc##1{\textcolor[rgb]{0.40,0.40,0.40}{##1}}}
\expandafter\def\csname PY@tok@il\endcsname{\def\PY@tc##1{\textcolor[rgb]{0.40,0.40,0.40}{##1}}}
\expandafter\def\csname PY@tok@mo\endcsname{\def\PY@tc##1{\textcolor[rgb]{0.40,0.40,0.40}{##1}}}
\expandafter\def\csname PY@tok@ch\endcsname{\let\PY@it=\textit\def\PY@tc##1{\textcolor[rgb]{0.25,0.50,0.50}{##1}}}
\expandafter\def\csname PY@tok@cm\endcsname{\let\PY@it=\textit\def\PY@tc##1{\textcolor[rgb]{0.25,0.50,0.50}{##1}}}
\expandafter\def\csname PY@tok@cpf\endcsname{\let\PY@it=\textit\def\PY@tc##1{\textcolor[rgb]{0.25,0.50,0.50}{##1}}}
\expandafter\def\csname PY@tok@c1\endcsname{\let\PY@it=\textit\def\PY@tc##1{\textcolor[rgb]{0.25,0.50,0.50}{##1}}}
\expandafter\def\csname PY@tok@cs\endcsname{\let\PY@it=\textit\def\PY@tc##1{\textcolor[rgb]{0.25,0.50,0.50}{##1}}}

\def\PYZbs{\char`\\}
\def\PYZus{\char`\_}
\def\PYZob{\char`\{}
\def\PYZcb{\char`\}}
\def\PYZca{\char`\^}
\def\PYZam{\char`\&}
\def\PYZlt{\char`\<}
\def\PYZgt{\char`\>}
\def\PYZsh{\char`\#}
\def\PYZpc{\char`\%}
\def\PYZdl{\char`\$}
\def\PYZhy{\char`\-}
\def\PYZsq{\char`\'}
\def\PYZdq{\char`\"}
\def\PYZti{\char`\~}
% for compatibility with earlier versions
\def\PYZat{@}
\def\PYZlb{[}
\def\PYZrb{]}
\makeatother


    % Exact colors from NB
    \definecolor{incolor}{rgb}{0.0, 0.0, 0.5}
    \definecolor{outcolor}{rgb}{0.545, 0.0, 0.0}



    
    % Prevent overflowing lines due to hard-to-break entities
    \sloppy 
    % Setup hyperref package
    \hypersetup{
      breaklinks=true,  % so long urls are correctly broken across lines
      colorlinks=true,
      urlcolor=urlcolor,
      linkcolor=linkcolor,
      citecolor=citecolor,
      }
    % Slightly bigger margins than the latex defaults
    
    \geometry{verbose,tmargin=1in,bmargin=1in,lmargin=1in,rmargin=1in}
    
    

    \begin{document}
    
    
    \maketitle
    
    

    
    \hypertarget{solutions---practical-lesson-2}{%
\section{Solutions}\label{solutions---practical-lesson-2}}

\hypertarget{exercises}{%
\subsection{Exercises}\label{exercises}}

\hypertarget{exercise-2.2}{%
\subsubsection{Exercise 2.2}\label{exercise-2.2}}

Write code which, given the following list

\begin{Shaded}
\begin{Highlighting}[]
\NormalTok{input_list }\OperatorTok{=}\NormalTok{ [}\DecValTok{3}\NormalTok{, }\DecValTok{5}\NormalTok{, }\DecValTok{2}\NormalTok{, }\DecValTok{1}\NormalTok{, }\DecValTok{13}\NormalTok{, }\DecValTok{5}\NormalTok{, }\DecValTok{5}\NormalTok{, }\DecValTok{1}\NormalTok{, }\DecValTok{3}\NormalTok{, }\DecValTok{4}\NormalTok{]}
\end{Highlighting}
\end{Shaded}

prints out the indices of every occurrence of

\begin{Shaded}
\begin{Highlighting}[]
\NormalTok{y }\OperatorTok{=} \DecValTok{5}
\end{Highlighting}
\end{Shaded}

\textbf{Solution}:

    \begin{Verbatim}[commandchars=\\\{\}]
{\color{incolor}In [{\color{incolor}1}]:} \PY{n}{input\PYZus{}list} \PY{o}{=} \PY{p}{[}\PY{l+m+mi}{3}\PY{p}{,} \PY{l+m+mi}{5}\PY{p}{,} \PY{l+m+mi}{2}\PY{p}{,} \PY{l+m+mi}{1}\PY{p}{,} \PY{l+m+mi}{13}\PY{p}{,} \PY{l+m+mi}{5}\PY{p}{,} \PY{l+m+mi}{5}\PY{p}{,} \PY{l+m+mi}{1}\PY{p}{,} \PY{l+m+mi}{3}\PY{p}{,} \PY{l+m+mi}{4}\PY{p}{]}
        \PY{n}{y} \PY{o}{=} \PY{l+m+mi}{5}
        \PY{k}{for} \PY{n}{i}\PY{p}{,} \PY{n}{value} \PY{o+ow}{in} \PY{n+nb}{enumerate}\PY{p}{(}\PY{n}{input\PYZus{}list}\PY{p}{)}\PY{p}{:}
            \PY{k}{if} \PY{n}{value} \PY{o}{==} \PY{l+m+mi}{5}\PY{p}{:}
                \PY{n+nb}{print} \PY{p}{(}\PY{n}{i}\PY{p}{)}
\end{Verbatim}

    \begin{Verbatim}[commandchars=\\\{\}]
1
5
6

    \end{Verbatim}

    \hypertarget{exercise-2.3}{%
\subsubsection{Exercise 2.3}\label{exercise-2.3}}

Given the following variables

\begin{Shaded}
\begin{Highlighting}[]
\NormalTok{S_t }\OperatorTok{=} \FloatTok{800.0} \CommentTok{# spot price of the underlying}
\NormalTok{K }\OperatorTok{=} \FloatTok{600.0} \CommentTok{# strike price}
\NormalTok{vol }\OperatorTok{=} \FloatTok{0.25} \CommentTok{# volatility}
\NormalTok{r }\OperatorTok{=} \FloatTok{0.01} \CommentTok{# interest rate}
\NormalTok{ttm }\OperatorTok{=} \FloatTok{0.5} \CommentTok{# time to maturity, in years}
\end{Highlighting}
\end{Shaded}

write out the Black Scholes formula and save the value of a call in a
variable named `call\_price' and the value of a put in a variable named
`put\_price'

\textbf{Solution:} The BS equation for the price of a call is:

\[ C(S, t) = S_tN(d_1)-Ke^{-r(T-t)}N(d_2) \]

where
\begin{itemize}
\item \(S_t\) is the spot price of the underlying
\item \(K\) is the strike price
\item \(r\) is the risk-free interest rate (expressed in terms of continous compounding)
\item \(N(\cdot)\) is the cumulative distribution function of the standard normal distribution
\item \(T - t\) is the time to maturity
\item \(\sigma\) is the volatility of the underlying
\end{itemize}

\[\begin{split}
d_1 & = \frac{\mathrm{ln}(\frac{S_t}{K}) + (r + \frac{1}{2}\sigma^{2})(T-t)}{\sigma\sqrt{T-t}}\\ \\
d_2 & = d_1 - \sigma\sqrt{T-t}\\
\end{split}\]

Remember that there are many modules available in python that let you
save a lot of time. In this case we need the cumulative distribution
function of the standard normal distribution which can be found in
\texttt{scipy.stats} module.

    \begin{Verbatim}[commandchars=\\\{\}]
{\color{incolor}In [{\color{incolor}8}]:} \PY{k+kn}{from} \PY{n+nn}{math} \PY{k}{import} \PY{n}{log}\PY{p}{,} \PY{n}{exp}\PY{p}{,} \PY{n}{sqrt}
        \PY{c+c1}{\PYZsh{} You\PYZsq{}ll need the Gaussian cumulative distribution function}
        \PY{k+kn}{from} \PY{n+nn}{scipy}\PY{n+nn}{.}\PY{n+nn}{stats} \PY{k}{import} \PY{n}{norm}
        
        \PY{n}{S\PYZus{}t} \PY{o}{=} \PY{l+m+mf}{800.0}
        \PY{n}{ttm} \PY{o}{=} \PY{l+m+mf}{0.5}
        \PY{n}{K} \PY{o}{=} \PY{l+m+mf}{600.0}
        \PY{n}{vol} \PY{o}{=} \PY{l+m+mf}{0.25}
        \PY{n}{r} \PY{o}{=} \PY{l+m+mf}{0.01}
        
        \PY{n}{d1\PYZus{}num} \PY{o}{=} \PY{p}{(}\PY{n}{log}\PY{p}{(}\PY{n}{S\PYZus{}t}\PY{o}{/}\PY{n}{K}\PY{p}{)}\PY{o}{+}\PY{p}{(}\PY{n}{r}\PY{o}{+}\PY{l+m+mf}{0.5}\PY{o}{*}\PY{n+nb}{pow}\PY{p}{(}\PY{n}{vol}\PY{p}{,} \PY{l+m+mi}{2}\PY{p}{)}\PY{p}{)}\PY{o}{*}\PY{n}{ttm}\PY{p}{)}
        \PY{n}{d1\PYZus{}den} \PY{o}{=} \PY{n}{vol}\PY{o}{*}\PY{n}{sqrt}\PY{p}{(}\PY{n}{ttm}\PY{p}{)}
        \PY{n}{d1} \PY{o}{=} \PY{n}{d1\PYZus{}num} \PY{o}{/}\PY{n}{d1\PYZus{}den}
        \PY{n}{d2} \PY{o}{=} \PY{n}{d1} \PY{o}{\PYZhy{}} \PY{n}{d1\PYZus{}den}          
        
        \PY{n}{call\PYZus{}price} \PY{o}{=} \PY{n}{S\PYZus{}t} \PY{o}{*} \PY{n}{norm}\PY{o}{.}\PY{n}{cdf}\PY{p}{(}\PY{n}{d1}\PY{p}{)} \PY{o}{\PYZhy{}} \PY{n}{K} \PY{o}{*} \PY{n}{exp}\PY{p}{(}\PY{o}{\PYZhy{}}\PY{n}{r}\PY{o}{*}\PY{n}{ttm}\PY{p}{)}\PY{o}{*}\PY{n}{norm}\PY{o}{.}\PY{n}{cdf}\PY{p}{(}\PY{n}{d2}\PY{p}{)}
        \PY{n}{put\PYZus{}price} \PY{o}{=} \PY{o}{\PYZhy{}} \PY{n}{S\PYZus{}t} \PY{o}{*} \PY{n}{norm}\PY{o}{.}\PY{n}{cdf}\PY{p}{(}\PY{o}{\PYZhy{}}\PY{n}{d1}\PY{p}{)} \PY{o}{+} \PY{n}{K} \PY{o}{*} \PY{n}{exp}\PY{p}{(}\PY{o}{\PYZhy{}}\PY{n}{r}\PY{o}{*}\PY{n}{ttm}\PY{p}{)}\PY{o}{*}\PY{n}{norm}\PY{o}{.}\PY{n}{cdf}\PY{p}{(}\PY{o}{\PYZhy{}}\PY{n}{d2}\PY{p}{)}
        
        \PY{n+nb}{print} \PY{p}{(}\PY{l+s+s2}{\PYZdq{}}\PY{l+s+si}{\PYZob{}:.3f\PYZcb{}}\PY{l+s+s2}{ }\PY{l+s+si}{\PYZob{}:.3f\PYZcb{}}\PY{l+s+s2}{\PYZdq{}}\PY{o}{.}\PY{n}{format}\PY{p}{(}\PY{n}{call\PYZus{}price}\PY{p}{,} \PY{n}{put\PYZus{}price}\PY{p}{)}\PY{p}{)}
\end{Verbatim}

    \begin{Verbatim}[commandchars=\\\{\}]
205.472 2.480

    \end{Verbatim}

    \hypertarget{exercise-2.4}{%
\subsubsection{Exercise 2.4}\label{exercise-2.4}}

Given the following dictionary mapping currencies to 2-year zero coupon
bond prices, build another dictionary mapping the same currencies to the
corresponding annualized interest rates.

\begin{Shaded}
\begin{Highlighting}[]
\NormalTok{discount_factors }\OperatorTok{=}\NormalTok{ \{}
\StringTok{'EUR'}\NormalTok{: }\FloatTok{0.98}\NormalTok{,}
\StringTok{'CHF'}\NormalTok{: }\FloatTok{1.005}\NormalTok{,}
\StringTok{'USD'}\NormalTok{: }\FloatTok{0.985}\NormalTok{,}
\StringTok{'GBP'}\NormalTok{: }\FloatTok{0.97}
\NormalTok{\}}
\end{Highlighting}
\end{Shaded}

\textbf{Solution:} The price of a n-years zero coupon bond is:

\[ P = \frac{M}{(1+r)^{n}} = M\cdot D \]

where * \(M\) is the value of the bond at the maturity * \(r\) is the
risk-free rate * \(n\) is the number of years untill maturity

Hence:

\[ D = \frac{1}{(1+r)^{n}} \Longrightarrow r = \Big(\frac{1}{D}\Big)^{\frac{1}{n}} - 1\]

    \begin{Verbatim}[commandchars=\\\{\}]
{\color{incolor}In [{\color{incolor}5}]:} \PY{k+kn}{from} \PY{n+nn}{math} \PY{k}{import} \PY{n}{exp}
        
        \PY{c+c1}{\PYZsh{} initialize an empty dictionary in which to store result}
        \PY{n}{rates} \PY{o}{=} \PY{p}{\PYZob{}}\PY{p}{\PYZcb{}}
        
        \PY{n}{maturity} \PY{o}{=} \PY{l+m+mi}{2}
        \PY{n}{discount\PYZus{}factors} \PY{o}{=} \PY{p}{\PYZob{}}
            \PY{l+s+s1}{\PYZsq{}}\PY{l+s+s1}{EUR}\PY{l+s+s1}{\PYZsq{}}\PY{p}{:} \PY{l+m+mf}{0.98}\PY{p}{,}
            \PY{l+s+s1}{\PYZsq{}}\PY{l+s+s1}{CHF}\PY{l+s+s1}{\PYZsq{}}\PY{p}{:} \PY{l+m+mf}{1.005}\PY{p}{,}
            \PY{l+s+s1}{\PYZsq{}}\PY{l+s+s1}{USD}\PY{l+s+s1}{\PYZsq{}}\PY{p}{:} \PY{l+m+mf}{0.985}\PY{p}{,}
            \PY{l+s+s1}{\PYZsq{}}\PY{l+s+s1}{GBP}\PY{l+s+s1}{\PYZsq{}}\PY{p}{:} \PY{l+m+mf}{0.97}
        \PY{p}{\PYZcb{}}
        
        \PY{c+c1}{\PYZsh{} loop over the input dictionary to get the currencies}
        \PY{k}{for} \PY{n}{currency}\PY{p}{,} \PY{n}{df} \PY{o+ow}{in} \PY{n}{discount\PYZus{}factors}\PY{o}{.}\PY{n}{items}\PY{p}{(}\PY{p}{)}\PY{p}{:}
            \PY{c+c1}{\PYZsh{} calculate the rate and store it in the output dictionary}
            \PY{n}{rates}\PY{p}{[}\PY{n}{currency}\PY{p}{]} \PY{o}{=} \PY{n+nb}{pow}\PY{p}{(}\PY{l+m+mi}{1}\PY{o}{/}\PY{n}{df}\PY{p}{,} \PY{l+m+mi}{1}\PY{o}{/}\PY{n}{maturity}\PY{p}{)} \PY{o}{\PYZhy{}} \PY{l+m+mi}{1}
            
        \PY{k}{for} \PY{n}{r} \PY{o+ow}{in} \PY{n}{rates}\PY{o}{.}\PY{n}{items}\PY{p}{(}\PY{p}{)}\PY{p}{:}    
            \PY{n+nb}{print} \PY{p}{(}\PY{n}{r}\PY{p}{)}
\end{Verbatim}

    \begin{Verbatim}[commandchars=\\\{\}]
('EUR', 0.010152544552210818)
('CHF', -0.002490663892367073)
('USD', 0.007585443719756668)
('GBP', 0.015346165133619083)

    \end{Verbatim}

    \hypertarget{exercise-2.5}{%
\subsubsection{Exercise 2.5}\label{exercise-2.5}}

Build again dates as in Exercise 2.1 (i.e.~the weekday of your birthdays
for the next 120 years) and count how many of your birthdays is a
Monday, Tuesday, \ldots{} , Sunday until 120 years of age. Print out the
result using a dictionary.

\textbf{Solution:}

    \begin{Verbatim}[commandchars=\\\{\}]
{\color{incolor}In [{\color{incolor}7}]:} \PY{k+kn}{import} \PY{n+nn}{datetime}
        \PY{k+kn}{from} \PY{n+nn}{dateutil}\PY{n+nn}{.}\PY{n+nn}{relativedelta} \PY{k}{import} \PY{n}{relativedelta}
        
        \PY{n}{name\PYZus{}of\PYZus{}day} \PY{o}{=} \PY{p}{\PYZob{}}\PY{l+m+mi}{0}\PY{p}{:}\PY{l+s+s2}{\PYZdq{}}\PY{l+s+s2}{Mon}\PY{l+s+s2}{\PYZdq{}}\PY{p}{,} \PY{l+m+mi}{1}\PY{p}{:}\PY{l+s+s2}{\PYZdq{}}\PY{l+s+s2}{Tue}\PY{l+s+s2}{\PYZdq{}}\PY{p}{,} \PY{l+m+mi}{2}\PY{p}{:}\PY{l+s+s2}{\PYZdq{}}\PY{l+s+s2}{Wed}\PY{l+s+s2}{\PYZdq{}}\PY{p}{,} \PY{l+m+mi}{3}\PY{p}{:}\PY{l+s+s2}{\PYZdq{}}\PY{l+s+s2}{Thu}\PY{l+s+s2}{\PYZdq{}}\PY{p}{,} \PY{l+m+mi}{4}\PY{p}{:}\PY{l+s+s2}{\PYZdq{}}\PY{l+s+s2}{Fri}\PY{l+s+s2}{\PYZdq{}}\PY{p}{,} \PY{l+m+mi}{5}\PY{p}{:}\PY{l+s+s2}{\PYZdq{}}\PY{l+s+s2}{Sat}\PY{l+s+s2}{\PYZdq{}}\PY{p}{,} \PY{l+m+mi}{6}\PY{p}{:}\PY{l+s+s2}{\PYZdq{}}\PY{l+s+s2}{Sun}\PY{l+s+s2}{\PYZdq{}}\PY{p}{\PYZcb{}}
        \PY{n}{birthday\PYZus{}weekdays} \PY{o}{=} \PY{p}{\PYZob{}}\PY{p}{\PYZcb{}}
        \PY{n}{birthday} \PY{o}{=} \PY{n}{datetime}\PY{o}{.}\PY{n}{date}\PY{p}{(}\PY{l+m+mi}{1974}\PY{p}{,} \PY{l+m+mi}{10}\PY{p}{,} \PY{l+m+mi}{20}\PY{p}{)}
        
        \PY{k}{for} \PY{n}{i} \PY{o+ow}{in} \PY{n+nb}{range}\PY{p}{(}\PY{l+m+mi}{121}\PY{p}{)}\PY{p}{:}
          \PY{n}{next\PYZus{}birthday} \PY{o}{=} \PY{n}{birthday} \PY{o}{+} \PY{n}{relativedelta}\PY{p}{(}\PY{n}{years}\PY{o}{=}\PY{n}{i}\PY{p}{)}
          \PY{n}{wd} \PY{o}{=} \PY{n}{next\PYZus{}birthday}\PY{o}{.}\PY{n}{weekday}\PY{p}{(}\PY{p}{)}
          \PY{k}{if} \PY{n}{wd} \PY{o+ow}{not} \PY{o+ow}{in} \PY{n}{birthday\PYZus{}weekdays}\PY{o}{.}\PY{n}{keys}\PY{p}{(}\PY{p}{)}\PY{p}{:}
            \PY{n}{birthday\PYZus{}weekdays}\PY{p}{[}\PY{n}{wd}\PY{p}{]} \PY{o}{=} \PY{l+m+mi}{1}
          \PY{k}{else}\PY{p}{:}
            \PY{n}{birthday\PYZus{}weekdays}\PY{p}{[}\PY{n}{wd}\PY{p}{]} \PY{o}{+}\PY{o}{=} \PY{l+m+mi}{1}
        
        \PY{n+nb}{print} \PY{p}{(}\PY{n}{birthday\PYZus{}weekdays}\PY{p}{)}
        
        \PY{c+c1}{\PYZsh{} if you want to be more precise you can map integers}
        \PY{c+c1}{\PYZsh{} from 0 to 6 to the day name}
        \PY{n+nb}{print} \PY{p}{(}\PY{p}{)}
        \PY{k}{for} \PY{n}{k} \PY{o+ow}{in} \PY{n+nb}{sorted}\PY{p}{(}\PY{n}{birthday\PYZus{}weekdays}\PY{o}{.}\PY{n}{keys}\PY{p}{(}\PY{p}{)}\PY{p}{)}\PY{p}{:}
          \PY{n+nb}{print} \PY{p}{(}\PY{n}{name\PYZus{}of\PYZus{}day}\PY{p}{[}\PY{n}{k}\PY{p}{]}\PY{p}{,} \PY{n}{birthday\PYZus{}weekdays}\PY{p}{[}\PY{n}{k}\PY{p}{]}\PY{p}{)}
\end{Verbatim}

    \begin{Verbatim}[commandchars=\\\{\}]
\{6: 17, 0: 18, 2: 18, 3: 17, 4: 17, 5: 17, 1: 17\}

Mon 18
Tue 17
Wed 18
Thu 17
Fri 17
Sat 17
Sun 17

    \end{Verbatim}


\hypertarget{exercise-2.6}{%
\subsubsection{Exercise 2.6}\label{exercise-2.6}}
\begin{itemize}
      
\item What is the statement that can be used in Python if the program requires no action but requires it syntactically?
  
The pass statement is a null operation. Nothing happens when it executes. You should use “pass” keyword in lowercase. If you write “Pass,” you’ll face an error like “NameError: name Pass is not defined.” Python statements are case sensitive.

\begin{verbatim}
letter = "hai sethuraman"
for i in letter:
    if i == "a":
        pass
        print("pass statement is execute ..............")
    else:
        print(i)
\end{verbatim}

\item What are the built-in types available in Python?
  
Here is the list of most commonly used built-in types that Python supports:
\begin{itemize}
\item Immutable built-in datatypes of Python
\item Numbers
\item Strings
\item Tuples
\end{itemize}

\begin{itemize}
\item Mutable built-in datatypes of Python
\item List
\item Dictionaries
\item Sets
\end{itemize}

\item What is the principal difference between a list and the tuple ?

The principal difference between a list and the tuple is that the former is mutable while the tuple is not.
A tuple is allowed to be hashed, for example, using it as a key for dictionaries.

What do you think is the output of the following code fragment? Is there any error in the code ?

list = ['a', 'b', 'c', 'd', 'e']
print (list[10:])

The result of the above lines of code is []. There won’t be any error like an IndexError.
You should know that trying to fetch a member from the list using an index that exceeds the member count (for example, attempting to access list[10] as given in the question) would yield an IndexError. By the way, retrieving only a slice at the starting index that surpasses the no. of items in the list won’t result in an IndexError. It will just return an empty list.



What is slicing in Python ?
Slicing is a string operation for extracting a part of the string, or some part of a list. In Python, a string (say text) begins at index 0, and the nth character stores at position text[n-1]. Python can also perform reverse indexing, i.e., in the backward direction, with the help of negative numbers. In Python, the slice() is also a constructor function which generates a slice object. The result is a set of indices mentioned by range(start, stop, step). The slice() method allows three parameters. 1. start – starting number for the slicing to begin. 2. stop – the number which indicates the end of slicing. 3. step – the value to increment after each index (default = 1).

Is a string immutable or mutable in Python ?
Python strings are indeed immutable.
Let’s take an example. We have an str variable holding a string value. We can’t mutate the container, i.e., the string, but can modify what it contains that means the value of the variable.

>>> a = "matteo"
>>> a[0] = "t"
Traceback (most recent call last):
  File "<stdin>", line 1, in <module>
TypeError: 'str' object does not support item assignment

What is a function in Python programming ?
A function is an object which represents a block of code and is a reusable entity. It brings modularity to a program and a higher degree of code reusability.
Python has given us many built-in functions such as print() and provides the ability to create user-defined functions.

How do we write a function in Python ?
We can create a Python function in the following manner.

Step-1: to begin the function, start writing with the keyword def and then mention the function name.

Step-2: We can now pass the arguments and enclose them using the parentheses. A colon, in the end, marks the end of the function header.

Step-3: After pressing an enter, we can add the desired Python statements for execution.

What is a function call or a callable object In Python ?
A function in Python gets treated as a callable object. It can allow some arguments and also return a value or multiple values in the form of a tuple. Apart from the function, Python has other constructs, such as classes or the class instances which fits in the same category.

What is the return keyword used for in Python ?
The purpose of a function is to receive the inputs and return some output.
The return is a Python statement which we can use in a function for sending a value back to its caller.

What is “call by value” in Python ?
In call-by-value, the argument whether an expression or a value gets bound to the respective variable in the function. Python will treat that variable as local in the function-level scope. Any changes made to that variable will remain local and will not reflect outside the function.

What is “call by reference” in Python ?
We use both “call-by-reference” and “pass-by-reference” interchangeably. When we pass an argument by reference, then it is available as an implicit reference to the function, rather than a simple copy. In such a case, any modification to the argument will also be visible to the caller. This scheme also has the advantage of bringing more time and space efficiency because it leaves the need for creating local copies. On the contrary, the disadvantage could be that a variable can get changed accidentally during a function call. Hence, the programmers need to handle in the code to avoid such uncertainty.

Is it mandatory for a python function to return a value ?
It is not at all necessary for a function to return any value. However, if needed, we can use None as a return value.

Does Python have a main method ?
The main() is the entry point function which happens to be called first in most programming languages.
Since Python is interpreter-based, so it sequentially executes the lines of the code one-by-one.
Python also does have a main() method. But it gets executed whenever we run our Python script either by directly clicking it or starts it from the command line.
We can also override the Python default main() function using the Python if statement. Please see the below code.

print("Welcome")
print("__name__ contains: ", __name__)
def main():
    print("Testing the main function")
if __name__ == '__main__':
    main()

The output:

Welcome
__name__ contains:  __main__
Testing the main function

What does the __name__ do in Python ?
The __name__ is a unique variable. Since Python doesn’t expose the main() function, so when its interpreter gets to run the script, it first executes the code which is at level 0 indentation.
To see whether the main() gets called, we can use the __name__ variable in an if clause compares with the value __main__.


What is the difference between pass and continue in Python ?
The continue statement makes the loop to resume from the next iteration.
On the contrary, the pass statement instructs to do nothing, and the remainder of the code executes as usual.

What does the len function do in Python ?
In Python, the len() is a primary string function. It determines the length of an input string.

>>> some_string = 'techbeamers'
>>> len(some_string)
11

How do you use the split function in Python ?
Python’s split() function works on strings to cut a large piece into smaller chunks, or sub-strings. We can specify a separator to start splitting, or it uses the space as one by default.

#Example
str = 'pdf csv json'
print(str.split(" "))
print(str.split())
The output:

['pdf', 'csv', 'json']
['pdf', 'csv', 'json']

What does the join method do in Python ?
Python provides the join() method which works on strings, lists, and tuples. It combines them and returns a united value.

What is a tuple in Python ?
A tuple is a collection type data structure in Python which is immutable.
They are similar to sequences, just like the lists. However, There are some differences between a tuple and list; the former doesn’t allow modifications whereas the list does.
Also, the tuples use parentheses for enclosing, but the lists have square brackets in their syntax.

What is the set object in Python ?
Sets are unordered collection objects in Python. They store unique and immutable objects. Python has its implementation derived from mathematics.

What is class in Python ?
Python supports object-oriented programming and provides almost all OOP features to use in programs.
A Python class is a blueprint for creating the objects. It defines member variables and gets their behavior associated with them.
We can make it by using the keyword “class.” An object gets created from the constructor. This object represents the instance of the class.
In Python, we generate classes and instances in the following way.

>>>class Human:  # Create the class
...     pass
>>>man = Human()  # Create the instance
>>>print(man)
<__main__.Human object at 0x0000000003559E10>

What are attributes and methods in a python class ?
A class is useless if it has not defined any functionality. We can do so by adding attributes. They work as containers for data and functions. We can add an attribute directly specifying inside the class body.

>>> class Human:
...     profession = "programmer" # specify the attribute 'profession' of the class
>>> man = Human()
>>> print(man.profession)
Programmer

After we added the attributes, we can go on to define the functions. Generally, we call them methods. In the method signature, we always have to provide the first argument with a self-keyword.

>>> class Human:
    profession = "programmer"
    def set_profession(self, new_profession):
        self.profession = new_profession      
>>> man = Human()
>>> man.set_profession("Manager")
>>> print(man.profession)
Manager

How to assign values for the class attributes at Runtime ?
We can specify the values for the attributes at runtime. We need to add an __init__ method and pass input to object constructor. See the following example demonstrating this.

>>> class Human:
    def __init__(self, profession):
        self.profession = profession
    def set_profession(self, new_profession):
        self.profession = new_profession

>>> man = Human("Manager")
>>> print(man.profession)
Manager



What does the self keyword do ?
The self is a Python keyword which represents a variable that holds the instance of an object.
In almost, all the object-oriented languages, it is passed to the methods as a hidden parameter.





    
    
    \end{document}

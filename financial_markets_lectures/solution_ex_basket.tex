\documentclass[]{article}

\title{Basket Default Swap}
\author{}
\usepackage{exsheets}
\SetupExSheets{counter-format=ch.qu}

\usepackage{epigraph}
\setlength{\epigraphwidth}{0.6\textwidth}
\setlength{\epigraphrule}{0pt}
\newcommand{\myopeningquote}{\smash{\raisebox{-0.25\height}{\llap{\scalebox{3}
				{\textcolor{silver}{``}}\,}}}}
\newcommand{\myclosingquote}{\raisebox{-0.86\height}{\scalebox{3}{\textcolor{silver}{”}}}}

\usepackage{sectsty}
\usepackage{titlesec}
\chapterfont{\color{rahmen}}
\subsectionfont{\color{rahmen}}
\subsubsectionfont{\color{rahmen}}
\titleformat{\section}
{\normalfont\Large\bfseries\color{rahmen}}{\textcolor{rahmen}\thesection}{1em}{}[\textcolor{rahmen}{\titlerule[0.1pt]}]

\patchcmd{\thebibliography}{\chapter*}{\section*}{}{}

\usepackage{afterpage}
\usepackage[inline]{enumitem}
\usepackage{cancel}
\usepackage{tabularx}
\usepackage{booktabs}
\usepackage{makecell}
\setcellgapes{4pt}

\usepackage{geometry}
\geometry{verbose,tmargin=2.5cm,bmargin=2.5cm,lmargin=2.5cm,rmargin=2.5cm,headheight=14pt}

\usepackage[most]{tcolorbox}
\newtcolorbox{attention}{breakable,enhanced,arc=2mm,colback=cream,colframe=goldenrod,leftrule=12mm,
	overlay={\node[anchor=north west,outer sep=2pt] at (frame.north west) {\includegraphics[width=8mm]{figures/alert.png}}; }}
\newtcolorbox{curiosity}{breakable,enhanced,arc=2mm,colback=lime,colframe=green,leftrule=12mm,
	overlay={\node[anchor=north west,outer sep=2pt] at (frame.north west) {\includegraphics[width=8mm]{figures/curiosity.png}}; }}
\newtcolorbox{finmarkets}{breakable,enhanced,arc=2mm,colback=sky,colframe=rahmen,leftrule=12mm,
	overlay={\node[anchor=north west,outer sep=2pt] at (frame.north west) {\includegraphics[width=8mm]{figures/python_logo.png}}; }}

%\usepackage{amssymb}
\usepackage{amsfonts} % for expectation and other chars
\usepackage{bbm} % for indicator function
\usepackage{eurosym} % defines \euro

\usepackage{xcolor}
\definecolor{mandellightgreen}{rgb}{0.77, 0.82, 0.73}
\definecolor{mandelpantone}{rgb}{0.2, 0.11, 0.27}
\definecolor{ipython-frame}{RGB}{207, 207, 207}
\definecolor{ipython-bg}{RGB}{247, 247, 247}
\definecolor{ipython-red}{RGB}{186, 33, 33}
\definecolor{ipython-green}{RGB}{0, 128, 0}
\definecolor{ipython-cyan}{RGB}{64, 128, 128}
\definecolor{ipython-purple}{RGB}{170, 34, 255}
\definecolor{cream}{RGB}{255,229,153}
\definecolor{goldenrod}{RGB}{218,165,32}
\definecolor{sky}{RGB}{150, 195, 255}
\definecolor{rahmen}{RGB}{0,73,114}
\definecolor{ansi-red}{HTML}{E75C58}
\definecolor{ansi-green}{HTML}{00A250}
\definecolor{silver}{HTML}{C0C0C0}

\newcommand\Pdef{P_{\text{def}}}
\newcommand\Psur{P_{\text{sur}}}
\newcommand\corr{\text{Corr}}
\newcommand\cov{\text{Cov}}
\newcommand\var{\text{Var}}
\newcommand\std{\text{Std}}

\usepackage{hyperref}
\hypersetup{
	breaklinks=true,  % so long urls are correctly broken across lines
	colorlinks=true,
	urlcolor=rahmen,
	linkcolor=rahmen,
	citecolor=rahmen,
}
\hypersetup{linkcolor=rahmen}

\providecommand{\tightlist}{\setlength{\itemsep}{0pt}\setlength{\parskip}{0pt}}

\usepackage{subfig} %for subfloat

\usepackage{cprotect}
\lstdefinelanguage{iPython} {
	morekeywords={access,and,del,except,exec,in,is,lambda,not,or,raise},
	morekeywords=[2]{for,print,abs,all,any,basestring,bin,bool,bytearray,callable,chr,classmethod,cmp,compile,complex,delattr,dict,dir,divmod,enumerate,eval,execfile,file,filter,float,format,frozenset,getattr,globals,hasattr,hash,help,hex,id,input,int,isinstance,issubclass,iter,len,list,locals,long,map,max,memoryview,min,next,object,oct,open,ord,pow,property,range,reduce,reload,repr,reversed,round,set,setattr,slice,sorted,staticmethod,str,sum,super,tuple,type,unichr,unicode,vars,xrange,zip,apply,buffer,coerce,intern,elif,else,if,continue,break,while,class,def,return,try,except,import,finally,try,except,from,global,pass, True, False},
	sensitive=true,
	morecomment=[l]\#,
	morestring=[b]',
	morestring=[b]",
	moredelim=**[is][\color{black}]{@@}{@@},
	identifierstyle=\color{black}\footnotesize\ttfamily,
    commentstyle=\color{ipython-cyan}\footnotesize\itshape\ttfamily,
	stringstyle=\color{ipython-red}\footnotesize\ttfamily,
	keepspaces=true,
	showspaces=false,
    showstringspaces=false,
	rulecolor=\color{ipython-frame},
	frame=single,
	frameround={t}{t}{t}{t},
	framexleftmargin=6mm,
	numbers=left,
	numberstyle=\color{black},
	basicstyle=\footnotesize\ttfamily,
	keywordstyle=[2]\color{ipython-green}\bfseries\footnotesize\ttfamily, 
	keywordstyle=\color{ipython-purple}\bfseries\footnotesize\ttfamily,
	mathescape=true,
	escapeinside={*@}{@*}
}

\lstdefinelanguage{iOutput} {
	sensitive=true,
	identifierstyle=\color{black}\small\ttfamily,
	stringstyle=\color{ipython-red}\small\ttfamily,
	keepspaces=true,
	showspaces=false,
	showstringspaces=false,
	rulecolor=\color{ipython-frame},
	%frame=single,
	%frameround={t}{t}{t}{t},
	%backgroundcolor=\color{ipython-bg},
	basicstyle=\small\ttfamily,
	mathescape=true,
	escapeinside={*@}{@*}
}

\lstdefinestyle{mycpp}{language=C++,
	basicstyle=\small\ttfamily,
	showstringspaces=false,
	keywordstyle=\small\color{blue}\ttfamily,
	stringstyle=\small\color{red}\ttfamily,
	commentstyle=\small\color{green}\ttfamily,
	rulecolor=\color{ipython-frame},
	frame=single,
	frameround={t}{t}{t}{t},
	backgroundcolor=\color{ipython-bg},
	morecomment=[l][\color{magenta}]{\#}
}

\lstdefinestyle{attstyle}{
	backgroundcolor=\color{cream},
	rulecolor=\color{goldenrod}
}

\lstnewenvironment{attpython}[1][]{
	\lstset{language=iPython,style=attstyle,mathescape=true}
}{}

\lstnewenvironment{ipython}[1][]{\lstset{language=iPython,backgroundcolor=\color{ipython-bg}}}{}

\lstnewenvironment{ipythonnon}[1][]{\lstset{language=iPython,backgroundcolor=\color{ipython-bg},numbers=none,framexleftmargin=0mm}}{}

\lstnewenvironment{ipythonbox}[1][]{\lstset{language=iPython,numbers=none,framexleftmargin=0mm}}{}

\lstnewenvironment{ioutput}[1][]{\lstset{language=iOutput}}{}

\newcommand{\pythoncode}[1]{
    \lstinputlisting[language=iPython,backgroundcolor=\color{ipython-bg}]{#1}
}

\newcommand{\pythoncodenon}[1]{
    \lstinputlisting[language=iPython,numbers=none,backgroundcolor=\color{ipython-bg},framexleftmargin=0mm]{#1}
}

\newcommand{\pythoncodebox}[1]{
	\lstinputlisting[language=iPython,numbers=none,framexleftmargin=0mm]{#1}
}

\usepackage{fancyhdr}
\pagestyle{fancy}
%\renewcommand{\chaptermark}[1]{\markboth{\thechapter.\ #1}{}}
\renewcommand{\sectionmark}[1]{\markboth{\thesection.\ #1}{}}
\fancypagestyle{main}{
	\fancyhf{}
    \fancyhead[LO,RE]{\leftmark}
	\fancyhead[LE,RO]{\thepage}
	%\fancyfoot[C]{\thepage}
	\renewcommand{\headrulewidth}{0.4pt}
}


\begin{document}

\maketitle

\section{Derivative Definition}
A \emph{basket default swap} (BDS) is a credit derivative on a portfolio of $n$ reference entities.
The simplest basket default swaps are first-to-default, second-to-default, \ldots, $n$th-to-default swaps. It is very similar to a standard CDS except for the protection it can offer, indeed:
\begin{itemize}
\item a first-to-default swap provides insurance for only the first default happening;
\item a second-to-default swap provides insurance for only the second default
\item \ldots
\end{itemize}

For example, in a $n$th-to-default swap, the contract seller does not make any payment to the protection buyer for the first $n - 1$ defaulted entities, but makes it only for the $n$th default. Like in a CDD once there has been this payment the swap terminates.

\subsection{Valuation}
Let's assume that the principals and the expected recovery rates are the same for all the underlying reference assets. Let's also consider that all the assets have the same default probability.

The valuation procedure is quite similar to that for a regular CDS:
\begin{itemize}
\item the value of a CDS depends on the probability of default ($P_{\textrm{def}}$) of the reference asset within the life of the contract;
\item the value of a BDS depends on the probability $P^{(n)}_{\textrm{def}}$ that $n$ assets, among those of the reference basket, defaulted within the life of the contract.
\end{itemize}

Similarly to the CDS, the protection buyer makes quarterly payments at a specified rate until the $n$th default occurs or the end of the contract is reached. In case the $n$th default occurred, the seller will pay $F\cdot(1 − R)$, the \emph{loss given default}.

\subsection{Algorithm Implementation}

BDS valuation implementation follows closely the code written for the CDS, except that we are not going to consider the credit curve of a single asset but rather the joint default probability of the entire basket focusing on the $n$th default according to the BDS definition.
So we can reuse entirely the CDS code and focus on the new credit curve definition.
Since we have already implemented a \texttt{CreditCurve} class which takes as parameters a set of pillar dates and the corresponding non-default probabilities, we are going to define the relevant credit curve for our BDS in a discrete way.

Without loss of generality we can assume Gaussian correlation among the $P_{\textrm{def}}$ of the entities and that each $P_{\textrm{def}}$ is modelled according to a Poisson process with a constant hazard rate $\lambda$. So we proceed as follows:
\begin{enumerate}
\item create a $N$ dimensional Gaussian copula, with $N$ being the number of entities in the basket. Then sample $m$ times from the distribution to get a $N\times m$ array of correlated $P_{def}$;
\begin{ipython}
import numpy as np
from finmarkets import GaussianCopula

m = 100000
N = 5
rho = 0.5
cov = np.ones(shape=(N, N))*rho
np.fill_diagonal(cov, 1)
g_copula = GaussianCopula(N, cov)
copula_sample = g_copula.sample(m)

print ("copula sample shape: ", copula_sample.shape)
print (copula_sample)
\end{ipython}
\begin{ioutput}
copula sample shape: (100000, 5)
[[0.60480404 0.79241128 0.7922326  0.58344479 0.45601573]
[0.97967882 0.47501304 0.70409046 0.14648779 0.33119248]
[0.5197925  0.90746284 0.71911478 0.64719254 0.42531869]
...
[0.20182568 0.05399022 0.6359692  0.40018493 0.11124106]
[0.05005798 0.60204771 0.08187652 0.91139689 0.10944598]
[0.60788262 0.34598408 0.57910356 0.10051283 0.60155192]]
\end{ioutput}
Notice that the covariance matrix used in the multivariate Gaussian distribution has been defined into two steps:
\begin{equation*}
	cov = \begin{bmatrix}
		1 & 1 & 1 & 1 & 1 \\
		1 & 1 & 1 & 1 & 1 \\
		1 & 1 & 1 & 1 & 1 \\
		1 & 1 & 1 & 1 & 1 \\
		1 & 1 & 1 & 1 & 1 \\
	\end{bmatrix}\cdot \rho = 
	\begin{bmatrix}
		\rho & \rho & \rho & \rho & \rho \\
		\rho & \rho & \rho & \rho & \rho \\
		\rho & \rho & \rho & \rho & \rho \\
		\rho & \rho & \rho & \rho & \rho \\
		\rho & \rho & \rho & \rho & \rho \\
	\end{bmatrix}
\end{equation*}
Then the diagonal elements have been replaced with 1s with \texttt{np.fill\_diagonal};
\item applying the PPF transformation of the Poisson process, convert the copula sample into a set of default times (an $N\times m$ array of $\tau_{def}$). Each row of this matrix represents a set default times, one for each of the $N$ entities of the basket;

\begin{ipython}
from finmarkets import PoissonProcess

poisson = PoissonProcess(l=0.06)
tau_def = poisson.ppf(copula_sample)

print (f"tau_def shape: {tau_def.shape}")
print (tau_def)
\end{ipython}
\begin{ioutput}
tau_def shape: (100000, 5)
[[15.47289228 26.20327421 26.18893479 14.59560447 10.14724927]
[64.93486319 10.73969756 20.29502463  2.63992387  6.7043162 ]
[12.22561614 39.669083   21.16348612 17.36388026  9.23232736]
...
[ 3.75713756  0.92503952 16.8419465   8.51889812  1.9654874 ]
[ 0.8559054  15.35705259  1.42372316 40.39313865  1.93185857]
[15.603234    7.07705965 14.42280786  1.76550815 15.33630145]]
\end{ioutput}

\item define a set of pillar dates $\{d_i\}$, in this example 1 to 6 years from the observation date;
\begin{ipython}
pillars = [i for i in range(1, 6)]
\end{ipython}

\item for each pillar $d_i$ and for each simulation of default times, count how many entities defaulted before the pillar date ($\tau_{def}\leq d_i$) in that particular simulation. In such a way we get a $N$ vector of number of defaults $e_{defs}$), one for each simulation;
\begin{ipython}
entity_defs = np.sum(tau_def <= pillars[0], axis=1)

print (f"entity_defs shape: {entity_defs.shape}")
print (entity_defs)
\end{ipython}
\begin{ioutput}
entity_defs shape: (100000,)
[0 0 0 ... 1 1 0]
\end{ioutput}
To compute the number of defaulted entities before the considered pillar date per simulation, we have first checked element-wise if the default time is earlier than the pillar date with \texttt{tau\_def <= pillars[0]}. Then with \texttt{np.sum} (with the parameter \texttt{axis=1} we have aggregated the results by simulation.
\item calculate the number of simulations in which there are at least $n$ defaulted entities \texttt{entity\_defs};
\begin{ipython}
nth_default = 3

n_defs = np.sum(entity_defs>=nth_default)
print (f"simulations with {nth_default} defaults: {n_defs}")
\end{ipython}
\begin{ioutput}
simulations with 3 defaults: 2505
\end{ioutput}
Again we have used \texttt{np.sum} this time to aggregate the results for the entire set of simulations. 
\item finally compute the nth-to-default survival probability for the pillar $d_i$ as ($P_{\textrm{sur}} = 1 - n_{defs}/m$);
\begin{ipython}
P_sur = 1 - n_defs/m
print (f"surv. prob.: {P_sur}")
\end{ipython}
\begin{ioutput}
surv. prob.: 0.97495
\end{ioutput}
\end{enumerate}
Putting all together we can define a \texttt{BasketDefaultSwap} class:
\begin{ipython}
import numpy as np
from finmarkets import CreditCurve, CreditDefaultSwap, PoissonProcess
from scipy.stats import multivariate_normal, norm

class BasketDefaultSwaps:
    def __init__(self, nominal, N, start_date, maturity, spread, 
                 tenor="3m", recovery=0.4):
        self.cds = CreditDefaultSwap(nominal, start_date, maturity, spread, tenor, recovery)
        self.N = N
        self.cc = None

    def credit_curve(self, n_defaults, copula_func, default_prob, 
                     obs_date, pillars, simulations=100000):
        copula_sample = copula_func.sample(simulations)
        default_times = default_prob.ppf(copula_sample)

        Ts = [(p-obs_date).days/365 for p in pillars]
        ndps = []
        for t in Ts:
            entity_defs_per_sim = np.sum(default_times <= t, axis=1)
            tot_defs = np.sum(entity_defs_per_sim >= n_defaults)
            ndps.append(1 - tot_defs/simulations)
        self.cc = CreditCurve(obs_date, pillars, ndps)

    def npv(self, dc):
        if self.cc is None:
            print ("Need to call credit_curve method first !")
            return None
        return self.cds.npv(dc, self.cc)
\end{ipython}

\section{Testing}
Consider a 2-years 3rd-to-default BDS on 10 reference entities with a Gaussian copula correlation of 0.3. The expected recovery rate, $R$, is $40\%$ and the spread of the contract is 0.01. The term structure of interest rates is assumed to be flat at 5%.

The 10 entities default probabilities are modelled by Poisson processes with constant hazard rates, $\lambda_i = 0.01$, such that
\begin{equation*}
Q(t) = 1 - e^{-\lambda t}	
\end{equation*}

\begin{ipython}
from finmarkets import DiscountCurve, CreditCurve
from datetime import date
from dateutil.relativedelta import relativedelta

n_cds = 10
rho = 0.3
cov = np.ones(shape=(n_cds, n_cds))*rho
np.fill_diagonal(cov, 1)
copula = GaussianCopula(n_cds, cov)

obs_date = date.today()
pillar_dates = [obs_date + relativedelta(years=i) for i in range(6)]
dfs = [1/(1+0.05)**i for i in range(6)]
dc = DiscountCurve(obs_date, pillar_dates, dfs)

np.random.seed(1)
basket = BasketDefaultSwaps(1, n_cds, obs_date, "2y", 0.01, "3m")
basket.credit_curve(n_defaults=3, copula_func=copula, default_prob=poisson, 
                    obs_date=obs_date, pillars=pillar_dates)
print (basket.npv(dc))
\end{ipython}
\begin{ioutput}
0.070757976355211
\end{ioutput}
\end{document}

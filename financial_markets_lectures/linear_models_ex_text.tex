
\cprotEnv\begin{question}
The Dow-Jones Industrial Average (DJIA) was first introduced by Charles Dow in 1896 and has since become one of the main references for stock market performance on the New York Stock Exchange.

In this post, we will use it to better understand the pros and cons of simple Linear Regression models, the assumptions they rely on, and how we can use them for feature selection.

Its value is determined with a weighted average of 30 stock prices. Since the Dow components change over time, using the provided dataset (\href{"https://github.com/matteosan1/finance_course/raw/master/input_files/dji.csv}{dji.csv}) and a list of 35 assets, determine which of them are included in the calculation given a time interval. Try to interpret the results.

\textbf{Hint}: in order to cross-check your result split the dataset in two parts (before and after August 2020), then fit the model with the first part and test it with the second.
\begin{ipython}
selected = ['AAPL', 'AMGN', 'AXP', 'BA', 'CAT', 'CRM', 'CSCO', 'CVX', 'DD', 'DIS', 
            'DOW', 'GE', 'GS', 'HD', 'HON', 'IBM', 'INTC', 'JNJ', 'JPM', 'KO', 'MCD', 
            'MMM', 'MRK', 'MSFT', 'NKE', 'PFE', 'PG', 'RTX', 'TRV', 'UNH', 'V', 'VZ', 
            'WBA', 'WMT', 'XOM']
\end{ipython}
\end{question}

\cprotEnv\begin{solution}
Load the dataset and split into two dataframes DJI and the other tickers.
\begin{ipython}
import pandas as pd

closing = pd.read_csv("dji.csv", index_col='Date')
DJI = closing['DJI'].copy()
closing.drop(columns='DJI', inplace=True)
\end{ipython} 
	
\begin{figure}[htbp]
\centering
\includegraphics[width=0.7\textwidth]{figures/dji}
\caption{DJIA index daily values in 2020. The dataset has been divided into two parts for training and testing purpose.}
\label{fig:dji-returns}
\end{figure}
	
We need to fit the linear model to the available data prior to August 2020 (\emph{notice that the dataset already contains the intercept column to take care of the $\alpha$ parameter}), see Fig.~/ref{fig:dji-returns}.
\begin{ipython}
import statsmodels.api as sm

model = sm.OLS(DJI[DJI.index <='2020-07-31'], 
               closing[closing.index <='2020-07-31']).fit()
print (model.summary())
\end{ipython}
\begin{ioutput}
                          OLS Regression Results                            
==============================================================================
Dep. Variable:                    DJI   R-squared:                       1.000
Model:                            OLS   Adj. R-squared:                  1.000
Method:                 Least Squares   F-statistic:                 2.107e+06
Date:                Tue, 08 Nov 2022   Prob (F-statistic):          1.99e-297
Time:                        18:16:13   Log-Likelihood:                -364.32
No. Observations:                 143   AIC:                             800.6
Df Residuals:                     107   BIC:                             907.3
Df Model:                          35                                         
Covariance Type:            nonrobust                                         
==============================================================================
coef    std err          t      P>|t|      [0.025      0.975]
------------------------------------------------------------------------------
AAPL          28.1315      0.238    118.405      0.000      27.660      28.602
AMGN           0.0701      0.098      0.714      0.477      -0.124       0.265
AXP            6.2508      0.261     23.940      0.000       5.733       6.768
BA             7.0296      0.047    149.349      0.000       6.936       7.123
CAT            6.8530      0.160     42.953      0.000       6.537       7.169
CRM            0.0452      0.123      0.367      0.714      -0.199       0.289
CSCO           6.9258      0.564     12.274      0.000       5.807       8.044
CVX            7.6117      0.260     29.233      0.000       7.096       8.128
DD             0.4722      0.318      1.484      0.141      -0.159       1.103
DIS            6.9450      0.178     39.057      0.000       6.593       7.298
DOW            6.3940      0.387     16.541      0.000       5.628       7.160
GE             0.2437      0.216      1.130      0.261      -0.184       0.671
GS             6.9331      0.122     56.636      0.000       6.690       7.176
HD             6.6309      0.096     69.009      0.000       6.440       6.821
HON           -0.3708      0.193     -1.918      0.058      -0.754       0.012
IBM            7.3397      0.202     36.317      0.000       6.939       7.740
INTC           6.5288      0.199     32.789      0.000       6.134       6.924
JNJ            6.7114      0.240     27.962      0.000       6.236       7.187
JPM            6.7418      0.236     28.578      0.000       6.274       7.209
KO             6.9641      0.572     12.173      0.000       5.830       8.098
MCD            6.8349      0.150     45.446      0.000       6.537       7.133
MMM            6.4935      0.146     44.401      0.000       6.204       6.783
MRK            7.5556      0.318     23.792      0.000       6.926       8.185
MSFT           6.4336      0.155     41.509      0.000       6.126       6.741
NKE            6.7713      0.231     29.369      0.000       6.314       7.228
PFE            8.0045      0.466     17.163      0.000       7.080       8.929
PG             7.0308      0.294     23.894      0.000       6.447       7.614
RTX            8.4790      0.301     28.136      0.000       7.882       9.076
TRV            6.7275      0.194     34.767      0.000       6.344       7.111
UNH            6.7560      0.073     93.183      0.000       6.612       6.900
V              7.2610      0.174     41.824      0.000       6.917       7.605
VZ             6.6514      0.631     10.546      0.000       5.401       7.902
WBA            6.2044      0.331     18.750      0.000       5.548       6.860
WMT            6.7940      0.211     32.258      0.000       6.376       7.212
XOM            6.0549      0.487     12.435      0.000       5.090       7.020
intercept     -4.1082     21.263     -0.193      0.847     -46.259      38.043
==============================================================================
Omnibus:                       19.720   Durbin-Watson:                   2.012
Prob(Omnibus):                  0.000   Jarque-Bera (JB):               62.557
Skew:                          -0.398   Prob(JB):                     2.61e-14
Kurtosis:                       6.141   Cond. No.                     5.70e+04
==============================================================================
\end{ioutput}

From the $p$-values in Figure~\ref{fig:p-values} (left), we can quickly identify the stocks whose weights are not meaningfully different from zero, namely: AMGN, CRM, DD, GE, and HON. The reason is simply that they are not part of the DJIA in the considered period.
Repeating the fit with the updated model shows that all the included tickers are still contributing to the DJIA index determination, see Fig.~\ref{fig:p-values} (right).
	
\begin{ipython}
selected_columns = list(model.pvalues[model.pvalues<0.05].index)

model_small = sm.OLS(DJI[DJI.index <= '2020-07-31'], 
                     closing[selected_columns][closing.index <= '2020-07-31']).fit()
print (model_small.summary())
\end{ipython}
\begin{ioutput}
                        OLS Regression Results                                
==============================================================================
Dep. Variable:                    DJI   R-squared (uncentered):          1.000
Model:                            OLS   Adj. R-squared (uncentered):     1.000
Method:                 Least Squares   F-statistic:                 2.434e+08
Date:                Thu, 26 Jan 2023   Prob (F-statistic):               0.00
Time:                        11:59:44   Log-Likelihood:                -369.56
No. Observations:                 143   AIC:                             799.1
Df Residuals:                     113   BIC:                             888.0
Df Model:                          30                                                  
Covariance Type:            nonrobust                                                  
==============================================================================
                 coef    std err          t      P>|t|      [0.025      0.975]
------------------------------------------------------------------------------
AAPL          28.1695      0.229    123.120      0.000      27.716      28.623
AXP            6.2176      0.250     24.878      0.000       5.722       6.713
BA             7.0379      0.045    157.710      0.000       6.949       7.126
CAT            6.7754      0.133     50.871      0.000       6.511       7.039
CSCO           7.2883      0.535     13.620      0.000       6.228       8.348
CVX            7.5201      0.239     31.522      0.000       7.047       7.993
DIS            6.8027      0.163     41.760      0.000       6.480       7.125
DOW            6.6917      0.338     19.793      0.000       6.022       7.361
GS             6.9423      0.116     59.693      0.000       6.712       7.173
HD             6.6851      0.089     75.149      0.000       6.509       6.861
IBM            7.2860      0.196     37.126      0.000       6.897       7.675
INTC           6.4310      0.189     34.007      0.000       6.056       6.806
JNJ            6.7935      0.215     31.583      0.000       6.367       7.220
JPM            6.8400      0.214     31.945      0.000       6.416       7.264
KO             6.5188      0.494     13.206      0.000       5.541       7.497
MCD            6.7657      0.139     48.518      0.000       6.489       7.042
MMM            6.4385      0.121     53.174      0.000       6.199       6.678
MRK            7.5965      0.296     25.622      0.000       7.009       8.184
MSFT           6.5577      0.134     48.935      0.000       6.292       6.823
NKE            6.7307      0.208     32.385      0.000       6.319       7.142
PFE            8.0817      0.454     17.806      0.000       7.182       8.981
PG             6.9862      0.267     26.144      0.000       6.457       7.516
RTX            8.6628      0.275     31.476      0.000       8.118       9.208
TRV            6.5894      0.181     36.417      0.000       6.231       6.948
UNH            6.6966      0.066    100.982      0.000       6.565       6.828
V              7.2778      0.160     45.487      0.000       6.961       7.595
VZ             6.3851      0.526     12.134      0.000       5.343       7.428
WBA            6.2725      0.304     20.631      0.000       5.670       6.875
WMT            6.9055      0.198     34.832      0.000       6.513       7.298
XOM            6.5498      0.443     14.782      0.000       5.672       7.428
==============================================================================
Omnibus:                       27.413   Durbin-Watson:                   1.905
Prob(Omnibus):                  0.000   Jarque-Bera (JB):               91.998
Skew:                          -0.630   Prob(JB):                     1.05e-20
Kurtosis:                       6.722   Cond. No.                     1.70e+03
==============================================================================
\end{ioutput}
\begin{figure}[htbp]
\centering
\includegraphics[width=0.4\textwidth]{figures/p-values}
\includegraphics[width=0.4\textwidth]{figures/delta_weights}
\caption{$p$-values resulting from the linear model fit to DJIA returns (left). Parameter percentage difference according to the two models described in the text (right).}
\label{fig:p-values}
\end{figure}
	
Now that we have a good model it is possible to check how it behaves on the test sample (i.e. from August to December 2020). The result is shown in Figure~\ref{fig:prediction}. The residuals plot, the difference between predicted and actual values of the DJIA, shows that there is a clear degradation of the performance after September, see Fig.~\ref{fig:residuals}.

\begin{ipython}
residuals = DJI-model_small.predict(closing[selected_columns])
\end{ipython}
	
\begin{figure}[htbp]
\centering
\includegraphics[width=0.7\textwidth]{figures/prediction}
\caption{Model prediction to DJIA index performance comparison.}
\label{fig:prediction}
\end{figure}

As we can see, our model works extremely well for the first few weeks of the testing period and then it completely falls apart. The reason for this puzzling behavior is surprisingly simple: the fundamental assumption underlying our model was no longer valid. The \textbf{composition of the DJIA changed on Aug 31st, 2020} with Pfizer, Raytheon, and Exxon Mobile being dropped and replaced by Amgen, Honeywell, and Salesforce.

\begin{figure}[htbp]
\centering
\includegraphics[width=0.7\textwidth]{figures/residuals}
\caption{Residuals of the linear model.}
\label{fig:residuals}
\end{figure}
\end{solution}

\begin{question}
Stock A has a $\beta$ of 1.2 and Stock B has a $\beta$ of 0.6. Which of the following statements is true? 
\begin{enumerate}[label=\emph{\alph*})]
\tightlist
\item Stock A has more unsystematic risk than Stock B;
\item Stock B has more systematic risk than Stock A; 
\item if the risk-free rate and the market risk premium are both positive, Stock A has a higher expected return than Stock B according to the CAPM;
\item both \emph{a} and \emph{b} are true;
\item both \emph{b} and \emph{c} are true
\end{enumerate}
\end{question}

\begin{solution}
The correct answer is \emph{c} indeed according to CAPM formula
\[r_i = r_f + \beta_i (r_M - r_f)\]
if both $r_f$ and $(r_M - r_f)$ are positive then $r_A > r_B$.
\end{solution}	

\begin{question}
Consider a stock with a $\beta$ of 1.5. Which of the following statements is true?

\begin{enumerate}[label=\emph{\alph*}]
\tightlist
\item when the market goes down by 1.5\%, on average, the stock goes down by 1\%;
\item when the market goes up by 1.5\%, on average, the stock goes up by 1\%;
\item when the market goes up by 1\%, on average, the stock goes down by 1.5\%;
\item when the market goes down by 1\%, on average, the stock goes down by 1.5\%. 
\item both \emph{a} and \emph{b}.	
\end{enumerate}
\end{question}

\begin{solution}
The correct answer is \emph{d} since $\beta$ measures the slope of the regression line of the expected return of the stock vs the expected return of the market. So by definition it represents the variation of the stock expected return given the a unit variation of the expected return of the market.
\end{solution}	

\begin{question}
Suppose that the risk-free rate is 3\% and the market risk premium is 8\%. According to the CAPM, what is the required rate of return on a stock with a $\beta$ of 2 ?
\end{question}

\begin{solution}
Careful! The market risk premium is 8\%. This means that $r_M - r_f  = 8\%$. Plug this into the CAPM equation to get

\begin{equation*}
r = r_f + \beta(r_M - r_f) = 3\% + 2\cdot(8\%) =19\%
\end{equation*}
\end{solution}

\begin{question}
You analyze the prospects of several companies and come to the following conclusions about the required return on each:

\begin{center}
\begin{tabular}{lc}
\textbf{Stock Required} & \textbf{Return} \\
Starbucks &18\% \\
Sears &8\% \\
Microsoft &16\% \\
Limited Brands &12\% \\
\end{tabular}
\end{center}

You decide to invest \$4000 in Starbucks, \$6000 in Sears, \$12000 in Microsoft, and \$3000 in Limited Brands. What is the required return on your portfolio?
\end{question}

\cprotEnv \begin{solution}
\begin{ipython}
P = 4000 + 6000 + 12000 + 3000
rp =(4000/P)*0.18 + (6000/P)*0.08 + (12000/P)*0.16 +(3000/P)*0.12
print ("Portfolio Return: {:.2f}%".format(rp*100))
\end{ipython}
\begin{ioutput}
Portfolio Return: 13.92%
\end{ioutput}
\end{solution}	

\begin{question}
You have a portfolio that consists of 35\% Microsoft stock, 35\% Amazon stock, and 30\% GE stock. Microsoft has a $\beta$ of 1, Amazon has a $\beta$ of 3.0, and GE has a $\beta$ of 0.5. Treasury bills (the risk-free asset) currently offer a return of 4\%, and the expected return on the market is 11.5\%. What return should you expect on your portfolio according to the CAPM ? 
\end{question}

\cprotEnv \begin{solution}
You can work this problem two different ways.

\textbf{Method 1}: calculate the $\beta$ of your portfolio and plug this $\beta$ into the CAPM formula to get the required return of your portfolio.

\begin{ipython}
rm = 0.115
rf = 0.04
beta_p = 0.35 * 1 + 0.35 * 3 + 0.3 * 0.5
rp = rf + beta_p*(rm - rf)

print ("Portfolio beta: {:.2f}".format(beta_p))
print ("Portfolio return: {:.3f}%".format(rp*100))
\end{ipython}
\begin{ioutput}
Portfolio beta: 1.55
Portfolio return: 15.625%
\end{ioutput}

\textbf{Method 2}: using the CAPM, calculate the required return on each individual stock. Then,calculate the weighted average of those required returns to get the required return of your portfolio.

\begin{ipython}
Er_msft = rf + 1*(rm-rf)
Er_amzn = rf + 3*(rm-rf)
Er_ge = rf + 0.5*(rm-rf)
rp2 = Er_msft * 0.35 + Er_amzn *0.35 + Er_ge * 0.3

print ("Er (MSFT): {:.2f}%".format(Er_msft*100))
print ("Er (AMZN): {:.2f}%".format(Er_amzn*100))
print ("Er (GE): {:.2f}%".format(Er_ge*100))
print ("Portfolio return: {:.3f}%".format(rp2*100))
\end{ipython}
\begin{ioutput}
Er (MSFT): 11.50%
Er (AMZN): 26.50%
Er (GE): 7.75%
Portfolio return: 15.625%
\end{ioutput}
Notice that this is the same answer we got using method 1. When calculating the required return of a portfolio, it does not matter which way you do it. But method 1 is a little less work.
\end{solution}

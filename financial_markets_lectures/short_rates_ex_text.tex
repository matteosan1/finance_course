\begin{question}
Assume that company A has agreed to pay a 6-month LIBOR and receive a fixed interest rate of 8\% per year (with interest payable every six months) from the face value of \$100 million. Swap is 1.25 years to expire. The interest rates for 3, 9 and 15 months are: 10\%, 10.5\% and 11\% respectively. Assume that interest rates are continuously compounded. The 6-month LIBOR is currently 10.2\%. 

Calculate the value of this swap for company A.
\end{question}


%  Add to \texttt{finmarkets} module a class to valuate swaptions, it should have as attributes the underlying IRS, the swap rate volatility and the question date.
%  The class should have the capability of estimating the payoff of the swaptions both with the Black-Scholes formula and Monte Carlo Simulation.
%\end{question}
%
%\begin{solution}
%	class InterestRateSwaption:
%	def __init__(self, exercise_date, irs):
%	self.exercise_date = exercise_date
%	self.irs = irs
%	def npv_bs(self, discount_curve, libor_curve, sigma):
%	A = self.irs.annuity(discount_curve)
%	S = self.irs.swap_rate(discount_curve, libor_curve)
%	T = (self.exercise_date - discount_curve.today).days / 365
%	d1 = (math.log(S/self.irs.fixed_rate) + 0.5 * sigma**2 * T)
%	/ (sigma * T**0.5)
%	d2 = d1 - (sigma * T**0.5)
%	npv = self.irs.notional * A * (S * scipy.stats.norm.cdf(d1) -
%	self.irs.fixed_rate * scipy.stats.norm.cdf(d2))
%	return npv
%	
%	def npv_mc(self, discount_curve, libor_curve, sigma, n_scenarios=10000):
%	A = self.irs.annuity(discount_curve)
%	S = self.irs.swap_rate(discount_curve, libor_curve)
%	T = (self.exercise_date - discount_curve.today).days / 365
%	discounted_payoffs = []
%	for i_scenario in range(n_scenarios):
%	S_simulated = S * math.exp(-0.5 * sigma * sigma * T +
%	sigma * math.sqrt(T) * numpy.random.normal())
%	
%	swap_npv = self.irs.notional * (S_simulated - self.irs.fixed_rate) * A
%	discounted_payoffs.append(max(0, swap_npv))
%	npv_mc = numpy.mean(discounted_payoffs)
%	npv_error = 3 * numpy.std(discounted_payoffs) / math.sqrt(n_scenarios)
%	return npv_mc, npv_error
%\end{ipython}

\begin{question}
Suppose that the LIBOR Forward rates and the discount curve are those defined in
\href{https://github.com/matteosan1/finance_course/raw/develop/libro/input_files/libor_curve.xlsx}{libor\_curve.xlsx} and \href{https://github.com/matteosan1/finance_course/raw/develop/libro/input_files/discount_curve.xlsx}{discount\_curve.xlsx} respectively.
Determine the value of an option to pay a fixed rate of 4\% and receives LIBOR on a 5 year swap starting in 1 year. Assume the notional is 100 EUR, the exercise date is on October, 30th 2020 and the swap rate volatility is 15\%.

\noindent\textbf{Hint:} add \texttt{InterestRateSwap} class to \texttt{finmarkets.py}.
\end{question}

\cprotEnv\begin{solution}
\begin{ipython}
from finmarkets import InterestRateSwap
from datetime import date
from dateutil.relativedelta import relativedelta
from curve_data import discount_curve, libor_curve
from scipy.stats import norm
import math

pricing_date = date.today()
start_date = date.today() + relativedelta(years=1)
exercise_date = date(2020, 10, 30)

irs = InterestRateSwap(start_date, 100, 0.04, 12, 5)
sigma = 0.15
A = irs.annuity(discount_curve, pricing_date)
S = irs.swap_rate(discount_curve, libor_curve, pricing_date)
T = (exercise_date - pricing_date).days / 365
d1 = (math.log(S/irs.fixed_rate) + 0.5 * sigma**2 * T) / (sigma * T**0.5)
d2 = (math.log(S/irs.fixed_rate) - 0.5 * sigma**2 * T) / (sigma * T**0.5)
npv = irs.notional * A * (S * norm.cdf(d1) - irs.fixed_rate * norm.cdf(d2))

print(f"Swaption NPV: {npv:.3f} EUR")
\end{ipython}
\begin{ioutput}
Swaption NPV: 13.587 EUR
\end{ioutput}
\end{solution}

\chapter{Interpolation, Discount Factors and Forward Rates}
\label{interpolation}

In this Chapter we will begin to see some \texttt{python} applications for finance, in particular we are going to deal with discount curves and forward rates. In doing so we will review a widely used mathematical tool: \emph{interpolation}.

\section{Discounting}
\label{discount-factors}

The core principle of \emph{time value of money} states that a dollar today is worth more than a dollar tomorrow. This is because money has the potential to earn interest over time.
The concept acknowledges that the ability to invest and grow your money today holds more value than receiving the same amount later.

\emph{Discounting} is the process of determining the present value of a future cash flow. It essentially takes the future amount and adjusts it downward to reflect its current worth based on the time value of money, i.e. a chosen discount rate.

With $D(T)$ (or $D(0,T)$) it is represented the discount factor corresponding to a spot rate $r$, and a future time $T$ (in year fraction). 

Consider a payment $N$ that is to be made $T$ years in the future, we calculate the present value ($PV$) as

\begin{equation}
PV = N\cdot D(T) = N \cdot \cfrac{1}{(1 + r)^t}~~~~~
(N \cdot e^{-rt}~~~\text{continuous compounding})
\label{eq:discount_factor}
\end{equation}

The discount rate $r$ which is used in financial calculations is usually chosen to be equal to the cost of capital, i.e. the risk-free rate.

\subsection{Risk Neutral Pricing Formula}

Given a derivative providing a payoff $H(t)$ at time $t$ it can be demonstrated that its value (NPV, net present value) can be expressed as

\begin{equation}
NPV = \mathbb{E}[e^{-rt} H(t)]
\end{equation}

In words we can states that the value of the derivative is given by the \textbf{sum of the discounted value of the future cash-flows}.
It is then apparent how important is to know the proper discount factors for pricing purposes.

Discount factors are usually presented as a \emph{discount curve} which represents the factors for various dates in the future. 
Since discount curves are made of a discrete set of discount factors derived only at some dates, we may need to find the factor at some different times: this is a typical application of interpolation.

\subsection{Linear Interpolation}
\label{linear-interpolation}

Consider to have a discount curve made of three points as in Fig.~\ref{fig:samples_for_interpolation}.

\begin{figure}[htbp]
  \centering
  \includegraphics[width=0.7\textwidth]{figures/interp_example1}
  \caption{Plot of the discount curve described in the text.}
  \label{fig:samples_for_interpolation}
\end{figure}

Given two discount factors $d_1$ and $d_2$ taken at two different times $t_1$ and $t_2$ a new factor between these two can be found as the point lying on the \textbf{line} connecting $d_1$ and $d_2$ at the desired time. 

\begin{attention}
\subsubsection{Derivation}
The equation of a line for two points $(t_1, d_1)$ and $(t_2, d_2)$ can be written as:

\begin{equation}
\frac{t - t_1}{t_2 - t_1} = \frac{d - d_1}{d_2 - d_1}
\end{equation}

Setting $w = \cfrac{t - t_1}{t_2 - t_1}$ and solving for $d$ we find the desired solution:

\begin{equation}
(d_2 - d_1)\cdot w = d - d_1\quad\implies\quad d = (1 - w)\cdot d_1 + w \cdot d_2
\end{equation}

This formula can be interpreted as a weighted average where the weights are inversely related to the distances from the end points to the unknown point ($w_1 = (1 - w) = \cfrac{t_2 - t}{t_2 -t_1}, w_2 = w$), which means that the closer point has more "influence" than the farther point on the result.
\end{attention}

You don't need to implement the interpolation formula from scratch, instead it can be used the one provided in \texttt{scipy.interpolate.interp1d}. 

\pythoncode{code/discount_1.py}

\begin{ioutput}
0.988
\end{ioutput}

\emph{Always interpret critically your results to guess if they make sense or not and avoid mistakes}. In the previous example we certainly expected something between 0.994 and 0.982 (our range ends) furthermore since we are looking for the discount factor at a time which is halfway the considered interval, the result will be somehow in the middle of $d_1$ and $d_2$. This simple kind of reasoning should be applied every time you have a result to quickly judge it.

\begin{curiosity}
\subsubsection{Epic Failure}
The Mars Climate Orbiter \emph{was} a 638~kg (1,407~lb), 326.7~M\$ space probe launched by NASA on December 11, 1998 to study Martian climate, atmosphere, and surface changes. 

On September 15, 1999, the necessary corrections to speed and direction of the probe were computed in order to place the spacecraft at an optimal position for an orbital insertion maneuver that would bring it around Mars at the proper altitude. 
But one week later communication with the spacecraft was permanently lost as it went into Martian orbital insertion. 

A committee of experts was created to investigate the reasons of 
such failure and they found out that the spacecraft encountered Mars at a lower than foreseen altitude causing either its destruction by atmospheric friction or making it bouncing against the atmosphere re-entering heliocentric orbit after leaving Mars.

The primary cause of this discrepancy was found in one piece of software (supplied by Lockheed Martin) that produced results in "Imperial" units,  while a second system (supplied by NASA) expected those results to be in SI units. Specifically, the software calculated the total impulse produced by thrust in \emph{pound-force seconds}. The trajectory calculation software then used these results, expected to be in \emph{newton seconds}, thus incorrect by a factor of 4.45, to update the predicted position of the spacecraft.
	
NASA took the entire responsibility for having vaporized about 300~M\$ in the Martian atmosphere, mainly for failing to make the appropriate checks and tests that would have caught this unit discrepancy~\cite{bib:mars}.	
\end{curiosity}

\subsubsection{Extrapolation Risk}

Interpolation is primarily designed to estimate values \textbf{within} the range of known data points. Attempting to extrapolate beyond the range of the data can lead to inaccurate results and is generally discouraged. Extrapolation assumes that the relationship between data points remains constant outside the known range, which may not be valid in general.

\subsection{Log-linear Interpolation}
\label{log-linear-interpolation}

When there is an exponential relationship between the two variables (e.g. $t$ and $d$) we can fall back to the linear case by a simple variable transformation. 

Consider the discount factor in case of continuous compounding: $D=e^{-rt}$. Applying the logarithm to both sides of the equation gives:

\begin{equation}
f = \log(D) = \log(\exp(-rt)) = -rt
\end{equation}
so there is linear relation between the new variable $f$ and $t$. At this point we can use the results of the previous Section to interpolate for values of $f$, just remember to exponentiate the final result to get the correct $D$.

\subsection{Limitations of Interpolation}
Interpolation is just an approximation and works well when we are trying to interpolate between two points that are close enough to believe that the function is almost linear in that interval.

It can be easily demonstrated that the linear approximation between two points of a given function $f(x)$ gets worse with the second derivative of the function that is approximated ($f''(x)$). This is intuitively correct: the "curvier" the function is, the worse the approximation made with simple linear interpolation becomes, see Fig.~\ref{fig:sine_interp} where we try to interpolate a sine function.

\begin{figure}
  \centering
  \includegraphics[width=0.7\textwidth]{figures/wrong_interp.png}
  \caption{Trying to approximate a sine function with a line is clearly not going to work unless the interpolation interval is very small.}
  \label{fig:sine_interp}
\end{figure}

To improve the approximation accuracy with complicated curves e.g. in the evaluation of natural logarithm or trigonometric functions), an higher order polynomial can be used ($p(x)=a_0 + a_1\cdot x + a_2\cdot x^2+\cdots$). It has to be clear however that going to higher degrees does not always help~\cite{bib:runge}.
Some interpolation methods, indeed, can result in \emph{overfitting}. This means the interpolating function fits the data points very closely but may not generalize well to unseen data, i.e. it can lead to excessive fluctuations between data points.

\subsection{Discount Curve}

\begin{finmarkets}
Since discount factors are an essential part for every financial calculation and we will keep using them everywhere, a \texttt{python} class which manages discount factors and curves is developed using an object oriented approach.

This class, that we name \texttt{DiscountCurve}, have as attributes
\begin{itemize}
	\tightlist
    \item an observation date which corresponds to $t=0$;
	\item a list of pillars dates specifying the value dates of the given discount factors, $t_0,...,t_{n-1}$;
	\item a list of given discount factors, $D(t_0),...,D(t_{n-1})$.
\end{itemize}

and a method to interpolate discount factors at a generic date. The input argument to this function will be the value date at which we want to interpolate. Since the discount factor can be expressed as an exponential the log-linear interpolation can be used:

\begin{equation}
	\begin{gathered}
		d(t_i)=\mathrm{ln}(D(t_i))\\
		d(t) = (1-w)d(t_i) + wd(t_{i+1});\quad w=\frac{t-t_i}{t_{i+1}-t_i}\\
		D(t) = \mathrm{exp}(d(t))
	\end{gathered}
\end{equation}
where $i$ is such that $t_i \le t \le t_{i+1}$

When dealing with discount factors we need to be careful though. \texttt{scipy.interpolate.interp1d} only accepts list of numbers as argument, i.e. it doesn't know how to interpolate them. So we need to convert the dates before going through the interpolation function. This transformation is implemented directly in the class constructor snd replace each date with the number of days from a reference using the method \texttt{toordinal()} available for each \texttt{datetime} object.

To protect against extrapolation problems an exception is raised if the provided date is outside the range of known discount factor dates.
\end{finmarkets}

\pythoncode{code/discount_2.py}

Using the discount curve data in \href{https://github.com/matteosan1/finance_course/raw/master/input_files/discount_factors_2022-10-05.xlsx}{\texttt{discount\_factors.xlsx}} instantiate the corresponding object and compute a discount factor.
The following example is going to use also the \texttt{TimeInterval} utility which converts strings like "10d" or "5y" into actual time deltas to be used with \texttt{python} dates.

\begin{finmarkets}
The \texttt{finmarkets} module is currently in the test repository of PyPi, the official collection of \texttt{python} modules. 
To install it in your working area just type 

\begin{ioutput}
pip install --index-url https://test.pypi.org/simple finmarkets
\end{ioutput}

(if you want to install it on Colab just prepend an exclamation mark to the previous command). The tool \texttt{pip} will take care of picking up the latest version of the package and of all the required dependencies. You can explore the module with \texttt{help} command and in your program just import what you need with something like

\begin{ipythonnon}
from finmarkets import DiscountCurve
\end{ipythonnon}
\end{finmarkets}

\pythoncode{code/discount_3.py}

\begin{ioutput}
discount factor at 2023-04-21: 0.9902
\end{ioutput}

A very useful way to check the correctness of a result is by plotting it. So by inspecting Fig.~\ref{fig:linear_discount_curve} we can see that the computed discount factor is compatible with the original discount curve.

\begin{figure}[htb]
	\centering
	\includegraphics[width=0.7\textwidth]{figures/linear_discount_curve}
	\caption{Plot of the discount curve and of the computed discount factor.}
	\label{fig:linear_discount_curve}
\end{figure}

\section{Forward Rates}
\label{calculating-forward-rates}
A forward rate is an interest rate applicable to a financial transaction that will take place in the future. It can be considered as the market's expectation for future prices and can serve as an indicator of how it believes will perform. Contrary the \emph{spot rate} is used by buyers and sellers looking to make an immediate purchase or sale.

Forward rates are calculated from the spot rate by exploiting the no arbitrage condition which states that investing at rate $r_1$ for the period $(0, T_1)$ and then \emph{re-investing} at rate $r_{1,2}$ for the time period $(T_1, T_2)$ should be equivalent to invest at rate $r_2$ for the full time period $(0, T_2)$. Essentially two investors shouldn't be able to earn money from arbitraging between different interest periods. That said:

\begin{equation}
(1+r_1 T_1)[1+r_{1,2}(T_2 - T_1)] = 1 + r_2 T_2
\label{eq:no_arbitrage_r}
\end{equation}
Solving for $r_{1,2}$ leads to

\begin{equation}
F(T_1, T_2) = r_{1,2} = \frac{1}{T_2 - T_1}\Big(\frac{1+r_2 T_2}{1+r_1 T_1} - 1 \Big)
\label{eq:forward_rate_simple}
\end{equation}
The same expression in terms of discount factors becomes
\begin{equation}
F(T_1, T_2) = \frac{1}{T_2 - T_1}\Big(\frac{D(0, T_1)}{D(0, T_2)} - 1 \Big)
\end{equation}
%Considering continuously compounded rates instead Eq.~\ref{eq:no_arbitrage_r} can be written as
%\begin{equation}
%\begin{gathered}
%e^{r_{2}T_{2}}=e^{r_{1}T_{1}}\cdot \ e^{r_{1,2} (T_{2}-T_{1})}\quad(\textrm{then equating the exponents})\\
%%\textrm{log}\left(e^{r_{2}T_{2}}\right)=\textrm{log}\left(e^{r_{1}T_{1}+ r_{1,2} (T_{2}-T_{1})}\right) \\
%r_{2}T_{2}=r_{1}T_{1}+r_{1,2}(T_{2}-T_{1})\implies r_{1,2} = \cfrac{r_2 T_2 - r_1 T_1}{T_2 - T_1} 
%\end{gathered}
%\end{equation}
%and the corresponding expression for the forward rate is
%\begin{equation}
%F(T_1, T_2) = r_{1,2} = \frac {1}{T_{2}-T_{1}}(\ln D(0,T_{1})-\ln D(0,T_{2}))
%\quad(\textrm{since now } D(0, T_i)=e^{-r_i T_i})
%\label{eq:forward_rate_continous}
%\end{equation}

\begin{finmarkets}
The \texttt{TermStructure} class of the \texttt{finmarkets} module computes forward rates at any arbitrary date, given a set of spot rates. This class is quite similar to \texttt{DiscountCurve}, but its logic has been split into two methods: one to interpolate spot rates, another to actually compute the forward rate.

In order to successfully interpolate the rates, dates need to be converted to numbers by evaluating the number of days since a reference date. Also the interpolation date is checked to avoid extrapolations, if it is outside the known range am exception will be raisedw and the calculation stopped.
\end{finmarkets}

\pythoncode{code/discount_4.py}

As an example let's compute

\pythoncode{code/discount_5.py}

\begin{ioutput}
f_12 = 0.100
f_23 = 0.110
f_13 = 0.105
\end{ioutput}

\section{Multi-curve Framework}
\label{sec:financial-crisis}

% aggiungere un pezzettino dell'articolo di Mercurio

Prior to the 2008 financial crisis, inter-bank deposits posed little credit/liquidity issues, inter-bank lending rates (e.g. LIBOR, EURIBOR) were essentially a good proxy for risk free rates. Basis swap spreads were negligible and thereby neglected. 

Looking at the historical series of the EURIBOR (6M) rate versus the EONIA Overnight Indexed Swap (OIS-6M) rate over the time interval 2006-2011 in Fig.~\ref{fig:credit_crunch} it becomes apparent how before August 2007 the two rates display strictly overlapping trends differing of no more than 6 bps.

\begin{figure}[htb]
	\centering
	\includegraphics[width=0.9\linewidth]{figures/credit_crunch.png}
	\caption{Historical series of EURIBOR 6M rate versus EONIA OIS 6M rate. The corresponding spread 
		is shown on the right axis (Jan. 06 - Dec. 10 window, source: Bloomberg).}
	\label{fig:credit_crunch}
\end{figure}

A single yield curve constructed out of selected deposit, FRA and swap rates, served both the cash flow projection and discounting purposes.

During the 2008 financial crisis, the failure of some banks however proved that inter-bank lending rates were not risk-free. Meanwhile there was also significant counter-party credit risk arising from derivative transactions that were not subjected to collateral. Basis swap spreads greatly widened, and persist to this day. 

Still looking at Fig.~\ref{fig:credit_crunch} it is clear how in August 2007 a sudden increase of the EURIBOR rate and a simultaneous decrease of the OIS rate led to the explosion of the corresponding basis spread, touching the peak of 222 bps in October 2008, when Lehman Brothers filed for bankruptcy. Successively the basis has sensibly reduced and stabilized between 40 and 60 bps (notice that the pre-crisis level has never been recovered). The same effect is observed for other similar couples of series, e.g. EURIBOR 3M vs OIS 3M.

The existence of such significant basis swap spread reflects the fact that after the crisis interest rate market has been segmented into subareas corresponding to instruments with different underlying rate tenors, characterized by different rate dynamics. 

Traditional single curve based pricing approach ignores these differences. It mixes different underlying rate tenors and incorporates different rate dynamics, eventually leading to inconsistency.
%After the crisis, the market practice has thus evolved to take into account the new market information (e.g. the basis swap spreads, collateralization, etc.), that translate into the additional requirement of homogeneity and funding. The homogeneity requirement means that interest rate derivatives with a given underlying rate tenor must be priced and hedged using vanilla interest rate market instruments with the same underlying. The funding requirement means that the discount rate of any cash flow generated by the derivative must be consistent, by no-arbitrage, with the funding rate associated with that cash flow. 

Driven by the crisis, many derivative contracts have been updated to include permissible credit mitigants for a transaction, such as netting and collateralization in cash. Since standard agreements stipulate daily margination on collateral and the cash collateral earns a return at overnight rate, overnight rate becomes a natural choice for the risk-free discount rate or the funding rate. This is referred to as \emph{OIS discounting}.

Due to the large spread between risk free rate and inter-bank lending rate during and after 2008 financial crisis it is not possible anymore to use a single curve for discounting and derivative valuation. The traditional single curve used for both cash flow projection and discounting turned out to be obsolete. The markets have since nearly switched to \emph{multi-curve framework}. 

For example, if we want to calculate the net present value (NPV) of a forward 6-month EURIBOR coupon, we need to simultaneously use two different discount curves: 

\begin{itemize}
\tightlist
\item the 6-month EURIBOR curve for determining the forward rate;
\item the \euro STR curve for discounting the expected cash flow.
\end{itemize}

%The reason of the abrupt divergence between the Euribor and OIS rates can be explained by considering both the monetary policy decisions adopted by international authorities in response to the financial turmoil, and the impact of the credit crunch on both credit and liquidity risk perception of the market, coupled with the different financial meaning and dynamics of these rates.

%\begin{itemize}
%\tightlist
%\item
%  The Euribor rate is the reference rate for over-the-counter (OTC)
%  transactions in the Euro area. It is defined as the rate at which
%  Euro inter-bank deposits are being offered within the EMU zone by one
%  prime bank to another at 11:00 a.m. Brussels time. The rate fixings
%  for a strip of 15 maturities (from one day to one year) are
%  constructed as the average of the rates submitted (excluding the
%  highest and lowest 15\% tails) by a panel of 42 banks, selected
%  among the EU banks with the highest volume of business in the Euro
%  zone money markets, plus some large international bank from non-EU
%  countries with important euro zone operations. \emph{Thus, Euribor
%  rates reflect the average cost of funding of banks in the inter bank
%  market at each given maturity. During the crisis the solvency and
%  solidity of the whole financial sector was brought into question and
%  the credit and liquidity risk and uremia associated to inter-bank
%  counter-parties sharply increased.} The Euribor rates immediately
%  reflected these dynamics and raise to their highest values over more
%  than 10 years. As seen in the plot above, the Euribor 6M rate suddenly
%  increased on August 2007 and reached 5.49\% on 10th October 2008.
%\item
%  The EONIA rate is the reference rate for overnight OTC transactions in
%  the Euro area. It is constructed as the average rate of the overnight
%  transactions (one day maturity deposits) executed during a given
%  business day by a panel of banks on the inter-bank money market,
%  weighted with the corresponding transaction volumes. \emph{The EONIA
%  Contribution Panel coincides with the Euribor Contribution Panel, thus
%  EONIA rate includes information on the short term (overnight)
%  liquidity expectations of banks in the Euro money market. It is also
%  used by the European Central Bank (ECB) as a method of effecting and
%  observing the transmission of its monetary policy actions. During the
%  crisis the central banks were mainly concerned about stabilizing the
%  level of liquidity in the market, thus they reduced the level of the
%  official rates.} Furthermore, the daily tenor of the EONIA rate makes
%  negligible the credit and liquidity risks reflected on it: for this
%  reason the OIS rates are considered the best proxies available in the
%  market for the risk-free rate.
%\end{itemize}

%Our financial library has to implement the following calculation
%
%\[\mathrm{NPV} = D_{\mathrm{EONIA}}(T_1) \cdot \frac{1}{T_2-T_1}\Big(\frac{D_{\mathrm{LIBOR}}(T_1)}{D_{\mathrm{LIBOR}}(T_2)} - 1 \Big)\]
%\noindent
%In order to do so we can extend the \texttt{DiscountCurve} class with a \texttt{forward\_rate} method
%
%\begin{ipython}
%class DiscountCurve:
%    ...
%    def forward_rate(self, d1, d2):
%        return (self.df(d1) / self.df(d2) - 1.0) * \
%            (365.0 / ((d2 - d1).days))
%\end{ipython}

%As an example let's define \euro STR and EURIBOR curves and compute the net present value of the forward 6-month EURIBOR coupon mentioned before.
%
%\begin{ipython}
%from finmarkets import DiscountCurve, ForwardRateCurve
%from numpy import exp
%from dateutil.relativedelta import relativedelta
%from datetime import date
%
%obs_date = date.today()
%t1 = obs_date + relativedelta(months=3)
%t2 = obs_date + relativedelta(months=9)
%pillar_dates_estr = [obs_date, 
%                     obs_date + relativedelta(months=12),
%                     obs_date + relativedelta(months=34)]
%estr_rates = [1.0, 0.97, 0.72]
%pillar_dates_euribor = [obs_date, 
%                        obs_date + relativedelta(months=5), 
%                        obs_date + relativedelta(months=12)]
%euribor = [0.005, 0.01, 0.015]
%
%estr_curve = DiscountCurve(obs_date, pillar_dates_estr, estr_rates) 
%euribor_curve = ForwardRateCurve(obs_date, pillar_dates_euribor, euribor) 
%
%C = estr_curve.df(t1) * euribor_curve.forward_rate(t1, t2)
%t1_frac, r1 = euribor_curve.interp_rate(t1)
%C_pre2008 = exp(-r1*t1_frac) * euribor_curve.forward_rate(t1, t2)
%
%print (f"C post 2008: {C:.5f} EUR")
%print (f"C pre 2008: {C_pre2008:.5f} EUR")
%\end{ipython}
%\begin{ioutput}
%C post 2008: 0.01513 EUR
%C pre 2008: 0.01522 EUR
%\end{ioutput}

%\subsection{Transitioning away from LIBOR~\cite{bib:str}}
%A working group on euro risk-free rates was established to identify and recommend risk free rates that could serve as a basis for an alternative to current benchmarks used in a variety of financial instruments and contracts in the euro area, such as the euro overnight index average (EONIA) and the euro inter-bank offered rate (EURIBOR). 
%
%The group recommended on September 2018 that the euro short-term rate (\euro STR) be used as the risk-free rate for the euro area and is now focused on supporting the market with transitioning.
%The ECB published the \euro STR for the first time on 2nd October 2019, reflecting trading activity on 1st October 2019.
%
%The working group recommends that market participants should gradually replace EONIA with the \euro STR as a reference rate for all products and contracts and make all necessary adjustments for using the \euro STR as their standard benchmark The working group recommends the \euro STR plus a fixed spread of 8.5
%basis points as the EONIA fallback rate for all products and purposes. The working group recommends that market participants should: consider, whenever feasible and appropriate, no longer entering into new contracts referencing EONIA, in particular new contracts maturing after 31 December 2021, as EONIA will cease to exist after that date.
%
%%The working group is also looking at identifying fallbacks for
%%EURIBOR based on the STR. Both backward and forward-looking
%%options are being considered. As part of its work on forward-looking
%%options, in March 2019 the working group recommended a
%%methodology based on (tradable) overnight index swap (OIS) quotes
%%for calculating a STR-based forward-looking term structure and later
%%invited benchmark administrators to express their interest in
%%producing such a term structure.

\section*{Exercises}
\cprotEnv\begin{question}
\label{ex:yield_discount}
Using the \texttt{DiscountCurve} class, implemented in this Chapter, build a curve with the given inputs (pillar dates and yields)

\begin{ipython}
today = date(2020, 10, 15)
dates = [date(2021, 1, 15), date(2021, 4, 15), date(2021, 7, 15),
         date(2021, 10, 15), date(2022, 10, 15), date(2023, 10, 15),
         date(2024, 10, 15), date(2025, 10, 15), date(2026, 10, 15),
         date(2027, 10, 15), date(2028, 10, 15), date(2029, 10, 15),
         date(2030, 10, 15), date(2031, 10, 15), date(2032, 10, 15),
         date(2033, 10, 15), date(2034, 10, 15), date(2035, 10, 15),
         date(2036, 10, 15), date(2037, 10, 15), date(2038, 10, 15),
         date(2039, 10, 15), date(2040, 10, 15), date(2041, 10, 15),
         date(2042, 10, 15), date(2043, 10, 15), date(2044, 10, 15),
         date(2045, 10, 15), date(2046, 10, 15), date(2047, 10, 15),
         date(2048, 10, 15), date(2049, 10, 15), date(2050, 10, 15)]
yields = [-0.652548, -0.687966, -0.718319, -0.744011, -0.807362,
          -0.822144, -0.803715, -0.763496, -0.709892, -0.649001,
          -0.585169, -0.521425, -0.459808, -0.401628, -0.347657,
          -0.298283, -0.253620, -0.213593, -0.178005, -0.146578,
          -0.118993, -0.094911, -0.073989, -0.055896, -0.040317,
          -0.026957, -0.015546, -0.005840,  0.002383,  0.009320,
           0.015145,  0.020013,  0.024059]
\end{ipython}
\noindent
By interpolation, find the 18m yield, then draw the yield curve. Finally instantiate a \texttt{DiscountCurve} object, draw the curve and discount factors at the following dates
\begin{ipython}
df_dates = [date(2025, 4, 15), date(2031, 4, 15)]
\end{ipython}
\end{question}

\cprotEnv\begin{solution}
To interpolate the given yields the \texttt{interp} function can be used. Before doing that it is necessary to convert the dates into numbers (i.e. the difference between the actual date and "today").
\begin{ipython}
from dateutil.relativedelta import relativedelta
import numpy as np

num_dates = [(d-today).days for d in dates]
target = (today+relativedelta(months=18)-today).days
print (np.interp(target, num_dates, yields))
\end{ipython}
\begin{ioutput}
-0.7755997178082192
\end{ioutput}

Figure~\ref{fig:ex_yield} shows the yield curve and the interpolated point of the 18m yield.

\begin{figure}[htpb]
\centering
\includegraphics[width=0.7\linewidth]{figures/ex_yield}
\caption{Yield curve resulting from the exercise inputs.}
\label{fig:ex_yield}
\end{figure}

To construct the discount curve first it is necessary to define the discount factors from the yields. Then you need to instantiate a \texttt{DiscountCurve} object with the mandatory inputs and call the \texttt{df} method to compute the discount factors at the required dates.
\begin{ipython}
from finmarkets import DiscountCurve

dfs = []
for i, d in enumerate(dates):
    dfs.append(np.exp(-yields[i]/100*((d-today).days/365)))
    
dc = DiscountCurve(today, dates, dfs)
print (dc.df(date(2025, 4, 15)))
print (dc.df(date(2031, 4, 15)))
\end{ipython}
\begin{ioutput}
1.035800870152254
1.0461387588227613
\end{ioutput}

Figure~\ref{fig:ex_discount} shows the discount curve and the interpolated discount factors.

\begin{figure}[htpb]
\centering
\includegraphics[width=0.7\linewidth]{figures/ex_discount}
\caption{Discount curve resulting from the exercise inputs.}
\label{fig:ex_discount}
\end{figure}
\end{solution}

%\cprotEnv\begin{question}
%Calculate the forward interest for the period from six months (180/360) from now to nine months (270/360) from now if the six month rate is 4.50\% p.a. and the nine month rate is 4.25\% p.a.
%\end{question}
%
%\cprotEnv\begin{solution}
%FV
%FV
%6
%9
%1 +0.045* 180/360 1025
%1 +0.0425 *270-360 10525
%
%360/90 *1.03188/1.02250 = 0.0367 = 3.67
%\end{solution}

\begin{question}
Given the same inputs of Exercise~\ref{ex:yield_discount} and using the \texttt{ForwardRateCurve} class compute the 10y forward rate in 1y from today. 
\end{question}

\cprotEnv\begin{solution}
\begin{ipython}
from finmarkets import ForwardRateCurve

fc = ForwardRateCurve(today, dates, yields)
print ("{:.4f}%".format(fc.forward_rate(date(2021, 10, 15), date(2031, 10, 15))/100))
\end{ipython}
\begin{ioutput}
-0.0037%
\end{ioutput}
\end{solution}

<<<<<<< HEAD
\cprotEnv\begin{question}
Mario wants to lease a car for 3 years. The car dealer has given him two payment options to choose from:

\begin{table}[htbp]
\centering
\begin{tabular}{l|c|c|c|c|c|c|c}
Year & 0 & 0.5 & 1 & 1.5 & 2 & 2.5 & 3 \\
\hline
Option 1 &	3000.00	& 500.00 & 500.00 &	500.00 & 500.00 & 500.00 & 500.00 \\
\hline
Option 2 &	5000.00	& 350.00 & 350.00 &	350.00 & 350.00 & 350.00 & 350.00 \\
\end{tabular}
\end{table}

Which payment option should he choose? (Assume an annual discount rate of 10\%)
\end{question}

\cprotEnv\begin{solution}
In order to decide which payment to choose from, he would need to calculate the PV of the payment stream, and look at which costs him the least. 

\begin{gather*}
DF_n = 1/(1 + r)^n \\
PV_n = [\textrm{payment in year n}] / DF_n \\
Total PV = \sum PV_n
\end{gather*}

\begin{ipython}
def discount_factor(year):
    return 1/(1 + 0.1)**year

opt1 = {0:3000, 0.5:500, 1:500, 1.5:500, 2:500, 2.5:500, 3:500}
opt2 = {0:5000, 0.5:350, 1:350, 1.5:350, 2:350, 2.5:350, 3:350}

npv1 = sum([discount_factor(k)*v for k,v in opt1.items()])
npv2 = sum([discount_factor(k)*v for k,v in opt2.items()])

print ("Option1: {:.1f}".format(npv1))
print ("Option2: {:.1f}".format(npv2))
\end{ipython}
\begin{ioutput}
Option1: 5547.5
Option2: 6783.3
\end{ioutput}
Hence he should choose option 1 as it costs him less than option 2.
\end{solution}

\begin{question}
Tots and Toys, Inc. offers Maria an investment plan that requires 10 yearly payments of 1000 dollars, starting from today. The plan promises an annual return of 8\% on her investment. The money will be available to her at the end of 10 years (no withdrawls are allowed before that time.) 
What amount will she get at the end of 10 years?
\end{question}

\cprotEnv\begin{solution}
The question requires you to find the future value (FV) of the stream of payments. The rate given is 8\%.
In order to find the FV, you need to multiply each amount by its respective FV factor, and then sum the results.

\begin{ipython}
def fv_factor(year):
    return (1 + 0.08)**year

fv = 0
for year in range(11):
    fv += 1000*fv_factor(year)

print ("future value: {:.1f}".format(fv))
\end{ipython}
\begin{ioutput}
future value: 16645.5
\end{ioutput}
Amount at the end of 10 years is 16645.5 dollars.
\end{solution}

%\begin{question}
%Copy into the file \texttt{finmarkets.py} the function used to compute Black Scholes formula used in Ex.~\ref{ex:BS2}. This is another utility for our financial library. Then repeat Ex.~\ref{ex:BS2} now using the version of the Black and Scholes formula in the \texttt{finmarkets} module.
%\end{question}
%
%\begin{solution}
%\begin{tcolorbox}[size=fbox, boxrule=1pt, colback=cellbackground, colframe=cellborder]
%\begin{Verbatim}[commandchars=\\\{\}]
%\PY{k}{import} \PY{n}{finmarkets}
%        
%\PY{n}{s} \PY{o}{=} \PY{l+m+mi}{800}
%\PY{c+c1}{\PYZsh{} strikes expressed as \PYZpc{} of spot price}
%\PY{n}{moneyness} \PY{o}{=} \PY{p}{[} \PY{l+m+mf}{0.5}\PY{p}{,} \PY{l+m+mf}{0.75}\PY{p}{,} \PY{l+m+mf}{0.825}\PY{p}{,} \PYZbs{}
%             \PY{l+m+mf}{1.0}\PY{p}{,} \PY{l+m+mf}{1.125}\PY{p}{,} \PY{l+m+mf}{1.25}\PY{p}{,} \PY{l+m+mf}{1.5} \PY{p}{]}
%\PY{n}{vol} \PY{o}{=} \PY{l+m+mf}{0.3}
%\PY{n}{ttm} \PY{o}{=} \PY{l+m+mf}{0.75}
%\PY{n}{r} \PY{o}{=} \PY{l+m+mf}{0.005}
%
%\PY{n}{result} \PY{o}{=} \PY{p}{\PYZob{}}\PY{p}{\PYZcb{}}
%\PY{k}{for} \PY{n}{m} \PY{o+ow}{in} \PY{n}{moneyness}\PY{p}{:}
%    \PY{n}{result}\PY{p}{[}\PY{n}{s}\PY{o}{*}\PY{n}{m}\PY{p}{]} \PY{o}{=} \PY{n}{finmarkets.call}\PY{p}{(}\PY{n}{s}\PY{p}{,} \PY{n}{m}\PY{o}{*}\PY{n}{s}\PY{p}{,} \PY{n}{r}\PY{p}{,} \PY{n}{vol}\PY{p}{,} \PY{n}{ttm}\PY{p}{)}
%\PY{n}{result}
%
%\{400.0: 401.66074527896365,
%  600.0: 213.9883852521275,
%  660.0: 166.85957363897393,
%  800.0: 84.03697017660357,
%  900.0: 47.61880394696229,
%  1000.0: 25.632722952585738,
%  1200.0: 6.655275227771156\}
%\end{Verbatim}
%\end{tcolorbox}
%\end{solution}



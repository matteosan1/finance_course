\cprotEnv\begin{question}
Calculate the prices of calls with various strikes, using the following input data.

\begin{ipython}
s = 800 
# strikes expressed as percentage of spot price
moneyness = [0.5, 0.75, 0.825, 1.0, 1.125, 1.25, 1.5]
vol = 0.3
ttm = 0.75
r = 0.005
\end{ipython}
\noindent
The output should be a dictionary mapping strikes to call prices.
\end{question}

\cprotEnv\begin{solution}
\begin{ipython}
import numpy as np
from scipy.stats import norm

def d1(S, K, r, vol, ttm):
    num = np.log(S/K) + (r + 0.5*np.power(vol, 2)) * ttm
    den = vol * np.sqrt(ttm)
    return num/den

def d2(S, K, r, vol, ttm):
    return d1(S, K, r, vol, ttm) - vol*np.sqrt(ttm)

def call(S, K, r, vol, ttm):
    return (S * norm.cdf(d1(S, K, r, vol, ttm)) - \
           K*np.exp(-r*ttm)*norm.cdf(d2(S, K, r, vol, ttm)))

S = 800
# strikes expressed as % of spot price
moneyness = [0.5, 0.75, 0.825, 1.0, 1.125, 1.25, 1.5]
vol = 0.3
ttm = 0.75
r = 0.005

result = {}
for m in moneyness:
    result[S*m] = round(call(S, S*m, r, vol, ttm), 2)
print(result)
\end{ipython}
\begin{ioutput}
{400.0: 401.66, 
 600.0: 213.99, 
 660.0: 166.86, 
 800.0: 84.04, 
 900.0: 47.62, 
 1000.0: 25.63, 
 1200.0: 6.66}
\end{ioutput}
\end{solution}

\cprotEnv\begin{question}
Write two classes, \texttt{Circle} and \texttt{Rectangle} that given the radius and height, width respectively allow to compute area and perimeter of the two shapes. Test them with the following:

\begin{ipython}
a_circle = Circle(5)
print(f"My circle has an area of {a_circle.area()} m2")

a_rectangle = Rectangle(3, 6)
print (f"My rectangle has a perimeter of {a_rectangle.perimeter()} m and an area of {a_rectangle.area()} m2")
\end{ipython}
\end{question}

\cprotEnv\begin{solution}
\begin{ipython}
from math import pi

class Circle:
    def __init__(self, radius):
        self.radius = radius

    def area(self):
        return pi*self.radius**2

class Rectangle:
    def __init__(self, width, height):
        self.height = height
        self.width = width

    def area(self):
        return self.width*self.height

    def perimeter(self):
        return self.width*2 + self.height*2

circle = Circle(5)
print (f"My circle area is {circle.area():.1f} m**2")

rect = Rectangle(3, 6)
print (f"My rect area is {rect.area():.1f} m**2 and the perimeter is {rect.perimeter()} m")
\end{ipython}
\begin{ioutput}
My circle area is 78.5 m**2
My rect area is 18.0 m**2 and the perimeter is 18 m
\end{ioutput}
\end{solution}

\cprotEnv\begin{question}
Write a class \texttt{Menu} which is a component of a restaurant ordering system. Through this program the waiters should be able to: register client orders adding or deleting dishes and print the final bill.
The only attribute of the class is the following list of dishes with an \texttt{id} to identify them, their name and cost.

\begin{ipython}
dishes = {'1':("Spaghetti", 6.50), '2':("Pizza", 7.00), '3':("Water", 0.50),
                       '4':("Red Wine 0.5l", 5.00)}
\end{ipython}
\noindent
To test the class simulate an order and print the corresponding bill.
\end{question}

\cprotEnv\begin{solution}
\begin{ipython}
class Menu:
    def __init__(self):
        self.dishes = {'1':("Spaghetti", 6.50), '2':("Pizza", 7.00), '3':("Water", 0.50),
                       '4':("Red Wine 0.5l", 5.00)}
        self.current_choice = {}
        
    def clear_order(self):
        self.current_choice = {}
        
    def add_dish(self, dish_id, quantity = 1):
        if dish_id not in self.dishes.keys():
            print (f"Dish {dish_id} currently not available.")
        else:
            self.current_choice[dish_id] = 
                self.current_choice.setdefault(dish_id, 0) + quantity
            
    def remove_dish(self, dish_id, quantity = 1):
        if dish_id in self.current_choice.keys():
            self.current_choice[dish_id] = max(0, self.current_choice[dish_id] - quantity)
            
    def bill(self):
        cost = 0
        for id, qty in self.current_choice.items():
            cost += self.dishes[id][1] * qty
            print (f"{self.dishes[id][0]:15} {qty:2} {self.dishes[id][1]:>6.2f}")
        print ("{'Tot:':15}    {cost:>6.2f}")

m = Menu()
m.add_dish('1')
m.add_dish('2')
m.add_dish('3')
m.bill()
\end{ipython}
\begin{ioutput}
Spaghetti        1   6.50
Pizza            1   7.00
Water            1   0.50
Tot:                14.00
\end{ioutput}
\end{solution}

\begin{question}
Define a Point2D class that represent a point in a plane. Its \texttt{\_\_init\_\_} method should accept the point coordinates \texttt{x} and \texttt{y}. Write a method \texttt{distanceTo} that compute the distance of the point to another passed as input. Test the class by printing the distance of the point $P=(4, 5)$ to the origin $P=(0,0)$ and to $P=(3,4)$.

\noindent\textbf{Hint:} in the Cartesian plane the distance between two points is: $\sqrt{(x_1 - x_2)^2 + (y_1 - y_2)^2}$.
\end{question}

\cprotEnv\begin{solution}
\begin{ipython}
from math import sqrt

class Point2D:
    def __init__(self, x, y):
        self.x = x
        self.y = y

    def distanceTo(self, x, y):
        dist = sqrt((self.x-x)**2 + (self.y - y)**2)
        return dist

    def distanceTo_v2(self, p):
        dist = sqrt((self.x-p[0])**2 + (self.y - p[1])**2)
        return dist

    def distanceTo_v3(self, p):
        dist = sqrt((self.x-p.x)**2 + (self.y - p.y)**2)
        return dist

point = Point2D(4, 5)
p0 = (0, 0)
point0 = Point2D(0, 0)
print (f"distance to p0: {point.distanceTo(p0[0], p0[1]):.2f}")
print (f"distance_v2 to p0: {point.distanceTo_v2(p0):.2f}")
print (f"distance_v3 to p0: {point.distanceTo_v3(point0):.2f}")
\end{ipython}
\begin{ioutput}
distance to p0: 6.40
distance_v2 to p0: 6.40
distance_v3 to p0: 6.40
\end{ioutput}
\begin{ipython}
p1 = (3, 4)
point1 = Point2D(3, 4)
print (f"distance to p1: {point.distanceTo(p1[0], p1[1]):.2f}")
print (f"distance v2 to p1: {point.distanceTo_v2(p1):.2f}")
print (f"distance v3 to p1: {point.distanceTo_v3(point1):.2f}")
\end{ipython}
\begin{ioutput}
distance to p1: 1.41
distance v2 to p1: 1.41
distance v3 to p1: 1.41
\end{ioutput}
\end{solution}

\cprotEnv\begin{question}
Write a class \texttt{Student} which inherits from the class \texttt{Person} defined in Chapter Object Oriented Programming in Python. This new class should have two new attributes: \texttt{grade} which keeps the type of school and \texttt{votes} a dictionary which will record the student's votes and the corresponding course. Then add two methods, one to add votes and another to compute the average vote.Instantiate a "student" add some votes and show how good it has been.

\noindent\textbf{Hint:} this is the \texttt{Person} class already developed.

\begin{ipython}
class Person:
    def __init__(self, name, birthday):
        self.name = name
        self.birthday = birthday
        self.employment = None

    def age(self, d=date.today()):
        age = (d - self.birthday).days/365
        print (f"{self.name} is {age:.0f} years old")

    def mainOccupation(self, occupation):
        self.employment = occupation
        print ("{self.name}'s main occupation is: {self.employment}")
\end{ipython}
\end{question}

\cprotEnv\begin{solution}
\begin{ipython}
class Student(Person):
    def __init__(self, name, birthday, school):
        Person.__init__(self, name, birthday)
        self.grade = school
        self.votes = {}

    def addVote(self, subject, vote):
        self.votes[subject] = vote

    def average(self):
        print ("List of votes")
        print ("-------------")
        for k, v in self.votes.items():
            print (f"{k}: {v}")
        avg = sum(self.votes.values())/len(self.votes)
        print ("-------------")
        print (f"Avg: {avg:.1f}")

student = Student("Mario", date(1980, 5, 6), "Liceo Scientifico G. Galilei")
student.addVote("Calculus", 8)
student.addVote("Literature", 5.5)
student.addVote("Latin", 6.5)
student.average()
\end{ipython}
\begin{ioutput}
List of votes
-------------
Calculus: 8
Literature: 5.5
Latin: 6.5
-------------
Avg: 6.7
\end{ioutput}
\end{solution}

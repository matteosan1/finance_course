\documentclass[]{article}

\title{Bootstrapping}
\author{}
\usepackage{exsheets}
\SetupExSheets{counter-format=ch.qu}

\usepackage{epigraph}
\setlength{\epigraphwidth}{0.6\textwidth}
\setlength{\epigraphrule}{0pt}
\newcommand{\myopeningquote}{\smash{\raisebox{-0.25\height}{\llap{\scalebox{3}
				{\textcolor{silver}{``}}\,}}}}
\newcommand{\myclosingquote}{\raisebox{-0.86\height}{\scalebox{3}{\textcolor{silver}{”}}}}

\usepackage{sectsty}
\usepackage{titlesec}
\chapterfont{\color{rahmen}}
\subsectionfont{\color{rahmen}}
\subsubsectionfont{\color{rahmen}}
\titleformat{\section}
{\normalfont\Large\bfseries\color{rahmen}}{\textcolor{rahmen}\thesection}{1em}{}[\textcolor{rahmen}{\titlerule[0.1pt]}]

\patchcmd{\thebibliography}{\chapter*}{\section*}{}{}

\usepackage{afterpage}
\usepackage[inline]{enumitem}
\usepackage{cancel}
\usepackage{tabularx}
\usepackage{booktabs}
\usepackage{makecell}
\setcellgapes{4pt}

\usepackage{geometry}
\geometry{verbose,tmargin=2.5cm,bmargin=2.5cm,lmargin=2.5cm,rmargin=2.5cm,headheight=14pt}

\usepackage[most]{tcolorbox}
\newtcolorbox{attention}{breakable,enhanced,arc=2mm,colback=cream,colframe=goldenrod,leftrule=12mm,
	overlay={\node[anchor=north west,outer sep=2pt] at (frame.north west) {\includegraphics[width=8mm]{figures/alert.png}}; }}
\newtcolorbox{curiosity}{breakable,enhanced,arc=2mm,colback=lime,colframe=green,leftrule=12mm,
	overlay={\node[anchor=north west,outer sep=2pt] at (frame.north west) {\includegraphics[width=8mm]{figures/curiosity.png}}; }}
\newtcolorbox{finmarkets}{breakable,enhanced,arc=2mm,colback=sky,colframe=rahmen,leftrule=12mm,
	overlay={\node[anchor=north west,outer sep=2pt] at (frame.north west) {\includegraphics[width=8mm]{figures/python_logo.png}}; }}

%\usepackage{amssymb}
\usepackage{amsfonts} % for expectation and other chars
\usepackage{bbm} % for indicator function
\usepackage{eurosym} % defines \euro

\usepackage{xcolor}
\definecolor{mandellightgreen}{rgb}{0.77, 0.82, 0.73}
\definecolor{mandelpantone}{rgb}{0.2, 0.11, 0.27}
\definecolor{ipython-frame}{RGB}{207, 207, 207}
\definecolor{ipython-bg}{RGB}{247, 247, 247}
\definecolor{ipython-red}{RGB}{186, 33, 33}
\definecolor{ipython-green}{RGB}{0, 128, 0}
\definecolor{ipython-cyan}{RGB}{64, 128, 128}
\definecolor{ipython-purple}{RGB}{170, 34, 255}
\definecolor{cream}{RGB}{255,229,153}
\definecolor{goldenrod}{RGB}{218,165,32}
\definecolor{sky}{RGB}{150, 195, 255}
\definecolor{rahmen}{RGB}{0,73,114}
\definecolor{ansi-red}{HTML}{E75C58}
\definecolor{ansi-green}{HTML}{00A250}
\definecolor{silver}{HTML}{C0C0C0}

\newcommand\Pdef{P_{\text{def}}}
\newcommand\Psur{P_{\text{sur}}}
\newcommand\corr{\text{Corr}}
\newcommand\cov{\text{Cov}}
\newcommand\var{\text{Var}}
\newcommand\std{\text{Std}}

\usepackage{hyperref}
\hypersetup{
	breaklinks=true,  % so long urls are correctly broken across lines
	colorlinks=true,
	urlcolor=rahmen,
	linkcolor=rahmen,
	citecolor=rahmen,
}
\hypersetup{linkcolor=rahmen}

\providecommand{\tightlist}{\setlength{\itemsep}{0pt}\setlength{\parskip}{0pt}}

\usepackage{subfig} %for subfloat

\usepackage{cprotect}
\lstdefinelanguage{iPython} {
	morekeywords={access,and,del,except,exec,in,is,lambda,not,or,raise},
	morekeywords=[2]{for,print,abs,all,any,basestring,bin,bool,bytearray,callable,chr,classmethod,cmp,compile,complex,delattr,dict,dir,divmod,enumerate,eval,execfile,file,filter,float,format,frozenset,getattr,globals,hasattr,hash,help,hex,id,input,int,isinstance,issubclass,iter,len,list,locals,long,map,max,memoryview,min,next,object,oct,open,ord,pow,property,range,reduce,reload,repr,reversed,round,set,setattr,slice,sorted,staticmethod,str,sum,super,tuple,type,unichr,unicode,vars,xrange,zip,apply,buffer,coerce,intern,elif,else,if,continue,break,while,class,def,return,try,except,import,finally,try,except,from,global,pass, True, False},
	sensitive=true,
	morecomment=[l]\#,
	morestring=[b]',
	morestring=[b]",
	moredelim=**[is][\color{black}]{@@}{@@},
	identifierstyle=\color{black}\footnotesize\ttfamily,
    commentstyle=\color{ipython-cyan}\footnotesize\itshape\ttfamily,
	stringstyle=\color{ipython-red}\footnotesize\ttfamily,
	keepspaces=true,
	showspaces=false,
    showstringspaces=false,
	rulecolor=\color{ipython-frame},
	frame=single,
	frameround={t}{t}{t}{t},
	framexleftmargin=6mm,
	numbers=left,
	numberstyle=\color{black},
	basicstyle=\footnotesize\ttfamily,
	keywordstyle=[2]\color{ipython-green}\bfseries\footnotesize\ttfamily, 
	keywordstyle=\color{ipython-purple}\bfseries\footnotesize\ttfamily,
	mathescape=true,
	escapeinside={*@}{@*}
}

\lstdefinelanguage{iOutput} {
	sensitive=true,
	identifierstyle=\color{black}\small\ttfamily,
	stringstyle=\color{ipython-red}\small\ttfamily,
	keepspaces=true,
	showspaces=false,
	showstringspaces=false,
	rulecolor=\color{ipython-frame},
	%frame=single,
	%frameround={t}{t}{t}{t},
	%backgroundcolor=\color{ipython-bg},
	basicstyle=\small\ttfamily,
	mathescape=true,
	escapeinside={*@}{@*}
}

\lstdefinestyle{mycpp}{language=C++,
	basicstyle=\small\ttfamily,
	showstringspaces=false,
	keywordstyle=\small\color{blue}\ttfamily,
	stringstyle=\small\color{red}\ttfamily,
	commentstyle=\small\color{green}\ttfamily,
	rulecolor=\color{ipython-frame},
	frame=single,
	frameround={t}{t}{t}{t},
	backgroundcolor=\color{ipython-bg},
	morecomment=[l][\color{magenta}]{\#}
}

\lstdefinestyle{attstyle}{
	backgroundcolor=\color{cream},
	rulecolor=\color{goldenrod}
}

\lstnewenvironment{attpython}[1][]{
	\lstset{language=iPython,style=attstyle,mathescape=true}
}{}

\lstnewenvironment{ipython}[1][]{\lstset{language=iPython,backgroundcolor=\color{ipython-bg}}}{}

\lstnewenvironment{ipythonnon}[1][]{\lstset{language=iPython,backgroundcolor=\color{ipython-bg},numbers=none,framexleftmargin=0mm}}{}

\lstnewenvironment{ipythonbox}[1][]{\lstset{language=iPython,numbers=none,framexleftmargin=0mm}}{}

\lstnewenvironment{ioutput}[1][]{\lstset{language=iOutput}}{}

\newcommand{\pythoncode}[1]{
    \lstinputlisting[language=iPython,backgroundcolor=\color{ipython-bg}]{#1}
}

\newcommand{\pythoncodenon}[1]{
    \lstinputlisting[language=iPython,numbers=none,backgroundcolor=\color{ipython-bg},framexleftmargin=0mm]{#1}
}

\newcommand{\pythoncodebox}[1]{
	\lstinputlisting[language=iPython,numbers=none,framexleftmargin=0mm]{#1}
}

\usepackage{fancyhdr}
\pagestyle{fancy}
%\renewcommand{\chaptermark}[1]{\markboth{\thechapter.\ #1}{}}
\renewcommand{\sectionmark}[1]{\markboth{\thesection.\ #1}{}}
\fancypagestyle{main}{
	\fancyhf{}
    \fancyhead[LO,RE]{\leftmark}
	\fancyhead[LE,RO]{\thepage}
	%\fancyfoot[C]{\thepage}
	\renewcommand{\headrulewidth}{0.4pt}
}


\begin{document}

\maketitle

\section{Bootstrapping the ESTR Curve}
The \emph{bootstrap algorithm} is a technique that allows to derive interest rates (or discount factors) from market quotes of traded products, e.g. bonds, FRA, swaps.
In this exercise we would like to bootstrap the ESTR discount curve from a set of Overnight Index Swaps quotes. The algorithm relies on the fact that the market quotes (prices) of the swaps are the \emph{fair} quotes, i.e. those that makes the NPV of the contract 0. Note that the market quotes of an OIS are the \emph{rate} of the fixed leg ($K$). 

In the case of the Overnight Index Swaps the NPV is a function of the fixed rate (market quote) and the discount curve $NPV(K, dc)=0$, the bootstrap provides a way to "invert" that function to get the discount factors of the discount curve given $K$: $dc = NPV^{-1}(K)$.
In order to simplify the entire procedure, instead of inverting each NPV function for each OIS, we are going to do that simultaneously by looking for the minimum of the function 
\begin{equation}
\sum_{i=1}^{N}\mathrm{NPV}^\mathrm{OIS}_i( \mathcal{C}(\mathbf{x}))^2
\label{eq:obj}
\end{equation}
with respect to the unknown discount factors ($x$).

The \texttt{python} implementation follows these steps:
\begin{enumerate}
	\item define the set of Overnight Index Swaps according to the available market quotes;
\begin{ipython}
from finmarkets import generate_dates, OvernightIndexSwap

def make_swaps(data, start_date):
    pillar_dates = []
    swaps = []
    for i in range(len(data)):
        swap = OvernightIndexSwap(1e5, start_date,
                                  "{}M".format(data.loc[i, 'months']),
                                  data.loc[i, 'quotes']*0.01)
        swaps.append(swap)
        pillar_dates.append(swap.payment_dates[-1])
    return swaps, pillar_dates
\end{ipython}
	
	\item define the objective function for the minimization problem from Eq.~\ref{eq:obj};
\begin{ipython}
from finmarkets import DiscountCurve

def of(dfs, obs_date, pillars, swaps):
    dc = DiscountCurve(obs_date, pillars, dfs)
    val = 0
    for s in swaps:
        val += s.npv(dc)**2
    return val
\end{ipython}
	\item set the initial guesses for the discount factors and their boundaries;
	\item run the minimization algorithm.
\end{enumerate}

\section{Testing}
The bootstrap algorithm can be tested from the market quotes of October 2021 in \href{"https://github.com/matteosan1/finance\_course/raw/develop/input\_files/ois\_2021_10_14.xlsx"}{ois\_2021\_10\_14.xlsx}.
\begin{ipython}
import pandas as pd
from datetime import date

dataframe = pd.read_excel("ois_2021_10_14.xlsx")
print (dataframe.head())
\end{ipython}
\begin{ioutput}
   months   quotes
0       1  0.70575
1       2  1.03170
2       3  1.23200
3       4  1.42975
4       5  1.59400
\end{ioutput}

Then we can create the OIS, set the parameters and finally to run the minimization.
\begin{ipython}
from scipy.optimize import minimize

obs_date = date.today()
swaps, pillar_dates = make_swaps(dataframe, obs_date)

dfs0 = [0.5 for _ in range(len(swaps))]
bounds = [(0.01, 10) for _ in range(len(swaps))]
	
res = minimize(of, dfs0, bounds=bounds, args=(obs_date, pillar_dates, swaps))
print res()
\end{ipython}
\begin{ioutput}
 message: CONVERGENCE: NORM_OF_PROJECTED_GRADIENT_<=_PGTOL
 success: True
  status: 0
     fun: 1.5567916515678768e-12
       x: [ 9.994e-01  9.983e-01 ...  4.428e-01  3.901e-01]
     nit: 7
     jac: [ 2.641e-07  4.092e-07 ...  1.840e-07  4.270e-07]
    nfev: 279
    njev: 9
hess_inv: <30x30 LbfgsInvHessProduct with dtype=float64>
\end{ioutput}

Finally we can create the discount curve implied by the market quote of our swaps and, if needed, determine the
corresponding yield curve using the relation between rate and discount factor
\begin{equation}
\mathrm{df} = e^{-rt} \quad\implies \mathrm{log(df)} = -rt\quad\implies r = -\mathrm{log(df)}/t
\end{equation}

\begin{ipython}
from dateutil.relativedelta import relativedelta
from finmarkets import DiscountCurve, ForwardRateCurve

dc = DiscountCurve(obs_date, pillars, res.x)

dates = [date.today() + relativedelta(months=i) for i in range(600)]
yields = [-np.log(dc.df(d))/((d-date.today()).days/365) for d in dates]

fc = ForwardRateCurve(obs_date, dates, yields)
\end{ipython}
\end{document}

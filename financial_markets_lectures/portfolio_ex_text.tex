\begin{question}
One of the approaches to finding the optimal point on the efficient frontier for a given investor is to maximize the investor's utility ($U$). This is a measure of how much benefit investors obtain from portfolio performance. Utility is a measure of relative satisfaction that an investor derives from different portfolios. Utility is a function of the portfolio expected return, the portfolio variance and a measure of risk aversion

\begin{equation*}
U = \mathbb{E}(R_{p}) - \frac{1}{2}A\sigma^2
\end{equation*}
where $U$ = utility, $\mathbb{E}(R)$ = portfolio expected return, $A$ = risk aversion coefficient and $\sigma^2$ = portfolio variance.
In determining the risk aversion ($A$), we measure the marginal reward an investor needs in order to take on more risk. A risk-averse investor will need a high margin reward for taking on more risk. The utility equation shows the following:
\begin{itemize}
\tightlist
\item $U$ can be positive or negative, i.e. it is unbounded;
\item high returns add to utility;
\item high variance reduces utility;
\item $U$ does not measure satisfaction but can be used to rank portfolios.
\end{itemize}

The risk aversion coefficient, $A$, ranges between 1 and 10 (1 for the aggressive investor, 4 for the moderate investor, and 10 for the risk-averse investor.

Given the historical series in \href{https://github.com/matteosan1/finance_course/raw/master/input_files/share_price.csv}{share\_prices.csv} find the best portfolio for aggressive, moderate and risk-averse investors.

\end{question}

\cprotEnv \begin{solution}
\begin{ipython}
import pandas as pd
df = pd.read_csv("share_prices.csv", index_col='Date')
daily_returns = df.pct_change()
returns = daily_returns.mean()*252
covariance = daily_returns.cov()*252

print (returns)
print (covariance)
\end{ipython}
\begin{ioutput}
CHD     0.062666
COST    0.247429
CTRE    0.165151
ENTR    0.002445
HI      0.223906
LDOS    0.075464
TDS    -0.003953
TMUS    0.283815
UEC     0.999521
WM      0.163298
dtype: float64

           CHD     COST       CTRE      ENTR        HI      LDOS       TDS
CHD   0.067815  0.036114  0.016454  0.022583  0.022301  0.029738  0.032921   
COST  0.036114  0.073004  0.031599  0.049279  0.034606  0.034481  0.034043   
CTRE  0.016454  0.031599  0.313706  0.074262  0.136719  0.068927  0.088424   
ENTR  0.022583  0.049279  0.074262  0.113892  0.062266  0.043155  0.040843   
HI    0.022301  0.034606  0.136719  0.062266  0.202751  0.068284  0.097420   
LDOS  0.029738  0.034481  0.068927  0.043155  0.068284  0.110635  0.059296   
TDS   0.032921  0.034043  0.088424  0.040843  0.097420  0.059296  0.185842   
TMUS  0.026069  0.042121  0.073199  0.055806  0.057519  0.046704  0.068516   
UEC   0.045100  0.099269  0.164647  0.147985  0.165341  0.096700  0.143037   
WM    0.034987  0.034676  0.049294  0.034555  0.057760  0.047310  0.048338   
...
\end{ioutput}
\begin{ipython}
import numpy as np
from scipy.optimize import minimize

def sum_weights(w): 
    return np.sum(w) - 1

def utility(w, returns, cov, risk_aversion):
    return -(returns.dot(w) - 0.5*w.T.dot(cov.dot(w))*risk_aversion)

num_assets = 10
constraints = [{'type': 'eq', 'fun': sum_weights},] 
bounds = tuple((0, 1) for _ in range(num_assets))
weights = [1./num_assets for _ in range(num_assets)]

for risk_aversion in (1, 4, 10):
  opts = minimize(utility, weights, args=(returns, covariance, risk_aversion),
                bounds=bounds, constraints=constraints)
  for i, c in enumerate(df.columns):
    print ("{} {:.1f}".format(c, opts.x[i]*100), end=" ")
  print()
  print ("Expected return: {:.3f}".format(opts.x.dot(returns)))
  print ("Variance: {:.3f}".format(opts.x.T.dot(covariance.dot(opts.x))))
\end{ipython}
\begin{ioutput}
# aggressive
CHD 0.0 COST 0.0 CTRE 0.0 ENTR 0.0 HI 0.0 LDOS 0.0 TDS 0.0 TMUS 12.8 UEC 87.2 WM 0.0 
Expected return: 0.908
Variance: 0.734

# moderate
CHD 0.0 COST 47.2 CTRE 0.0 ENTR 0.0 HI 0.0 LDOS 0.0 TDS 0.0 TMUS 33.1 UEC 18.3 WM 1.4 
Expected return: 0.396
Variance: 0.104

# risk averse
CHD 7.1 COST 40.2 CTRE 0.0 ENTR 0.0 HI 1.3 LDOS 0.0 TDS 0.0 TMUS 22.7 UEC 4.1 WM 24.6 
Expected return: 0.252
Variance: 0.055
\end{ioutput}
\end{solution}

\begin{question}
Your are portfolio manager and one of your clients is asking to design her investment in order to reduce at the minimum the risk. Her position involves these shares: 'TMUS', 'TDS', 'ENTR', 'UEC', 'WM', 'CTRE', 'LDOS', 'COST', 'CHD', 'HI'.
Find the weights corresponding to each asset.

\noindent\textbf{Hint:} the historical series of the corresponding companies are stored in \href{https://github.com/matteosan1/finance_course/raw/master/input_files/share_price.csv}{share\_prices.csv}.

\end{question}

\cprotEnv \begin{solution}
\begin{ipython}
import pandas as pd
df = pd.read_csv("share_prices.csv", index_col='Date')

daily_returns = df.pct_change()
returns = daily_returns.mean()*252
covariance = daily_returns.cov()*252
\end{ipython}
\begin{ipython}
import numpy as np
from scipy.optimize import minimize

def sum_weights(w):
    return np.sum(w) - 1

def min_risk(w, cov):
    return np.sqrt(w.T.dot(cov.dot(w)))

num_assets = 10
constraints = ({'type': 'eq', 'fun': sum_weights},)
bounds = tuple((0, 1) for asset in range(num_assets))
weights = [1./num_assets for _ in range(num_assets)]
opts = minimize(min_risk, weights, args=(covariance,),
                bounds=bounds, constraints=constraints)

print (opts)
\end{ipython}
\begin{ioutput}
     fun: 0.20887087318996655
     jac: array([0.2088003 , 0.20897953, 0.20895358, 0.20911743, 0.21106115,
                 0.20858843, 0.20914311, 0.20901022, 0.40519709, 0.20881298])
 message: 'Optimization terminated successfully'
    nfev: 110
     nit: 10
    njev: 10
  status: 0
 success: True
       x: array([3.33744344e-01, 1.68530140e-01, 8.18977235e-03, 9.91808728e-02,
                 0.00000000e+00, 7.10377889e-02, 6.05724516e-03, 8.44657341e-02,
                 3.89635156e-20, 2.28794102e-01])
\end{ioutput}
\begin{ipython}
print ("Portfolio composition")
for i, n in enumerate(df.columns):
    print ("{:5}: {:4.1f}%".format(n, opts.x[i]*100))
print ("Portfolio variance: {:.4f}".format(opts.fun**2))
print ("Expected Portfolio return: {:.3f}".format(opts.x.dot(returns)))
\end{ipython}
\begin{ioutput}
Portfolio composition
CHD  : 33.4%
COST : 16.9%
CTRE :  0.8%
ENTR :  9.9%
HI   :  0.0%
LDOS :  7.1%
TDS  :  0.6%
TMUS :  8.4%
UEC  :  0.0%
WM   : 22.9%
Portfolio variance: 0.0436
Expected Portfolio return: 0.131
\end{ioutput}
\end{solution}

%\begin{question}[title={(Return Allocation Portfolio)}]
%After few months the same client of the previous question, contacted you again asking for larger returns. In particular she would like to increase it by 40\%.
%Find the new portfolio composition that makes your client happy.
%\end{question}
%\begin{solution}
%\begin{tcolorbox}[size=fbox, boxrule=1pt, colback=cellbackground, colframe=cellborder]
%\begin{Verbatim}[commandchars=\\\{\}]
%\PY{k}{def} \PY{n+nf}{efficient\PYZus{}frontier}\PY{p}{(}\PY{n}{w}\PY{p}{,} \PY{n}{asset\PYZus{}returns}\PY{p}{,} \PY{n}{target\PYZus{}return}\PY{p}{)}\PY{p}{:} 
%    \PY{n}{portfolio\PYZus{}return} \PY{o}{=} \PY{n}{np}\PY{o}{.}\PY{n}{sum}\PY{p}{(}\PY{n}{asset\PYZus{}returns} \PY{o}{*} \PY{n}{w}\PY{p}{)} 
%    \PY{k}{return} \PY{p}{(}\PY{n}{portfolio\PYZus{}return} \PY{o}{\PYZhy{}} \PY{n}{target\PYZus{}return}\PY{p}{)}
%		
%\PY{n}{results} \PY{o}{=} \PY{p}{[}\PY{p}{]}
%\PY{n}{bounds} \PY{o}{=} \PY{n+nb}{tuple}\PY{p}{(}\PY{p}{(}\PY{l+m+mi}{0}\PY{p}{,} \PY{l+m+mi}{1}\PY{p}{)} \PY{k}{for} \PY{n}{asset} \PY{o+ow}{in} \PY{n+nb}{range}\PY{p}{(}\PY{n}{num\PYZus{}assets}\PY{p}{)}\PY{p}{)}
%\PY{n}{target} \PY{o}{=} \PY{l+m+mf}{0.172}\PY{o}{*}\PY{l+m+mf}{1.4}
%\PY{n}{constraints} \PY{o}{=} \PY{p}{(}\PY{p}{\PYZob{}}\PY{l+s+s1}{\PYZsq{}}\PY{l+s+s1}{type}\PY{l+s+s1}{\PYZsq{}}\PY{p}{:} \PY{l+s+s1}{\PYZsq{}}\PY{l+s+s1}{eq}\PY{l+s+s1}{\PYZsq{}}\PY{p}{,} \PY{l+s+s1}{\PYZsq{}}\PY{l+s+s1}{fun}\PY{l+s+s1}{\PYZsq{}}\PY{p}{:} \PY{n}{efficient\PYZus{}frontier}\PY{p}{,} 
%                \PY{l+s+s1}{\PYZsq{}}\PY{l+s+s1}{args}\PY{l+s+s1}{\PYZsq{}}\PY{p}{:}\PY{p}{(}\PY{n}{returns}\PY{p}{,} \PY{n}{target}\PY{p}{)}\PY{p}{\PYZcb{}}\PY{p}{,}
%               \PY{p}{\PYZob{}}\PY{l+s+s1}{\PYZsq{}}\PY{l+s+s1}{type}\PY{l+s+s1}{\PYZsq{}}\PY{p}{:} \PY{l+s+s1}{\PYZsq{}}\PY{l+s+s1}{eq}\PY{l+s+s1}{\PYZsq{}}\PY{p}{,} \PY{l+s+s1}{\PYZsq{}}\PY{l+s+s1}{fun}\PY{l+s+s1}{\PYZsq{}}\PY{p}{:} \PY{n}{sum\PYZus{}weights}\PY{p}{\PYZcb{}}\PY{p}{)}
%\PY{n}{weights} \PY{o}{=} \PY{p}{[}\PY{l+m+mf}{1.}\PY{o}{/}\PY{n}{num\PYZus{}assets} \PY{k}{for} \PY{n}{\PYZus{}} \PY{o+ow}{in} \PY{n+nb}{range}\PY{p}{(}\PY{n}{num\PYZus{}assets}\PY{p}{)}\PY{p}{]}
%\PY{n}{opts} \PY{o}{=} \PY{n}{minimize}\PY{p}{(}\PY{n}{markowitz}\PY{p}{,} \PY{n}{weights}\PY{p}{,} \PY{n}{args}\PY{o}{=}\PY{p}{(}\PY{n}{covariance}\PY{p}{,}\PY{p}{)}\PY{p}{,}
%\PY{n}{bounds}\PY{o}{=}\PY{n}{bounds}\PY{p}{,} \PY{n}{constraints}\PY{o}{=}\PY{n}{constraints}\PY{p}{)} 
%		
%\PY{n}{p\PYZus{}variance} \PY{o}{=} \PY{n}{np}\PY{o}{.}\PY{n}{dot}\PY{p}{(}\PY{n}{opts}\PY{o}{.}\PY{n}{x}\PY{o}{.}\PY{n}{T}\PY{p}{,} \PY{n}{np}\PY{o}{.}\PY{n}{dot}\PY{p}{(}\PY{n}{covariance}\PY{p}{,} \PY{n}{opts}\PY{o}{.}\PY{n}{x}\PY{p}{)}\PY{p}{)}
%\PY{n}{ret} \PY{o}{=} \PY{n}{np}\PY{o}{.}\PY{n}{sum}\PY{p}{(}\PY{n}{returns}\PY{o}{*}\PY{n}{opts}\PY{o}{.}\PY{n}{x}\PY{p}{)}
%\PY{n+nb}{print} \PY{p}{(}\PY{l+s+s2}{\PYZdq{}}\PY{l+s+s2}{Portfolio variance: }\PY{l+s+si}{\PYZob{}:.3f\PYZcb{}}\PY{l+s+s2}{\PYZdq{}}\PY{o}{.}\PY{n}{format}\PY{p}{(}\PY{n}{variance}\PY{p}{)}\PY{p}{)}
%\PY{n+nb}{print} \PY{p}{(}\PY{l+s+s2}{\PYZdq{}}\PY{l+s+s2}{Portfolio return: }\PY{l+s+si}{\PYZob{}:.3f\PYZcb{}}\PY{l+s+s2}{\PYZdq{}}\PY{o}{.}\PY{n}{format}\PY{p}{(}\PY{n}{ret}\PY{p}{)}\PY{p}{)}
%
%Portfolio variance: 0.026
%Portfolio return: 0.241
%\end{Verbatim}
%\end{tcolorbox}	
%\end{solution}

\begin{question}
The client is not yet satisfied and decides to move to a risk parity portfolio. Compute the correct weights to have each asset contributing equally to the portfolio risk.
\end{question}

\cprotEnv \begin{solution}
\begin{ipython}
def risk_parity(w, cov):
    variance = w.T.dot(cov.dot(w))
    sum = 0 
    N = len(w)
    for i in range(N):
        sum += (w[i] - (variance/(N*(cov.dot(w))[i])))**2
    return sum

args = (covariance,)
constraints = ({'type': 'eq', 'fun': sum_weights})
bounds = tuple((0, 1) for asset in range(num_assets))
weights = [1./num_assets for _ in range(num_assets)]
opts = minimize(risk_parity, weights, args=(covariance,),
                bounds=bounds, constraints=constraints)

print (opts)
\end{ipython}
\begin{ioutput}
     fun: 3.5766519241779855e-07
     jac: array([ 1.24834913e-03,  1.00388202e-03, -2.28052647e-04,  6.19051279e-04,
                 -5.68149005e-04, -1.03169022e-03, -1.24285291e-03, -4.09550915e-04,
                 -2.02491654e-03,  6.26250036e-06])
 message: 'Optimization terminated successfully'
    nfev: 80
     nit: 7
    njev: 7
  status: 0
 success: True
       x: array([0.161952  , 0.12997992, 0.07022277, 0.10331577, 0.07717795,
                 0.10490077, 0.08475445, 0.10276264, 0.04079525, 0.12413849])
\end{ioutput}
\begin{ipython}
sigma_i = []
for i in range(num_assets):
    std = np.sqrt(opts.x.T.dot(covariance.dot(opts.x)))
    a = opts.x[i]*covariance.dot(opts.x)[i]
    sigma_i.append(a/std)

for i in range(num_assets):
    print ("Risk contribution for asset {}: {:6.3f}%"
           .format(i, sigma_i[i]/sum(sigma_i)*100))
\end{ipython}
\begin{ioutput}
Risk contribution for asset 0:  9.986%
Risk contribution for asset 1: 10.003%
Risk contribution for asset 2:  9.971%
Risk contribution for asset 3:  9.987%
Risk contribution for asset 4: 10.014%
Risk contribution for asset 5:  9.998%
Risk contribution for asset 6: 10.047%
Risk contribution for asset 7:  9.976%
Risk contribution for asset 8: 10.023%
Risk contribution for asset 9:  9.995%		
\end{ioutput}
\end{solution}


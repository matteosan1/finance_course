\chapter{Risk Neutral Measure}
\label{risk-neutral}

\section{Discrete single-period model}
\label{sec:risk_neutral1}
Assume that today's stock price is $𝑆_0$, and one period from now, the stock price can be $𝑆_0\cdot u = 𝑆_𝑢$ or $𝑆_0\cdot 𝑑 = 𝑆_𝑑$, with $𝑢$ and $𝑑$ being "up" and "down" multiplicative factors. Assume also that the risk-free rate is $𝑟$.

Without imposing some conditions on $𝑢$, $𝑑$ and $𝑟$, there might be some arbitrage opportunities. If for example $𝑒^{r} > 𝑢$, I could short the stock at time $𝑡_0$ and invest the proceeds $𝑆_0$ into the risk-free account: in both future states at time $𝑡_1$, I could then buy the stock back for less than my proceeds $𝑆_0 𝑒^{𝑟}$ from the risk free (because $𝑆_𝑢$ and $𝑆_𝑑$ would both be less than $𝑆_0 𝑒^{r}$).

If, on the other hand, $𝑒^{r} < 𝑑$, I could borrow exactly $𝑆_0$ money at $𝑡_0$ to buy the stock, and in both future states of the world at time $𝑡_1$, the stock value would be higher than what I would have to repay ($𝑆_0 𝑒^{r}$): so there would be arbitrage again.

Imposing $𝑑 \le e^{r} \le 𝑢$, will ensure no arbitrage in the one-period model.

Now I am going to perform the following algebraic manipulation:

\begin{align}
\begin{split}
𝑆_0 &=𝑆_0 \frac{(𝑢−𝑑)}{(𝑢−𝑑)} = \\
&=\frac{1}{𝑒^{r}}𝑆_0 \frac{(𝑢−𝑑)𝑒^{r}}{(𝑢−𝑑)}= \\
&=\frac{1}{𝑒^{r}}𝑆_0 \frac{(𝑢−𝑑)𝑒^{r}+(𝑆_0𝑢𝑑−𝑆_0𝑢𝑑)}{(𝑢−𝑑)}= \\
&=\frac{1}{𝑒^{r}}\left(\frac{𝑆_0𝑢𝑒^{𝑟}−(𝑆_0𝑢𝑑)}{𝑢−𝑑}+\frac{−𝑆_0𝑑𝑒^{𝑟}+(𝑆_0𝑢𝑑)}{𝑢−𝑑}\right)= \\
&=\frac{1}{𝑒^{r}}(𝑆_0𝑢 \left(\frac{𝑒^{𝑟}−𝑑}{𝑢−𝑑}\right)+𝑆_0𝑑\left(\frac{𝑢−𝑒^{𝑟}}{𝑢−𝑑}\right))
\end{split}
\end{align}

The no arbitrage condition $𝑑 \le e^{r} \le 𝑢$ will result in the following bounds 

\begin{equation}
\begin{cases}
0 \le \dfrac{𝑒^{𝑟}−𝑑}{𝑢−𝑑} \le 1 \\ 
0 \le \dfrac{𝑢−𝑒^{r}}{𝑢−𝑑} \le 1
\end{cases}
\end{equation}

Furthermore:

\begin{equation*}
\frac{𝑒^{𝑟}−𝑑}{𝑢−𝑑}+\frac{𝑢−𝑒^{𝑟}}{𝑢−𝑑}=1
\end{equation*}

Let's call $𝑒^{𝑟}−𝑑𝑢−𝑑 = 𝑝_𝑢$ and $𝑢−𝑒^{𝑟}𝑢−𝑑 = 𝑝_𝑑$. In the one period model, the stock going up and the stock going down are two different states of the world, i.e. there is no "intersection" between these states in the probabilistic sense. Therefore 𝑝𝑢 and 𝑝𝑑 are additive over disjoint sets and they are within the zero-one range, therefore mathematically, these parameters qualify as a probability measure.

Rewriting the algebraic manipulation above in terms of $𝑝_𝑢$ and $𝑝_𝑑$ yields the following:

\begin{equation}
𝑆_0=\frac{𝑆_𝑢 𝑝_𝑢 +𝑆_𝑑 𝑝_𝑑}{𝑒^{𝑟}}=\frac{1}{𝑒^{r}}\mathbb{E}[𝑆_1]
\end{equation}

Also notice that in the entire construction above, we did not talk about the probabilities of the stock going up or down. Every market participant might have his or her Bayesian view of the world with different probabilities assigned to the stock going up or down. But the risk-neutral measure is agreed upon by the market as a whole as a consequence of no arbitrage.

This also brings up an interesting point: in my view, the risk neutral probabilities are probabilities only in the "mathematical object" sense. They do not actually represent "likelihoods", in the sense that we human beings like to interpret probabilistic events with.

\section{Pricing Derivatives}
Let's assume we want to price a derivative on the stock with pay-off function $𝑉(𝑆_𝑡)$ (could be a forward, option, whatever). The derivative pay-off in the two states will trivially be $𝑉(𝑆_𝑢)$ and $𝑉(𝑆_𝑑)$. We have two states, two underlying instruments: let's try to replicate the derivative pay-off in both states ($𝑥$ is the number of stocks and $𝑦$ is the amount invested in the risk-free account: I want to replicate the derivative pay-off in both states with $𝑥$ stocks and $𝑦$ risk-free investment):

\begin{equation}
\begin{split}
𝑥𝑆_𝑢+𝑦𝑒^{𝑟}=𝑉(𝑆_𝑢)\\
𝑥𝑆_𝑑+𝑦𝑒^{𝑟}=𝑉(𝑆_𝑑)
\end{split}
\end{equation}

Solving gives:

\begin{align}
\begin{split}
𝑥&=\frac{𝑉(𝑆_𝑢)−𝑉(𝑆_𝑑)}{𝑆_0(𝑢−𝑑)}\\
𝑦&=\frac{𝑢𝑉(𝑆_𝑑)−𝑑𝑉(𝑆_𝑢)}{(𝑢−𝑑)}\frac{1}{𝑒^{𝑟}}
\end{split}
\end{align}

Therefore the derivative price at time $𝑡_0$ is the $𝑥$ amount of the stock + $𝑦$ amount invested in the risk-free account:

\begin{equation}
𝑉(𝑆_0,𝑡_0) =𝑥\cdot 𝑆_0+𝑦\cdot 1 =\frac{𝑉(𝑆_𝑢)−𝑉(𝑆_𝑑)}{𝑆_0(𝑢−𝑑)}\cdot 𝑆_0 + \frac{𝑢𝑉(𝑆_𝑑)−𝑑𝑉(𝑆_𝑢)}{(𝑢−𝑑)}\frac{1}{e^{r}}\cdot 1
\end{equation}

The above evaluates to:
\begin{equation}
\frac{1}{e^{r}}\left(𝑉(𝑆_𝑢)\left(\frac{e^{r}−𝑑}{𝑢−𝑑}\right)+𝑉(𝑆_𝑑)\left(\frac{𝑢−e^{r}}{𝑢−𝑑}\right)\right)
\end{equation}

Notice that again we can write $𝑒^{𝑟}−𝑑𝑢−𝑑 = 𝑝_𝑢$ and $𝑢−𝑒^{𝑟}𝑢−𝑑 = 𝑝_𝑑$, where notably $𝑝_𝑢$ and $𝑝_𝑑$ are the same as in Section~\ref{sec:risk_neutral1} above, therefore, \textbf{instead of having to compute the replication portfolio weights $𝑥$ and $𝑦$}, the derivative can be priced as:

\begin{equation}
𝑉(𝑆_0,𝑡_0)=\frac{1}{e^{r}}(𝑉(𝑆_𝑢)𝑝_𝑢+𝑉(𝑆_𝑑)𝑝_𝑑)=\frac{1}{e^{r}}\mathbb{E}[𝑉(𝑆_1,𝑡_1)]
\end{equation}

The risk-neutral measure pricing technique has the following features:
\begin{itemize}
\item is a consequence of no-arbitrage assumptions in the model;
\item taking the expectation of a derivative pay-off and discounting it to today is the equivalent of computing "replication portfolio" weights at each time-step, and pricing the derivative using these replicating weights at time $𝑡_0$.
\end{itemize}

\section{Continous-time Models}
Extending the one-period model leads to a multi-period "binomial tree" discrete model. Pricing a derivative on a multi-period tree would require working "backwards" from the terminal pay-off and computing the replicating portfolio pay-off at each node. Alternatively, the more convenient way is to use the risk-neutral expectation of the terminal pay-off and discounting it to "today": as that will produce the same result (as shown above) and will save us having to worry about the replicating portfolio weights.

%As a sketch, we can show that the multi-period Binomial model for the stock converges to the well-known continuous Geometric Brownian Motion (GBM) model (which in turn can be used to derive the Black-Scholes formula directly by applying the risk-neutral expectation to the option pay-off at maturity where the stock process is simulated with GBM).
%
%For the multi-period Binomial model, we have:
%
%𝑆𝑛=𝑆0𝑢𝑘𝑑𝑛−𝑘
%
%with 𝑘∼𝐵𝑖𝑛(𝑛,𝑝). The "Cox-Ross-Rubinstein" parameters for "up" and "down" are then set as follows: 𝑢:=𝑒𝜂𝑇𝑛+𝑇𝑛√𝜎, 𝑑:=𝑒𝜂𝑇𝑛−𝑇𝑛√𝜎.
%
%It is well known that the Binomial distribution converges to Normal, in fact, using CLT, we get:
%
%lim𝑛→∞𝑘−→𝑑𝑁(𝑛𝑝,𝑛𝑝(1−𝑝)‾‾‾‾‾‾‾‾‾√)
%
%Going back to the equation for 𝑆𝑛, taking a log and substituting for 𝑢 and 𝑑, we get:
%
%𝑙𝑛(𝑆𝑛𝑆0)=𝑘(2𝑇𝑛‾‾‾√𝜎)+𝑛(𝜂𝑇𝑛−𝑇𝑛‾‾‾√𝜎)
%
%Which gives:
%
%𝔼[𝑙𝑛(𝑆𝑛𝑆0)]=𝑇𝑛‾‾‾√𝜎(2𝑝−1)+𝜂𝑇
%
%𝑉𝑎𝑟(𝑙𝑛(𝑋𝑛𝑋0))=4𝑇𝜎2𝑝(1−𝑝))
%
%So we see that:
%
%lim𝑛→∞(𝑋𝑛𝑋0)−→𝑑𝑁(𝑇𝑛‾‾‾√𝜎(2𝑝−1)+𝜂𝑇,2𝑇‾‾√𝜎𝑝(1−𝑝))‾‾‾‾‾‾‾‾‾√
%
%Taking 𝑝=0.5, we get:
%
%lim𝑛→∞𝑙𝑛(𝑋𝑛𝑋0)−→𝑑𝑁(𝜂𝑇,𝜎𝑇‾‾√)
%
%If we now set 𝜂:=𝜇−0.5𝜎2, we see that the discrete multiperiod Binomial model for the stock price converges exactly to the continuous Geometric-Brownian-Motion model in distribution.
%
%Another interesting fact to note is that the replicating weight of the stock in part 1, i.e. 𝑥, converges to 𝑁(𝑑1), i.e. the instantaneous option Delta.
%
%I will conclude by producing the same summary as Kevin, but with the following additional comments:

\section{Summary}

Risk-neutral probability measures are artificial and created by assuming the existence of no-arbitrage and completeness.

Derivatives can be priced relative to underlying assets. This hedging price can be computed as expectation with respect to the risk-neutral probability measure. Equivalent martingale measures are a consequence of arbitrage and completeness.

The risk-neutral density can be estimated from observed market data, i.e. twice differentiating the Implied Vol surface with respect to strike. The risk-neutral framework connects many different approaches to derivatives pricing.

\documentclass{beamer}
\usetheme{afm}

\title{Advanced Financial Modeling}
\author{\href{mailto:matteo.sani@unisi.it}{Matteo Sani}}

\begin{document}
\begin{frame}[plain]
	\maketitle
\end{frame}

\begin{frame}{The Course}
	\begin{itemize}
		\item Due to some delays the course schedule has been condensed. All the lessons will be held in just 6 weeks.
		
		\begin{table}[bt]
			\begin{tabular}{|l|c|c|} \hline
				\textbf{Weekday} & \textbf{Room} & \textbf{Time} \\ \hline
				Monday & 1\footnotemark & 10-12 \\ \hline
				Tuesday & 1 & 18-19.30 \\ \hline		
				Wednesday & 1 & 18-19.30 \\ \hline
				Thursday & 13 & 10-12\\ \hline
			\end{tabular}
		\end{table}
		\item What about next Monday ?
	\end{itemize}
	\footnotetext[1]{Except 15$^{th}$ of May Room 6}
\end{frame}

\begin{frame}{Syllabus}
	\begin{itemize}
	\item The aim of the course is to provide the students with the necessary background to start a career as a trader or a quantitative analyst in a Financial Institution.
	\item The focus is on Fixed Income Derivatives and XVAs. Detailed references will be given and teaching materials will be handed during the lectures.
	\item \textcolor{orange}{Attendance to the lectures is warmly recommended.}
\end{itemize}
\end{frame}

\begin{frame}{Syllabus}
	Main topics covered:
	\begin{enumerate}	
	\item \textcolor{maincolor}{Review of the pricing approaches to Fixed Income Derivatives in the old framework with a critical focus on the underlying hypotheses}
	\item \textcolor{maincolor}{The change of measure technique}
	\item \textcolor{maincolor}{A primer on market models and Black Formula for Caps and Swaptions}
	\item The segmentation of the interest rate markets after the 2008 Crisis and the new multicurve approach
	\item The monetary Policies of the Central Banks, \textcolor{maincolor}{the advent of negative rates and the need for models allowing for negative yields}
	\item Pricing and hedging with the Smile: stochastic volatility and local volatility models
	\item Short Rate Models 
	\item Topics in Credit Risk Modeling with a crash Primer
	\item Introduction to XVAs. Definitions, methods and open problems
	\end{enumerate}
\end{frame}

\begin{frame}{Recommended Books}
	Here is a list of Books we recommend to anyone who wants to become an accomplished professional in the field of Quantitative Finance (Trader, Quant, Risk Manager):
	\begin{enumerate}
	\item Hull, John. Options, \textit{Futures And Other Derivatives}, Pearson College, last edition
	\item Bjork, Thomas. \textit{Arbitrage Theory in Continuous Time}, OUP, 4th edition
	\item Brigo Damiano and Mercurio Fabio, \textit{Interest Rate Models Theory and Practice, with Smile, Inflation and Credit}, Springer Verlag, 2006
	\item Rebonato, Riccardo. \textit{Volatility and Correlation: The Perfect Hedger and the Fox}, Wiley and Sons, 2004
	\item Derman, Emanuel. \textit{The Volatility Smile: An Introduction for Students and Practitioners}, Wiley, 2016
	\item Taleb, Nassim. \textit{Dynamic Hedging: Managing Vanilla and Exotic Options}, Wiley and Sons, 1997
	\item Coen, Guy. \textit{The Bible of Options Strategies: The Definitive Guide for Practical Trading Strategies}, Ft Pr; Reprint 2015-06-05
\end{enumerate}
\end{frame}

\begin{frame}{Recommended Books}
	\begin{enumerate}
		\item Bennet, Colin. \textit{Trading Volatility: Trading Volatility, Correlation, Term Structure and Skew}, CreateSpace Independent Publishing Platform, 2014
		\item Brigo Damiano, Morini Massimo, Pallavicini Andrea. \textit{Counterparty Credit Risk}, Collateral and Funding: With Pricing Cases for All Assets. Wiley and Sons, 2013
		\item Gregory, John. \textit{The xVa Challenge: Counterparty Risk, Funding, Collateral, Capital and Initial Margin}, Wiley and Sons, 2020
		\item Neftci, Salih and Hirsa, Ali. \textit{An Introduction to the Mathematics of Financial Derivatives}, Academic Press, 2013
	\end{enumerate}
\end{frame}

%\begin{frame}{Final Mark}
%	\begin{}
%	\end{}
%\end{frame}
\end{document}

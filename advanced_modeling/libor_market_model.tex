\documentclass{beamer}
\usetheme{CambridgeUS}

\title{Libor Market Model}
\author{Matteo Sani}
\begin{document}
	\begin{frame}[plain]
		\maketitle
	\end{frame}

\begin{frame}{Market Models}
\begin{itemize}
\item Before market models were introduced, there were no interest-rate dynamics compatible with the Black formula for Caps and Swaptions.
\item The (forward) LIBOR market model assumes lognormality of the forward rates under their own measure
\item The (forward) SWAP market model assumes lognormality of the swap rates, under their own measure.
\item They are not compatible each other.
\end{itemize}
\end{frame}

\begin{frame}{Map}
\begin{itemize}
\item Vanilla-quoted in the market (FRAs-Swaps: only the yield curve is needed)
\item CAPS\&FLOORS EUROPEAN SWAPTIONS
\item Standardized contracts for many maturities. Used as input to determine the Set of ($P(o,T)$ needed to price not quoted instruments. Here models are needed mainly for hedging.)
\item No Vanilla-not quaoted in the market BERMUDAN SWAPTIONS  LIBRO EXOTICS
\item For these Exotics you need a model to price and Hedge
\end{itemize}
\end{frame}

\begin{frame}{Basic Valuation Formula}
\begin{itemize}
\item Absence of arbitrage implies the existence of a martingale measure $\mathcal{Q}$ such that, for every contingent claim whose payoff at time $t$ is $\chi$, we have the following formula for its price $\Pi$:
\begin{equation}
\Pi(0,\chi) = \mathbb{E}^{\mathcal{Q}}\left[e^{-\int_0^T r_s ds}\chi\right]
\end{equation}
\item The main issue of this formula is the interaction between the two terms: the discount factor $e^{-\int_0^T r_s ds}$ and the contingent claim payoff $\chi$.
\item We then devise a technique which allows the separation of the two.
\end{itemize}
\end{frame}


\begin{frame}{Numeraires}
\begin{block}{Definition}
A numeraire is any positive non-dividend-paying asset.
\end{block}
\begin{itemize}
\item In general, a numeraire is a portfolio or a self-financing strategy.
\item The condition for a strategy to be self-financing in an economy with $K$ assets is
\begin{equation}
dV_t = \sum_{k=0}^K \phi^k_t dS^k_t
\end{equation}
which implies that form every numeraire $Z_t$
\begin{equation}
d\left(\frac{V_t}{Z_t}\right) = \sum_{k=0}^K \phi^k_t d\left(\frac{S^k_t}{Z_t}\right)
\end{equation}
\end{itemize}
\end{frame}


\begin{frame}{Numeraires}
\begin{itemize}
\item Intuitively, the numeraire is a reference asset that normalizes all other assets in the economy.
\item The simplest numeraire is the money market account
\begin{equation}
B_t = e^{\int_0^t r_s ds}
\end{equation} 
\item Also the ZCB price $(P(0,T)$ can be a numeraire.
\item And also your friend BPV.
\end{itemize}
\end{frame}


\begin{frame}{Numeraire Pricing}
\begin{block}{Theorem (German, El Karoui and Rochet, 1995)}
Assume that there exists a numeraire $N$ and a probability measure $\mathcal{Q}^N$ which is equivalente to $\mathcal{P}$ such that, for every traded asset $X$:
\begin{equation}
\frac{X_t}{N_t} = \mathbb{E}^{\mathcal{Q}^N}\left[\frac{X_T}{N_T}|\mathcal{F}_t\right]
\end{equation}
Now, given a second arbitrary numeraire $U$, there exists a probability measure $\mathcal{Q}^U$ which is equivalent to $\mathcal{P}$ and such that:
\begin{equation}
\frac{X_t}{U_t} = \mathbb{E}^{\mathcal{Q}^U}\left[\frac{X_T}{U_T}|\mathcal{F}_t\right]
\end{equation}
\end{block}
\end{frame}


\begin{frame}{Numeraire Pricing}
\begin{block}{Theorem (German, El Karoui and Rochet, 1995)}
Moreover, the Radon-Nykodin derivative of the probability change from $\mathcal{Q}^N$ to $\mathcal{Q}^U$ is given by:
\begin{equation}
\frac{d\mathcal{Q}^U}{d\mathcal{Q}^N} = \frac{U_T N_0}{U_0 N_T} 
\end{equation}
\end{block}
\end{frame}


\begin{frame}{title}
\begin{itemize}
	\item Absence of arbitrage insures the existence of at least one numeraire: the money market account.
	\item The numeraire theorem allows other numeraires to be used to price a contingent claim.
	\item It is the natural to look for the most convenient numeraire, that one which minimize the mathematical difficulties.
	\item When changing numeraire, also the drift will change (convexity adjustments).
\end{itemize}
\end{frame}


\begin{frame}{How to Use the Change of Numeraire}
\begin{itemize}
	\item We have to compute
	\begin{equation}
		\Pi(0,\chi)=\mathbb{E}^{\mathcal{Q}}\left[e^{-\int_0^T r_s ds}\chi\right]=\mathbb{E}^{\mathcal{Q}}\left[\frac{\chi}{B_T}\right]
	\end{equation}
\item By changing numeraire and using the new one $S_t$
	\begin{equation}
	\Pi(0,\chi)=S_0\mathbb{E}^{\mathcal{Q}^S}\left[\frac{\chi}{S_T}\right]
\end{equation}
\item The chosen numeraire should then make the quantity $\frac{\chi}{S_T}$ particularly simple.
\end{itemize}
\end{frame}

\begin{frame}{title}
\begin{itemize}
\item In many concrete situations, the best numerarie is the ZCB with the same maturity of the derivative to price.
In this case $S_t = P(T,T)=1$.
\item The forward measure $\mathcal{Q}^T$ (the T-measure) is defined as the martingale measure for the numeraire process $P(t,T)$, where $P(t,T)$ is the ZCB maturing in T.
\item It is easy to see that, in this case the Radon-Nykodin derivative is given by
\begin{equation}
\frac{d\mathcal{Q}^T}{d\mathcal{Q}} = \frac{P(t,T)}{B_t P(0,T)} 
\end{equation}
\item Using the pricing formula after the change of numeraire, we finally have
\begin{equation}
\Pi(0,\chi)=P(t,T)\mathbb{E}^{\mathcal{Q}^T}[\chi]
\end{equation}
which achieves the desired separation (although under a new measure).
\item Notice that if the interest rates are deterministic, then $\mathcal{Q} = \mathcal{Q}^T$
\end{itemize}
\end{frame}


\begin{frame}{The Expectation Hypothesis}
\begin{itemize}
\item It is possible to prove the following
\begin{equation}
f(t, T) = \mathbb{E}^{\mathcal{Q}^T}[r(T)|\mathcal{F}_t]
\end{equation}
\item According to the pure expectation hypothesis, the above formula is valid if the expected value is taken under the real probability.
\item Absence of arbitrage makes this incompatible with stochastic interest rates.
\end{itemize}
DA CAPIRE MEGLIO
\end{frame}


\begin{frame}{Drift Changes}
\begin{itemize}
\item Assume that, under the $S$-measure, we have
\begin{equation}
dX_t = \mu^S(X_t)dt + \sigma(X_t)dW^S_t
\end{equation}
where $dW^S_t$ is n-dimensional standard brownian motion.
\item Under the $U$-measure, we have
\begin{equation}
\mu^U_t(X_t) = \mu^S_t(X_t) - \rho\sigma(X_t)\left(\frac{\sigma^S_t}{S_t}-{\sigma^U_t}{U_t}\right) 
\end{equation}
or
\begin{equation}
dW^U_t = dW^S_t + \rho\sigma\left(\frac{\sigma^S_t}{S_t}-{\sigma^U_t}{U_t}\right) 
\end{equation}
$\rho$ is the correlation matrix of $<dW^S,dW^U>$ and $\sigma^S_t$ and $\sigma^U_t$ are the (vector) volatilities of numeraires $S$ and $U$ (one component for each brownian motion).
\end{itemize}
\end{frame}



\begin{frame}{A General Option Pricing Formula}
\begin{itemize}
	\item Consider an European call on an asset $S$ which is also a numeraire
	\begin{equation}
		\chi = \max[S_T-K,0]
	\end{equation}
\item Denote by $\mathcal{Q}^S$ the martingale measure for the numeraire $S$, and by $\mathcal{Q}^T$ the forward measure.
\begin{block}{Theorem (German, El Karoui and Rochet, 1995)}
	\begin{equation}
\Pi(0,\chi) = S_0\mathcal{Q}^S(S_T \geq K) - KP(0,T)\mathcal{Q}^T(S_T\geq K)
\end{equation}
\end{block}
\item This is a general-purpouse Black-Scholes formula.
\item Put prices can be computed by put-call parity.
\end{itemize}
\end{frame}


\begin{frame}
\begin{itemize}
\item To make this formula simpler in a special (but relevant case), we assume that the process 
\begin{equation}
Z_t = \frac{S_t}{P(t,T)}
\end{equation}
follows the diffusion
\begin{equation}
dZ_t = Z_tm_tdt + Z_t\sigma_tdW_t
\end{equation}
where $sigma_t$ is non-stochastic (can you quote a famous example of this kind ?).
\item Under $\mathcal{Q}^T$, $Z-t$ is a martingale, then
\begin{equation}
	dZ_t = Z_t\sigma_tdW^T_t
\end{equation}
which implies
\begin{equation}
Z_t = Z_0 e^{-\frac{1}{2}\int_0^T\sigma_t^2 dt + \int_0^T\sigma_tdW^T_t}
\end{equation}
which in turn implies that
\begin{equation}
\mathcal{Q}^T(S_T\geq K) = \mathcal{N}(d_2)
\end{equation}
\end{itemize}
\end{frame}


\begin{frame}{title}
\begin{itemize}
	\item Now define
	\begin{equation}
		Y_t = \frac{P(t,T)}{S_t}=\frac{1}{Z_t}
	\end{equation}
Again, under $\mathcal{Q}^S$ $Y_t$ is a martingale, whose volatility must be $-\sigma_t$ by Ito's lemma, so we have
\begin{equation}
	dY_t = -Y_t\sigma_tdW^S_t
\end{equation} 
which implies
\begin{equation}
Y_t = Y_0 e^{-\frac{1}{2}\int_0^T\sigma_t^2 dt - \int_0^T\sigma_tdW^S_t}
\end{equation}
\item This implies in turn that
\begin{equation}
\mathcal{Q}^S(S_T\geq K) = \mathcal{N}(d_1)
\end{equation}
\end{itemize}
\end{frame}


\begin{frame}
\begin{block}{Extended Black and Scholes formula (deterministic volatility)}
\begin{equation}
\Pi(0,\chi) = S_0\mathcal{N}(d_1) - KP(0,T)\mathcal{N}(d_2)
\end{equation}
where $d_1$ and $d_2$ are computed with $\sigma^2 = \int_0^T \sigma_s^2 ds$
\end{block}
\end{frame}




\begin{frame}{Toward the Black-76 Formula}
\begin{itemize}
	\item Consider a caplet resetting at $T_1$ and payed at $T_2$. The payout is 
	\begin{equation}
		\tau(L(T_1,T_2)-X)^+ = \tau(F(T_1,T_1,T_2)-X)^+
	\end{equation}
\item Its value in $t$ is given by
	\begin{equation}
	\mathbb{E}\left[e^{-\int_t^T2r_s ds}\tau(F(T_1,T_1,T_2)-X)^+\right]
\end{equation}
\item Notice that $P(0,T_2)F(T_1,T_1,T_2)$ is a tradable asset since
\begin{equation}
P(t,T2)F(T_1,T_1,T_2)=\frac{P(t,T_1)-P(t,T_2)}{\tau}
\end{equation}
Hence changing numeraire to $P(t,T_2)$ we get
\begin{equation}
P(t,T2)\tau \mathbb{E}^{\mathcal{T_2}}\left[(F(T_1,T_1,T_2)-X)^+\right]
\end{equation}
\item Now we assume lognormality of $F(t,T_1,T_2)$ under the $T_2$-measure
\begin{equation}
dF(t,T_1,T_2)=v_2F(t,T_1,T_2) dW^{T_2}_t
\end{equation}
\end{itemize}
\end{frame}


\begin{frame}
\begin{itemize}
	\item This implies the Black-76 formula for caps.
	\item The LMM is automatically fitted to cap prices.
	\item To price a more complicated derivative, we have to compute the drift of the forward rates under a unique measure.
	\item For example, $F(t,T_1,T_2)$ is driftless in the $T_2$-measure. In the $T_1$-measure
	\begin{equation}
	dF(t,T_1,T_2)=\frac{v_2F(t,T_1,T_2)\tau}{1+\tau F(t,T_1,T_2)}dt + v_2F(t,T_1,T_2)dW^{T_1}_t
	\end{equation}
\item We need simulation of the forward rates to price exotic instruments.
\end{itemize}
\end{frame}


\begin{frame}{The Libor Market Model}
\begin{itemize}
\item We now specify the LIBOR market model. We have a set $T_0,\ldots,T_M$ of dates and $\tau_i=T_i-T_{i-1}$. Settlement date is fixed to 0.
\item Now consider forward rate $F_k(t)=F(t,T_{k-1},T_k)$. They last from $t$ to $T_{k-1}$.
\item $F_k(t)$ is lognormally distributed under the $T_k$-measure, that is 
\begin{equation}
dF_k(t) = \sigma_k(t)F_k(t)dW^k(t),\quad t\leq T_{k-1}
\end{equation} 
where $dW^k(t)$ is an M-dimensional brownian motion with correlation matrix $\rho$. $\sigma_k$ is an M-vector of volatilities whose only non-zero entry is at the $k$-th place.
\item A typical specification of $\sigma_k(t)$ is piece-wise constant.
\end{itemize}
\end{frame}



\begin{frame}{Decorrelation}
\begin{itemize}
	\item In the LMM, we can assume that the brownian motions driving the dynamics of forward rate are correlated
	\begin{equation}
	<dW_t^{T_i}, dW_t^{T_i}> = \rho_{ij}dt
	\end{equation}
\item In models for short rate or for i.f.r. it is assumed full correlation $\rho_{ij}=1$, which is a tight constraint on the dynamics of the forward rates.
\item In the LMM, we can allow decorrelation to better fit the derivatives at hand.
\item The change of measure with decorrelation implies
\begin{equation}
dF(t,T_1,T_2)=\rho_{12}\frac{v_2F(t,T_1,T_2)\tau}{1+\tau F(t,T_1,T_2)}dt + v_2F(t,T_1,T_2)dW^{T_1}_t
\end{equation}
\item Correlations have no impacts on Caps.
\end{itemize}
\end{frame}






%\begin{frame}{A General Option Pricing Formula}
%	\begin{itemize}
	%		\item Consider an European call on an asset $S$ which is also a numeraire
	%		\begin{equation}
		%			\chi = \max[S_T-K,0]
		%		\end{equation}
	%		\item Denote by $\mathcal{Q}^S$ the martingale measure for the numeraire $S$, and by $\mathcal{Q}^T$ the forward measure.
	%		\begin{block}{Theorem (German, El Karoui and Rochet, 1995)}
		%			\begin{equation}
			%				\Pi(0,\chi) = S_0\mathcal{Q}^S(S_T \geq K) - KP(0,T)\mathcal{Q}^T(S_T\geq K)
			%			\end{equation}
		%		\end{block}
	%		\item This is a general-purpouse Black-Scholes formula.
	%		\item Put prices can be computed by put-call parity.
	%	\end{itemize}
%\end{frame}
%
%\begin{frame}
%To prove the theorem, note that the value of the option is given by
%
%\begin{equation}
%\Pi(0,\chi) = \mathbb{E}^{\mathcal{Q}}[B_TS_T I_{\{S_T\geq K\}}] - K\mathbb{E}^{\mathcal{Q}}[B_TI_{\{S_T\geq K\}}]
%\end{equation}
%\end{frame}
%
%\begin{frame}
%	\begin{itemize}
	%		\item To make this formula simpler in a special (but relevant case), we assume that the process 
	%		\begin{equation}
		%			Z_t = \frac{S_t}{P(t,T)}
		%		\end{equation}
	%		follows the diffusion
	%		\begin{equation}
		%			dZ_t = Z_tm_tdt + Z_t\sigma_tdW_t
		%		\end{equation}
	%		where $sigma_t$ is non-stochastic (can you quote a famous example of this kind ?).
	%		\item Under $\mathcal{Q}^T$, $Z-t$ is a martingale, then
	%		\begin{equation}
		%			dZ_t = Z_t\sigma_tdW^T_t
		%		\end{equation}
	%		which implies
	%		\begin{equation}
		%			Z_t = Z_0 e^{-\frac{1}{2}\int_0^T\sigma_t^2 dt + \int_0^T\sigma_tdW^T_t}
		%		\end{equation}
	%		which in turn implies that
	%		\begin{equation}
		%			\mathcal{Q}^T(S_T\geq K) = \mathcal{N}(d_2)
		%		\end{equation}
	%	\end{itemize}
%\end{frame}
%
%
%\begin{frame}{title}
%	\begin{itemize}
	%		\item Now define
	%		\begin{equation}
		%			Y_t = \frac{P(t,T)}{S_t}=\frac{1}{Z_t}
		%		\end{equation}
	%		Again, under $\mathcal{Q}^S$ $Y_t$ is a martingale, whose volatility must be $-\sigma_t$ by Ito's lemma, so we have
	%		\begin{equation}
		%			dY_t = -Y_t\sigma_tdW^S_t
		%		\end{equation} 
	%		which implies
	%		\begin{equation}
		%			Y_t = Y_0 e^{-\frac{1}{2}\int_0^T\sigma_t^2 dt - \int_0^T\sigma_tdW^S_t}
		%		\end{equation}
	%		\item This implies in turn that
	%		\begin{equation}
		%			\mathcal{Q}^S(S_T\geq K) = \mathcal{N}(d_1)
		%		\end{equation}
	%	\end{itemize}
%\end{frame}
%
%\begin{frame}
%	\begin{block}{Extended Black and Scholes formula (deterministic volatility)}
	%		\begin{equation}
		%			\Pi(0,\chi) = S_0\mathcal{N}(d_1) - KP(0,T)\mathcal{N}(d_2)
		%		\end{equation}
	%		where $d_1$ and $d_2$ are computed with $\sigma^2 = \int_0^T \sigma_s^2 ds$
	%	\end{block}
%\end{frame}
%
%\begin{frame}{Toward the Black-76 Formula}
%	\begin{itemize}
	%		\item Consider a caplet resetting at $T_1$ and payed at $T_2$. The payout is 
	%		\begin{equation}
		%			\tau(L(T_1,T_2)-X)^+ = \tau(F(T_1,T_1,T_2)-X)^+
		%		\end{equation}
	%		\item Its value in $t$ is given by
	%		\begin{equation}
		%			\mathbb{E}\left[e^{-\int_t^T2r_s ds}\tau(F(T_1,T_1,T_2)-X)^+\right]
		%		\end{equation}
	%		\item Notice that $P(0,T_2)F(T_1,T_1,T_2)$ is a tradable asset since
	%		\begin{equation}
		%			P(t,T2)F(T_1,T_1,T_2)=\frac{P(t,T_1)-P(t,T_2)}{\tau}
		%		\end{equation}
	%		Hence changing numeraire to $P(t,T_2)$ we get
	%		\begin{equation}
		%			P(t,T2)\tau \mathbb{E}^{\mathcal{T_2}}\left[(F(T_1,T_1,T_2)-X)^+\right]
		%		\end{equation}
	%		\item Now we assume lognormality of $F(t,T_1,T_2)$ under the $T_2$-measure
	%		\begin{equation}
		%			dF(t,T_1,T_2)=v_2F(t,T_1,T_2) dW^{T_2}_t
		%		\end{equation}
	%	\end{itemize}
%\end{frame}
%
%
%\begin{frame}
%	\begin{itemize}
	%		\item This implies the Black-76 formula for caps.
	%		\item The LMM is automatically fitted to cap prices.
	%		\item To price a more complicated derivative, we have to compute the drift of the forward rates under a unique measure.
	%		\item For example, $F(t,T_1,T_2)$ is driftless in the $T_2$-measure. In the $T_1$-measure
	%		\begin{equation}
		%			dF(t,T_1,T_2)=\frac{v_2F(t,T_1,T_2)\tau}{1+\tau F(t,T_1,T_2)}dt + v_2F(t,T_1,T_2)dW^{T_1}_t
		%		\end{equation}
	%		\item We need simulation of the forward rates to price exotic instruments.
	%	\end{itemize}
%\end{frame}
%
%
%\begin{frame}{The Libor Market Model}
%	\begin{itemize}
	%		\item We now specify the LIBOR market model. We have a set $T_0,\ldots,T_M$ of dates and $\tau_i=T_i-T_{i-1}$. Settlement date is fixed to 0.
	%		\item Now consider forward rate $F_k(t)=F(t,T_{k-1},T_k)$. They last from $t$ to $T_{k-1}$.
	%		\item $F_k(t)$ is lognormally distributed under the $T_k$-measure, that is 
	%		\begin{equation}
		%			dF_k(t) = \sigma_k(t)F_k(t)dW^k(t),\quad t\leq T_{k-1}
		%		\end{equation} 
	%		where $dW^k(t)$ is an M-dimensional brownian motion with correlation matrix $\rho$. $\sigma_k$ is an M-vector of volatilities whose only non-zero entry is at the $k$-th place.
	%		\item A typical specification of $\sigma_k(t)$ is piece-wise constant.
	%	\end{itemize}
%\end{frame}
%
%\begin{frame}{Decorrelation}
%	\begin{itemize}
	%		\item In the LMM, we can assume that the brownian motions driving the dynamics of forward rate are correlated
	%		\begin{equation}
		%			<dW_t^{T_i}, dW_t^{T_i}> = \rho_{ij}dt
		%		\end{equation}
	%		\item In models for short rate or for i.f.r. it is assumed full correlation $\rho_{ij}=1$, which is a tight constraint on the dynamics of the forward rates.
	%		\item In the LMM, we can allow decorrelation to better fit the derivatives at hand.
	%		\item The change of measure with decorrelation implies
	%		\begin{equation}
		%			dF(t,T_1,T_2)=\rho_{12}\frac{v_2F(t,T_1,T_2)\tau}{1+\tau F(t,T_1,T_2)}dt + v_2F(t,T_1,T_2)dW^{T_1}_t
		%		\end{equation}
	%		\item Correlations have no impacts on Caps.
	%	\end{itemize}
%\end{frame}

\end{document}

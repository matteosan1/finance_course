\documentclass{beamer}
\usetheme{afm}

\title{Change of Measure and Its Applications}
\course{Advanced Financial Modeling}
\author{\href{mailto:matteo.sani@unisi.it}{Matteo Sani}}

\begin{document}
\begin{frame}[plain]
  \maketitle
\end{frame}

\section{Change of Measure}
\begin{frame}{Few Definitions}
  It may be helpful to explain (and recall) some of the more technical terms we are going to use.\newline
  
  \textbf{Sample space}: all possible future states or outcomes ($\Omega$) of a random process.\newline
  
  \textbf{(Probability) Measure} ($\mathcal{P}, \mathcal{Q}\ldots$): is a mapping which associates a probability to each element in the sample space. Two measures are \textbf{equivalent} if they agree "on what is possible". Note the word \emph{possible}: the two measures can have different probabilities for the same event, but must have the same \emph{null-set} $\{x\in {\mathcal {P}}\mid p (x)=0\}$. 
\end{frame}

\begin{frame}{Few Definitions}
  \textbf{Contingent claim}: is a derivative whose future payoff depends on the value of another “underlying” asset, or more generally, that is dependent on the realization of some uncertain future event $(S, X\ldots)$.\newline
  
  \textbf{Filtrations}: are totally ordered collections of subsets that are used to model the information that is available at a given point in time ($\mathcal{F}_t$). \newline
  
  \textbf{Martingale}: is a stochastic process for which, at a particular time, the conditional expectation of the next value in the sequence is equal to the present value, regardless of all prior values. It can be imagined as a drift-less process.
\end{frame}

\begin{frame}{Real World Measure $\mathcal{P}$}
  \begin{itemize}
  \item<1-> When we model derivative prices, we take as given some "true" probability measure $\mathcal{P}$, which assigns probabilities to different states of the world. 
  \item<2-> These states in turn affect the path of security prices. 
  \item<3-> And these states plus the corresponding probabilities are supposed to reflect the subjective beliefs of traders or investors about what will happen in the future.
  \item<4-> Unfortunately, under $\mathcal{P}$ it is usually quite complicated to price derivatives, and the probabilities themselves cannot easily be derived.
  \item<5-> This makes it hard to work out the price processes and it is necessary to use simulation techniques.
  \end{itemize}
\end{frame}

\begin{frame}{Changing Measure}
  \begin{itemize}
  \item<1-> Then it makes sense to look for a new set of probabilities (measure) which could simplify the pricing process giving the correct result at the same time.
    %(Actually, no-arbitrage constructions or the Feynman-Kac formula will give you an explicit PDE whose solution is $C_t$, which will not generally be analytical solvable.)
  \item<2-> That's why \textcolor{red}{changes of probability measure are important in mathematical finance}: allow to express derivative prices in closed-form.
  \item<3-> At the beginning of the course we have seen how by simply assuming the absence of arbitrage it is possible to define a \textcolor{red}{new measure} under which the price of a derivative is given by the discounted expectation of its payoff.
  \item<4-> This result has been formalized by \emph{Harrison} and \emph{Pliska} in 1981. 
  \end{itemize}
\end{frame}

\begin{frame}{Equivalent Martingale Measure}
  \begin{block}{Definition}
    An \textcolor{red}{equivalent martingale measure} $\mathcal{Q}$ is a probability measure on the space $\Omega$ such that
    \begin{enumerate}
    \item $\mathcal{Q}$ is equivalent to $\mathcal{P}$;
    \item for any asset $A$ and for each time $t$, $0\le t\le T$ there exists a price $\pi_t$
	\begin{equation*}
	  \pi_t = \mathbb{E}^{\mathcal{Q}^0}[D(t,T)V_A|\mathcal{F}_t]
	\end{equation*}
      %\item the \textcolor{red}{Radon-Nikodym} derivative is square integrable;
    \item the "discounted asset price" is a $\mathcal{Q}$-martingale
      \begin{equation*}
	\pi_u = \mathbb{E}^{\mathcal{Q}^0}[D(0,t)V_A(t)|\mathcal{F}_u], \quad\text{with }(t>u)
      \end{equation*}
    \end{enumerate}
  \end{block}
\end{frame}

\begin{frame}{Fundamental Results Summary}
  Harrison and Pliska proved and formalized also the following results:
  \begin{itemize}
  \item the market is free of arbitrage if and only if there exists an equivalent martingale measure;
  \item the market is complete if and only if the martingale measure is unique;
  \item in an arbitrage-free market the price of any claim is uniquely given, either by the value of an associated replicating strategy, or by the expectation of the discounted payoff under any of the equivalent martingale measures.
  \end{itemize}
\end{frame}

\subsection{Numeraires}
\begin{frame}{Numeraires}
  \begin{itemize}
  \item<1-> Ideally we would like to define a probability measure $\mathcal{Q}^X$ under which the discounted derivative process IS a martingale, such that the expectation of the payoff is analytically tractable (i.e. easy to compute).
  \item<2-> The question that arises is: how can we determine such measure $\mathcal{Q}^X$ ?
  \item<3-> The answer is through \textcolor{red}{change of numeraire}.
  \item<4-> A \textcolor{red}{numeraire} is any strictly positive stochastic process $N_t$ that is taken as a unit of reference when pricing an asset $S_t$
    \begin{equation*}
      \tilde{S_t}:=\frac{S_t}{N_t}, \quad t \ge 0
    \end{equation*}
  \end{itemize}
\end{frame}

\begin{frame}{Numeraires}
  \begin{itemize}
  \item We may compute asset values w.r.t. USD, EUR or JPY.
  \item Others might prefer use commodities: 1~oz of gold could be a numeraire.
  \item In any case, once we choose a numerarie e.g. 1 USD, we determine the value of other assets:
    \begin{columns}
    	\column{0.5\linewidth}
      \includegraphics[width=0.9\linewidth]{usd_eur}
    	\column{0.5\linewidth}
      \includegraphics[width=0.9\linewidth]{usd_chy}
    \end{columns}
  \end{itemize}
\end{frame}

\begin{frame}{Numeraires}
  \begin{itemize}
  \item Of course, in practice there are reasons to prefer gold to other commodities e.g. corn, live cattle, or one currency with respect to another (i.e. political reasons)\ldots
  \item But intuitively there should be no theoretical reason (at least at the scale of investors) to measure value in gold or USD (there used to be also the ``gold standard'')
  \item Basically we want to exploit this fact in our financial models.
  \end{itemize}
\end{frame}

\begin{frame}{Numeraires}
  \begin{itemize}
  \item<1-> \textcolor{red}{Deterministic numeraires} are easy to handle as they imply just an algebraic transformation (i.e. do not involve any risk),
    \begin{itemize}
    \item the exchange rate of the Italian Lira to the Euro was locked at EUR 1 = ITL 1936.27 on 31 December 1998.
    \end{itemize}
  \item<2-> When the numeraire is a stochastic process, and we want to move to it, the pricing of a claim has to be changed in order to take into account the new risks (the intrinsic randomness of the new numeraire).
  \item<3-> In particular a change of numeraire implies also a change in the measure (probability distribution). Indeed (starting with the bank account numeraire)
    \begin{equation*}
      \begin{cases}
        \cfrac{S_1}{B_t} = \cfrac{\sum_i S^i_1 p_i}{e^{rt}}\\
        \cfrac{S_2}{B_t} = \cfrac{\sum_i S^i_2 p_i}{e^{rt}}
      \end{cases}\implies
      \frac{S_1}{S_2} = \frac{\sum_i S^i_1 p_i}{\sum_j S^j_2 p_j} = \sum_i S_1^i \frac{p_i}{\sum_j S^j_2 p_j} = \sum_i S_1^1 \pi_i
    \end{equation*}
  \end{itemize}
\end{frame}

\begin{frame}{Numeraire Examples (I)}
  \begin{itemize}
  \item<1-> \textbf{Money market account.} Given $r_t$, a possibly random and time dependent risk-free interest rate process, let
    \begin{equation*}
      N_t := \exp\left(\int_0^t r_s ds\right)
    \end{equation*}
    In this case 
    \begin{equation*}
      \tilde{S_t}:=\frac{S_t}{N_t}=e^{-\int_0^t r_s ds}S_t, \quad t \ge 0
    \end{equation*}
    represents the discounted price of the asset at time 0.
  \item<2-> \textbf{Currency exchange rate.} In this case $N_t := R_t$ denotes e.g. the EUR/SGD exchange rate. Let
    \begin{equation*}
      \tilde{S_t}:=\frac{S_t}{R_t}, \quad t \ge 0
    \end{equation*}
    denotes the price of a local (SG) asset quoted in units of the foreign currency (EUR) (notice the difference with previous ITL/EUR example above).
  \end{itemize}
\end{frame}

\begin{frame}{Numeraire Examples (II)}
  \begin{itemize}
  \item<1-> \textbf{Forward numeraire.} The price $P(t,T)$ of a bond paying $P(T,T)=1$ at maturity $T$. In this case
    \begin{equation*}
      N_t := P(t,T)=\mathbb{E}\left[e^{\int_t^T r_s ds}\right]
    \end{equation*}
    %Notice that the process $t\rightarrow e^{\int_0^t r_s ds} P(t,T)=\mathbb{E}\left[e^{\int_0^T r_s ds}\right], \quad 0 \le t \le T$ is a martingale.
  \item<2-> \textbf{Annuity numeraire.} Processes of the form
    \begin{equation*}
      N_t = P(t, T_0, T_n) := \sum_{k=1}^{n}(T_k - T_{k-1})P(t, T_k), \quad 0 \le t \le T
    \end{equation*}
    where $P(t,T_1),P(t,T_2),\ldots,P(t,T_n)$ are bond prices with maturities $T_1 < T_2 < \ldots < T_n$.
  \end{itemize}
\end{frame}

\subsection{Radon-Nikodym Derivative}
\begin{frame}{Radon-Nikodym Derivative}
  \begin{itemize}
  \item Now we need to understand how to pass from a numeraire to another, and hence by a measure to another, in an arbitrage free setting.
  \item Notice that until now we have implicitly assumed the bank account $B$ as numeraire.
  \end{itemize}
	\pause
  \begin{block}{Definition}
    When two measures are equivalent it is possible to express the first in terms of the second through the \textcolor{red}{Radon-Nikodym derivative}. Indeed there exists a \textcolor{red}{martingale process $\zeta_t$} such that
    \begin{equation*}
      \mathcal{Q}^* =\int_{A} \zeta_t(\omega)d\mathcal{Q}(\omega)
    \end{equation*}
    which can be written in a more concise form as
    \begin{equation}
      \frac{d\mathcal{Q}^*}{d\mathcal{Q}} = \zeta_t
      \label{eq:radon_nikodym_der}
    \end{equation}
  \end{block}
\end{frame}

\begin{frame}{Intuition from Expected Value}
  \begin{itemize}
  \item To get a sense of what a Radon-Nikodym derivative is we can write the expected value of a generic function $\Pi(x)$ under a measure $\mathcal{F}$, with associated density function $f(x)$ as
    \begin{equation*}
      \mathbb{E}^\mathcal{F}=\int\Pi(x)f(x)dx
    \end{equation*}
  \item Suppose there exists a function $g(x)$, which satisfies the mathematical conditions required to be a density function. Then we can write
    \begin{equation*}
      \mathbb{E}^\mathcal{F}=\int\Pi(x)f(x)\frac{g(x)}{g(x)}dx
    \end{equation*}
  \item If we define $\psi(x)=\Pi(x)\frac{f(x)}{g(x)}$ the expected value can be written as 
    \begin{equation*}
      \mathbb{E}^\mathcal{F}[\Pi(x)]=\int\psi(x)g(x)dx=\mathbb{E}^\mathcal{G}[\psi(x)]=\mathbb{E}^\mathcal{G}\left[\Pi(x)\frac{f(x)}{g(x)}\right]
    \end{equation*}
  \end{itemize}
\end{frame}

\begin{frame}{Radon-Nikodym Derivative}
  \begin{itemize}
  \item The expectations corresponding to the two measures are related by
    \begin{equation*}
      \mathbb{E}^{\mathcal{Q}^*}[X] = \mathbb{E}^\mathcal{Q}\left[X\frac{d\mathcal{Q}^*}{d\mathcal{Q}}\right]
    \end{equation*}
	\pause
  \item In case of a conditioned expectation
    \begin{equation}
      \mathbb{E}^*[X|\mathcal{F}_t] = \frac{\mathbb{E}\left[X\cfrac{d\mathcal{Q}^*}{d\mathcal{Q}}\bigg|\mathcal{F}_t\right]}{\mathbb{E}[\zeta_t|\mathcal{F}_t]}
      \label{eq:conditioned_expectation}
    \end{equation}
    which is an "equivalent" formulation of the famous \emph{Bayes theorem}
    \begin{equation*}
      P(A|B)=\frac{P(B|A)P(A)}{P(B)}
    \end{equation*}
  \end{itemize}
\end{frame}

\subsection{Change of Numeraire}
\begin{frame}{Change of Numeraire}
  \begin{block}{Theorem}
    Assume exists a numeraire $N_t$ and the associated measure $\mathcal{Q}^N$, equivalent to $\mathcal{P}$, such that the price of any traded asset $S_t$ relative to $N$ is a martingale under $\mathcal{Q}^N$
    \begin{equation*}
      \frac{S_t}{N_t} = \mathbb{E}^N\left[\frac{S_T}{N_T}\bigg|\mathcal{F}_t\right],\quad 0\le t \le T
    \end{equation*}
    Let $U$ be another arbitrary numeraire. Then there exists a measure $\mathcal{Q}^U$, also equivalent to $\mathcal{P}$, such that the price of any traded asset $S_t$, normalized to $U$, is a martingale under $\mathcal{Q}^U$
    \begin{equation*}
      \frac{S_t}{U_t} = \mathbb{E}^U\left[\frac{S_T}{U_T}\bigg|\mathcal{F}_t\right],\quad 0\le t \le T
    \end{equation*}
  \end{block}
\end{frame}	

\begin{frame}{Change of Numeraire}
  \begin{block}{...continued}
    The Radon-Nikodym derivative defining the measure $\mathcal{Q}^U$ is given by
    \begin{equation}
      \frac{d\mathcal{Q}^U}{d\mathcal{Q}^N} = \frac{U_T N_0}{U_0 N_T}
      \label{eq:radon_nikodym_der2}
    \end{equation}
  \end{block}
\end{frame}

\begin{frame}{Change of Numeraire (Proof p.2)}
  Let's prove first this second part.
  By definition of $\mathcal{Q}^N$, for any asset price $S_t$ holds
  \begin{equation*}
    \begin{cases}
      \cfrac{S_0}{N_0} = 
      \mathbb{E}^N\left[\cfrac{S_T}{N_T}\right] \\
      \cfrac{U_0}{N_0}\mathbb{E}^U\left[\cfrac{S_T}{U_T}\right] = \cfrac{\cancel{U_0} S_0}{N_0 \cancel{U_0}} = \cfrac{S_0}{N_0}
    \end{cases}\implies \mathbb{E}^N\left[\frac{S_T}{N_T}\right] = \mathbb{E}^U\left[\frac{U_0 S_T}{N_0 U_T}\right]
  \end{equation*}
  since both equal $S_0/N_0$. 
\end{frame}

\begin{frame}{Change of Numeraire (Proof p.2)}
  Also, by definition of Radon-Nikodym derivative
  \begin{equation*}
    \mathbb{E}^N\left[\frac{S_T}{N_T}\right] = \mathbb{E}^U\left[\frac{S_T}{N_T} \frac{d\mathcal{Q}^N}{d\mathcal{Q}^U}\right]
  \end{equation*}
	\pause
  But from the previous result
  \begin{equation*}
    \mathbb{E}^N\left[\frac{S_T}{N_T}\right] = \mathbb{E}^U\left[\frac{U_0 S_T}{N_0 U_T}\right]
  \end{equation*}
  The expectation arguments under $U$ must equal, so we get \cref{eq:radon_nikodym_der2}. 
  \pause
  
  This last equation shows that \textcolor{red}{the risk-neutral price is invariant under change of numeraire.}
    \begin{equation*}
    Price_0 = \mathbb{E}^N\left[\frac{N_0 S_T}{N_T}\right] = \mathbb{E}^U\left[\frac{U_0 S_T}{U_T}\right]
  \end{equation*}
  
\end{frame}

\begin{frame}{Change of Numeraire (Proof p.1)}
  \begin{itemize}
  \item<1-> Now we can prove the first part of the change of numeraire theorem. The conditional expectation formula~\cref{eq:conditioned_expectation} gives
    \begin{equation*}
      \mathbb{E}^U\left[\cfrac{S_T}{U_T}\bigg|\mathcal{F}_t\right]=\cfrac{\mathbb{E}^N\left[\cfrac{d\mathcal{Q}^U}{d\mathcal{Q}^N}\cfrac{S_T}{U_T}\bigg|\mathcal{F}_t\right]}{\mathbb{E}^N\left[\cfrac{d\mathcal{Q}^U}{d\mathcal{Q}^N}\bigg|\mathcal{F}_t\right]}
    \end{equation*}
  \item<2-> But 
    \begin{equation*}
      \begin{cases}
	\mathbb{E}^N\left[\cfrac{d\mathcal{Q}^U}{d\mathcal{Q}^N}\cfrac{S_T}{U_T}\bigg|\mathcal{F}_t\right]= \mathbb{E}^N\left[\cfrac{\cancel{U_T} N_0}{N_T U_0}\cfrac{S_T}{\cancel{U_T}}\bigg|\mathcal{F}_t\right]=\cfrac{N_0 S_t}{U_0 N_t} \\
	\mathbb{E}^N\left[\cfrac{d\mathcal{Q}^U}{d\mathcal{Q}^N}\bigg|\mathcal{F}_t\right]= \mathbb{E}^N\left[\cfrac{U_T N_0}{N_T U_0}\bigg|\mathcal{F}_t\right] = \cfrac{N_0 U_t}{U_0 N_t}
      \end{cases}\implies
      \frac{S_t}{\cancel{N_t}}=\mathbb{E}^U\left[\cfrac{S_T}{U_T}\bigg|\mathcal{F}_t\right]\frac{U_t}{\cancel{N_t}}
    \end{equation*}
  \end{itemize}
\end{frame}	

\begin{frame}{Change of Numeraire Remarks}
  \begin{itemize}
  \item The powerful theorem we have just proved allows to
    \begin{itemize}
    \item find a characterization of our process by means of which we can work-out more easily the fundamental pricing formula. In particular allows to find a measure associated to the new numeraire such that \textcolor{red}{the price of any asset divided by that numeraire is a martingale};
    \item give a simple rule to write (the otherwise difficult to derive) Radon-Nikodym derivative;
    \item show that \textcolor{red}{the risk-neutral price is invariant under change of numeraire.}
    \end{itemize}
  \end{itemize}
\end{frame}

\begin{frame}{Asset Price divided by Numeraire}
	\begin{itemize}
 	\item<1-> Let $B$ be the money bank numeraire and $\mathcal{Q}^B$ the corresponding risk-neutral measure. Also let $N$ be another numeraire (note that $N/B$ is a $\mathcal{Q}^B$-martingale). 
	\item<2-> From the Change of Numeraire Theorem we can define a new measure by mean of
  \begin{equation*}
    \frac{d\mathcal{Q}^N}{d\mathcal{Q}^B} = \frac{N_TB_0}{B_TN_0}
  \end{equation*}
	\item<3-> Then, for any asset $S$ such that $S/B$ is a $\mathcal{Q}^B$-martingale
  \begin{equation*}
    \mathbb{E}^N\left[\frac{S_T}{N_T}\bigg|\mathcal{F}_t\right] = \cfrac{\mathbb{E}^B\left[\cfrac{S_T}{N_T}\cfrac{N_TB_0}{B_TN_0}\bigg|\mathcal{F}_t\right]}{\mathbb{E}^B\left[\cfrac{N_TB_0}{B_TN_0}\bigg|\mathcal{F}_t\right]}
    =\cfrac{\mathbb{E}^B\left[\cfrac{S_T}{B_T}\bigg|\mathcal{F}_t\right]}
    {\mathbb{E}^B\left[\cfrac{N_T}{B_T}\bigg|\mathcal{F}_t\right]}
    =\frac{S_tB_t}{N_tB_t}=\frac{S_t}{N_t}
  \end{equation*}
	\item<4-> \textcolor{red}{So $S/N$ is a $\mathcal{Q}^N$-martingale.}
	\end{itemize}
\end{frame}

\subsection{Applications}
\begin{frame}{Examples}
  \footnotesize{\tiny {\tiny }}{
    \begin{table}[bt]
      \renewcommand*{\arraystretch}{1.4}
      \begin{tabular}{|l|l|} \hline
	\begin{tabular}{@{}l@{}}
	  Any asset divided by the bank account
	  $B_t$\\(recall $dB_t = r_t B_t dt$)
	  \boxed{\cfrac{S_t}{B_t} = e^{-\int_0^t r_s ds}S_t}
	\end{tabular}
	& \begin{tabular}{l}
	    It is a martingale under the\\
	    measure $Q^B$ associated to \\
	    the bank account numeraire,\\
  	    i.e. the risk neutral measure.
	  \end{tabular} \\ \hline
	\begin{tabular}{@{}l@{}}
	  The forward rate\\
	  \boxed{F(t; T_1, T_2) = \frac{1}{T_2-T_1}\left(\frac{P(t,T_1) - P(t,T_2)}{P(t,T_2)}\right)}\\
	  can be interpreted as a portfolio of two ZCBs\\
	  divided by another ZCB.		
	\end{tabular}
	& \begin{tabular}{l}
	    Under the measure $\mathcal{Q}^2$\\ 
	    associated to the numeraire\\ 
	    $P(\cdot,T_2)$ it is a martingale.\end{tabular}\\ \hline  
	\begin{tabular}{@{}l@{}}
	  The swap rate
	  \boxed{S_{\alpha,\beta}(t) = \frac{P(t,T_\alpha)-P(t,T_\beta)}{\sum_{i=\alpha+1}^{\beta}\tau_i P(t,T_i)}}
	  \\can be interpreted as a portfolio of two ZCBs\\
	  divided by a portfolio of ZCBs.		
	\end{tabular}
	& \begin{tabular}{l}
	    It is a martingale under the\\
	    measure associated to the\\
	    annuity numeraire.
	  \end{tabular} \\ \hline
      \end{tabular}
  \end{table}}
\end{frame}

\begin{frame}{A Useful Separation}
  \begin{itemize}
  \item<1-> Until now we have used $B(t)$, the money market account, as numeraire. But as we have seen it is natural to look for the most convenient numeraire, which minimizes the mathematical difficulties according to the problem at hand.
  \item<2-> Given a contingent claim whose payoff at time $T$ is $\chi$, we have the following formula for its price $\Pi$
    \begin{equation*}
      \Pi_\chi(t,T)=\mathbb{E}^B\left[e^{-\int_t^T r_s ds}\chi\bigg|\mathcal{F}_t \right]=\mathbb{E}^B\left[e^{\int_0^t r_s ds}e^{-\int_0^T r_s ds}\chi\bigg|\mathcal{F}_t \right]=B_t\mathbb{E}^B\left[B^{-1}_T\chi|\mathcal{F}_t\right]
    \end{equation*}
  \item<3-> If $\chi$ and the short rate process were independent under $\mathcal{Q}^B$ (recall $\mathbb{E}[XY]=\mathbb{E}[X]\mathbb{E}[Y]$) then we could write
    \begin{equation*}
      \Pi_\chi(t,T)=\mathbb{E}^B\left[e^{-\int_t^T r_s ds}\bigg|\mathcal{F}_t\right]\mathbb{E}^B\left[\chi|\mathcal{F}_t\right] = P(t,T)\mathbb{E}^B\left[\chi|\mathcal{F}_t\right]
    \end{equation*}
  \end{itemize}
\end{frame}

\begin{frame}{A Useful Separation}
  \begin{itemize}
  \item<1-> In general the above separation is not possible due to the interaction between the discount factors and the claim payoff. 
  \item<2-> In this, like in other concrete situations, a better numeraire is indeed the ZCB with the same maturity $T$ of the derivative to price (recall $P(T,T)=1)$.
  \item<3-> The \textcolor{red}{forward measure $\mathcal{Q}^T$ (also called the $T$-measure)} is defined as the martingale measure for the numeraire process $P(\cdot,T)$, the ZCB maturing in T indeed, i.e. what we called $\mathcal{Q}^2$ in the example above.
  \item<4-> It is easy to see that using \cref{eq:radon_nikodym_der2}, the Radon-Nykodim derivative is given in this case by
    \begin{equation}
      \zeta_t = \frac{d\mathcal{Q}^T}{d\mathcal{Q}^B} = \frac{P(t,T)\overbrace{B(0)}^{=1}}{B_t P(0,T)} ,\quad\left(\zeta_T=\frac{\overbrace{P(T,T)}^{=1}B(0)}{B(T)P(0,T)}=\frac{1}{B(T)P(0,T)}\right)
      \label{eq:radon_nikodym_t_forward}
    \end{equation}
  \end{itemize}
\end{frame}

\begin{frame}{A Useful Separation}
  \begin{itemize}
  \item<1-> Applying the change of numeraire to the pricing formula, we get
    \begin{equation*}
      \begin{aligned}
	\Pi_\chi(t,T) & = B_t\mathbb{E}^B\left[B^{-1}_T\chi|\mathcal{F}_t\right] \\
	& = B_t\mathbb{E}^B\left[P(0,T)\zeta_T\chi|\mathcal{F}_t\right]\quad\text{(using } B_T^{-1} = \zeta_TP(0,T))\\
	& = B_tP(0,T)\mathbb{E}^B\left[\zeta_T|\mathcal{F}_t\right]\mathbb{E}^T\left[\chi|\mathcal{F}_t\right]\quad\text{(by \cref{eq:conditioned_expectation})}\\
	& = \cancel{B_tP(0,T)}\frac{P(t,T)}{\cancel{B_tP(0,T)}}\mathbb{E}^T\left[\chi|\mathcal{F}_t\right] \\
	& = P(t,T)\mathbb{E}^T\left[\chi|\mathcal{F}_t\right] \\
      \end{aligned}
    \end{equation*}
    which achieves the desired separation (although under a new measure).
  \item<2-> Clearly this kind of transformation is useful when $\chi$ has known dynamics under the forward measure.
  \end{itemize}
\end{frame}

\begin{frame}{Identity between $\mathcal{Q}^B$ and $\mathcal{Q}^T$}
  By construction of the martingale measure $\mathcal{Q}^B$, the following relationship holds
  \begin{equation*}
    \begin{gathered}
      \frac{P(t,T)}{B_t}=\mathbb{E}^B\left[\frac{P(T,T)}{B_T}\right]\\[0.3cm]
      P(t,T)=\mathbb{E}^B\left[\frac{P(T,T)}{B_T}B_t\right] = \mathbb{E}^B\left[\frac{B_t}{B_T}\right]
    \end{gathered}
  \end{equation*}
	\pause
  Plugging the result into the Radon-Nikodym derivative gives
  \begin{equation*}
    \frac{d\mathcal{Q}^T}{d\mathcal{Q}^B} = \frac{B_t}{B_T}\frac{1}{P(t,T)} =\frac{B_t/B_T}{\mathbb{E}^B[B_t/B_T]}
  \end{equation*}
	\pause
  \begin{block}{Proposition}
    If interest rates are deterministic (i.e. the Radon-Nikodym derivative is 1), then the measures $\mathcal{Q}^B$ and $\mathcal{Q}^T$ are identical.
  \end{block}
\end{frame}

\begin{frame}{Clarification on Time}
  \begin{itemize}
  \item Clearly as the Radon-Nikodym derivative is a martingale for valuation time $t$, we have
    \begin{equation*}
      \frac{d\mathcal{Q}^U}{d\mathcal{Q}^N}=\frac{U_tN_0}{U_0N_t}
    \end{equation*}
  \item Do not confuse the maturity of the numeraire bond $T$ with the times at which you have to take the values of the numeraire, in this case $t$ and 0.
  \item If you want to switch from the $T$ measure to the $S$ measure, i.e. the one induced by the bond $P(.,S)$, for the valuation time $t$ we get
    \begin{equation*}
      \frac{d\mathcal{Q}^S}{d\mathcal{Q}^T}=\frac{P(t,S)P(0,T)}{P(t,T)P(0,S)}
    \end{equation*}
  \end{itemize}
\end{frame}

\begin{frame}{The Forward Rate Under $\mathcal{Q}^T$}
  \begin{block}{Proposition}
    Consider the forward numeraire $P(t,T)$ and denote with $\mathcal{Q}^T$ its associated measure.
    The forward rate spanning the interval $[S,T]$ is the $\mathcal{Q}^T$-expectation of the future spot rate at time $S$ for the maturity $T$
    \begin{equation}
      F(t;S,T) =\mathbb{E}^T[L(S,T)|\mathcal{F}_t]
    \end{equation}
  \end{block}
\end{frame}

\begin{frame}{The Forward Rate Under $\mathcal{Q}^T$ (Proof)}
  \begin{equation*}
    \begin{gathered}
      F(t;S,T) = \frac{1}{\tau}\left[\frac{P(t,S)-P(t,T)}{P(t,T)}\right] \\[0.3cm]
      F(t;S,T)P(t,T) = \frac{P(t,S)-P(t,T)}{\tau}
    \end{gathered}
  \end{equation*}

  This is the price at time $t$ of an asset (difference of two bonds). Therefore by the change of numeraire theorem and by definition of forward measure
  \begin{equation*}
    \frac{F(t;S,T)P(t,T)}{P(t,T)} = F(t,S,T)
  \end{equation*}
  is a \textcolor{red}{martingale} under $\mathcal{Q}^T$-measure. \pause Hence
  \begin{equation*}
    F(t;S,T) = \mathbb{E}^T[F(S;S,T)|\mathcal{F}_t] = \mathbb{E}^T\left[\frac{1}{\tau}\left(\frac{1-P(S,T)}{P(S,T)}\right)\bigg|\mathcal{F}_t\right] = \mathbb{E}^T[L(S,T)|\mathcal{F}_t]
  \end{equation*}
\end{frame}

\begin{frame}{The Forward Rate Under $\mathcal{Q}^T$}
  A similar result can be derived for the corresponding instantaneous quantities
  \begin{equation}
    f(t,T)=\mathbb{E}^T[r_t|\mathcal{F}_t]
  \end{equation}
	\pause
  Indeed from the definition of $\mathcal{Q}^B$
  \begin{equation*}
    \frac{P(t,T)}{B_t}=\mathbb{E}^B\left[\frac{P(T,T)}{B_T}\bigg|\mathcal{F}_t\right]
  \end{equation*}
  but $P(T,T)=1$ so
  \begin{equation*}
    P(t,T)=\mathbb{E}^B\left[\frac{B_tP(T,T)}{B_T}\bigg|\mathcal{F}_t\right]=\mathbb{E}^B\left[e^{-\int_t^Tr_u du}\big|\mathcal{F}_t\right]
  \end{equation*}
	\pause
  Differentiating with respect to $T$ ($\frac{d}{dx}\int_c^x f(t)dt=f(x)$)
  \begin{equation*}
    \frac{\partial P(t,T)}{\partial T}=
    \mathbb{E}^B\left[-r(T)e^{-\int_t^Tr_u du}\big|\mathcal{F}_t\right]
  \end{equation*}
\end{frame}

\begin{frame}{The Forward Rate Under $\mathcal{Q}^T$}
  Now we can change numeraire to $P(\cdot,T)$ so that, using reciprocal of \cref{eq:radon_nikodym_t_forward} ($\zeta^{-1}=\frac{B_t/B_T}{P(t,T)/P(T,T)}$)
  \begin{equation*}
    \frac{\partial P(t,T)}{\partial T}=
    \mathbb{E}^T\left[-r(T)\cancel{e^{-\int_t^Tr_u du}}\frac{P(t,T)}{\cancel{e^{-\int_t^Tr_u du}}}\bigg|\mathcal{F}_t\right]=
    P(t,T)\mathbb{E}^T\left[-r(T)|\mathcal{F}_t\right]
  \end{equation*}
	\pause
  Hence
  \begin{equation*}
    \begin{aligned}
      f(t,T)&=\frac{1}{P(t,T)}\frac{\partial P(t,T)}{\partial T}=
      -\frac{\partial \ln P(t,T)}{\partial T}
      = \mathbb{E}^T\left[r(T)|\mathcal{F}_t\right]=	\mathbb{E}^T\left[f(T,T)|\mathcal{F}_t\right]
    \end{aligned}
  \end{equation*}
  Which demonstrates the initial statement and also shows that \textcolor{red}{the instantaneous forward rate is a martingale under the $T$-forward measure}.
\end{frame}

\begin{frame}{The Expectation Hypothesis}
	\begin{itemize}
		\item Previous result
		\begin{equation*}
			f(t, T) = \mathbb{E}^T[r(T)|\mathcal{F}_t]
		\end{equation*}
		has a nice connection with the \textcolor{red}{expectation hypothesis of the term structure of interest rates}.
		\item Its basic idea is that the long-term rate is determined purely by current and future expected short-term rates.
		\item We cannot dive into it, but there are tons of papers on the subject, among which I suggest
		\begin{itemize}
			\item \href{https://pages.stern.nyu.edu/~sternfin/asangvin/ExpHyp.pdf}{\emph{The Expectation Hypothesis, A. Sangvinatsos}}
			\item \href{https://www.jstor.org/stable/2327547}{\emph{A Re-Examination of Traditional Hypotheses about the Term Structure of Interest Rates}, J.C. Cox, J.E. Ingersoll, and S.A. Ross}
			%\item \href{https://www.albany.edu/~bd445/Economics_802_Financial_Economics_Slides_Fall_2013/Risk_Neutrality_and_the_Expectations_Theory.pdf}{\emph{Slides from Fall 2013}}
		\end{itemize}
	%\item According to the pure expectation hypothesis, the above formula is valid if the expected value is taken under the real probability.
	%\item Absence of arbitrage makes this incompatible with stochastic interest rates.
	\end{itemize}
\end{frame}

%The Relationship between Forward Rate over Second Year
%and Spot Rate Expected over Second Year
%Given a forecast of bond B’s price, an investor can choose one of two strategies at date 0:
% 1. Buy a one-year bond. Proceeds at date 1 would be:
% $1,080 = $1,000 x 1.08 (A.10)

% 2. Buy a two-year bond but sell at date 1. Expected proceeds would be:
% $1,000 x (1.10)2
%____________________________
%1 +  Spot rate expected over year 2 (A.11)
% Given our discussion of forward rates, we can rewrite Equation A.11 as:
% $1,000 x 1.08 x 1.1204
%____________________________
%1  Spot rate expected over year 2 (A.12)
% (Remember that 12.04 percent was the forward rate over year 2; that is, f2 = 12.04%.)
% Under what condition will the return from strategy 1 equal the expected return from
%strategy 2? In other words, under what condition will Equation A.10 equal Equation A.12?
% The two strategies will yield the same expected return only when:
% 12.04%  Spot rate expected over year 2 (A.13)
%In other words, if the forward rate equals the expected spot rate, one would expect to earn
%the same return over the first year whether one
%• Invested in a one-year bond.
%• Invested in a two-year bond but sold after one year.


%The Expectations Hypothesis
%Equation A.13 seems fairly reasonable. That is, it is reasonable that investors would set
%interest rates in such a way that the forward rate would equal the spot rate expected by the
%marketplace a year from now.
% For example, imagine that individuals in the marketplace do
%not concern themselves with risk. If the forward rate, f2, is less than the spot rate expected
%over year 2, individuals desiring to invest for one year would always buy a one-year bond.
%That is, our work shows that an individual investing in a two-year bond but planning to sell
%at the end of one year would expect to earn less than if he simply bought a one-year bond.
% Equation A.13 was stated for the specific case where the forward rate was 12.04 percent.
%We can generalize this as follows:
%Expectations Hypothesis:
% f2 = Spot rate expected over year 2 (A.14)
%Equation A.14 says that the forward rate over the second year is set to the spot rate that
%people expect to prevail over the second year. This is called the expectations hypothesis. It
%states that investors will set interest rates such that the forward rate over the second year is
%equal to the one-year spot rate expected over the second year.


%Liquidity Preference Hypothesis
%At this point, many students think that Equation A.14 must hold. However, note that we
%developed Equation A.14 by assuming that investors were risk-neutral. Suppose, alternatively, that investors are averse to risk.
% Which strategy would appear more risky for an individual who wants to invest for one year?
% 1. Invest in a one-year bond.
% 2. Invest in a two-year bond but sell at the end of one year.

%Strategy 1 has no risk because the investor knows that the rate of return must be r1. Conversely,
%strategy 2 has much risk: The fi nal return is dependent on what happens to interest rates.
% Because strategy 2 has more risk than strategy 1, no risk-averse investor will choose
%strategy 2 if both strategies have the same expected return. Risk-averse investors can have
%no preference for one strategy over the other only when the expected return on strategy
%2 is above the return on strategy 1. Because the two strategies have the same expected
%return when f2 equals the spot rate expected over year 2, strategy 2 can have a higher rate
%of return only when the following condition holds:
%Liquidity Preference Hypothesis:
% f2 >  Spot rate expected over year 2 (A.15)

%That is, to induce investors to hold the riskier two-year bonds, the market sets the forward
%rate over the second year to be above the spot rate expected over the second year. Equation A.15 is called the liquidity preference hypothesis.
% We developed the entire discussion by assuming that individuals are planning to invest
%over one year. We pointed out that for these types of individuals, a two-year bond has extra
%risk because it must be sold prematurely. What about individuals who want to invest for two
%years? (We call these people investors with a two-year time horizon.)
% They could choose one of the following strategies:
% 3. Buy a two-year zero coupon bond.
% 4. Buy a one-year bond. When the bond matures, immediately buy another one-year bond.
% Strategy 3 has no risk for an investor with a two-year time horizon because the proceeds
%to be received at date 2 are known as of date 0. However, strategy 4 has risk because the
%spot rate over year 2 is unknown at date 0. It can be shown that risk-averse investors will
%prefer neither strategy 3 nor strategy 4 over the other when:
% f2 < Spot rate expected over year 2 (A.16)

% Note that the assumption of risk aversion gives contrary predictions. Relationship A.15
%holds for a market dominated by investors with a one-year time horizon. Relationship A.16
%holds for a market dominated by investors with a two-year time horizon. Financial economists have generally argued that the time horizon of the typical investor is generally much
%shorter than the maturity of typical bonds in the marketplace. Thus, economists view A.15
%as the best depiction of equilibrium in the bond market with risk-averse investors.
% However, do we have a market of risk-neutral or risk-averse investors? In other words,
%can the expectations hypothesis of Equation A.14 or the liquidity preference hypothesis of
%Equation A.15 be expected to hold? As we will learn later in this book, economists view
%investors as being risk-averse for the most part. Yet, economists are never satisfi ed with a
% casual examination of a theory’s assumptions. To them, empirical evidence of a theory’s
%predictions must be the fi nal arbiter.
% There has been a great deal of empirical evidence about the term structure of interest
%rates. Unfortunately (perhaps fortunately for some students), we will not be able to present
%the evidence in any detail. Suffi ce it to say that, in our opinion, the evidence supports the
% liquidity preference hypothesis over the expectations hypothesis. One simple result might
%give students the fl avor of this research. Consider an individual choosing between one of
%the following two strategies:
% 1. Invest in a one-year bond.
% 2. Invest in a 20-year bond but sell at the end of one year.
%www.mhhe.com/rwj r
%Chapter 5 How to Value Bonds and Stocks 5A-9
% [Strategy 2 is identical to strategy 2, except that a 20-year bond is substituted for a
%2-year bond.]
% The expectations hypothesis states that the expected returns on both strategies are
%iden tical. The liquidity preference hypothesis states that the expected return on strategy
%2 should be above the expected return on strategy 1. Though no one knows what returns
%are actually expected over a particular time period, actual returns from the past may allow
%us to infer expectations. The results from January 1926 to December 1999 are illuminating. The average yearly return on strategy 1 is 3.8 percent and 5.5 percent on strategy 2
%over this time period.7,8 This evidence is generally considered to be consistent with the
% liquidity preference hypothesis and inconsistent with the expectations hypothesis.
%\end{frame}

\subsection{Girsanov Theorem}
\begin{frame}{Which Dynamics ?}
  \begin{itemize}
  \item<1-> We're left with one important question:
    \textcolor{red}{what does the path of an asset price $S_t$ look like under a new measure $\mathcal{Q}$ ?} (we need to know in order to be able to really compute its expectation under $\mathcal{Q}$.)
  \item<2-> \emph{Girsanov's theorem} answers to this question since it tells us, when we change from $\mathcal{P}$ to some other measure $\mathcal{Q}$, how the stochastic part of a process ($W_t$) changes under $\mathcal{Q}$.
  \item<3-> Will see that it evolves as the sum of a Brownian motion under $\mathcal{Q}$ and a drift related to the Radon-Nikodym derivative characterizing $\mathcal{Q}$.
  \item<4-> We therefore want to choose the Radon-Nikodym derivative so that the drift of $W_t$ w.r.t. $\mathcal{Q}$ exactly cancels out the drift of $S_t$, leaving us with a pure diffusion process (martingale). 
  \end{itemize}
\end{frame}

\begin{frame}{Girsanov Theorem}
  \begin{block}{Theorem}
    Consider the SDE 
    \begin{equation*}
      dX_t = f_t dt + \sigma_t dW_t
    \end{equation*}
    under $\mathcal{P}$. 
    
    Let be given a new drift $f^*_t$ and assume $\gamma_t=\frac{f_t^*-f_t}{\sigma_t}$ such that $\mathbb{E}\left[\exp\left(\frac{1}{2}\int_0^t\gamma_t^2dt\right)\right]<\infty$.
    Define the measure 
    \begin{equation}
      \frac{d\mathcal{P}^*}{d\mathcal{P}}=\exp\left(-\frac{1}{2}\int_0^t \gamma_s^2 ds + \int_0^t \gamma_s dW_s \right)
    \end{equation}
    Then $\mathcal{P}^*$ is equivalent to $\mathcal{P}$. 
    The Radon-Nikodym derivative process is an \textcolor{red}{exponential martingale}.
  \end{block}
\end{frame}

\begin{frame}{Girsanov Theorem}
  \begin{block}{Theorem continued}
    Also the process
    \begin{equation}
      dW^*_t = -\gamma_s dt + dW_t
    \end{equation} 
    is a Brownian motion under $\mathcal{P}^*$, and 
    \begin{equation*}
      dX_t = f^*_t dt + \sigma_t dW^*_t
    \end{equation*}
    The condition $\mathbb{E}\left[\exp\left(\frac{1}{2}\int_0^t\gamma_t^2dt\right)\right]<\infty$ is a sufficient but non-necessary, and it is know as the \textcolor{red}{Novikov condition}.
  \end{block}
\end{frame}

%\begin{frame}{Girsanov Theorem (Proof)}
%  \begin{itemize}
%  \item We have already seen that the solution of the SDE $dS_t=\mu S_t dt + \sigma S_t dW_t$ is
%    \begin{equation*}
%      \frac{S_t}{S_0} = e^{(\mu-\frac{1}{2}\sigma^2)t+\sigma W_t}
%    \end{equation*}
%  \item If we now assume the drift coefficient $\mu=0$ the solution becomes
%    \begin{equation*}
%      \frac{S_t}{S_0} = e^{(-\frac{1}{2}\sigma^2)t+\sigma W_t} \approxeq \frac{d\mathcal{P}^*}{d\mathcal{P}}
%    \end{equation*}
%  \item So the Radom-Nikodym is a solution of a driftless process hence is a martingale.
%  \end{itemize}
%\end{frame}

\begin{frame}{An Example}
  \begin{itemize}
  \item Consider the stochastic differential equation
    \begin{equation*}
      dX_t = b(X_t, t) dt + a(X_t, t) dW_t
    \end{equation*}
  \item Let's assume that the drift and diffusion coefficients are such that there exists a unique solution to the equation which is $X$.
  \item We want to find a probability measure $\mathcal{P}^*$ such that the drift of $X$ is $\tilde{b}(X_t,t)$ instead of $b(X_t,t)$.
  \end{itemize}
\end{frame}

\begin{frame}{An Example}
  \begin{equation*}
    \begin{aligned}
      dX_t &= \tilde{b}(X_t,t) dt+b(X_t,t) dt -\tilde{b}(X_t,t) dt + a(X_t,t) dW_t = \\
      &=\tilde{b_t} dt + (b_t -\tilde{b_t})dt + a(X_t,t) dW_t =\\
      &=\tilde{b_t}dt+ a_t\overbrace{\left(\frac{b_t-\tilde{b_t}}{a_t}\right)}^{-\gamma_t}dt + a_t dW_t = \\
      &= \tilde{b_t}dt+a_t dW_t - a_t\gamma_t dt\\
      &=\tilde{b_t}dt+a_t d\tilde{W_t}
    \end{aligned}
  \end{equation*}
  where $d\tilde{W_t}=dW_t-\gamma_t dt$.
\end{frame}

\begin{frame}{An Example}
  \begin{itemize}
  \item<1-> If the Novikov condition is satisfied then we can apply the Girsanov theorem and we have that
    \begin{equation}
      \mathcal{P}^* = \mathbb{E}^\mathcal{P}\left[\exp\left(-\frac{1}{2}\int_0^t \gamma_s^2 ds + \int_0^t \gamma_s dW_s \right)\right]
    \end{equation}
    and that $\tilde{W}$ is a Brownian motion on $\mathcal{P}^*$.
  \item<2-> In practice, don't need to determine the new measure $\mathcal{P}^*$.
  \item<3-> It is enough to know it exists, such that we can work with the resulting SDE of the process of interest under the new measure.	
  \begin{tikzpicture}[remember picture,overlay]
	\node[xshift=5cm,yshift=-3.9cm] (image) at (current page.center) {\includegraphics[width=20px]{python_logo}};
	\node[align = center, yshift=1.45cm, below=of image] {\tiny{\href{shorturl.at/ctCF7}{shorturl.at/ctCF7}}};
\end{tikzpicture}
  \end{itemize}
\end{frame}

%\begin{frame}{title}
%\begin{itemize}
%	\item Let $(\Omega,\mathcal{F}_t, \mathbb{P})$ be a probability space with a standard Brownian motion $W^{\mathbb{P}}$.
%	\item The stochastic process $S_t$ represents the evolution of a risky security price satisfying stochastic differential equation (SDE)
%	\begin{equation*}
%		dS_t = \mu S_t dt + \sigma S_t dW^{\mathbb{P}}_t
%	\end{equation*}
%	\item Let's assume that interest rate $r$ is constant. Therefore
%	\begin{equation*}
%		D(0,t) = e^{-rt}
%	\end{equation*} 	
%	which implies $dD = -re^{-rtdt}$.  
%\end{itemize}
%\end{frame}
%
%\begin{frame}{title}
%	\begin{itemize}
%		\item Define then 
%		\begin{equation*}
%			Y_t = D_t S_t
%		\end{equation*} 
%		that is the present value at time $t$ of the risky security.
%		\item Using Ito's lemma
%		\begin{equation*}
%			\begin{aligned}
%			dY_t &= D_t dS_t + dD_t S_t \\ 
%			&= D_t (\mu S_t dt + \sigma S_t dW^{\mathbb{P}}_t) + S_t (-rD_t dt) \\
%			&= (\mu - r) Y_t dt + \sigma Y_t dW_t^{\mathbb{P}}		
%			\end{aligned}
%		\end{equation*}
%		\item In its integral form it becomes
%		\begin{equation*}
%			Y_t = Y_0 + (\mu - r)\int_0^t Y_s ds + \sigma \int_0^t Y_s dW_s^{\mathbb{P}}
%		\end{equation*}
%	\end{itemize}
%\end{frame}

%\begin{frame}{Numeraire Pricing}
%	\begin{block}{Theorem (German, El Karoui and Rochet, 1995)}
%		Assume that there exists a numeraire $N$ and a probability measure $\mathcal{Q}^N$ which is equivalent to $\mathcal{P}$ such that, for every traded asset $X$:
%		\begin{equation}
%			\frac{X_t}{N_t} = \mathbb{E}^{\mathcal{Q}^N}\left[\frac{X_T}{N_T}|\mathcal{F}_t\right]
%		\end{equation}
%		Now, given a second arbitrary numeraire $U$, there exists a probability measure $\mathcal{Q}^U$ which is equivalent to $\mathcal{P}$ and such that:
%		\begin{equation}
%			\frac{X_t}{U_t} = \mathbb{E}^{\mathcal{Q}^U}\left[\frac{X_T}{U_T}|\mathcal{F}_t\right]
%		\end{equation}
%	\end{block}
%\end{frame}


%\begin{frame}{Example Approach to Numeraire Change}
%\begin{itemize}
%	\item In lieu of the fundamental theorem we can write
%	\begin{equation}
%	\Pi(0,X)=S_0(0)\mathbb{E}^0\left[\frac{X}{S_0(T)}\right]
%	\end{equation}
%	\item But also
%	\begin{equation}
%	\Pi(0,X)=S_1(0)\mathbb{E}^1\left[\frac{X}{S_1(T)}\right]
%	\end{equation}
%	\item We define the Radon-Nikodym derivative
%	\begin{equation}
%	L_0^1(T)=\frac{dQ^1}{dQ^0}
%	\end{equation}
%\end{itemize}
%\end{frame}
%
%\begin{frame}{Example Approach to Numeraire Change}
%	\begin{itemize}
%		\item Hence we can write ?????????????????
%		\begin{equation}
%			\Pi(0,X)=S_1(0)\mathbb{E}^0\left[\frac{X}{S_1(T)}L_0^1(T)\right]
%		\end{equation}
%		\item After some trivial manipulations
%		\begin{equation}
%			S_0(0)\mathbb{E}^0\left[\frac{X}{S_0(T)}\right]=
%			S_1(0)\mathbb{E}^0\left[\frac{X}{S_1(T)}L_0^1(T)\right]
%		\end{equation}
%		\item Finally
%		\begin{equation}
%			\frac{S_0(0)}{S_0(T)}=\frac{S_1(0)}{S_1(T)}L_0^1(T)		
%		\end{equation}
%	\end{itemize}
%\end{frame}
%
%\begin{frame}{Example Approach to Numeraire Change}
%	\begin{itemize}
%		\item Hence 
%		\begin{equation}
%			L_0^1(T) = \frac{dQ^1}{dQ^0}=			\frac{S_0(0)S_1(0)}{S_1(T)S_0(T)}
%		\end{equation}
%
%	\end{itemize}
%\end{frame}

\begin{frame}{Moving Away from $\mathcal{P}$ Measure}
  \begin{itemize}
  \item<1-> Let's go back to the real-world probabilities and assume that a stock price has the following dynamics (Geometric Brownian Motion) under the real-world measure $\mathcal{P}$
    \begin{equation*}
      dS_t = \mu S_t dt + \sigma S_t dW_t
    \end{equation*}
  \item<2-> By definition the bank account dynamics is described by (for simplicity let's consider deterministic rates)
    \begin{equation*}
      dB_t = rB_tdt\implies B_t = e^{rt}
    \end{equation*}
  \item<3-> We want to know what happens to the stock SDE when moving to two different numeraires:
    \begin{itemize}
    \item<4-> risk-neutral measure (bank account numeraire);
    \item<5-> stock measure (stock numeraire).
    \end{itemize}
  \end{itemize}
\end{frame}

\begin{frame}{Risk-Neutral Measure Dynamics}
  \begin{itemize}
  \item<1-> We have seen that under the bank account induced measure the process defined as an asset divided by the numeraire is a martingale
    \begin{equation*}
      \cfrac{S_t}{B_t} = \mathbb{E}^B\left[\frac{S_T}{B_T}\bigg|\mathcal{F}_t\right]
    \end{equation*}
  \item<2-> So if we define $Z_t=\cfrac{S_t}{B_t}$, since $Z_t$ is a martingale (no-drift process) it's evolution could be described by
    \begin{equation}
      dZ_t = \sigma Z_t dW_t^B
      \label{eq:z_martingale1}
    \end{equation}
    where $dW_t^B$ is a Brownian motion under the $\mathcal{Q}^B$ measure.
  \end{itemize}
\end{frame}

\begin{frame}{Risk-Neutral Measure Dynamics}
  \begin{itemize}
  \item<1-> Computing directly the $Z_t$ differential (by It$\hat{o}$'s rule at first order)
    \begin{equation*}
      \begin{aligned}
	d\left(\frac{S_t}{B_t}\right) &= \frac{dS_t}{B_t} + S_t d\left(\frac{1}{B_t}\right) = \\ 
	&=\frac{dS_t}{B_t} + S_t d\left(e^{-rt}\right) = \frac{dS_t}{B_t} - S_t re^{-rt}dt \\
	&= \frac{dS_t}{B_t} - r\frac{S_t}{B_t}dt 
      \end{aligned}
    \end{equation*}
  \item<2-> Now substitute for $dS_t$
    \begin{equation*}
      d\left(\frac{S_t}{B_t}\right)= \frac{ \mu S_t dt + \sigma S_t dW_t}{B_t} - r\frac{S_t}{B_t}dt = \sigma\frac{S_t}{B_t}\left(\frac{\mu - r}{\sigma}dt + dW_t \right)
    \end{equation*}	
  \end{itemize}
\end{frame}

\begin{frame}{Risk-Neutral Measure Dynamics}
  \begin{itemize}
  \item<1-> In terms of $Z_t$ it becomes
    \begin{equation}
      dZ_t = \sigma Z_t\left(\frac{\mu - r}{\sigma}dt + dW_t \right)
      \label{eq:z_martingale2}
    \end{equation}
  \item<2-> Both \cref{eq:z_martingale2} and~\cref{eq:z_martingale1} represent the dynamics of $Z_t$ so they must be equal
    \begin{equation*}
      \cancel{\sigma Z_t}dW_t^B = \cancel{\sigma Z_t}\left(\frac{\mu - r}{\sigma}dt + dW_t\right)
    \end{equation*}
  \item<3-> Replacing the Brownian Motion into the real-world dynamics
    \begin{equation*}
      \begin{aligned}
	dS_t &= \mu S_t dt + \sigma S_t \left(dW_t^B - \frac{\mu - r}{\sigma}dt\right) =\\
	& = \cancel{\mu S_t dt} \cancel{-\mu S_t dt} + rS_t dt + \sigma S_t dW_t^B  = \boxed{rS_t dt + \sigma S_t dW_t^B}
      \end{aligned}
    \end{equation*}
  \item<4-> So \textcolor{red}{under the risk-neutral measure the drift equals the risk-free rate}.
  \end{itemize}
\end{frame}

\begin{frame}{Stock Numeraire Measure Dynamics}
  \begin{itemize}
  \item<1-> Now let's see what happens under the stock numeraire.
  \item<2-> Under the risk-neutral measure $\mathcal{Q}^B$
    \begin{equation*}
      \frac{S_0}{B_0} = \mathbb{E}^B\left[\frac{S_t}{B_t}\bigg|\mathcal{F}_0\right] \implies
      S_0 = \mathbb{E}^B\left[B_0\frac{S_t}{B_t}\bigg|\mathcal{F}_0\right]
    \end{equation*}
  \item<3-> By the Change of Numeraire Theorem under the measure $\mathcal{Q}^A$ induced by asset numeraire $A$
    \begin{equation*}
      S_0 = \mathbb{E}^A\left[A_0\frac{S_t}{A_t}\bigg|\mathcal{F}_0\right]
    \end{equation*}
  \item<4-> Since both expressions represent a price of an asset they must be the same and we can equal the terms inside the expectations. Note that the expectations are computed according two different measures so we keep the factors $d\mathcal{Q}^X$. 
  \end{itemize}
\end{frame}

\begin{frame}{Stock Numeraire Measure Dynamics}
  \begin{equation*}
    \frac{B_0}{B_t}d\mathcal{Q}^B = \frac{A_0}{A_t}d\mathcal{Q}^A\implies \frac{d\mathcal{Q}^A}{d\mathcal{Q}^B}=\frac{B_0A_t}{B_tA_0}
  \end{equation*}
  \begin{itemize}
  \item<2-> We have already derived the analytical GBM solution in the risk-neutral measure (see pg. 14 to 18 of "no\_arbitrage" slides)
    \begin{equation*} 
      A_t = A_0 \exp\left(rt-\frac{1}{2}\sigma^2 t + \sigma W^B_t\right)
    \end{equation*}
    \item<3-> So we can replace the numeraire definition into the Radon-Nikodym derivative
    \begin{equation*}
      \frac{d\mathcal{Q}^A}{d\mathcal{Q}^B}=\frac{\cancel{A_0}e^{\cancel{rt}-\frac{1}{2}\sigma^2 t + \sigma W^B_t}}{\cancel{e^{rt}}\cancel{A_0}}=\exp\left(-\frac{1}{2}\sigma^2 t + \sigma W^B_t\right)
    \end{equation*}
  \end{itemize}
\end{frame}

\begin{frame}{Stock Numeraire Measure Dynamics}
  \begin{itemize}
  \item<1-> From the Girsanov theorem, setting the function $y_t = \sigma$, we can get the transformed diffusion process
    \begin{equation*}
      dW_t^A = dW_t^B - \sigma dt 
    \end{equation*}
  \item<2-> Substituting back into the risk-neutral dynamics we get
    \begin{equation*}
      \begin{aligned}
	dS_t &= r S_t dt + \sigma S_t dW_t^B = 
	rS_t dt + \sigma S_t (dW_t^A + \sigma dt) \\
	& = (r + \sigma^2)S_t dt + \sigma S_t dW^A_t
      \end{aligned}
    \end{equation*}
  \end{itemize}
  %\vspace{3cm}
%\footnotesize{Exercise: calculate the drift change for the Vasicek Model in case you change from B(t) to P(0,T) numeraire}
\end{frame}

\begin{frame}{Summarizing the Results}
  \begin{itemize}
  \item<1-> To summarize all the results
    \vspace{0.5cm}
    \begin{table}
      \begin{tabular}{llr}
	$(i)$&$dS_t = \textcolor{red}{\mu} S_t dt + \sigma S_t dW_t$ & Real-world measure \\
	$(ii)$&$dS_t = \textcolor{red}{r}S_t dt + \sigma S_t dW_t^B$ & Risk-neutral measure \\
	$(iii)$&$dS_t = \textcolor{red}{(r + \sigma^2)}S_t dt + \sigma S_t dW^A_t$ & Stock measure\\
      \end{tabular}
    \end{table}
    \vspace{0.5cm}
  \item<2-> In practical terms, this means that it possible to use equation $(ii)$, instead of equation $(i)$, to simulate future payoffs, and hence that it is possible to get rid of the big problem of the equity premium estimation. Equation $(ii)$ just needs estimates of the risk-free rate $r$ and of the volatility $\sigma$, which can be derived from real market quotes.
  \end{itemize}
	\pause
  \begin{tikzpicture}[remember picture,overlay]
    \node[xshift=5cm,yshift=-3.7cm] (image) at (current page.center) {\includegraphics[width=20px]{python_logo}};
    \node[align = center, yshift=1.45cm, below=of image] {\tiny{\href{shorturl.at/rGJQ3}{shorturl.at/rGJQ3}}};
  \end{tikzpicture}
\end{frame}


\begin{frame}{Drift Changes Generalization}
  \begin{block}{Proposition}
    Assume that, under a generic $N$-measure, we have the following dynamics for a $n$-vector diffusion process $X$
    \begin{equation*}
      dX_t = \mu_t^N(X_t)dt + \sigma_t(X_t)CdW^N_t
    \end{equation*}
    where $dW^N_t$ is a $n$-dimensional standard Brownian motion whose correlation is modeled by the $n\times n$ matrix $C$, $\mu$ is an $n\times 1$ vector and $\sigma_t$ a $n\times n$ diagonal matrix. Under the $U$-measure, we have
    \begin{equation}
      \mu^U_t(X_t) = \mu^N_t(X_t) - \rho\sigma(X_t)\left(\frac{\sigma^N_t}{N_t}-\frac{\sigma^U_t}{U_t}\right)'
    \end{equation}
	or \ldots
	\end{block}
\end{frame}

\begin{frame}{Drift Changes Generalization}
	\begin{block}{Proposition contd.}
		\begin{equation}
			CdW^U_t = CdW^N_t + \rho\left(\frac{\sigma^N_t}{N_t}-\frac{\sigma^U_t}{U_t}\right)' dt
		\end{equation}
		$\rho=CC'$ is the correlation matrix of $<dW^N_i,dW^N_j>$ and $\sigma^N_t$ and $\sigma^U_t$ are the (vector) volatilities of numeraires $N$ and $U$. %(one component for each Brownian motion).
	\end{block}
\end{frame}

\begin{frame}{Drift Changes (Proof)}
  %We now provide a formal proof of the above proposition in the special case of \textcolor{red}{$n=1$}, in which \textcolor{red}{$\rho=1$}.
  
  Indicate by $\mathcal{Q}^N$ and $\mathcal{Q}^U$ the $N$-measure and $U$-measure. By Girsanov theorem we have the following expression for the Radon-Nikodym derivative
  \begin{equation*}
    \zeta_t = \frac{d\mathcal{Q}^N}{d\mathcal{Q}^U} = e^{-\frac{1}{2}\int_0^t\gamma_s^2 ds + \int_0^t\gamma_s dW_s^U}
  \end{equation*}
  with 
  \begin{equation}
    \gamma_t=\frac{[\mu^N_t(X_t)-\mu_t^U(X_t)]'}{(\sigma_t(X_t)C)'}
    \label{eq:gamma_t}
  \end{equation}
	\pause
  We also know that $\zeta_t$ is an exponential martingale hence its dynamics is such that 
  \begin{equation}
    d\zeta_t=\gamma_t\zeta_tdW_t^U
    \label{eq:dzeta1}
  \end{equation}
\end{frame}

\begin{frame}{Drift Changes (Proof)}
  By the main theorem on numeraire change \cref{eq:radon_nikodym_der2}, and using the fact that $\zeta_t$ is a $\mathcal{Q}^U$-martingale, 
  \begin{equation}
    \zeta_t = \frac{d\mathcal{Q}^N}{d\mathcal{Q}^U} = \frac{U_0N_t}{N_0U_t}
    \label{eq:zeta_numeraire}
  \end{equation}
  thus
  \begin{equation}
    d\zeta_t= \frac{U_0}{N_0}d\left(\frac{N_t}{U_t}\right)= \frac{U_0}{N_0}\sigma_t^{N/U}CdW_t^U
    \label{eq:dzeta2}
  \end{equation}
  where $\sigma^{N/U}_t$ is the volatility of the process $N_t/U_t$, which is also a martingale under $\mathcal{Q}^U$.
\end{frame}
  
\begin{frame}{Drift Changes (Proof)}
  Comparing the two results for $d\zeta_t$ (\cref{eq:dzeta1}, \cref{eq:dzeta2} and using \cref{eq:zeta_numeraire}) we get
  \begin{equation*}
    \begin{gathered}
      \gamma_t\zeta_tdW_t^U = \gamma_t\frac{\cancel{U_0}N_t}{\cancel{N_0}U_t}\cancel{dW_t^U}=	\frac{\cancel{U_0}}{\cancel{N_0}}\sigma^{N/U}_tC\cancel{dW_t^U} \implies 
      \gamma_t = \frac{U_t}{N_t}\sigma^{N/U}_tC
    \end{gathered}
  \end{equation*}
  \pause
  Using the definition of $\gamma_t$ (\cref{eq:gamma_t}) and remembering that given two matrices $A$ and $B$ it holds ($(AB)' = B'A'$)
  \begin{equation}
    \begin{gathered}
    (\mu_t^N(X_t)-\mu_t^U(X_t))'= \gamma_t (\sigma_t(X_t)C)'=\frac{U_t}{N_t}\sigma^{N/U}_t CC'(\sigma_t(X_t))'\\
    \mu_t^U(X_t)=\mu_t^N(X_t)-\frac{U_t}{N_t}\sigma_t(X_t)\rho(\sigma^{N/U}_t)'
    \end{gathered}
  \label{eq:gamma}
\end{equation}
\end{frame}

\begin{frame}{Intermezzo}
  \begin{itemize}
  \item One of the classical formulas of differential calculus is th Leibniz rule $d(x y) = x(dy) + y(dx)$
  \item For stochastic processes this becomes, applying It$\hat{o}$'s formula to the function $F(X,Y) = XY$
    \begin{equation*}
      dF(x_i)=\sum_{i=1}^n \frac{\partial F}{\partial x_i}dx_i
      +\frac{1}{2}\sum_{i,j=1}^2 \frac{\partial^2 F}{\partial x_i \partial x_j}dx_i dx_j
    \end{equation*}
  For $n=2$:
    \begin{equation*}
      \begin{gathered}
        \frac{\partial F}{\partial X}=Y,\frac{\partial F}{\partial Y}=X \\
        \frac{\partial^2 F}{\partial X^2}=0,\frac{\partial^2 F}{\partial X\partial Y}=\frac{\partial^2 F}{\partial Y\partial X}=1,\frac{\partial^2 F}{\partial Y^2}=0\\
        \\
        d(XY) = XdY + YdX + dXdY
      \end{gathered}
    \end{equation*}
  \end{itemize}
\end{frame}

\begin{frame}{Drift Changes (Proof)}	
Now let $N_t$ and $U_t$ have dynamics under $\mathcal{Q}^U$ given by 
\begin{equation*}
	\begin{gathered}
		dN_t = (\ldots) dt + \sigma_t^NCdW^U_t\\
		dU_t = (\ldots) dt + \sigma_t^UCdW^U_t 
	\end{gathered}
\end{equation*}
	\pause
        From what we have just seen about the product rule in stochastic calculus
        \begin{equation*}
          \begin{gathered}
            d\left(\frac{N_t}{U_t}\right)=\frac{1}{U_t}dN_t+N_td\frac{1}{U_t}+dN_td\frac{1}{U_t} \quad \left(d\frac{1}{U_t}=-\frac{1}{U^2_t}dU_t+\cancel{\frac{1}{U^3_t}dU_tdU_t}\right) \\
          \end{gathered}
  \end{equation*}
	\pause
  Replacing the dynamics for $N_t$ and $U_t$ (ignoring the terms in $dt$ since we know that $d\cfrac{N_t}{U_t}$ is a martingale)
  \begin{equation}
    d\left(\frac{N_t}{U_t}\right) = \frac{dN}{U_t}-\frac{N_tdU}{U^2_t}\cancel{-\frac{dNdU}{U_t^2}}=\frac{\sigma^N_t CdW^U_t}{U_t} - \frac{N_t\sigma^U_t C dW^U_t}{U^2_t}
    \label{eq:dnu}
  \end{equation}
\end{frame}

\begin{frame}{Drift Changes (Proof)}   
  Taking $d(N_t/U_t)$ definition from \cref{eq:dzeta2} and comparing it with \cref{eq:dnu}
  \begin{equation}
    d\left(\frac{N_t}{U_t}\right)=\sigma_t^{N/U}\cancel{C dW^U_t} = \frac{\sigma^N_t \cancel{CdW^U_t}}{U_t} - \frac{N_t\sigma^U_t \cancel{C dW^U_t}}{U^2_t}\implies \sigma_t^{N/U}= \frac{\sigma^N_t}{U_t} - \frac{N_t\sigma^U_t}{U^2_t}
  \end{equation}
  Replacing above expression for $\sigma_t^{N/U}$ into \cref{eq:gamma}
  \begin{equation}
    \begin{aligned}
      \mu_t^U(X_t)&=\mu_t^N(X_t)-\frac{\cancel{U_t}}{N_t}\sigma_t(X_t)\rho\left(\frac{\sigma^N_t}{\cancel{U_t}} - \frac{N_t}{U_t^{\cancel{2}}}\sigma^U_t\right)'\\
      &=\mu_t^N(X_t)-\sigma_t(X_t)\rho\left(\frac{\sigma^N_t}{N_t} - \frac{\sigma^U_t}{U_t}\right)'
    \end{aligned}
  \end{equation}
  \textcolor{red}{which proves the first part of the statement}.
\end{frame}

\begin{frame}{Drift Changes (Proof)}
  Expressing $\gamma_t$ coefficient in terms of the numeraires volatilities
  \begin{equation*}
    \begin{cases}
      \gamma_t = \frac{[\mu_t^N(X_t) - \mu_t^U(X_t)]'}{(\sigma_t(X_t)C)'}\\
      \mu_t^N(X_t) - \mu_t^U(X_t) = \sigma_t(X_t)\rho \left(\frac{\sigma^N_t}{N_t} - \frac{\sigma^U_t}{U_t}\right)'
    \end{cases}\implies \gamma_t = \frac{\left(\frac{\sigma^N_t}{N_t} - \frac{\sigma^U_t}{U_t}\right)CC'(\sigma_t(X_t))}{C'(\sigma_t(X_t))'}
  \end{equation*}
  \begin{equation}
    \gamma_t = \left(\frac{\sigma^N_t}{N_t} - \frac{\sigma^U_t}{U_t}\right)C = C'\left(\frac{\sigma^N_t}{N_t} - \frac{\sigma^U_t}{U_t}\right)'\quad(\text{$C$ is a symmetric matrix})    
    \label{eq:gamma_3}
  \end{equation}
	\pause
  Finally from the Girsanov theorem we get the diffusion process under the new numeraire
  \begin{equation}
    \begin{gathered}
      CdW^N_t = CdW^U_t - C\gamma_t dt \\
      CdW^N_t = CdW^U_t - \rho\left(\frac{\sigma^N_t}{N_t}-\frac{\sigma^U_t}{U_t}\right)' dt
    \end{gathered}
	\label{eq:girsanov_shock_extended}
  \end{equation}
  \textcolor{red}{which proves also the second part of the proposition}.
  
  %As an exercise, once you know the Vasicek short rate model, try to determine the new drift when moving from bank account to forward meeasure.%the result is an application of the previous formula with $X = r$, $Q = P^T$ , $\sigma(X_t,t) = \sigma$, $\sigma_B (t) = -A(t, T)\sigma$ and $m(X_t,t) = a(b-rt)$.
\end{frame}

\begin{frame}{Asset/Numeraraire by Girsanov}
  Assuming an asset $S$ and a numeraire $N$ with the following evolutions under the risk-neutral measure
  \begin{equation}
    \begin{cases}
      \cfrac{dS_t}{S_t} = r_tdt + \sigma^S_t dW^B_t\quad\text{(asset)} \\
      \cfrac{dN_t}{N_t} = r_tdt + \sigma^N_t dW^B_t\quad\text{(numeraire)}
      %\frac{d\mathcal{Q}^N}{d\mathcal{Q}^B} = \frac{N_TB_0}{B_TN_0} = \exp\left(\int_0^T\sigma_t^N dW^B_t - \frac{1}{2}\int_0^T(\sigma^N_t)^2 dt\right)
    \end{cases}
    \label{eq:S_N_dynamics}
  \end{equation}
  
  by Girsanov Theorem (\cref{eq:girsanov_shock_extended}), under $\mathcal{Q}^N$, we get
  \begin{equation}
    dW^N_t = dW_t^B - \sigma_t^N dt
    \label{eq:girsanov_ex}
  \end{equation}
  which is a Brownian motion.
\end{frame}

\begin{frame}{Asset/Numeraraire by Girsanov}
  Now let's apply It$\hat{o}$'s lemma to $S_t/N_t$
  \begin{equation*}
    \begin{aligned}
      d\left(\frac{S}{N}\right) &= \frac{1}{S}dS - \frac{S}{N^2}dN + \frac{S}{N^3}dN^2-\frac{1}{N^2}dSdN = \frac{S}{N}\left(\frac{dS}{S}-\frac{dN}{N}+\frac{dN^2}{N^2}-\frac{dSdN}{SN} \right) = \\
      &=\frac{S}{N}\left(rdt+\sigma^S dW^B - rdt - \sigma^N dW^B + (\sigma^N)^2 dt - \sigma^S\sigma^N dt \right) = \quad\textit{(with \cref{eq:S_N_dynamics})}\\
      &=\frac{S}{N}((\sigma^N)^2 - \sigma^S\sigma^N) dt + \sigma^S dW^B - \sigma^N dW^B = \quad\textit{(with \cref{eq:girsanov_ex})}\\
      &=\frac{S}{N}((\sigma^N)^2 - \sigma^S\sigma^N) dt + \sigma^S (dW^N + \sigma^N dt) - \sigma^N(dW^N+\sigma^Ndt) = \\
      &=\frac{S}{N}(\sigma^S - \sigma^N)dW^N 
      \end{aligned}
  \end{equation*}
  which shows that $\cfrac{S}{N}$ is a $\mathcal{Q}^N$-martingale (no drift in dynamics).
\end{frame}

%\begin{frame}{}
%	Let $\mathcal{P}$ be an equivalent martingale measure (EMM) for the numeraire $H_t$ and $\mathcal{Q}$ an EMM for the numeraire $J_t$.Then given a probability space $(\Omega , \mathcal{F}_T , \mathcal{P}/\mathcal{Q})$, for the numeraire change theorem holds 
%	\begin{equation}
	%			V_t = \mathbb{E}^P\left[V_T\frac{H_t}{H_T}|\mathcal{F}_t\right]= \mathbb{E}^Q\left[V_T\frac{J_t}{J_T}|\mathcal{F}_t\right]
	%		\end{equation}	
%	Assume that $\mathcal{P}$ and $\mathcal{Q}$ are equivalent and denote the Radon-Nikodym by $\eta_t$. 
%	We then have
%	\begin{equation}
	%			V_t = \mathbb{E}^Q\left[V_T\frac{J_t}{J_T}|\mathcal{F}_t\right]= \mathbb{E}^P\left[V_T\frac{H_t}{H_T}\eta_t|\mathcal{F}_t\right]
	%		\end{equation}
%\end{frame}

%\begin{frame}{Drift Change General Form}
%	In particular, $\eta_t = \frac{H_TJ_t}{H_tJ_T}$ for $t < T$. 
%	
%	Suppose that the process under some measure $\mathcal{Q}^H$ associated with the numeraire $H_t$ is given by $dX_t = \mu(X_t,t)dt +
%	\sigma (X_t,t)dW^H_t$.
%	
%	We are interested on the process followed by $X_t$ under another measure $\mathcal{Q}^J$ with numeraire $J_t$. 
%	
%	From Girsanov’s theorem, 
%	\begin{equation}
%		\eta_t = \frac{d\mathcal{Q}^H}{d\mathcal{Q}^J}|\mathcal{F}_t=\exp(-\frac{1}{2}\int_0^t [\mu^H-\mu^J]^2/\sigma ds ) + \int_0^t [\mu^H-\mu^J]/\sigma dW^J_s
%	\end{equation}
%	we know that $\eta$ is an exponential martingale with dynamics 
%	\begin{equation}
%		d\eta_t = \alpha_t\eta_tdW^J_t\quad \left(\text{with } 	\alpha_t = [\mu^H-\mu^J]/\sigma \right)
%	\end{equation}
%\end{frame}
%
%\begin{frame}{Drift Change General Form}
%	Since 
%	%	Let $\mathcal{P}$ be an equivalent martingale measure (EMM) for the numeraire $H_t$ and $\mathcal{Q}$ an EMM for the numeraire $J_t$.Then given a probability space $(\Omega , \mathcal{F}_T , \mathcal{P}/\mathcal{Q})$, for the numeraire change theorem holds 
%	%	\begin{equation}
%		%			V_t = \mathbb{E}^P\left[V_T\frac{H_t}{H_T}|\mathcal{F}_t\right]= \mathbb{E}^Q\left[V_T\frac{J_t}{J_T}|\mathcal{F}_t\right]
%		%		\end{equation}	
%	%	Assume that $\mathcal{P}$ and $\mathcal{Q}$ are equivalent and denote the Radon-Nikodym by $\eta_t$. 
%	%	We then have
%	%	\begin{equation}
%		%			V_t = \mathbb{E}^Q\left[V_T\frac{J_t}{J_T}|\mathcal{F}_t\right]= \mathbb{E}^P\left[V_T\frac{H_t}{H_T}\eta_t|\mathcal{F}_t\right]
%		%		\end{equation}
%	%\end{frame}
%	
%	We also know that 
%	
%	\begin{equation}
%		\eta_T = \frac{J_0H_T}{H_0J_T}
%	\end{equation}
%	Since $\eta$ is a $\mathcal{Q}^J$ martingale
%	\begin{equation}
%		d\eta_t = \mathbb{E}^J[\eta_T]=\frac{J_0H_t}{H_0J_t}
%	\end{equation}
%\end{frame}
%
%
%\begin{frame}{Drift Change General Form}
%	Differentiating
%	\begin{equation}
%		d\eta_t =\frac{J_0}{H_0}d\left(\frac{H_t}{J_t}\right)=\frac{J_0}{H_0}\sigma^{H/J}dW^J_t
%	\end{equation}
%	since $H/J$ is a martingale under $J$ and $d(\frac{H_t}{J_t})=\sigma^{H/J}dW^J_t$.
%	By comparing the two dynamics for $\eta$
%	\begin{equation}
%		\alpha\eta_t = \frac{J_0}{H_0}\sigma^{H/J}\implies \frac{H_t}{J_t}\alpha=\sigma^{H/J}
%	\end{equation}
%	or by definition
%	\begin{equation}
%		\mu^J = \mu^H - \frac{J_t}{H_t}\sigma_t\sigma^{H/J}
%	\end{equation}
%\end{frame}
%
%
%
%\begin{frame}{Drift Change General Form}
%	The last step
%	
%	\begin{equation}
%		\sigma^{H/J} = \frac{\sigma_H}{J_t} - \frac{H_t}{J^2_t}\sigma_J
%	\end{equation}
%	
%	dWJ = -g dt + dWH
%	
%	If $W^J_t$ is a Wiener process under $\mathcal{Q}^J$, then $W^J_t = W^H_t - \int_0^t \theta_s ds$ where $d\eta_{t,T} = \Gamma_{t,T} \theta_T dW^P_T$.
%	
%	Let $H_t$ and $J_t$ have dynamics under $\mathcal{P}$ given by 
%	\begin{equation}
%		\begin{gathered}
%			dH_t = m_H dt + \sigma_H dW^P_t\\
%			dJ_t = m_J dt + \sigma_J dW^P_t 
%		\end{gathered}
%	\end{equation}
%\end{frame}
%
%Using these dynamics and noting that $\eta_t$ is a martingale under $\mathcal{P}$, it can be verified that
%\begin{equation}
%	\frac{d\eta_T}{\eta_T} = \left(\frac{\sigma_J}{J_T}-\frac{\sigma_H}{H_T}\right)dW^P_T
%\end{equation}
%
%So, $\theta_t = \frac{\sigma_J}{J_T}-\frac{\sigma_H}{H_T}$.
%Suppose $\mathcal{P}$ is the measure induced by the bank account numeraire and $\mathcal{Q}$ is the forward measure. For $s < t < T$
%\begin{equation}
%	\Gamma_t = \frac{J_t}{J_s}\frac{H_s}{H_T} = \frac{P(t,T)}{P(s,T)}\exp\left(-\int_s^t r_u du\right)
%\end{equation}
%
%Under measure $\mathcal{P}$, $\frac{dP(t,T)}{P(t,T)} = r_t dt + \sigma_P(t)dW_t$.
%It is a straightforward calculation to show that the process $\Gamma_t$ satifies
%\begin{equation}
%	\frac{\Gamma_t}{\Gamma_t}=\frac{dP(t,T)}{P(t,T)} - r_t dt = \sigma_P(t)dW_t
%\end{equation}
%
%This implies that$W^Q_t = W^P_t-\int_0^t\sigma_B(u)du$. Hence,if under $\mathcal{P}$ we have the dynamics $dX_t = m(X_t,t)dt + \sigma(X_t,t)dW_t^P$ then the $\mathcal{Q}$~-process for $X_t$
%is $dX_t =(m(X_t,t)+\sigma_B(t)\sigma(X_t,t))dt+\sigma(X_t,t)dW^Q_t$.

%\begin{frame}{Decorrelation}
%	\begin{itemize}
	%		\item In the LMM, we can assume that the brownian motions driving the dynamics of forward rate are correlated
	%		\begin{equation}
		%			<dW_t^{T_i}, dW_t^{T_i}> = \rho_{ij}dt
		%		\end{equation}
	%		\item In models for short rate or for i.f.r. it is assumed full correlation $\rho_{ij}=1$, which is a tight constraint on the dynamics of the forward rates.
	%		\item In the LMM, we can allow decorrelation to better fit the derivatives at hand.
	%		\item The change of measure with decorrelation implies
	%		\begin{equation}
		%			dF(t,T_1,T_2)=\rho_{12}\frac{v_2F(t,T_1,T_2)\tau}{1+\tau F(t,T_1,T_2)}dt + v_2F(t,T_1,T_2)dW^{T_1}_t
		%		\end{equation}
	%		\item Correlations have no impacts on Caps.
	%	\end{itemize}
%\end{frame}

\end{document}

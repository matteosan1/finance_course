\documentclass{beamer}
\usetheme{CambridgeUS}

\usepackage{tikz}
\usetikzlibrary{calc,positioning}
\usepackage{amsmath}
\usepackage[bb=dsserif]{mathalpha}
\usepackage{bm}
\usepackage{cancel}

\title{Change of Measure}
\author{Matteo Sani}

\begin{document}
	\begin{frame}[plain]
		\maketitle
	\end{frame}

\begin{frame}{Few Definitions}

%equivalent measure
%martingale measure

It may be helpful to explain some of the more technical terms we are going to use.\newline

\textbf{Sample space}: all possible future states or outcomes ($\Omega$).\newline

\textbf{(Probability) Measure}: the measure gives the probabilities of each of the outcomes in the sample space ($\mathcal{P}, \mathcal{Q}\ldots$). Two measures are \textbf{equivalent} if they agree "on what is possible". Note the word \emph{possible}: the two measures can have different probabilities, bu must have the same \emph{null-set} $\{x\in {\mathcal {P}}\mid p (x)=0\}$. 
\end{frame}

\begin{frame}{Few Definitions}
\textbf{Contingent claim}: is a derivative whose future payoff depends on the value of another “underlying” asset, or more generally, that is dependent on the realization of some uncertain future event $(S, X\ldots)$.\newline

\textbf{Filtrations}: are totally ordered collections of subsets that are used to model the information that is available at a given point in time ($\mathcal{F}_t$). \newline

\textbf{Martingale}: is a stochastic process for which, at a particular time, the conditional expectation of the next value in the sequence is equal to the present value, regardless of all prior values. It can be imagined as a drift-less process.
\end{frame}

\begin{frame}{Real World Measure $\mathcal{P}$}
\begin{itemize}
	\item When we model derivative prices, we take as given some "true" probability measure $\mathcal{P}$, which assigns probabilities to different states of the world. 
	\item These states in turn affect the path of security prices. 
	\item And these states plus the corresponding probabilities are supposed to reflect the subjective beliefs of traders or investors about what will happen in the future.
	\item Unfortunately, under $\mathcal{P}$ we can't usually price derivatives as expected discounted cash-flows. (i.e. the discounted price process of a derivative is not a martingale: that is, if $C_t$ is the price of a derivative at time $t$ and $r$ is the short rate (let's take it to be constant for simplicity), it's NOT generally the case that
	\begin{equation*}
		C_t=\mathbb{E}_t^{\mathcal{P}}[e^{-r(s-t)}C_s]
	\end{equation*}
	\end{itemize}
\end{frame}

\begin{frame}{Risk Neutral Measure}
\begin{itemize}
	\item This makes it hard to work out the price process. (Actually, no-arbitrage constructions or the Feynman-Kac formula will give you an explicit PDE whose solution is $C_t$, which will not generally be analytical solvable.)
	\item Changes of probability measure are important in mathematical finance because allow to express derivative prices as an expected discounted sum of cash-flows. 
	\item This in turn allows many derivative prices to be computed in closed-form.
	\item What we want is an "artificial" probability measure $\mathcal{Q}$ (i.e. assigning different probabilities to states of the world) under which the discounted derivative process IS a martingale.
	\end{itemize}
\end{frame}

\begin{frame}{}
	\begin{block}{Risk-Neutral Measure Definition}
	A probability distribution (measure) $\mathcal{Q}^N$ on the sample space $\Omega$ of possible market scenarios is said to be a \textcolor{red}{risk-neutral measure} if, for every asset $A$, and \textcolor{red}{numeraire} $N$, the price of $A$ at time $t=0$ is the \emph{discounted expectation}, under $\mathcal{Q}^N$, of the price at time $t$, that is
	\begin{equation}
		\frac{S_0}{N_0} = \mathbb{E}^{\mathcal{Q}^N}\left[\frac{S_t}{N_t}\bigg| \mathcal{F}_t\right]
		\label{eq:risk_neutral_pricing}
	\end{equation}
	The discounted value process is then a martingale.
	\end{block}
	\begin{itemize}
	\item We could in general assign any probabilities we wanted to states of the world, but expectations wrt these measures would generally not correspond to anything in the real world. 
	\item For the right choice of measure $\mathcal{Q}^N$, expectations wrt $\mathcal{Q}^N$ actually give the (real-world) prices of derivatives.
	\end{itemize}
\end{frame}

\begin{frame}{Numeraires}
	\begin{itemize}
	\item The question that arises is: how can we determine the right measure $\mathcal{Q}^N$ ?
	\item The answer is through change of \emph{numeraire}.
		\item A \textcolor{red}{numeraire} is any strictly positive stochastic process $N_t$ that is taken as a unit of reference when pricing an asset $S_t$
		\begin{equation*}
			\tilde{S_t}:=\frac{S_t}{N_t}, \quad t \ge 0
		\end{equation*}
		\item \emph{Deterministic numeraires} are easy to handle as they imply just an algebraic transformation (i.e. do not involve any risk), e.g.
		\begin{itemize}
			\item the exchange rate of the Italian Lira to the Euro was locked at EUR 1 = ITL 1936.27 on 31 December 1998.
		\end{itemize}
		%\item On the other hand, a random numeraire may involve new risks, and can allow for arbitrage opportunities.
	\end{itemize}
\end{frame}

\begin{frame}{Numeraires}
	\begin{itemize}
		\item When the numeraire is a stochastic process, and we want to move to it, the pricing of a claim has to be changed in order to take into account the new risks (the intrinsic randomness of the new numeraire).
		\item In particular it has to be determined a new probability measure under which the transformed process $\tilde{S_t}$ will be a martingale (changing the numeraire implies a change of the probability measure).
		\item It remains to understand how to pass from a numeraire to another, and hence by a measure to another in an arbitrage free setting.
	\end{itemize}
\end{frame}

\begin{frame}{Numeraire Examples (I)}
	\begin{itemize}
		\item \textbf{Money market account.} Given $r_t$, a possibly random and time dependent risk-free interest rate process, let
		\begin{equation*}
			N_t := \exp\left(\int_0^t r_s ds\right)
		\end{equation*}
		In this case 
		\begin{equation*}
			\tilde{S_t}:=\frac{S_t}{N_t}=e^{-\int_0^t r_s ds}S_t, \quad t \ge 0
		\end{equation*}
		represents the discounted price of the asset at time 0.
		\item \textbf{Currency exchange rate.} In this case $N_t := R_t$ denotes e.g. the EUR/SGD exchange rate. Let
		\begin{equation*}
			\tilde{S_t}:=\frac{S_t}{R_t}, \quad t \ge 0
		\end{equation*}
		denotes the price of a local (SG) asset quoted in units of the foreign currency (EUR) (notice the difference with previous ITL/EUR example above).
	\end{itemize}
\end{frame}

\begin{frame}{Numeraire Examples (II)}
	\begin{itemize}
		\item \textbf{Forward numeraire.} The price $P(t,T)$ of a bond paying $P(T,T)=1$ at maturity $T$. In this case
		\begin{equation*}
			N_t := P(t,T)=\mathbb{E}\left[e^{\int_t^T r_s ds}\right]
		\end{equation*}
		Notice that the process $t\rightarrow e^{\int_0^t r_s ds} P(t,T)=\mathbb{E}\left[e^{\int_0^T r_s ds}\right], \quad 0 \le t \le T$
		is a martingale
		\item \textbf{Annuity numeraire.} Processes of the form
		\begin{equation*}
			N_t = P(t, T_0, T_n) := \sum_{k=1}^{n}(T_k - T_{k-1})P(t, T_k), \quad 0 \le t \le T
		\end{equation*}
		where $P(t,T_1),P(t,T_2),\ldots,P(t,T_n)$ are bond prices with maturities $T_1 < T_2 < \ldots < T_n$.
	\end{itemize}
\end{frame}

\begin{frame}{Radon-Nikodym Derivative}
\begin{block}{Definition}
When two measures are equivalent it is possible to express the first in terms of the second through the \textcolor{red}{Radon-Nikodym derivative}. Indeed there exists a martingale process $\rho_t$ such that
\begin{equation*}
	\mathcal{P}^* =\int_{A} \rho_t(\omega)d\mathcal{P}(\omega)
\end{equation*}
which can be written in a more concise form as
\begin{equation}
	\frac{d\mathcal{P}^*}{d\mathcal{P}} = \rho_t
	\label{eq:radon_nikodym_der}
\end{equation}
\end{block}
\end{frame}

\begin{frame}{Intuition from Expected Value}
	\begin{itemize}
		\item Notice that we can write the expected value of a generic function $\Pi(x)$ under a measure $\mathcal{F}$, with associated density function $f(x)$ as
		\begin{equation*}
			\mathbb{E}^{\mathcal{F}}=\int\Pi(x)f(x)dx
		\end{equation*}
		\item Suppose there exists a function $g(x)$, which satisfies the mathematical conditions required to be a density function. Then we can write
		\begin{equation*}
			\mathbb{E}^{\mathcal{F}}=\int\Pi(x)f(x)\frac{g(x)}{g(x)}dx
		\end{equation*}
		\item If we define $\psi(x)=\Pi(x)\frac{f(x)}{g(x)}$ the expected value can be written as 
		\begin{equation*}
			\int\psi(x)g(x)dx=\mathbb{E}^{\mathcal{G}}\left[\psi(x)\right]=\mathbb{E}^{\mathcal{G}}\left[\Pi(x)\frac{f(x)}{g(x)}\right]=\mathbb{E}^{\mathcal{F}}\left[\Pi(x)\right]
		\end{equation*}
	\end{itemize}
\end{frame}

\begin{frame}{Radon-Nikodym Derivative}
\begin{itemize}
	\item The expectations corresponding to the two measures are related by
	\begin{equation*}
		\mathbb{E}^*[X] = \mathbb{E}\left[X\frac{d\mathcal{P}^*}{d\mathcal{P}}\right]
	\end{equation*}
	\item In case we need to compute a conditioned expectation
	\begin{equation}
		\mathbb{E}^*[X|\mathcal{F}_t] = \frac{\mathbb{E}\left[X\cfrac{d\mathcal{P}^*}{d\mathcal{P}}\bigg|\mathcal{F}_t\right]}{\mathbb{E}[\rho_t|\mathcal{F}_t]}
		\label{eq:conditioned_expectation}
	\end{equation}
	which is an equivalent formulation of the famous \emph{Bayes theorem}
 	\begin{equation*}
		P(A|B)=\frac{P(B|A)P(A)}{P(B)}
	\end{equation*}
\end{itemize}
\end{frame}

\begin{frame}{Change of Numeraire}
	\begin{block}{Theorem}
	Assume exists a numeraire $N_t$ and a measure $\mathcal{Q}^N$, equivalent to $\mathcal{Q}^0$, such that the price of any traded asset $S_t$ relative to $N$ is a martingale under $\mathcal{Q}^N$
	\begin{equation*}
		\frac{S_t}{N_t} = \mathbb{E}^N\left[\frac{S_T}{N_T}\bigg|\mathcal{F}_t\right],\quad 0\le t \le T
	\end{equation*}
	Let $U$ be another arbitrary numeraire. Then there exists a measure $\mathcal{Q}^U$, also equivalent to $\mathcal{Q}^0$, such that the price of any traded asset $X_t$, normalized to $U$, is a martingale under $\mathcal{Q}^U$
	\begin{equation*}
		\frac{X_t}{U_t} = \mathbb{E}^U\left[\frac{X_T}{U_T}\bigg|\mathcal{F}_t\right],\quad 0\le t \le T
	\end{equation*}
	\textcolor{red}{Also the risk-neutral price is invariant under change of numeraire.}
	\end{block}
\end{frame}	

\begin{frame}{Change of Numeraire}
	\begin{block}{Theorem continued}
		The Radon-Nikodym derivative defining the measure $\mathcal{Q}^U$ is given by
		\begin{equation}
			\frac{d\mathcal{Q}^U}{d\mathcal{Q}^N} = \frac{U_T N_0}{U_0 N_T}
			\label{eq:radon_nikodym_der2}
		\end{equation}
	\end{block}	
	By definition of $\mathcal{Q}^N$, for any asset price $Z$ holds
	\begin{equation*}
		\frac{Z_0}{N_0} = 
		\mathbb{E}^N\left[\frac{Z_T}{N_T}\right] = \mathbb{E}^U\left[\frac{U_0 Z_T}{N_0 U_T}\right] =
		\frac{\cancel{U_0} Z_0}{N_0 \cancel{U_0}}
	\end{equation*}
	since both equal $Z_0/N_0$. Also, by definition of Radon-Nikodym derivative
	\begin{equation*}
		\mathbb{E}^N\left[\frac{Z_T}{N_T}\right] = \mathbb{E}^U\left[\frac{Z_T}{N_T} \frac{d\mathcal{Q}^N}{d\mathcal{Q}^U}\right]
	\end{equation*}
	Combining the last two we get Eq.\ref{eq:radon_nikodym_der2}.
\end{frame}

\begin{frame}{Change of Numeraire}
	\begin{block}{Proof}
	The conditional expectation formula~\ref{eq:conditioned_expectation} gives
	\begin{equation*}
		\mathbb{E}^U\left[\cfrac{S_T}{U_T}\bigg|\mathcal{F}_t\right]=\cfrac{\mathbb{E}^N\left[\cfrac{d\mathcal{Q}^U}{d\mathcal{Q}^N}\cfrac{S_T}{U_T}\bigg|\mathcal{F}_t\right]}{\mathbb{E}^N\left[\cfrac{d\mathcal{Q}^U}{d\mathcal{Q}^N}\bigg|\mathcal{F}_t\right]}
	\end{equation*}
	But 
	\begin{equation*}
		\begin{cases}
		\mathbb{E}^N\left[\cfrac{d\mathcal{Q}^U}{d\mathcal{Q}^N}\cfrac{S_T}{U_T}\bigg|\mathcal{F}_t\right]=\cfrac{S_t}{N_t} \\
		\mathbb{E}^N\left[\cfrac{d\mathcal{Q}^U}{d\mathcal{Q}^N}\bigg|\mathcal{F}_t\right]=\cfrac{U_t}{N_t}
		\end{cases}\implies
		\frac{S_t}{N_t}=\mathbb{E}^U\left[\cfrac{S_T}{U_T}\bigg|\mathcal{F}_t\right]\frac{U_t}{N_t}
	\end{equation*}
    \end{block}
\end{frame}	

\begin{frame}{Change of Numeraire}
	\begin{itemize}
		\item The power of the Theorem we have just proved stems from the fact that we can find a characterization of our process by means of which we can work-out more easily the fundamental pricing formula.
		\item In particular it tells how we can find a measure associated to the new numeraire such that \textcolor{red}{the price of any asset divided by that numeraire is a martingale}.
		\item Also it gives a simple rule to write (the otherwise difficult to derive) Radon-Nikodym derivative.
%		\item For instance, to change from the measure induced by numeraire $B(t)$, the bank account numeraire, to the one induced by the $P(t,T)$, the Radon-Nikodym derivative, $\frac{d\mathcal{Q}^T}{d\mathcal{Q}}$ can be obtained as
%		\begin{equation}
%			\frac{P(t,T) \cdot 1}{B(t)P(0,T)}
%		\end{equation}
	\end{itemize}
\end{frame}

\begin{frame}{title}
Proving the existence of a risk neutral measure is the difficult part. Once its existence is established, a simple calculation of conditional expectations allows to go from a numeraire to any other.

Let $B$ be the cash numeraire and $\mathcal{Q}^B$ the corresponding risk-neutral measure. Also let $N$ be a numeraire (so $N/B$ is a $\mathcal{Q}^B$ martingale). 

Define a new measure by
\begin{equation*}
	\frac{d\mathcal{Q}^N}{d\mathcal{Q}^B} = \frac{N_TB_0}{B_TN_0}
\end{equation*}

Then, for any $P$ such that $P/B$ is a $\mathcal{Q}^B$ martingale
\begin{equation*}
\mathbb{E}^N\left[\frac{P_T}{N_T}\bigg|\mathcal{F}_t\right] = \frac{\mathbb{E}^B\left[\frac{P_T}{N_T}\frac{N_TB_0}{B_TN_0}\bigg|\mathcal{F}_t\right]}{\mathbb{E}^B\left[\frac{N_TB_0}{B_TN_0}\bigg|\mathcal{F}_t\right]}
=\frac{\mathbb{E}^B\left[\frac{P_T}{B_T}\bigg|\mathcal{F}_t\right]}
{\mathbb{E}^B\left[\frac{N_T}{B_T}\bigg|\mathcal{F}_t\right]}
=\frac{P_tB_t}{N_tB_t}=\frac{P_t}{N_t}
\end{equation*}
So $P/N$ is a $\mathcal{Q}^N$ martingale.
\end{frame}

\begin{frame}{title}
If you assume that you have a Brownian market:
\begin{equation*}
\begin{gathered}
\frac{dN_t}{N_t} = r_tdt + \sigma^N_t dW^B_t \\
\frac{dP_t}{P_t} = r_tdt + \sigma^P_t dW^B_t \\
\frac{d\mathcal{Q}^N}{d\mathcal{Q}^B} = \frac{N_TB_0}{B_TN_0} = 
\exp\left(\int_0^T\sigma_t^N dW^B_t - \frac{1}{2}\int_0^T(\sigma^N_t)^2 dt\right)
\end{gathered}
\end{equation*}

By Girsanov, under $\mathcal{Q}^N$,
\begin{equation*}
dW^N_t = dW_t^B - \sigma_t^N dt
\end{equation*}
is a Brownian motion and using Ito's lemma you can check that
\begin{equation}
\frac{d(P_t/N_t)}{P_t/N_t)} = (\sigma^P_t \sigma^N_t)dW^N_t
\end{equation}
which also shows that it is a Brownian martingale under $\mathcal{Q}^N$.
\end{frame}

\begin{frame}{Change of Numeraire Examples}
\footnotesize{\tiny {\tiny }}{
\begin{table}[bt]
	 \renewcommand*{\arraystretch}{1.4}
	\begin{tabular}{|l|l|} \hline
		\begin{tabular}{@{}l@{}}
		Any asset divided by the bank account
		$B_t$\\(recall $dB_t = r_t B_t dt$)
		\boxed{\cfrac{S_t}{B_t} = e^{-\int_0^t r_s ds}S_t}
		\end{tabular}
		& \begin{tabular}{l}
		It is a martingale under the\\
		measure $Q^B$ associated to \\
		the bank account numeraire,\\
  		i.e. the risk neutral measure.
		\end{tabular} \\ \hline
		\begin{tabular}{@{}l@{}}
		The forward rate\\
		\boxed{F(t, T_1, T_2) = \frac{1}{T_2-T_1}\left(\frac{P(t,T_1) - P(t,T_2)}{P(t,T_2)}\right)}\\
		can be interpreted as a portfolio of two ZCBs\\
		divided by another ZCB.		
		\end{tabular}
		& \begin{tabular}{l}
		Under the measure $\mathcal{Q}^2$\\ 
		associated to the numeraire\\ 
		$P(\cdot,T_2)$ it is a martingale.\end{tabular}\\ \hline  
		\begin{tabular}{@{}l@{}}
		The swap rate
		\boxed{S_{\alpha,\beta}(t) = \frac{P(t,T_\alpha)-P(t,T_\beta)}{\sum_{i=\alpha+1}^{\beta}\tau_i P(t,T_i)}}
		\\can be interpreted as a portfolio of two ZCBs\\
		divided by a portfolio of ZCBs.		
		\end{tabular}
		& \begin{tabular}{l}
		It is a martingale under the\\
		measure associated to the\\
		annuity numeraire.
		\end{tabular} \\ \hline
	\end{tabular}
\end{table}}
\end{frame}

\begin{frame}{A Trick for a Useful Separation}
	\begin{itemize}
	\item Until now we have used $B(t)$, the money market account, as numeraire. But it is natural to look for the most convenient one, which minimizes the mathematical difficulties according to the problem at hand.
	\item Given a contingent claim whose payoff at time $T$ is $\chi$, we have the following formula for its price $\Pi$
	\begin{equation*}
	\Pi_\chi(t,T)=\mathbb{E}^B\left[e^{-\int_t^T r_s ds}\chi\bigg|\mathcal{F}_t \right]=B_t\mathbb{E}^B\left[B^{-1}_T\chi|\mathcal{F}_t\right]
	\end{equation*}
	\item If $\chi$ and the short rate process were independent under $\mathcal{Q}^B$ then we could write
	\begin{equation*}
	\Pi_\chi(t,T)=\mathbb{E}^B\left[e^{-\int_t^T r_s ds}\bigg|\mathcal{F}_t\right]\mathbb{E}^B\left[\chi|\mathcal{F}_t\right] = P(t,T)\mathbb{E}^B\left[\chi|\mathcal{F}_t\right]
	\end{equation*}
	\end{itemize}
\end{frame}

\begin{frame}{A Trick for a Useful Separation}
	\begin{itemize}
		\item In general the above separation is not possible due to the interaction between the discount factor and the contingent claim payoff. 
		\item In this, like in other concrete situations, a better numeraire is indeed the ZCB with the same maturity $T$ of the derivative to price $(P(T,T)=1)$.
		\item The forward measure $\mathcal{Q}^T$ (also called the $T$-measure) is defined as the martingale measure for the numeraire process $P(t,T)$, the ZCB maturing in T.
		\item It is easy to see that, in this case the Radon-Nykodim derivative is given by
		\begin{equation}
			\zeta_t = \frac{d\mathcal{Q}^T}{d\mathcal{Q}^B} = \frac{P(t,T)\overbrace{B(0)}^{=1}}{B_t P(0,T)} ,\quad\left(\eta_T=\frac{1}{B(T)P(0,T)}\right)
		\label{eq:radon_nikodym_t_forward}
		\end{equation}
	\end{itemize}
\end{frame}

\begin{frame}{A Trick for a Useful Separation}
	\begin{itemize}
		\item Applying the change of numeraire to the pricing formula, we get
		\begin{equation*}
			\begin{aligned}
			\Pi_\chi(t,T) & = B_t\mathbb{E}^B\left[B^{-1}_T\chi|\mathcal{F}_t\right] \\
			& = B_t\mathbb{E}^B\left[P(0,T)\zeta_T\chi|\mathcal{F}_t\right]\quad\text{(using RN expression above)}\\
			& = B_tP(0,T)\mathbb{E}^B\left[\zeta_T|\mathcal{F}_t\right]\mathbb{E}^T\left[\chi|\mathcal{F}_t\right]\quad\text{(by Eq.~\ref{eq:conditioned_expectation})}\\
			& = \cancel{B_tP(0,T)}\frac{P(t,T)}{\cancel{B_tP(0,T)}}\mathbb{E}^T\left[\chi|\mathcal{F}_t\right] \\
			& = P(t,T)\mathbb{E}^T\left[\chi|\mathcal{F}_t\right] \\
			\end{aligned}
		\end{equation*}
		which achieves the desired separation (although under a new measure).
		\item Clearly this particular trick is useful when $\chi$ has known dynamics under the forward measure.
	\end{itemize}
\end{frame}

\begin{frame}{Equivalence between $\mathcal{Q}^B$ and $\mathcal{Q}^T$}
By construction of the martingale measure $\mathcal{Q}^B$, the following relationship holds
\begin{equation*}
\begin{gathered}
\frac{P(t,T)}{B_t}=\mathbb{E}^B\left[\frac{P(T,T)}{B_T}\right]\\[0.3cm]
P(t,T)=\mathbb{E}^B\left[\frac{P(T,T)}{B_T}B_t\right] = \mathbb{E}^B\left[\frac{B_t}{B_T}\right]
\end{gathered}
\end{equation*}
Plugging the result into the Radon-Nikodym derivative gives
\begin{equation*}
	\frac{d\mathcal{Q}^T}{d\mathcal{Q}^B} = \frac{B_t}{B_T}\frac{1}{P(t,T)} =\frac{B_t/B_T}{\mathbb{E}^B[B_t/B_T]}
\end{equation*}	
If interest rates are deterministic then the Radon-Nikodym derivative is 1, hence the two measures are the same.
\end{frame}

\begin{frame}{Clarification on Time}
	\begin{itemize}
		\item Clearly as the Radon-Nikodym derivative is a martingale for valuation time $t$, we have
		\begin{equation}
			\frac{d\mathcal{Q}^U}{d\mathcal{Q}^N}=\frac{U_tN_0}{U_0N_t}
		\end{equation}
		\item So do not confuse the maturity of the numeraire bond $T$ with the times at which you have to take the values of the numeraire, in this case $t$ and 0.
		\item If you want to switch from the $T$ measure to the $S$ measure, i.e. the one induced by the bond $P(.,S)$, for the valuation time $t$ we get
		\begin{equation}
			\frac{d\mathcal{Q}^S}{d\mathcal{Q}^T}=\frac{P(t,S)P(0,T)}{P(t,T)P(0,S)}
		\end{equation}
	\end{itemize}
\end{frame}

\begin{frame}{The Forward Rate Under $\mathcal{Q}^T$}
	\begin{block}{Theorem}
		Consider the forward numeraire $P(t,T)$ and denote with $\mathcal{Q}^T$ its associated measure.
		The forward rate spanning the interval $[S,T]$ is the $\mathcal{Q}^T$ expectation of the future spot rate at time $S$ for the maturity $T$
		\begin{equation}
			\mathbb{E}^T[L(S,T)|\mathcal{F}_t] = F(t,S,T)
		\end{equation}
	\end{block}
\end{frame}

\begin{frame}{The Forward Rate Under $\mathcal{Q}^T$}
	\begin{block}{Proof}
		\begin{equation*}
			\begin{gathered}
				F(t,S,T) = \frac{1}{\tau}\left[\frac{P(t,S)-P(t,T)}{P(t,T)}\right] \\[0.3cm]
				F(t,S,T)P(t,T) = \frac{P(t,S)-P(t,T)}{\tau}
			\end{gathered}
		\end{equation*}
		This is the price at time $t$ of an asset (difference of two bonds). Therefore by the change of numeraire theorem and by definition of forward measure
		\begin{equation*}
			\frac{F(t,S,T)P(t,T)}{P(t,T)} = F(t,S,T)
		\end{equation*}
		is a martingale under such measure. Hence
		\begin{equation*}
			F(t,S,T) = F(S,S,T) = L(S,T) = \frac{1}{\tau}\left[\frac{1-P(S,T)}{P(S,T)}\right]
		\end{equation*}
	\end{block}
\end{frame}

\begin{frame}{The Forward Rate Under $\mathcal{Q}^T$}
	A similar result can be derived for the corresponding instantaneous quantities
	\begin{equation}
		\mathbb{E}^T[r_t|\mathcal{F}_t] = f(t,T)
	\end{equation}
	Indeed from the definition of $\mathcal{Q}^B$
	\begin{equation*}
		\frac{P(t,T)}{B_t}=\mathbb{E}^B\left[\frac{P(T,T)}{B_T}\bigg|\mathcal{F}_t\right]
	\end{equation*}
	but $P(T,T)=1$ so
	\begin{equation*}
	P(t,T)=\mathbb{E}^B\left[\frac{B_tP(T,T)}{B_T}\bigg|\mathcal{F}_t\right]=\mathbb{E}^B\left[e^{-\int_t^Tr_u du}\big|\mathcal{F}_t\right]
	\end{equation*}
	Differentiating with respect to $T$ ($\frac{d}{dx}\int_c^x f(t)dt=f(x)$)
	\begin{equation*}
	\frac{\partial P(t,T)}{\partial T}=
	\mathbb{E}^B\left[r(T)e^{-\int_t^Tr_u du}\big|\mathcal{F}_t\right]
	\end{equation*}
\end{frame}

\begin{frame}{The Forward Rate Under $\mathcal{Q}^T$}
	Now we can change numeraire to $P(t,T)$ so that, using reciprocal of Eq.~\ref{eq:radon_nikodym_t_forward} ($\eta^{-1}=\frac{B_t/B_T}{P(t,T)/P(T,T)}$)
	\begin{equation*}
	\frac{\partial P(t,T)}{\partial T}=
	\mathbb{E}^T\left[r(T)\cancel{e^{-\int_t^Tr_u du}}\frac{P(t,T)}{\cancel{e^{-\int_t^Tr_u du}}}\bigg|\mathcal{F}_t\right]=
	P(t,T)\mathbb{E}^T\left[r(T)|\mathcal{F}_t\right]
	\end{equation*}
	Hence
	\begin{equation*}
	\begin{aligned}
	f(t,T)&=\frac{1}{P(t,T)}\frac{\partial P(t,T)}{\partial T}=
	-\frac{\partial \ln P(t,T)}{\partial T}\\
	& = \mathbb{E}^T\left[r(T)|\mathcal{F}_t\right]=		\mathbb{E}^T\left[f(T,T)|\mathcal{F}_t\right]
	\end{aligned}
	\end{equation*}
	This implies that the instantaneous forward rate is a martingale under the $T$-forward measure.
	%
	%\begin{frame}{The Expectation Hypothesis}
	%	\begin{itemize}
		%		\item It is possible to prove the following
		%		\begin{equation}
			%			f(t, T) = \mathbb{E}^{\mathcal{Q}^T}[r(T)|\mathcal{F}_t]
			%		\end{equation}
		%		\item According to the pure expectation hypothesis, the above formula is valid if the expected value is taken under the real probability.
		%		\item Absence of arbitrage makes this incompatible with stochastic interest rates.
		%	\end{itemize}
	%	DA CAPIRE MEGLIO
	%\end{frame}
	%
	
\end{frame}


\begin{frame}{Girsanov Theorem}
	\begin{itemize}
		\item We're left with one important question:
		\textcolor{red}{what does the path of an asset price $S_t$ look like under a new measure $\mathcal{Q}$ ?} (we need to know this in order to be able to really compute its expectation under $\mathcal{Q}$.)
		\item \emph{Girsanov's theorem} answers to this question since it tells us, when we change from $\mathcal{P}$ to some other measure $\mathcal{Q}$, how the stochastic part of a process ($W_t$) changes under $\mathcal{Q}$.
		\item Will see that it evolves as the sum of a Brownian motion under $\mathcal{Q}$ and a drift process related to the Radon-Nikodym derivative characterizing $\mathcal{Q}$.
		%\item We therefore want to choose the Radon-Nikodym derivative so that the drift of $W_t$ wrt $\mathcal{Q}$ exactly cancels out the drift of $S_t$, leaving us with a pure diffusion process. 
		\item Under mild technical conditions, the resulting diffusion process will be a martingale, meaning $\mathcal{Q}$ is an \emph{equivalent martingale measure}.
%		\item It turns out that a price process $C_t$ for a derivative on $S_t$ avoids arbitrage opportunities only if a risk-neutral measure for the price process of the underlying $S$ is also a risk-neutral measure for $C$. 
%		\item So we want to construct a risk-neutral measure for $S$.
%		\item Changing measures change every SDE describing asset prices. The way in which this happens is characterized by the Girsanov theorem.
%		\item Obviously the drift of the SDE of the asset changes. The change of drift is characterized by the Girsanov theorem.
	\end{itemize}
\end{frame}
 
\begin{frame}{Girsanov Theorem}
	\begin{itemize}
		\item So Girsanov's theorem actually solves two problems at once: 
		\item it tells us what new probability measure we want to choose, 
		\item AND it tells us that $S_t$ evolves as a pure diffusion process under the new measure. 
		\item This allows us to straightforwardly compute the expectation with the risk-neutral pricing formula and explicitly solve for the derivative price.
		%\item The Girsanov theorem shows how a SDE changes due to the changes in the underlying probability measure.
		%Indeed if we change the measure, the drift of the SDE changes while the diffusion coefficient remains the same.
		%It can be useful when it is needed to change the drift coefficient of a SDE.
	\end{itemize}
\end{frame}

\begin{frame}{Girsanov Theorem}
	\begin{block}{Theorem}
	Consider the SDE 
	\begin{equation*}
		dX_t = f_t dt + \sigma_t dW_t
	\end{equation*}
	under $\mathcal{P}$. 
	
	Let be given a new drift $f^*_t$ and assume $\gamma_t=\frac{f_t^*-f_t}{\sigma_t}$ such that $\mathbb{E}\left[\exp\left(\frac{1}{2}\int_0^t\gamma_t^2dt\right)\right]<\infty$.
	Define the measure 
	\begin{equation}
	\frac{d\mathcal{P}^*}{d\mathcal{P}}=\exp\left(-\frac{1}{2}\int_0^t \gamma_s^2 ds + \int_0^t \gamma_s dW_s \right)
	\end{equation}
	Then $\mathcal{P}^*$ is equivalent to $\mathcal{P}$. 
	The Radon-Nikodym derivative process is an \emph{exponential martingale}.
	\end{block}
\end{frame}

\begin{frame}{Girsanov Theorem}
	\begin{block}{Theorem continued}
		Also the process
		\begin{equation}
			dW^*_t = -\gamma_s dt + dW_t
		\end{equation} 
		is a Brownian motion under $\mathcal{P}^*$, and 
		\begin{equation*}
			dX_t = f^*_t dt + \sigma_t dW^*_t
		\end{equation*}
	The condition $\mathbb{E}\left[\exp\left(\frac{1}{2}\int_0^t\gamma_t^2dt\right)\right]<\infty$ is a sufficient but non-necessary condition. It is know as the \textbf{Novikov condition}.
	\end{block}
	\begin{tikzpicture}[remember picture,overlay]
	\node[xshift=5cm,yshift=-3.7cm] (image) at (current page.center) {\includegraphics[width=20px]{python_logo}};
	\node[align = center, yshift=1.45cm, below=of image] {\tiny{\href{shorturl.at/ctCF7}{shorturl.at/ctCF7}}};
\end{tikzpicture}
\end{frame}

\begin{frame}{An Example}
\begin{itemize}
	\item Consider the stochastic differential equation
	\begin{equation*}
	dX_t = b(X_t, t) dt + a(X_t, t) dW_t
	\end{equation*}
	\item Let's assume that the drift and diffusion coefficients are such that there exists a unique solution to the equation which is $X$.
	\item We want to find a probability measure $\mathcal{Q}$ such that the drift of $X$ is $\tilde{b}(X_t,t)$ instead of $b(X_t,t)$.
\end{itemize}
	\begin{equation*}
	\begin{aligned}
		dX_t = \tilde{b}(X_t,t) dt &+ a(X_t,t) \left(\frac{b(X_t,t)-\tilde{b}(X_t,t)}{a(X_t,t)}\right)dt + \\ &+ a(X_t,t) dW_t = \ldots
	\end{aligned}
	\end{equation*}
\end{frame}

\begin{frame}{An Example}
	\begin{equation*}
		\begin{aligned}
		\ldots &= \tilde{b}(X_t,t)dt+a(X_t,t)d\left(W_t + \int_0^t \frac{b(X_s,s)-\tilde{b}(X_s,s)}{a(X_s,s)} ds\right) \\
		&=\tilde{b}(X_t,t)dt+a(X_t,t)d\tilde{W}_t
		\end{aligned}
		\end{equation*}
		where $\tilde{W}_t=W_t+\int_0^t\gamma_s ds$ and $\gamma_t =\frac{b(X_t,t)-\tilde{b}(X_t,t)}{a(X_t,t)}$
	\begin{itemize}
	\item If the Novikov condition is satisfied then by the Girsanov theorem we have that
	\begin{equation}
		\mathcal{Q} = \mathbb{E}^{\mathcal{P}}\left[\exp\left(-\frac{1}{2}\int_0^t \gamma_s^2 ds + \int_0^t \gamma_s dW_s \right)\right]
	\end{equation}
	and that $\tilde{W}$ is a Brownian motion on $\mathcal{Q}$.
	\item In practice, don't need to determine $\mathcal{Q}$. It is enough to know it exists, and the SDE of the process of interest according to the new measure.	
	\end{itemize}
\end{frame}


%%* A risk-neutral measure $\mathcal{Q}$ allows us to price things in the following way: 
%%* Suppose that at some time $T$ there is an easy no-arbitrage pricing argument pinning down $C_T$. For instance, if $C$ is the price process of a European call on stock $S$ and $T is the exercise date, then $C_T = (S_T - K)^+$, where $K$ is the strike price. 
%%* In this case, the martingale property wrt $\mathcal{Q}$ implies that
%%
%%𝐶_𝑡=\mathbb{𝐸}^{$\mathcal{Q}$}_𝑡[𝑒^{−𝑟(𝑇−𝑡)}𝐶_𝑇]=\mathbb{𝐸}^{$\mathcal{Q}$}_t[𝑒^{−𝑟(𝑇−𝑡)}(𝑆_𝑇−𝐾)^+]









%* Suppose we model $S$ as 
%
%𝑑𝑆_𝑡=\mu_𝑡 𝑑𝑡 + \sigma_t 𝑑𝑊^{$\mathcal{Q}$}_𝑡
%


%\begin{frame}{title}
%\begin{itemize}
%	\item Let $(\Omega,\mathcal{F}_t, \mathbb{P})$ be a probability space with a standard Brownian motion $W^{\mathbb{P}}$.
%	\item The stochastic process $S_t$ represents the evolution of a risky security price satisfying stochastic differential equation (SDE)
%	\begin{equation*}
%		dS_t = \mu S_t dt + \sigma S_t dW^{\mathbb{P}}_t
%	\end{equation*}
%	\item Let's assume that interest rate $r$ is constant. Therefore
%	\begin{equation*}
%		D(0,t) = e^{-rt}
%	\end{equation*} 	
%	which implies $dD = -re^{-rtdt}$.  
%\end{itemize}
%\end{frame}
%
%\begin{frame}{title}
%	\begin{itemize}
%		\item Define then 
%		\begin{equation*}
%			Y_t = D_t S_t
%		\end{equation*} 
%		that is the present value at time $t$ of the risky security.
%		\item Using Ito's lemma
%		\begin{equation*}
%			\begin{aligned}
%			dY_t &= D_t dS_t + dD_t S_t \\ 
%			&= D_t (\mu S_t dt + \sigma S_t dW^{\mathbb{P}}_t) + S_t (-rD_t dt) \\
%			&= (\mu - r) Y_t dt + \sigma Y_t dW_t^{\mathbb{P}}		
%			\end{aligned}
%		\end{equation*}
%		\item In its integral form it becomes
%		\begin{equation*}
%			Y_t = Y_0 + (\mu - r)\int_0^t Y_s ds + \sigma \int_0^t Y_s dW_s^{\mathbb{P}}
%		\end{equation*}
%	\end{itemize}
%\end{frame}
%
%\begin{frame}{title}
%\begin{itemize}
%\item Let's set 
%\begin{equation*}
%	
%\end{equation*}
%\end{frame}
%
%
%Radon-Nikodym theorem VIII
%Theorem
%Radon-Nikodym theorem. Given two equivalent probability
%measures P and Q constructed on the measurable space
%(Ω, F), there exists a positive-valued random variable Y such
%that
%Q (A) = E
%P
%[Y IA] .
%Such a random variable Y is often denoted by dQ
%dP
%.
%

%\begin{frame}{Intuition behind the Mathematics}
%	\begin{itemize}
%		\item The following theorem is at the heart of the \emph{no arbitrage} pricing theory.
%		\begin{block}{}
%		The market model does not allow for arbitrage if and only if there exist a martingale measure $\mathcal{Q}^0$ under which the processes
%		\begin{equation}
%		\frac{S_0(t)}{S_0(t)},\frac{S_1(t)}{S_0(t)},\ldots,\frac{S_N(t)}{S_0(t)}
%		\end{equation}
%		behave as martingales under $\mathcal{Q}^0$.
%		\end{block}
%	\end{itemize}
%\end{frame}
%\begin{frame}{Numeraire Pricing}
%	\begin{block}{Theorem (German, El Karoui and Rochet, 1995)}
%		Assume that there exists a numeraire $N$ and a probability measure $\mathcal{Q}^N$ which is equivalent to $\mathcal{P}$ such that, for every traded asset $X$:
%		\begin{equation}
%			\frac{X_t}{N_t} = \mathbb{E}^{\mathcal{Q}^N}\left[\frac{X_T}{N_T}|\mathcal{F}_t\right]
%		\end{equation}
%		Now, given a second arbitrary numeraire $U$, there exists a probability measure $\mathcal{Q}^U$ which is equivalent to $\mathcal{P}$ and such that:
%		\begin{equation}
%			\frac{X_t}{U_t} = \mathbb{E}^{\mathcal{Q}^U}\left[\frac{X_T}{U_T}|\mathcal{F}_t\right]
%		\end{equation}
%	\end{block}
%\end{frame}


%\begin{frame}{Example Approach to Numeraire Change}
%\begin{itemize}
%	\item In lieu of the fundamental theorem we can write
%	\begin{equation}
%	\Pi(0,X)=S_0(0)\mathbb{E}^0\left[\frac{X}{S_0(T)}\right]
%	\end{equation}
%	\item But also
%	\begin{equation}
%	\Pi(0,X)=S_1(0)\mathbb{E}^1\left[\frac{X}{S_1(T)}\right]
%	\end{equation}
%	\item We define the Radon-Nikodym derivative
%	\begin{equation}
%	L_0^1(T)=\frac{dQ^1}{dQ^0}
%	\end{equation}
%\end{itemize}
%\end{frame}
%
%\begin{frame}{Example Approach to Numeraire Change}
%	\begin{itemize}
%		\item Hence we can write ?????????????????
%		\begin{equation}
%			\Pi(0,X)=S_1(0)\mathbb{E}^0\left[\frac{X}{S_1(T)}L_0^1(T)\right]
%		\end{equation}
%		\item After some trivial manipulations
%		\begin{equation}
%			S_0(0)\mathbb{E}^0\left[\frac{X}{S_0(T)}\right]=
%			S_1(0)\mathbb{E}^0\left[\frac{X}{S_1(T)}L_0^1(T)\right]
%		\end{equation}
%		\item Finally
%		\begin{equation}
%			\frac{S_0(0)}{S_0(T)}=\frac{S_1(0)}{S_1(T)}L_0^1(T)		
%		\end{equation}
%	\end{itemize}
%\end{frame}
%
%\begin{frame}{Example Approach to Numeraire Change}
%	\begin{itemize}
%		\item Hence 
%		\begin{equation}
%			L_0^1(T) = \frac{dQ^1}{dQ^0}=			\frac{S_0(0)S_1(0)}{S_1(T)S_0(T)}
%		\end{equation}
%
%	\end{itemize}
%\end{frame}
%
%\begin{frame}{The Theorem}
%\begin{block}{}
%Let $\mathcal{Q}^0$ a martingale measure associated with the numeraire $S_0$ and further suppose that $S_1$ is the positive process of an asset such that
%\begin{equation}
%\frac{S_1(t)}{S_0(t)}
%\end{equation}
%is a martingale under $\mathcal{Q}^0$. Define $\mathcal{Q}^1$ by means of the Radon-Nikodym derivative
%		\begin{equation}
%	L_0^1(t) = \frac{S_0(0)S_1(0)}{S_1(t)S_0(t)}
%\end{equation}
%SBAGLIATI GLI INDICI 0 e t
%Then $\mathcal{Q}^1$ is a martingale measure for $S_1$.
%\end{block}
%\end{frame}










\begin{frame}{Real-World Measure}
\begin{itemize}
	\item Assume that a stock price has the following dynamic (Geometric Brownian Motion) under the real-world measure $\mathcal{P}$
	\begin{equation*}
	dS_t = \mu S_t dt + \sigma S_t dW_t
	\end{equation*}
	\item Also the bank account is assumed to be deterministic hence
	\begin{equation*}
	dB_t = rB_tdt\implies B_t = e^{rt}
	\end{equation*}
	\item We want to know what happens to the SDE with two different numeraires:
	\begin{itemize}
		\item risk-neutral measure (bank account numeraire);
		\item and stock measure (stock numeraire).
	\end{itemize}
\end{itemize}
\end{frame}

\begin{frame}{Risk-Neutral Measure Dynamics}
	\begin{itemize}
	\item We have seen that under the bank account induced measure
	\begin{equation*}
	\frac{S_t}{B_t} = \mathbb{E}^B\left[\frac{S_T}{B_T}\bigg|\mathcal{F}_t\right]
	\end{equation*}
	\item So if we define $Z_t=\frac{S_t}{B_t}$, since $Z_t$ is a martingale (no-drift process) we have 
	\begin{equation}
	dZ_t = \sigma Z_t dW_t^B
	\label{eq:z_martingale1}
	\end{equation}
	where $dW_t^B$ is a Brownian motion under the $\mathcal{Q}^B$ measure.
\end{itemize}
\end{frame}

\begin{frame}{Risk-Neutral Measure Dynamics}
\begin{itemize}
	\item By applying Ito's rule to $Z_t$
	\begin{equation*}
		\begin{aligned}
		d\left(\frac{S_t}{B_t}\right) &= \frac{dS_t}{B_t} + S_t d\left(\frac{1}{B_t}\right) \\ 
		&=\frac{dS_t}{B_t} + S_t d\left(e^{-rt}\right) = \frac{dS_t}{B_t} - S_t re^{-rt}dt \\
		&= \frac{dS_t}{B_t} - r\frac{dS_t}{B_t}dt 
		\end{aligned}
	\end{equation*}
	\item Now substitute for $dS_t$
	\begin{equation*}
		d\left(\frac{S_t}{B_t}\right)= \frac{ \mu S_t dt + \sigma S_t dW_t}{B_t} - r\frac{dS_t}{B_t}dt = \sigma\frac{S_t}{B_t}\left(\frac{\mu - r}{\sigma}dt + dW_t \right)
	\end{equation*}	
\end{itemize}
\end{frame}

\begin{frame}{Risk-Neutral Measure Dynamics}
	\begin{itemize}
	\item In terms of $Z_t$ it becomes
	\begin{equation}
	dZ_t = \sigma Z_t\left(\frac{\mu - r}{\sigma}dt + dW_t \right)
	\label{eq:z_martingale2}
	\end{equation}
	\item Both Eq.~\ref{eq:z_martingale2} and~\ref{eq:z_martingale1} represent the dynamics of $Z_t$ so they must be equal
	\begin{equation*}
		dW_t^B = \frac{\mu - r}{\sigma}dt + dW_t
	\end{equation*}
	\item Replacing the BM into the real-world dynamics
	\begin{equation*}
		\begin{aligned}
		dS_t &= \mu S_t dt + \sigma S_t \left(dW_t^B - \frac{\mu - r}{\sigma}dt\right) \\
		& = \mu S_t dt - \mu S_t dt + rS_t dt + \sigma S_t dW_t^B  = rS_t dt + \sigma S_t dW_t^B
		\end{aligned}
	\end{equation*}
	\item So \textcolor{red}{under the risk-neutral measure the drift equals the risk-free rate}.
\end{itemize}
\end{frame}

\begin{frame}{Stock Numeraire Measure Dynamics}
	\begin{itemize}
	\item Now let's see what happens under the stock numeraire.
	\item Again 
	\begin{equation*}
	\frac{S_0}{B_0} = \mathbb{E}^B\left[\frac{S_t}{B_t}\bigg|\mathcal{F}_0\right] \implies
	S_0 = \mathbb{E}^B\left[B_0\frac{S_t}{B_t}\bigg|\mathcal{F}_0\right]
	\end{equation*}
	\item For the stock numeraire $A$
	\begin{equation*}
	S_0 = \mathbb{E}^A\left[A_0\frac{S_t}{A_t}\bigg|\mathcal{F}_0\right]
	\end{equation*}
	\item Since both expressions represent a price of an asset they must be the same and we can equal the terms inside the expectations.
\end{itemize}
\end{frame}

\begin{frame}{Stock Numeraire Measure Dynamics}
	\begin{equation*}
	\frac{B_0}{B_t}d\mathcal{Q}^B = \frac{A_0}{A_t}d\mathcal{Q}^A\implies \frac{d\mathcal{Q}^A}{d\mathcal{Q}^B}=\frac{B_0A_t}{B_tA_0}
	\end{equation*}
	\begin{itemize}
	\item We know that the solution in the risk-neutral measure of the GBM is (see slides on GBM)
	\begin{equation*} 
	A_t = A_0 e^{rt-\frac{1}{2}\sigma^2 t + \sigma W^B_t}
	\end{equation*}
	so we can re-write
	\begin{equation*}
	\frac{d\mathcal{Q}^A}{d\mathcal{Q}^B}=e^{-rt}e^{rt-\frac{1}{2}\sigma^2 t + \sigma W^B_t}=e^{-\frac{1}{2}\sigma^2 t + \sigma W^B_t}
	\end{equation*}
	\item From the Girsanov theorem the function $y_t = \sigma$ hence
	\begin{equation*}
	dW_t^A = dW_t^B - \sigma dt 
	\end{equation*}
\end{itemize}
\end{frame}

\begin{frame}{Summarizing the Results}
\begin{itemize}
	\item Substituting back into the risk-neutral dynamics we get
	\begin{equation*}
		\begin{aligned}
		dS_t &= r S_t dt + \sigma S_t dW_t^B = 
		rS_t dt + \sigma S_t (dW_t^A + \sigma dt) \\
		& = (r + \sigma^2)S_t dt + \sigma S_t dW^A_t
		\end{aligned}
	\end{equation*}

	\item To summarize all the results
	\end{itemize}

	\begin{table}
		\begin{tabular}{lr}
		$dS_t = \textcolor{red}{\mu} S_t dt + \sigma S_t dW_t$ & Real-world measure \\
		$dS_t = \textcolor{red}{r}S_t dt + \sigma S_t dW_t^B$ & Risk-neutral measure \\
		$dS_t = \textcolor{red}{(r + \sigma^2)}S_t dt + \sigma S_t dW^A_t$ & Stock measure\\
		\end{tabular}
	\end{table}
\end{frame}


\begin{frame}{Drift Changes (Generalization)}
	\begin{block}{Proposition}
	Assume that, under the $S$-measure, we have
	\begin{equation*}
	dX_t = \mu^S(X_t)dt + \sigma(X_t)dW^S_t
	\end{equation*}
	where $dW^S_t$ is a $n$-dimensional standard Brownian motion. Under the $U$-measure, we have
	\begin{equation}
	\mu^U_t(X_t) = \mu^S_t(X_t) - \rho\sigma(X_t)\left(\frac{\sigma^S_t}{S_t}-\frac{\sigma^U_t}{U_t}\right)
	\end{equation}
	or
	\begin{equation}
	dW^U_t = dW^S_t + \rho\left(\frac{\sigma^S_t}{S_t}-\frac{\sigma^U_t}{U_t}\right) dt
	\end{equation}
	$\rho$ is the correlation matrix of $<dW^S,dW^U>$ and $\sigma^S_t$ and $\sigma^U_t$ are the (vector) volatilities of numeraires $S$ and $U$. %(one component for each Brownian motion).
	\end{block}
	%Exercise: calculate the drift change for the Vasicek Model in case you change from B(t) to P(0,T) numeraire
\end{frame}

\begin{frame}{Drift Changes (Proof)}
	We now provide a formal proof of the above proposition in the special case of \textcolor{red}{$n=1$}, in which \textcolor{red}{$\rho=1$}.
	
	Indicate by $\mathcal{Q}^S$ and $\mathcal{Q}^U$ the $S$-measure and $U$-measure. By Girsanov theorem we have the following expression for the Radon-Nikodym derivative
	\begin{equation*}
	\zeta_t = \frac{d\mathcal{Q}^S}{d\mathcal{Q}^U} = e^{-\frac{1}{2}\int_0^t\zeta_s^2 ds + \int_0^t\zeta_s dW_s^U}
	\end{equation*}
	with 
	\begin{equation*}
	\alpha_t=\frac{\mu^S_t(X_t)-\mu_t^U(X_t)}{\sigma_t(X_t)}
	\end{equation*}
	We also know that $\zeta_t$ is an exponential martingale hence its dynamics is such that 
	\begin{equation}
	d\zeta_t=\alpha_t\zeta_tdW_t^U
	\label{eq:dzeta1}
	\end{equation}
\end{frame}

\begin{frame}{Drift Changes (Proof)}
	By the main theorem on numeraire change Eq.~\ref{eq:radon_nikodym_der2}, and using the fact that $Z_t$ is a $\mathcal{Q}^U$ martingale, 
	\begin{equation}
	\zeta_t = \frac{d\mathcal{Q}^S}{d\mathcal{Q}^U} = \frac{U_0S_t}{S_0U_t}
	\label{eq:zeta_numeraire}
	\end{equation}
	thus
	\begin{equation}
	d\zeta_t= \frac{U_0}{S_0}d\left(\frac{S_t}{U_t}\right)= \frac{U_0}{S_0}\sigma_t^{S/U}dW_t^U
	\label{eq:dzeta2}
	\end{equation}
	where $\sigma^{S/U}_t$ is the volatility of the process $S_t/U_t$, which is also a martingale under $\mathcal{Q}^U$. Comparing the two results for $d(S_t/U_t)$ (Eqs.~\ref{eq:dzeta1},~\ref{eq:dzeta2} and using Eq.~\ref{eq:zeta_numeraire}) we get
	\begin{equation}
		\begin{gathered}
		\alpha_t\zeta_tdW_t^U = \alpha_t\frac{\cancel{U_0}S_t}{\cancel{S_0}U_t}\cancel{dW_t^U}=	\frac{\cancel{U_0}}{\cancel{S_0}}\sigma^{S/U}_t\cancel{dW_t^U} \\
		\alpha_t = \frac{U_t}{S_t}\sigma^{S/U}_t
		\end{gathered}
	\end{equation}
\end{frame}

\begin{frame}{Drift Changes (Proof)}
	Using the definition of $\alpha_t$
	\begin{equation}
	\mu_t^U(X_t)=\mu_t^S(X_t)-\frac{U_t}{S_t}\sigma_t(X_t)\sigma^{S/U}_t
	\label{eq:alpha}
	\end{equation}
	
	Now let $S_t$ and $U_t$ have dynamics under $\mathcal{Q}^U$ given by 
	\begin{equation*}
		\begin{gathered}
			dS_t = (\ldots) dt + \sigma^S dW^U_t\\
			dU_t = (\ldots) dt + \sigma^U dW^U_t 
		\end{gathered}
	\end{equation*}
	
	From Leibniz rule of stochastic calculus and using Ito's lemma (neglecting higher order terms)
	\begin{equation*}
		\begin{gathered}
		d\left(\frac{S_t}{U_t}\right)=\frac{1}{U_t}dS_t+S_td\frac{1}{U_t}+dS_td\frac{1}{U_t} \\
		d\frac{1}{U_t}=-\frac{1}{U^2_t}dU_t+\cancel{\frac{1}{U^3_t}dU_tdU_t}
		\end{gathered}
	\end{equation*}
\end{frame}

\begin{frame}{Drift Changes (Proof)}
	Replacing the dynamics for $S_t$ and $U_t$ (ignoring the terms in $dt$ since we know that $d\frac{S_t}{U_t}$ is a martingale)
	\begin{equation}
		d\left(\frac{S_t}{U_t}\right) = \frac{\sigma^S_t dW^U_t}{U_t} - \frac{S_t}{U^2_t}\sigma^U_t dW^U_t
	\end{equation}
	Taking $d(S_t/U_t)$ definition from Eq.~\ref{eq:dzeta2}
	\begin{equation}
	\sigma_t^{S/U} = \frac{\sigma^S_t}{U_t} - \frac{S_t}{U^2_t}\sigma^U_t
	\end{equation}
	Replacing above expression for $\sigma^{S/B}$ in Eq.~\ref{eq:alpha}
	\begin{equation}
		\begin{aligned}
			\mu_t^U(X_t)&=\mu_t^S(X_t)-\frac{U_t}{S_t}\sigma_t(X_t)\left(\frac{\sigma^S_t}{U_t} - \frac{S_t}{U^2_t}\sigma^U_t\right)\\
			&=\mu_t^S(X_t)-\sigma_t(X_t)\left(\frac{\sigma^S_t}{S_t} - \frac{\sigma^U_t}{U_t}\right)
		\end{aligned}
	\end{equation}
	\textcolor{red}{which proves the first part of the statement}.
\end{frame}

\begin{frame}{Drift Changes (Proof)}
	Expressing $\alpha_t$ coefficient in terms of the numeraires volatilities
	\begin{equation}
	\begin{aligned}
		\alpha_t = \frac{\mu_t^S(X_t) - \mu_t^U(X_t)}{\sigma_t(X_t)} = \left(\frac{\sigma^S_t}{S_t} - \frac{\sigma^U_t}{U_t}\right)
	\end{aligned}
	\end{equation}
	Finally from the Girsanov theorem we get the new diffusion under the new numeraire
	\begin{equation}
		\begin{gathered}
		dW^U_t = dW^S_t - \alpha_t dt \\
		dW^U_t = dW^S_t - \left(\frac{\sigma^S_t}{S_t}-\frac{\sigma^U_t}{U_t}\right) dt
		\end{gathered}
	\end{equation}
	\textcolor{red}{which proves also the second part of the proposition}.
	
	As an exercise, once you know the Vasicek short rate model, try to determine the new drift when moving from bank account to forward meeasure.%the result is an application of the previous formula with $X = r$, $Q = P^T$ , $\sigma(X_t,t) = \sigma$, $\sigma_B (t) = -A(t, T)\sigma$ and $m(X_t,t) = a(b-rt)$.
\end{frame}


%\begin{frame}{}
%	Let $\mathcal{P}$ be an equivalent martingale measure (EMM) for the numeraire $H_t$ and $\mathcal{Q}$ an EMM for the numeraire $J_t$.Then given a probability space $(\Omega , \mathcal{F}_T , \mathcal{P}/\mathcal{Q})$, for the numeraire change theorem holds 
%	\begin{equation}
	%			V_t = \mathbb{E}^P\left[V_T\frac{H_t}{H_T}|\mathcal{F}_t\right]= \mathbb{E}^Q\left[V_T\frac{J_t}{J_T}|\mathcal{F}_t\right]
	%		\end{equation}	
%	Assume that $\mathcal{P}$ and $\mathcal{Q}$ are equivalent and denote the Radon-Nikodym by $\eta_t$. 
%	We then have
%	\begin{equation}
	%			V_t = \mathbb{E}^Q\left[V_T\frac{J_t}{J_T}|\mathcal{F}_t\right]= \mathbb{E}^P\left[V_T\frac{H_t}{H_T}\eta_t|\mathcal{F}_t\right]
	%		\end{equation}
%\end{frame}

%\begin{frame}{Drift Change General Form}
%	In particular, $\eta_t = \frac{H_TJ_t}{H_tJ_T}$ for $t < T$. 
%	
%	Suppose that the process under some measure $\mathcal{Q}^H$ associated with the numeraire $H_t$ is given by $dX_t = \mu(X_t,t)dt +
%	\sigma (X_t,t)dW^H_t$.
%	
%	We are interested on the process followed by $X_t$ under another measure $\mathcal{Q}^J$ with numeraire $J_t$. 
%	
%	From Girsanov’s theorem, 
%	\begin{equation}
%		\eta_t = \frac{d\mathcal{Q}^H}{d\mathcal{Q}^J}|\mathcal{F}_t=\exp(-\frac{1}{2}\int_0^t [\mu^H-\mu^J]^2/\sigma ds ) + \int_0^t [\mu^H-\mu^J]/\sigma dW^J_s
%	\end{equation}
%	we know that $\eta$ is an exponential martingale with dynamics 
%	\begin{equation}
%		d\eta_t = \alpha_t\eta_tdW^J_t\quad \left(\text{with } 	\alpha_t = [\mu^H-\mu^J]/\sigma \right)
%	\end{equation}
%\end{frame}
%
%\begin{frame}{Drift Change General Form}
%	Since 
%	%	Let $\mathcal{P}$ be an equivalent martingale measure (EMM) for the numeraire $H_t$ and $\mathcal{Q}$ an EMM for the numeraire $J_t$.Then given a probability space $(\Omega , \mathcal{F}_T , \mathcal{P}/\mathcal{Q})$, for the numeraire change theorem holds 
%	%	\begin{equation}
%		%			V_t = \mathbb{E}^P\left[V_T\frac{H_t}{H_T}|\mathcal{F}_t\right]= \mathbb{E}^Q\left[V_T\frac{J_t}{J_T}|\mathcal{F}_t\right]
%		%		\end{equation}	
%	%	Assume that $\mathcal{P}$ and $\mathcal{Q}$ are equivalent and denote the Radon-Nikodym by $\eta_t$. 
%	%	We then have
%	%	\begin{equation}
%		%			V_t = \mathbb{E}^Q\left[V_T\frac{J_t}{J_T}|\mathcal{F}_t\right]= \mathbb{E}^P\left[V_T\frac{H_t}{H_T}\eta_t|\mathcal{F}_t\right]
%		%		\end{equation}
%	%\end{frame}
%	
%	We also know that 
%	
%	\begin{equation}
%		\eta_T = \frac{J_0H_T}{H_0J_T}
%	\end{equation}
%	Since $\eta$ is a $\mathcal{Q}^J$ martingale
%	\begin{equation}
%		d\eta_t = \mathbb{E}^J[\eta_T]=\frac{J_0H_t}{H_0J_t}
%	\end{equation}
%\end{frame}
%
%
%\begin{frame}{Drift Change General Form}
%	Differentiating
%	\begin{equation}
%		d\eta_t =\frac{J_0}{H_0}d\left(\frac{H_t}{J_t}\right)=\frac{J_0}{H_0}\sigma^{H/J}dW^J_t
%	\end{equation}
%	since $H/J$ is a martingale under $J$ and $d(\frac{H_t}{J_t})=\sigma^{H/J}dW^J_t$.
%	By comparing the two dynamics for $\eta$
%	\begin{equation}
%		\alpha\eta_t = \frac{J_0}{H_0}\sigma^{H/J}\implies \frac{H_t}{J_t}\alpha=\sigma^{H/J}
%	\end{equation}
%	or by definition
%	\begin{equation}
%		\mu^J = \mu^H - \frac{J_t}{H_t}\sigma_t\sigma^{H/J}
%	\end{equation}
%\end{frame}
%
%
%
%\begin{frame}{Drift Change General Form}
%	The last step
%	
%	\begin{equation}
%		\sigma^{H/J} = \frac{\sigma_H}{J_t} - \frac{H_t}{J^2_t}\sigma_J
%	\end{equation}
%	
%	dWJ = -g dt + dWH
%	
%	If $W^J_t$ is a Wiener process under $\mathcal{Q}^J$, then $W^J_t = W^H_t - \int_0^t \theta_s ds$ where $d\eta_{t,T} = \Gamma_{t,T} \theta_T dW^P_T$.
%	
%	Let $H_t$ and $J_t$ have dynamics under $\mathcal{P}$ given by 
%	\begin{equation}
%		\begin{gathered}
%			dH_t = m_H dt + \sigma_H dW^P_t\\
%			dJ_t = m_J dt + \sigma_J dW^P_t 
%		\end{gathered}
%	\end{equation}
%\end{frame}
%
%Using these dynamics and noting that $\eta_t$ is a martingale under $\mathcal{P}$, it can be verified that
%\begin{equation}
%	\frac{d\eta_T}{\eta_T} = \left(\frac{\sigma_J}{J_T}-\frac{\sigma_H}{H_T}\right)dW^P_T
%\end{equation}
%
%So, $\theta_t = \frac{\sigma_J}{J_T}-\frac{\sigma_H}{H_T}$.
%Suppose $\mathcal{P}$ is the measure induced by the bank account numeraire and $\mathcal{Q}$ is the forward measure. For $s < t < T$
%\begin{equation}
%	\Gamma_t = \frac{J_t}{J_s}\frac{H_s}{H_T} = \frac{P(t,T)}{P(s,T)}\exp\left(-\int_s^t r_u du\right)
%\end{equation}
%
%Under measure $\mathcal{P}$, $\frac{dP(t,T)}{P(t,T)} = r_t dt + \sigma_P(t)dW_t$.
%It is a straightforward calculation to show that the process $\Gamma_t$ satifies
%\begin{equation}
%	\frac{\Gamma_t}{\Gamma_t}=\frac{dP(t,T)}{P(t,T)} - r_t dt = \sigma_P(t)dW_t
%\end{equation}
%
%This implies that$W^Q_t = W^P_t-\int_0^t\sigma_B(u)du$. Hence,if under $\mathcal{P}$ we have the dynamics $dX_t = m(X_t,t)dt + \sigma(X_t,t)dW_t^P$ then the $\mathcal{Q}$~-process for $X_t$
%is $dX_t =(m(X_t,t)+\sigma_B(t)\sigma(X_t,t))dt+\sigma(X_t,t)dW^Q_t$.

\end{document}

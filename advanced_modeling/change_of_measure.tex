\documentclass{beamer}
\usetheme{CambridgeUS}

\title{Libor Market Model}
\author{Matteo Sani}
\begin{document}
	\begin{frame}[plain]
		\maketitle
	\end{frame}


\begin{frame}{Intuition behind the Mathematics}
	\begin{itemize}
		\item The following theorem is at the heart of the n\emph{no arbitrage} pricing theory
		\begin{block}{}
		The market model does not allow for arbitrage if and only if there exist a martingale measure $\mathcal{Q}^0$ under which the processes
		\begin{equation}
		\frac{S_0(t)}{S_0(t)},\frac{S_1(t)}{S_0(t)},\ldots,\frac{S_N(t)}{S_0(t)}
		\end{equation}
		behave as martingales under $\mathcal{Q}^0$.
		\end{block}
	\end{itemize}
\end{frame}

\begin{frame}{Intuition behind the Mathematics}
	\begin{itemize}
		\item Moreover
		\begin{block}{}
			In order to avoid arbitrages on the markets a generic derivative contract with a known payoff at time $T$ must be priced according to the following expression
			\begin{equation}
				\Pi(t, X) = S_)(t)\mathbb{E}^0\left[\frac{X}{S_0(T)}\right]
			\end{equation}
			where $\mathbb{E}^0$ stands for the expectation operator under $\mathcal{Q}^0$.
		\end{block}
	\item Until now we have used $B(t)$ as numeraire. But it does not represent the only possible choice.
	\end{itemize}
\end{frame}

\begin{frame}{Implications of the Fundamental Theorem}
	\begin{itemize}
		\item The price of any asset divided by an active reference follows a martingale process under the measure associated with that numeraire.
		\item Changing the numeraire implies a change of the probability measure.
		\item It remains to understand how to pass from a numeraire to another, and hence by a measure to another in an arbitrage free setting.
	\end{itemize}
\end{frame}

\begin{frame}{Example Approach to Numeraire Change}
\begin{itemize}
	\item In lieu of the fundamental theorem we can write
	\begin{equation}
	\Pi(0,X)=S_0(0)\mathbb{E}^0\left[\frac{X}{S_0(T)}\right]
	\end{equation}
	\item But also
	\begin{equation}
	\Pi(0,X)=S_1(0)\mathbb{E}^1\left[\frac{X}{S_1(T)}\right]
	\end{equation}
	\item We define the Radom-Nikodym derivative
	\begin{equation}
	L_0^1(T)=\frac{dQ^1}{dQ^0}
	\end{equation}
\end{itemize}
\end{frame}

\begin{frame}{Example Approach to Numeraire Change}
	\begin{itemize}
		\item Hence we can write
		\begin{equation}
			\Pi(0,X)=S_1(0)\mathbb{E}^0\left[\frac{X}{S_1(T)}L_0^1(T)\right]
		\end{equation}
		\item After some trivial manipulations
		\begin{equation}
			S_0(0)\mathbb{E}^0\left[\frac{X}{S_0(T)}\right]=
			S_1(0)\mathbb{E}^0\left[\frac{X}{S_1(T)}L_0^1(T)\right]
		\end{equation}
		\item Finally
		\begin{equation}
			\frac{S_0(0)}{S_0(T)}=\frac{S_1(0)}{S_1(T)}L_0^1(T)		
		\end{equation}
	\end{itemize}
\end{frame}

\begin{frame}{Example Approach to Numeraire Change}
	\begin{itemize}
		\item Hence 
		\begin{equation}
			L_0^1(T) = \frac{dQ^1}{dQ^0}=			\frac{S_0(0)S_1(0)}{S_1(T)S_0(T)}
		\end{equation}
	\item In general the theorem of the following page holds and gives a simple rule to wrtie (the otherwise diffiult to derive) Radon-Nykodym derivative
	\end{itemize}
\end{frame}

\begin{frame}{The Theorem}
\begin{block}{}
Let $\mathcal{Q}^0$ a martingale measure associated with the numeraire $S_0$ and further suppouse that $S_1$ is the positive process of an asset such that
\begin{equation}
\frac{S_1(t)}{S_0(t)}
\end{equation}
is a martingale under $\mathcal{Q}^0$. Define $\mathcal{Q}^1$ by means of the Radom-Nikodym derivative
		\begin{equation}
	L_0^1(t) = \frac{S_0(0)S_1(0)}{S_1(t)S_0(t)}
\end{equation}
Then $\mathcal{Q}^1$ is a martingale measure for $S_1$.
\end{block}
\end{frame}

\begin{frame}{Intuition from Expected Value}
\begin{itemize}
	\item Notice that we can write the expected value of a generic derivative $\Pi(x)$ under a measure $F$, with associated density function $f(x)$ as
	\begin{equation}
		\mathbb{E}^{F}=\int\Pi(x)f(x)dx
	\end{equation}
	\item Suppouse there exists a function $g(x)$, which satisfies the mathematical conditions required to be a density function. Then we can write
	\begin{equation}
		\mathbb{E}^{F}=\int\Pi(x)f(x)\frac{g(x)}{g(x)}dx
	\end{equation}
	\item If we define $\psi(x)=\Pi\frac{f(x)}{g(x)}$ the expected value can be written as 
	\begin{equation}
		\int\psi(x)g(x)dx=\mathbb{E}^G\left[\psi(x)\right]=\mathbb{E}^G\left[\Pi(x)\frac{f(x)}{g(x)}\right]=\mathbb{E}^F\left[\Pi(x)\right]
	\end{equation}
\end{itemize}
\end{frame}

\begin{frame}{Catching Up}
\begin{itemize}
	\item The power of the previous theorem stems from the fact that we can find a characterization of our process by means of which we can work more easily.
	\item For instance, to change from the measure induced by numeraire $B(t)$, the bank account numeraire to the one induced by the $P(t,T)$, the Radom-Nikodym derivative, $\frac{dQ^T}{dQ}$ can be obtained as
	\begin{equation}
		\frac{P(t,T) \cdot 1}{B(t)P(0,T)}
	\end{equation}
	\item To prove the result, just put $S_0=B$ and $S_1=P$.
	\item Changing measures change the SDE which describes asset prices. The way in which this happens is characterized by the Girsanov theorem.
\end{itemize}
\end{frame}

\begin{frame}{A Trick for a Useful Separation}
\begin{itemize}
	\item Absence of arbitrage implies the existence of a martingale measure $\mathcal{Q}$ such that, for every contingent claim whose payoff at time $T$ is $\chi$, we have the following formula for its price $\Pi$
	\begin{equation}
	\Pi(0,\chi)=\mathbb{E}^{\mathcal{Q}}\left[e^{-\int_0^T r_s ds}\chi\right]
	\end{equation}
	\item The main issue with this formula resides in the interaction between the two terms, the discount factor $e^{-\int_0^Tr_sds}$ and the contingent claim payoff $\chi$.
	\item We then devise a technique which allows the separation of the two.
\end{itemize}
\end{frame}

\begin{frame}{The Actual Trick}
\begin{itemize}
	\item If instead of the bank account, $P(t,T)$ is choosen ad numerarire, the derivative price can be discounted with an observable quantity and there is no need to mind the discounting. It is enough to estimate the expected value of $\chi(T)$ under the new measure induced by $P(t,T)$	
	\item If a closed form expression cannot be found, simulation cannot be avoided, but however we don-t need it for the discount factor, as it happens when $B(t)$ is the numeraire.
	\item Obviously the drift of the SDE of the asset changes. The change of drift is characterized by the Girsanov theorem.
	\item As we will see $\lambda(t)$, the market price of risk, is the object connecting drifts across different measures.
\end{itemize}
\end{frame}

\begin{frame}{Numeraires}
	\begin{block}{Definition}
		A numeraire is any positive non-dividend-paying asset.
	\end{block}
	\begin{itemize}
		\item In general, a numeraire is a portfolio or a self-financing strategy.
		\item The condition for a strategy to be self-financing in an economy with $K$ assets is
		\begin{equation}
			dV_t = \sum_{k=0}^K \phi^k_t dS^k_t
		\end{equation}
		which implies that form every numeraire $Z_t$
		\begin{equation}
			d\left(\frac{V_t}{Z_t}\right) = \sum_{k=0}^K \phi^k_t d\left(\frac{S^k_t}{Z_t}\right)
		\end{equation}
	\end{itemize}
\end{frame}

\begin{frame}{Numeraire Pricing}
	\begin{block}{Theorem (German, El Karoui and Rochet, 1995)}
		Assume that there exists a numeraire $N$ and a probability measure $\mathcal{Q}^N$ which is equivalente to $\mathcal{P}$ such that, for every traded asset $X$:
		\begin{equation}
			\frac{X_t}{N_t} = \mathbb{E}^{\mathcal{Q}^N}\left[\frac{X_T}{N_T}|\mathcal{F}_t\right]
		\end{equation}
		Now, given a second arbitrary numeraire $U$, there exists a probability measure $\mathcal{Q}^U$ which is equivalent to $\mathcal{P}$ and such that:
		\begin{equation}
			\frac{X_t}{U_t} = \mathbb{E}^{\mathcal{Q}^U}\left[\frac{X_T}{U_T}|\mathcal{F}_t\right]
		\end{equation}
	\end{block}
\end{frame}

\begin{frame}{Clarificaion on Time}
\begin{itemize}
	\item Clearly as the Radon-Nikodyn derivative is a martingale for valuation time $t$, we have
	\begin{equation}
		\frac{d\mathcal{Q}^U}{d\mathcal{Q}^N}=\frac{U_tN_0}{U_0N_t}
	\end{equation}
\item So do not confuse the maturity of the numeraire bond $T$ with the times at which ou have to take the values of the numeraire, in this case $t$ and 0.
\item If you want to switch from the $T$ measure to the $S$ measure, i.e. the one induced by the bond $P(.,S)$, for the valuation time $t$ we get
\begin{equation}
\frac{d\mathcal{Q}^S}{d\mathcal{Q}^T}=\frac{P(t,S)P(0,T)}{P(t,T)P(0,S)}
\end{equation}
\end{itemize}
CLARIFY
\end{frame}

\begin{frame}{title}
\begin{itemize}
	\item Absence of arbitrage insures the existence of at least one numeraire: the money market account.
	\item The numeraire theorem allows other numeraires to be used to price a contingent claim.
	\item It is then natural to look for the most convenient numeraire, that one which minimizes the mathematical difficulties.
	\item When changing numeraires, also the drift will change (convexity adjustments).
\end{itemize}
\end{frame}

\begin{frame}{title}
\begin{itemize}
	\item Following Brigo-Mercurio (2006) we can state: The risk neutral (so arbitrage free) price is invariant by change of numeraire, i.e. the price of a pyoff $\Psi(T)$ can be written on both ways
	\begin{equation}
	\mathbb{E}^B\left[B(t)\frac{\Psi(T)}{B(T)}\right] = 	\mathbb{E}^S\left[S(t)\frac{\Psi(T)}{S(T)}\right]
	\end{equation}
	\item So changing $B$ with $S$ the price does not change.
	\item So we choose the one which is easier to calculate given the problem at hand.
\end{itemize}
\end{frame}

\begin{frame}{title}
	\begin{itemize}
		\item In many concrete situations, the best numerarie is the ZCB with the same maturity of the derivative to price.
		In this case $S_t = P(T,T)=1$.
		\item The forward measure $\mathcal{Q}^T$ (the T-measure) is defined as the martingale measure for the numeraire process $P(t,T)$, where $P(t,T)$ is the ZCB maturing in T.
		\item It is easy to see that, in this case the Radon-Nykodin derivative is given by
		\begin{equation}
			\frac{d\mathcal{Q}^T}{d\mathcal{Q}} = \frac{P(t,T)}{B_t P(0,T)} 
		\end{equation}
		\item Using the pricing formula after the change of numeraire, we finally have
		\begin{equation}
			\Pi(0,\chi)=P(t,T)\mathbb{E}^{\mathcal{Q}^T}[\chi]
		\end{equation}
		which achieves the desired separation (although under a new measure).
		\item Notice that if the interest rates are deterministic, then $\mathcal{Q} = \mathcal{Q}^T$
	\end{itemize}
\end{frame}

\begin{frame}{The Expectation Hypothesis}
	\begin{itemize}
		\item It is possible to prove the following
		\begin{equation}
			f(t, T) = \mathbb{E}^{\mathcal{Q}^T}[r(T)|\mathcal{F}_t]
		\end{equation}
		\item According to the pure expectation hypothesis, the above formula is valid if the expected value is taken under the real probability.
		\item Absence of arbitrage makes this incompatible with stochastic interest rates.
	\end{itemize}
	DA CAPIRE MEGLIO
\end{frame}

\begin{frame}{Drift Changes}
	\begin{itemize}
		\item Assume that, under the $S$-measure, we have
		\begin{equation}
			dX_t = \mu^S(X_t)dt + \sigma(X_t)dW^S_t
		\end{equation}
		where $dW^S_t$ is n-dimensional standard brownian motion.
		\item Under the $U$-measure, we have
		\begin{equation}
			\mu^U_t(X_t) = \mu^S_t(X_t) - \rho\sigma(X_t)\left(\frac{\sigma^S_t}{S_t}-{\sigma^U_t}{U_t}\right) 
		\end{equation}
		or
		\begin{equation}
			dW^U_t = dW^S_t + \rho\sigma\left(\frac{\sigma^S_t}{S_t}-{\sigma^U_t}{U_t}\right) 
		\end{equation}
		$\rho$ is the correlation matrix of $<dW^S,dW^U>$ and $\sigma^S_t$ and $\sigma^U_t$ are the (vector) volatilities of numeraires $S$ and $U$ (one component for each brownian motion).
	\end{itemize}
Exercise: calculate the drift change for the Vasicek Model in case you change from B(t) to P(0,T) numeraire
\end{frame}

\begin{frame}{Formal Proof}
We now provide a formal proof of the above statement in the case $n=1$, in which $\rho=1$. To generalize to the multivariate case, see Brigo and Mercurio, Chapter 2.

Indicate by $\mathcal{Q}^S$ and $\mathcal{Q}^U$ the $S$-measure and $T$-measure. By Girsanov theorem we have
\begin{equation}
Z_t = \frac{d\mathcal{Q}^S}{d\mathcal{Q}^U} = e^{-\frac{1}{2}\int_0^t\xi_s^0 ds + \int_0^t\xi_s dW_s^U}
\end{equation}
where 
\begin{equation}
\xi_s=\frac{\mu^S_s(X_s)-\mu_s^U(X_s)}{\sigma_s(X_s)}
\end{equation}
is such that 
\begin{equation}
dZ_t=\xi_tZ_tdW_t^U
\end{equation}
\end{frame}

\begin{frame}{Formal Proof}
By the main theorem on numeraire change, and using the fact that $Z_t$ is $\mathcal{Q}^U$ martingale, 
\begin{equation}
Z_t = \frac{d\mathcal{Q}^S}{d\mathcal{Q}^U} = \frac{U_0S_t}{S_0U_t}
\end{equation}
thus
\begin{equation}
	dZ_t=\frac{U_0}{S_0}\bar{\sigma_t}dW_t^U
\end{equation}
where $\bar{\sigma_t}$ is the volatility of the process $S_t/U_t$, which is also a martingale under $\mathcal{Q}^U$. Comparing the two results for $dZ_t$ we get
\begin{equation}
\mu_t^U(X_t)=\mu_t^S(X_t)-\frac{U_t}{S_t}\sigma_t(X_t)\bar{\sigma_t}
\end{equation}
\end{frame}

\begin{frame}
\begin{itemize}
	\item Now, just need to compute
	\begin{equation}
		d\frac{S_t}{U_t}\bar{\sigma_t}= \frac{\sigma^S_t}{U_t}-\frac{S_t}{U_t}\frac{\sigma_t^U}{U_t}
	\end{equation}
\item Substitute above and get the result.
\item The calculation is left as an exercise: you could find at the exam: do it !
\item Hint: use the stochastic Leibniz rule and Ito's lemma.
\item Exercise: given $dS(t)=r(t)S(t)dt + \sigma S(t)dW(t)$ under $\mathcal{Q}$ induced by the numeraire $B$, calculate the dynamics if the numeraire is $S$.
\end{itemize}
\end{frame}

\begin{frame}{A General Option Pricing Formula}
	\begin{itemize}
		\item Consider an European call on an asset $S$ which is also a numeraire
		\begin{equation}
			\chi = \max[S_T-K,0]
		\end{equation}
		\item Denote by $\mathcal{Q}^S$ the martingale measure for the numeraire $S$, and by $\mathcal{Q}^T$ the forward measure.
		\begin{block}{Theorem (German, El Karoui and Rochet, 1995)}
			\begin{equation}
				\Pi(0,\chi) = S_0\mathcal{Q}^S(S_T \geq K) - KP(0,T)\mathcal{Q}^T(S_T\geq K)
			\end{equation}
		\end{block}
		\item This is a general-purpouse Black-Scholes formula.
		\item Put prices can be computed by put-call parity.
	\end{itemize}
\end{frame}

\begin{frame}
To prove the theorem, note that the value of the option is given by

\begin{equation}
\Pi(0,\chi) = \mathbb{E}^{\mathcal{Q}}[B_TS_T I_{\{S_T\geq K\}}] - K\mathbb{E}^{\mathcal{Q}}[B_TI_{\{S_T\geq K\}}]
\end{equation}
\end{frame}

\begin{frame}
	\begin{itemize}
		\item To make this formula simpler in a special (but relevant case), we assume that the process 
		\begin{equation}
			Z_t = \frac{S_t}{P(t,T)}
		\end{equation}
		follows the diffusion
		\begin{equation}
			dZ_t = Z_tm_tdt + Z_t\sigma_tdW_t
		\end{equation}
		where $sigma_t$ is non-stochastic (can you quote a famous example of this kind ?).
		\item Under $\mathcal{Q}^T$, $Z-t$ is a martingale, then
		\begin{equation}
			dZ_t = Z_t\sigma_tdW^T_t
		\end{equation}
		which implies
		\begin{equation}
			Z_t = Z_0 e^{-\frac{1}{2}\int_0^T\sigma_t^2 dt + \int_0^T\sigma_tdW^T_t}
		\end{equation}
		which in turn implies that
		\begin{equation}
			\mathcal{Q}^T(S_T\geq K) = \mathcal{N}(d_2)
		\end{equation}
	\end{itemize}
\end{frame}


\begin{frame}{title}
	\begin{itemize}
		\item Now define
		\begin{equation}
			Y_t = \frac{P(t,T)}{S_t}=\frac{1}{Z_t}
		\end{equation}
		Again, under $\mathcal{Q}^S$ $Y_t$ is a martingale, whose volatility must be $-\sigma_t$ by Ito's lemma, so we have
		\begin{equation}
			dY_t = -Y_t\sigma_tdW^S_t
		\end{equation} 
		which implies
		\begin{equation}
			Y_t = Y_0 e^{-\frac{1}{2}\int_0^T\sigma_t^2 dt - \int_0^T\sigma_tdW^S_t}
		\end{equation}
		\item This implies in turn that
		\begin{equation}
			\mathcal{Q}^S(S_T\geq K) = \mathcal{N}(d_1)
		\end{equation}
	\end{itemize}
\end{frame}

\begin{frame}
	\begin{block}{Extended Black and Scholes formula (deterministic volatility)}
		\begin{equation}
			\Pi(0,\chi) = S_0\mathcal{N}(d_1) - KP(0,T)\mathcal{N}(d_2)
		\end{equation}
		where $d_1$ and $d_2$ are computed with $\sigma^2 = \int_0^T \sigma_s^2 ds$
	\end{block}
\end{frame}

\begin{frame}{Toward the Black-76 Formula}
	\begin{itemize}
		\item Consider a caplet resetting at $T_1$ and payed at $T_2$. The payout is 
		\begin{equation}
			\tau(L(T_1,T_2)-X)^+ = \tau(F(T_1,T_1,T_2)-X)^+
		\end{equation}
		\item Its value in $t$ is given by
		\begin{equation}
			\mathbb{E}\left[e^{-\int_t^T2r_s ds}\tau(F(T_1,T_1,T_2)-X)^+\right]
		\end{equation}
		\item Notice that $P(0,T_2)F(T_1,T_1,T_2)$ is a tradable asset since
		\begin{equation}
			P(t,T2)F(T_1,T_1,T_2)=\frac{P(t,T_1)-P(t,T_2)}{\tau}
		\end{equation}
		Hence changing numeraire to $P(t,T_2)$ we get
		\begin{equation}
			P(t,T2)\tau \mathbb{E}^{\mathcal{T_2}}\left[(F(T_1,T_1,T_2)-X)^+\right]
		\end{equation}
		\item Now we assume lognormality of $F(t,T_1,T_2)$ under the $T_2$-measure
		\begin{equation}
			dF(t,T_1,T_2)=v_2F(t,T_1,T_2) dW^{T_2}_t
		\end{equation}
	\end{itemize}
\end{frame}


\begin{frame}
	\begin{itemize}
		\item This implies the Black-76 formula for caps.
		\item The LMM is automatically fitted to cap prices.
		\item To price a more complicated derivative, we have to compute the drift of the forward rates under a unique measure.
		\item For example, $F(t,T_1,T_2)$ is driftless in the $T_2$-measure. In the $T_1$-measure
		\begin{equation}
			dF(t,T_1,T_2)=\frac{v_2F(t,T_1,T_2)\tau}{1+\tau F(t,T_1,T_2)}dt + v_2F(t,T_1,T_2)dW^{T_1}_t
		\end{equation}
		\item We need simulation of the forward rates to price exotic instruments.
	\end{itemize}
\end{frame}


\begin{frame}{The Libor Market Model}
	\begin{itemize}
		\item We now specify the LIBOR market model. We have a set $T_0,\ldots,T_M$ of dates and $\tau_i=T_i-T_{i-1}$. Settlement date is fixed to 0.
		\item Now consider forward rate $F_k(t)=F(t,T_{k-1},T_k)$. They last from $t$ to $T_{k-1}$.
		\item $F_k(t)$ is lognormally distributed under the $T_k$-measure, that is 
		\begin{equation}
			dF_k(t) = \sigma_k(t)F_k(t)dW^k(t),\quad t\leq T_{k-1}
		\end{equation} 
		where $dW^k(t)$ is an M-dimensional brownian motion with correlation matrix $\rho$. $\sigma_k$ is an M-vector of volatilities whose only non-zero entry is at the $k$-th place.
		\item A typical specification of $\sigma_k(t)$ is piece-wise constant.
	\end{itemize}
\end{frame}

\begin{frame}{Decorrelation}
	\begin{itemize}
		\item In the LMM, we can assume that the brownian motions driving the dynamics of forward rate are correlated
		\begin{equation}
			<dW_t^{T_i}, dW_t^{T_i}> = \rho_{ij}dt
		\end{equation}
		\item In models for short rate or for i.f.r. it is assumed full correlation $\rho_{ij}=1$, which is a tight constraint on the dynamics of the forward rates.
		\item In the LMM, we can allow decorrelation to better fit the derivatives at hand.
		\item The change of measure with decorrelation implies
		\begin{equation}
			dF(t,T_1,T_2)=\rho_{12}\frac{v_2F(t,T_1,T_2)\tau}{1+\tau F(t,T_1,T_2)}dt + v_2F(t,T_1,T_2)dW^{T_1}_t
		\end{equation}
		\item Correlations have no impacts on Caps.
	\end{itemize}
\end{frame}


\end{document}

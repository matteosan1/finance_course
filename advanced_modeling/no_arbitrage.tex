\documentclass{beamer}
\usetheme{CambridgeUS}

\usepackage{tikz}
\usetikzlibrary{calc,positioning}

\title{Arbitrage-Free Pricing Theory}
\author{Matteo Sani}
\begin{document}
	\begin{frame}[plain]
		\maketitle
	\end{frame}

\AtBeginSection{
\begin{frame}{Outline}
	\tableofcontents[currentsection]
\end{frame}
}
\section{Fundamental Definitions}
\subsection{Portfolio, Numeraire, Arbitrage\ldots}
\begin{frame}{Portfolio}
	\begin{block}{Portfolio Definition}
		A \textcolor{red}{portfolio} is a vector $\mathbf{\theta}\in \mathbb{R}^K$ whose $j$ components represent the number of shares of asset $A_j$. It's value is
		\begin{equation}
			V_t(\mathbf{\theta}, \omega)=\sum_{j=1}^K\theta_jS^j_t(\omega)
		\end{equation} 
		where $S_t^j$ is the value of $j$th asset, and $\omega$ a market situation. A portfolio is \textcolor{red}{self-financing} if its value changes only due to variations of the asset prices.
	\end{block}
\end{frame}

\begin{frame}{Arbitrage}
	\begin{block}{Arbitrage Definition}
		An \textcolor{red}{arbitrage} is a self financing portfolio that \emph{makes money from nothing}, formally a portfolio $\mathbf{\theta}$ such that
		\begin{equation}
				V_0(\mathbf{\theta}, \omega)\le 0 \text{ and } \mathcal{P}\{V_t(\mathbf{\theta}, \omega) > 0\} > 0
		\end{equation}
	\end{block}
	Informally, an arbitrage is a way to make a guaranteed profit from nothing, by short-selling certain assets at time $t = 0$, using the proceeds to buy other assets, and then settling accounts at time $t$.
\end{frame}

\subsection{Fundamental Theorems of Arbitrage Pricing}
\begin{frame}{Fundamental Theorem of Arbitrage Pricing (I)}
	\begin{block}{Fundamental Theorem of Arbitrage Pricing I}
		There exists a risk-neutral measure if and only \textcolor{red}{if arbitrages do not exist}.
	\end{block}
	\vspace{1cm}
	The market does not allow for risk-free profits with no initial investment.
	Arbitrage opportunities rarely exist in practice. If and when they do, gains are extremely small (not for small investors). Situations when the No-Arbitrage Principle is violated are typically short-lived and difficult to spot. 
	The exclusion of arbitrage in the mathematical model is close enough to reality and turns out to be the most important and fruitful assumption.
\end{frame}

\begin{frame}{Few Definitions}
	\begin{block}{Numeraire Definition}
		A \textcolor{red}{numeraire} is any positive non-dividend-paying asset. It is a reference asset chosen to normalize all other asset prices to it. Having a numeraire allows for the comparison of the value of goods against one another.
	\end{block}
TOGLIERE NUMERAIRE, AGGIUNGERE CONTINGENT CLAIM e SAMPLE SPACE
\begin{block}{Probability Measure}
		A \textcolor{red}{probability measure} is a real-valued function that assignes probabilities to a set of events in a sample space that satisfies measure properties such as countable additivity, and assigning value 1 to the entire space.
	\end{block}	
\end{frame}

\begin{frame}{Risk-Neutral Measure}
Assume today a stock price is $S_0$. In one period from now, the price could be 
\begin{equation*}
	\begin{cases}
		S_0\cdot u = S_u \\
		S_0\cdot d = S_d \\ 
	\end{cases}
\end{equation*}
Assume also that the risk-free rate is $r$.

To \emph{avoid arbitrage opportunities}, conditions must be imposed on $u$ and $d$. (Example: $e^r > u$, I could short the stock in $t_0$ and invest the proceeds $S_0$ into the risk-free account: in both future states in $t_1$, I could buy the stock back for less than my proceeds $S_0e^r$ because $S_u$ and $S_d$ would both be lower. Similarly for $e^r < d$\ldots)

So \textcolor{red}{imposing $d\le e^r \le d$, will ensure no arbitrage}.
\end{frame}


\begin{frame}{Risk-Neutral Measure}
\begin{equation*}
	\begin{aligned}
	S_0 &= \frac{S_0(u-d)e^r}{(u-d)e^r} = \frac{S_0(u-d)e^r + (S_0ud - S_0ud)}{(u-d)e^r}\\
	&= \frac{1}{e^r}\left(\frac{S_0ue^r - S_0ud}{u-d} + \frac{-S_0de^r + S_0ud}{u-d}\right) \\
	&= \frac{1}{e^r}\left(S_0u\frac{e^r - d}{u-d} + S_0d\frac{u - e^r}{u-d}\right)
	\end{aligned}
\end{equation*}
The no arbitrage condition implies the following bounds
\begin{equation*}
0\le\frac{e^r -d}{u-d}\le 1,\quad 0\le\frac{u - e^r}{u-d}\le 1
\end{equation*}
Also
\begin{equation*}
 \frac{e^r -d}{u-d} + \frac{u - e^r}{u-d} = 1
\end{equation*}
\end{frame}

\begin{frame}{Risk-Neutral Measure}
So we can interpret $p_u=\frac{e^r -d}{u-d}$ and $p_d=\frac{u - e^r}{u-d}$ as \textcolor{red}{probability measure}.

\begin{equation*}
S_0 = \frac{S_up_u + S_dp_d}{e^r} = e^{-r}\mathbb{E}[S_1]
\end{equation*}

Notice that we have never talked about the probabilities of the stock going up or down. Every market participant might have her view of the world with different probabilities assigned to the stock going up or down. But the \textcolor{red}{risk-neutral measure} is agreed upon by the market as a whole just as a consequence of no arbitrage.
\end{frame}

\begin{frame}{Risk-Neutral Measure}
	\begin{block}{Price Uniqueness}
	Assume there exists a \textcolor{red}{risk-neutral measure} $\mathcal{Q}$ %on the set $\Omega$ of possible market scenarios 
	and let $A$ be an asset. Then, for each time $t$, $0\le t\le T$ there exists a unique price $\pi_t$ associated with $A$
	\begin{equation}
		\pi_t = \mathbb{E}[D(t,T)A|\mathcal{F}_t]
		\label{eq:risk_neutral_pricing}
	\end{equation}
\end{block}
Such a price is given by the expectation of the discounted payoff under the measure $\mathcal{Q}$.
\end{frame}

\begin{frame}{Hedging}
	\begin{itemize}
		\item A portfolio $\mathbf{\theta}$ in the assets $\mathbf{A}$ is a \textcolor{red}{replicating portfolio} for the asset $B$ if
		\begin{equation}
			S_t^{B}(\omega_i) = \sum_{j=1}^K \theta_j S_t^j(\omega_i)\quad\forall i=1,2,\ldots,N
		\end{equation}
		\item In particular if the market is \emph{arbitrage-free} then the relation holds for all $t$.
		\item The importance of replicating portfolios is that they enable financial institutions that sell
		asset $B$ (e.g. a call options) to \textcolor{red}{hedge}: for each sold share of asset $B$, buy $\theta_j$ shares
		of asset $A_j$ and hold them to time $t + 1$. Then at time $t + 1$, 
		\begin{equation*}
			\text{net gain }= \text{ net loss } = 0
		\end{equation*}
	\end{itemize}
\end{frame}

\begin{frame}{Fundamental Theorem of Arbitrage Pricing (II)}
	\begin{itemize}
		\item In some circumstances, an arbitrage-free market may admit more than one risk-neutral measure, i.e. \textcolor{red}{incomplete markets}.
		\item By contrast, a \textcolor{red}{complete market} is one that has a unique risk-neutral measure.
	\end{itemize}
	\begin{block}{Completeness Theorem Definition}
		Let $\mathcal{M}$ be an arbitrage-free market with a risk-less asset. If for every derivative security there is a replicating portfolio in the assets $A_j$ then the market $\mathcal{M}$ is complete. Conversely, if the market $\mathcal{M}$ is complete, and if the unique risk-neutral measure $\mathcal{P}$ gives positive probability to every market scenario $\omega$, then for every
		derivative security there is a replicating portfolio in the assets $A_j$.
	\end{block}
\end{frame}

\begin{frame}{Summary of Basic Definitions}
	\begin{itemize}
		\item The market is free of arbitrage if (and only if) there exists an \textcolor{red}{equivalent martingale measure} (i.e. a risk-neutral measure).
		\item The market is complete if and only if the martingale measure is unique.
		\item In an arbitrage-free market the price of any derivative is uniquely given, either by the value of the associated replicating strategy, or by the expectation of the discounted payoff under the risk-neutral measure. 
	\end{itemize}
\end{frame}

\subsection{Money Market Account}
\begin{frame}{Money Market Account}
	\begin{itemize}
		\item A \textcolor{red}{money market account} represents a risk-less investment, where profit is accrued continuously at the risk-free rate, and its value is denoted by $B(t)$.
		\item We assume $B(0)=1$ and by definition
		\begin{equation}
			dB(t) = r(t)B(t)dt
		\end{equation}
		\item This evolution process has the solution 
		\begin{equation}
			B(t) = \exp\left(\int_0^t r_s ds\right)
		\end{equation}
		where $r_t$ is referred to as \textcolor{red}{instantaneous spot rate} or  \textcolor{red}{short rate}.
	\end{itemize}
\end{frame}

\begin{frame}{Money Market Account}
	\begin{itemize}
		\item In case $r_t$ is \emph{deterministic}, from the definition of money market account it follows that 
		\begin{equation*}
			V(0) = A \implies V(t) = A\cdot B(t)
		\end{equation*}
		\item If we want at time $T$ exactly 1 unit of currency
		\begin{equation*}
			AB(T) = 1 \implies AB(t) = \frac{B(t)}{B(T)} 
		\end{equation*}
		hence $\frac{B(t)}{B(T)}$ is the value of one unit of currency payable at time $T$ seen from $t$.
		
		%%		\item We now define the abstract quantity $r(t)$, the \textbf{short rate}, as
		%%		\begin{equation}
			%%			r(t) = \lim_{T\rightarrow t^+} L(t,T) \simeq L(t, t+\epsilon)
			%%		\end{equation}\quad with $\epsilon$ small
	\end{itemize}
\end{frame}

\subsection{Stochastic Discount Factor and Zero Coupon Bond}
\begin{frame}{Stochastic Discount Factor}
	\begin{itemize}
		\item The \textcolor{red}{(stochastic) discount factor} $D(t, T)$ is the amount at time $t$ that is \emph{equivalent} to one unit of currency payable at time $T$ and is given by
		\begin{equation}
			D(t, T) = \frac{B(t)}{B(T)} = e^{-\int_t^T r_s ds}
		\end{equation}
		\item Can you guess which are its dimension ?	
		\item In many pricing application (e.g. Black-Scholes formula) $r$ is assumed to be a deterministic function of time, and so are $B(t)$ and $D(t,T)$.
		\begin{itemize}
			\item This is motivated by the small influence interest rate variations have on equity prices.
		\end{itemize}
		\item When dealing with interest rate products, $r$ becomes the main actor, so the deterministic assumption must be dropped.
	\end{itemize}	
\end{frame}

\begin{frame}{Money Market Account as Numeraire}
	In the following we are going to use $B(t)$ as numeraire then  Eq.~\ref{eq:risk_neutral_pricing} becomes
	\begin{equation*}
		\begin{aligned}
			S_t = \mathbb{E}^{\mathcal{Q}^B}\left[e^{-\int_t^T r_s ds}S_T|\mathcal{F}_t\right]
		\end{aligned}
	\end{equation*}
	where $\mathcal{Q}^B$ is the risk-neutral measure.% that makes $\frac{S_t}{B_t}$ a martingale. 
	
	When \textcolor{red}{interest rates are deterministic}, the exponential can be brought out of the expectation
	\begin{equation*}
		S_t = e^{-\int_t^T r_s ds} \mathbb{E}^B\left[S_T|\mathcal{F}_t\right]
	\end{equation*}
	And finally when \textcolor{red}{rates are constant} (e.g. Black-Scholes or Heston models)
	\begin{equation*}
		S_t = e^{-r(T-t)}\mathbb{E}^B\left[S_T|\mathcal{F}_t\right]
	\end{equation*}
\end{frame}

\begin{frame}{Zero Coupon Bond}
	\begin{itemize}
		\item \textcolor{red}{Zero Coupon Bond} (ZCB) is a contract that pays one unit of money at time $T$. Its price at time $t$ is denoted by $P(t,T)$, and by definition $P(T,T) = 1$.
		\item What is the relation between $P(t,T)$ and $D(t,T)$ ? \\
		\renewcommand{\arraystretch}{1.2}
		\footnotesize{\tiny {\tiny }}{
			\begin{table}[bt]
				\begin{tabular}{|c|c|} \hline
					$D(t, T)$ & $P(t, T)$ \\ \hline
					equivalent amount of money & value of a contract \\ \hline
					\multicolumn{2}{|c|}{\textcolor{red}{deterministic rates}} \\ \hline
					\multicolumn{2}{|c|}{D(t,T)=P(t,T)} \\ \hline 
					\multicolumn{2}{|c|}{\textcolor{red}{stochastic rates}} \\ \hline
					random quantity at time $t$ &
					\renewcommand{\arraystretch}{1.0} 
					\begin{tabular}{@{}c@{}}
						being the value of a contract\\ with payoff at time $T$ must be known in $t$ \\
					\end{tabular} \\ \hline
				\end{tabular}
			\end{table}
		}
		\item We will see that in absence of arbitrage
		\begin{equation*}
			P(t, T) = \mathbb{E}^Q[D(t, T)|\mathcal{F}_t]
		\end{equation*}	
	QUANDO ??????
	\end{itemize}
\end{frame}

\begin{frame}{Intermezzo}
	\begin{itemize}
		\item Time to maturity: $T-t$
		\item Year fraction, day-time convention. Denote by $\tau(t, T)$ the chosen time measure between $t$ and $T$. In theory, you can think of $\tau(t, T)=T-t$. In practice, among the plethora of convention, the following are worth mentioning:
		\begin{itemize}
			\item Actual/365
			\item Actual/360
			\item 30/360
		\end{itemize}
		\item \textcolor{red}{Go and find their definitions: they are also embedded in Excel as financial functions.}
	\end{itemize}
	\begin{tikzpicture}[remember picture,overlay]
	\node[xshift=5cm,yshift=-3.7cm] (image) at (current page.center) {\includegraphics[width=20px]{python_logo}};
	\node[align = center, yshift=1.45cm, below=of image] {\tiny{\href{shorturl.at/chzT2}{shorturl.at/chzT2}}};
	\end{tikzpicture}
\end{frame}

\subsection{Spot Interest Rate}
\begin{frame}{Compounding}
	\begin{itemize}
		\item Simply-compounded \textcolor{red}{spot interest rate}
		\begin{equation}
			L(t,T)=\frac{(1-P(t,T))}{\tau(t,T)P(t,T)}		
		\end{equation}
		\item The \textcolor{red}{yield curve} at time $t$ is the graph of:
		\begin{equation*}
			T\rightarrowtail L(t, T)
		\end{equation*}
		\item In the market (and in the \emph{books}) these are the so called LIBOR (or EURIBOR) rates and are typically compounded with the actual/360 convention. They are the main rates underlying interest rate derivatives.
	\end{itemize}
\end{frame}

\begin{frame}{Compounding}
	In the following we recall some useful definition \textbf{you should be always familiar with}.
	\begin{itemize}
		\item Annually-compounded spot interest rate
		\begin{equation}
			Y(t, T)= \frac{1}{P(t, T)^{\frac{1}{\tau(t,T)}}}-1
		\end{equation}
		\item $k$-times-per-year compounded spot interest rate
		\begin{equation}
			Y^k(t, T)= \frac{k}{P(t, T)^{\frac{1}{k\tau(t,T)}}}-1
		\end{equation}
		\item When $k\rightarrow\infty$, we get the continuously compounded rate
		\begin{equation}
			R(t,T)=-\frac{\log P(t,T)}{\tau(t,T)} \implies P(t,T)=e^{-R(t,T)\tau(t,T)}
		\end{equation}
	\end{itemize}
\end{frame}

%\begin{frame}{Fiddling with Rates (2/2)}
%	MIGLIORARE
%	\begin{itemize}
	%	\item Does it ring a bell ?
	%	\item We can also write
	%	\begin{equation} 
		%	f(t, T)=-\frac{\partial \log P(t, T)}{\partial T} = -\frac{\partial\log \mathbb{E}^Q[e^{-\int_t^T r(s) ds} | \mathcal{F}_t]}{\partial T}
		%		\end{equation}
	%	\item So what is the difference between $r(t)$ and $f(t, T)$ ?
	%	\end{itemize}
%\end{frame}

\section{Forward Rate Agreement}
\begin{frame}{Forward Rate Agreement: Definition}
	\begin{itemize}	
		\item A \textcolor{red}{Forward Rate Agreement} (FRA) is a contract involving three time instants: %the current time $t$, the fixing time $T>t$, and the maturity time $S>T$.
		\begin{equation*}
			\underbrace{t}_{\text{current time}} \leq \underbrace{T}_{\text{fixing time}} \leq\underbrace{S}_{\text{maturity}}
		\end{equation*}
		\item The FRA payout consists of an exchange of interest rate flows calculated for the time period $\tau=S-T$. At the maturity $S$, a fixed payment based on a fixed rate $K$ is exchanged against a floating payment based on the spot rate $L(T, S)$ whose value is known only in $T$.
		\item Basically, this contract allows one to lock-in the interest rate between times $T$ and $S$ at a desired value $K$. Interest rate flows are calculated using the simple compounding law.
	\end{itemize}
\end{frame}

\begin{frame}{FRA: Formalization of the Contract}
	\begin{itemize}
		\item Formally, at time $S$ one receives $\tau(T, S)KN$ units of currency and pays the amount $\tau(T,S)L(T,S)N$, where $N$ is the contract nominal value.
		\item Thus, at time $S$, the (future and today unknown) payout of the contract is: 
		\begin{equation}
			N\tau(T,S)(K-L(T,S))
			\label{eq:fra_payoff}
		\end{equation}
		\item In order to assign a value at this contract we have to tackle two issues:
		\begin{itemize}
			\item how to estimate $L(T, S)$
			\item how to discount it from $S$ to $t$
		\end{itemize}.
		\item There are several ways to arrive at the final result: \textbf{no arbitrage is the common denominator.}
	\end{itemize}
\end{frame}

%\begin{frame}{FRA: Numerical Example}
%	A company enters into a FRA to receive 4\% on \$100M for a 3-month period starting in 3 years. Imagine that LIBOR is 4.5\% in 3 years.
%	
%	So $T=3y$, $S=3.25y$, $K=4\%$, and $L(T,S)=4.5\%$.
%	
%	The net cash-flow at time $S$ is
%	\begin{equation*}
%		100000000\times(0.04-0.045)\times0.25 = -\$125000
%	\end{equation*} 
%\end{frame}

\subsection{FRA Valuation}
\begin{frame}{FRA: the Standard Derivation}
	\begin{itemize}
		\item Following the approach reported in Brigo-Mercurio (2006) we can proceed as follows:
		\item Substitute in Eq.~\ref{eq:fra_payoff} $L(T,S)$ with its expression as a function of $P$ and get
		\begin{equation*}
			N[\tau K - \frac{1}{P(T, S)} + 1]
		\end{equation*}
		\item Interpret $A=\frac{1}{P(T,S)}$ as an amount of money held at time $S$. At $T$ it's worth 1, which in turn, at time $t$, equals $A=P(t,T)$.
		\item On the other hand $B=\tau K + 1$ in $S$ becomes at time $t$
		\begin{equation*}
			P(t,S)\tau K + P(t, S)
		\end{equation*}
		\item Collecting the terms we get
		\begin{equation}
			\mathbf{FRA}(t,T,S,\tau,N,K)=N[P(t,S)\tau K–P(t,T)+P(t,S)]
		\end{equation}
	\end{itemize}
\end{frame}

\subsection{FRA Valuation ($2^{nd}$ Approach)}
\begin{frame}{FRA Valuation: a Different Approach}
	Now we will try to estimate the value of a FRA using a simple replication approach:
	\begin{itemize}
		\item at time $t$: 
		\begin{equation*}
			\begin{cases}
				\text{lend}~P(t,T)\\
				\text{borrow}~P(t,S)(1+(S-T)K)
			\end{cases};
		\end{equation*}
		\item at time $T$: you receive 1 which you will invest at the (in $t$ unknown) EURIBOR rate $L(T,S)$;
		\item at time $S$: 
		\begin{equation*}
			\begin{cases}
				\text{receive}~(1+L(T,S)(S-T))\\
				\text{pay}~1 + (S-T)K
			\end{cases}.
		\end{equation*}
	\end{itemize}
\end{frame}

\begin{frame}{FRA Valuation: a Different Approach}
	Adding the cash-flows together and evaluating in $t$ we get:
	\begin{equation*}
		\begin{gathered}
			1+L(T,S)(S-T) - 1 - (S-T)K = (L(T,S)-K)(S-T) \\
			\mathbb{E}^Q[D(t, S)(L(T, S)-K)(S-T)]
		\end{gathered}
	\end{equation*}
	
	But its value must equal to the value of the replicating portfolio in $t$ (we are in a complete market\ldots). Hence
	\begin{equation}
		\begin{aligned}
			\mathbb{E}^Q[D(t,S)&(L(T,S)-K)(S-T)]=\\
			&[P(t,S)(S-T)K-P(t,T)+P(t,S)]
		\end{aligned}
	\end{equation}
\end{frame}

\subsection{Forward Rate Definition}
\begin{frame}{FRA: the Forward Rate as a Break-even Rate}
	\begin{block}{Forward Rate}
		The value of $K$ which makes the value of the FRA equal to zero is called \textcolor{red}{simply-compounded forward rate}
		\begin{equation}
			F(t;T,S):=\frac{1}{\tau(T,S)}\left[\frac{P(t,T)-P(t,S)}{P(t,S)}\right]
		\end{equation}
	\end{block}
	
	The forward rate can be interpreted as a rate observed in $t$ spanning the time period $S-T$.
	
	%Its value depends on no-arbitrage consideration.
	
\end{frame}

\begin{frame}{Forward Rate}
	\begin{itemize}
		\item We can then rewrite the value of the FRA in terms of the simply-compounded forward rate
		\begin{equation}
			\mathbf{FRA}(t,T,S,\tau,N,K)=NP(t,S)\tau(T,S)(K-F(t;T,S))
			\label{eq:fram_payoff_withF}
		\end{equation}
		(this formula will be used for the swap evaluation)
		\item It is just like we had replaced the LIBOR rate $L(T,S)$ in the payoff with the forward rate $F(t,T,S)$ and then taken the present value of the (deterministic) quantity.
		\item The forward rate can be interpreted as an \emph{estimate} of the future spot rate, which is random at time $t$ (based on the market conditions).
		\item We'll see later that indeed $F(t,T,S)$ is the expectation of $L(T,S)$ under a particular probability measure.
	\end{itemize}
\end{frame}

\begin{frame}{Instantaneous Forward Rate}
	\begin{itemize}
		\item Now we introduce the \textcolor{red}{instantaneous forward rate} $f(t, T)$.
		It is defined as 
		\begin{equation}
			\begin{aligned}
			f(t, T) &:= \lim_{S\rightarrow T^+} F(t,T,S) \\
			& = \lim_{\epsilon\rightarrow 0}  \frac{1}{\tau(T,T+\epsilon)}\frac{P(t,T)-P(t,T+\epsilon)}{P(t,T+\epsilon)} \\
			& = \lim_{\epsilon\rightarrow 0} - \frac{1}{P(t,T)} \frac{P(t,T+\epsilon)-P(t,T)}{\epsilon} \\
			& = -\frac{\partial \log P(t, T)}{\partial T}
			\end{aligned}
		\end{equation}
	\end{itemize}
\end{frame}

\begin{frame}{Instantaneous Forward Rate}
	\begin{itemize}
		\item From the previous equation we can derive
		\begin{equation}
			P(t, T) = e^{-\int_t^T f(t, s) ds}
		\end{equation}
		\item The instantaneous forward rate is the forward rate for a forward contract with an infinitesimal investment period after the settlement date.
		\item Notice that
		\begin{equation}
			r(t) = f(t,t)
		\end{equation}
	\end{itemize}
\end{frame}

%\begin{frame}{}
%\begin{block}{}
%This derivation seems very general but in reality it relies on some strong assumptions about the absence of counter-party liquidity credit risk. If that book were written today may be the authors would give a different version of the proof. The next derivation is a bit more explicit in the assumptions needed to get to the result.
%\end{block}
%\end{frame}
%
%\begin{frame}{Derivation of the Forward Rate by no arbitrage}
%\begin{itemize}
%	\item Following Filipovic (2009) we can also directly define the \textcolor{red}{forward rate} by implementing the following:
%	\item at time $t$ sell one $T$-maturity ZCB and buy $\frac{P(t,T)}{P(t,S)}$ ZCBs;
%	\item in $T$ you must pay 1;
%	\item in $S$ you receive $\frac{P(t,T)}{P(t,S)}$, it is equivalent to a forward start investment of 1 in $T$;
%	\item so 
%	\begin{equation*}
	%	1 + (S-T)F(t;T, S)= \frac{P(t,T)}{P(t,S)}
	%	\end{equation*}
%	\item hence we get as before the expression for the forward rate.
%	\item Are there risks in the implementation of this strategy ?
%\end{itemize}	
%\end{frame}
%
%\begin{frame}{FRA: detailed proof}
%We start by
%\begin{equation}
%\mathbb{E}_t^Q[D(t, S)\tau K – D(t,S) \tau L(T,S)]
%\end{equation}
%From which we derive easily
%\begin{align}
%\tau K \mathbb{E}_t^Q[D(t, S)]&-\mathbb{E}_t^Q[D(t, S)\tau L(T,S)]=\\
%& \tau K P(t,S) - \mathbb{E}_t^Q[D(t, S)\tau L(T,S)]
%\end{align}
%as $P(t,S)=\mathbb{E}_t^Q[D(t, S)]$
%
%But remembering the definition given above
%$D(t,S) = D(t,T)D(T,S)$
%\end{frame}
%
%\begin{frame}{}
%Substituting in (1)
%\begin{equation}
%\tau KP(t,S)-\mathbb{E}_t^Q[\tau D(t, T)D(T,S)L(T,S)]
%\end{equation}
%Then using the properties of the expectation operator
%\begin{equation}
%\tau K P(t,S)-\mathbb{E}_t^Q[\mathbb{E}_t^Q[\tau D(t, T)D(T, S)L(T,S)]]
%\end{equation}
%From the last equation we get (using another property of expectation)
%\begin{equation}
%\tau K P(t,S)-\mathbb{E}_t^Q[ \tau D(t, T) L(T,S) \mathbb{E}_t^Q[D(T, S)]]
%\end{equation}
%From which we can write by the definition of $P$
%\begin{equation}
%\tau K P(t,S)-\mathbb{E}_t^Q[ \tau D(t, T) L(T,S) P(T,S)]]
%\end{equation}
%\end{frame}
%
%\begin{frame}{}
%From the last equation I can write by the property of Expectation
%\begin{equation}
%\tau K P(t,S)-\mathbb{E}_t^Q[D(t,T)]+\mathbb{E}_t^Q[D(t, T) \mathbb{E}_t^Q[D(T, S)]]
%\end{equation}
%Perversely splitting 
%\begin{equation}
%\tau K P(t,S)-\mathbb{E}_t^Q[D(t,T)]+\mathbb{E}_t^Q[\mathbb{E}_t^Q[D(t, T) D(T, S)]]
%\end{equation}
%More perversely
%\begin{equation}
%\tau K P(t,S)-\mathbb{E}_t^Q[D(t,T)]+\mathbb{E}_t^Q[\mathbb{E}_t^Q[D(t, S)]
%\end{equation}
%Then applying the lay of iterated expectations we get
%\begin{equation}
%\tau K P(t,S)-\mathbb{E}_t^Q[D(t,T)]+\mathbb{E}_t^Q[D(t, S)]
%\end{equation}
%\end{frame}
%
%\begin{frame}{}
%Finally we get what we are looking for
%\begin{equation}
%\tau K P(t,S)-P(t, T)+ P(t, S)
%\end{equation}
%\end{frame}

%\begin{frame}{Forward rates: shedding light on an underlying relation}
%\begin{itemize}
%\item The starting point is the observation that in a FRA, we must evaluate the future value of the rate $L(T,S)$
%\item Let us define the forward zero coupon bond in valuation date $t$ as
%\begin{equation}
%P(t,T,S):= \mathbb{E}_t^Q[P(T,S)]
%\end{equation}
%\item In order to avoid arbitrage in the calculation of the price $V(t,G(S))$ in $t$ of a generic and certain sum $G(S)$, the following must hold
%\end{itemize}
%\end{frame}
%
%\begin{frame}{}
%\begin{equation}
%V(t,G(T))=P(t,S)G(T)=G(T)P(t,T)P(t,T,S)
%\end{equation}
%Hence we get $P(t,T,S)+\frac{P(t,S)}{P(t,T)}$
%And then we can define $F(t,T,S)$ inverting
%$P(t,T,S)+\frac{P(t,S)}{P(t,T)}:=\frac{1}{1+F(t,T,S)(S-T)}$
%We invert and find the same no arbitrage formula above. The relation
%$P(t,S)=P(t,T)P(t,T,S)$
%Relies on the fact that there is no counterparty risk. Basically the crisis has broken this relationship because the $P(…)$ may be risky: between $t$, $T$ and $S$ many things can happen to one of the participants of the trade.
%\end{frame}

\end{document}

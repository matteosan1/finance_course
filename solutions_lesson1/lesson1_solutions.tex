
% Default to the notebook output style

    


% Inherit from the specified cell style.




    
\documentclass[11pt]{article}

    
    
    \usepackage[T1]{fontenc}
    % Nicer default font (+ math font) than Computer Modern for most use cases
    \usepackage{mathpazo}

    % Basic figure setup, for now with no caption control since it's done
    % automatically by Pandoc (which extracts ![](path) syntax from Markdown).
    \usepackage{graphicx}
    % We will generate all images so they have a width \maxwidth. This means
    % that they will get their normal width if they fit onto the page, but
    % are scaled down if they would overflow the margins.
    \makeatletter
    \def\maxwidth{\ifdim\Gin@nat@width>\linewidth\linewidth
    \else\Gin@nat@width\fi}
    \makeatother
    \let\Oldincludegraphics\includegraphics
    % Set max figure width to be 80% of text width, for now hardcoded.
    \renewcommand{\includegraphics}[1]{\Oldincludegraphics[width=.8\maxwidth]{#1}}
    % Ensure that by default, figures have no caption (until we provide a
    % proper Figure object with a Caption API and a way to capture that
    % in the conversion process - todo).
    \usepackage{caption}
    \DeclareCaptionLabelFormat{nolabel}{}
    \captionsetup{labelformat=nolabel}

    \usepackage{adjustbox} % Used to constrain images to a maximum size 
    \usepackage{xcolor} % Allow colors to be defined
    \usepackage{enumerate} % Needed for markdown enumerations to work
    \usepackage{geometry} % Used to adjust the document margins
    \usepackage{amsmath} % Equations
    \usepackage{amssymb} % Equations
    \usepackage{textcomp} % defines textquotesingle
    % Hack from http://tex.stackexchange.com/a/47451/13684:
    \AtBeginDocument{%
        \def\PYZsq{\textquotesingle}% Upright quotes in Pygmentized code
    }
    \usepackage{upquote} % Upright quotes for verbatim code
    \usepackage{eurosym} % defines \euro
    \usepackage[mathletters]{ucs} % Extended unicode (utf-8) support
    \usepackage[utf8x]{inputenc} % Allow utf-8 characters in the tex document
    \usepackage{fancyvrb} % verbatim replacement that allows latex
    \usepackage{grffile} % extends the file name processing of package graphics 
                         % to support a larger range 
    % The hyperref package gives us a pdf with properly built
    % internal navigation ('pdf bookmarks' for the table of contents,
    % internal cross-reference links, web links for URLs, etc.)
    \usepackage{hyperref}
    \usepackage{longtable} % longtable support required by pandoc >1.10
    \usepackage{booktabs}  % table support for pandoc > 1.12.2
    \usepackage[inline]{enumitem} % IRkernel/repr support (it uses the enumerate* environment)
    \usepackage[normalem]{ulem} % ulem is needed to support strikethroughs (\sout)
                                % normalem makes italics be italics, not underlines
    \usepackage{mathrsfs}
    

    
    
    % Colors for the hyperref package
    \definecolor{urlcolor}{rgb}{0,.145,.698}
    \definecolor{linkcolor}{rgb}{.71,0.21,0.01}
    \definecolor{citecolor}{rgb}{.12,.54,.11}

    % ANSI colors
    \definecolor{ansi-black}{HTML}{3E424D}
    \definecolor{ansi-black-intense}{HTML}{282C36}
    \definecolor{ansi-red}{HTML}{E75C58}
    \definecolor{ansi-red-intense}{HTML}{B22B31}
    \definecolor{ansi-green}{HTML}{00A250}
    \definecolor{ansi-green-intense}{HTML}{007427}
    \definecolor{ansi-yellow}{HTML}{DDB62B}
    \definecolor{ansi-yellow-intense}{HTML}{B27D12}
    \definecolor{ansi-blue}{HTML}{208FFB}
    \definecolor{ansi-blue-intense}{HTML}{0065CA}
    \definecolor{ansi-magenta}{HTML}{D160C4}
    \definecolor{ansi-magenta-intense}{HTML}{A03196}
    \definecolor{ansi-cyan}{HTML}{60C6C8}
    \definecolor{ansi-cyan-intense}{HTML}{258F8F}
    \definecolor{ansi-white}{HTML}{C5C1B4}
    \definecolor{ansi-white-intense}{HTML}{A1A6B2}
    \definecolor{ansi-default-inverse-fg}{HTML}{FFFFFF}
    \definecolor{ansi-default-inverse-bg}{HTML}{000000}

    % commands and environments needed by pandoc snippets
    % extracted from the output of `pandoc -s`
    \providecommand{\tightlist}{%
      \setlength{\itemsep}{0pt}\setlength{\parskip}{0pt}}
    \DefineVerbatimEnvironment{Highlighting}{Verbatim}{commandchars=\\\{\}}
    % Add ',fontsize=\small' for more characters per line
    \newenvironment{Shaded}{}{}
    \newcommand{\KeywordTok}[1]{\textcolor[rgb]{0.00,0.44,0.13}{\textbf{{#1}}}}
    \newcommand{\DataTypeTok}[1]{\textcolor[rgb]{0.56,0.13,0.00}{{#1}}}
    \newcommand{\DecValTok}[1]{\textcolor[rgb]{0.25,0.63,0.44}{{#1}}}
    \newcommand{\BaseNTok}[1]{\textcolor[rgb]{0.25,0.63,0.44}{{#1}}}
    \newcommand{\FloatTok}[1]{\textcolor[rgb]{0.25,0.63,0.44}{{#1}}}
    \newcommand{\CharTok}[1]{\textcolor[rgb]{0.25,0.44,0.63}{{#1}}}
    \newcommand{\StringTok}[1]{\textcolor[rgb]{0.25,0.44,0.63}{{#1}}}
    \newcommand{\CommentTok}[1]{\textcolor[rgb]{0.38,0.63,0.69}{\textit{{#1}}}}
    \newcommand{\OtherTok}[1]{\textcolor[rgb]{0.00,0.44,0.13}{{#1}}}
    \newcommand{\AlertTok}[1]{\textcolor[rgb]{1.00,0.00,0.00}{\textbf{{#1}}}}
    \newcommand{\FunctionTok}[1]{\textcolor[rgb]{0.02,0.16,0.49}{{#1}}}
    \newcommand{\RegionMarkerTok}[1]{{#1}}
    \newcommand{\ErrorTok}[1]{\textcolor[rgb]{1.00,0.00,0.00}{\textbf{{#1}}}}
    \newcommand{\NormalTok}[1]{{#1}}
    
    % Additional commands for more recent versions of Pandoc
    \newcommand{\ConstantTok}[1]{\textcolor[rgb]{0.53,0.00,0.00}{{#1}}}
    \newcommand{\SpecialCharTok}[1]{\textcolor[rgb]{0.25,0.44,0.63}{{#1}}}
    \newcommand{\VerbatimStringTok}[1]{\textcolor[rgb]{0.25,0.44,0.63}{{#1}}}
    \newcommand{\SpecialStringTok}[1]{\textcolor[rgb]{0.73,0.40,0.53}{{#1}}}
    \newcommand{\ImportTok}[1]{{#1}}
    \newcommand{\DocumentationTok}[1]{\textcolor[rgb]{0.73,0.13,0.13}{\textit{{#1}}}}
    \newcommand{\AnnotationTok}[1]{\textcolor[rgb]{0.38,0.63,0.69}{\textbf{\textit{{#1}}}}}
    \newcommand{\CommentVarTok}[1]{\textcolor[rgb]{0.38,0.63,0.69}{\textbf{\textit{{#1}}}}}
    \newcommand{\VariableTok}[1]{\textcolor[rgb]{0.10,0.09,0.49}{{#1}}}
    \newcommand{\ControlFlowTok}[1]{\textcolor[rgb]{0.00,0.44,0.13}{\textbf{{#1}}}}
    \newcommand{\OperatorTok}[1]{\textcolor[rgb]{0.40,0.40,0.40}{{#1}}}
    \newcommand{\BuiltInTok}[1]{{#1}}
    \newcommand{\ExtensionTok}[1]{{#1}}
    \newcommand{\PreprocessorTok}[1]{\textcolor[rgb]{0.74,0.48,0.00}{{#1}}}
    \newcommand{\AttributeTok}[1]{\textcolor[rgb]{0.49,0.56,0.16}{{#1}}}
    \newcommand{\InformationTok}[1]{\textcolor[rgb]{0.38,0.63,0.69}{\textbf{\textit{{#1}}}}}
    \newcommand{\WarningTok}[1]{\textcolor[rgb]{0.38,0.63,0.69}{\textbf{\textit{{#1}}}}}
    
    
    % Define a nice break command that doesn't care if a line doesn't already
    % exist.
    \def\br{\hspace*{\fill} \\* }
    % Math Jax compatibility definitions
    \def\gt{>}
    \def\lt{<}
    \let\Oldtex\TeX
    \let\Oldlatex\LaTeX
    \renewcommand{\TeX}{\textrm{\Oldtex}}
    \renewcommand{\LaTeX}{\textrm{\Oldlatex}}
    % Document parameters
    % Document title
    \title{Solutions - Practical Lesson 1}
    \author {Matteo Sani \\ \href{mailto:matteosan1@gmail.com}{matteosan1@gmail.com}}
    
    \begin{document}
    
    
    \maketitle

\hypertarget{exercise-1.1}{%
\section{Exercise 1.1}\label{exercise-1.1}}
\textbf{Solution:}
\begin{itemize}
\item What is a built-in function that Python uses to iterate over a number sequence ?
  
\verb;range(); generates a list of numbers, which is used to iterate over for loops.

\begin{verbatim}
for i in range(5):
    print(i)
\end{verbatim}

The range() function accompanies two sets of parameters.

\verb;range([start], stop[, step]);

where
\begin{itemize}
  \item start: It is the starting no. of the sequence.
  \item stop: It specifies the upper limit of the sequence.
  \item step: It is the incrementing factor for generating the sequence.
\end{itemize}

Points to note: only integer arguments are allowed. Parameters can be positive or negative.
The range() function in Python starts from the zeroth index.

\item What is a string in Python ?

  A string in Python is a sequence of alpha-numeric characters. They are immutable objects. It means that they don’t allow modification once they get assigned a value. Python provides several methods, such as \verb;join(), replace() or split(); to alter strings. But none of these change the original object.

\item What does the continue do in Python ?

  The \verb;continue; is a jump statement in Python which moves the control to execute the next iteration in a loop leaving all the remaining instructions in the block unexecuted. The \verb;continue; statement is applicable for both the while and for loops.

\item When should you use the break in Python ? 

  Python provides a \verb;break; statement to exit from a loop. Whenever the \verb;break; hits in the code, the control of the program immediately exits from the body of the loop. The break statement in a nested loop causes the control to exit from the inner iterative block.

\item What is a dictionary in Python programming ?

  A dictionary is a data structure known as an associative array in Python which stores a collection of objects.
The collection is a set of keys having a single associated value. We can call it a hash, a map, or a hashmap as it gets called in other programming languages.

\item What is the use of the dictionary in Python ?
  
  A dictionary has a group of objects (the keys) map to another group of objects (the values). A Python dictionary represents a mapping of unique Keys to Values. They are mutable and hence will not change. The values associated with the keys can be of any Python types.

\item How do you create a dictionary in Python ?
  
  Let’s take the example of building site statistics. For this, we first need to break up the key-value pairs using a colon(“:”). The keys should be of an immutable type, i.e., so we’ll use the data-types which don’t allow changes at runtime. We’ll choose from an int, string, or tuple. However, we can take values of any kind. For distinguishing the data pairs, we can use a comma(“,”) and keep the whole stuff inside curly braces({…}).

\begin {verbatim}
>>> site_stats = {'site': 'tecbeamers.com', 'traffic': 10000, "type": "organic"}
>>> type(site_stats)
<class 'dict'>
>>> print(site_stats)
{'type': 'organic', 'site': 'tecbeamers.com', 'traffic': 10000}
\end{verbatim}

\item How do you read from a dictionary in Python ?
  
  To fetch data from a dictionary, we can directly access using the keys. We can enclose a key using brackets […] after mentioning the variable name corresponding to the dictionary.

\begin{verbatim}
>>> site_stats = {'site': 'tecbeamers.com', 'traffic': 10000, "type": "organic"}
>>> print(site_stats["traffic"])
\end{verbatim}

We can even call the get method to fetch the values from a dict. It also let us set a default value. If the key is missing, then the KeyError would occur.

\begin{verbatim}
>>> site_stats = {'site': 'tecbeamers.com', 'traffic': 10000, "type": "organic"}
>>> print(site_stats.get('site'))
tecbeamers.com
\end{verbatim}

\item How do you traverse through a dictionary object in Python?
  
  We can use the \verb;for; and \verb;in; loop for traversing the dictionary object.

\begin{verbatim}
>>> site_stats = {'site': 'tecbeamers.com', 'traffic': 10000, "type": "organic"}
>>> for k, v in site_stats.items():
    print("The key is: %s" % k)
    print("The value is: %s" % v)
    print("++++++++++++++++++++++++")
The key is: type
The value is: organic
++++++++++++++++++++++++
The key is: site
The value is: tecbeamers.com
++++++++++++++++++++++++
The key is: traffic
The value is: 10000
++++++++++++++++++++++++
\end{verbatim}

\item How do you add elements to a dictionary in Python ?
  
  We can add elements by modifying the dictionary with a fresh key and then set the value to it.

\begin{verbatim}
>>> # Setup a blank dictionary
>>> site_stats = {}
>>> site_stats['site'] = 'google.com'
>>> site_stats['traffic'] = 10000000000
>>> site_stats['type'] = 'Referral'
>>> print(site_stats)
{'type': 'Referral', 'site': 'google.com', 'traffic': 10000000000}
\end{verbatim}

We can even join two dictionaries to get a bigger dictionary with the help of the \verb;update(); method.

\begin{verbatim}
>>> site_stats['site'] = 'google.co.in'
>>> print(site_stats)
{'site': 'google.co.in'}
>>> site_stats_new = {'traffic': 1000000, "type": "social media"}
>>> site_stats.update(site_stats_new)
>>> print(site_stats)
{'type': 'social media', 'site': 'google.co.in', 'traffic': 1000000}
\end{verbatim}

\item How do you delete elements of a dictionary in Python ?
  
  We can delete a key in a dictionary by using the \verb;del; method.

\begin{verbatim}
>>> site_stats = {'site': 'tecbeamers.com', 'traffic': 10000, "type": "organic"}
>>> del site_stats["type"]
>>> print(site_stats)
{'site': 'google.co.in', 'traffic': 1000000}
\end{verbatim}

Another method, we can use is the \verb;pop; function. It accepts the key as the parameter. Also, a second parameter, we can pass a default value if the key doesn’t exist.

\begin{verbatim}
>>> site_stats = {'site': 'tecbeamers.com', 'traffic': 10000, "type": "organic"}
>>> print(site_stats.pop("type", None))
organic
>>> print(site_stats)
{'site': 'tecbeamers.com', 'traffic': 10000}
\end{verbatim}

\item How do you check the presence of a key in a Dictionary?
  
  We can use Python’s \verb;in; operator to test the presence of a key inside a dict object.

\begin{verbatim}
>>> site_stats = {'site': 'tecbeamers.com', 'traffic': 10000, "type": "organic"}
>>> 'site' in site_stats
True
>>> 'traffic' in site_stats
True
>>> "type" in site_stats
True
\end{verbatim}

\item What is the syntax for list comprehension in Python?
  
  The signature for the list comprehension is as follows:

\verb;[ expression(var) for var in iterable ];

For example, the below code will return all the numbers from 10 to 20 and store them in a list.

\begin{verbatim}
>>> alist = [var for var in range(10, 20)]
>>> print(alist)
\end{verbatim}

\item What is the syntax for dictionary comprehension in Python ?
  
  A dictionary has the same syntax as was for the list comprehension but the difference is that it uses curly braces:

\verb;{ aKey, itsValue for aKey in iterable };

For example, the below code will return all the numbers 10 to 20 as the keys and will store the respective squares of those numbers as the values.

\begin{verbatim}
>>> adict = {var:var**2 for var in range(10, 20)}
>>> print(adict)
\end{verbatim}

\item How do you write a conditional expression in Python ?
  
  We can utilize the following single statement as a conditional expression.
  \verb;default_statment if condition else another_statement;

\begin{verbatim}
>>> no_of_days = 366
>>> is_leap_year = "Yes" if no_of_days == 366 else "No"
>>> print(is_leap_year)
Yes
\end{verbatim}

\item Which python function will you use to convert a number to a string?
  
  For converting a number into a string, you can use the built-in function \verb;str();.  If you want an octal or hexadecimal representation, use the inbuilt function \verb;oct(); or \verb;hex();.
\end{itemize}
    
    
    \end{document}

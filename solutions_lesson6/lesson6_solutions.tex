
% Default to the notebook output style

    


% Inherit from the specified cell style.




    
\documentclass[11pt]{article}

    
    
    \usepackage[T1]{fontenc}
    % Nicer default font (+ math font) than Computer Modern for most use cases
    \usepackage{mathpazo}

    % Basic figure setup, for now with no caption control since it's done
    % automatically by Pandoc (which extracts ![](path) syntax from Markdown).
    \usepackage{graphicx}
    % We will generate all images so they have a width \maxwidth. This means
    % that they will get their normal width if they fit onto the page, but
    % are scaled down if they would overflow the margins.
    \makeatletter
    \def\maxwidth{\ifdim\Gin@nat@width>\linewidth\linewidth
    \else\Gin@nat@width\fi}
    \makeatother
    \let\Oldincludegraphics\includegraphics
    % Set max figure width to be 80% of text width, for now hardcoded.
    \renewcommand{\includegraphics}[1]{\Oldincludegraphics[width=.8\maxwidth]{#1}}
    % Ensure that by default, figures have no caption (until we provide a
    % proper Figure object with a Caption API and a way to capture that
    % in the conversion process - todo).
    \usepackage{caption}
    \DeclareCaptionLabelFormat{nolabel}{}
    \captionsetup{labelformat=nolabel}

    \usepackage{adjustbox} % Used to constrain images to a maximum size 
    \usepackage{xcolor} % Allow colors to be defined
    \usepackage{enumerate} % Needed for markdown enumerations to work
    \usepackage{geometry} % Used to adjust the document margins
    \usepackage{amsmath} % Equations
    \usepackage{amssymb} % Equations
    \usepackage{textcomp} % defines textquotesingle
    % Hack from http://tex.stackexchange.com/a/47451/13684:
    \AtBeginDocument{%
        \def\PYZsq{\textquotesingle}% Upright quotes in Pygmentized code
    }
    \usepackage{upquote} % Upright quotes for verbatim code
    \usepackage{eurosym} % defines \euro
    \usepackage[mathletters]{ucs} % Extended unicode (utf-8) support
    \usepackage[utf8x]{inputenc} % Allow utf-8 characters in the tex document
    \usepackage{fancyvrb} % verbatim replacement that allows latex
    \usepackage{grffile} % extends the file name processing of package graphics 
                         % to support a larger range 
    % The hyperref package gives us a pdf with properly built
    % internal navigation ('pdf bookmarks' for the table of contents,
    % internal cross-reference links, web links for URLs, etc.)
    \usepackage{hyperref}
    \usepackage{longtable} % longtable support required by pandoc >1.10
    \usepackage{booktabs}  % table support for pandoc > 1.12.2
    \usepackage[inline]{enumitem} % IRkernel/repr support (it uses the enumerate* environment)
    \usepackage[normalem]{ulem} % ulem is needed to support strikethroughs (\sout)
                                % normalem makes italics be italics, not underlines
    \usepackage{mathrsfs}
    

    
    
    % Colors for the hyperref package
    \definecolor{urlcolor}{rgb}{0,.145,.698}
    \definecolor{linkcolor}{rgb}{.71,0.21,0.01}
    \definecolor{citecolor}{rgb}{.12,.54,.11}

    % ANSI colors
    \definecolor{ansi-black}{HTML}{3E424D}
    \definecolor{ansi-black-intense}{HTML}{282C36}
    \definecolor{ansi-red}{HTML}{E75C58}
    \definecolor{ansi-red-intense}{HTML}{B22B31}
    \definecolor{ansi-green}{HTML}{00A250}
    \definecolor{ansi-green-intense}{HTML}{007427}
    \definecolor{ansi-yellow}{HTML}{DDB62B}
    \definecolor{ansi-yellow-intense}{HTML}{B27D12}
    \definecolor{ansi-blue}{HTML}{208FFB}
    \definecolor{ansi-blue-intense}{HTML}{0065CA}
    \definecolor{ansi-magenta}{HTML}{D160C4}
    \definecolor{ansi-magenta-intense}{HTML}{A03196}
    \definecolor{ansi-cyan}{HTML}{60C6C8}
    \definecolor{ansi-cyan-intense}{HTML}{258F8F}
    \definecolor{ansi-white}{HTML}{C5C1B4}
    \definecolor{ansi-white-intense}{HTML}{A1A6B2}
    \definecolor{ansi-default-inverse-fg}{HTML}{FFFFFF}
    \definecolor{ansi-default-inverse-bg}{HTML}{000000}

    % commands and environments needed by pandoc snippets
    % extracted from the output of `pandoc -s`
    \providecommand{\tightlist}{%
      \setlength{\itemsep}{0pt}\setlength{\parskip}{0pt}}
    \DefineVerbatimEnvironment{Highlighting}{Verbatim}{commandchars=\\\{\}}
    % Add ',fontsize=\small' for more characters per line
    \newenvironment{Shaded}{}{}
    \newcommand{\KeywordTok}[1]{\textcolor[rgb]{0.00,0.44,0.13}{\textbf{{#1}}}}
    \newcommand{\DataTypeTok}[1]{\textcolor[rgb]{0.56,0.13,0.00}{{#1}}}
    \newcommand{\DecValTok}[1]{\textcolor[rgb]{0.25,0.63,0.44}{{#1}}}
    \newcommand{\BaseNTok}[1]{\textcolor[rgb]{0.25,0.63,0.44}{{#1}}}
    \newcommand{\FloatTok}[1]{\textcolor[rgb]{0.25,0.63,0.44}{{#1}}}
    \newcommand{\CharTok}[1]{\textcolor[rgb]{0.25,0.44,0.63}{{#1}}}
    \newcommand{\StringTok}[1]{\textcolor[rgb]{0.25,0.44,0.63}{{#1}}}
    \newcommand{\CommentTok}[1]{\textcolor[rgb]{0.38,0.63,0.69}{\textit{{#1}}}}
    \newcommand{\OtherTok}[1]{\textcolor[rgb]{0.00,0.44,0.13}{{#1}}}
    \newcommand{\AlertTok}[1]{\textcolor[rgb]{1.00,0.00,0.00}{\textbf{{#1}}}}
    \newcommand{\FunctionTok}[1]{\textcolor[rgb]{0.02,0.16,0.49}{{#1}}}
    \newcommand{\RegionMarkerTok}[1]{{#1}}
    \newcommand{\ErrorTok}[1]{\textcolor[rgb]{1.00,0.00,0.00}{\textbf{{#1}}}}
    \newcommand{\NormalTok}[1]{{#1}}
    
    % Additional commands for more recent versions of Pandoc
    \newcommand{\ConstantTok}[1]{\textcolor[rgb]{0.53,0.00,0.00}{{#1}}}
    \newcommand{\SpecialCharTok}[1]{\textcolor[rgb]{0.25,0.44,0.63}{{#1}}}
    \newcommand{\VerbatimStringTok}[1]{\textcolor[rgb]{0.25,0.44,0.63}{{#1}}}
    \newcommand{\SpecialStringTok}[1]{\textcolor[rgb]{0.73,0.40,0.53}{{#1}}}
    \newcommand{\ImportTok}[1]{{#1}}
    \newcommand{\DocumentationTok}[1]{\textcolor[rgb]{0.73,0.13,0.13}{\textit{{#1}}}}
    \newcommand{\AnnotationTok}[1]{\textcolor[rgb]{0.38,0.63,0.69}{\textbf{\textit{{#1}}}}}
    \newcommand{\CommentVarTok}[1]{\textcolor[rgb]{0.38,0.63,0.69}{\textbf{\textit{{#1}}}}}
    \newcommand{\VariableTok}[1]{\textcolor[rgb]{0.10,0.09,0.49}{{#1}}}
    \newcommand{\ControlFlowTok}[1]{\textcolor[rgb]{0.00,0.44,0.13}{\textbf{{#1}}}}
    \newcommand{\OperatorTok}[1]{\textcolor[rgb]{0.40,0.40,0.40}{{#1}}}
    \newcommand{\BuiltInTok}[1]{{#1}}
    \newcommand{\ExtensionTok}[1]{{#1}}
    \newcommand{\PreprocessorTok}[1]{\textcolor[rgb]{0.74,0.48,0.00}{{#1}}}
    \newcommand{\AttributeTok}[1]{\textcolor[rgb]{0.49,0.56,0.16}{{#1}}}
    \newcommand{\InformationTok}[1]{\textcolor[rgb]{0.38,0.63,0.69}{\textbf{\textit{{#1}}}}}
    \newcommand{\WarningTok}[1]{\textcolor[rgb]{0.38,0.63,0.69}{\textbf{\textit{{#1}}}}}
    
    
    % Define a nice break command that doesn't care if a line doesn't already
    % exist.
    \def\br{\hspace*{\fill} \\* }
    % Math Jax compatibility definitions
    \def\gt{>}
    \def\lt{<}
    \let\Oldtex\TeX
    \let\Oldlatex\LaTeX
    \renewcommand{\TeX}{\textrm{\Oldtex}}
    \renewcommand{\LaTeX}{\textrm{\Oldlatex}}
    % Document parameters
    % Document title
    \title{Solutions - Practical Lesson 6}
    \author{Matteo Sani \\ \href{mailto:matteosan1@gmail.com}{matteosan1@gmail.com}}
    
    
    
    

    % Pygments definitions
    
\makeatletter
\def\PY@reset{\let\PY@it=\relax \let\PY@bf=\relax%
    \let\PY@ul=\relax \let\PY@tc=\relax%
    \let\PY@bc=\relax \let\PY@ff=\relax}
\def\PY@tok#1{\csname PY@tok@#1\endcsname}
\def\PY@toks#1+{\ifx\relax#1\empty\else%
    \PY@tok{#1}\expandafter\PY@toks\fi}
\def\PY@do#1{\PY@bc{\PY@tc{\PY@ul{%
    \PY@it{\PY@bf{\PY@ff{#1}}}}}}}
\def\PY#1#2{\PY@reset\PY@toks#1+\relax+\PY@do{#2}}

\expandafter\def\csname PY@tok@w\endcsname{\def\PY@tc##1{\textcolor[rgb]{0.73,0.73,0.73}{##1}}}
\expandafter\def\csname PY@tok@c\endcsname{\let\PY@it=\textit\def\PY@tc##1{\textcolor[rgb]{0.25,0.50,0.50}{##1}}}
\expandafter\def\csname PY@tok@cp\endcsname{\def\PY@tc##1{\textcolor[rgb]{0.74,0.48,0.00}{##1}}}
\expandafter\def\csname PY@tok@k\endcsname{\let\PY@bf=\textbf\def\PY@tc##1{\textcolor[rgb]{0.00,0.50,0.00}{##1}}}
\expandafter\def\csname PY@tok@kp\endcsname{\def\PY@tc##1{\textcolor[rgb]{0.00,0.50,0.00}{##1}}}
\expandafter\def\csname PY@tok@kt\endcsname{\def\PY@tc##1{\textcolor[rgb]{0.69,0.00,0.25}{##1}}}
\expandafter\def\csname PY@tok@o\endcsname{\def\PY@tc##1{\textcolor[rgb]{0.40,0.40,0.40}{##1}}}
\expandafter\def\csname PY@tok@ow\endcsname{\let\PY@bf=\textbf\def\PY@tc##1{\textcolor[rgb]{0.67,0.13,1.00}{##1}}}
\expandafter\def\csname PY@tok@nb\endcsname{\def\PY@tc##1{\textcolor[rgb]{0.00,0.50,0.00}{##1}}}
\expandafter\def\csname PY@tok@nf\endcsname{\def\PY@tc##1{\textcolor[rgb]{0.00,0.00,1.00}{##1}}}
\expandafter\def\csname PY@tok@nc\endcsname{\let\PY@bf=\textbf\def\PY@tc##1{\textcolor[rgb]{0.00,0.00,1.00}{##1}}}
\expandafter\def\csname PY@tok@nn\endcsname{\let\PY@bf=\textbf\def\PY@tc##1{\textcolor[rgb]{0.00,0.00,1.00}{##1}}}
\expandafter\def\csname PY@tok@ne\endcsname{\let\PY@bf=\textbf\def\PY@tc##1{\textcolor[rgb]{0.82,0.25,0.23}{##1}}}
\expandafter\def\csname PY@tok@nv\endcsname{\def\PY@tc##1{\textcolor[rgb]{0.10,0.09,0.49}{##1}}}
\expandafter\def\csname PY@tok@no\endcsname{\def\PY@tc##1{\textcolor[rgb]{0.53,0.00,0.00}{##1}}}
\expandafter\def\csname PY@tok@nl\endcsname{\def\PY@tc##1{\textcolor[rgb]{0.63,0.63,0.00}{##1}}}
\expandafter\def\csname PY@tok@ni\endcsname{\let\PY@bf=\textbf\def\PY@tc##1{\textcolor[rgb]{0.60,0.60,0.60}{##1}}}
\expandafter\def\csname PY@tok@na\endcsname{\def\PY@tc##1{\textcolor[rgb]{0.49,0.56,0.16}{##1}}}
\expandafter\def\csname PY@tok@nt\endcsname{\let\PY@bf=\textbf\def\PY@tc##1{\textcolor[rgb]{0.00,0.50,0.00}{##1}}}
\expandafter\def\csname PY@tok@nd\endcsname{\def\PY@tc##1{\textcolor[rgb]{0.67,0.13,1.00}{##1}}}
\expandafter\def\csname PY@tok@s\endcsname{\def\PY@tc##1{\textcolor[rgb]{0.73,0.13,0.13}{##1}}}
\expandafter\def\csname PY@tok@sd\endcsname{\let\PY@it=\textit\def\PY@tc##1{\textcolor[rgb]{0.73,0.13,0.13}{##1}}}
\expandafter\def\csname PY@tok@si\endcsname{\let\PY@bf=\textbf\def\PY@tc##1{\textcolor[rgb]{0.73,0.40,0.53}{##1}}}
\expandafter\def\csname PY@tok@se\endcsname{\let\PY@bf=\textbf\def\PY@tc##1{\textcolor[rgb]{0.73,0.40,0.13}{##1}}}
\expandafter\def\csname PY@tok@sr\endcsname{\def\PY@tc##1{\textcolor[rgb]{0.73,0.40,0.53}{##1}}}
\expandafter\def\csname PY@tok@ss\endcsname{\def\PY@tc##1{\textcolor[rgb]{0.10,0.09,0.49}{##1}}}
\expandafter\def\csname PY@tok@sx\endcsname{\def\PY@tc##1{\textcolor[rgb]{0.00,0.50,0.00}{##1}}}
\expandafter\def\csname PY@tok@m\endcsname{\def\PY@tc##1{\textcolor[rgb]{0.40,0.40,0.40}{##1}}}
\expandafter\def\csname PY@tok@gh\endcsname{\let\PY@bf=\textbf\def\PY@tc##1{\textcolor[rgb]{0.00,0.00,0.50}{##1}}}
\expandafter\def\csname PY@tok@gu\endcsname{\let\PY@bf=\textbf\def\PY@tc##1{\textcolor[rgb]{0.50,0.00,0.50}{##1}}}
\expandafter\def\csname PY@tok@gd\endcsname{\def\PY@tc##1{\textcolor[rgb]{0.63,0.00,0.00}{##1}}}
\expandafter\def\csname PY@tok@gi\endcsname{\def\PY@tc##1{\textcolor[rgb]{0.00,0.63,0.00}{##1}}}
\expandafter\def\csname PY@tok@gr\endcsname{\def\PY@tc##1{\textcolor[rgb]{1.00,0.00,0.00}{##1}}}
\expandafter\def\csname PY@tok@ge\endcsname{\let\PY@it=\textit}
\expandafter\def\csname PY@tok@gs\endcsname{\let\PY@bf=\textbf}
\expandafter\def\csname PY@tok@gp\endcsname{\let\PY@bf=\textbf\def\PY@tc##1{\textcolor[rgb]{0.00,0.00,0.50}{##1}}}
\expandafter\def\csname PY@tok@go\endcsname{\def\PY@tc##1{\textcolor[rgb]{0.53,0.53,0.53}{##1}}}
\expandafter\def\csname PY@tok@gt\endcsname{\def\PY@tc##1{\textcolor[rgb]{0.00,0.27,0.87}{##1}}}
\expandafter\def\csname PY@tok@err\endcsname{\def\PY@bc##1{\setlength{\fboxsep}{0pt}\fcolorbox[rgb]{1.00,0.00,0.00}{1,1,1}{\strut ##1}}}
\expandafter\def\csname PY@tok@kc\endcsname{\let\PY@bf=\textbf\def\PY@tc##1{\textcolor[rgb]{0.00,0.50,0.00}{##1}}}
\expandafter\def\csname PY@tok@kd\endcsname{\let\PY@bf=\textbf\def\PY@tc##1{\textcolor[rgb]{0.00,0.50,0.00}{##1}}}
\expandafter\def\csname PY@tok@kn\endcsname{\let\PY@bf=\textbf\def\PY@tc##1{\textcolor[rgb]{0.00,0.50,0.00}{##1}}}
\expandafter\def\csname PY@tok@kr\endcsname{\let\PY@bf=\textbf\def\PY@tc##1{\textcolor[rgb]{0.00,0.50,0.00}{##1}}}
\expandafter\def\csname PY@tok@bp\endcsname{\def\PY@tc##1{\textcolor[rgb]{0.00,0.50,0.00}{##1}}}
\expandafter\def\csname PY@tok@fm\endcsname{\def\PY@tc##1{\textcolor[rgb]{0.00,0.00,1.00}{##1}}}
\expandafter\def\csname PY@tok@vc\endcsname{\def\PY@tc##1{\textcolor[rgb]{0.10,0.09,0.49}{##1}}}
\expandafter\def\csname PY@tok@vg\endcsname{\def\PY@tc##1{\textcolor[rgb]{0.10,0.09,0.49}{##1}}}
\expandafter\def\csname PY@tok@vi\endcsname{\def\PY@tc##1{\textcolor[rgb]{0.10,0.09,0.49}{##1}}}
\expandafter\def\csname PY@tok@vm\endcsname{\def\PY@tc##1{\textcolor[rgb]{0.10,0.09,0.49}{##1}}}
\expandafter\def\csname PY@tok@sa\endcsname{\def\PY@tc##1{\textcolor[rgb]{0.73,0.13,0.13}{##1}}}
\expandafter\def\csname PY@tok@sb\endcsname{\def\PY@tc##1{\textcolor[rgb]{0.73,0.13,0.13}{##1}}}
\expandafter\def\csname PY@tok@sc\endcsname{\def\PY@tc##1{\textcolor[rgb]{0.73,0.13,0.13}{##1}}}
\expandafter\def\csname PY@tok@dl\endcsname{\def\PY@tc##1{\textcolor[rgb]{0.73,0.13,0.13}{##1}}}
\expandafter\def\csname PY@tok@s2\endcsname{\def\PY@tc##1{\textcolor[rgb]{0.73,0.13,0.13}{##1}}}
\expandafter\def\csname PY@tok@sh\endcsname{\def\PY@tc##1{\textcolor[rgb]{0.73,0.13,0.13}{##1}}}
\expandafter\def\csname PY@tok@s1\endcsname{\def\PY@tc##1{\textcolor[rgb]{0.73,0.13,0.13}{##1}}}
\expandafter\def\csname PY@tok@mb\endcsname{\def\PY@tc##1{\textcolor[rgb]{0.40,0.40,0.40}{##1}}}
\expandafter\def\csname PY@tok@mf\endcsname{\def\PY@tc##1{\textcolor[rgb]{0.40,0.40,0.40}{##1}}}
\expandafter\def\csname PY@tok@mh\endcsname{\def\PY@tc##1{\textcolor[rgb]{0.40,0.40,0.40}{##1}}}
\expandafter\def\csname PY@tok@mi\endcsname{\def\PY@tc##1{\textcolor[rgb]{0.40,0.40,0.40}{##1}}}
\expandafter\def\csname PY@tok@il\endcsname{\def\PY@tc##1{\textcolor[rgb]{0.40,0.40,0.40}{##1}}}
\expandafter\def\csname PY@tok@mo\endcsname{\def\PY@tc##1{\textcolor[rgb]{0.40,0.40,0.40}{##1}}}
\expandafter\def\csname PY@tok@ch\endcsname{\let\PY@it=\textit\def\PY@tc##1{\textcolor[rgb]{0.25,0.50,0.50}{##1}}}
\expandafter\def\csname PY@tok@cm\endcsname{\let\PY@it=\textit\def\PY@tc##1{\textcolor[rgb]{0.25,0.50,0.50}{##1}}}
\expandafter\def\csname PY@tok@cpf\endcsname{\let\PY@it=\textit\def\PY@tc##1{\textcolor[rgb]{0.25,0.50,0.50}{##1}}}
\expandafter\def\csname PY@tok@c1\endcsname{\let\PY@it=\textit\def\PY@tc##1{\textcolor[rgb]{0.25,0.50,0.50}{##1}}}
\expandafter\def\csname PY@tok@cs\endcsname{\let\PY@it=\textit\def\PY@tc##1{\textcolor[rgb]{0.25,0.50,0.50}{##1}}}

\def\PYZbs{\char`\\}
\def\PYZus{\char`\_}
\def\PYZob{\char`\{}
\def\PYZcb{\char`\}}
\def\PYZca{\char`\^}
\def\PYZam{\char`\&}
\def\PYZlt{\char`\<}
\def\PYZgt{\char`\>}
\def\PYZsh{\char`\#}
\def\PYZpc{\char`\%}
\def\PYZdl{\char`\$}
\def\PYZhy{\char`\-}
\def\PYZsq{\char`\'}
\def\PYZdq{\char`\"}
\def\PYZti{\char`\~}
% for compatibility with earlier versions
\def\PYZat{@}
\def\PYZlb{[}
\def\PYZrb{]}
\makeatother


    % Exact colors from NB
    \definecolor{incolor}{rgb}{0.0, 0.0, 0.5}
    \definecolor{outcolor}{rgb}{0.545, 0.0, 0.0}



    
    % Prevent overflowing lines due to hard-to-break entities
    \sloppy 
    % Setup hyperref package
    \hypersetup{
      breaklinks=true,  % so long urls are correctly broken across lines
      colorlinks=true,
      urlcolor=urlcolor,
      linkcolor=linkcolor,
      citecolor=citecolor,
      }
    % Slightly bigger margins than the latex defaults
    
    \geometry{verbose,tmargin=1in,bmargin=1in,lmargin=1in,rmargin=1in}
    
    

    \begin{document}
    
    
    \maketitle
    
    

    
    \hypertarget{solutions---practical-lesson-6}{%
\section{Solutions}\label{solutions---practical-lesson-6}}

\hypertarget{exercises}{%
\subsection{Exercises}\label{exercises}}

\hypertarget{exercise-6.1}{%
\subsubsection{Exercise 6.1}\label{exercise-6.1}}

Write a ForwardRateCurve (for EURIBOR/LIBOR rate curve) which doesn't
compute discount factors but only interplatates forward rates; then add
it to the finmarkets module.

\hypertarget{solution}{%
\paragraph{Solution}\label{solution}}

In this case it is enough to write a new \texttt{class} that has three
attributes: a today date, a set of pillar\_dates and the corresponding
rates. There will be just a single method \texttt{forward\_rate} which
returns the corresponding interpolated rate.

    \begin{Verbatim}[commandchars=\\\{\}]
{\color{incolor}In [{\color{incolor}1}]:} \PY{k+kn}{import} \PY{n+nn}{numpy}
        
        \PY{c+c1}{\PYZsh{} an EURIBOR or LIBOR rate curve}
        \PY{c+c1}{\PYZsh{} doesn\PYZsq{}t calculate discount factors, only interpolates forward rates}
        \PY{k}{class} \PY{n+nc}{ForwardRateCurve}\PY{p}{(}\PY{n+nb}{object}\PY{p}{)}\PY{p}{:}
            
            \PY{c+c1}{\PYZsh{} the special \PYZus{}\PYZus{}init\PYZus{}\PYZus{} method defines how to}
            \PY{c+c1}{\PYZsh{} construct instances of the class}
            \PY{k}{def} \PY{n+nf}{\PYZus{}\PYZus{}init\PYZus{}\PYZus{}}\PY{p}{(}\PY{n+nb+bp}{self}\PY{p}{,} \PY{n}{pillar\PYZus{}dates}\PY{p}{,} \PY{n}{rates}\PY{p}{)}\PY{p}{:}
                
                \PY{c+c1}{\PYZsh{} we just store the arguments as attributes of the instance}
                \PY{n+nb+bp}{self}\PY{o}{.}\PY{n}{today} \PY{o}{=} \PY{n}{pillar\PYZus{}dates}\PY{p}{[}\PY{l+m+mi}{0}\PY{p}{]}
                \PY{n+nb+bp}{self}\PY{o}{.}\PY{n}{rates} \PY{o}{=} \PY{n}{rates}
                
                \PY{n+nb+bp}{self}\PY{o}{.}\PY{n}{pillar\PYZus{}days} \PY{o}{=} \PY{p}{[}
                    \PY{p}{(}\PY{n}{pillar\PYZus{}date} \PY{o}{\PYZhy{}} \PY{n+nb+bp}{self}\PY{o}{.}\PY{n}{today}\PY{p}{)}\PY{o}{.}\PY{n}{days}
                    \PY{k}{for} \PY{n}{pillar\PYZus{}date} \PY{o+ow}{in} \PY{n}{pillar\PYZus{}dates}
                \PY{p}{]}
                
                
            \PY{c+c1}{\PYZsh{} interpolates the forward rates stored in the instance}
            \PY{k}{def} \PY{n+nf}{forward\PYZus{}rate}\PY{p}{(}\PY{n+nb+bp}{self}\PY{p}{,} \PY{n}{d}\PY{p}{)}\PY{p}{:}
                \PY{n}{d\PYZus{}days} \PY{o}{=} \PY{p}{(}\PY{n}{d} \PY{o}{\PYZhy{}} \PY{n+nb+bp}{self}\PY{o}{.}\PY{n}{today}\PY{p}{)}\PY{o}{.}\PY{n}{days}
                \PY{k}{return} \PY{n}{numpy}\PY{o}{.}\PY{n}{interp}\PY{p}{(}\PY{n}{d\PYZus{}days}\PY{p}{,} \PY{n+nb+bp}{self}\PY{o}{.}\PY{n}{pillar\PYZus{}days}\PY{p}{,} \PY{n+nb+bp}{self}\PY{o}{.}\PY{n}{rates}\PY{p}{)}
\end{Verbatim}

    \hypertarget{exercise-6.2}{%
\subsubsection{Exercise 6.2}\label{exercise-6.2}}

Using the function \texttt{randint} of the module \texttt{random} make a
Monte Carlo simulation of rolling three dices to check the probability
of getting the same values on the three of them.

From the probability theory you should expect:

\[P_{d1=d2=d3} = \frac{1}{6}\cdot\frac{1}{6}\cdot\frac{1}{6}\cdot 6 = \frac{1}{36} = 0.0278\]

    \begin{Verbatim}[commandchars=\\\{\}]
{\color{incolor}In [{\color{incolor}2}]:} \PY{k+kn}{from} \PY{n+nn}{random} \PY{k}{import} \PY{n}{seed}\PY{p}{,} \PY{n}{randint}
        
        \PY{n}{seed}\PY{p}{(}\PY{l+m+mi}{1}\PY{p}{)}
        
        \PY{n}{trials} \PY{o}{=} \PY{l+m+mi}{100000}
        \PY{n}{success} \PY{o}{=} \PY{l+m+mi}{0}
        \PY{k}{for} \PY{n}{\PYZus{}} \PY{o+ow}{in} \PY{n+nb}{range}\PY{p}{(}\PY{n}{trials}\PY{p}{)}\PY{p}{:}
          \PY{n}{d1}\PY{p}{,} \PY{n}{d2}\PY{p}{,} \PY{n}{d3} \PY{o}{=} \PY{n}{randint}\PY{p}{(}\PY{l+m+mi}{1}\PY{p}{,} \PY{l+m+mi}{6}\PY{p}{)}\PY{p}{,} \PY{n}{randint}\PY{p}{(}\PY{l+m+mi}{1}\PY{p}{,} \PY{l+m+mi}{6}\PY{p}{)}\PY{p}{,} \PY{n}{randint}\PY{p}{(}\PY{l+m+mi}{1}\PY{p}{,} \PY{l+m+mi}{6}\PY{p}{)}
          
          \PY{k}{if} \PY{n}{d1} \PY{o}{==} \PY{n}{d2} \PY{o+ow}{and} \PY{n}{d2} \PY{o}{==} \PY{n}{d3}\PY{p}{:}
            \PY{n}{success} \PY{o}{+}\PY{o}{=} \PY{l+m+mi}{1}
            
        \PY{n+nb}{print} \PY{p}{(}\PY{l+s+s2}{\PYZdq{}}\PY{l+s+s2}{The probability to get three equal dice is }\PY{l+s+si}{\PYZob{}:.4f\PYZcb{}}\PY{l+s+s2}{\PYZdq{}}\PY{o}{.}\PY{n}{format}\PY{p}{(}\PY{n}{success}\PY{o}{/}\PY{n}{trials}\PY{p}{)}\PY{p}{)}
\end{Verbatim}

    \begin{Verbatim}[commandchars=\\\{\}]
The probability to get three equal dice is 0.0276

    \end{Verbatim}

    \hypertarget{exercise-6.3}{%
\subsubsection{Exercise 6.3}\label{exercise-6.3}}

Using the function \texttt{normal} of \texttt{numpy.random} simulate the
price of a stock which evolves according to a log-normal stochastic
process with a daily rate of return \(\mu=0.1\) and a volatility
\(\sigma=0.15\) for 30 days.

Also plot the price. Try to play with \(\mu\) and \(\sigma\) to see how
the plot changes.

    \begin{Verbatim}[commandchars=\\\{\}]
{\color{incolor}In [{\color{incolor}3}]:} \PY{k+kn}{from} \PY{n+nn}{numpy}\PY{n+nn}{.}\PY{n+nn}{random} \PY{k}{import} \PY{n}{normal}\PY{p}{,} \PY{n}{seed}
        \PY{k+kn}{from} \PY{n+nn}{matplotlib} \PY{k}{import} \PY{n}{pyplot} \PY{k}{as} \PY{n}{plt}
        \PY{k+kn}{import} \PY{n+nn}{math}
        
        \PY{n}{S} \PY{o}{=} \PY{l+m+mi}{100}
        \PY{n}{mu} \PY{o}{=} \PY{l+m+mf}{0.1}
        \PY{n}{sigma} \PY{o}{=} \PY{l+m+mf}{0.15}
        \PY{n}{T} \PY{o}{=} \PY{l+m+mi}{1}
        
        \PY{n}{seed}\PY{p}{(}\PY{l+m+mi}{1}\PY{p}{)}
        \PY{n}{historical\PYZus{}series} \PY{o}{=} \PY{p}{[}\PY{n}{S}\PY{p}{]}
        \PY{k}{for} \PY{n}{i} \PY{o+ow}{in} \PY{n+nb}{range}\PY{p}{(}\PY{l+m+mi}{30}\PY{p}{)}\PY{p}{:}
            \PY{n}{S} \PY{o}{=} \PY{n}{S} \PY{o}{*} \PY{n}{math}\PY{o}{.}\PY{n}{exp}\PY{p}{(}\PY{p}{(}\PY{n}{mu} \PY{o}{\PYZhy{}} \PY{l+m+mf}{0.5} \PY{o}{*} \PY{n}{sigma} \PY{o}{*} \PY{n}{sigma}\PY{p}{)} \PY{o}{*} \PY{n}{T} \PY{o}{+} \PY{n}{sigma} \PY{o}{*} \PY{n}{math}\PY{o}{.}\PY{n}{sqrt}\PY{p}{(}\PY{n}{T}\PY{p}{)} \PY{o}{*} \PY{n}{normal}\PY{p}{(}\PY{p}{)}\PY{p}{)}
            \PY{n}{historical\PYZus{}series}\PY{o}{.}\PY{n}{append}\PY{p}{(}\PY{n}{S}\PY{p}{)}
            
        \PY{n}{plt}\PY{o}{.}\PY{n}{plot}\PY{p}{(}\PY{n+nb}{range}\PY{p}{(}\PY{l+m+mi}{31}\PY{p}{)}\PY{p}{,} \PY{n}{historical\PYZus{}series}\PY{p}{)}
        \PY{n}{plt}\PY{o}{.}\PY{n}{xlabel}\PY{p}{(}\PY{l+s+s2}{\PYZdq{}}\PY{l+s+s2}{days}\PY{l+s+s2}{\PYZdq{}}\PY{p}{)}
        \PY{n}{plt}\PY{o}{.}\PY{n}{ylabel}\PY{p}{(}\PY{l+s+s2}{\PYZdq{}}\PY{l+s+s2}{Price of stock X}\PY{l+s+s2}{\PYZdq{}}\PY{p}{)}
        \PY{n}{plt}\PY{o}{.}\PY{n}{show}\PY{p}{(}\PY{p}{)}
            
\end{Verbatim}

    
    \begin{verbatim}
<Figure size 640x480 with 1 Axes>
    \end{verbatim}

    
    \hypertarget{exercise-6.4}{%
\subsubsection{Exercise 6.4}\label{exercise-6.4}}

Suppouse that the Libor Forward rates are those defined
\(\href{}{here}\). Determine the value of an option to pay a fixed rate
of 4\% and receives LIBOR on a 5 year swap starting in 1 year. Assume
the notional is 100 EUR, the exercise date is on October, 30th 2020 and
the swap rate volatility is 15\%.

    \begin{Verbatim}[commandchars=\\\{\}]
{\color{incolor}In [{\color{incolor}4}]:} \PY{k+kn}{from} \PY{n+nn}{finmarkets} \PY{k}{import} \PY{n}{InterestRateSwap}
        \PY{k+kn}{from} \PY{n+nn}{datetime} \PY{k}{import} \PY{n}{date}
        \PY{k+kn}{from} \PY{n+nn}{dateutil}\PY{n+nn}{.}\PY{n+nn}{relativedelta} \PY{k}{import} \PY{n}{relativedelta}
        \PY{k+kn}{from} \PY{n+nn}{curve\PYZus{}data} \PY{k}{import} \PY{n}{discount\PYZus{}curve}\PY{p}{,} \PY{n}{libor\PYZus{}curve}
        \PY{k+kn}{from} \PY{n+nn}{scipy}\PY{n+nn}{.}\PY{n+nn}{stats} \PY{k}{import} \PY{n}{norm}
        \PY{k+kn}{import} \PY{n+nn}{math}
        
        \PY{n}{pricing\PYZus{}date} \PY{o}{=} \PY{n}{date}\PY{o}{.}\PY{n}{today}\PY{p}{(}\PY{p}{)}
        \PY{n}{start\PYZus{}date} \PY{o}{=} \PY{n}{date}\PY{o}{.}\PY{n}{today}\PY{p}{(}\PY{p}{)} \PY{o}{+} \PY{n}{relativedelta}\PY{p}{(}\PY{n}{years}\PY{o}{=}\PY{l+m+mi}{1}\PY{p}{)}
        \PY{n}{exercise\PYZus{}date} \PY{o}{=} \PY{n}{date}\PY{p}{(}\PY{l+m+mi}{2020}\PY{p}{,} \PY{l+m+mi}{10}\PY{p}{,} \PY{l+m+mi}{30}\PY{p}{)}
        
        \PY{n}{irs} \PY{o}{=} \PY{n}{InterestRateSwap}\PY{p}{(}\PY{n}{start\PYZus{}date}\PY{p}{,} \PY{l+m+mi}{100}\PY{p}{,} \PY{l+m+mf}{0.04}\PY{p}{,} \PY{l+m+mi}{12}\PY{p}{,} \PY{l+m+mi}{5}\PY{p}{)}
        \PY{n}{sigma} \PY{o}{=} \PY{l+m+mf}{0.15}
        
        \PY{n}{A} \PY{o}{=} \PY{n}{irs}\PY{o}{.}\PY{n}{annuity}\PY{p}{(}\PY{n}{discount\PYZus{}curve}\PY{p}{)}
        \PY{n}{S} \PY{o}{=} \PY{n}{irs}\PY{o}{.}\PY{n}{swap\PYZus{}rate}\PY{p}{(}\PY{n}{discount\PYZus{}curve}\PY{p}{,} \PY{n}{libor\PYZus{}curve}\PY{p}{)}
        \PY{n}{T} \PY{o}{=} \PY{p}{(}\PY{n}{exercise\PYZus{}date} \PY{o}{\PYZhy{}} \PY{n}{pricing\PYZus{}date}\PY{p}{)}\PY{o}{.}\PY{n}{days} \PY{o}{/} \PY{l+m+mi}{365}
        \PY{n}{d1} \PY{o}{=} \PY{p}{(}\PY{n}{math}\PY{o}{.}\PY{n}{log}\PY{p}{(}\PY{n}{S}\PY{o}{/}\PY{n}{irs}\PY{o}{.}\PY{n}{fixed\PYZus{}rate}\PY{p}{)} \PY{o}{+} \PY{l+m+mf}{0.5} \PY{o}{*} \PY{n}{sigma}\PY{o}{*}\PY{o}{*}\PY{l+m+mi}{2} \PY{o}{*} \PY{n}{T}\PY{p}{)} \PY{o}{/} \PY{p}{(}\PY{n}{sigma} \PY{o}{*} \PY{n}{T}\PY{o}{*}\PY{o}{*}\PY{l+m+mf}{0.5}\PY{p}{)}
        \PY{n}{d2} \PY{o}{=} \PY{p}{(}\PY{n}{math}\PY{o}{.}\PY{n}{log}\PY{p}{(}\PY{n}{S}\PY{o}{/}\PY{n}{irs}\PY{o}{.}\PY{n}{fixed\PYZus{}rate}\PY{p}{)} \PY{o}{\PYZhy{}} \PY{l+m+mf}{0.5} \PY{o}{*} \PY{n}{sigma}\PY{o}{*}\PY{o}{*}\PY{l+m+mi}{2} \PY{o}{*} \PY{n}{T}\PY{p}{)} \PY{o}{/} \PY{p}{(}\PY{n}{sigma} \PY{o}{*} \PY{n}{T}\PY{o}{*}\PY{o}{*}\PY{l+m+mf}{0.5}\PY{p}{)}
        \PY{n}{npv} \PY{o}{=} \PY{n}{irs}\PY{o}{.}\PY{n}{notional} \PY{o}{*} \PY{n}{A} \PY{o}{*} \PY{p}{(}\PY{n}{S} \PY{o}{*} \PY{n}{norm}\PY{o}{.}\PY{n}{cdf}\PY{p}{(}\PY{n}{d1}\PY{p}{)} \PY{o}{\PYZhy{}} \PY{n}{irs}\PY{o}{.}\PY{n}{fixed\PYZus{}rate} \PY{o}{*} \PY{n}{norm}\PY{o}{.}\PY{n}{cdf}\PY{p}{(}\PY{n}{d2}\PY{p}{)}\PY{p}{)}
        
        \PY{n+nb}{print}\PY{p}{(}\PY{l+s+s2}{\PYZdq{}}\PY{l+s+s2}{Swaption NPV: }\PY{l+s+si}{\PYZob{}:.3f\PYZcb{}}\PY{l+s+s2}{ EUR}\PY{l+s+s2}{\PYZdq{}}\PY{o}{.}\PY{n}{format}\PY{p}{(}\PY{n}{npv}\PY{p}{)}\PY{p}{)}
\end{Verbatim}

    \begin{Verbatim}[commandchars=\\\{\}]
Swaption NPV: 13.587 EUR

    \end{Verbatim}


    % Add a bibliography block to the postdoc
    
    
    
    \end{document}
